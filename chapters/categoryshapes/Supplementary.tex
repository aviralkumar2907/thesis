% last updated in April 2002 by Antje Endemann
% Based on CVPR 07 and LNCS, with modifications by DAF, AZ and elle, 2008 and AA, 2010, and CC, 2011; TT, 2014

\documentclass[runningheads]{llncs}
\usepackage{graphicx}
\usepackage{amsmath,amssymb} % define this before the line numbering.
\usepackage{ruler}
\usepackage{color}
\usepackage{caption}
%\usepackage[caption=false]{subfig}
\usepackage{subcaption}
\captionsetup{compatibility=false}
\usepackage[width=122mm,left=12mm,paperwidth=146mm,height=193mm,top=12mm,paperheight=217mm]{geometry}

% Helper commands
\newcommand{\BLUE}[1]{{\textcolor{blue}{#1}}}
\newcommand{\RED}[1]{{\textcolor{red}{#1}}}
\newcommand{\rgbd}[0]{RGB-D }
\newcommand{\etal}[0]{\textit{et al. }}
\newcommand{\nrsfm}[0]{NRSfM}
\begin{document}
% \renewcommand\thelinenumber{\color[rgb]{0.2,0.5,0.8}\normalfont\sffamily\scriptsize\arabic{linenumber}\color[rgb]{0,0,0}}
% \renewcommand\makeLineNumber {\hss\thelinenumber\ \hspace{6mm} \rlap{\hskip\textwidth\ \hspace{6.5mm}\thelinenumber}}
% \linenumbers
\pagestyle{headings}
\mainmatter
\def\ECCV14SubNumber{612}  % Insert your submission number here

%\title{}
%\title{Combining Top-Down and Bottom-up Reconstruction}
%\title{Morphing PASCAL VOC}
%\title{Unleashing Rich 3D Shape Models}
%\title{Object Reconstruction Unleashed}
%\title{Morphable Models Unleashed}
%\title{Learning 3D Shape Priors from 2D Annotations}
\title{Category-Specific 3D Shape Reconstruction from a Single Image : \\ Supplementary Material}

\titlerunning{ECCV-14 submission ID \ECCV14SubNumber}

\authorrunning{ECCV-14 submission ID \ECCV14SubNumber}

\author{Anonymous ECCV submission}
\institute{Paper ID \ECCV14SubNumber}

\maketitle

%\section{Organization}
%We describe in detail the dense shape models proposed in our work in sections \ref{OPTIMIZ} and \ref{VH}. We aim for these sections to be self-sufficient and therefore there is some overlap with the summarized descriptions given in our submission. In section \ref{Experiements} we extensively report the  results obtained using different levels of curation for the test set. Section \ref{Reconstructions} shows the final shapes obtained using our approach for a number of instances.

%\section{Basis Shape Models}
%\label{OPTIMIZ}
%%\textcolor{red}{ 'occluding contours' to 'silhouettes', shape model as 'set' of points to matrix, more intuition for optimization }

\subsection{Shape Model}
We learn a shape model $M = (\overline{S},V)$ comprising a mean shape $\overline{S}$ and deformation bases $V = \{ V_1,.,V_K \} $. An explicit point coordinate based representation is used and therefore $\overline{S}, V_i \in \mathbb{R}^{N \times 3}$. Our basis shape model represents a shape $S$ as a the mean shape deformed by a linear combination of the deformation bases i.e. $S = \overline{S} + \underset{k}{\sum}\alpha_{k} V_k$. The expressive power of our model is the similar to PCA as they both cover a $K$ dimensional subspace of $\mathbb{R}^{N \times 3}$ using $K$ basis vectors (though we do not enforce orthogonality of deformation basis as the various sematically meaningful deformation modes across a category are not necessarily orthogonal).

\subsection{Inputs}
We learn the shape model using training set $T:\{(O^i,P^i)\}$ where $O^i$ is the corresponding silhouette and $P^i$ is the function to project points from the world coordinates to image coordinates. The silhouette for an instance is represented as the coordinates of the set of boundary points in the image frame. To infer the shape $S$ for new instance, we require the same input - $(O,P)$ (it's silhouette and projection function). Note that the $P^i$ we obtain using NRSfM corresponds to orthographic projection but our algorithm to infer shape models can handle perspective projections as well.

\subsection{Energy Formulation}
We formulate an objective function based on three properties which a proposed solution should intuitively satisfy  - 1)\textit{Silhouette Consistency:} the shape inferred for an instance does not project outside its silhouette. 2) \textit{Silhouette Coverage:} every point on the silhouette is adjacent to at least some (say $m$)  points on the projected model-induced shape for the instance. 3) \textit{Local Consistency:} distance to a point's nearest neighbors is small and neighboring points have similar deformations. We capture these properties by defining corresponding energy terms as follows (here $\Delta(Q,p,k)$ corresponds to the average distance of $p$ to its nearest $k$ neighbors in $Q$):

\begin{gather}
 \label{eq:local_con}\text{Local Consistency: }E_{L}(\bar{S},V)=\underset{i}{\sum}\underset{j\epsilon N(i)}{\sum}(\|\bar{S}_{i}-\bar{S}_{j}\|^2 +\underset{k}{\sum}\|V_{ki}-V_{kj}\|^2) \\
 \label{eq:sil_con} \text{Silhouette Consistency: }E_{S}(S,O,P)=\underset{p \in P(S)\setminus Mask(O)}{\sum}\Delta(O,p,1)^{2}\\
  \label{eq:sil_cov}\text{Silhouette Coverage: }E_{C}(S,O,P)=\underset{p\epsilon O}{\sum}\Delta(P(S),p,m)^2
\end{gather}

Equation~\eqref{eq:sil_con} penalizes the projected points according to the Chamfer distance field of the mask corresponding to silhouette $O$. Equation~\eqref{eq:sil_cov} penalizes large discrepancies from every boundary pixel to its $m$ nearest neighbors in the projected shape.  We use the objective in equation \ref{eq:formulation} to learn a shape model $M = (\overline{S},V)$ such that the shapes inferred using the model for all instances in the training set satisfy the above properties while penalizing the amount of deformation to regularize the objective.

\begin{equation}
\label{eq:formulation}
\underset{\bar{S},V,\alpha}{\min} E_{L}(\bar{S},V)+\underset{i}{\sum}(\lambda_{1}E_{S}(S^{i},O^{i},P^{i})+\lambda_{2}E_{C}(S^{i},O^{i},P^{i})+\lambda_3\underset{k}{\sum}(\|\alpha_{ik}V_k\|_F^{2}))
\end{equation}

where 

\[
S^i = \overline{S} + \underset{k}{\sum}\alpha_{ik} V_k \text{  : Shape for }  i^{th} \text{ training instance}
\]

\subsection{Learning and Inference}
As can be seen in fig.~\ref{fig:basisShapesModel}, the shape models learned using our formulation are able to model the coarse level category shapes as well as the intra-class variation using the mean shapes and deformation bases respectively.

\begin{figure}[htb!]
\centering
\begin{subfigure}{.45\textwidth}
  \includegraphics[width=\linewidth, trim=8 8 8 8,clip]{figures/meanShapes.pdf}
  \caption{Mean shapes for chair, aeroplane, bicycle and car classes.}
  \label{fig:meanDense}
\end{subfigure}%
\hfill
\begin{subfigure}{.45\textwidth}
  \includegraphics[width=\linewidth]{figures/aeroplaneDeformations.jpg}
  \caption{Deformation modes for aeroplanes.}
  \label{fig:denseDeformations}
\end{subfigure}
\caption{Our basis shapes model learns a mean shape and a deformation basis for each class. a) Learned mean shapes for various categories. b) Mean shape for aeroplane deformed according to the learned deformations bases. }
\label{fig:basisShapesModel}
\end{figure}
To infer the shape for a novel image using the learned model $M=(\bar{S},V)$, we solve for the deformation weights $\alpha$ by optimizing equation~\eqref{eq:formulation} wrt $\alpha$ for fixed $\bar{S},V$.

\subsection{Initialization}
Our training objective is non-convex and  non-smooth and is susceptible to local minima. We follow the suggestion of \cite{esteban2004snake} and initialize our mean shape with the visual hull computed using \textit{all} training instances. The deformation bases and deformation weights are initialized randomly.

\subsection{Optimization Implementation}
To address the scale ambiguity between $V$ and $\alpha$ in our formulation, we restrict $\|V_k\|_F$ to be a constant. The mean shape and deformation basis are inferred via block-coordinate descent on the objective using sub-gradient computations over the training set. 

%\section{Prototype-Shape Models}
%\label{VH}
%
Parametric models with simple deformations might not sufficiently capture large intra-class variance which is frequently present in some categories. Allowing complex deformations, on the other hand, would result in a significant increase in the computational complexity as well as the possibility of overfitting. We explore whether representing category shapes via multiple learned prototypes can alleviate these issues. We propose to learn a representative set of 3D prototypes for a category employing variations of visual hull techniques that we make more robust to intra-class variation and noisy camera estimates.

A key observation which enables this approach is that an estimate of a coarse rotation invariant shape parameter allows grouping of visually similar instances into ``visual clusters". Deriving our prototypes from these visual clusters provides an effective way of modeling the various modes of the distribution underlying the shapes of an object category. Formally, we infer our shape model from a training set $T:\{(O^i,P^i,\alpha^i)\}$, where the quantities $O_i$ and $P_i$ are as described in the previous section. The additional input $\alpha^i$ corresponds to the coarse rotation invariant shape parameter which in our implementation is the deformation weight vector obtained using the NRSfM algorithm. Note that the proposed approach can exploit any coarse shape parameter (e.g. simple cues based on the silhouette shape).

\subsubsection{Learning:}
\begin{figure}[htb!]
\centering
\begin{minipage}{.45\textwidth}

  \includegraphics[width=.45\linewidth]{figures/visClusters1.jpg}
  \label{fig:VisCluster1}
\quad
%\begin{subfigure}{.22\textwidth}
  \includegraphics[width=.45\linewidth]{figures/visClustersShape1.jpg}
  \label{fig:VisShape1}
%\end{subfigure}

%\begin{subfigure}{.22\textwidth}
  \includegraphics[width=.45\linewidth]{figures/visClusters2.jpg}
  \label{fig:VisCluster2}
%\end{subfigure}%
\quad
%\begin{subfigure}{.22\textwidth}
  \includegraphics[width=.45\linewidth]{figures/visClustersShape2.jpg}
  \label{fig:VisShape2}
%\end{subfigure}
\vspace{7pt}
  
\caption{Learned prototype shapes corresponding to different visual clusters.}
\label{fig:visClusterShapes}
\end{minipage}
\hspace{0.05\textwidth}
%\vspace*{\fill}
%\begin{center}
\begin{minipage}{.45\textwidth}

\centering
 \includegraphics[width = \linewidth]{figures/visHullPlanes.jpg}
%\vspace{-5}
\caption{Illustration of visual cones induced by silhouettes.}
\label{fig:visHullPlanes}

\end{minipage}
%\end{center}

%\vfill
\end{figure}


%\begin{figure}[htb!]
%\centering
% \includegraphics[scale = 0.2]{figures/visHullPlanes.jpg}
%\caption{Illustration of visual cones induced by silhouettes.}
%\label{fig:visHullPlanes}
%\end{figure}

We represent the shape model $M = \{ (\bar{S_k},\alpha_k) | k=1..K\}$ as a set of volumetric shape representations $\bar{S_k}$ and associated shape parameters $\alpha_k$. The individual prototype $\bar{S_k}$ is learned from the corresponding visual cluster and the associated shape parameter $\alpha_k$ represents the mean deformation parameter for the corresponding cluster.

In fig.~\ref{fig:visClusterShapes}, we visualize some training instances and their ``visual cones" - a visual cone is the volume that projects inside the silhouette under the corresponding orthographic transformation. The visual hull approach for learning a shape model is to take the intersection of all the visual cones and the object volume is then represented as a binary value associated with each voxel. We use an extensible representation based on the truncated signed distance function (TSDF) which is a field over voxel space such that a level set describes the object surface. A TSDF represents the signed distances of points from a surface and the distances are truncated at some minimum and maximum thresholds. We denote a TSDF $S$ whose values are restricted between $[l,L]$ as $S[l,L]$. Thus the prototype $\bar{S_k}$ visual hull from the associated visual cluster $C_k$ is simply
\begin{equation}
\label{eq:VisLearning}
\bar{S_k} = \underset{i \in C_k}{\sum}S^i[0,\infty]
\end{equation}
where $S_i[0,\infty]$ is the TSDF representing the distances from the surface of the visual cone for  $O^i$ under transformation $P^i$. We prefer the TSDF representation because it can be easily extended to handle variations and noise as we describe later. Fig.~\ref{fig:visHullPlanes} shows some visual clusters and the corresponding prototype models that we learn.

\subsubsection{Inference:}
Given an instance $I = (O,P,\alpha)$, we hypothesize that its dense shape would be similar to the shape of the instances in the most similar visual cluster in our model $M = \{ (\bar{S_k},\alpha_k)\}$. Thus, we find its closest cluster $C_j$ in the training set and combine the shape information obtained using the silhouette for the particular instance with the prototype model of $C_j$. We first compute the TSDF $S[0,\infty]$ for the particular instance using $(O,P)$. This volumetric representation is then updated according to equation \ref{eq:inferenceVis} to infer the dense shape for the given instance.
\begin{equation}
\label{eq:inferenceVis}
S \leftarrow S + \lambda \bar{S_j}
\end{equation}
\subsubsection{Allowing for Intra-class Variations and Camera Estimation Error:}
The approach described above learns the visual hull for each visual cluster. However, this may not be the desirable as our camera estimates are not perfect and a visual cluster does not exactly correspond to an instance. We suggest two modifications to make our method more robust -
\begin{itemize}
\item Instead of using a minimum truncation value 0 for the TSDF computations, we allow a small negative value. This results in learned/inferred shapes which differ from the visual hull approach by allowing us to retain the voxels which are consistent with a majority of the instances and violate very few constraints, presumably due to shape variations or noisy camera estimates.
\item The distribution of views for an object class is often non-uniform and to account for this, we modify the learning in equation \ref{eq:VisLearning} to assign an importance weight to each instance. Since we aim to negate the viewpoint bias using these weights, we first cluster the camera viewpoints within each visual cluster. The weight for each instance is then assigned to be inversely proportional to the size of its corresponding viewpoint cluster to normalize for the view distribution.
\end{itemize}


\section{Experimental Results}
\label{Experiements}

We used as ground truth the 3D meshes from PASCAL3D+ \cite{pascal3d}, that are manually aligned to all the objects in the 12 rigid categories of VOC. PASCAL3D+ provides between 4 different meshes for "tvmonitor" and "train" and 10 meshes for ``car" and ``chair" but they do not always align perfectly with the objects pictured in the images. In order to get more meaningful ground truth data we curated the annotations and kept those that were more accurate with respect to a 2D proxy for 3D alignment: the intersection-over-union (IoU) overlap between each projected silhouette and the corresponding ground-truth object segmentation which we have. The obtained errors for the reconstructed meshes for different IoU thresholds used for curation of the test set are shown in fig.~\ref{fig:meshScore}. We compare the dense shape inference methods presented in our work with \cite{carvi14} and Puffball \cite{twarog2012playing}.
In fig.~\ref{fig:depthScore}, we report the mean correlation between the ground truth depth and the depth maps obtained using various methods for different IoU thresholds for curation. We compare our results with \cite{carvi14} and SIRFS \cite{barronPAMI13}. Note that we use the performances corresponding to IoU 0.5 in our submission as not all classes have enough examples at higher thresholds.

\begin{figure}[htb!]
\includegraphics[width=0.44\linewidth]{figures/plots/aeroplaneMeshScores} \hfill
 \includegraphics[width=0.44\linewidth]
  {figures/plots/bicycleMeshScores}

\includegraphics[width=0.44\linewidth]{figures/plots/boatMeshScores} \hfill
 \includegraphics[width=0.44\linewidth]{figures/plots/busMeshScores}

\includegraphics[width=0.44\linewidth]{figures/plots/carMeshScores} \hfill
 \includegraphics[width=0.44\linewidth]{figures/plots/chairMeshScores}
 
 \includegraphics[width=0.44\linewidth]{figures/plots/motorbikeMeshScores} \hfill
 \includegraphics[width=0.44\linewidth]{figures/plots/sofaMeshScores}
 
 \includegraphics[width=0.44\linewidth]{figures/plots/trainMeshScores} \hfill
 \includegraphics[width=0.44\linewidth]{figures/plots/tvmonitorMeshScores}
\caption{\label{fig:meshScore} Mean error for obtained meshes}
 
\end{figure}

\begin{figure}[htb!]
\includegraphics[width=0.44\linewidth]{figures/plots/aeroplaneDepthScores} \hfill
 \includegraphics[width=0.44\linewidth]
  {figures/plots/bicycleDepthScores}

\includegraphics[width=0.44\linewidth]{figures/plots/boatDepthScores} \hfill
 \includegraphics[width=0.44\linewidth]{figures/plots/busDepthScores}

\includegraphics[width=0.44\linewidth]{figures/plots/carDepthScores} \hfill
 \includegraphics[width=0.44\linewidth]{figures/plots/chairDepthScores}
 
 \includegraphics[width=0.44\linewidth]{figures/plots/motorbikeDepthScores} \hfill
 \includegraphics[width=0.44\linewidth]{figures/plots/sofaDepthScores}
 
 \includegraphics[width=0.44\linewidth]{figures/plots/trainDepthScores} \hfill
 \includegraphics[width=0.44\linewidth]{figures/plots/tvmonitorDepthScores}
\caption{\label{fig:depthScore} Measure of correlation of obtained depth maps with groundtruth}
 
\end{figure}

\section{Obtained Reconstructions}
\label{Reconstructions}

In fig. \ref{fig:recons1}, \ref{fig:recons2} and \ref{fig:recons3} we visualize the reconstructions and depth maps obtained for objects of various classes.

\begin{figure}[htb!]
  \includegraphics[width = \textwidth]{figures/reconstructions/1}
  
 \includegraphics[width = \textwidth]{figures/reconstructions/2}
 
 \includegraphics[width = \textwidth]{figures/reconstructions/3}

  \includegraphics[width = \textwidth]{figures/reconstructions/4} 
 
 \includegraphics[width = \textwidth]{figures/reconstructions/5}    
 
\includegraphics[width = \textwidth]{figures/reconstructions/6} 
 
 \includegraphics[width = \textwidth]{figures/reconstructions/28}
 
\includegraphics[width = \textwidth]{figures/reconstructions/29}

\includegraphics[width = \textwidth]{figures/reconstructions/30}

 \includegraphics[width = \textwidth]{figures/reconstructions/7}
  
   \includegraphics[width = \textwidth]{figures/reconstructions/8}

\includegraphics[width = \textwidth]{figures/reconstructions/9}
       
\caption{\label{fig:recons1}Example reconstructions of objects shown on the left column. Columns 3 and 6  show the image-aligned 3D reconstruction obtained using our basis shape model and prototype shape model respectively. Columns 4 and 7 show the shape obtained in  columns 3 and 6 respectively from a different view. Columns 2 and 5 show the corresponding final depth maps combining both top-down and bottom-up cues. Blue is closer to the camera, red is farther away.}
 
\end{figure}

\begin{figure}[htb!]
  \includegraphics[width = \textwidth]{figures/reconstructions/10}
  
\includegraphics[width = \textwidth]{figures/reconstructions/11}

\includegraphics[width = \textwidth]{figures/reconstructions/12}

 \includegraphics[width = \textwidth]{figures/reconstructions/13}
  
 \includegraphics[width = \textwidth]{figures/reconstructions/14}

\includegraphics[width = \textwidth]{figures/reconstructions/15}

 \includegraphics[width = \textwidth]{figures/reconstructions/16}

\includegraphics[width = \textwidth]{figures/reconstructions/17}

\includegraphics[width = \textwidth]{figures/reconstructions/18}

\caption{\label{fig:recons2}Example reconstructions of objects shown on the left column. Columns 3 and 6  show the image-aligned 3D reconstruction obtained using our basis shape model and prototype shape model respectively. Columns 4 and 7 show the shape obtained in  columns 3 and 6 respectively from a different view. Columns 2 and 5 show the corresponding final depth maps combining both top-down and bottom-up cues. Blue is closer to the camera, red is farther away.}

\end{figure}

\begin{figure}[htb!]

\includegraphics[width = \textwidth]{figures/reconstructions/19}

\includegraphics[width = \textwidth]{figures/reconstructions/20}

\includegraphics[width = \textwidth]{figures/reconstructions/21}

\includegraphics[width = \textwidth]{figures/reconstructions/22}

\includegraphics[width = \textwidth]{figures/reconstructions/23}

\includegraphics[width = \textwidth]{figures/reconstructions/24}

\includegraphics[width = \textwidth]{figures/reconstructions/25}

\includegraphics[width = \textwidth]{figures/reconstructions/26}

\includegraphics[width = \textwidth]{figures/reconstructions/27}

\caption{\label{fig:recons3}Example reconstructions of objects shown on the left column. Columns 3 and 6  show the image-aligned 3D reconstruction obtained using our basis shape model and prototype shape model respectively. Columns 4 and 7 show the shape obtained in  columns 3 and 6 respectively from a different view. Columns 2 and 5 show the corresponding final depth maps combining both top-down and bottom-up cues. Blue is closer to the camera, red is farther away.}
\end{figure}

\bibliographystyle{splncs}
\bibliography{references}
\end{document}