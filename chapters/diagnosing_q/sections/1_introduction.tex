\section{Introduction}

In principle, off-policy reinforcement learning (RL) methods provide an effective way to learn optimal policies from static data: by learning value-functions, Q-learning and actor-critic algorithms, can learn optimal policies from even sub-optimal offline data. Via an empirical investigation, in this section, we aim to understand if this promise continues to hold in practice, in an attempt to isolate practical problems that must be solved to realize the usability of off-policy RL methods for learning from entirely static data. We will focus on answering the following questions: 

% Q-learning algorithms, which are based on approximating state-action value functions, are an efficient and commonly used class of RL methods. Q-learning methods have several very appealing properties: they are relatively sample-efficient when compared to policy gradient methods, and they allow for off-policy learning. This makes them an appealing choice for a wide range of tasks, from robotic control~\citep{kalashnikov18} and video game AI~\citep{Mnih2015} to off-policy learning from historical data for recommender systems~\citep{shani2005recommender}. However, although the basic tabular Q-learning algorithm is convergent and admits theoretical analysis~\cite{suttonrlbook}, its counterpart with function approximation is poorly understood. 
% In this paper, we aim to investigate the degree to which potential issues with Q-learning manifest in practice. 
% We empirically analyze aspects of the Q-learning method in a \emph{unit testing} framework, where we can employ oracle solvers to obtain ground truth Q-functions and distributions for exact analysis. We investigate the following questions:

% \textbf{1) What is the effect of function approximation on convergence?}
% Many practical RL problems require function approximation to handle large or continuous state spaces. However, the behavior of Q-learning methods under function approximation is not well understood -- there are simple counterexamples where the method diverges~\citep{Baird1995}. 
% To investigate these problems, we study the convergence behavior of Q-learning methods with function approximation by varying the function approximator power and analyzing the quality of the solution found. 
% We find, somewhat surprisingly, that divergence rarely occurs, and that function approximation error is not a major problem in Q-learning algorithms when the function approximator is powerful. This makes sense in light of the theory: a high-capacity approximator can perform an accurate projection of the bellman Backup, thus mitigating potential convergence issues due to function approximation. (Section~\ref{sec:function_approx})

~

\niparagraph{\large{(1) What is the effect of distributional shift?}}

~

The standard formulation of off-policy Q-learning and actor-critic prescribes an update rule, with no corresponding objective function~\citep{Sutton09b}. This results in a learning process which optimizes an objective that suffers from distributional shift: the target values computed by the Bellman backup query actions from the learned policy, which is not necessarily identical to the data collection policy (and in general, we would expect that the learned policy would differ greatly from the data collection policy since it is supposed to improve over the behaviors in the offline dataset). We perform controlled experiments to investigate the amount of distributional shift and its impact on performance, observing that deviating away from the distributions of states and actions in the offline dataset can lead to significant instability over the course of learning.

~

\niparagraph{\large{(2) What is the effect of sampling error?}}

~

In general, it is impossible to infer the true dynamics of the underlying MDP using just a finite amount of static, offline data provided in the problem. Like any standard supervised learning setting, by reusing samples in the offline data multiple times for learning, we would also expect off-policy RL methods to overfit to the training data. To what extent, does this overfitting hurt performance? We experimentally show that overfitting exists in practice by performing ablation studies on the number of gradient steps utilized for learning, and by demonstrating that oracle based early stopping techniques can be used to improve performance of Q-learning algorithms.
%Thus, in our experiments we quantify the amount of overfitting which happens in practice, incorporating a variety of metrics, and performing a number of ablations and investigate methods to mitigate its effects.


% \textbf{4) What is the best sampling or weighting distribution?}
% Deeply tied to the distribution shift problem is the choice of which distribution to sample from. Do moving distributions cause instability, as Q-values trained on one distribution are evaluated under another in subsequent iterations?
% Researchers have often noted that on-policy samples are typically superior to off-policy samples~\citep{suttonrlbook}, and there are several theoretical results that highlight favorable convergence properties under on-policy samples. However, there is little theoretical guidance on how to pick distributions so as to maximize learning. To this end, we investigate several choices for the sampling distribution. {Surprisingly, we find that on-policy training distributions are not always preferable, and that broader, higher-entropy distributions often perform better, regardless of distributional shift. Motivated by our findings, we propose a novel weighting distribution, adversarial feature matching (AFM), which is explicitly compensates for function approximator error, while still producing high-entropy sampling distributions.}


%%AK: say this para in some different way
% In summary, we introduce a unit testing framework for Q-learning to disentangle potential bottlenecks where approximate components are replaced by oracles. This allows for controlled analysis of different sources of error. We study various sources of instability and error in Q-learning algorithms on tabular domains, and show that many of these trends hold true in high dimensional domains. We then propose a novel sampling distribution that improve performance even on high-dimensional tasks. %Our overall aim is to offer practical guidance for designing RL algorithms, as well as to identify important issues to solve in future research.
