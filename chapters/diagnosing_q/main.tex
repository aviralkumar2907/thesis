%%%%%%%% ICML 2019 EXAMPLE LATEX SUBMISSION FILE %%%%%%%%%%%%%%%%%

\documentclass{article}

% Recommended, but optional, packages for figures and better typesetting:
\usepackage{microtype}
\usepackage{graphicx}
\usepackage{subcaption}
\usepackage{booktabs} % for professional tables
\usepackage{multicol}
\usepackage[dvipsnames]{xcolor}

% hyperref makes hyperlinks in the resulting PDF.
% If your build breaks (sometimes temporarily if a hyperlink spans a page)
% please comment out the following usepackage line and replace
% \usepackage{icml2019} with \usepackage[nohyperref]{icml2019} above.
\usepackage{hyperref}
\usepackage{amsmath}
\usepackage{amssymb}
\usepackage{mathtools}
\usepackage{natbib}
\usepackage{enumerate}
\usepackage{tikz}
\usepackage{graphicx}

\DeclarePairedDelimiter\abs{\lvert}{\rvert}%
\DeclarePairedDelimiter\norm{\lVert}{\rVert}%
\DeclarePairedDelimiter\ceil{\lceil}{\rceil}
\DeclarePairedDelimiter\floor{\lfloor}{\rfloor}

\newcommand{\TODO}[1]{\textcolor{red}{TODO: #1}}
\newcommand{\textdiff}[1]{\textcolor{red}{#1}}
\newcommand{\citemissing}{\textcolor{red}{(cite?)}}

\newcommand{\normt}[1]{\left\lVert#1\right\rVert_2}
\newcommand{\normtmu}[1]{\left\lVert#1\right\rVert_{2, \mu}}
\newcommand{\normm}[1]{\left\lVert#1\right\rVert}
\newcommand{\norminf}[1]{\left\lVert#1\right\rVert_\infty}
\newcommand{\normtt}[1]{\left\lVert#1\right\rVert^2_2}

\newcommand{\half}{\frac{1}{2}}
\newcommand{\fourth}{\frac{1}{4}}
\newcommand{\vect}[1]{\overrightafrrow{\textbf{#1}}}
\newcommand{\phat}{\hat{p}}
\newcommand{\KL}[2]{D_{KL}(#1||#2)}
\newcommand{\TV}[2]{D_{TV}(#1||#2)}
\newcommand{\ind}[1]{1[#1]}
\newcommand{\pardiv}[1]{\frac{\partial}{\partial #1}}
\newcommand{\parHess}[1]{\frac{\partial^2}{\partial #1 ^2}}
\newcommand{\Xhat}{{\hat{X}}}
\newcommand{\xhat}{{\hat{x}}}
\newcommand{\defeq}{\mathrel{\stackrel{\makebox[0pt]{\mbox{\normalfont\tiny def}}}{=}}}

\newcommand{\argmax}[1]{\underset{#1}{\textrm{argmax}}\ }
\newcommand{\argmin}[1]{\underset{#1}{\textrm{argmin}}\ }
\newcommand{\grad}[1]{\nabla{#1}}
\newcommand{\innerp}[2]{\langle{#1,#2}\rangle}
\newcommand{\Hess}[1]{\nabla^2{#1}}
\newcommand{\EXP}[1]{\text{exp}\{#1\}}

% debug q
\newcommand{\Proj}{\Pi}
\newcommand{\Projmu}{\Pi_\mu}
\newcommand{\trans}{T}
\newcommand{\backup}{\mathcal{T}}
\newcommand{\Qclass}{\mathcal{Q}}
\newcommand{\ReplayBuffer}{\mathcal{B}}
\newcommand{\ltwonorm}{L_2}
\newcommand{\lpnorm}{L_p}
\newcommand{\linfnorm}{L_\infty}

\newcommand{\UniformVec}{U[-1,1]^{32}}
\usepackage{semicrunch}
\usepackage{wrapfig,lipsum,booktabs}

% Attempt to make hyperref and algorithmic work together better:
\newcommand{\theHalgorithm}{\arabic{algorithm}}

\usepackage{amsthm}
\newtheorem{theorem}{Theorem}[section]
\newtheorem{proposition}{Proposition}[section]
\newtheorem{corollary}{Corollary}[theorem]
\newtheorem{lemma}{Lemma}[theorem]
\newtheorem{corollaryp}{Corollary}[proposition]
\newtheorem{definition}{Definition}[section]


% Use the following line for the initial blind version submitted for review:
 %\usepackage{icml2019}

% If accepted, instead use the following line for the camera-ready submission:
\usepackage[accepted]{icml2019}

% The \icmltitle you define below is probably too long as a header.
% Therefore, a short form for the running title is supplied here:
%\icmltitlerunning{Diagnosing Bottlenecks in Deep Q-learning Algorithms}

\begin{document}

\twocolumn[
\icmltitle{Diagnosing Bottlenecks in Deep Q-learning Algorithms}

% It is OKAY to include author information, even for blind
% submissions: the style file will automatically remove it for you
% unless you've provided the [accepted] option to the icml2019
% package.

% List of affiliations: The first argument should be a (short)
% identifier you will use later to specify author affiliations
% Academic affiliations should list Department, University, City, Region, Country
% Industry affiliations should list Company, City, Region, Country

% You can specify symbols, otherwise they are numbered in order.
% Ideally, you should not use this facility. Affiliations will be numbered
% in order of appearance and this is the preferred way.
\icmlsetsymbol{equal}{*}

\begin{icmlauthorlist}
\icmlauthor{Justin Fu}{equal,to}
\icmlauthor{Aviral Kumar}{equal,to}
\icmlauthor{Matthew Soh}{to}
\icmlauthor{Sergey Levine}{to}
% \icmlauthor{Fiuea Rrrr}{to}
% \icmlauthor{Tateu H.~Yasehe}{ed,to,goo}
% \icmlauthor{Aaoeu Iasoh}{goo}
% \icmlauthor{Buiui Eueu}{ed}
% \icmlauthor{Aeuia Zzzz}{ed}
% \icmlauthor{Bieea C.~Yyyy}{to,goo}
% \icmlauthor{Teoau Xxxx}{ed}
% \icmlauthor{Eee Pppp}{ed}
\end{icmlauthorlist}

\icmlaffiliation{to}{UC Berkeley}
% \icmlaffiliation{goo}{Google Brain}
% \icmlaffiliation{ed}{School of Computation, University of Edenborrow, Edenborrow, United Kingdom}

\icmlcorrespondingauthor{Justin Fu}{\texttt{justinjfu@eecs.berkeley.edu}}
\icmlcorrespondingauthor{Aviral Kumar}{\texttt{aviralk@berkeley.edu}}

% You may provide any keywords that you
% find helpful for describing your paper; these are used to populate
% the "keywords" metadata in the PDF but will not be shown in the document
% \icmlkeywords{Machine Learning, ICML}

\vskip 0.3in
]

% this must go after the closing bracket ] following \twocolumn[ ...

% This command actually creates the footnote in the first column
% listing the affiliations and the copyright notice.
% The command takes one argument, which is text to display at the start of the footnote.
% The \icmlEqualContribution command is standard text for equal contribution.
% Remove it (just {}) if you do not need this facility.

\printAffiliationsAndNotice{\icmlEqualContribution}  % leave blank if no need to mention equal contribution

%\printAffiliationsAndNotice{$^*$Equal Contribution} % otherwise use the standard text.

\begin{abstract}
Q-learning methods are a common class of algorithms used in reinforcement learning (RL). However, their behavior with function approximation, especially with neural networks, is poorly understood theoretically and empirically. In this work, we aim to experimentally investigate potential issues in Q-learning, by means of a "unit testing" framework where we can utilize oracles to disentangle sources of error. 
Specifically, we investigate questions related to function approximation, sampling error and nonstationarity, and where available, verify if trends found in oracle settings hold true with deep RL methods.
We find that large neural network architectures have many benefits with regards to learning stability; offer several practical compensations for overfitting; and develop a novel sampling method based on explicitly compensating for function approximation error that yields fair improvement on high-dimensional continuous control domains. 
\end{abstract}

\section{Introduction}
Q-learning algorithms, which are based on approximating state-action value functions, are an efficient and commonly used class of RL methods. Q-learning methods have several very appealing properties: they are relatively sample-efficient when compared to policy gradient methods, and they allow for off-policy learning. This makes them an appealing choice for a wide range of tasks, from robotic control~\citep{kalashnikov18} and video game AI~\citep{Mnih2015} to off-policy learning from historical data for recommender systems~\citep{shani2005recommender}. However, although the basic tabular Q-learning algorithm is convergent and admits theoretical analysis~\cite{suttonrlbook}, its counterpart with function approximation is poorly understood. 
In this paper, we aim to investigate the degree to which potential issues with Q-learning manifest in practice. 
We empirically analyze aspects of the Q-learning method in a \emph{unit testing} framework, where we can employ oracle solvers to obtain ground truth Q-functions and distributions for exact analysis. We investigate the following questions:

\textbf{1) What is the effect of function approximation on convergence?}
Many practical RL problems require function approximation to handle large or continuous state spaces. However, the behavior of Q-learning methods under function approximation is not well understood -- there are simple counterexamples where the method diverges~\citep{Baird1995}. 
To investigate these problems, we study the convergence behavior of Q-learning methods with function approximation by varying the function approximator power and analyzing the quality of the solution found. 
We find, somewhat surprisingly, that divergence rarely occurs, and that function approximation error is not a major problem in Q-learning algorithms when the function approximator is powerful. This makes sense in light of the theory: a high-capacity approximator can perform an accurate projection of the bellman Backup, thus mitigating potential convergence issues due to function approximation. (Section~\ref{sec:function_approx})

\textbf{2) What is the effect of sampling error and overfitting?}
RL is used to solve problems where we do not have access to the transition function of the MDP. Thus, Q-learning methods need to learn by collecting samples in the environment, and minimizing errors on samples potentially leads to overfitting. We experimentally show that overfitting exists in practice by performing ablation studies on the number of gradient steps, and by demonstrating that oracle based early stopping techniques can be used to improve performance of Q-learning algorithms. (Section~\ref{sec:overfitting}).
%Thus, in our experiments we quantify the amount of overfitting which happens in practice, incorporating a variety of metrics, and performing a number of ablations and investigate methods to mitigate its effects.

\textbf{3) What is the effect of distribution shift and a moving target?}
The standard formulation of Q-learning prescribes an update rule, with no corresponding objective function~\citep{Sutton09b}. This results in a process which optimizes an objective that is non-stationary in two ways: the target values are updated during training (the \emph{moving target} problem), and the distribution under which the Bellman error is optimized changes, as samples are drawn from different policies (the \emph{distribution shift} problem). These properties can make convergence behavior difficult to understand, and prior works have hypothesized that nonstationarity is a source of instability~\citep{Mnih2015, Lillicrap2015}. {We develop metrics to quantify the amount of distribution shift and performance change due to non-stationary targets. Surprisingly, we find that in a controlled experiment, distributional shift and non-stationary targets do not correlate with a reduction in performance, and some well-performing methods incur high distribution shift.}

\textbf{4) What is the best sampling or weighting distribution?}
Deeply tied to the distribution shift problem is the choice of which distribution to sample from. Do moving distributions cause instability, as Q-values trained on one distribution are evaluated under another in subsequent iterations?
Researchers have often noted that on-policy samples are typically superior to off-policy samples~\citep{suttonrlbook}, and there are several theoretical results that highlight favorable convergence properties under on-policy samples. However, there is little theoretical guidance on how to pick distributions so as to maximize learning. To this end, we investigate several choices for the sampling distribution. {Surprisingly, we find that on-policy training distributions are not always preferable, and that broader, higher-entropy distributions often perform better, regardless of distributional shift. Motivated by our findings, we propose a novel weighting distribution, adversarial feature matching (AFM), which is explicitly compensates for function approximator error, while still producing high-entropy sampling distributions.}

In summary, we introduce a unit testing framework for Q-learning to disentangle potential bottlenecks where approximate components are replaced by oracles. This allows for controlled analysis of different sources of error. We study various sources of instability and error in Q-learning algorithms on tabular domains, and show that many of these trends hold true in high dimensional domains. We then propose a novel sampling distribution that improve performance even on high-dimensional tasks. %Our overall aim is to offer practical guidance for designing RL algorithms, as well as to identify important issues to solve in future research.

\section{Preliminaries}
\label{sec:backrgound}
Q-learning algorithms aim to solve a Markov decision process (MDP) by learning the optimal state-action value function, or Q-function. We define an MDP as a tuple $(\mathcal{S}, \mathcal{A}, \trans, R, \gamma)$. $\mathcal{S}, \mathcal{A}$ represent the state and action spaces, respectively. $\trans(s' | s, a)$ and $R(s,a)$ represent the dynamics (transition distribution) and reward function, respectively, and $\gamma \in (0,1)$ represents the discount factor. Letting $\rho_0(s)$ denote the initial state distribution, the goal in RL is to find a policy $\pi(a|s)$ that maximizes the expected cumulative discounted rewards, known as the \textit{returns}:
 $\pi^* = \argmax{\pi} E_{s_0 \sim \rho_0, s_{t+1} \sim \trans, a_t \sim \pi}\left[\sum_{t=0}^\infty \gamma^t R(s_t, a_t)\right] $
The quantity of interest in many Q-learning methods are the optimal state ($V^*(s)$) and state-action ($Q^*(s,a)$) value functions, which give the expected future return of the optimal policy starting from a particular state or state-action pair. Q-learning algorithms are based on iterating the Bellman backup operator $\backup$, defined as
\begin{align*}
(\backup Q)(s, a) &= R(s, a) + \gamma E_{s' \sim \trans}[V(s')]\\
V(s) &= \max_{a'} Q(s, a')
\end{align*}
Q-iteration is a dynamic programming algorithm that iterates the Bellman backup $Q^{t+1} \leftarrow \backup Q^t$. Because the Bellman backup is a $\gamma$-contraction in the $\linfnorm$ norm, and $Q^*$ is its fixed point, Q-iteration can be shown to converge to $Q^*$~\citep{suttonrlbook}. A deterministic optimal policy can then be obtained as $\pi^*(s) = \textrm{argmax}_{a} Q^*(s,a)$.

When state spaces are large, function approximation is needed to represent the Q-values. This corresponds to \textit{fitted Q-iteration} (FQI)~\citep{Ernst05}, a form of approximate dynamic programming (ADP), which forms the basis of modern deep RL methods such as DQN~\citep{Mnih2015}.
FQI projects the values of the Bellman backup onto a family of Q-function approximators $\Qclass$: $Q^{t+1} \leftarrow \Projmu(\backup Q^t)$.
$\Projmu$ denotes a $\mu$-weighted $\ltwonorm$ projection, which minimizes the \textit{Bellman error} via supervised learning:
\begin{equation}
\label{eqn:bellman_projection} 
\Projmu(Q) \defeq 
\argmin{Q' \in \Qclass} E_{s,a \sim \mu}[(Q'(s,a) - Q(s,a))^2]
 .\end{equation}
The values produced by the Bellman backup, $(\backup Q^t)(s,a)$ are commonly referred to as \textit{target values}, and when neural networks are used for function approximation, the previous Q-function $Q^t(s,a)$ is referred to as the \textit{target network}. In this work, we distinguish between the cases when the Bellman error is estimated with Monte-Carlo sampling or computed exactly (see Section~\ref{sec:setup_algos}). The sampled variant corresponds to FQI as described in the literature~\citep{Ernst05,Riedmiller2005}, while the exact variant is an example of conventional ADP methods~\citep{Bertsekas96}. 
Convergence guarantees for Q-iteration do not cleanly translate to FQI. $\Projmu$ is an $\ltwonorm$ projection, but $\backup$ is a contraction in the $\linfnorm$ norm -- this norm mismatch means the composition of the backup and projection is no longer guaranteed to be a contraction under any norm~\citep{Bertsekas96}, and hence convergence is not guaranteed.

A related branch of Q-learning methods are \textit{online Q-learning} methods,
in which Q-values are updated while samples are being collected in the MDP. This includes classic algorithms such as Watkin's Q-learning~\citep{Watkins1992}. Online Q-learning methods can be viewed as a form of stochastic approximation applied to Q-iteration and FQI~\citep{Bertsekas96}, and share many of its theoretical properties~\citep{tsitsiklis1994asynchronous,szepesvari1998asymptotic}.
Modern deep RL algorithms such as DQN~\citep{Mnih2015} have characteristics of both online Q-learning and FQI -- using replay buffers means the sampling distribution $\mu$ changes very little between target updates (see Section~\ref{sec:distr_shift}), and target networks are justified from the viewpoint of FQI. Because FQI corresponds to the case when the sampling distribution is static between target updates, the behavior of modern deep RL methods more closely resembles FQI than a true online method without target networks.


\section{Experimental Setup}
\label{sec:setup}
Our experimental setup is centered around \emph{unit-testing}. We evaluate a spectrum of Q-learning algorithms, starting with exact dynamic programming and replacing exact components with practical approximations, until the algorithm resembles modern deep Q-learning methods. 
%We then introduce a suite of tabular environments where oracle solutions can be computed, to aid in diagnosis, as well as testing in high-dimensional environments to verify our hypotheses.


\subsection{Algorithms}
\label{sec:setup_algos}
We use three different Q-learning variants, each of which controls for a specific source of approximation error -- 
Exact-FQI, Sampling-FQI, and Replay-FQI. 
We use FQI as a basis for our controlled analysis, as it strongly resembles modern deep RL algorithms while allowing us to separately isolate target values, update rates, and the number of samples used for each iteration. We then confirm that the observed trends hold with several state-of-the-art deep RL methods (SAC~\citep{Haarnoja2017}, TD3~\citep{pmlr-v80-fujimoto18a}) on standard benchmark problems.
%Although FQI is not exactly identical to commonly used deep RL methods, such as DQN~\citep{Mnih2015}, DDPG~\citep{Lillicrap2015},TD3~\citep{pmlr-v80-fujimoto18a}, and SAC~\citep{Haarnoja2017}, it is structurally similar and, when the replay buffer for the commonly used methods becomes large, the difference becomes negligible, since the sampling distribution changes very little between target network updates. 
%However, FQI methods are much more amenable for controlled analysis, since we can separately isolate target values, update rates, and the number of samples used for each iteration. We therefore use variants of FQI as the basis for our analysis, but we also confirm that similar trends hold with more commonly used algorithms on standard benchmark problems.

\begin{figure*}[ttt!]
\begin{small}
\begin{minipage}[t]{0.33\linewidth}
\begin{algorithm}[H]
\small
\caption{Exact-FQI}
\label{alg:fqiexact}
\begin{algorithmic}[1]
    \STATE Initialize Q-value approximator $Q_\theta(s,a)$.
    \FOR{step $t$ in \{1, \dots, N\}}
        \item[]
        \item[]
        \item[]
        \STATE Evaluate $Q_{\theta^t}(s,a)$ at all states.
        \STATE Compute exact target values at all states. \\
        $y(s,a) = r(s,a) + \gamma E_{s'}[ V_{\theta^t}(s')]$ 
        \STATE Minimize projection loss with respect to $\mu$: \\
        $\argmin{\theta} E_\mu[(Q_\theta(s,a) - y(s,a))^2]$
    \ENDFOR
\end{algorithmic}
\end{algorithm}
\end{minipage}
\begin{minipage}[t]{0.33\linewidth}
\begin{algorithm}[H]
\small
\caption{Sampled-FQI}
\label{alg:fqisampled}
\begin{algorithmic}[1]
    \STATE Initialize Q-value approximator $Q_\theta(s,a)$.
    \FOR{step $t$ in \{1, \dots, N\}}
        \item[]
        \item[]
        \STATE \textdiff{Collect $M$ samples from $\mu$.}
        \STATE Evaluate $Q_{\theta^t}(s,a)$ \textdiff{on samples.}
        \STATE Compute sampled target values \textdiff{on samples.}\\
        $\hat{y}_i = r_i + \gamma V_{\theta^t}(s'_i)$ 
        \STATE Minimize projection loss with respect to \textdiff{samples}: \\
        $ \argmin{\theta} \frac{1}{M}\sum_{i=1}^M (Q_\theta(s_i,a_i) - y_i)^2$
    \ENDFOR
\end{algorithmic}
\end{algorithm}
\end{minipage}
\begin{minipage}[t]{0.33\linewidth}
\begin{algorithm}[H]
\small
\caption{Replay-FQI}
\label{alg:fqireplay}
\begin{algorithmic}[1]
    \STATE Initialize Q-value approximator $Q_\theta(s,a)$, \textdiff{replay buffer $\ReplayBuffer$}.
    \FOR{step $t$ in \{1, \dots, N\}}
        \STATE \textdiff{Collect $K$ online samples from $\mu$.}
        \STATE \textdiff{Append online samples to buffer $\ReplayBuffer$.}
        \STATE Collect $M$ samples from $\ReplayBuffer$.
        %\STATE \textbf{Collect $K$ replay samples $(s_k, a_k, s'_k, r_k)$ from $\ReplayBuffer$}.
        \STATE Evaluate $Q_{\theta^t}(s,a)$ on samples.
        \STATE Compute sampled target values on samples\\
        $\hat{y}_i = r_i + \gamma V_{\theta^t}(s'_i)$ 
        \STATE Minimize projection loss with respect to samples: \\
        $ \argmin{\theta} \frac{1}{M}\sum_{i=1}^M (Q_\theta(s_i,a_i) - y_i)^2$
    \ENDFOR
\end{algorithmic}
\end{algorithm}
\end{minipage}
\end{small}
\end{figure*}


\textbf{Exact-FQI} (Algorithm~\ref{alg:fqiexact}): Exact-FQI uses known dynamics and reward functions and computes the backup and projection on all state-action tuples, without sampling error. We use Exact-FQI to study convergence, distribution shift (by varying weighting distributions on transitions), and function approximation in the absence of sampling error. 
%Exact-FQI eliminates errors due to sampling states, and computing inexact, sampled backups.

\textbf{Sampled-FQI} (Algorithm~\ref{alg:fqisampled}): Sampled-FQI is a special case of Exact-FQI,
where the Bellman error is approximated with Monte-Carlo estimates from a sampling distribution $\mu$, and the Bellman backup is approximated with samples from the dynamics as $r(s,a) + \gamma \max_{a'}Q(s', a')$. We use Sampled-FQI to study effects of overfitting. Sampled-FQI incorporates errors arising from function approximation, sampling, and distribution shift.

\textbf{Replay-FQI} (Algorithm~\ref{alg:fqireplay}): Replay-FQI is a special case of Sampled-FQI that uses a \textit{replay buffer}~\citep{lin1992replay},
that saves past transition samples $(s, a, s', r)$, which are used for computing Bellman error. Replay-FQI strongle resembles DQN~\cite{Mnih2015}, but lacking the online updates that allow $\mu$ to change within an FQI iteration. 
With large replay buffers, we expect the difference between Replay-FQI and DQN to be minimal as $\mu$ changes slowly.

We additionally investigate the following choices of weighting distributions ($\mu$) for the Bellman error:
%When sampling the Bellman error, these can be implemented by sampling directly from the distribution, or via importance sampling.

\textbf{Unif$(s,a)$}: Uniform weights over state-action space. 
%This is the weighting distribution typically used by dynamic programming algorithms, such as FQI.

\textbf{$\pi(s,a)$}: The on-policy state-action marginal induced by $\pi$.

\textbf{$\pi^*(s,a)$}: The state-action marginal induced by $\pi^*$.

\textbf{Random$(s,a)$}: State-action marginal induced by executing uniformly random actions.

\textbf{Prioritized$(s,a)$}: Weights Bellman errors proportional to $|Q(s,a)-\backup Q(s,a)|$. This is similar to prioritized replay~\citep{Schaul2015} with $\beta=0$.

\textbf{Replay$(s,a)$} and \textbf{Replay10$(s,a)$}: Averaged state-action marginal of all policies (or the last 10) produced during training. This simulates sampling from a replay buffer. 

\subsection{Domains}

We evaluate our methods on suite of tabular environments where we can compute oracle values. 
%This will help us compare, analyze and fix various sources of error by means of comparing the learned Q-functions to the true, oracle-compute Q-functions.
We selected 8 tabular domains, each with different qualitative attributes, including: gridworlds of varying sizes and observations, blind Cliffwalk~\citep{Schaul2015},
discretized Pendulum and Mountain Car based on OpenAI Gym~\citep{gym},
and a sparsely connected graph. Additional details can be found in Appendix~\ref{app:domains}. In order to provide consistent metrics across domains, we normalize returns and errors involving Q-functions (such as Bellman error) by the returns of the expert policy $\pi^*$ on each environment.


\subsection{Function Approximators}
Throughout our experiments, we use 2-layer ReLU networks, denoted by a tuple $(N, N)$ where N represents the number of units in a layer. The ``Tabular'' architecture refers to the case when no function approximation is used. 

\subsection{High-Dimensional Testing}

In addition to diagnostic experiments on tabular domains, we also wish to see if the observed trends hold true on high-dimensional environments. Thus, we include experiments on continuous control tasks in the OpenAI Gym benchmark~\citep{gym} (HalfCheetah-v2, Hopper-v2, Ant-v2, Walker2d-v2). In continuous domains, computing the maximum over actions of the Q-value is difficult ($\max_a Q(s,a)$). A common choice in this case is to use an ``actor'' function to approximate $\arg\max_a Q(s,a)$~\cite{Lillicrap2015,pmlr-v80-fujimoto18a,Haarnoja18}. This approach resembles Replay-FQI, but using the actor network in place of the max.
%\vspace{-20pt}
\section{Function Approximation and Convergence}
\label{sec:function_approx}

%The first issue we investigate is the connection between function approximation and convergence properties.

%\subsection{Technical Background}
This interaction between approximation and convergence has been a long-studied topic in RL and ADP. In control theory, it is closely related to the problems of state-aliasing or interference~\citep{Farrell95}. \citet{Baird1995} introduces a counterexample in which Watkin's Q-learning with linear approximators causes unbounded divergence. However, several works have noted that divergence need not occur. \citet{munos2005erroravi,antos2008concentrability,farahmand2010error} address the norm-mismatch problem by showing convergence guarantees in $\lpnorm$-norms, at the price of introducing a \textit{concentrability coefficient} that worsens the error bound (and is potentially infinite for deterministic MDPs). In policy evaluation, \citet{Tsitsiklis1997} prove that on-policy TD-learning with linear approximators converges, and methods such as GTD~\citep{Sutton09b} and ETD~\citep{Sutton2016} have extended results to off-policy cases.
In the control scenario, convergent algorithms such as SBEED~\citep{Dai2018} and Greedy-GQ~\citep{Maei2010} have been developed. Concurrently to us,\citet{VanHesselt2018} experimentally find that unbounded divergence rarely occurs with DQN variants on Atari games. 

\subsection{How does function approximation affect convergence and solution suboptimality?}
The crucial quantities we wish to measure are a trend between function approximation and performance, and a measure for the bias in the learning procedure introduced by function approximation.
Using Exact-FQI with uniform weighting (to remove sampling error), we plot the returns of the learned policy, and the $\linfnorm$ error between $Q^*$ and the solution found by Exact-FQI ($\lim_{t \to \infty} (\Projmu \backup)^t Q^0$) and the projection of the optimal solution  ($\Projmu Q^*$) in Fig.~\ref{fig:function_approx}. $\Projmu Q^*$ represents the best solution within model class, in absence of bootstrapping error. Thus, the difference between FQI error and projection error represents the additional bias introduced by FQI (the \textit{inherent Bellman error} of the function class~\citep{munos2008finite}). This is the potential gap that can be improved via better algorithm design.

\begin{figure}
\vspace{-0.5cm}
\caption{\label{fig:function_approx} Normalized returns and Q-function error with function approximation, averaged across domains and seeds. Small architectures show a large gap between the solution found by FQI (FQI Error) and the best solution within model class (Project Error).}
\centering
\includegraphics[width=0.7\linewidth]{chapters/diagnosing_q/images/fig_1.pdf}
% generated by plot_doodad_exact_fqi.py:returns_qstar_vs_arch
% data_dir = east1//2019-01-18-newenv-exact-weighting
\vspace{-0.8cm}
\end{figure}

We first note the obvious trend that smaller architectures produce lower returns and converge to worse solutions. 
However, we also find that smaller architectures introduce significant bias in the \textit{learning process}, and there is a significant gap between the solution found by Exact-FQI and the best solution within the model class. 
One explanation is that when the target is bootstrapped, we must represent all Q-functions along the path to the solution, and not only the final result~\citep{Bertsekas96}.
This observation implies that using large architectures is crucial not only because they have capacity to represent a better solution, but also because they are easier to train using bootstrapping. 
We also note that divergence rarely occurs in practice, occuring in only 0.9\% of our experiments (measured as the Q-values growing larger than 10 times $Q^*(s,a)$). 

For high-dimensional problems, we present experiments on varying the architecture of the Q-network in SAC~\cite{Haarnoja18} in Appendix Fig.~\ref{fig:size_sac}. We still observe that large networks have the best performance, and that divergence rarely happens even in high-dimensional continuous spaces. We briefly discuss theoretical intuitions on apparent discrepancy between the lack of unbounded divergence in relation known counterexamples in Appendix~\ref{app:bounded_error}.

\section{Sampling Error and Overfitting}
\label{sec:overfitting}

%A second source of error in minimizing the Bellman error, orthogonal to function approximation, is that of sampling or generalization error. The next issue we investigate is the effect of sampling error on Q-learning methods.

%\subsection{Technical Background}
\begin{wrapfigure}{r}{0.6\columnwidth}
\centering
\vspace{-10pt}
\includegraphics[scale=0.27]{chapters/diagnosing_q/images/samples_arch.pdf}
\vspace{-0.2cm}
\caption{\label{fig:sampling_256} Samples plotted with returns for a 256x256 network. More samples yields better performance.}
\vspace{-0.6cm}
%plot_sampling
%east1//2019-01-20-newenv-sample-weighting
\end{wrapfigure}

Approximate dynamic programming assumes that the projection of the Bellman backup (Eqn.~\ref{eqn:bellman_projection}) is computed exactly, but in reinforcement learning we can normally only compute the \textit{empirical Bellman error} over a finite set of samples. In the PAC framework, overfitting can be quantified by a bounded error in between the empirical and expected loss with high probability, which decays with sample size~\citep{Shalev2014}. \citet{munos2008finite, maillard2010finite, tosatto2017boosted} provide such PAC-bounds which account for sampling error in the context of Q-learning and value-based methods, and quantify the quality of the final solution in terms of sample complexity.

We analyze several key points that relate to sampling error. First, we show that Q-learning is prone to overfitting, and that this overfitting has a real impact on performance. We also show that the replay buffer is in fact an effective technique in addressing this issue, and discuss several methods to migitate the effects of overfitting in practice.

\subsection{Quantifying Overfitting}
We first quantify the amount of overfitting that happens during training, by varying the number of samples. In order provide comparable validation errors across different experiments, we fix a reference sequence of Q-functions, $Q^1, ... , Q^N$, obtained during an arbitrary run of FQI. We then retrace the training sequence, and minimize the projection error $\Projmu(Q^t)$ at each training iteration, using varying amounts of on-policy data or sampling from a replay buffer. We measure the validation error (the expected Bellman error) at each iteration under the on-policy distribution, plotted in Fig.~\ref{fig:sampling_validation_loss}. We note the obvious trend that more samples leads to significantly lower validation loss. A more interesting observation is that sampling from the replay buffer results in the lowest on-policy validation loss, despite bias due to distribution mismatch from sampling off-policy data. As we elaborate in Section~\ref{sec:analysis_nonstationarity}, we believe that replay buffers are effective because they reduce overfitting and have good sample coverage over the state space, not necessarily due to reducing the effects of nonstationarity.

Next, Fig.~\ref{fig:sampling_256} shows the relationship between number of samples and returns. We see that more samples leads to improved learning speed and a better final solution. A full sweep including architectures is presented in Appendix Fig.~\ref{fig:sampling_arch_sweep}. Despite overfitting being an issue, larger architectures still perform better because the bias introduced by smaller architectures dominates.

\begin{figure*}[ttt!]
\begin{minipage}[t]{0.32\linewidth}

\includegraphics[width=0.95\columnwidth]{chapters/diagnosing_q/images/overfitting.pdf}
%plot_overfitting
%central1//2019-01-21-overfitting-coupled
\caption{\label{fig:sampling_validation_loss} On-policy validation losses for varying amounts of on-policy data (or replay buffer), averaged across environments and seeds. Note that sampling from the replay buffer has lower on-policy validation loss, despite bias from distribution shift.}
\end{minipage}
~\vline~
\begin{minipage}[t]{0.32\linewidth}
\includegraphics[trim={0 0 7.0cm 0},clip,width=0.95\columnwidth]{chapters/diagnosing_q/images/grad_steps_fqi}
%plot_overfitting
%central1//2019-01-21-overfitting-coupled
\caption{\label{fig:fqi_grad_sweep}Normalized returns plotted over training iterations (32 samples are taken per iteration), for different ratios of gradient steps per sample using Replay-FQI. We observe that intermediate values of gradient steps work best, and too many gradient steps hinders performance.}
\end{minipage}
~\vline~
\begin{minipage}[t]{0.32\linewidth}
\includegraphics[trim={0 0 4.4cm 0},clip,width=0.95\columnwidth]{chapters/diagnosing_q/images/validation_stop}
%plot_validation_stop
%west1//2019-02-20-replay-validation-stop
\caption{\label{fig:validation_stop}Normalized returns plotted over training iterations (32 samples are taken per iteration), for different early stopping methods using Replay-FQI. We observe that using proper early stopping can result in a modest performance increase.}
\end{minipage}
\end{figure*}

\subsection{How can we compensate for overfitting?}

Finally, we discuss methods to compensate for overfitting. One common method for reducing overfitting is to regularize the function approximator to reduce its capacity. However, we have seen that weaker architectures can give rise to suboptimal convergence. Instead, we study \textit{early stopping} methods to mitigate overfitting without reducing model size.
We first note that the number of gradient steps taken per sample in the projection step has an important effect on performance -- too few steps and the algorithm learns slowly, but too many steps and the algorithm may initially learn quickly but overfit. To show this, we run an ablation over the number of gradient steps taken per sample in Replay-FQI and TD3 (TD3 uses 1 by default). Results for FQI are shown in Fig.~\ref{fig:fqi_grad_sweep}, and for TD3 in Appendix Fig.~\ref{fig:td3_grad_sweep}.

In order to understand whether early stopping criteria can reduce overfitting, we employ \emph{oracle} stopping rules to provide an ``upper bound'' on the best potential improvement. We try two criteria for setting the number of gradient steps: the expected Bellman error and the expected returns of the greedy policy (oracle returns). We implement both methods by running the projection step of FQI to convergence, and retroactively selecting the intermediate Q-function which is judged best by the evaluation metric. Using oracle stopping metrics results in a modest boost in performance in tabular domains (Fig.~\ref{fig:validation_stop}). Thus, we believe that there is promise in further improving such early-stopping methods for reducing overfitting in deep RL algorithms.

We can draw a few conclusions from these experiments. First, overfitting is indeed an issue with Q-learning, and too many gradient steps or too few samples can lead to poor performance. Second, replay buffers and early stopping can mitigate the effects of overfitting. Third, although overfitting is a problem, large architectures are still preferred, because the bias from function approximation outweighs the increased overfitting from using large models.
\section{Non-Stationarity}
\label{sec:analysis_nonstationarity}

%In this section, we discuss issues related to the non-stationarity of the Q-learning process (relating to the Bellman backup and Bellman error minimization).

%\subsection{Technical Background}
Instability in Q-learning methods is often attributed to the nonstationarity of the objective \citep{Lillicrap2015,Mnih2015}. 
Nonstationarity occurs in two places: in the changing target values $\backup Q$, and in a changing weighting distribution $\mu$ (``distribution shift''). 
Note that a non-stationary objective, by itself, is not indicative of instability. For example, gradient descent can be viewed as successively minimizing linear approximations to a function: for gradient descent on $f$ with parameter $\theta$ and learning rate $\alpha$, we have the ``moving'' objective $\theta^{t+1} = \argmin{\theta}\{ \theta^T \nabla_\theta f(\theta^t) - \frac{1}{2\alpha} \normtt{\theta - \theta^t} \} = \theta^t - \alpha \nabla_\theta f(\theta^t)$. 
However, the fact that the Q-learning algorithm prescribes an update rule and not a stationary objective complicates analysis. Indeed, the motivation behind algorithms such as GTD~\citep{Sutton09a, Sutton09b}, approximate linear programming~\citep{de2002alp}, and residual methods~\citep{Baird1995,scherrer2010residual} can be seen as introducing a stationary objective that can be optimized with standard methods such as gradient descent.
%Therefore, a key question to investigate is whether these non-stationarities are detrimental to the learning process.
% \vspace{-25pt}

\subsection{Does a moving target cause instability in the absence of a moving distribution?}

To study the moving target problem, we first isolate the speed at which the target changes. To this end, we define the $\alpha$-smoothed Bellman backup, $\backup^\alpha$, which computes an exponentially smoothed update as follows: 
$\backup^{\alpha}Q = \alpha \backup Q + (1-\alpha)Q$.
% \vspace{-5pt}
This scheme is inspired by the soft target update used in algorithms such as DDPG~\citep{Lillicrap2015} and SAC~\citep{Haarnoja2017} to improve the stability of learning. Standard Q-iteration uses a ``hard'' update where $\alpha=1$. A soft target update weakens the contraction of Q-iteration from $\gamma$ to $1-\alpha+\alpha\gamma$ (see Appendix~\ref{app:alpha_smoothed_q}),
so we expect slower convergence, but perhaps it is more stable under heavy function approximation error. We performed experiments with this modified backup using Exact-FQI under the $\text{Unif}(s,a)$ weighting distribution.

Our results are presented in Appendix Fig.~\ref{fig:smooth_fqi}.
We find that the most cases, the hard update with $\alpha=1$ results in the fastest convergence and highest asymptotic performance. However, for smaller architectures, $4 \times 4$ and $16 \times 16$, lower values of $\alpha$ (such as 0.1) achieve slightly higher asymptotic performance. Thus, while more expressive architectures are still stable under fast-changing targets, we believe that a slowly moving target may have benefits under heavy approximation error. This evidence points to either using large function approximators, in line with the conclusions drawn in the previous sections, or slowing the target updates on problems with high approximation error.

\subsection{Does distribution shift impact performance?}
\label{sec:distr_shift}

\begin{figure}[tb]
\caption{\label{fig:distribution_shift_tv_loss} Distribution shift and loss shift plotted against time. Prioritized and on-policy distributions induce the greatest shift, whereas replay buffers greatly reduce the amount of shift.}
\includegraphics[width=0.51\columnwidth]{chapters/diagnosing_q/images/dist_shift_tv3.pdf}
\includegraphics[width=0.47\columnwidth]{chapters/diagnosing_q/images/dist_shift_loss_tv3.pdf}
% \vspace{-10pt}
% generated by plot_distribution_shift.py:plot_tv_loss_over_time
% data_dir = east1//2019-01-18-newenv-exact-weighting
\end{figure}


To study the distribution shift problem, we exactly compute the amount of distribution shift between iterations in total-variation distance, $D_{TV}(\mu^{t+1} || \mu^{t})$ and the ``loss shift'':
$\mathbb{E}_{\mu^{t+1}}[ (Q^{t} - \backup Q^{t})^2] - \mathbb{E}_{\mu^{t}}[ (Q^{t} - \backup Q^{t})^2]$.
The loss shift quantifies the Bellman error objective when evaluated under a new distribution - if the distribution shifts to previously unseen states, we would expect a highly inaccurate Q-value in such states, leading to high loss shift.

% \begin{figure}[ht]

% \end{figure}


We run our experiments using Exact-FQI with a 256x256 layer architecture, and plot the distribution discrepancy and the loss discrepancy in Fig.~\ref{fig:distribution_shift_tv_loss}. 
We find that Prioritized$(s,a)$ has the greatest shift, followed by on-policy variants. Replay buffers greatly reduce distribution shift compared to on-policy learning, which is similar to the de-correlation argument cited for its use by~\citet{Mnih2015}.
However, we find that this metric correlates little with the performance of FQI (Fig.~\ref{fig:distribution_shift_vs_returns}). For example, prioritized weighting performs well yet has high distribution shift.

Overall, our experiments indicate that nonstationarities in both distributions and target values, when isolated, do not cause significant stability issues. Instead, other factors such as sampling error and function approximation appear to have more significant effects on performance.
%Therefore, we investigate how to design a \emph{better} sampling distribution, without regard to nonstationarity, in the next section.
%In the light of these findings, we might therefore ask: can we design a \emph{better} sampling distribution, without regard for distributional shift and with regard for high-entropy, that results in better final performance, and is realizable in practice? We investigate this in the following section.

\vspace{-0.2cm}
\section{Impact of Distributional Shift}
\label{sec:sampling_distributions}
\vspace{-0.2cm}

%As alluded to in Section~\ref{sec:analysis_nonstationarity}, the choice of sampling distribution $\mu$ is an important design decision can have a large impact on performance. Indeed, it is not immediately clear which distribution is ideal for Q-learning. In this section, we hope to shed some light on this issue.

%\subsection{Technical Background}
% Off-policy data has been cited as one of the ``deadly triads'' for Q-learning~\citep{suttonrlbook}, which has potential to cause instabilities in learning. On-policy distributions~\citep{Tsitsiklis1997} and fixed behavior distributions~\citep{Sutton09a,Maei2010} have often been targeted for theoretical convergence analysis, and many works use importance sampling to correct for off-policyness~\citep{precup2001offpol, munos2016safe}
% However, to our knowledge, there is relatively little guidance which compares how different weighting distributions compare in terms of convergence rate and final solutions.

% Off-policy RL methods applied to the offline RL problem would typically attempt to learn an optimal policy, even though the static dataset may not be generated from an optimal policy. As a result, one issue to emerge is that of \emph{distributional shift}: while these methods can only train a model of the value-function and the policy using state-action tuples from the data, these models  will need to make correct predictions on states and actions sampled from a different distribution encountered upon deployment. In general, models trained via machine learning are not robust when the distribution of inputs changes, indicating that distributional shift can be challenge for off-policy RL methods. Is distributional shift a challenge in offline RL?   

Off-policy RL methods applied to the offline RL problem would typically attempt to learn an optimal policy, even though the static dataset may not be generated from an optimal policy. As a result, one issue that emerges is that of \emph{distributional shift}: while these methods train a model of the value-function and the policy only using state-action tuples from the data, upon deployment, these models will need to make correct predictions on states and actions sampled from a different distribution of the learned policy. In general, models trained via machine learning are not robust when the distribution of inputs changes, indicating that distributional shift can be a challenge for off-policy RL methods. Is distributional shift a challenge in offline RL?

Indeed, theoretically, the effects of distributional shift have been quantified using the notion of a concentrability coefficient~\citep{munos2005erroravi}, a constant typically $\gg 1$, which provides a worst-case error bound on the performance of an off-policy RL method due to distributional shift. To evaluate if this challenge persists empirically as well, we will analyze Q-learning methods for various choices of data distributions in this section.

A crucial design decision we must make is to consider setups that do not confound distributional shift with access to limited data. Therefore, for our study, we provide the underlying algorithm oracle access to all state-action transitions in the MDP, but vary the \emph{distribution} over state-action pairs from which these transitions are sampled.

% Indeed, theoretically, the effects of distributional shift have been quantified using the notion of a concentrability coefficient~\citep{munos2005erroravi}, a constant typically $\ggt 1$, which provides a worst-case error bound on the performance of an off-policy RL method due to distributional shift. To evaluate if this challenge persists empirically as well, we will analyze Q-learning methods for various choices of data distributions in this section. 

% A crucial design decision we must make is to consider setups that do not confound distributional shift with access to limited data. Therefore, for our study, we provide the underlying algorithm oracle access to all state-action transitions in the MDP, but vary the \emph{distribution} over state-action pairs that these transitions are sampled according to.   
% ~\citep{NIPS2017_6913} suggests that when the state-distribution is fixed, the action distribution should be weighted by the optimal policy for residual Bellman errors. In deep RL, prioritized replay~\citep{Schaul2015}, and mixing replay buffer with on-policy data~\citep{hausknecht2016policy,zhang2018deeper} have been found to be effective. %We aim to empirically analyze multiple choices for weighting distributions to determine which are the most effective.

%\vspace{-10pt}
\vspace{-0.2cm}
\subsection{What Are the Best Data Distributions Without Sampling Error?}
\label{subsec:dist_shift_exact}
\vspace{-0.2cm}

% \begin{figure*}[ttt!]
% \begin{subfigure}{0.3\linewidth}
% \caption{\footnotesize \label{fig:distribution_shift_vs_returns} Average distribution shift across time for different data distributions, plotted against returns for a 256x256 model. We find that distribution shift does not have strong correlation with returns.}
% \includegraphics[width=0.99\columnwidth]{chapters/diagnosing_q/images/returns_vs_shift}
% % generated by plot_distribution_shift.py
% % data_dir = east1//2019-02-12-exact-weighting-distr-shift
% \vspace{-0.2in}
% \end{subfigure}
% ~\vline~
% \begin{subfigure}{0.3\linewidth}
% \caption{\footnotesize \label{fig:weighting_schemes} Weighting distribution versus architecture in Exact-FQI. Replay(s, a) consistently provides the highest performance. Note that Adversarial Feature Matching is comparable to Replay(s, a), but surprisingly better for small networks. }
% \includegraphics[width=0.99\columnwidth]{chapters/diagnosing_q/images/exact_fqi_schemes.pdf}
% % generated by plot_exact_weighting.py
% % data_dir = east1//2019-01-18-newenv-exact-weighting
% \vspace{-0.3in}
% \end{subfigure}
% ~\vline~
% \begin{subfigure}{0.3\linewidth}
% \caption{\footnotesize \label{fig:weighting_entropy_vs_returns} Normalized returns plotted against normalized entropy for different weighting distributions. All experiments use Exact-FQI with a 256x256 network. We see a general trend that high-entropy distributions lead to greater performance.}
% \includegraphics[width=0.99\columnwidth]{chapters/diagnosing_q/images/returns_vs_entropy}
% % generated by plot_distribution_shift.py
% % data_dir = east1//2019-02-12-exact-weighting-distr-shift
% \end{subfigure}
% \end{figure*}

\begin{wrapfigure}{r}{0.45\linewidth}
    \vspace{-0.2cm}
    \includegraphics[width=0.99\linewidth]{chapters/diagnosing_q/images/returns_vs_entropy}
    \caption{\footnotesize \label{fig:weighting_entropy_vs_returns} Normalized returns plotted against normalized entropy for different weighting distributions. All experiments assume access to all state-action pairs with a 256x256 Q-network. We see a general trend that high-entropy distributions lead to greater performance, corroborating the uniformity hypothesis.}
    \vspace{-0.2cm}
    % generated by plot_distribution_shift.py
    % data_dir = east1//2019-02-12-exact-weighting-distr-shift
\end{wrapfigure}
We begin by studying the effect of data distributions when disentangled from sampling error. We run Q-learning with an ablation over various Q-function network architectures and data distributions and report our aggregate results in Fig.~\ref{fig:weighting_entropy_vs_returns}. $\text{Unif}(\bs, \mathbf{a})$, $\text{Replay}(\bs, \mathbf{a})$ (using a replay buffer consisting of data from a mixture of policies with different degrees of optimality), and $\text{Prioritized}(\bs, \mathbf{a})$ (weighting induced by prioritized experience replay~\citep{Schaul2015}) consistently result in the highest returns across all architectures. On the other hand, relatively ``narrow'' data distributions, such as those induced the optimal policy ($\pi^*$) or only using a mixture of a few policies ($\text{Replay}(10)$) results in poor performance.
% We believe that these results are in favor of the \textit{uniformity} hypothesis: intuitively, the best distributions spread weight across larger support of the state-action space, reducing the amount of possible distributional shift. On the other hand, less-uniform distributions such as the state-action distirbution induced by the optimal policy, present multiple avenues to deviate away from the offline data distribution, resulting in larger distributional shift.
We believe that these results favor the \textit{uniformity} hypothesis: intuitively, the best distributions spread weight across a larger support of the state-action space, reducing the amount of possible distributional shift. On the other hand, less-uniform distributions, such as the state-action distribution induced by the optimal policy, present multiple avenues to deviate away from the offline data distribution, resulting in larger distributional shift.

\textbf{To summarize}, this indicates that narrow data distributions lead to worse performance compared to higher-entropy data distributions, indicating that distributional shift can have a significant impact on the performance of off-policy RL in the offline setting.
%These distributions generally result in the tightest contraction rates, and allow the Q-function to focus on high-error regions. 
%In the sampled setting, this observation motivates exploration algorithms that maximize state coverage (for example, ~\citet{hazan2018} solve an exploration objective which maximizes state-space entropy).
%However, note that in this particular experiment is distinct from exploration, as there is no sampling involved. 
%All states are observed, just with different weights, thus isolating the issue of distributions from the issue of sampling.
\documentclass[../thesis.tex]{subfiles}
\begin{document}
In this thesis, we have presented a number of advances towards learning based 3D object reconstruction. In Chapter \ref{chapter:CategoryShapes}, we presented an algorithm to learn category-specific deformable 3D models for diverse object categories and showed how to use them in conjunction with recognition systems to achieve fully automatic object localization and reconstruction from a single image. In Chapter \ref{chapter:Amodal}, we worked towards calibrating the size of the reconstructed objects by inferring their relative sizes and depths from a single image. We did this by leveraging amodal bounding boxes and reasoning about object co-occurences in image collections. We worked towards unifying single and multi-view reconstruction with a CNN in Chapter \ref{chapter:LSM} with Learnt Stereo Machines.

We have still just scratched the tip of the iceberg as alluded to in the limitations of current works in the chapters above. A number of challenges remain in tightly integrating semantic reasoning and 3D reconstruction (some of which we summarize below) and present for exciting directions for future work. 

\paragraph{Shape Representations:} In our works, we have explored using deformable meshes, voxel occupancy grids and depth maps as representations for shape. While each presents its own benefits, it is still unclear whether one has all the desirable properties for shape representations. For example, part compositionality plays a crucial role in the human visual system which none of the above representations exhibit. There have been promising works towards modelling shapes with primitives such as planes and simple shapes~\cite{abstractionTulsiani17} (cubes, cylinders \etc). Automatic discovery of such primitives which can be shared across object categories would allow for greater interpretability in reconstructions.

\paragraph{Learning without explicit supervision:} We, as humans, almost never have ``ground truth'' data to learn from - especially for 3D shapes. Our mental models are built from observing objects from different viewpoints, in various lighting conditions, by interacting with them \etc This remains a critical problem to solve for current learning systems to scale beyond available 3D datasets. Some recent works~\cite{tulsiani2017multi,zhou2017unsupervised} have investigated using motion and projection into novel views as a proxy for learning shapes. An exciting direction to investigate would be coupling active exploration with 3D shape inference. For example, a robot could derive shape cues for an object by trying to grasp/poke/manipulate it in certain ways.

\paragraph{Quality of reconstructions:} A common issue with current methods for learning-based shape inference methods is that they dont produce high quality outputs (in terms of accuracy, fidelity and resolution). This is in contrast with classical systems which produce extremely detailed models, albeit from far more images. Recent attempts at modelling high resolution grids with CNNs~\cite{hane2017hierarchical} present a promising direction towards detailed reconstructions with learning-based cues. A related task is modelling large spaces with such systems which couples with the question of what shape representations are amenable to such problems.

\paragraph{Related tasks:} While inferring 3D shape from 2D views is an end in itself, it can be especially useful for a number of complementary and upstream tasks. For example, estimating lighting and reflectance in scenes could benefit from learning-based 3D inference systems, particularly when learnt jointly~\cite{barronPAMI13}. More examples of properties for which are difficult to \textit{ground-truth} and tightly coupled with 3D reasoning are optical flow and scene flow. Learning systems could provide a very natural way of jointly modelling these variety of tasks with strong inter-dependencies. Implicit 3D reasoning could also play a critical role in end-to-end learning of policies in robotics~\cite{finn2016end}, \eg for navigating in 3D scenes or manipulating objects.

\end{document}

% Acknowledgements should only appear in the accepted version.
%\section*{Acknowledgements}

%\textbf{Do not} include acknowledgements in the initial version of
%the paper submitted for blind review.

%If a paper is accepted, the final camera-ready version can (and
%probably should) include acknowledgements. In this case, please
%place such acknowledgements in an unnumbered section at the
%end of the paper. Typically, this will include thanks to reviewers
%who gave useful comments, to colleagues who contributed to the ideas,
%and to funding agencies and corporate sponsors that provided financial
%support.
%\clearpage
\bibliography{example_paper}
\bibliographystyle{icml2019}


%%%%%%%%%%%%%%%%%%%%%%%%%%%%%%%%%%%%%%%%%%%%%%%%%%%%%%%%%%%%%%%%%%%%%%%%%%%%%%%
%%%%%%%%%%%%%%%%%%%%%%%%%%%%%%%%%%%%%%%%%%%%%%%%%%%%%%%%%%%%%%%%%%%%%%%%%%%%%%%
% DELETE THIS PART. DO NOT PLACE CONTENT AFTER THE REFERENCES!
%%%%%%%%%%%%%%%%%%%%%%%%%%%%%%%%%%%%%%%%%%%%%%%%%%%%%%%%%%%%%%%%%%%%%%%%%%%%%%%
%%%%%%%%%%%%%%%%%%%%%%%%%%%%%%%%%%%%%%%%%%%%%%%%%%%%%%%%%%%%%%%%%%%%%%%%%%%%%%%
\clearpage
\appendix
\section{COMBO Proofs from Section~\ref{sec:combo_theory}}
\label{app:combo_proofs}

In this section, we provide proofs for theoretical results in Section~\ref{sec:combo_theory}. Before the proofs, we note that all statements are proven in the case of finite state space (i.e., $|\states| < \infty$) and finite action space (i.e., $|\actions| < \infty$) we define some commonly appearing notation symbols appearing in the proof: 
\begin{itemize}
\vspace{-5pt}
    \item $P_{\mdp}$ and $r_{\mdp}$ (or $P$ and $r$ with no subscript for notational simplicity) denote the dynamics and reward function of the actual MDP $\mdp$
    \vspace{-5pt}
    \item $P_{\mdpbar}$ and $r_{\mdpbar}$ denote the dynamics and reward of the empirical MDP $\mdpbar$ generated from the transitions in the dataset
    \vspace{-5pt}
    \item $P_{\mdphat}$ and $r_{\mdphat}$ denote the dynamics and reward of the MDP induced by the learned model $\mdphat$
\end{itemize}
\vspace{-5pt}
We also assume that whenever the cardinality of a particular state or state-action pair in the offline dataset $\data$, denoted by $|\mathcal{D}(\bs, \mathbf{a})|$, appears in the denominator, we assume it is non-zero. For any non-existent $(\bs, \mathbf{a}) \notin \data$, we can simply set $|\data(\bs, \mathbf{a})|$ to be a small value $< 1$, which prevents any bound from producing trivially $\infty$ values.

\subsection{A Useful Lemma and Its Proof}
\label{app:proof_lemma}

Before proving our main results, we first show that the penalty
term in equation \ref{eqn:combo_iterate} is positive in expectation. Such a positive penalty is important to combat any overestimation that may
arise as a result of using $\bellmanhat$.

\begin{lemma}[(Interpolation Lemma]
\label{thm:line_thm}
For any $f \in [0, 1]$, and any given $\rho(\bs, \mathbf{a}) \in \Delta^{|\states||\actions|}$, let $d_f$ be an f-interpolation of $\rho$ and $\data$, i.e., $d_f(\bs, \mathbf{a}) := f d(\bs, \mathbf{a}) + (1-f) \rho(\bs, \mathbf{a})$. For a given iteration $k$ of Equation~\ref{eqn:combo_iterate}, we restate the definition of the expected penalty under $\rho(\bs, \mathbf{a})$ in Eq.~\ref{eqn:expected_penalty}: 
\begin{equation*}
 \nu(\rho, f) := \E_{\bs, \mathbf{a} \sim \rho(\bs, \mathbf{a})}\left[\frac{\rho(\bs, \mathbf{a}) - d(\bs, \mathbf{a})}{d_f(\bs, \mathbf{a})} \right].
\end{equation*}
Then $\nu(\rho, f)$ satisfies, (1) $\nu(\rho, f) \geq 0,~~ \forall \rho, f$, (2) $\nu(\rho, f)$ is monotonically increasing in $f$ for a fixed $\rho$, and (3) $\nu(\rho, f) = 0$ iff $\forall~ \bs, \mathbf{a}, ~\rho(\bs, \mathbf{a}) = d(\bs, \mathbf{a}) \text{~or~} f = 0$. 
\end{lemma}
\begin{proof}
To prove this lemma, we use algebraic manipulation on the expression for quantity $\nu(\rho, f)$ and show that it is indeed positive and monotonically increasing in $f \in [0, 1]$.
\begin{align}
    \nu(\rho, f) &= \sum_{\bs, \mathbf{a}} \rho(\bs, \mathbf{a}) \left(\frac{\rho(\bs, \mathbf{a}) - d(\bs, \mathbf{a})}{f d(\bs, \mathbf{a}) + (1 - f) \rho(\bs, \mathbf{a})}\right)\nonumber \\
    &= \sum_{\bs, \mathbf{a}} \rho(\bs, \mathbf{a}) \left(\frac{\rho(\bs, \mathbf{a}) - d(\bs, \mathbf{a})}{\rho(\bs, \mathbf{a}) + f ( d(\bs, \mathbf{a}) - \rho(\bs, \mathbf{a}))}\right)\\
    \implies \frac{d \nu(\rho, f)}{d f} &= \sum_{\bs, \mathbf{a}} \rho(\bs, \mathbf{a}) \left(\rho(\bs, \mathbf{a}) - d(\bs, \mathbf{a})\right)^2 \cdot \left(\frac{1}{(\rho(\bs, \mathbf{a}) + f ( d(\bs, \mathbf{a}) - \rho(\bs, \mathbf{a}))}\right)^2 \geq 0\nonumber\\
    &~~~\forall f \in [0, 1].
\end{align}
Since the derivative of $\nu(\rho, f)$ with respect to $f$ is always positive, it is an increasing function of $f$ for a fixed $\rho$, and this proves the second part (2) of the Lemma. Using this property, we can show the part (1) of the Lemma as follows:
\begin{align}
    \forall f \in (0, 1],~ \nu(\rho, f) \geq \nu(\rho, 0) = \sum_{\bs, \mathbf{a}} \rho(\bs, \mathbf{a}) \frac{\rho(\bs, \mathbf{a}) - d(\bs, \mathbf{a})}{\rho(\bs, \mathbf{a})} &= \sum_{\bs, \mathbf{a}} \left( \rho(\bs, \mathbf{a}) - d(\bs, \mathbf{a}) \right)\nonumber\\
    &= 1 - 1 = 0.
\end{align}
Finally, to prove the third part (3) of this Lemma, note that when $f = 0$, $\nu(\rho, f) = 0$ (as shown above), and similarly by setting $\rho(\bs, \mathbf{a}) = d(\bs, \mathbf{a})$ note that we obtain $\nu(\rho, f) = 0$. To prove the only if side of (3), assume that $f \neq 0$ and $\rho(\bs, \mathbf{a}) \neq d(\bs, \mathbf{a})$ and we will show that in this case $\nu(\rho,f) \neq 0$. When $d(\bs, \mathbf{a}) \neq \rho(\bs, \mathbf{a})$, the derivative $\frac{d \nu(\rho,f)}{d f} > 0$ (i.e., strictly positive) and hence the function $\nu(\rho, f)$ is a strictly increasing function of $f$. Thus, in this case, $\nu(\rho, f) > 0 = \nu(\rho, 0)~ \forall f > 0$. Thus we have shown that if $\rho(\bs, \mathbf{a}) \neq d(\bs, \mathbf{a})$ and $f > 0$, $\nu(\rho, f) \neq 0$, which completes our proof for the only if side of (3). 
\end{proof}

\subsection{Proof of Proposition~\ref{thm:lower_bound}}
\label{app:proof_lower_bound}
Before proving this proposition, we provide a bound on the Bellman backup in the empirical MDP, $\bellman_{\mdpbar}$. To do so, we formally define the standard concentration properties of the reward and transition dynamics in the empirical MDP, $\mdpbar$, that we assume so as to prove Proposition~\ref{thm:line_thm}. Following prior work~\citep{osband2017posterior,jaksch2010near,kumar2020conservative}, we assume:
\begin{assumption}
\label{assumption:conc}
    $\forall~ \bs, \mathbf{a} \in \mdp$, the following relationships hold with high probability, $\geq 1 - \delta$
    \begin{equation*}
        |r_{\mdpbar}(\bs, \mathbf{a}) - r(\bs, \mathbf{a})| \leq \frac{C_{r, \delta}}{\sqrt{|\mathcal{D}(\bs, \mathbf{a})|}}, ~~~ ||P_{\mdpbar}(\bs'|\bs, \mathbf{a}) - P(\bs'|\bs, \mathbf{a})||_{1} \leq \frac{C_{P, \delta}}{\sqrt{|\mathcal{D}(\bs, \mathbf{a})|}}.
    \end{equation*}
\end{assumption}
Under this assumption and assuming that the reward function in the MDP, $r(\bs, \mathbf{a})$ is bounded, as $|r(\bs, \mathbf{a})| \leq R_{\max}$, we can bound the difference between the empirical Bellman operator, $\bellman_{\mdpbar}$ and the actual MDP, $\bellman_\mdp$,
\begin{align*}
    \left\vert\left({\bellman_{\mdpbar}}^\policy \hat{Q}^k \right) - \left({\bellman}^\policy_\mdp \hat{Q}^k \right)\right\vert &= \left\vert\left(r_{\mdpbar}(\bs, \mathbf{a}) - r_\mdp(\bs, \mathbf{a})\right)\right.\\
    &\left.+ \gamma \sum_{\bs'} \left({P}_{\mdpbar}(\bs'|\bs, \mathbf{a}) - P_\mdp(\bs'|\bs,\mathbf{a})\right) \E_{\policy(\mathbf{a}'|\bs')}\left[\hat{Q}^k(\bs' , \mathbf{a}')\right]\right\vert\\
    &\leq \left\vert r_{\mdpbar}(\bs, \mathbf{a}) - r_\mdp(\bs, \mathbf{a})\right\vert\\
    &+ \gamma \left\vert \sum_{\bs'} \left({P}_{\mdpbar}(\bs'|\bs, \mathbf{a}) - P_\mdp(\bs'|\bs,\mathbf{a})\right) \E_{\policy(\mathbf{a}'|\bs')}\left[\hat{Q}^k(\bs' , \mathbf{a}')\right]\right\vert\\
    &\leq \frac{C_{r, \delta} + \gamma C_{P, \delta} 2R_{\max} / (1 - \gamma)}{\sqrt{|\mathcal{D}(\bs, \mathbf{a})|}}. 
\end{align*}
Thus the overestimation due to sampling error in the empirical MDP, $\mdpbar$ is bounded as a function of a bigger constant, $C_{r, P, \delta}$ that can be expressed as a function of $C_{r, \delta}$ and $C_{P, \delta}$, and depends on $\delta$ via a $\sqrt{\log (1/\delta)}$ dependency. For the purposes of proving Proposition~\ref{thm:Q_bound}, we assume that:
\begin{equation}
\label{eqn:sampling_error}
    \forall \bs, \mathbf{a}, ~~\left\vert\left({\bellman_{\mdpbar}}^\policy \hat{Q}^k \right) - \left({\bellman}^\policy_\mdp \hat{Q}^k \right)\right\vert  \leq \frac{C_{r, T, \delta} R_{\max}}{(1 - \gamma) \sqrt{|\mathcal{D}(\bs, \mathbf{a})|}}.
\end{equation}

Next, we provide a bound on the error between the bellman backup induced by the learned dynamics model and the learned reward, $\bellman_{\mdphat}$, and the actual Bellman backup, $\bellman_{\mdp}$. To do so, we note that:
\begin{align}
    \left\vert\left({\bellman_{\mdphat}}^\policy \hat{Q}^k \right) - \left({\bellman}^\policy_\mdp \hat{Q}^k \right)\right\vert &= \left\vert\left(r_{\mdphat}(\bs, \mathbf{a}) - r_\mdp(\bs, \mathbf{a})\right)\right.\\
    &\left.+ \gamma \sum_{\bs'} \left({P}_{\mdphat}(\bs'|\bs, \mathbf{a}) - P_\mdp(\bs'|\bs,\mathbf{a})\right) \E_{\policy(\mathbf{a}'|\bs')}\left[\hat{Q}^k(\bs' , \mathbf{a}')\right]\right\vert \nonumber\\ 
    &\leq |r_{\mdphat}(\bs, \mathbf{a}) - r_\mdp(\bs, \mathbf{a})| + \gamma \frac{2 R_{\max}}{1 - \gamma} D(P, P_{\mdphat}),
    \label{eqn:model_error} 
\end{align}
where $D(P, P_{\mdphat})$ is the total-variation divergence between the learned dynamics model and the actual MDP. Now, we show that the asymptotic Q-function learned by COMBO lower-bounds the actual Q-function of any
policy $\pi$ with high probability for a large enough $\beta \geq 0$. We will use Equations~\ref{eqn:sampling_error} and \ref{eqn:model_error} to prove such a result.

\begin{proposition}[Asymptotic lower-bound]
\label{thm:Q_bound}
Let $P^\pi$ denote the Hadamard product of the dynamics $P$ and a given policy $\pi$ in the actual MDP and let $S^\pi := (I - \gamma P^\pi)^{-1}$. Let $D$ denote the total-variation divergence between two probability distributions. For any $\pi(\mathbf{a}|\bs)$, the Q-function obtained by recursively applying Equation~\ref{eqn:combo_iterate}, with $\hat{{\bellman}}^\pi = f \bellman_{\mdpbar}^\pi + (1 - f) \bellman_{\mdphat}^\pi$, with probability at least $1 - \delta$, results in $\hat{Q}^\pi$ that satisfies:
\begin{align*}
    \forall \bs, \mathbf{a},~ \hat{Q}^\pi(\bs, \mathbf{a}) \leq  Q^\pi(\bs, \mathbf{a}) &- \beta \cdot \left[ S^\pi \left[ \frac{\rho - d}{d_f} \right] \right](\bs, \mathbf{a}) + f \left[ S^\pi \left[ \frac{C_{r, T, \delta} R_{\max}}{(1 - \gamma) \sqrt{|\data|}} \right] \right](\bs, \mathbf{a})\\
    +&~ (1 - f) \left[ S^\pi \left[ |r - r_{\mdphat}| + \frac{ 2 \gamma  R_{\max}}{1 - \gamma} D(P, P_{\mdphat}) \right]  \right]\!\! (\bs, \mathbf{a}).
\end{align*}
\end{proposition}
\begin{proof}
We first note that the Bellman backup $\hat{\bellman}^\pi$ induces the following Q-function iterates as per Equation~\ref{eqn:combo_iterate},
\begin{align*}
    \hat{Q}^{k+1}(\bs, \mathbf{a}) &= \left(\hat{\bellman}^\pi \hat{Q}^k\right)(\bs, \mathbf{a}) - \beta \frac{\rho(\bs, \mathbf{a}) - d(\bs, \mathbf{a})}{d_f(\bs, \mathbf{a})}\\
    &=  f \left(\bellman^\pi_{\mdpbar} \hat{Q}^k \right) (\bs, \mathbf{a}) + (1 - f) \left(\bellman^\pi_{\mdphat} \hat{Q}^k \right) (\bs, \mathbf{a}) - \beta \frac{\rho(\bs, \mathbf{a}) - d(\bs, \mathbf{a})}{d_f(\bs, \mathbf{a})}\\
    &= \left(\bellman^\pi \hat{Q}^k\right)(\bs, \mathbf{a}) - \beta \frac{\rho(\bs, \mathbf{a}) - d(\bs, \mathbf{a})}{d_f(\bs, \mathbf{a})} + (1 - f) \left({\bellman_{\mdphat}}^\policy \hat{Q}^k - {\bellman}^\policy \hat{Q}^k \right)(\bs, \mathbf{a})\\
    &+ f  \left({\bellman_{\mdpbar}}^\policy \hat{Q}^k - {\bellman}^\policy \hat{Q}^k \right)(\bs, \mathbf{a})\\
   \forall \bs, \mathbf{a},~ \hat{Q}^{k+1} &\leq \left(\bellman^\pi \hat{Q}^k\right) - \beta \frac{\rho - d}{d_f} + (1 - f) \left[|r_{\mdphat} - r_\mdp| + \frac{2 \gamma R_{\max}}{1 - \gamma} D(P, P_{\mdphat}) \right] + f \frac{C_{r, T, \delta} R_{\max}}{(1 - \gamma) \sqrt{|\data|}} 
\end{align*}
Since the RHS upper bounds the Q-function pointwise for each $(\bs, \mathbf{a})$, the fixed point of the Bellman iteration process will be pointwise smaller than the fixed point of the Q-function found by solving for the RHS via equality. Thus, we get that
\begin{align*}
    \hat{Q}^\pi(\bs, \mathbf{a}) &\leq \underbrace{ S^\pi r_{\mdp}}_{= Q^\pi(\bs, \mathbf{a})} -\beta \left[ S^\pi \left[ \frac{\rho - d}{d_f} \right] \right](\bs, \mathbf{a}) +~ f \left[ S^\pi \left[ \frac{C_{r, T, \delta} R_{\max}}{(1 - \gamma) \sqrt{|\data|}} \right] \right](\bs, \mathbf{a})\\
    &+~ (1 - f) \left[ S^\pi \left[ |r - r_{\mdphat}| + \frac{ 2 \gamma  R_{\max}}{1 - \gamma} D(P, P_{\mdphat}) \right]  \right]\!\! (\bs, \mathbf{a}),  
\end{align*}
which completes the proof of this proposition.
\end{proof}

Next, we use the result and proof technique from Proposition~\ref{thm:Q_bound} to prove Corollary~\ref{thm:lower_bound}, that in expectation under the initial state-distribution, the expected Q-value is indeed a lower-bound. 

\begin{corollary}[Corollary~\ref{thm:lower_bound} restated]
For a sufficiently large $\beta$, we have a lower-bound that
$\E_{\bs \sim \mu_0, \mathbf{a} \sim \policy(\cdot|\bs)}[\hat{Q}^\pi(\bs, \mathbf{a})] \leq \E_{\bs \sim \mu_0, \mathbf{a} \sim \policy(\cdot|\bs)}[Q^\pi(\bs, \mathbf{a})]$, 
where $\mu_0(\bs)$ is the initial state distribution. 
Furthermore, when $\epsilon_{\text{s}}$ is small, such as in the large sample regime; or when the model bias $\epsilon_{\text{m}}$ is small, a small $\beta$ is sufficient along with an appropriate choice of $f$.
\end{corollary}

\begin{proof}
To prove this corollary, we note a slightly different variant of Proposition~\ref{thm:Q_bound}. To observe this, we will deviate from the proof of Proposition~\ref{thm:Q_bound} slightly and will aim to express the inequality using $\bellman_{\mdphat}$, the Bellman operator defined by the learned model and the reward function. Denoting $(I - \gamma P_{\mdphat})^{-1}$ as $S_{\mdphat}^\pi$, doing this will intuitively allow us to obtain $\beta \left(\mu(\bs) \policy(\mathbf{a}|\bs)\right)^T \left(S_{\mdphat}^\pi \left[\frac{\rho - d}{d_f} \right]\right)(\bs, \mathbf{a})$ as the conservative penalty which can be controlled by choosing $\beta$ appropriately so as to nullify the potential overestimation caused due to other terms. Formally,
\begin{align*}
    \hat{Q}^{k+1}(\bs, \mathbf{a}) &= \left(\hat{\bellman}^\pi \hat{Q}^k\right)(\bs, \mathbf{a}) - \beta \frac{\rho(\bs, \mathbf{a}) - d(\bs, \mathbf{a})}{d_f(\bs, \mathbf{a})} = \left(\bellman^\pi_{\mdphat} \hat{Q}^k \right)(\bs, \mathbf{a}) -  \beta \frac{\rho(\bs, \mathbf{a}) - d(\bs, \mathbf{a})}{d_f(\bs, \mathbf{a})}\\
    &+ f \underbrace{\left(\bellman^\pi_{\mdpbar} - \bellman^\pi_{\mdphat} \hat{Q}^k \right)(\bs, \mathbf{a})}_{:= \Delta(\bs, \mathbf{a})}
\end{align*}
By controlling $\Delta(\bs, \mathbf{a})$ using the pointwise triangle inequality:
\begin{equation}
    \forall \bs, \mathbf{a}, ~\left\vert \bellman^\pi_{\mdpbar} \hat{Q}^k - \bellman^\pi_{\mdphat} \hat{Q}^k \right\vert \leq \left\vert \bellman^\pi \hat{Q}^k - \bellman^\pi_{\mdphat} \hat{Q}^k \right\vert + \left\vert \bellman^\pi_{\mdpbar} \hat{Q}^k - \bellman^\pi \hat{Q}^k \right\vert,
\end{equation}
and then iterating the backup $\bellman^\pi_{\mdphat}$ to its fixed point and finally noting that $\rho(\bs, \mathbf{a}) = \left((\mu \cdot \pi)^T S^\pi_{\mdphat}\right)(\bs, \mathbf{a})$, we obtain:
\begin{equation}
    \E_{\mu, \pi}[\hat{Q}^\pi(\bs, \mathbf{a})] \leq \E_{\mu, \pi}[Q^\pi_{\mdphat}(\bs, \mathbf{a})] - \beta~ \E_{\rho(\bs, \mathbf{a})}\left[\frac{\rho(\bs, \mathbf{a}) - d(\bs, \mathbf{a})}{d_f(\bs, \mathbf{a})}\right] + \mathrm{terms~ independent~ of~} \beta.
\end{equation}
%%AK: there is one more term in the equation above, fit it in one line...
The terms marked as ``terms independent of $\beta$'' correspond to the additional positive error terms obtained by iterating $\left\vert \bellman^\pi \hat{Q}^k - \bellman^\pi_{\mdphat} \hat{Q}^k \right\vert$ and $\left\vert \bellman^\pi_{\mdpbar} \hat{Q}^k - \bellman^\pi \hat{Q}^k \right\vert$, which can be bounded similar to the proof of Proposition~\ref{thm:Q_bound} above. Now by replacing the model Q-function, $\E_{\mu, \pi}[Q^\pi_{\mdphat}(\bs, \mathbf{a})]$ with the actual Q-function, $\E_{\mu, \pi}[Q^\pi(\bs, \mathbf{a})]$ and adding an error term corresponding to model error to the bound, we obtain that:
\begin{equation}
\label{eqn:lower_bound_eqn}
    \E_{\mu, \pi}[\hat{Q}^\pi(\bs, \mathbf{a})] \leq \E_{\mu, \pi}[Q^\pi(\bs, \mathbf{a})] + \mathrm{terms~ independent~ of~} \beta - \beta~ \underbrace{\E_{\rho(\bs, \mathbf{a})}\left[\frac{\rho(\bs, \mathbf{a}) - d(\bs, \mathbf{a})}{d_f(\bs, \mathbf{a})}\right]}_{= \nu(\rho, f) > 0}.
\end{equation}
Hence, by choosing $\beta$ large enough, we obtain the desired lower bound guarantee. 
\end{proof}

\begin{remark}[\underline{\textbf{COMBO does not underestimate at every $\bs \in \mathcal{D}$ unlike CQL.}}]
\label{remak:tighter_lower_bound}
Before concluding this section, we discuss how the bound obtained by COMBO (Equation~\ref{eqn:lower_bound_eqn}) is tighter than CQL. CQL learns a Q-function such that the value of the policy under the resulting Q-function lower-bounds the true value function at each state $\bs \in \mathcal{D}$ individually (in the absence of no sampling error), i.e., $\forall \bs \in \mathcal{D}, \hat{V}^\pi_{\text{CQL}}(\bs) \leq V^\pi(\bs)$, whereas the bound in COMBO is only valid in expectation of the value function over the initial state distribution, i.e., $\E_{\bs \sim \mu_0(\bs)}[\hat{V}^\pi_{\text{COMBO}}(\bs)] \leq \E_{\bs \sim \mu_0(\bs)}[V^\pi(\bs)]$, and the value function at a given state may not be a lower-bound. For instance, COMBO can overestimate the value of a state more frequent in the dataset distribution $d(\bs, \mathbf{a})$ but not so frequent in the $\rho(\bs, \mathbf{a})$ marginal distribution of the policy under the learned model $\mdphat$. To see this more formally, note that the expected penalty added in the effective Bellman backup performed by COMBO (Equation~\ref{eqn:combo_iterate}), in expectation under the dataset distribution $d(\bs, \mathbf{a})$, $\widetilde{\nu}(\rho, d, f)$ is actually \textbf{\textit{negative}}:
\begin{align*}
    \widetilde{\nu}(\rho, d, f) = \sum_{\bs, \mathbf{a}} d(\bs, \mathbf{a}) \frac{\rho(\bs, \mathbf{a}) - d(\bs, \mathbf{a})}{d_f(\bs, \mathbf{a})} = - \sum_{\bs, \mathbf{a}} d(\bs, \mathbf{a}) \frac{d(\bs, \mathbf{a}) - \rho(\bs, \mathbf{a})}{f d(\bs, \mathbf{a}) + (1 - f) \rho(\bs, \mathbf{a})} < 0,
\end{align*}
where the final inequality follows via a direct application of the proof of Lemma~\ref{thm:line_thm}. Thus, COMBO actually \emph{overestimates} the values at atleast some states (in the dataset) unlike CQL.   
\end{remark}

\subsection{Proof of Proposition~\ref{prop:less_conservative}}
\label{app:proof_less_conservative}

In this section, we will provide a proof for Proposition~\ref{prop:less_conservative}, and show that the COMBO can be less conservative in terms of the estimated value. To recall, let $\Delta^\pi_\text{COMBO} := \E_{\bs, \mathbf{a} \sim d_{\mdpbar}(\bs), \pi(\mathbf{a}|\bs)}\left[\hat{Q}^\pi(\bs, \mathbf{a} \right]$ and let $\Delta^\pi_\text{CQL} := \E_{\bs, \mathbf{a} \sim d_{\mdpbar}, \pi(\mathbf{a}|\bs)} \left[\hat{Q}^\pi_\text{CQL}(\bs, \mathbf{a}) \right]$. From \citet{kumar2020conservative}, we obtain that $\hat{Q}^\pi_\text{CQL}(\bs, \mathbf{a}) := Q^\pi(\bs, \mathbf{a}) - \beta \frac{\pi(\mathbf{a}|\bs) - \pi_\beta(\mathbf{a}|\bs)}{\pi_\beta(\mathbf{a}|\bs)}$. We shall derive the condition for the real data fraction $f=1$ for COMBO, thus making sure that $d_f(\bs) = d^{\pi_\beta}(\bs)$. To derive the condition when $\Delta^\pi_\text{COMBO} \geq \Delta^\pi_\text{CQL}$, we note the following simplifications:
\begin{align}
    & \Delta^\pi_\text{COMBO} \geq \Delta^\pi_\text{CQL} \\
    \implies & \sum_{\bs, \mathbf{a}} d_{\mdpbar}(\bs) \pi(\mathbf{a}|\bs) \hat{Q}^\pi(\bs, \mathbf{a}) \geq \sum_{\bs, \mathbf{a}} d_{\mdpbar}(\bs) \pi(\mathbf{a}|\bs) \hat{Q}^\pi_\text{CQL}(\bs, \mathbf{a}) \\
    \label{eqn:cql_vs_combo_terms}
    \implies & \beta \sum_{\bs, \mathbf{a}} d_{\mdpbar}(\bs)\pi(\mathbf{a}|\bs) \left( \frac{\rho(\bs, \mathbf{a}) - d^{\pi_\beta}(\bs) \pi_\beta(\mathbf{a}|\bs)}{d^{\pi_\beta}(\bs) \pi_\beta(\mathbf{a}|\bs)} \right) \leq \beta \sum_{\bs, \mathbf{a}} d_{\mdpbar}(\bs)\pi(\mathbf{a}|\bs) \left(\frac{\pi(\mathbf{a}|\bs) - \pi_\beta(\mathbf{a}|\bs)}{\pi_\beta(\mathbf{a}|\bs)} \right).
\end{align}
Now, in the expression on the left-hand side, we add and subtract $d^{\pi_\beta}(\bs) \pi(\mathbf{a}|\bs)$ from the numerator inside the paranthesis.
\begin{align}
    & \sum_{\bs, \mathbf{a}} d_{\mdpbar}(\bs, \mathbf{a}) \left( \frac{\rho(\bs, \mathbf{a}) - d^{\pi_\beta}(\bs) \pi_\beta(\mathbf{a}|\bs)}{d^{\pi_\beta}(\bs) \pi_\beta(\mathbf{a}|\bs)} \right)\\
    &= \sum_{\bs, \mathbf{a}} d_{\mdpbar}(\bs, \mathbf{a}) \left( \frac{\rho(\bs, \mathbf{a}) - d^{\pi_\beta}(\bs) \pi(\mathbf{a}|\bs) + d^{\pi_\beta}(\bs) \pi(\mathbf{a}|\bs) - d^{\pi_\beta}(\bs) \pi_\beta(\mathbf{a}|\bs)}{d^{\pi_\beta}(\bs) \pi_\beta(\mathbf{a}|\bs)} \right)\\
    &= \underbrace{\sum_{\bs, \mathbf{a}} d_{\mdpbar}(\bs, \mathbf{a}) \frac{\pi(\mathbf{a}|\bs) - \pi_\beta(\mathbf{a}|\bs)}{\pi_\beta(\mathbf{a}|\bs)}}_{(1)} + \sum_{\bs, \mathbf{a}} d_{\mdpbar}(\bs, \mathbf{a}) \cdot \frac{\rho(\bs) - d^{\pi_\beta}(\bs)}{d^{\pi_\beta}(\bs)} \cdot \frac{\pi(\mathbf{a}|\bs)}{\pi_\beta(\mathbf{a}|\bs)}
\end{align}
The term marked $(1)$ is identical to the CQL term that appears on the right in Equation~\ref{eqn:cql_vs_combo_terms}. Thus the inequality in Equation~\ref{eqn:cql_vs_combo_terms} is satisfied when the second term above is negative. To show this, first note that $d^{\pi_\beta}(\bs) = d_{\mdpbar}(\bs)$ which results in a cancellation. Finally, re-arranging the second term into expectations gives us the desired result. An analogous condition can be derived when $f \neq 1$, but we omit that derivation as it will be hard to interpret terms appear in the final inequality.

\subsection{Proof of Proposition~\ref{thm:policy_improvement}}
\label{app:proof_policy_improvement}

To prove the policy improvement result in Proposition~\ref{thm:policy_improvement}, we first observe that using Equation~\ref{eqn:combo_iterate} for Bellman backups amounts to finding a policy that maximizes the return of the policy in the a modified ``f-interpolant'' MDP which admits the Bellman backup $\bellmanhat^\pi$, and is induced by a linear interpolation of backups in the empirical MDP $\mdpbar$ and the MDP induced by a dynamics model $\mdphat$ and the return of a policy $\pi$ in this effective f-interpolant MDP is denoted by $J(\mdpbar, \mdphat, f, \pi)$. Alongside this, the return is penalized by the conservative penalty where $\rho^\pi$ denotes the marginal state-action distribution of policy $\pi$ in the learned model $\mdphat$. 
\begin{equation}
    \hat{J}(f, \pi) = J(\mdpbar, \mdphat, f, \pi)  - \beta \frac{\nu(\rho^\pi, f)}{1 - \gamma}.
\label{eqn:penalized_objective}
\end{equation}
We will require bounds on the return of a policy $\pi$ in this f-interpolant MDP, $J(\mdpbar, \mdphat, f, \pi)$, which we first prove separately as Lemma~\ref{lemma:interpolant_regular_bound} below and then move to the proof of Proposition~\ref{thm:policy_improvement}.

\begin{lemma}[Bound on return in f-interpolant MDP]
\label{lemma:interpolant_regular_bound}
For any two MDPs, $\mdp_1$ and $\mdp_2$, with the same state-space, action-space and discount factor, and for a given fraction $f \in [0, 1]$, define the f-interpolant MDP $\mdp_f$ as the MDP on the same state-space, action-space and with the same discount as the MDP with dynamics: $P_{\mdp_f} := f P_{\mdp_1} + (1 - f) P_{\mdp_2}$ and reward function: $r_{\mdp_f} := f r_{\mdp_1} + (1 - f) r_{\mdp_2}$. Then, given any auxiliary MDP, $\mdp$, the return of any policy $\pi$ in $\mdp_f$, $J(\pi, \mdp_f)$, also denoted by $J(\mdp_1, \mdp_2, f, \pi)$, lies in the interval:
\begin{equation*}
    \big[ J(\pi, \mdp) - \alpha,~~ J(\pi, \mdp)+ \alpha \big], \text{~~~~~~~~~~~~where~} \alpha \text{~is given by:~}
\end{equation*}
\begin{align}
    \alpha &= \frac{2 \gamma (1 - f)}{(1 - \gamma)^2} R_{\max} D \left(P_{\mdp_2}, P_{\mdp}\right) + \frac{\gamma f}{1 - \gamma} \left\vert \E_{d^\pi_{\mdp} \pi} \left[ \left(P^\pi_{\mdp} - P^\pi_{\mdp_1}\right) Q^\pi_{\mdp} \right]\right\vert  \nonumber\\
   & + \frac{f}{1 - \gamma} \E_{\bs, \mathbf{a} \sim d^\pi_{\mdp} \pi}[|r_{\mdp_1}(\bs, \mathbf{a}) - r_{\mdp}(\bs, \mathbf{a})|] + \frac{1 - f}{1 - \gamma} \E_{\bs, \mathbf{a} \sim d^\pi_{\mdp} \pi}[|r_{\mdp_2}(\bs, \mathbf{a}) - r_{\mdp}(\bs, \mathbf{a})|].  \label{eqn:alpha_expr}
\end{align}
\end{lemma}
\begin{proof}
To prove this lemma, we note two general inequalities. First, note that for a fixed transition dynamics, say $P$, the return decomposes linearly in the components of the reward as the expected return is linear in the reward function:
\begin{equation*}
    J(P, r_{\mdp_f}) = J(P, f r_{\mdp_1} + (1 - f) r_{\mdp_2}) = f J (P, r_{\mdp_1}) + (1 - f) J(P, r_{\mdp_2}).  
\end{equation*}
As a result, we can bound $J(P, r_{\mdp_f})$ using $J(P, r)$ for a new reward function $r$ of the auxiliary MDP, $\mdp$, as follows
\begin{align*}
     J(P, r_{\mdp_f}) &= J(P, f r_{\mdp_1} + (1 - f) r_{\mdp_2}) = J (P, r + f (r_{\mdp_1} - r) + (1 -f) (r_{\mdp_2} - r)\\
     &= J(P, r) + f J(P, r_{\mdp_1} - r) + (1 - f) J(P, r_{\mdp_2} - r)\\
     &= J(P, r) + \frac{f}{1 - \gamma} \E_{\bs, \mathbf{a} \sim d^\pi_{\mdp}(\bs) \pi(\mathbf{a}|\bs)}\left[ r_{\mdp_1}(\bs, \mathbf{a}) - r(\bs, \mathbf{a}) \right]\\
     &+ \frac{1 - f}{1 - \gamma} \E_{\bs, \mathbf{a} \sim d^\pi_{\mdp}(\bs) \pi(\mathbf{a}|\bs)} \left[ r_{\mdp_2}(\bs, \mathbf{a}) - r(\bs, \mathbf{a}) \right].
\end{align*}
Second, note that for a given reward function, $r$, but a linear combination of dynamics, the following bound holds:
\begin{align*}
    J(P_{\mdp_f}, r) &= J(f P_{\mdp_1} + (1 - f) P_{\mdp_2}, r)\\
    &= J ( P_{\mdp} +  f( P_{\mdp_1} - P_{\mdp}) + (1 - f) (P_{\mdp_2} - P_{\mdp}), r)\\ 
    &= J (P_{\mdp}, r) - \frac{\gamma (1 - f)}{1 - \gamma} \E_{\bs, \mathbf{a} \sim d^\pi_{\mdp}(\bs) \pi(\mathbf{a}|\bs)} \left[ \left(P^\pi_{\mdp_2} - P^\pi_{\mdp}\right) Q^\pi_{\mdp}  \right]\\
    &- \frac{\gamma f}{1 - \gamma} \E_{\bs, \mathbf{a} \sim d^\pi_{\mdp}(\bs) \pi(\mathbf{a}|\bs)} \left[ \left(P^\pi_{\mdp} - P^\pi_{\mdp_1}\right) Q^\pi_{\mdp}  \right]\\
    &\in \left[ J( P_{\mdp}, r) ~\pm~ \left(\frac{\gamma f}{(1 - \gamma)} \left\vert \E_{\bs, \mathbf{a} \sim d^\pi_{\mdp}(\bs) \pi(\mathbf{a}|\bs)}\left[ \left(P^\pi_{\mdp} - P^\pi_{\mdp_1}\right) Q^\pi_{\mdp} \right] \right\vert\right.\right.\\
    &\left.\left.+ \frac{2 \gamma (1 -f) R_{\max}}{(1 - \gamma)^2} D(P_{\mdp_2}, P_{\mdp}) \right) \right].
    % &\in \left[J (P_{\mdp_1}, r) ~~\pm~~ \frac{\gamma (1 -f) R_{\max}}{(1 - \gamma)^2} D(P_{\mdp}, P_{\mdp_2}) ~\pm~ (1 - f) \frac{\gamma}{1 - \gamma}  \E_{\bs, \mathbf{a} \sim d^\pi_{\mdp_1}(\bs) \pi(\mathbf{a}|\bs)} \left[ \left(P^\pi_{\mdp} - P^\pi_{\mdp_1}\right) Q^\pi  \right] \right]. 
\end{align*}
To observe the third equality, we utilize the result on the difference between returns of a policy $\pi$ on two different MDPs, $P_{\mdp_1}$ and $P_{\mdp_f}$ from \citet{ajksbook} (Chapter 2, Lemma 2.2, Simulation Lemma), and additionally incorporate the auxiliary MDP $\mdp$ in the expression via addition and subtraction in the previous (second) step. In the fourth step, we finally bound one term that corresponds to the learned model via the total-variation divergence $D(P_{\mdp_2}, P_{\mdp})$ and the other term corresponding to the empirical MDP $\mdpbar$ is left in its expectation form to be bounded later. 

Using the above bounds on return for reward-mixtures and dynamics-mixtures, proving this lemma is straightforward:
\begin{align*}
    & J(\mdp_1, \mdp_2, f, \pi) := J(P_{\mdp_f}, f r_{\mdp_1} + (1 - f) r_{\mdp_2}) = J(f P_{\mdp_1} + (1 -f) P_{\mdp_2}, r_{\mdp_f})\\
    &\in \left[ J(P_{\mdp_f}, r_{\mdp}) ~\pm\right.\\
    &\left.~ \underbrace{\left(\frac{f}{1 - \gamma} \E_{\bs, \mathbf{a} \sim d^\pi_{\mdp} \pi}[|r_{\mdp_1}(\bs, \mathbf{a}) - r_{\mdp}(\bs, \mathbf{a})|] + \frac{1 - f}{1 - \gamma} \E_{\bs, \mathbf{a} \sim d^\pi_{\mdp} \pi}[|r_{\mdp_2}(\bs, \mathbf{a}) - r_{\mdp}(\bs, \mathbf{a})|] \right)}_{:= \Delta_R} \right],
    % ~\pm~ \left(\frac{2 \gamma f (1 - f)}{(1 - \gamma)^2} R_{\max} D \left(P_{\mdp_2}, P_{\mdp}\right) + \frac{2 \gamma f (1 - f)}{1 - \gamma} \E_{d^\pi_{\mdp_1}} \left\vert \left[ \left(P^\pi_{\mdp} - P^\pi_{\mdp_1}\right) Q^\pi  \right] \right\vert \right) \right],
\end{align*}
where the second step holds via linear decomposition of the return of $\pi$ in $\mdp_f$ with respect to the reward interpolation, and bounding the terms that appear in the reward difference. For convenience, we refer to these offset terms due to the reward as $\Delta_R$. For the final part of this proof, we bound $J(P_{\mdp_f}, r_{\mdp})$ in terms of the return on the actual MDP, $J(P_{\mdp}, r_{\mdp})$, using the inequality proved above that provides intervals for mixture dynamics but a fixed reward function. Thus, the overall bound is given by $J(\pi, \mdp_f) \in [J(\pi, \mdp) - \alpha, J(\pi, \mdp) + \alpha]$, where $\alpha$ is given by:
\begin{align}
\label{eqn:alpha_expr_repeat}
    \alpha = \frac{2 \gamma (1 - f)}{(1 - \gamma)^2} & R_{\max} D \left(P_{\mdp_2}, P_{\mdp}\right) + \frac{\gamma f}{1 - \gamma} \left\vert \E_{d^\pi_{\mdp} \pi} \left[ \left(P^\pi_{\mdp} - P^\pi_{\mdp_1}\right) Q^\pi_{\mdp} \right]\right\vert + \Delta_R.
\end{align}
This concludes the proof of this lemma.
\end{proof}



Finally, we prove Theorem~\ref{thm:policy_improvement} that shows how policy optimization with respect to $\hat{J}(f, \pi)$ affects the performance in the actual MD by using Equation~\ref{eqn:penalized_objective} and building on the  analysis of pure model-free algorithms from \citet{kumar2020conservative}. We restate a more complete statement of the theorem below and present the constants at the end of the proof. 

\begin{theorem}[Formal version of Proposition~\ref{thm:policy_improvement}]
Let $\hat{\pi}_{\text{out}}(\mathbf{a}|\bs)$ be the policy obtained by COMBO.
%%CF: Would be nice to have this definition outside of the theorem so that the theorem is shorter/simpler
Then, the policy ${\pi}_{\text{out}}(\mathbf{a}|\bs)$ is a $\zeta$-safe policy improvement over ${\behavior}$ in the actual MDP $\mdp$, i.e., $J({\pi}_{\text{out}}, \mdp) \geq J({\behavior}, \mdp) - \zeta$, with probability at least $1 - \delta$, where $\zeta$ is given by (where $\rho^\beta(\bs, \mathbf{a}) := d^\behavior_{\mdphat}(\bs, \mathbf{a})$):
\begin{align*}
&\mathcal{O}\left(\frac{\gamma f}{(1 - \gamma)^2}\right) {\left[ \E_{\bs \sim d^{\pi_{\text{out}}}_{\mdp}}\left[ \sqrt{\frac{|\actions|}{|\data(\bs)|} (\mathrm{D}_{\text{CQL}}({\pi}_{\text{out}}, \behavior) + 1)} \right] \right]}\\
&+ \mathcal{O}\left(\frac{\gamma (1 - f)}{(1 - \gamma)^2}\right) {\mathrm{D_{TV}}(P_{\mdp}, P_{\mdphat})} - \beta \frac{\nu(\rho^{\pi_{\text{out}}}, f) - \nu(\rho^\beta, f)}{(1 - \gamma)}.
    % &- \underbrace{\left({J}(\mdpbar, \mdphat, f, \pi) - {J}(\mdpbar, \mdphat, f, \behavior) \right)}_{:= (3),~~ \geq \beta \frac{\nu(\rho, f)}{(1 - \gamma)}} 
\end{align*}
\end{theorem}

\begin{proof}
We first note that since policy improvement is not being performed in the same MDP, $\mdp$ as the f-interpolant MDP, $\mdp_f$, we need to upper and lower bound the amount of improvement occurring in the actual MDP due to the f-interpolant MDP. As a result our first is to relate $J(\pi, \mdp)$ and $J(\pi, \mdp_f) := J(\mdpbar, \mdphat, f, \pi)$ for any given policy $\pi$.

\textbf{Step 1: Bounding the return in the actual MDP due to optimization in the f-interpolant MDP.} By directly applying Lemma~\ref{lemma:interpolant_regular_bound} stated and proved previously, we obtain the following upper and lower-bounds on the return of a policy $\pi$:
\begin{equation*}
    J(\mdpbar, \mdphat, f, \pi) \in \left[ J(\pi, \mdp) - \alpha,~~ J(\pi, \mdp) + \alpha \right],
\end{equation*}
where $\alpha$ is shown in Equation~\ref{eqn:alpha_expr}. As a result, we just need to bound the terms appearing the expression of $\alpha$ to obtain a bound on the return differences. We first note that the terms in the expression for $\alpha$ are of two types: \textbf{(1)} terms that depend only on the reward function differences (captured in $\Delta_R$ in Equation~\ref{eqn:alpha_expr_repeat}), and \textbf{(2)} terms that depend on the dynamics (the other two terms in Equation~\ref{eqn:alpha_expr_repeat}). 

To bound $\Delta_R$, we simply appeal to concentration inequalities on reward (Assumption~\ref{assumption:conc}), and bound $\Delta_R$ as:
\begin{align*}
\Delta_R &:= \frac{f}{1 - \gamma} \E_{\bs, \mathbf{a} \sim d^\pi_{\mdp} \pi}[|r_{\mdp_1}(\bs, \mathbf{a}) - r_{\mdp}(\bs, \mathbf{a})|] + \frac{1 - f}{1 - \gamma} \E_{\bs, \mathbf{a} \sim d^\pi_{\mdp} \pi}[|r_{\mdp_2}(\bs, \mathbf{a}) - r_{\mdp}(\bs, \mathbf{a})|]\\
&\leq \frac{C_{r, \delta}}{1 - \gamma} \E_{\bs, \mathbf{a} \sim d^\pi_{\mdp}\pi} \left[\frac{1}{\sqrt{D(\bs, \mathbf{a})}}\right] + \frac{1}{1 - \gamma} ||R_{\mdp} - R_{\mdphat}|| := \Delta_R^u.
\end{align*}
Note that both of these terms are of the order of $\mathcal{O}(1/ (1 - \gamma))$ and hence they don't figure in the informal bound in Theorem~\ref{thm:policy_improvement} in the main text, as these are dominated by terms that grow quadratically with the horizon.
% First, we use algebraic manipulation to obtain the following decompositionof the difference in the return of $\pi_{\text{out}}$ and $\pi_\beta$ in the actual MDP, $\mdp$:
% \begin{align*}
%     J(\pi_{\text{out}}, \mdp) - J(\behavior, \mdp) &= f \left(J(\pi_{\text{out}}, \mdp) - J(\pi_{\text{out}}, \mdpbar) \right) + (1 - f) \left(J(\pi_{\text{out}}, \mdp) - J(\pi_{\text{out}}, \mdphat) \right)~~~ \text{(a): policy difference}\\
%     &+ f (J(\pi_{\text{out}}, \mdpbar) - J(\behavior, \mdpbar)) + (1 - f) \left(J(\pi_{\text{out}}, \mdphat) - J(\behavior, \mdphat) \right)~~~\text{(b): policy improvement} \\
%     &+ f \left(J(\behavior, \mdpbar) - J(\behavior, \mdp) \right) + (1 - f) \left(J(\behavior, \mdphat) - J(\behavior, \mdp) \right)~~~~~~ \text{(c): behavior difference}
% \end{align*}
% Terms (a) and (c) correspond to a weighted sum of the difference in the return estimates of the policies in the empirical MDP $\mdpbar$ and the actual MDP $\mdp$ and the model-induced MDP $\mdphat$, and the actual MDP $\mdp$. 
To bound the remaining terms in the expression for $\alpha$, we utilize a result directly from \citet{kumar2020conservative} for the empirical MDP, $\mdpbar$, which holds for any policy $\pi(\mathbf{a}|\bs)$, as shown below.
\begin{align*}
   &\frac{\gamma}{(1 - \gamma)} \left\vert \E_{\bs, \mathbf{a} \sim d^\pi_{\mdp}(\bs) \pi(\mathbf{a}|\bs)}\left[ \left(P^\pi_{\mdp} - P^\pi_{\mdp_1}\right) Q^\pi_{\mdp} \right] \right\vert \\
   &\leq \frac{2 \gamma R_{\max} C_{P, \delta}}{(1 - \gamma)^2} \mathbb{E}_{\bs \sim d^{\policy}_{\mdpbar}(\bs)}\left[ \frac{\sqrt{|\mathcal{A}|}}{\sqrt{|\mathcal{D}(\bs)|}} \sqrt{ D_{\text{CQL}}(\policy, \behavior)(\bs) + 1} \right].
    %%AK: technically I think this is a naive result and certainly the MOReL paper was not the first one to come up with this... so unclear if we should be citing it for this result...
\end{align*}

\textbf{Step 2: Incorporate policy improvement in the f-inrerpolant MDP.} Now we incorporate the improvement of policy $\pi_{\text{out}}$ over the policy $\behavior$ on a weighted mixture of $\mdphat$ and $\mdpbar$. In what follows, we derive a lower-bound on this improvement by using the fact that policy $\pi_{\text{out}}$ is obtained by maximizing $\hat{J}(f, \pi)$ from Equation~\ref{eqn:penalized_objective}. As a direct consequence of Equation~\ref{eqn:penalized_objective}, we note that 
\begin{equation}
\label{eqn:improvement_expanded}
    \hat{J}(f, \pi_{\text{out}}) =  J(\mdpbar, \mdphat, f, \pi_{\text{out}}) - \beta \frac{\nu(\rho^\pi, f)}{1 - \gamma} \geq \hat{J}(f, \behavior) =  J(\mdpbar, \mdphat, f, \behavior) - \beta {\frac{\nu(\rho^\beta, f)}{1 - \gamma}}
\end{equation}
% Now, observe that we can both upper and lower-bound $J(\mdpbar, \mdphat, f, \pi)$ in terms of the return of policy $\pi$, individually in each MDP, $\mdpbar$ and $\mdphat$. We state this result more formally in Lemma~\ref{lemma:interpolant_regular_bound}.

% Next, we will use the upper bound on $J(\mdpbar, \mdphat, f, \pi)$ from Lemma~\ref{lemma:interpolant_regular_bound} for policy $\pi = \pi_{\text{out}}$ and a lower-bound on $J(\mdpbar, \mdphat, f, \pi)$ for policy $\pi = \behavior$, in the case when the auxiliary MDP is given by $\mdp$ (the actual MDP) to replace the expressions for $J(\mdpbar, \mdphat, f, \pi_{\text{out}})$ and $J(\mdpbar, \mdphat, f, \behavior)$ in the improvement equation~\ref{eqn:improvement_expanded}. Thus using Lemma~\ref{lemma:interpolant_regular_bound} we obtain the following inequality:
Following \textbf{Step 1}, we will use the upper bound on $J(\mdpbar, \mdphat, f, \pi)$ for policy $\pi = \pi_{\text{out}}$ and a lower-bound on $J(\mdpbar, \mdphat, f, \pi)$ for policy $\pi = \behavior$ and obtain the following inequality:
\begin{align*}
    J(\pi_{\text{out}}, \mdp) - \beta \frac{\nu(\rho^\pi, f)}{1 - \gamma} ~&\geq~ \Big\{ J(\behavior, \mdp) - \beta \frac{\nu(\rho^\beta, f)}{1 - \gamma}
    - \frac{4 \gamma (1 - f) R_{\max}}{(1 - \gamma)^2} D(P_{\mdp}, P_{\mdphat}) \\ 
    &- \underbrace{\frac{2 \gamma f}{(1 - \gamma)}\left\vert\E_{d^{\pi_{\text{out}}}_{\mdp}} \left[ \left(P^{\pi_{\text{out}}}_{\mdp} - P^{\pi_{\text{out}}}_{\mdpbar}\right) Q^{\pi_{\text{out}}}_{\mdp}  \right] \right\vert}_{:= (*)}\nonumber\\
    &- \underbrace{\frac{4 \gamma R_{\max} C_{P, \delta} f}{(1 - \gamma)^2} \E_{\bs \sim d^\behavior_{\mdp}}\left[ \sqrt{\frac{|\actions|}{|\data(\bs)|}}\right]}_{:= (\wedge)} - \Delta_R^u \Big\}.
\end{align*}
The term marked by $(*)$ in the above expression can be upper bounded by the concentration properties of the dynamics as done in Step 1 in this proof: 
\begin{align}
\label{eqn:bound_mdp_mdphat}
    (*) \leq \frac{4 \gamma f C_{P, \delta} R_{\max}}{(1 - \gamma)^2} \mathbb{E}_{\bs \sim d^{{\pi_{\text{out}}}}_{\mdp}(\bs)}\left[ \frac{\sqrt{|\mathcal{A}|}}{\sqrt{|\mathcal{D}(\bs)|}} \sqrt{ D_{\text{CQL}}({\pi_{\text{out}}}, \behavior)(\bs) + 1} \right]. 
\end{align}
Finally, using Equation~\ref{eqn:bound_mdp_mdphat}, we can lower-bound the policy return difference as:
\begin{align*}
\begin{small}
    J(\pi_{\text{out}}, \mdp) - J(\behavior, \mdp) \geq \beta \frac{\nu(\rho^\pi, f)}{1 - \gamma} - \beta \frac{\nu(\rho^\beta, f)}{1 - \gamma} - \frac{4 \gamma (1 -f) R_{\max}}{(1 - \gamma)^2} D(P_{\mdp}, P_{\mdphat}) - (*) - \Delta_R^u.
\end{small}
\end{align*}
Plugging the bounds for terms (a), (b) and (c) in the expression for $\zeta$ where $J(\pi_{\text{out}}, \mdp) - J(\behavior, \mdp) \geq \zeta$, we obtain:
\begin{align}
\zeta &= \left({\frac{4f \gamma R_{\max} C_{P, \delta}}{(1 - \gamma)^2}} \right)\mathbb{E}_{\bs \sim d^{\policy_{\text{out}}}_{\mdp}(\bs)}\left[ \frac{\sqrt{|\mathcal{A}|}}{\sqrt{|\mathcal{D}(\bs)|}} \sqrt{ D_{\text{CQL}}(\policy_{\text{out}}, \behavior)(\bs) + 1} \right]  + (\wedge) - \Delta_R^u \nonumber\\
\label{eqn:zeta_expression}
&~~~~~~~~~~~~+ \frac{4 (1 -f) \gamma R_{\max}}{(1 - \gamma)^2} D(P_{\mdp}, P_{\mdphat}) - \beta \frac{\nu(\rho^\pi, f)}{1 - \gamma} + \beta \frac{\nu(\rho^\beta, f)}{1 - \gamma}.
\end{align}
\end{proof}

\begin{remark}[\underline{\textbf{Interpretation of Proposition~\ref{thm:policy_improvement}}}] 
\label{remark:remark1}
Now we will interpret the theoretical expression for $\zeta$ in Equation~\ref{eqn:zeta_expression}, and discuss the scenarios when it is \emph{negative}. When the expression for $\zeta$ is negative, the policy $\pi_{\text{out}}$ is an improvement over $\behavior$ in the original MDP, $\mdp$. 

\begin{itemize}
    \item First note that we have never used the fact that the learned model $P_{\mdphat}$ is close to the actual MDP, $P_{\mdp}$ on the states visited by the behavior policy $\behavior$ in our analysis. We will use this fact now: in practical scenarios, $\nu(\rho^\beta, f)$ is expected to be smaller than $\nu(\rho^\pi, f)$, since $\nu(\rho^\beta, f)$ is directly controlled by the difference and density ratio of $\rho^\beta(\bs, \mathbf{a})$ and $d(\bs, \mathbf{a})$: $\nu(\rho^\beta, f) \leq \nu(\rho^\beta, f=1) = \sum_{\bs, \mathbf{a}} d^\behavior_{\mdphat}(\bs, \mathbf{a}) \left(d^\behavior_{\mdphat}(\bs, \mathbf{a})/d^\behavior_{\mdpbar}(\bs, \mathbf{a}) - 1\right)^2$ by Lemma~\ref{thm:line_thm} which is expected to be small for the behavior policy $\behavior$ in cases when the behavior policy marginal in the empirical MDP, $d^\behavior_{\mdpbar}(\bs, \mathbf{a})$, is broad. This is a direct consequence of the fact that the learned dynamics integrated with the policy under the learned model: $P_{\mdphat}^\behavior$ is closer to its counterpart in the empirical MDP:  $P_{\mdpbar}^\behavior$ for $\behavior$. Note that this is not true for any other policy besides the behavior policy that performs several counterfactual actions in a rollout and deviates from the data. For such a learned policy $\pi$, we incur an extra error which depends on the importance ratio of policy densities, compounded over the horizon and manifests as the $D_{\mathrm{CQL}}$ term (similar to Equation~\ref{eqn:bound_mdp_mdphat}, or Lemma D.4.1 in \citet{kumar2020conservative}). Thus, in practice, we argue that we are interested in situations where $\nu(\rho^\pi, f) > \nu(\rho^\beta, f)$, in which case by increasing $\beta$, we can make the expression for $\zeta$ in Equation~\ref{eqn:zeta_expression} negative, allowing for policy improvement.
    \item In addition, note that when $f$ is close to 1, the bound reverts to a standard model-free policy improvement bound and when $f$ is close to 0, the bound reverts to a typical model-based policy improvement bound. In scenarios with high sampling error (i.e. smaller $|\mathcal{D}(\bs)|$), if we can learn a good model, i.e., $D(P_{\mdp}, P_{\mdphat})$ is small, we can attain policy improvement better than model-free methods by relying on the learned model by setting $f$ closer to 0. A similar argument can be made in reverse for handling cases when learning an accurate dynamics model is hard. 
\end{itemize}
\end{remark}

% \begin{theorem}[Upper bound on $\nu(\rho, f)$]
% If the distributions $\rho(\bs, \mathbf{a})$ and $d(\bs, \mathbf{a})$ are such that $\sum_{\bs, \mathbf{a}} (\rho(\bs, \mathbf{a}) - d(\bs, \mathbf{a}))^2 \leq \varepsilon$, then the value of $\nu(\rho, f) \leq $. 
% \end{theorem}
% \begin{proof}
% To obtain a bound on $\nu(\rho, f)$, we solve the following optimization problem over $\rho$:
% \begin{align*}
%     \max_{\rho}&~~~ \nu(\rho, f):= \sum_{\bs, \mathbf{a}} \rho(\bs, \mathbf{a}) \frac{\rho(\bs, \mathbf{a}) - d(\bs, \mathbf{a})}{f d (\bs, \mathbf{a}) + (1 - f) \rho(\bs, \mathbf{a})}\\
%     &\text{s.t.}~~ \sum_{\bs, \mathbf{a}} (\rho(\bs, \mathbf{a}) - d(\bs, \mathbf{a}))^2 \leq \varepsilon, ~~ \sum_{\bs, \mathbf{a}} \rho(\bs, \mathbf{a}) = 1, ~~ \rho(\bs, \mathbf{a}) \geq 0.
% \end{align*}
% We first note that for any optimal $\rho=\rho^*$, the objective value is largest for $f = 1$, and thus, solving the above optimization problem for $f=1$ gives an upper bound on the objective value. Converting the problem for $f=1$ to a minimization problem and writing out the Lagrangian for optimization, we obtain:
% \begin{multline}
%     \mathcal{L}(\rho; \lambda, \alpha, \eta) = -\sum_{\bs, \mathbf{a}} d(\bs, \mathbf{a}) \frac{\rho(\bs, \mathbf{a})}{d(\bs, \mathbf{a})}  \left( \frac{\rho(\bs, \mathbf{a})}{d(\bs, \mathbf{a})} - 1 \right) + \lambda \left(\sum_{\bs, \mathbf{a}} d(\bs, \mathbf{a})^2 \left(\frac{\rho(\bs, \mathbf{a})}{d(\bs, \mathbf{a})} - 1 \right)^2 - \varepsilon \right) \\ - \eta \left(\sum_{\bs, \mathbf{a}} d(\bs, \mathbf{a}) \frac{\rho(\bs, \mathbf{a})}{d(\bs, \mathbf{a})} - 1 \right) - \sum_{\bs, \mathbf{a}} \alpha(\bs, \mathbf{a}) \frac{\rho(\bs, \mathbf{a})}{d(\bs, \mathbf{a})}.
% \end{multline}
% Noting the change of variable transformation: $w(\bs, \mathbf{a}) := \frac{\rho(\bs, \mathbf{a})}{d(\bs, \mathbf{a})} - 1$, we obtain the following optimization problem:
% \begin{equation*}
%     \mathcal{L}(w; \lambda, \alpha, \eta) = -\sum_{\bs, \mathbf{a}} d(\bs, \mathbf{a}) w(\bs, \mathbf{a})^2 + \lambda \left( \sum_{\bs, \mathbf{a}} d(\bs, \mathbf{a})^2 w(\bs, \mathbf{a})^2 - \varepsilon \right) - \eta \sum_{\bs, \mathbf{a}} d(\bs, \mathbf{a}) w(\bs, \mathbf{a}) - \sum_{\bs, \mathbf{a}} \alpha(\bs, \mathbf{a}) (w(\bs, \mathbf{a}) + 1).
% \end{equation*}
% Taking the derivative with respect to $w(\bs, \mathbf{a})$ and utilizing KKT conditions we obtain
% \begin{align}
% &- 2 d(\bs, \mathbf{a}) w(\bs, \mathbf{a}) + 2 \lambda d(\bs, \mathbf{a})^2 w(\bs, \mathbf{a}) - \eta d(\bs, \mathbf{a}) - \alpha(\bs, \mathbf{a}) = 0   \label{eq:grad}\\
% &\lambda \left( \sum_{\bs, \mathbf{a}} d(\bs, \mathbf{a})^2 w(\bs, \mathbf{a})^2 - \varepsilon \right) = 0.  \label{eq:slack1}\\
% & \alpha(\bs, \mathbf{a}) (w(\bs, \mathbf{a}) + 1) = 0 ~~ \forall \bs, \mathbf{a}.  \label{eq:slack2}
% \end{align}
% Multiplying Equation~\ref{eq:grad} by $w(\bs, \mathbf{a})$ and adding both LHS and RHS over $(\bs, \mathbf{a})$ we obtain:
% \begin{equation}
%     \label{eq:temp_add}
%     - 2 \sum_{\bs, \mathbf{a}} d(\bs, \mathbf{a}) w(\bs, \mathbf{a})^2 + 2 \underbrace{\lambda \sum_{\bs, \mathbf{a}} d(\bs, \mathbf{a})^2 w(\bs, \mathbf{a})^2}_{= \lambda \varepsilon} - \underbrace{\eta \sum_{\bs, \mathbf{a}} d(\bs, \mathbf{a}) w(\bs, \mathbf{a})}_{= \eta \times 0 = 0} = \sum_{\bs, \mathbf{a}} \alpha(\bs, \mathbf{a}) w(\bs, \mathbf{a}),
% \end{equation}
% and similarly, adding Equation~\ref{eq:grad} over $(\bs, \mathbf{a})$ we get:
% \begin{equation}
%     \label{eq:simple_add}
%     - 2 \underbrace{\sum_{\bs, \mathbf{a}} d(\bs, \mathbf{a}) w(\bs, \mathbf{a})}_{= 0} + 2 \lambda \sum_{\bs, \mathbf{a}} d(\bs, \mathbf{a})^2 w(\bs, \mathbf{a}) - \eta = \sum_{\bs, \mathbf{a}} \alpha(\bs, \mathbf{a}).
% \end{equation}
% Finally, from Equation~\ref{eq:grad}, we get that the value of $w(\bs, \mathbf{a})$ is given by:
% \begin{equation*}
%     w(\bs, \mathbf{a}) = \frac{\eta d(\bs, \mathbf{a}) + \alpha (\bs, \mathbf{a})}{2 \lambda d(\bs, \mathbf{a})^2 - 2 d(\bs, \mathbf{a})}
% \end{equation*}
% Adding Equations~\ref{eq:temp_add} and \ref{eq:simple_add}, we obtain:
% \begin{equation*}
%     2 \sum_{\bs, \mathbf{a}} d(\bs, \mathbf{a}) w(\bs, \mathbf{a}) \left[\lambda d(\bs, \mathbf{a}) - w(\bs, \mathbf{a}) \right] + \lambda \varepsilon - \eta = 0. 
% \end{equation*}
% \end{proof}

\section{Experimental Details for COMBO}
\label{app:details}

In this section, we include all details of our empirical evaluations of COMBO.

\subsection{Practical algorithm implementation details}
\label{app:combo_details}

\paragraph{Model training.}

In the setting where the observation space is low-dimensional, as mentioned in Section~\ref{sec:combo},  we represent the model as a probabilistic neural network that outputs a Gaussian distribution over the next state and reward given the current state and action: $$\widehat{T}_\theta(\bs_{t+1}, r| \bs, \mathbf{a}) = \mathcal{N}(\mu_\theta(\bs_t, \mathbf{a}_t), \Sigma_\theta(\bs_t, \mathbf{a}_t)).$$ We train an ensemble of $7$ such dynamics models following \cite{janner2019mbpo} and pick the best $5$ models based on the validation prediction error on a held-out set that contains $1000$ transitions in the offline dataset $\data$. During model rollouts, we randomly pick one dynamics model from the best $5$ models. Each model in the ensemble is represented as a 4-layer feedforward neural network with $200$ hidden units. For the generalization experiments in Section~\ref{sec:generalization_exps}, we additionally use a two-head architecture to output the mean and variance after the last hidden layer following \cite{yu2020mopo}.

In the image-based setting, we follow \citet{Rafailov2020LOMPO} and use a variational model with the following components:

\begin{gather}
\begin{aligned}
&\text{Image encoder:} && \mathbf{h}_t=E_\theta(\bo_t) \\
&\text{Inference model:} && \bs_t \sim q_\theta(\bs_t|\mathbf{h}_t, \bs_{t-1}, \mathbf{a}_{t-1})\\
&\text{Latent transition model:} &&\bs_t \sim \widehat{T}_\theta(\bs_t| \bs_{t-1}, \mathbf{a}_{t-1})\\
&\text{Reward predictor:} && r_t \sim p_\theta(r_t|\bs_t) \\
&\text{Image decoder:} && \bo_t \sim D_\theta(\bo_t|\bs_t).
\label{eq:latent_model}
\end{aligned}
\end{gather}%

We train the model using the evidence lower bound:

$$\max_{\theta}\sum_{\tau=0}^{T-1}\Big[\mathbb{E}_{q_{\theta}}[\log D_{\theta}(\bo_{\tau+1}|\bs_{\tau+1})]\Big]-\mathbb{E}_{q_{\theta}}\Big[D_{KL}[q_{\theta}(\bo_{\tau+1}, \bs_{\tau+1}|\bs_{\tau}, \mathbf{a}_{\tau})\|\widehat{T}_{\theta_{\tau}}(\bs_{\tau+1}, a_{\tau+1})]\Big]$$

At each step $\tau$ we sample a latent forward model $\widehat{T}_{\theta_{\tau}}$ from a fixed set of $K$ models $[\widehat{T}_{\theta_1},\ldots, \widehat{T}_{\theta_K}]$. For the encoder $E_{\theta}$ we use a convolutional neural network with kernel size 4 and stride 2. For the Walker environment we use 4 layers, while the Door Opening task has 5 layers. The $D_{\theta}$ is a transposed convolutional network with stride 2 and kernel sizes $[5,5,6,6]$ and $[5,5,5,6,6]$ respectively. The inference network has a two-level structure similar to \citet{Hafner2019PlanNet} with a deterministic path using a GRU cell with 256 units and a stochastic path implemented as a conditional diagonal Gaussian with 128 units. We only train an ensemble of stochastic forward models, which are also implemented as conditional diagonal Gaussians.


\paragraph{Policy Optimization.} We sample a batch size of $256$ transitions for the critic and policy learning. We set $f = 0.5$, which means we sample $50\%$ of the batch of transitions from $\data$ and another $50\%$ from $\data_\text{model}$. The equal split between the offline data and the model rollouts strikes the balance between conservatism and generalization in our experiments as shown in our experimental results in Section~\ref{sec:combo_exp}. We represent the Q-networks and policy as 3-layer feedforward neural networks with $256$ hidden units.

For the choice of $\rho(\bs,\mathbf{a})$ in Equation~\ref{eq:implicit_update}, we can obtain the Q-values that lower-bound the true value of the learned policy $\pi$ by setting $\rho(\bs,\mathbf{a}) = d^\policy_{\mdphat} (\bs) \pi(\mathbf{a} | \bs)$. However, as discussed in \cite{kumar2020conservative}, computing $\pi$ by alternating the full off-policy evaluation for the policy $\hat{\pi}^k$ at each iteration $k$ and one step of policy improvement is computationally expensive. Instead, following \cite{kumar2020conservative}, we pick a particular distribution $\psi(\mathbf{a}|\bs)$ that approximates the the policy that maximizes the Q-function at the current iteration and set $\rho(\bs,\mathbf{a}) = d^\policy_{\mdphat} (\bs) \psi(\mathbf{a} | \bs)$. We formulate the new objective as follows:
\begin{small}
\begin{align}
    \hat{Q}^{k+1} \leftarrow& \arg\min_{Q}\beta\left(\E_{\bs \sim d^\policy_{\mdphat} (\bs), \mathbf{a}\sim \psi(\mathbf{a} | \bs)}\!\left[Q(\bs,\mathbf{a})\right]-\E_{\bs, \mathbf{a} \sim \data}\left[Q(\bs,\mathbf{a})\right]\right)\nonumber\\
    &+ \frac{1}{2}\E_{\bs, \mathbf{a}, \bs' \sim d_f}\left[ \left(Q(\bs, \mathbf{a}) - \widehat{\bellman}^\policy\hat{Q}^k(\bs, \mathbf{a}))\right)^2 \right] + \mathcal{R}(\psi),
    \label{eq:combo_update_practical}
\end{align}
\end{small}
where $\mathcal{R}(\psi)$ is a regularizer on $\psi$. In practice, we pick $\mathcal{R}(\psi)$ to be the $-D_\text{KL}(\psi(\mathbf{a}|\bs)\|\text{Unif}(\mathbf{a}))$ and under such a regularization, the first term in Equation~\ref{eq:combo_update_practical} corresponds to computing softmax of the Q-values at any state $\bs$ as follows:
\begin{small}
\begin{align}
    \hat{Q}^{k+1} \leftarrow& \arg\min_{Q}\max_\psi\beta\left(\E_{\bs \sim d^\policy_{\mdphat} (\bs)}\!\left[\log\sum_\mathbf{a} Q(\bs,\mathbf{a})\right]-\E_{\bs, \mathbf{a} \sim \data}\left[Q(\bs,\mathbf{a})\right]\right) \nonumber\\
    &+ \frac{1}{2}\E_{\bs, \mathbf{a}, \bs' \sim d_f}\left[ \left(Q(\bs, \mathbf{a}) - \widehat{\bellman}^\policy\hat{Q}^k(\bs, \mathbf{a}))\right)^2 \right].
    \label{eq:combo_logsumexp}
\end{align}
\end{small}
We estimate the \texttt{log-sum-exp} term in Equation~\ref{eq:combo_logsumexp} by sampling $10$ actions at every state $\bs$ in the batch from a uniform policy $\text{Unif}(\mathbf{a})$ and the current learned policy $\pi(\mathbf{a}|\bs)$ with importance sampling following \cite{kumar2020conservative}.

\subsection{Hyperparameter Selection for COMBO}
\label{app:hyperparameter}

\neurips{In this section, we discuss the hyperparameters that we use for COMBO. In the D4RL and generalization experiments, our method are built upon the implementation of MOPO provided at: \url{https://github.com/tianheyu927/mopo}. The hyperparameters used in COMBO that relates to the backbone RL algorithm SAC such as twin Q-functions and number of gradient steps follow from those used in MOPO with the exception of smaller critic and policy learning rates, which we will discuss below. In the image-based domains, COMBO is built upon LOMPO without any changes to the parameters used there. For the evaluation of COMBO, we follow the evaluation protocol in D4RL~\citep{fu2020d4rl} and a variety of prior offline RL works~\citep{kumar2020conservative,yu2020mopo,kidambi2020morel} and report the normalized score of the smooth undiscounted averaged return over $3$ random seeds for all environments except \texttt{sawyer-door-close} and \texttt{sawyer-door} where we report the average success rate over $3$ random seeds.}

\neurips{We now list the additional hyperparameters as follows.
\begin{itemize}
    \item \textbf{Rollout length $h$.} We perform a short-horizon model rollouts in COMBO similar to \citet{yu2020mopo} and \citet{Rafailov2020LOMPO}. For the D4RL experiments and generalization experiments, we followed the defaults used in MOPO and used $h = 1$ for walker2d and \texttt{sawyer-door-close}, $h=5$ for hopper, halfcheetah and \texttt{halfcheetah-jump}, and $h=25$ for \texttt{ant-angle}. In the image-based domain we used rollout length of $h=5$ for both the the \texttt{walker-walk} and \texttt{sawyer-door-open} environments following the same hyperparameters used in \citet{Rafailov2020LOMPO}.
    \item \textbf{Q-function and policy learning rates.} On state-based domains, we searched over $\{1e-4, 3e-4\}$ for the Q-function learning rate and $\{1e-5, 3e-5, 1e-4\}$ for the policy learning rate. 
    We found that $3e-4$ for the Q-function learning rate (also used previously in \citet{kumar2020conservative}) and $1e-4$ for the policy learning rate (also recommended previously in \citet{kumar2020conservative} for gym domains) work well for almost all domains except that on walker2d where a smaller Q-function learning rate of $1e-4$ and a correspondingly smaller policy learning rate of $1e-5$ works the best. In the image-based domains, we followed the defaults from prior work \citep{Rafailov2020LOMPO} and used $3e-4$ for both the policy and Q-function.
    
    \item \textbf{Conservative coefficient $\beta$.} 
    % As noted in our theoretical results in Lemma~\ref{thm:line_thm}, the amount of conservatism depends on the choice of fraction $f$ and $\rho(\bs, \mathbf{a})$. In principle, we only need to control one of these factors, $\rho$, $f$, $\beta$ to obtain the right degree of conservatism. Since we do not alter $f$ and $\rho(\bs, \mathbf{a})$ for different quality datasets (see Appendix~\ref{app:combo_details} for our choice of $f$; $\rho$ was chosen based on model-prediction error as discussed next) we instead choose values of $\beta$ for different dataset types.
    We searched over $\{0.5, 1.0, 5.0\}$ for $\beta$, which correspond to low conservatism, medium conservatism and high conservatism.  A larger $\beta$ would be desirable in more narrow dataset distributions with lower-coverage of the state-action space that propagates error in a backup whereas a smaller $\beta$ is desirable with diverse dataset distributions. On the D4RL experiments, we found that $\beta = 0.5$ works well for halfcheetah agnostic of dataset quality, while on hopper and walker2d, we found that the more ``narrow'' dataset distributions: medium and medium-expert datasets work best with larger $\beta = 5.0$ whereas more ``diverse'' dataset distributions: random and medium-replay datasets work best with smaller $\beta$ ($\beta = 0.5$ for walker2d and $\beta = 1.0$ for hopper) which is consistent with the intuition. 
    % An intuitive explanation would be that on medium and medium-expert datasets where the data distribution is narrow, we need to be more conservative and hence large $\beta$ while on random and medium-replay datasets where the distribution is diverse and cover most of the state space, we require less conservatism. 
    On generalization experiments, $\beta = 1.0$ works best for all environments. In the image-domains we use $\beta=0.5$ for the medium-replay \texttt{walker-walk} task and and $\beta=1.0$ for all other domains, which again is in accordance with the impact of $\beta$ on performance.
    
    
    \item \textbf{Choice of $\rho(\bs,\mathbf{a})$.} We first decouple $\rho(\bs,\mathbf{a}) = \rho(\bs)\rho(\mathbf{a}|\bs)$ for convenience. As discussed in Appendix~\ref{app:combo_details}, we use $\rho(\mathbf{a}|\bs)$ as the soft-maximum of the Q-values and estimated with \texttt{log-sum-exp}. For $\rho(\bs)$, we searched over $\{d^\policy_{\mdphat}, \rho(\bs)=d_f\}$.  We found that $d^\policy_{\mdphat}$ works better the hopper task in D4RL while $d_f$ is better for the rest of the environments. For the remaining domains, we found $\rho(\bs)=d_f$ works well.
    
    
    \item \textbf{Choice of $\mu(\mathbf{a}|\bs)$.} For the rollout policy $\mu$, we searched $\{\text{Unif}(\mathbf{a}), \pi(\mathbf{a}|\bs)\}$, i.e. the set that contains a random policy and a current learned policy. We found that $\mu(\mathbf{a}|\bs) = \text{Unif}(\mathbf{a})$ works well on the hopper task in D4RL and also in the $\texttt{ant-angle}$ generalization experiment. For the remaining state-based environments, we discovered that $\mu(\mathbf{a}|\bs) = \pi(\mathbf{a}|\bs)$ excels. In the image-based domain, we found that $\mu(\mathbf{a}|\bs) = \text{Unif}(\mathbf{a})$ works well in the \texttt{walker-walk} domain and  $\mu(\mathbf{a}|\bs) = \pi(\mathbf{a}|\bs)$ is better for the \texttt{sawyer-door} environment. 
    % Similar to the choice of $\rho(\bs)$, 
    We observed that
    $\mu(\mathbf{a}|\bs) = \text{Unif}(\mathbf{a})$ behaves less conservatively and is suitable to tasks where dynamics models can be learned fairly precisely.
    \item \textbf{Choice of Backup.} Following CQL~\citep{kumar2020conservative}, we use the standard deterministic backup for COMBO.
    \item \textbf{Choice of $f$.} For the ratio between model rollouts and offline data $f$, we searched $\{0.5, 0.8\}$. We found that $f = 0.8$ works well on the medium and medium-expert in the walker2d task in D4RL. For the remaining tasks, we find $f = 0.5$ works well.
\end{itemize}}

\subsection{Automatic Hyperparameter Selection Rule for COMBO}

It is common in prior work on offline RL to select various hyperparameters using online policy rollouts~\citep{yu2020mopo,kidambi2020morel,argenson2020model,lee2021representation}. Requiring online rollouts to tune hyperparameters contradicts the main aim of offline RL, which is to learn entirely from offline data. Therefore, we attempted to devise an automated rule for tuning important hyperparameters such as $\beta$ and $f$ in a fully offline manner in COMBO. We search over a discrete set of hyperparameters for each task as dicussed above, and use the value of the regularization term $\mathbb{E}_{\mathbf{s}, \mathbf{a} \sim \rho(\mathbf{s},\mathbf{a})}\!\left[Q(\mathbf{s},\mathbf{a})\right]\!-\!\mathbb{E}_{\mathbf{s}, \mathbf{a} \sim \data}\!\left[Q(\mathbf{s},\mathbf{a})\right]$ (shown in Eq.~\ref{eq:implicit_update}) to evaluate the hyperparameters. This automated rule picks the hyperparameter setting which achieves the lowest regularization objective, which indicates that the Q-values on unseen model-predicted state-action tuples are not overestimated.
%%CF.9.30: The ICLR AC also wanted to see a discussion of how this offline selection scheme compared to prior methods for offline selection. Maybe discuss this somewhere? (perhaps in the appendix if space is short?)
%%TY.10.1: I discussed this in Appendix B.2.
%%SL.10.2: I slightly rephrased the paragraph above in a way that hopefully further avoids potential misunderstandings.

Below, we provide additional experimental validation showing the effiacy of this automatic hyperparameter selection rule from above. As shown in Table~\ref{tab:beta_selection},~\ref{tab:mu_selection}, ~\ref{tab:rho_selection} and \ref{tab:f_selection}, the above proposed automatic hyperparameter selection rule is able to pick the hyperparameters $\beta$, $\mu(\mathbf{a}|\bs)$, $\rho(\bs)$ and $f$ and  that correspond to the best policy performance based on the regularization value.

\begin{table}[ht]
    \centering
    \scriptsize
    \resizebox{1.0\textwidth}{!}{\begin{tabular}{l|r|r|r|r|}
    \toprule
    Task & $\beta=0.5$ & $\beta=0.5$ & $\beta=5.0$ & $\beta=5.0$\\
 & performance & regularizer value & performance & regularizer value\\
 \midrule
halfcheetah-medium &  \textbf{54.2}  & \textbf{-778.6}  & 40.8  & -236.8  \\
halfcheetah-medium-replay &  \textbf{55.1} & \textbf{28.9} & 9.3 & 283.9\\ 
halfcheetah-medium-expert & 89.4 & 189.8 & \textbf{90.0}  & \textbf{6.5}\\
hopper-medium      &  75.0  & -740.7  &\textbf{97.2}  & \textbf{-2035.9}\\
hopper-medium-replay & \textbf{89.5} & \textbf{37.7} & 28.3       & 107.2\\
hopper-medium-expert & \textbf{111.1}       & \textbf{-705.6}    & 75.3 &       -64.1\\
walker2d-medium        &  1.9  & 51.5  & \textbf{81.9}  & \textbf{-1991.2}\\
walker2d-medium-replay & \textbf{56.0}       & \textbf{-157.9}    & 27.0       & 53.6\\
walker2d-medium-expert & 10.3       & -788.3    &\textbf{103.3}       & \textbf{-3891.4}\\
    \bottomrule
\end{tabular}}
\caption{\footnotesize We include our automatic hyperparameter selection rule of $\beta$ on a set of representative D4RL environments. We show the policy performance (bold with the higher number) and the regularizer value (bold with the lower number). Lower regularizer value consistently corresponds to the higher policy return, suggesting the effectiveness of our automatic selection rule.}
\label{tab:beta_selection}
\end{table}

\begin{table}[ht]
    \centering
    \scriptsize
    \resizebox{1.0\textwidth}{!}{\begin{tabular}{l|r|r|r|r|}
    \toprule
    Task & $\mu(\mathbf{a}|\bs)=\text{Unif}(\mathbf{a})$ & $\mu(\mathbf{a}|\bs)=\text{Unif}(\mathbf{a})$            &$\mu(\mathbf{a}|\bs)=\pi(\mathbf{a}|\bs)$&$\mu(\mathbf{a}|\bs)=\pi(\mathbf{a}|\bs)$\\
 & performance & regularizer value & performance & regularizer value\\
 \midrule
hopper-medium        & \textbf{97.2}  & \textbf{-2035.9} &  52.6  & -14.9  \\
walker2d-medium        &  7.9  & -106.8  & \textbf{81.9}  & \textbf{-1991.2} \\
    \bottomrule
    \end{tabular}}
    \caption{\footnotesize We include our automatic hyperparameter selection rule of $\mu(\mathbf{a}|\bs)$ on the medium datasets in the hopper and walker2d environments from D4RL. We follow the same convention defined in Table~\ref{tab:beta_selection} and find that our automatic selection rule can effectively select $\mu$ offline.}
    \label{tab:mu_selection}
\end{table}

\begin{table}[ht]
    \centering
    \scriptsize
    \resizebox{0.9\textwidth}{!}{\begin{tabular}{l|r|r|r|r|}
    \toprule
    Task & $\rho(\bs) = d^\pi_{\hat{\mathcal{M}}} $&$\rho(\bs) = d^\pi_{\hat{\mathcal{M}}}$            &$\rho(\bs) = d_f$&$\rho(\bs) = d_f$\\
 & performance & regularizer value & performance & regularizer value\\
 \midrule
hopper-medium        & \textbf{97.2}  & \textbf{-2035.9} &  56.0  & -6.0  \\
walker2d-medium        &  1.8  & 14617.4  & \textbf{81.9}  & \textbf{-1991.2} \\
    \bottomrule
    \end{tabular}}
    \caption{\footnotesize We include our automatic hyperparameter selection rule of $\rho(\bs)$ on the medium datasets in the hopper and walker2d environments from D4RL. We follow the same convention defined in Table~\ref{tab:beta_selection} and find that our automatic selection rule can effectively select $\rho$ offline.}
    \label{tab:rho_selection}
\end{table}

\begin{table}[ht]
    \centering
    \scriptsize
    \resizebox{0.9\textwidth}{!}{\begin{tabular}{l|r|r|r|r|}
    \toprule
    Task & $f = 0.5 $&$f = 0.5$            &$f = 0.8$&$f = 0.8$\\
 & performance & regularizer value & performance & regularizer value\\
 \midrule
hopper-medium        & \textbf{97.2}  & \textbf{-2035.9} &  93.8  & -21.3  \\
walker2d-medium        &  70.9  & -1707.0  & \textbf{81.9}  & \textbf{-1991.2} \\
    \bottomrule
    \end{tabular}}
    \caption{\footnotesize We include our automatic hyperparameter selection rule of $f$ on the medium datasets in the hopper and walker2d environments from D4RL. We follow the same convention defined in Table~\ref{tab:beta_selection} and find that our automatic selection rule can effectively select $f$ offline.}
    \label{tab:f_selection}
\end{table}

\subsection{Details of generalization environments}
\label{app:ood_details}

For \texttt{halfcheetah-jump} and \texttt{ant-angle}, we follow the same environment used in MOPO. For \texttt{sawyer-door-close}, we train the \texttt{sawyer-door} environment in \url{https://github.com/rlworkgroup/metaworld} with dense rewards for opening the door until convergence. We collect $50000$ transitions with half of the data collected by the final expert policy and a policy that reaches the performance of about half the expert level performance. We relabel the reward such that the reward is $1$ when the door is fully closed and $0$ otherwise. Hence, the offline RL agent is required to learn the behavior that is different from the behavior policy in a sparse reward setting. We provide the datasets in the following anonymous link\footnote{The datasets of the generalization environments are available at the following link: \url{https://drive.google.com/file/d/1pn6dS5OgPQVp_ivGws-tmWdZoU7m_LvC/view?usp=sharing}.}.

\subsection{Details of image-based environments}
\label{app:image_details}

\begin{figure}[ht]
    \centering
    \includegraphics[width=0.25\textwidth]{chapters/combo/walker_task.png}
    \includegraphics[width=0.25\textwidth]{chapters/combo/dooropen_task.png}
    \vspace{-0.2cm}
    \caption{\footnotesize Our image-based environments: The observations are $64\times 64$ and $128\times 128$ raw RGB images for the \texttt{walker-walk} and \texttt{sawyer-door} tasks respectively. The \texttt{sawyer-door-close} environment used in in Section~\ref{sec:generalization_exps} also uses the \texttt{sawyer-door} environment.}
    \label{fig:visual}
\end{figure}


We visualize our image-based environments in Figure~\ref{fig:visual}. We use the standard \texttt{walker-walk} environment from \citet{tassa2018deepmind} with $64\times64$ pixel observations and an action repeat of 2. Datasets were constructed the same way as \citet{fu2020d4rl} with 200 trajectories each. For the \texttt{sawyer-door} we use $128\times128$ pixel observations. The medium-expert dataset contains 1000 rollouts (with a rollout length of 50 steps) covering the state distribution from grasping the door handle to opening the door. The expert dataset contains 1000 trajectories samples from a fully trained (stochastic) policy. The data was obtained from the training process of a stochastic SAC policy using dense reward function as defined in \citet{yu2020metaworld}. However, we relabel the rewards, so an agent receives a reward of 1 when the door is fully open and 0 otherwise. This aims to evaluate offline-RL performance in a sparse-reward setting. All the datasets are from \citep{Rafailov2020LOMPO}.


\section{Comparing COMBO to the Naive Combination of CQL and MBPO}
\label{app:cql_mbpo}

\iclr{In this section, we stress the distinction between COMBO and a direct combination of two previous methods CQL and MBPO (denoted as CQL + MBPO). CQL+MBPO performs Q-value regularization using CQL while expanding the offline data with MBPO-style model rollouts. While COMBO utilizes Q-value regularization similar to CQL, the effect is very different. CQL only penalizes the Q-value on unseen actions on the states observed in the dataset whereas COMBO penalizes Q-values on states generated by the learned model while maximizing Q values on state-action tuples in the dataset. Additionally, COMBO also utilizes MBPO-style model rollouts for also augmenting samples for training Q-functions.

To empirically demonstrate the consequences of this distinction, CQL + MBPO performs quite a bit worse than COMBO on generalization experiments (Section~\ref{sec:generalization_exps}) as shown in Table~\ref{tbl:cql_mbpo}. The results are averaged across 6 random seeds ($\pm$ denotes 95\%-confidence interval of the various runs). This suggests that carefully considering the state distribution, as done in COMBO, is crucial.}

\begin{table}[ht]
    \centering
    \scriptsize
    \resizebox{0.7\textwidth}{!}{\begin{tabular}{l|r|r|r|r|}
    \toprule 
    %
    %
    %
    \textbf{Environment} & \stackanchor{\textbf{Batch}}{\textbf{Mean}} & \stackanchor{\textbf{Batch}}{\textbf{Max}} & \stackanchor{\textbf{COMBO}}{\textbf{(Ours)}} & \textbf{CQL+MBPO}\\ \midrule
    halfcheetah-jump & -1022.6 & 1808.6 & \textbf{5392.7}$\pm$575.5 & 4053.4$\pm$176.9\\
    ant-angle & 866.7 & 2311.9 & \textbf{2764.8}$\pm$43.6 & 809.2$\pm$135.4\\
    sawyer-door-close & 5\% & 100\% & \textbf{100}\%$\pm$0.0\% & 62.7\%$\pm$24.8\%\\
    \bottomrule
    \end{tabular}}
    \vspace{-0.2cm}
    \caption{
    \footnotesize Comparison between COMBO and CQL+MBPO on tasks that require out-of-distribution generalization. Results are in average returns of \texttt{halfcheetah-jump} and \texttt{ant-angle} and average success rate of \texttt{sawyer-door-close}. All results are averaged over 6 random seeds, $\pm$ the $95\%$-confidence interval.
    }
    \vspace{-0.3cm}
    \label{tbl:cql_mbpo}
    \normalsize
    \end{table}
    

% \subsection{Computation Complexity}

% For the D4RL and generalization experiments, COMBO is trained on a single NVIDIA GeForce RTX 2080 Ti for one day. For the image-based experiments, we utilized a single NVIDIA GeForce RTX 2070. We trained the \texttt{walker-walk} tasks for a day and the \texttt{sawyer-door-open} tasks for about two days.

% \subsection{License of datasets}

% We acknowledge that all datasets used in this paper use the MIT license.

% % \vspace{1cm}
% \section{Empirical Evidence on Challenges of Uncertainty Quantification}
% \label{app:uq}

% \begin{figure}[t]
%     \centering
%     \includegraphics[width=0.47\linewidth]{halfcheetah_medium_corr_var_ood.png}
%     \includegraphics[width=0.47\linewidth]{halfcheetah_medium_corr_lip_ens_ood.png}
%     \includegraphics[width=0.47\linewidth]{hopper_medium_corr_var_ood.png}
%     \includegraphics[width=0.47\linewidth]{hopper_medium_corr_lip_ens_ood.png}
%     \includegraphics[width=0.47\linewidth]{walker_medium_corr_var_ood.png}
%     \includegraphics[width=0.47\linewidth]{walker_medium_corr_lip_ens_ood.png}
%     \vspace{-0.2cm}
%     \caption{\footnotesize
%     %
%     We visualize the correlation between the model error and two uncertainty quantification methods maximum learned variance over the ensemble (left column) and variance of the model prediction over the ensemble (right column) on three D4RL medium datasets (from the top row to the bottom row: halfcheetah, hopper and walker) where MOPO performs poorly compared to model-free methods. We show that \textbf{Max Var} tends to be overly conservative and overestimating the model error while \textbf{Ens. Var} is on the opposite. Such visualizations corroborate that uncertainty quantification is challenging with deep neural networks and could lead to poor performance in model-based offline RL. In the meantime, COMBO addresses this issue by removing the burden of performing uncertainty quantification.}
%     \label{fig:uq}
%     \vspace{-0.3cm}
% \end{figure}

% In this section, we perform empirical evaluations to show that uncertainty quantification with deep neural networks, especially in the setting of dynamics model learning, is challenging and could cause problems with uncertainty-based model-based offline RL methods such as MOReL~\citep{kidambi2020morel} and MOPO~\citep{yu2020mopo}. In our evaluations, we consider two uncertainty quantification methods, maximum learned variance over the ensemble (denoted as \textbf{Max Var}) $\max_{i=1,\dots,N}\|\Sigma^i_\theta(\bs,\mathbf{a})\|_\text{F}$ (used in MOPO) and the variance of the model prediction over the ensemble (denoted as \textbf{Ens. Var}) $\max_{i=1,\dots,N}\|\mu^i_\theta(\bs,\mathbf{a}) - \frac{1}{N}\sum_{j=1}^N\mu^j_\theta(\bs,\mathbf{a})\|_2$ (used in MOPO and MOReL) where we use an ensemble of $N$ probabilistic dynamics models $\{\widehat{T}^i_\theta(\bs_{t+1}, r| \bs, \mathbf{a}) = \mathcal{N}(\mu^i_\theta(\bs_t, \mathbf{a}_t), \Sigma^i_\theta(\bs_t, \mathbf{a}_t))\}_{i=1}^N$.

% As shown in Table~\ref{tbl:d4rl}, MOPO performs underwhelmingly on medium datasets in the D4RL datasets where the dataset is collected with a single policy and hence with relatively narrow data coverage of the whole state space. To empirically analyze the poor performance of MOPO on those datasets, we visualize the correlation between the true model error and two uncertainty quantification methods \textbf{Max Var} and \textbf{Ens. Var}. We normalize both the model error and the uncertainty estimates to be within scale $[0, 1]$. As shown in Figure~\ref{fig:uq}, on all three medium datasets, \textbf{Max Var} tends to be overly conservative and \textbf{Ens. Var} behaves too optimistic to correctly quantify the true model error, suggesting that uncertainty estimation used by MOPO is not accurate and might be the major factor that results in its poor performance. Meanwhile, COMBO circumvents challenging uncertainty quantification problem and achieves much better performances on those medium datasets, indicating the effectiveness and the robustness of the method.
%%%%%%%%%%%%%%%%%%%%%%%%%%%%%%%%%%%%%%%%%%%%%%%%%%%%%%%%%%%%%%%%%%%%%%%%%%%%%%%
%%%%%%%%%%%%%%%%%%%%%%%%%%%%%%%%%%%%%%%%%%%%%%%%%%%%%%%%%%%%%%%%%%%%%%%%%%%%%%%


\end{document}


% This document was modified from the file originally made available by
% Pat Langley and Andrea Danyluk for ICML-2K. This version was created
% by Iain Murray in 2018, and modified by Alexandre Bouchard in
% 2019. Previous contributors include Dan Roy, Lise Getoor and Tobias
% Scheffer, which was slightly modified from the 2010 version by
% Thorsten Joachims & Johannes Fuernkranz, slightly modified from the
% 2009 version by Kiri Wagstaff and Sam Roweis's 2008 version, which is
% slightly modified from Prasad Tadepalli's 2007 version which is a
% lightly changed version of the previous year's version by Andrew
% Moore, which was in turn edited from those of Kristian Kersting and
% Codrina Lauth. Alex Smola contributed to the algorithmic style files.
