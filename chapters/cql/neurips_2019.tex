
\documentclass{article}

% if you need to pass options to natbib, use, e.g.:
%     \PassOptionsToPackage{numbers, compress}{natbib}
% before loading neurips_2019

% ready for submission
% \usepackage{neurips_2019}

% to compile a preprint version, e.g., for submission to arXiv, add add the
% [preprint] option:
%     \usepackage[preprint]{neurips_2019}

% to compile a camera-ready version, add the [final] option, e.g.:
\usepackage[nonatbib,final]{neurips_2019}
\usepackage[numbers]{natbib}

\usepackage[colorlinks=true,allcolors=blue]{hyperref}


% to avoid loading the natbib package, add option nonatbib:
%     \usepackage[nonatbib]{neurips_2019}
\usepackage{graphicx}
\usepackage[utf8]{inputenc} % allow utf-8 input
\usepackage[T1]{fontenc}    % use 8-bit T1 fonts
\usepackage{hyperref}       % hyperlinks
\usepackage{url}            % simple URL typesetting
\usepackage{booktabs}       % professional-quality tables
\usepackage{amsfonts}       % blackboard math symbols
\usepackage{nicefrac}       % compact symbols for 1/2, etc.
\usepackage{microtype}      % microtypography
\usepackage{amsmath}
\usepackage{color}
\usepackage{algorithm}
\usepackage{algorithmic}
\usepackage{subcaption}
\usepackage{wrapfig}
\usepackage{multirow}
\usepackage{tikz}
\usepackage{amsthm}
\usepackage{amssymb}
\usetikzlibrary{positioning}
\usepackage{mathtools}


\newtheorem{theorem}{Theorem}[section]
\newtheorem{assumption}{Assumption}[section]
\newtheorem{proposition}{Proposition}[section]
\newtheorem{definition}{Definition}[section]
\newtheorem{corollary}{Corollary}[theorem]
\newtheorem{lemma}{Lemma}[theorem]
% \newtheorem{corollary}{Corollary}[proposition]
\newtheorem{definition1}{Definition}[section]

%% editing comment
\newcommand{\cmt}[1]{{\footnotesize\textcolor{red}{#1}}}
\newcommand{\cmto}[1]{{\footnotesize\textcolor{orange}{#1}}}
\newcommand{\note}[1]{\cmt{Note: #1}}
\newcommand{\todo}[1]{\cmt{TO-DO: #1}}
\newcommand{\question}[1]{\cmto{Question: #1}}
\newcommand{\sergey}[1]{{\footnotesize\textcolor{blue}{Sergey: #1}}}
\newcommand{\ak}[1]{{\textcolor{black}{#1}}}
\newcommand{\edits}[1]{\textcolor{blue}{#1}}
\newcommand{\editsred}[1]{\textcolor{red}{#1}}
\newcommand{\editsp}[1]{\textcolor{purple}{#1}}
\newcommand{\editsv}[1]{\textcolor{magenta}{#1}}


\newcommand{\methodname}{Cal-QL}
\newcommand{\aliasingproblemname}{bootstrapping aliasing}
\newcommand{\Aliasingproblemname}{Bootstrapping aliasing}
\newcommand{\AliasingProblemName}{Bootstrapping Aliasing}
\newcommand{\simnorm}{\mathrm{sim}_{\mathrm{n}}^\pi}
\newcommand{\simunnorm}{\mathrm{sim}_{\mathrm{u}}^\pi}

%% abbreviations
\newcommand{\x}{\mathbf{x}}
\newcommand{\z}{\mathbf{z}}
\newcommand{\y}{\mathbf{y}}
\newcommand{\w}{\mathbf{w}}
\newcommand{\data}{\mathcal{D}}

\newcommand{\etal}{{et~al.}\ }
\newcommand{\eg}{e.g.\ }
\newcommand{\ie}{i.e.\ }
\newcommand{\nth}{\text{th}}
\newcommand{\pr}{^\prime}
\newcommand{\tr}{^\mathrm{T}}
\newcommand{\inv}{^{-1}}
\newcommand{\pinv}{^{\dagger}}
\newcommand{\real}{\mathbb{R}}
\newcommand{\gauss}{\mathcal{N}}
\newcommand{\norm}[1]{\left|#1\right|}
\newcommand{\trace}{\text{tr}}

%% specifics for the paper
\newcommand{\reward}{r}
\newcommand{\policy}{\pi}
\newcommand{\mdp}{\mathcal{M}}
\newcommand{\states}{\mathcal{S}}
\newcommand{\actions}{\mathcal{A}}
\newcommand{\observations}{\mathcal{O}}
\newcommand{\transitions}{T}
\newcommand{\initstate}{d_0}
\newcommand{\freq}{d}
\newcommand{\obsfunc}{E}
\newcommand{\initial}{\mathcal{I}}
\newcommand{\horizon}{H}
\newcommand{\rewardevent}{R}
\newcommand{\probr}{p_\rewardevent}
\newcommand{\metareward}{\bar{\reward}}
\newcommand{\discount}{\gamma}
\newcommand{\behavior}{{\pi_\beta}}
\newcommand{\bellman}{\mathcal{B}}
\newcommand{\qparams}{\phi}
\newcommand{\qparamset}{\Phi}
\newcommand{\qset}{\mathcal{Q}}
\newcommand{\batch}{B}
\newcommand{\qfeat}{\mathbf{f}}
\newcommand{\Qfeat}{\mathbf{F}}
\newcommand{\hatbehavior}{\hat{\pi}_\beta}

\newcommand{\traj}{\tau}

\newcommand{\pihi}{\pi^{\text{hi}}}
\newcommand{\pilo}{\pi^{\text{lo}}}
\newcommand{\ah}{\mathbf{w}}

\newcommand{\proj}{\Pi}

\newcommand{\loss}{\mathcal{L}}
\newcommand{\eye}{\mathbf{I}}

\newcommand{\model}{\hat{p}}
\newcommand{\mhat}{\hat{\mathcal{M}}}
\newcommand{\mdphat}{\widehat{\mathcal{M}}}
\newcommand{\mdpbar}{\overline{\mathcal{M}}}

\newcommand{\pimix}{\pi_{\text{mix}}}

\newcommand{\pib}{\bar{\pi}}
\newcommand{\epspi}{\epsilon_{\pi}}
\newcommand{\epsmodel}{\epsilon_{m}}

\newcommand{\return}{\mathcal{R}}

%% math
\newcommand{\cY}{\mathcal{Y}}
\newcommand{\cX}{\mathcal{X}}
\newcommand{\en}{\mathcal{E}}
\newcommand{\bu}{\mathbf{u}}
\newcommand{\bv}{\mathbf{v}}
\newcommand{\be}{\mathbf{e}}
\newcommand{\by}{\mathbf{y}}
\newcommand{\bx}{\mathbf{x}}
\newcommand{\bz}{\mathbf{z}}
\newcommand{\bw}{\mathbf{w}}
\newcommand{\bo}{\mathbf{o}}
\newcommand{\bs}{\mathbf{s}}
\newcommand{\ba}{\mathbf{a}}
\newcommand{\bM}{\mathbf{M}}
\newcommand{\ot}{\bo_t}
\newcommand{\st}{\bs_t}
\newcommand{\at}{\ba_t}
\newcommand{\op}{\mathcal{O}}
\newcommand{\opt}{\op_t}
\newcommand{\kl}{D_\text{KL}}
\newcommand{\tv}{D_\text{TV}}
\newcommand{\ent}{\mathcal{H}}
\newcommand{\bG}{\mathbf{G}}
\newcommand{\byk}{\mathbf{y_k}}
\newcommand{\bI}{\mathbf{I}}
\newcommand{\bg}{\mathbf{g}}
\newcommand{\bV}{\mathbf{V}}
\newcommand{\bD}{\mathbf{D}}
\newcommand{\bR}{\mathbf{R}}
\newcommand{\bQ}{\mathbf{Q}}
\newcommand{\bA}{\mathbf{A}}
\newcommand{\bN}{\mathbf{N}}
\newcommand{\bS}{\mathbf{S}}
\newcommand{\bW}{\mathbf{W}}
\newcommand{\bU}{\mathbf{U}}
\newcommand{\bO}{\mathbf{O}}

\newcommand{\bzhi}{\bz^\text{hi}}

\newcommand{\expected}{\mathbb{E}}
\newcommand{\E}{\mathbb{E}}
\newcommand{\srank}{\text{srank}}
\newcommand{\rank}{\text{rank}}
\newcommand{\deepnet}{\bW_N(k, t) \bW_\phi(k, t)}
\newcommand{\features}{\bW_\phi(k, t)}
\newcommand{\stateactioni}{[\bs_i; \ba_i]}
\newcommand{\diag}{\text{\textbf{diag}}}

\def\thetaP{\theta^{\prime}}
\def\cf{\emph{c.f.}\ }
\def\vs{\emph{vs}.\ }
\def\etc{\emph{etc.}\ }
\def\Eqref#1{Equation~\ref{#1}}

\newenvironment{repeatedthm}[1]{\@begintheorem{#1}{\unskip}}{\@endtheorem}
\newcommand{\algname}{COMBO\xspace}

\newtheorem{theorem}{Theorem}[section]
\newtheorem{sketchtheorem}{Sketch Theorem}[section]
\newtheorem{lemma}[theorem]{Lemma}
\newtheorem{corollary}[theorem]{Corollary}
\newtheorem{proposition}[theorem]{Proposition}
\newtheorem{definition}[theorem]{Definition}
\newtheorem{conjecture}[theorem]{Conjecture}
\newtheorem{problem}[theorem]{Problem}
\newtheorem{formulation}[theorem]{Formulation}
\newtheorem{claim}[theorem]{Claim}
\newtheorem{remark}[theorem]{Remark}
\newtheorem{example}[theorem]{Example}
\newtheorem{assumption}[theorem]{Assumption}
\newtheorem{exercise}[theorem]{Exercise}


\newcommand{\indep}{\rotatebox[origin=c]{90}{$\models$}}

\renewcommand{\mathbf}{\boldsymbol}

\newcommand{\conv}{\circledast}
\newcommand{\mb}{\mathbf}
\newcommand{\mc}{\mathcal}
\newcommand{\mf}{\mathfrak}
\newcommand{\md}{\mathds}
\newcommand{\bb}{\mathbb}
\newcommand{\msf}{\mathsf}
\newcommand{\mcr}{\mathscr}
\newcommand{\magnitude}[1]{ \left| #1 \right| }
\newcommand{\set}[1]{\left\{ #1 \right\}}
\newcommand{\condset}[2]{ \left\{ #1 \;\middle|\; #2 \right\} }


\newcommand{\reals}{\bb R}
\newcommand{\proj}{\mathrm{proj}}

\newcommand{\eps}{\varepsilon}
\newcommand{\R}{\reals}
\newcommand{\Cp}{\bb C}
\newcommand{\Z}{\bb Z}
\newcommand{\N}{\bb N}
\newcommand{\Sp}{\bb S}
\newcommand{\Ba}{\bb B}
\newcommand{\indicator}[1]{\mathbbm 1\left\{#1\right\}}
\renewcommand{\P}{\mathbb{P}}
\newcommand{\rvline}{\hspace*{-\arraycolsep}\vline\hspace*{-\arraycolsep}}
\makeatletter
\def\Ddots{\mathinner{\mkern1mu\raise\p@
\vbox{\kern7\p@\hbox{.}}\mkern2mu
\raise4\p@\hbox{.}\mkern2mu\raise7\p@\hbox{.}\mkern1mu}}
\makeatother
% to declare new operator
% \DeclareMathOperator{\xxx}{xxx}

%% Other definitions

\newcommand{\event}{\mc E}

\newcommand{\e}{\mathrm{e}}
\newcommand{\im}{\mathrm{i}}
\newcommand{\rconcave}{r_\fgecap}
\newcommand{\Lconcave}{\mc L^\fgecap}
\newcommand{\rconvex}{r_\fgecup}
\newcommand{\Rconvex}{R_\fgecup}
\newcommand{\Lconvex}{\mc L^\fgecup}

\newcommand{\wh}{\widehat}
\newcommand{\wt}{\widetilde}
\newcommand{\ol}{\overline}


\newcommand{\betaconcave}{\beta_\fgecap}
\newcommand{\betagrad}{\beta_{\mathrm{grad}}}

\newcommand{\norm}[2]{\left\| #1 \right\|_{#2}}
\newcommand{\abs}[1]{\left| #1 \right|}
\newcommand{\row}[1]{\text{row}\left( #1 \right)}
\newcommand{\innerprod}[2]{\left\langle #1,  #2 \right\rangle}
\newcommand{\prob}[1]{\bb P\left[ #1 \right]}
\newcommand{\expect}[1]{\bb E\left[ #1 \right]}
\newcommand{\function}[2]{#1 \left(#2\right}
\newcommand{\integral}[4]{\int_{#1}^{#2}\; #3\; #4}
\newcommand{\paren}[1]{\left( #1 \right)}
\newcommand{\brac}[1]{\left[ #1 \right]}
\newcommand{\Brac}[1]{\left\{ #1 \right\}}

% Adding new defs here
\newcommand{\moff}{m_\mr{off}}
\newcommand{\mon}{m_\mr{on}}
\newcommand{\EmpiricalOffline}{\wh{\bb E}_{\mc D^\nu_h}}
\newcommand{\EmpiricalOnline}{\wh{\bb E}_{\mc D^\tau_h}}
\newcommand{\Deltaoff}{\Delta_\mr{off}}
\newcommand{\Deltaon}{\Delta_\mr{on}}
\newcommand{\Vmax}{V_{\max}}
\newcommand{\regret}{\mr{Reg}}
\newcommand{\regreton}{\mr{Sub}_{\mr {on}}}
\newcommand{\regretoff}{\mr{Sub}_{\mr {off}}}
\newcommand{\Doff}{\mc D_\mr{off}}
\newcommand{\Don}{\mc D_\mr{on}}
\newcommand{\piref}{\pi_\mr{ref}}

\newcommand{\nt}[1]{{\color{purple}{\bf [Next: #1]}}}
\newcommand{\here}{{\color{purple}{\bf [Writing here]}}}
% \newcommand{\todo}{{\color{purple}{\bf [TODO]}}}
\newcommand{\sz}[1]{{\color{blue}{\bf [Simon: #1]}}}
% \newcommand{\note}[1]{{\color{red}{\bf [note: #1]}}}
\newcommand{\mr}{\mathrm}
\newcommand{\sym}{\mathrm{Sym}}
\newcommand{\sks}{\mathrm{Skew}}
\newcommand{\inprod}[2]{\langle#1,#2\rangle}
\newcommand{\parans}[1]{\left(#1\right}
\newcommand{\clip}{\msf{clipped}}
\newcommand{\beha}{\msf b}
\numberwithin{equation}{section}

% \def \endprf{\hfill {\vrule height6pt width6pt depth0pt}\medskip}

\newcommand{\cdsmethodname}{CDS}
\newcommand{\udsmethodname}{UDS}
\newcommand{\ptrmethodname}{PTR~}
\newcommand{\primemethodname}{PRIME~}
\newcommand{\arxiv}[1] {{\color{black} #1}}

% \newenvironment{proof}{\noindent {\bf Proof} }{\endprf\par}

% \newcommand{\qed}{{\unskip\nobreak\hfil\penalty50\hskip2em\vadjust{}
%            \nobreak\hfil$\Box$\parfillskip=0pt\finalhyphendemerits=0\par}}


\newcommand\myworries[1]{\textcolor{red}{#1}}
\newcommand\running[1]{\textcolor{blue}{#1}}

\newcommand{\benchl}[1]{{\scriptsize\textsf{#1}}\normalsize\xspace}
\newcommand{\bench}[1]{{\fontsize{8.5}{10}\selectfont\textsf{#1}}\normalsize\xspace}
\newcommand{\code}[1]{{\fontsize{8.5}{1}\selectfont{\tt #1}}\xspace}
\newcommand{\codebold}[1]{{\fontsize{8.5}{1}\selectfont{\tt \textbf{#1}}}\xspace}
\newcommand{\xx}[1]{\textcolor{black}{#1}}
\newcommand{\xxs}[1]{\textcolor{black}{\scriptsize\textsf{#1}}}
\newcommand{\supertiny}[1]{\fontsize{5}{4}\selectfont{#1}}
\newcommand{\xxred}[1]{\textcolor{red}{\textsf{#1}}}
% \DeclareMathOperator{\EX}{\mathbb{\hat{E}}}% expected value
% \makeatletter
% \newcommand\footnoteref[1]{\protected@xdef\@thefnmark{\ref{#1}}\@footnotemark}
% \makeatother
% \usepackage{scrextend}

% \deffootnote[1em]{1em}{1em}{\textsuperscript{\thefootnotemark}\,}
% \newcolumntype{P}[1]{>{\centering\arraybackslash}p{#1}}
% \newcommand\mycommfont[1]{\footnotesize\ttfamily\textcolor{blue}{#1}}

\newcommand\aviral[1]{\textcolor{red}{aviralkumar@: #1}}
\newcommand\ayazdan[1]{\textcolor{red}{ayazdan@: #1}}
\newcommand\sv[1]{\textcolor{red}{SV@: #1}}
\newcommand\fix[1]{\textcolor{green}{#1}}

\newcommand{\round}[1]{\ensuremath{\lfloor#1\rceil}}

\newcommand{\tgray}[1]{\colorbox{lightgray}{\textbf{#1}}}
\newcommand{\finalcheck}[1]{\textcolor{green}{#1}}

\newcommand{\review}[1]{#1}

\newcommand{\niparagraph}[1]{\vspace{2pt}\noindent\textbf{#1}}

\let\svthefootnote\thefootnote

\setlength{\abovedisplayskip}{1pt}
\setlength{\belowdisplayskip}{1pt}

\title{Conservative Q-Learning\\ for Offline Reinforcement Learning}



% The \author macro works with any number of authors. There are two commands
% used to separate the names and addresses of multiple authors: \And and \AND.
%
% Using \And between authors leaves it to LaTeX to determine where to break the
% lines. Using \AND forces a line break at that point. So, if LaTeX puts 3 of 4
% authors names on the first line, and the last on the second line, try using
% \AND instead of \And before the third author name.

\author{%
  Aviral Kumar$^1$, Aurick Zhou$^1$, George Tucker$^2$, Sergey Levine$^{1,2}$ \\
  $^1$UC Berkeley, $^2$Google Research, Brain Team\\
  \texttt{aviralk@berkeley.edu}
  % Coauthor \\
  % Affiliation \\
  % Address \\
  % \texttt{email} \\
  % \AND
  % Coauthor \\
  % Affiliation \\
  % Address \\
  % \texttt{email} \\
  % \And
  % Coauthor \\
  % Affiliation \\
  % Address \\
  % \texttt{email} \\
  % \And
  % Coauthor \\
  % Affiliation \\
  % Address \\
  % \texttt{email} \\
}

\begin{document}


\part*{Representative Publications}

% \part*{\Large{Significant Publication 1: \\ Conservative Q-Learning for Offline Reinforcement Learning}}
In this document, two of my representative publications are attached one after the other.

\section*{Publication 1:} 
Conservative Q-Learning for Offline Reinforcement Learning, NeurIPS 2020.

\section*{Publication 2:}
DR3: Value-Based Deep Reinforcement Learning Requires Explicit Regularization, ICLR 2022.

% \section*{Author List} Aviral Kumar, Aurick Zhou, George Tucker, Sergey Levine

% \section*{Significance of the Paper} 

% {\textbf{Summary.}} This paper presents a method directly address offline reinforcement learning by learning a \emph{pessimistic} value function that estimates a lower bound on the actual utility of the learned policy (instead of constraining the policy). The key insight is to add a regularizer inspired by min-max formulations for adversarial training to the training loss in an off-the-shelf RL algorithm. 

% This paper has \textbf{500+ citations} till date (paper released in June 2020).  

% {\textbf{Theoretical impact.}} My theoretical analysis of CQL provided a framework for studying pessimistic algorithms under the lens of safe policy improvement, and this framework was used to make substantial progress on many aspects of offline reinforcement learning including the development of state-of-the-art model-based reinforcement learning algorithms and data sharing in multi-task reinforcement learning. Other works have also shown that CQL attains strong performance guarantees compared to other offline RL approaches. 

% {\textbf{Real-world application impact.}} CQL has served as a fundamental building block of many applications, I list some of them below:
% \begin{itemize}
%     \item In my follow-up work, I show that CQL is an effective way to leverage diverse robot datasets to boost generalization in {robot learning}, along with several works from others. 
%     \item At LinkedIn, CQL policies trained on historical user-interaction data are deployed in {mobile notification systems} to \textbf{hundreds of millions of users} and lead to higher notification quality.
%     \item CQL is also deployed in {digital marketing systems} (DMS), where it reduces costs and improves over contextual bandit and other RL methods.
%     \item CQL is also used for optimizing treatment regimes for chemotherapy.
% \end{itemize}
% References for each of these applications can be found in my research statement.






\newpage

\maketitle


\begin{abstract}
Effectively leveraging large, previously collected datasets in reinforcement learning (RL) is a key challenge for large-scale real-world applications. Offline RL algorithms promise to learn effective policies from previously-collected, static datasets without further interaction. However, in practice, offline RL presents a major challenge, and standard off-policy RL methods can fail due to overestimation of values induced by the distributional shift between the dataset and the learned policy, especially when training on complex and multi-modal data distributions. In this paper, we propose \emph{conservative Q-learning (CQL)}, which aims to address these limitations by learning a conservative Q-function such that the expected value of a policy under this Q-function lower-bounds its true value. 
% This conservatism reduces the effect of errors introduced by distributional shift. 
% CQL lower-bounds the value, thus preventing erroneously high values at unobserved actions, which can give rise to poor policy performance.     
We theoretically show that CQL produces a lower bound on the value of the current policy and that it can be incorporated into a policy learning procedure with theoretical improvement guarantees. In practice, CQL augments the standard Bellman error objective with a simple Q-value regularizer which is straightforward to implement on top of existing deep Q-learning and actor-critic implementations. On both discrete and continuous control domains, we show that CQL substantially outperforms existing offline RL methods, often learning policies that attain 2-5 times higher final return, especially when learning from complex and multi-modal data distributions\footnote{Our follow-up on CQL for vision-based robotic manipulation can be found here: \url{https://arxiv.org/abs/2010.14500}.}.
%%SL: rewrote some of the material below a bit
%based on Q-function estimation, that alleviates these limitations. Our algorithm simply augments a traditional Q-learning algorithm with an additional penalty the Q-function during training. We theoretically show that this method obtains provably lower-bounded value estimates, and as a result, avoids a number of problems that arise in offline settings. We empirically demonstrate its superior performance over existing methods, especially in harder settings with complex data distributions. We also show how our approach can be used for conservative off-policy policy evaluation, and empirically demonstrate its efficacy in providing meaningful value estimates on a number of complex domains.      
\end{abstract}

\vspace{-0.3cm}

\vspace{-0.25cm}
\section{Introduction}
\vspace{-0.25cm}
% motivation: generic RL, and offline RL
Recent advances in reinforcement learning (RL), especially when combined with expressive deep network function approximators, have produced promising results in domains ranging from robotics~\citep{kalashnikov2018qtopt} to strategy games~\citep{alphastar} and recommendation systems~\citep{li2010contextual}. However, applying RL to real-world problems consistently poses practical challenges: in contrast to the kinds of data-driven methods that have been successful in supervised learning~\citep{resnet,bert}, RL is classically regarded as an active learning process, where each training run requires active interaction with the environment. Interaction with the real world can be costly and dangerous, and the quantities of data that can be gathered online are substantially lower than the offline datasets that are used in supervised learning~\citep{imagenet}, which only need to be collected once. Offline RL, also known as batch RL, offers an appealing alternative~\citep{ernst2005tree,fujimoto2018off,kumar2019stabilizing,agarwal2019optimistic,jaques2019way,siegel2020keep,levine2020offline}. Offline RL algorithms learn from large, previously collected datasets, without interaction. This in principle can make it possible to leverage large datasets, but in practice fully offline RL methods pose major technical difficulties, stemming from the distributional shift between the policy that collected the data and the learned policy. This has made current results fall short of the full promise of such methods.

Directly utilizing existing value-based off-policy RL algorithms in an offline setting generally results in poor performance, due to issues with bootstrapping from out-of-distribution actions~\citep{kumar2019stabilizing,fujimoto2018off} and overfitting~\citep{fu2019diagnosing,kumar2019stabilizing,agarwal2019optimistic}. This typically manifests as erroneously optimistic value function estimates.
% motivating the technical approach
If we can instead learn a \emph{conservative} estimate of the value function, which provides a lower bound on the true values, this overestimation problem could be addressed. In fact, because policy evaluation and improvement typically only use the value of the policy, we can learn a less conservative lower bound Q-function, such that only the expected value of Q-function under the policy is lower-bounded, as opposed to a point-wise lower bound.
We propose a novel method for learning such conservative Q-functions via a simple modification to standard value-based RL algorithms. The key idea behind our method is to minimize values under an appropriately chosen distribution over state-action tuples, and then further tighten this bound by also incorporating a \emph{maximization} term over the data distribution.
%%SL.5.2: See the language I used in the abstract to explain this lower bound stuff. I think we could describe the nuance (that we lower bound the expected return of a policy, vs lower bounding the whole Q-function) quite concisely here using similar language.
%% added that

% last para, contributions
Our primary contribution is an algorithmic framework, which we call conservative Q-learning (CQL), for learning conservative, lower-bound estimates of the value function, by regularizing the Q-values during training.
%%SL.5.2: I don't think it's accurate to characterize the method as "simply penalizing" -- but maybe see the language I used in the abstract for inspiration?
Our theoretical analysis of CQL shows that \textit{only} the expected value of this Q-function under the policy lower-bounds the true policy value, preventing extra under-estimation that can arise with point-wise lower-bounded Q-functions, that have typically been explored in the opposite context in exploration literature~\citep{osband2017posterior,jaksch2010near}.
%%SL.5.17: The above sentence is a bit problematic, because I think most readers won't understand its significance. Perhaps we can instead say something like: We present a simplified version of our method that can learn point-wise lower bounds, and then extend this approach to provide a tighter bound that only lower-bounds the expected value under the policy, which also provides improved empirical performance.
%%AK.5.22: I would not prefer calling that a simplified version of our method since that seems like a detail. ALso I wanted to refer more generally to exploration methods that learn point-wise lower bounds, rather than just referring to CQL without the data term. I edited it a bit, but if it does not seem fine to you I can also remove this line.
We also empirically demonstrate the robustness of our approach to Q-function estimation error.
%%SL.5.17: I think we could probably remove this entire sentence.
Our practical algorithm uses these conservative estimates for policy evaluation and offline RL. CQL can be implemented with less than \textbf{20} lines of code on top of a number of standard, online RL algorithms~\citep{haarnoja,dabney2018distributional}, simply by adding the CQL regularization terms to the Q-function update. In our experiments, we demonstrate the efficacy of CQL for offline RL, in domains with complex dataset compositions, where prior methods are typically known to perform poorly~\citep{d4rl} and domains with high-dimensional visual inputs~\citep{bellemare2013arcade,agarwal2019optimistic}. CQL outperforms prior methods by as much as \textbf{2-5x} on many benchmark tasks, and is the only method that can outperform simple behavioral cloning on a number of realistic datasets collected from human interaction.

\section{Preliminaries}
\label{sec:background}
\vspace{-8pt}
The goal in reinforcement learning is to learn a policy that maximizes the expected cumulative discounted reward in a Markov decision process (MDP), which is defined by a tuple $(\mathcal{S}, \mathcal{A}, \transitions, r, \gamma)$.
$\mathcal{S}, \mathcal{A}$ represent state and action spaces, $\transitions(\bs' | \bs, \mathbf{a})$ and $r(\bs,\mathbf{a})$ represent the dynamics and reward function, and $\gamma \in (0,1)$ represents the discount factor. $\behavior(\mathbf{a}|\bs)$ represents the behavior policy, $\mathcal{D} = \{(\bs, \mathbf{a}, r, \bs')\}$ is the dataset of tuples from trajectories collected under a behavior policy $\behavior$, and $d^\behavior(\bs)$ is the discounted marginal state-distribution of $\behavior(\mathbf{a}|\bs)$. The dataset $\mathcal{D}$ is sampled from $d^\behavior(\bs) \behavior(\mathbf{a}|\bs)$. {On all states $\bs \in \mathcal{D}$, let $\hatbehavior(\mathbf{a}|\bs) := \frac{\sum_{\bs,\mathbf{a} \in \mathcal{D}} \mathbf{1} [\bs = \bs , \mathbf{a} = \mathbf{a}]}{\sum_{\bs \in \mathcal{D}} \mathbf{1}[\bs = \bs]}$ denote the empirical behavior policy, at that state.} We assume that the rewards $r$ satisfy: $|r(\bs, \mathbf{a})| \leq R_{\max}$.

% off-policy RL
Off-policy RL algorithms based on dynamic programming maintain a parametric Q-function $Q_\theta(s, a)$ and, optionally, a parametric policy, $\pi_\phi(a|s)$. 
Q-learning methods train the Q-function by iteratively applying the Bellman optimality operator $\mathcal{B}^*Q(\bs, \mathbf{a}) = r(\bs, \mathbf{a}) + \gamma \E_{\bs' \sim P(\bs'|\bs, \mathbf{a})}[\max_{\mathbf{a}'} Q(\bs', \mathbf{a}')]$, and use exact or an approximate maximization scheme, such as CEM~\citep{kalashnikov2018qtopt} to recover the greedy policy. In an actor-critic algorithm, a separate policy is trained to maximize the Q-value.
Actor-critic methods alternate between computing $Q^\policy$ via (partial) policy evaluation,
by iterating the Bellman operator, $\mathcal{B}^\pi Q= r + \gamma P^\pi Q$, where $P^\pi$ is the transition matrix coupled with the policy: $P^\pi Q(\bs, \mathbf{a}) = \E_{\bs' \sim \transitions(\bs' | \bs, \mathbf{a}), \mathbf{a}' \sim \pi(\mathbf{a}'|\bs')} \left[ Q(\bs', \mathbf{a}') \right],$
and improving the policy
$\policy(\mathbf{a}|\bs)$ by updating it towards actions that maximize the expected Q-value. Since $\mathcal{D}$ typically does not contain all possible transitions $(\bs, \mathbf{a}, \bs')$, the policy evaluation step actually uses an empirical Bellman operator that only backs up a single sample. We denote this operator $\hat{\bellman}^\policy$. Given the dataset $\mathcal{D} = \{(\bs, \mathbf{a}, r, \bs')\}$ of tuples from trajectories collected under a behavior policy $\behavior$:
\begin{small}

\begin{align*}
    \hat{Q}^{k+1} \leftarrow& \arg\min_{Q} \E_{\bs, \mathbf{a}, \bs' \sim \mathcal{D}}\left[ \left((r(\bs, \mathbf{a}) + \gamma \E_{\mathbf{a}' \sim \hat{\policy}^k(\mathbf{a}'|\bs')}[\hat{Q}^{k}(\bs', \mathbf{a}')]) - Q(\bs, \mathbf{a})\right)^2 \right]~\text{(policy evaluation)}\\ 
    \hat{\policy}^{k+1} \leftarrow& \arg\max_{\policy} \E_{\bs \sim \mathcal{D}, \mathbf{a} \sim \policy^k(\mathbf{a}|\bs)}\left[\hat{Q}^{k+1}(\bs, \mathbf{a})\right]~~~ \text{(policy improvement)} 
\end{align*}
\end{small}
% \ak{If $\mathcal{D}$ does not cover all possible transitions $(\bs, \ba, \bs')$ in the MDP, then, the policy evaluation step as described above effectively uses an ``empirical'' Bellman operator, denoted by $\hat{\bellman}^\policy$, that uses the reward $r$ and the next state $\bs'$ samples observed.} 
%%SL.5.30: replace with: 
%%SL.5.30: However, this is somewhat awkward here -- the equation is not expressed in terms of a Bellman operator, and therefore putting this comment here is quite confusing. Maybe it's better to instead put this right after the sentence in the previous paragraph that introduces the Bellman operator?
% Offline RL algorithms typically modify the policy improvement step with an additional divergence regularizer. We refer the reader to \citet{levine2020offline} for details.
%%SL.5.2: I think it would be helpful in the preliminaries section to briefly review why using these methods for offline RL is actually hard (i.e., why we get OOD actions). Thsi will serve to motivate the discussion that comes next.
% done, next.
% OOD actions and offline RL preliminaries
Offline RL algorithms based on this basic recipe suffer from action distribution shift~\citep{kumar2019stabilizing,wu2019behavior,jaques2019way,levine2020offline} during training, because the target values for Bellman backups in policy evaluation use actions sampled from the learned policy, $\policy^k$, but the Q-function is trained only on actions sampled from the behavior policy that produced the dataset $\mathcal{D}$, $\behavior$. Since $\policy$ is trained to maximize Q-values, it may be biased towards out-of-distribution (OOD) actions with erroneously high Q-values. In standard RL, such errors can be corrected by attempting an action in the environment and observing its actual value. However, the inability to interact with the environment makes it challenging to deal with Q-values for OOD actions in offline RL. Typical offline RL methods~\citep{kumar2019stabilizing,jaques2019way,wu2019behavior,siegel2020keep} mitigate this problem by constraining the learned policy~\citep{levine2020offline} away from OOD actions. {Note that Q-function training in offline RL does not suffer from state distribution shift, as the Bellman backup never queries the Q-function on out-of-distribution states. However, the policy may suffer from state distribution shift at test time.}

\vspace{-7pt}
\section{\methodname: Explicitly Regularizing Representations in Deep Q-Learning}
\label{sec:method}
\vspace{-7pt}
In Section~\ref{sec:problem}, we observed that implicit regularization of gradient descent and neural networks can induce co-adaption of features at state-action tuples appearing on the two sides of a Bellman backup. If the implicit regularizer that arises from GD does not lead to effective learning in the TD setting, as it does in the supervised setting, can we instead introduce \emph{explicit} regularization to alleviate this issue? In this section, we derive an explicit regularizer  directly from Theorem~\ref{thm:implicit_noise_reg}, that alleviates these issues and can be utilized in conjunction with any offline RL method.

\begin{remark}
Let $R_{\mathrm{SL}}(\theta): = \sum_i ||\nabla_\theta Q_\theta(\bx_i)||_2^2$ denote the implicit regularizer for supervised learning. Then, $\Delta(\theta) := R_\mathrm{TD}(\theta) - R_\mathrm{SL}(\theta) = - \gamma \sum_i \nabla_\theta Q_\theta(\bx_i)^\top \nabla_\theta Q_\theta(\bx'_i)$ is the offset between implicit regularizers induced by noisy updates in supervised and TD learning. 
\label{remark:remark_diff}
\end{remark}
Since our goal is to devise an explicit regularizer that mitigates issues with implicit regularization in TD-learning, we choose to add an \emph{explicit} regularizer equal to $\Delta(\theta)$ from Remark~\ref{remark:remark_diff}. This explicit regularizer minimizes $\Delta(\theta) = \sum_i \nabla_\theta Q_\theta(\bs_i, \ba_i)^\top \nabla_\theta Q_\theta(\bs'_i, \ba'_i) $, and thus the total of implicit and explicit regularization closely mimics the one in supervised learning ($R_\mathrm{SL}(\theta)$) which is known to be beneficial. We can combine $\Delta(\theta)$ with various RL algorithms. Since this regularizer by itself does not capture overestimation due to distributional shift, but only prevents feature co-adaptation, to obtain an effective offline RL method we must combine $\Delta(\theta)$ with another method that mitigates the effects of out-of-distribution action selection.  
For a generic offline RL algorithm, \textsc{Alg}, with objective $\mathcal{L}_{\textsc{Alg}}(\theta)$, the training objective with \methodname\ is given by: $\mathcal{L}(\theta) := \mathcal{L}_{\textsc{Alg}}(\theta) + \alpha \Delta(\theta)$.

\textbf{Practical implementation details.} Because per-example gradient dot products are costly to compute, we find that it suffices to approximate  $\mathcal{R}_\mathrm{exp}(\theta)$ with the contribution from the last layer parameters (\ie $\sum_i \nabla_\bw Q_\theta(\bs_i, \ba_i)^\top \nabla_\bw Q_\theta(\bs'_i, \ba'_i)$), thus, the practical version of our explicit regularizer is: $\bar{\mathcal{R}}_\mathrm{exp}(\theta) = \sum_{i \in \mathcal{D}} \phi(\bs_i, \ba_i)^\top \phi(\bs'_i, \ba'_i)$. In discrete action environments, where Q-learning algorithms utilize multi-head networks to parameterize Q-functions (one head per action), we apply \methodname\ on the state features $\phi(\bs)$. On continuous action domains, we apply \methodname\ to state-action features. The weighting factor, $\alpha$, is a constant across different tasks and depends on the base offline RL method. More details can be found in Appendix~\ref{app:method_details}.  

%%AK: thinking of removing the stuff below -- the consequences of why the penalty works is pretty clear given the section 3 analysis and while I can show other properties like adaptive discounts, noise compounding prevention, etc, I dont think this is worth the space.
\iffalse
\textbf{Theoretical analysis of \methodname.} Now we shall theoretically analyze \methodname\ with the goal of answering the following questions: \textbf{(1)} How does \methodname\ regularize the learning dynamics of deep Q-learning thereby preventing aliasing?, \textbf{(2)} Does \methodname\ prevent the Q-function from diverging away from a good solution with more training, thereby inducing stability? and \textbf{((3)} Does \methodname\ robustify the learning dynamics against noise arising from the limited size of data in sampled settings, thereby leading to better solutions? 
%%AK: the notion of stability is a bit overloaded, so we need to make sure when we are talking about which stability
We provide theoretical results answering these questions next.

To answer \textbf{(1)}, we utilize the abstract model from Section~\ref{sec:theory_evidence} and show in Theorem~\ref{thm:penalty_removes_aliasing} that utilizing \methodname\ with the ``optimal'' weight $\alpha_{\text{opt}}$ learns features $\Phi$ such that the optimal features are close to a projection of the optimal Q-function $\bQ^*$ on the function class $\Phi$.

\textcolor{red}{Add theorem here: maybe closeness to $Q^*$ is not the best approach, but more about something else?}

In our next result, we answer \textbf{(2)} by showing that, whenever Bellman backups are performed with features $\Phi$ for which the value of the \methodname\ regularizer,
%%SL.5.13: really complex clause, reorder this sentence to split it into shorter simpler clauses
$\mathcal{R}(\phi) = \E_{\bs, \ba, \bs' \sim \data, \ba' \sim \pi}[\phi(\bs, \ba)^\top \phi(\bs', \ba')]$, is bounded above by a certain constant $C^*$, then  Bellman backups originating in the vicinity of a good Q-function that does not alias states will converge to this fixed point. Note the direct contrast against Theorem ?? from Section~\ref{sec:bootstrapping_evidence},
%%AK: this theorem shows that Q-learning is unstable near optima because it can minimize the implicit regularizer more than the TD error.
which shows that naive TD learning will still pursue aliased solutions in the abstract model and will not converge to a good solution even starting close to it.

\textcolor{red}{Add theorem showing stable convergence to a good fixed point close to it. The previous theorem shows that regularization finds the right fixed point, and the second one shows that it converges to it.}

While the analysis so far has primarily focused on the benefits of \methodname\ in curbing the excessive implicit regularization effects that arise when optimizing deep networks with Bellman backups, our next result shows that in addition, \methodname\ reduces the degree to which error due to sampling and distributional shift compound over time as more backups are performed. By making distributional assumptions on the sampling error, we show that in certain cases, utilizing \methodname\ is absolutely essential to learn a meaningful policy that is better than random behavior.

\textcolor{red}{This theorem is similar to the stuff below, we will just assume that $\hat{P}^\pi- P^\pi$ is distributed according to a (truncated) normal distribution and show that the \methodname\ regularizer directly controls the amount of noise in the Q-function.}

\subsection{Effect of regularizer}
With the regularizer, our TD loss is
\[ \frac{1}{2}\left(\phi_\theta(s, a)w - r(s, a) - \gamma\phi_{\bar{\theta}}(s', a')\right)^2 + \alpha \phi_\theta(s, a)^\top \phi_{\bar{\theta}}(s', a'),\]
so the gradient wrt to $\theta$ is
\begin{align*}
  &\left(\phi_\theta(s, a)^\top w - r(s, a) - \gamma\phi_{\bar{\theta}}(s', a')^\top \bar{w} \right)w^\top\frac{\partial \phi_\theta(s, a)}{\partial \theta} + \alpha \phi_{\bar{\theta}}(s', a')^\top\frac{\partial \phi_\theta(s, a)}{\partial \theta}\\
  &=\left(\phi_\theta(s, a)^\top ww^\top - r(s, a)w^\top - \gamma\phi_{\bar{\theta}}(s', a')^\top \bar{w}w^\top + \alpha\phi_{\bar{\theta}}(s', a')^\top \right)\frac{\partial \phi_\theta(s, a)}{\partial \theta} \\ 
  &=\left(\phi_\theta(s, a)^\top ww^\top - r(s, a)w^\top - \phi_{\bar{\theta}}(s', a')^\top \left(\gamma\bar{w}w^\top - \alpha I \right)\right)\frac{\partial \phi_\theta(s, a)}{\partial \theta} \\ 
\end{align*}
\fi

%%%%%%%%%%%%%%%%%%%%%%%%%%%%%%%%%%%%%%%%%%%
%%%%%%%%%%%%%%%             OLD STUFF
%%%%%%%%%%%%%%%%%%%%%%%%%%%%%%%%%%%%%%%%%%%

\iffalse
\section{\methodname: Mitigating \AliasingProblemName\ via Orthogonality Regularization}
\label{sec:method}

%%AK.1.31: I don't really like the title of this section, and no we are also committed to an acronym that "orthogonality regularization" in it because of the abstract...

If bootstrapping induces features that alias state-action tuples drawn from the offline dataset and the learned policy, which can have pathological consequences (Section~\ref{sec:consequences_of_feature_sim}), can we develop a better offline RL method by explicitly reducing this aliasing? While encouraging aliasing on two vectors is relatively straightforward, here we are interested in the inverse problem of reducing aliasing, which can be tackled in potentially many ways, not all of which are ideal for offline Q-learning. This makes it less straightforward to address this problem.
%%AK.1.31: just stressing why removing aliasing is hard, while introducing aliasing is easy... else it might appear as if our fix is straightforward -- we took a diagnostic metric and penalized that, which will seem like a hack.

%%AK.1.31: I am not a fan of the word "fortunately" here, but I couldn't find a better word to kickstart this para..
Fortunately, as we will also theoretically show in Section~\ref{sec:analysis},
%%SL.2.1: Maybe we're getting ahead of ourselves -- how about we state the point here first, some intuition, and only then say: In Section ??, we will make this intuition more precise with a theoretical proof that [something].
it turns out that simply adding a regularizer that encourages the Q-function features to have a small unnormalized similarity
%%SL.2.1: the presence of s' here is I think a little confusing...
is theoretically sufficient to address this issue. This means that features on actions drawn from the learned policy, $\ba'$, are forced to be dissimilar to $\ba$, on directional similarity, provided the norm of the feature vectors is large enough.
%%SL.2.1: not sure if "on a combination of direction and norm" entirely makes sense -- at first I thought you meant "on both the direction and norm," but that's actually not right...
%%AK.2.3: removed that phrase and expanded it. is it better now?
Based on this observation, our method, \methodname\,
penalizes the expected value of $\simunnorm(\bs, \ba, \bs') = |\phi(\bs, \ba)^T \E_{\pi}[\phi(\bs', \cdot)]|$ in a training batch, in addition to TD-error. 
\methodname\ can be easily combined with existing offline RL methods, such as CQL~\citep{kumar2020conservative}, and REM~\citep{agarwal2019optimistic}, and offline policy evaluation algorithms such as FQE~\citep{le2019batch}.
%%AK.1.31: I have cited BRAC at a bunch of places, perhaps we can do some experiments with it, or we can just remove this...

While \methodname\ does not by itself curb overestimation in Q-values due to distributional shift, it instead aims to prevent feature aliasing with an additional loss that penalizes the absolute value of dot-products of features $\phi(\bs, \ba)$ and $\phi(\bs', \ba')$. Thus, for an effective offline method, we combine \methodname\ with existing methods for mitigating distributional shift. %We will discuss how this can be theoretically derived in Section~\ref{sec:analysis}.
When combined with an offline RL algorithm, \textsc{Alg}, with objective $\mathcal{L}_{\textsc{Alg}}(\theta)$, the training objective for the Q-function, with the \methodname\ penalty marked in red, is: 
\begin{align}
    \min_\theta~ \mathcal{L}(\theta) := &\  \mathcal{L}_{\textsc{Alg}}(\theta) + \textcolor{red}{\alpha \E_{\bs, \ba, \bs' \sim \data}[\simunnorm(\bs, \ba, \bs')]}.
    %\big|\phi_\theta(\bs, \ba)^T \E_{\pi}[\phi_\theta(\bs', \cdot)]\big|.}
\label{eqn:our_method}
\end{align}
We next discuss some practical implementation details. After this, we will theoretically show in Section~\ref{sec:analysis} that optimizing the policy $\pi$ against the Q-function obtained via Equation~\ref{eqn:our_method} amounts to maximizing a lower-bound on the policy return, and the lower-bound is tighter if the value of the \methodname\ regularizer is small.

%%SL.2.3: I do think that it would help to include the full equation for the CQL version of FOQL in the implementation paragraph, just to make it really clear how it works (if we have space of course)
\textbf{Practical Implementation of \methodname.} We discuss two variants of our method: a Q-learning variant that only parameterizes the Q-function for discrete-action tasks, and an actor-critic variant that trains a separately parameterized policy for continuous action tasks. The actor-critic variant of \methodname\ trains the Q-function using Equation~\ref{eqn:our_method} and optimizes the policy against it.
The Q-learning variant of \methodname, which we use in discrete action spaces, parameterizes a multi-head Q-network~(one head per action) that learns features $\phi(\bs)$ dependent only on the state $\bs$ and not the action $\ba$. In this case, \methodname\ is applied on the features $\phi(\bs)$ and $\phi(\bs')$ directly.
%%AK.1.26: do we need to have an algorithm box?
%%SL.1.26: Yes, I think an algorithm box would help. I think there was also a little bit of slaight of hand -- we were talking about how our method can be combined with any RL method, but now it looks like it's specific to CQL?
%%AK.1.31: TODO

%Implementation details
For our experiments, we build on top of prior offline RL methods: CQL~\citep{kumar2020conservative} and REM~\citep{agarwal2019optimistic}. On all the tasks, we use a default set of hyperparameters for the base method, and additionally set $\alpha_{\mathrm{\methodname}}$ to a constant that is kept identical across different task groups. The values of $\alpha_{\mathrm{\methodname}}$ can be found in Appendix ??.
%%AK.1.31: Clarify the bit about hparams since they are not all the same for the base method

%%SL.2.1: Generally, I think this section reads well. The second paragraph in the section (Fortunately...) is a little bit rough. Content-wise it's fine, but maybe go over it a couple more times to smooth out the sentence structure. The main clarity thing that I recommend addressing in this section is to clarify the point about s/a/s'/a' -- right now, the regularizer is written as a function of s/a/s', but you motivated this loss by talking about similarity between policy and data actions -- maybe it would be good to explain this leap. For example, you could define similarity in terms of pi(a|s) vs (s,a)\in D, but then explain how in practice, we already sample from pi from the target value, so it makes sense to contrast those (or explain some other reason why this is a good idea). Otherwise it comes across as a little arbitrary that the similarity of actions from data and pi is a function of s/a/s'
%%AK.2.3: this is TODO

\fi
\subsection{Conservative Q-Learning for Offline Policy Optimization}
\label{sec:framework}
% Define the optimiziation problem
We now present a general approach for offline policy learning, which we refer to as conservative Q-learning (CQL). 
As discussed in Section~\ref{sec:policy_eval}, we can obtain Q-values that lower-bound the value of a policy $\policy$
%%SL.5.17: but we don't use the "every state" version?
% it is the same version actually, but I removed the term "every state" from the above sentence
by solving Equation~\ref{eqn:modified_policy_eval} with $\mu = \policy$. How should we utilize this for policy optimization? We could alternate between performing full off-policy evaluation for each policy iterate, $\hat{\policy}^k$, and one step of policy improvement. However, this can be computationally expensive. Alternatively, since the policy $\hat{\policy}^k$ is typically derived from the Q-function, we could instead choose $\mu(\mathbf{a}|\bs)$ to approximate the policy that would maximize the current Q-function iterate, 
% \ak{alternating partial policy evaluation and improvement},
%%SL.5.30: I don't think the bit in red is necessary -- what is that supposed to fix?
thus giving rise to an online algorithm.
%%SL.5.17: I think the above intuition is not entirely clear. Perhaps we can phrase it more along these lines: we could instead choose $\mu(\mathbf{a}|\bs)$ to approximate the policy that would maximize the current Q-values, giving rise to an online algorithm.
% done

% Stating the final optimization framework
We can formally capture such online algorithms by defining a family of optimization problems over $\mu(\mathbf{a}|\bs)$, presented below, with modifications from Equation~\ref{eqn:modified_policy_eval} marked in red. An instance of this family is denoted by CQL($\mathcal{R}$) and is characterized by a particular choice of regularizer $\mathcal{R}(\mu)$:
\begin{multline}
    \label{eqn:cql_framework}
    \small{\min_{Q} \textcolor{red}{\max_{\mu}}~~ \alpha \left(\E_{\bs \sim \mathcal{D}, \mathbf{a} \sim \textcolor{red}{\mu(\mathbf{a}|\bs)}}\left[Q(\bs, \mathbf{a})\right] - \E_{\bs \sim \mathcal{D}, \mathbf{a} \sim \hatbehavior(\mathbf{a}|\bs)}\left[Q(\bs, \mathbf{a})\right] \right)}\\
    \small{+ \frac{1}{2}~ \E_{\bs, \mathbf{a}, \bs' \sim \mathcal{D}}\left[\left(Q(\bs, \mathbf{a}) - \hat{\bellman}^{\policy_k} \hat{Q}^{k} (\bs, \mathbf{a}) \right)^2 \right] + \textcolor{red}{\mathcal{R}(\mu)} ~~~ \left(\text{CQL}(\mathcal{R})\right).}
\end{multline}

% provide examples, to build intuition, and mention the one we use in practice
\textbf{Variants of CQL.} To demonstrate the generality of the CQL family of objectives, we discuss two specific instances within this family that are of special interest, and we evaluate them empirically in Section~\ref{sec:experiments}. 
%%AK.5.29: commenitng out since we don't evaluate this in practice
% First, if we choose $\mathcal{R}(\mu) = 0$, we obtain \mbox{$\mu(\mathbf{a}|\bs) = \delta(\mathbf{a} = \arg\max_{a} {Q}(\bs, \mathbf{a}))$}, which is the ``greedy" distribution (analogous to a greedy policy) corresponding to $Q$. 
If we choose $\mathcal{R}(\mu)$ to be the KL-divergence against a prior distribution, $\rho(\mathbf{a}|\bs)$, i.e., $\mathcal{R}(\mu) = -D_{\mathrm{KL}}(\mu, \rho)$, then we get $\mu(\mathbf{a}|\bs) \propto \rho(\mathbf{a}|\bs) \cdot \exp (Q(\bs, \mathbf{a}))$ (for a derivation, see Appendix~\ref{app:cql_variants}). Frist, if $\rho = \text{Unif}(\mathbf{a})$,
then the first term in Equation~\ref{eqn:cql_framework} corresponds to a soft-maximum of the Q-values at any state $\bs$ and gives rise to the following variant of Equation~\ref{eqn:cql_framework}, called CQL($\mathcal{H}$):
\begin{equation}
    \small{\min_{Q}~ \alpha \E_{\bs \sim \mathcal{D}}\left[\log \sum_{\mathbf{a}} \exp(Q(\bs, \mathbf{a}))\!-\!\E_{\mathbf{a} \sim \hatbehavior(\mathbf{a}|\bs)}\left[Q(\bs, \mathbf{a})\right]\right]\!+\!\frac{1}{2} \E_{\bs, \mathbf{a}, \bs' \sim \mathcal{D}}\left[\left(Q - \hat{\bellman}^{\policy_k} \hat{Q}^{k} \right)^2 \right]\!.}
    \label{eqn:practical_objective}
\end{equation}
Second, if $\rho(\mathbf{a}|\bs)$ is chosen to be the previous policy $\hat{\policy}^{k-1}$, the first term in Equation~\ref{eqn:practical_objective} is replaced by an exponential weighted average of Q-values of actions from the chosen $\hat{\policy}^{k-1}(\mathbf{a}|\bs)$. Empirically, we find that this variant can be more stable with high-dimensional action spaces (e.g., Table~\ref{table:adroit_antmaze}) where it is challenging to estimate $\log \sum_{\mathbf{a}} \exp$ via sampling due to high variance. In Appendix~\ref{app:cql_variants}, we discuss an additional variant of CQL, drawing connections to distributionally robust optimization~\citep{namkoong2017variance}.  
We will discuss a practical instantiation of a CQL deep RL algorithm in Section~\ref{sec:practical_alg}. CQL can be instantiated as either a Q-learning algorithm (with $\bellman^*$ instead of $\bellman^{\policy}$ in Equations~\ref{eqn:cql_framework}, \ref{eqn:practical_objective}) or as an actor-critic algorithm. 

\textbf{Theoretical analysis of CQL.}
Next, we will theoretically analyze CQL to show that the policy updates derived in this way are indeed ``conservative'', in the sense that each successive policy iterate is optimized against a lower bound on its value. For clarity, we state the results in the absence of finite-sample error, in this section, but sampling error can be incorporated in the same way as Theorems~\ref{thm:min_q_underestimates} and \ref{thm:cql_underestimates}, and we discuss this in Appendix~\ref{app:missing_proofs}.   Theorem~\ref{thm:cql_underestimation} shows that CQL($\mathcal{H}$) learns Q-value estimates that lower-bound the actual Q-function under the action-distribution defined by the policy, $\policy^{k}$, under mild regularity conditions (slow updates on the policy).
%% TODO: mentioning we don't have sampling error here, but we can account for that in the appendix.
%%% TODO: add in the appendix
%%AK.5.29: perhaps we can remove this completely?
\begin{theorem}[CQL learns lower-bounded Q-values]
\label{thm:cql_underestimation}
Let $\policy_{\hat{Q}^k}(\mathbf{a} | \bs) \propto \exp(\hat{Q}^k(\bs, \mathbf{a}))$ and assume that $D_{\mathrm{TV}}(\hat{\policy}^{k+1}, \pi_{\hat{Q}^k}) \leq \varepsilon$ (i.e., $\hat{\policy}^{k+1}$ changes slowly w.r.t to $\hat{Q}^k$). Then, the policy value under $\hat{Q}^k$, lower-bounds the actual policy value, \mbox{$\hat{V}^{k+1}(\bs) \leq V^{k+1}(\bs)~ \forall \bs \in \mathcal{D}$} if
\begin{equation*}
\small{\E_{\pi_{\hat{Q}^k}(\mathbf{a} | \bs)} \left[ \frac{\pi_{\hat{Q}^k}(\mathbf{a} | \bs)}{\hatbehavior(\mathbf{a}|\bs)} -1 \right] \geq \max_{\mathbf{a} \text{~s.t.~} \hatbehavior(\mathbf{a}|\bs) > 0} \left(\frac{\pi_{\hat{Q}^k}(\mathbf{a} | \bs)}{\hatbehavior(\mathbf{a}|\bs)} \right) \cdot \varepsilon}. 
\end{equation*}
\end{theorem}
The LHS of this inequality is equal to the amount of conservatism induced in the value, $\hat{V}^{k+1}$ in iteration $k+1$ of the CQL update,
%% TODO: conservatism caused in the particular $k$ of the backup...
if the learned policy were equal to soft-optimal policy for $\hat{Q}^k$, i.e., when $\hat{\policy}^{k+1} = \pi_{\hat{Q}^k}$. However, as the actual policy, $\hat{\policy}^{k+1}$, may be different, the RHS is the maximal amount of potential overestimation due to this difference. To get a lower bound, we require the amount of underestimation to be higher, which is obtained if $\varepsilon$ is small, i.e. the policy changes slowly.  
%%SL.5.27: This is pretty opaque. Is there any way to provide a bit more intuition for what this inequality actually means? Or some way to define this inequality in terms of some variables that we'll define, so that it's simpler? Right now I really can't tell whether it's reasonable to surmise that this inequality actually holds or not.
% done, although it is a bit verbose.

%%SL.5.30: In general, I'm getting concerned that the theory section is getting extremely long, to the point where it swallows the rest of the paper. This paragraph is hard to understand, and it's not entirely clear why this part is important. Maybe in the balance, it would take too much time to explain it fully, and perhaps including it here is not really effective. Maybe better to informally summarize the intuition, and refer to an appendix, instead freeing up more space for empirical results.
% changed, added more intuition.
% This was also when this para was too long and had some stuff that we didnt quite justify...
Our final result shows that CQL Q-function update is ``gap-expanding'', by which we mean that the difference in Q-values at in-distribution actions and over-optimistically erroneous out-of-distribution actions is higher than the corresponding difference under the actual Q-function. This implies that the policy $\policy^k(\mathbf{a}|\bs) \propto \exp(\hat{Q}^k(\bs, \mathbf{a}))$, is constrained to be closer to the dataset distribution, $\hatbehavior(\mathbf{a}|\bs)$, thus the CQL update implicitly prevents the detrimental effects of OOD action and distribution shift, which has been a major concern in offline RL settings~\citep{kumar2019stabilizing,levine2020offline,fujimoto2018off}.
%%SL.5.30: It's not at all clear why this is a good thing -- why should we care how big the gap is if OOD actions always have lower Q-values?
% This is the reason why Q-functions learned via CQL caOOD actions and action distribution shift, which has been a major problem  with offline RL methods~\citep{kumar2019stabilizing,levine2020offline}.
%%SL.5.30: I don't think most readers will understand the above sentence. How does having a bigger gap between in-distribution and OOD actions help combat these detrimental effects? And which particular detrimental effects do you have in mind here?
\begin{theorem}[CQL is gap-expanding] 
\label{thm:gap_amplify}
At any iteration $k$, CQL expands the difference in expected Q-values under the behavior policy $\behavior(\mathbf{a}|\bs)$ and $\mu_k$, such that for large enough values of $\alpha_k$, we have that \mbox{$\forall \bs \in \mathcal{D}, ~\E_{\behavior(\mathbf{a}|\bs)}[\hat{Q}^k(\bs, \mathbf{a})] - \E_{\mu_k(\mathbf{a}|\bs)}[\hat{Q}^k(\bs, \mathbf{a})] > \E_{\behavior(\mathbf{a}|\bs)}[{Q}^k(\bs, \mathbf{a})] - \E_{\mu_k(\mathbf{a}|\bs)}[{Q}^k(\bs, \mathbf{a})]$}.
\end{theorem}
% We restate this theorem formally, along with an expression for $\alpha_k$, in Appendix~\ref{app:missing_proofs}. Theorem~\ref{thm:gap_amplify} shows that the suboptimality of the in-distribution actions ($\mathbf{a} \sim \behavior$) under the learned Q-values is reduced relative to actions from $\mu_k$.
%%SL.5.30: Reduced relative to what?
When function approximation or sampling error makes OOD actions have higher learned Q-values, CQL backups are expected to be more robust, in that the policy is updated using Q-values that prefer in-distribution actions. 
%%SL.5.17: Above sentence is very hard to parse because it has too many clauses, split it into two sentences.
As we will empirically show in Appendix~\ref{app:gap_amplify}, prior offline RL methods that do not explicitly constrain or regularize the Q-function may not enjoy such robustness properties.

\textbf{To summarize}, we showed that the CQL algorithm learns lower-bound Q-values with large enough $\alpha$, meaning that the final policy attains \emph{at least} the estimated value. We also showed that the Q-function is \emph{gap-expanding}, meaning that it should only ever \emph{over-estimate} the gap between in-distribution and out-of-distribution actions, preventing OOD actions.

\subsection{Safe Policy Improvement Guarantees}
In Section~\ref{sec:policy_eval} we proposed novel objectives for Q-function training such that the expected value of a policy under the resulting Q-function lower bounds the actual performance of the policy. In Section~\ref{sec:framework}, we used the learned conservative Q-function for policy improvement. In this section, we show that this procedure actually optimizes a well-defined objective and provide a safe policy improvement result for CQL, along the lines of Theorems 1 and 2 in \citet{laroche2017safe}.

To begin with, we define \emph{empirical return} of any policy $\policy$, ${J}(\policy, \hat{M})$, which is equal to the discounted return of a policy $\policy$ in the \emph{empirical} MDP, $\hat{M}$, that is induced by the transitions observed in the dataset $\mathcal{D}$, i.e. $\hat{M} = \{s, a, r, s' \in \mathcal{D}\}$. $J(\policy, M)$ refers to the expected discounted return attained by a policy $\policy$ in the actual underlying MDP, $M$. In Theorem~\ref{thm:well_defined_obj}, we first show that CQL (Equation~\ref{eqn:modified_policy_eval}) optimizes a well-defined penalized RL empirical objective. All proofs are found in Appendix~\ref{app:safe_pi}. 

\begin{theorem}
\label{thm:well_defined_obj}
Let $\hat{Q}^\pi$ be the fixed point of Equation~\ref{eqn:modified_policy_eval}, then $\policy^*(\mathbf{a}|\bs) := \arg\max_{\policy} \mathbb{E}_{\bs \sim \rho(\bs)}[\hat{V}^\pi(\bs)]$ is equivalently represented as:
% \begin{equation}
% \label{eqn:policy_optimality_main}
\mbox{$\small{\policy^*(\mathbf{a}|\bs) \leftarrow \arg \max_{\policy}~~ J(\pi, \hat{M}) - \alpha \frac{1}{1 - \gamma} \mathbb{E}_{\bs \sim d^\policy_{\hat{M}}(\bs)}\left[D_{\text{CQL}}(\policy, \hatbehavior)(\bs) \right]}$},
% \end{equation}
where $D_{\text{CQL}}(\policy, \behavior)(\bs) := \sum_{\mathbf{a}} \policy(\mathbf{a}|\bs) \cdot \left(\frac{\policy(\mathbf{a}|\bs)}{\behavior(\mathbf{a}|\bs)} - 1 \right)$.
\end{theorem}
Intuitively, Theorem~\ref{thm:well_defined_obj} says that CQL optimizes the return of a policy in the empirical MDP, $\hat{M}$, while also ensuring that the learned policy $\policy$ is not too different from the behavior policy, $\hatbehavior$ via a penalty that depends on $D_{\text{CQL}}$. Note that this penalty is implicitly introduced by virtue by the gap-expanding (Theorem~\ref{thm:gap_amplify}) behavior of CQL. Next, building upon Theorem~\ref{thm:well_defined_obj} and the analysis of CPO~\citep{achiam2017constrained}, we show that CQL provides a $\zeta$-safe policy improvement over $\hatbehavior$.   

\begin{theorem}
\label{thm:zeta_safe}
Let $\policy^*(\mathbf{a}|\bs)$ be the policy obtained in Theorem~\ref{thm:well_defined_obj}. Then, the policy $\policy^*(\mathbf{a}|\bs)$ is a $\zeta$-safe policy improvement over $\hatbehavior$ in the actual MDP $M$, i.e., $J(\policy^*, M) \geq J(\hatbehavior, M) - \zeta$ with high probability $1 - \delta$, where $\zeta$ is given by,
% \sqrt{\log \left(\frac{1}{\delta}\right)}
\begin{multline*}
    \label{eqn:performance_relation}
    \small{\zeta = \mathcal{O}\left({\frac{\gamma}{(1 - \gamma)^2}}  \right) \mathbb{E}_{\bs \sim d^{\policy^*}_{\hat{M}}(\bs)}\left[ \frac{\sqrt{|\mathcal{A}|}}{\sqrt{|\mathcal{D}(\bs)|}} \sqrt{ D_{\text{CQL}}(\policy^*, \hatbehavior)(\bs) + 1} \right]} - \small{\underbrace{\left( J(\policy^*, \hat{M}) - J(\hatbehavior, \hat{M})  \right)}_{\geq \frac{\alpha }{1 - \gamma} \mathbb{E}_{\bs \sim d^{\policy^*}_{\hat{M}}(\bs)}\left[D_{\text{CQL}}(\policy^*, \hatbehavior)(\bs) \right]}}
\end{multline*}
\end{theorem}
\vspace{-5pt}
The expression of $\zeta$ in Theorem~\ref{thm:zeta_safe} consists of two terms: the first term captures the decrease in policy performance in $M$, that occurs due to the mismatch between $\hat{M}$ and $M$, also referred to as \emph{sampling error}. The second term captures the increase in policy performance due to CQL in empirical MDP, $\hat{M}$. The policy $\policy^*$ obtained by optimizing $\policy$ against the CQL Q-function improves upon the behavior policy, $\hatbehavior$ for suitably chosen values of $\alpha$. When sampling error is small, i.e., $|\mathcal{D}(\bs)|$ is large, then smaller values of $\alpha$ are enough to provide an improvement over the behavior policy. 

\textbf{To summarize,} CQL optimizes a well-defined, penalized empirical RL objective, and performs high-confidence safe policy improvement over the behavior policy. The extent of improvement is negatively influenced by higher sampling error, which decays as more samples are observed.  

\vspace{-7pt}
\section{Practical Algorithm and Implementation Details}
\label{sec:practical_alg}
\vspace{-7pt}
\begin{wrapfigure}{r}{0.6\textwidth}
\begin{small}
\vspace{-24pt}
\begin{minipage}[t]{0.99\linewidth}
\begin{algorithm}[H]
\small
\caption{Conservative Q-Learning (both variants)}
\label{alg:practical_alg}
\begin{algorithmic}[1]
    \STATE Initialize Q-function, $Q_\theta$, and optionally a policy, $\pi_\phi$.
    \FOR{step $t$ in \{1, \dots, N\}}
        \STATE Train the Q-function using $G_Q$ gradient steps on objective from Equation~\ref{eqn:practical_objective} \\
        \mbox{$\theta_t := \theta_{t-1} - \eta_Q \nabla_\theta \textcolor{red}{\text{CQL}(\mathcal{R})(\theta)}$}\\
        (Use $\bellman^*$ for Q-learning, $\bellman^{\policy_{\phi_t}}$ for actor-critic)
        \STATE \underline{(only with actor-critic)} Improve policy $\pi_\phi$ via $G_\pi$ gradient steps on $\phi$ with SAC-style entropy regularization:\\
        \mbox{$\phi_{t} := \phi_{t-1} + \eta_\pi \mathbb{E}_{\bs \sim \mathcal{D}, \mathbf{a} \sim \pi_\phi(\cdot|\bs)}[Q_\theta(\bs, \mathbf{a})\! -\! \log \pi_\phi(\mathbf{a}|\bs)] $}
    \ENDFOR
\end{algorithmic}
\end{algorithm}
\end{minipage}
\vspace{-14pt}
\end{small}
\end{wrapfigure}
We now describe two practical offline deep reinforcement learning methods based on CQL: an actor-critic variant and a Q-learning variant. Pseudocode is shown in Algorithm~\ref{alg:practical_alg}, with differences from conventional actor-critic algorithms (e.g., SAC~\citep{haarnoja}) and deep Q-learning algorithms (e.g., DQN~\citep{mnih2013playing}) in red.
Our algorithm uses the CQL($\mathcal{H}$) (or CQL($\mathcal{R}$) in general) objective from the CQL framework for training the Q-function $Q_\theta$, which is parameterized by a neural network with parameters $\theta$. For the actor-critic algorithm, a policy $\pi_\phi$ is trained as well. Our algorithm modifies the objective for the Q-function (swaps out Bellman error with CQL($\mathcal{H}$)) or CQL($\rho$)
in a standard actor-critic or Q-learning setting, as shown in Line 3. As discussed in Section~\ref{sec:framework}, due to the explicit penalty on the Q-function, CQL methods do not use a policy constraint,
unlike prior offline RL methods~\citep{kumar2019stabilizing,wu2019behavior,siegel2020keep,levine2020offline}.
Hence, we do not require fitting an additional behavior policy estimator, simplifying our method. %This prevents the detrimental effects of using an inaccurately fitted behavior policy in an offline RL algorithm~\citep{levine2020offline}.

\textbf{Implementation details.} Our algorithm requires an addition of only \textbf{20} lines of code on top of standard implementations of soft actor-critic (SAC)~\citep{haarnoja} for continuous control experiments and on top of QR-DQN~\citep{dabney2018distributional} for the discrete control. The tradeoff factor, $\alpha$
is is fixed at constant values described in Appendix~\ref{sec:experimental_details} for gym tasks and discrete control and is automatically tuned via Lagrangian dual gradient descent for other domains. We use default hyperparameters from SAC, except that the learning rate for the policy was chosen from \{3e-5, 1e-4, 3e-4\}, and is less than or equal to the Q-function, as dictated by Theorem~\ref{thm:cql_underestimation}. 
Elaborate details are provided in Appendix~\ref{sec:experimental_details}.  

\section{Related Work}
\label{sec:related}
\vspace{-10pt}
In this work, we study off-policy reinforcement learning with static datasets. Errors arising from inadequate sampling, distributional shift, and function approximation have been rigorously studied as ``error propagation'' in approximate dynamic programming (ADP)~\citep{bertsekas1996ndp,munos2003errorapi,farahmand2010error,bruno2015approximate}. These works often study how Bellman errors accumulate and propagate to nearby states via bootstrapping. In this work, we build upon tools from this analysis to show that performing Bellman backups on static datasets leads to error accumulation due to out-of-distribution values. Our approach is motivated as reducing the rate of propagation of error propagation between states.

% One way to reduce off-policy errors in practice is using importance sampling~\cite{precup2001offpol,sutton2016etd,hallak2017coptd,gelada2019off,munos2016safe} and minimize an importance sampled objective. In this work, we tackle the problem of inaccurate target values in Q-learning and the developments in importance sampling for off-policy evaluation are orthogonal and complementary to our method.
%%SL.5.20: I think we can heavily trim this paragraph down to basically one sentence with a row of citations if we don't end up making a big deal out of importance sampling. This will also help to mitigate our current problem with the new stuff beginning very very late in the paper.

Our approach constrains actor updates so that the actions remain in the support of the training dataset distribution. Several works have explored similar ideas in the context of off-policy learning learning in online settings. \citet{kakade2002cpi} shows that large policy updates can be destructive, and propose a conservative policy iteration scheme which constrains actor updates to be small for provably convergent learning. \citet{grau-moya2018soft} use a learned prior over actions in the maximum entropy RL framework~\citep{levine2018rlasinference} and justify it as a regularizer based on mutual information. However, none of these methods use static datasets. Importance Sampling based distribution re-weighting~\cite{munos2016safe,gelada2019off,precup2001offpol,mahmood2015emphatic} has also been explored primarily in the context of off-policy policy evaluation.

Most closely related to our work is batch-constrained Q-learning (BCQ)~\citep{fujimoto2018off} and SPIBB~\citep{laroche2019spibb},
which also discuss instability arising from previously unseen actions. \citet{fujimoto2018off} show convergence properties of an action-constrained Bellman backup operator in tabular, error-free settings. We prove stronger results under approximation errors and provide a bound on the \emph{suboptimality} of the solution. This is crucial as it drives the design choices for a practical algorithm. 
As a consequence, although we experimentally find that \citep{fujimoto2018off} outperforms standard Q-learning methods when the off-policy data is collected by an expert, BEAR outperforms \cite{fujimoto2018off} when the off-policy data is collected by a suboptimal policy, as is common in real-life applications. Empirically, we find BEAR  achieves stronger and more consistent results than BCQ across a wide variety of datasets and environments. As we explain below, the BCQ constraint is too aggressive;  BCQ generally fails to substantially improve over the behavior policy, while our method actually improves when the data collection policy is suboptimal or random. SPIBB~\citep{laroche2019spibb}, like BEAR, is an algorithm based on constraining the learned policy to the support of a behavior policy. However, the authors do not extend safe performance guarantees from the batch-constrained case to the relaxed support-constrained case, and do not evaluate on high-dimensional control tasks. REM~\citep{agarwal19striving} is a concurrent work that uses a random convex combination of an ensemble of Q-networks to perform offline reinforcement learning from a static dataset consisting of interaction data generated while training a DQN agent.

% % now lets talk about the section that compares prior methods and CQL with function approximation in terms of action gap
%%SL.5.11: I wonder if it could make sense to have a single "Related Work and Connections to Policy Constraints" section, where we could have a subsection on "Related Prior Work" subsection on "Prior Work on Offline RL with Constraints" and "Comparative Analysis of CQL and Policy Constraint Methods" or something? Not certain about this, but maybe we try it and see how it looks? It's a bit unconventional...

% setup the stage, what we want to do in this section and why
In this section, we discuss in detail the consequences of the gap-expanding behavior of CQL backups over prior methods based on policy constraints that, as we show in this section, may not exhibit such gap-expanding behavior in practice. To recap, Theorem~\ref{thm:gap_amplify} shows that the CQL backup operator increases the difference between expected Q-value at in-distribution ($\ba \sim \behavior(\ba|\bs)$) and out-of-distribution ($\ba \text{~s.t.~} \frac{\mu_k(\ba|\bs)}{\behavior(\ba|\bs)} << 1$) actions. We refer to this property as the gap-expanding property of the CQL update operator.

% This gap-expanding behavior plays a central role when learning in the presence of function approximation:  

% In this section, we perform an analysis of CQL with deep neural networks, and compare it to prior offline RL methods based on policy constraints. As noted in Section~\ref{sec:related}, some variants of CQL can be viewed as applying a policy constraint on the greedy policy induced by the Q-function. We therefore aim to analyze the effect of direct regularization on the Q-function as opposed to only constraining the policy, with primary focus on settings where function approximation is employed.

% discuss why policy constraint methods may fail
\textbf{Function approximation may give rise to erroneous Q-values at OOD actions.} We start by discussing the behavior of prior methods based on policy constraints~\citep{kumar2019stabilizing,fujimoto2018off,jaques2019way,wu2019behavior} in the presence of function approximation.
To recap, because computing the target value requires $\E_\policy[\hat{Q}(\bs,\ba)]$, constraining $\policy$ to be close to $\behavior$ will avoid evaluating $\hat{Q}$ on OOD actions. These methods typically do not impose any further form of regularization on the learned Q-function.
Even with policy constraints, because function approximation used to represent the Q-function, learned Q-values at two distinct state-action pairs are coupled together. As prior work has argued and shown~\citep{achiam2019towards,fu2019diagnosing,kumar2020discor}, the ``generalization'' or the coupling effects of the function approximator may be heavily influenced by the properties of the data distribution~\citep{fu2019diagnosing,kumar2020discor}. For instance, \citet{fu2019diagnosing} empirically shows that when the dataset distribution is narrow (i.e. state-action marginal entropy, $\mathcal{H}(d^\behavior(\bs, \ba))$, is low~\citep{fu2019diagnosing}), the coupling effects of the Q-function approximator can give rise to incorrect Q-values at different states, though this behavior is absent without function approximation, and is not as severe with high-entropy (e.g. Uniform) state-action marginal distributions.
%%SL.5.27: The above sentence is really speculative -- I don't think it's at all clear why narrow datasets couple different state action pairs. It's also very hard to understand, because it's long, with multiple clauses, and nested parentheses. I would really recommend just deleting the whole sentence. But if you don't want to delete it, try to rewrite to more clearly explain the point, with shorter sentences, avoiding complex clauses, and avoiding parens whenever possible.
% done, cited prior work diagnosing bottlenecks and DisCor

In offline RL, we will shortly present empirical evidence on high-dimensional MuJoCo tasks showing that certain dataset distributions, $\mathcal{D}$, may cause the learned Q-value at an OOD action $\ba$ at a state $\bs$, to in fact take on high values than Q-values at in-distribution actions at intermediate iterations of learning. This problem persists even when a large number of samples (e.g. $1M$) are provided for training, and the agent cannot correct these errors due to no active data collection.  

%%AK.5.29: Toned down to one para, with mostly pointing to the analysis, not creating any hypotheses for why something might be going wrong.
Since actor-critic methods, including those with policy constraints, use the learned Q-function to train the policy, in an iterative online policy evaluation and policy improvement cycle, as discussed in Section~\ref{sec:background}, the errneous Q-function may push the policy towards OOD actions, especially when no policy constraints are used. Of course, policy constraints should prevent the policy from choosing OOD actions, however, as we will show that in certain cases, policy constraint methods might also fail to prevent the effects on the policy due to incorrectly high Q-values at OOD actions. 

% how do these Q-values affect the policy?
% \textbf{How can erroneous Q-values affect the quality of the resulting policy?} When these erroneous Q-values -- with higher relative values
% at out-of-distribution actions -- are used to then update the policy, the policy is pushed towards OOD actions, since a policy improvement update, shown below, trains the policy to maximize Q-values:
% \begin{equation}
%     \label{eqn:policy_constraint_repeated}
%     \policy^{k+1}~~ \leftarrow \arg \max_{\textcolor{red}{\policy}} \E_{\bs \sim d^\behavior(\bs)}\left[ \E_{\ba \sim \textcolor{red}{\policy}}[\hat{Q}^{k+1}(\bs, \ba)] - \nu_k \underbrace{D(\textcolor{red}{\policy}(\ba|\bs), \behavior(\ba|\bs))}_{\text{policy constraint}} \right]
% \end{equation}
% % Of course, when policy constraints are used, the policy is prevented from choosing OOD actions. 
% % However, a gradient signal obtained from such an erroneous Q-function will still push the policy towards OOD actions, since the Q-values at these OOD actions are relatively higher than in-distribution actions. We discuss empirical evidence justifying this in Appendix~\ref{app:empirical_evidence_gap_expanding}. 
% %%SL.5.27: A reviewer might say this should not be an issue, since you won't get OOD actions due to the constraint
% % We have empirical evidence for this, that atleast in some cases this could be an issue, but the example, and the experiments, I feel should justify this in practice
% However, the low fidelity of the Q-function combined with the inability to correct errors in the function this case, may just push the policy towards OOD actions, directly conflicting with the policy-constraint. While this might not be potentially harmful when the Q-function provides enough improvement signal to improve the policy when the policy constraint is satisfied, but this property may not be guaranteed. 

% As a result, the improvement signal obtained from the Q-function might directly conflicts with the policy constraint, whose role is to prevent the policy from choosing OOD actions. We would instead desire that the Q-function provides enough improvement signal to the policy within the space of in-distribution actions, so that the policy can improve within the set of observed actions. But since Q-values may be higher at OOD actions, the Q-function gradient may push the policy towards OOD actions as a result, and hence directly conflict with the policy constraint (which aims at keeping the policy within in a close neighbourhood of the behavior cloned policy).  
% When gradient based optimization is used to train the policy in this setting, this could amount to conflicting gradient (as we show via empirical evidence), and hence, it is likely that the overall update shown in Equation~\ref{eqn:policy_constraint_repeated} does not improve the policy meaningfully. 
%%SL.5.27: this seems very vague and imprecise, and it's not clear how CQL does this
% restated below

\textbf{How can CQL address this problem?} As we show in Theorem~\ref{thm:gap_amplify}, the difference between expected Q-values at in-distribution actions and out-of-distribution actions is expanded by the CQL update. This property is a direct consequence of the specific nature of the CQL regularizer -- that maximizes Q-values under the dataset distribution, and minimizes them otherwise. This difference depends upon the choice of $\alpha_k$, which can directly be controlled, since it is a free parameter. Thus, by effectively controlling $\alpha_k$, CQL can push down the learned Q-value at out-of-distribution actions as much is desired, correcting for the erroneous overestimation error in the process.  


%%SL.5.27: Overall, I really don't find the discussion above to be very convincing. It seems very speculative, and it's very easy to argue with and disagree with. Are you sure we can't somehow delete the stuff above, and instead present this appendix as a summary of the evidence that we'll be discussing below? I also think we should really consider simply deleting this appendix. It was a nice idea, but the quality of the evidence here is really not up to the standard of rigor, and perhaps it's better to omit it than to include things that are too debatable or too hand-wavy. Basically, I have a hard time imagining how this appendix will make anyone happy -- anyone who is skeptical about our method will become even more skeptical, whereas anyone who likely our method will probably not read this appendix (because it's not about our method).

% why is this problem more relevant in offline settings?
% We also remark that while the problem induced due to Q-function approximation also afflicts standard online Q-function training (and this is a motivation behind gap-increasing operators~\citep{bellemare2016increasing}), we would expect this problem to more severely affect performance in offline RL. Q-function errors in standard online RL can be corrected in most cases~\citep{kumar2020discor,levine2020offline}, since the agent can collect new transitions that correspond to highly erroneous Q-values and then train on them. However, the algorithm cannot perform any online data collection and moreover, has no control over the dataset, $\mathcal{D}$ either in offline RL settings, making the impact of this problem severe. We next present a simple didactic example to demonstrate this problem.

% example
% \subsection{Didactic Example} 
% \label{app:didactic_example}
% In order to build intuition for the discussion presented above, we consider a didactic three-state, two-action MDP shown in Figure~\ref{fig:didactic}. Action $\ba_1$ at state $\bs_0$ deterministically transits to state $\bs_2$, and action $\ba_2$ at state $\bs_0$ transits to state $\bs_1$. Both actions at state $\bs_1$ and $\bs_2$ induce a self-loop at $\bs_1$ and $\bs_2$ respectively. The MDP provides the following reward values: $r(\bs_0, \ba_1) = 10, r(\bs_1, \ba_2) = -5$ and all other rewards, $r(\bs_1, \cdot) = 0, r(\bs_2, \cdot) = 0$. Assume that the offline dataset at state $\bs_0$ is distributed according to the following density function: $\mathcal{D}(\ba_1|\bs_0) = 0.8, \mathcal{D}(\ba_2|\bs_0) = 0.2$, and also assume that the size of the dataset $\mathcal{D}$ is so large, that the dataset empirical densities match the actual distribution of the behavior policy, i.e., $\mathcal{D}(\bs, \ba) = d^\behavior(\bs, \ba)$. The Q-function is modeled as a linear 
% \begin{wrapfigure}{r}{0.35\textwidth}
% \vspace{-10pt}
% \begin{center}
% \begin{tikzpicture}[auto,node distance=8mm,>=latex,font=\small]
%     \tikzstyle{round}=[thick,draw=black,circle]

%     \node[round] (s0) {$\bs_0$};
%     \node[round,above right=0mm and 20mm of s0] (s1) {$\bs_1$};
%     \node[round,below right=0mm and 20mm of s0] (s2) {$\bs_2$};

%     \draw[->] (s0) -- (s1) node[midway,sloped,above] {$\ba_2, -5$};
%     \draw[->] (s0) -- (s2) node[midway,sloped,below] {$\ba_1, +10$};
% \end{tikzpicture}
% \end{center}
% \caption{\small{Didactic three-state, two-action example demonstrating how the generalization effects of the Q-function approximator can hurt policy learning in offline RL, even with a (support-based) policy constraint.}}
% \vspace{-25pt}
% \label{fig:didactic}
% \end{wrapfigure}
% function on a scalar-valued given feature $\phi(\bs, \ba)$, such that $\hat{Q}(\bs, \ba) = w \cdot \phi(\bs, \ba) + b$. Assume $\phi(\bs_0, \ba_1) = 1$ and $\phi(\bs_0, \ba_2) = 1 + \varepsilon$, for some $\varepsilon > 0$, $\phi(\bs_1, \ba_1) = \phi(\bs_1, \ba_2) = 1 + \delta$, and $\phi(\bs_2, \ba_1) = \phi(\bs_2, \ba_2) = 0$. 

% %%AK: read this para and make more concrete
% When a policy-constraint that constrains the policy to the support of the behavior policy is used to update the Q-function, the updated Q-function parameters satisfy: $w_1 > 0$ and $b_1 > 0$. This is because an action with a high reward is used to minimize the Bellman error, and this pushes the Q-function to output positive values. Observe that the Q-function corresponding to these new-parameters also satisfies, $\hat{Q}^{1}(\bs_0, \ba_2) > \hat{Q}^{1}(\bs_0, \ba_1)$, which erroneously makes action $\ba_2$ have a higher Q-value. Since both actions are in the support of the behavior policy at state $\bs_0$, the incorrect Q-function updates the policy towards selecting action $\ba_2$ at state $\bs_0$, which gives rise to a negative reward. On the other hand, if the overestimation in the value $\hat{Q}(\bs_0, \ba_2)$ is controlled, as is the case with CQL (since, $\ba_2$ has a lower density under the behavior policy, and CQL would expand the gap, $\hat{Q}(\bs_0, \ba_1) - \hat{Q})(\bs_0, \ba_2)$, then we obtain $w^1 < 0$ and $b^1 >0$, thus preventing this issue. 

\textbf{Empirical evidence on high-dimensional benchmarks with neural networks.}  
We next empirically demonstrate the existence of of such Q-function estimation error on high-dimensional MuJoCo domains when deep neural network function approximators are used with stochastic optimization techniques. In order to measure this error, we plot the difference in expected Q-value under actions sampled from the behavior distribution, $\ba \sim \behavior(\ba|\bs)$, and the maximum Q-value over actions sampled from a uniformly random policy, $\ba \sim \text{Unif}(\ba|\bs)$. That is, we plot the quantity
\begin{equation}
\label{eqn:delta_eqn}
    \hat{\Delta}^k = \E_{\bs, \ba \sim \mathcal{D}}\left[\max_{\ba'_1, \cdots, \ba'_N \sim \text{Unif}(\ba')}[\hat{Q}^k(\bs, \ba')]- \hat{Q}^k(\bs, \ba)\right]
\end{equation}
over the iterations of training, indexed by $k$. This quantity, intuitively, represents an estimate of the ``advantage'' of an action $\ba$, under the Q-function, with respect to the optimal action $\max_{\ba'} \hat{Q}^k(\bs, \ba')$. Since, we cannot perform exact maximization over the learned Q-function in a continuous action space to compute $\Delta$, we estimate it via sampling described in Equation~\ref{eqn:delta_eqn}.

We present these plots in Figure~\ref{fig:delta_plots} on two datasets: hopper-expert and hopper-medium. The expert dataset is generated from a near-deterministic, expert policy, exhibits a narrow coverage of the state-action space, and limited to only a few directed trajectories. On this dataset, we find that $\hat{\Delta}^k$ is always positive for the policy constraint method (Figure~\ref{fig:delta_plots}(a)) and increases during training -- note, the continuous rise in $\hat{\Delta}^k$ values, in the case of the policy-constraint method, shown in Figure~\ref{fig:delta_plots}(a). This means that even if the dataset is generated from an expert policy, and policy constraints correct target values for OOD actions,
%%SL.5.27: Maybe it's because this sentence has too many clauses, but I don't actually understand what "near-optimal policy and policy constraints are used" means
% edited
incorrect Q-function generalization may make an out-of-distribution action appear promising. For the more stochastic hopper-medium dataset, that consists of a more diverse set of trajectories, shown in Figure~\ref{fig:delta_plots}(b), we still observe that $\hat{\Delta}^k > 0$ for the policy-constraint method, however, the relative magnitude is smaller than hopper-expert.

In contrast, Q-functions learned by CQL, generally satisfy $\hat{\Delta}^k < 0$, as is seen  and these values are clearly smaller than those for the policy-constraint method. This provides some empirical evidence for Theorem~\ref{thm:gap_amplify}, in that, the maximum Q-value at a randomly chosen action from the uniform distribution the action space is smaller than the Q-value at in-distribution actions.

On the hopper-expert task, as we show in Figure~\ref{fig:delta_plots}(a) (right), we eventually observe an ``unlearning'' effect, in the policy-constraint method where the policy performance deteriorates after a extra iterations in training. This ``unlearning'' effect is similar to what has been observed when standard off-policy Q-learning algorithms without any policy constraint are used in the offline regime~\citep{kumar2019stabilizing,levine2020offline}, on the other hand this effect is absent in the case of CQL, even after equally many training steps. The performance in the more-stochastic hopper-medium dataset fluctuates, but does not deteriorate.
%%AK.5.30: need some ending.. what do these plots suggest.

To summarize this discussion, we concretely observed the following points via empirical evidence:
\begin{itemize}
\vspace{-10pt}
    \item CQL backups are gap expanding in practice, as justified by the negative $\hat{\Delta}^k$ values in Figure~\ref{fig:delta_plots}.
    \item Policy constraint methods, that do not impose any regularization on the Q-function may observe highly positive $\hat{\Delta}^k$ values during training, especially with narrow data distributions, indicating that gap-expansion may  be absent.
    \item When $\hat{\Delta}^k$ values continuously grow during training, the policy might eventually suffer from an unlearning effect~\citep{levine2020offline}, as shown in Figure~\ref{fig:delta_plots}(a).
    \vspace{-10pt}
\end{itemize}

\begin{figure}
    \begin{subfigure}[h]{0.49\linewidth}
      \centering
      \includegraphics[width=0.47\linewidth]{NeuRIPS2019/images/hopper-expert-v0bear_vs_cql.pdf}
      \includegraphics[width=0.47\linewidth]{NeuRIPS2019/images/hopper-expert-v0bear_vs_cql_return.pdf}
      \caption{hopper-expert-v0}
    \end{subfigure}
    ~
    \begin{subfigure}[h]{0.49\linewidth}
      \centering
      \includegraphics[width=0.47\linewidth]{NeuRIPS2019/images/hopper-medium-v0bear_vs_cql_again.pdf}
      \includegraphics[width=0.47\linewidth]{NeuRIPS2019/images/hopper-medium-v0bear_vs_cql_again_return.pdf}
      \caption{hopper-medium-v0}
      %%SL.5.27: label the plots -- label both axes and title
    \end{subfigure}
    \caption{$\Delta^k$ as a function of training iterations for hopper-expert and hopper-medium datasets. Note that CQL (left) generally has negative values of $\Delta$, whereas BEAR (right) generally has positive $\Delta$ values, which also increase during training with increasing $k$ values.}
    %%SL.5.27: Make sure caption at least briefly summarizes the implications of this
    \label{fig:delta_plots}
\end{figure}

% \textbf{Why does CQL solve this problem?} Theorem~\ref{thm:gap_amplify} indicates that, by appropriately controlling for $\alpha_k$, CQL can ensure that the learned $\hat{\Delta}^k$ is larger than the actual value of $\Delta^k$ in the MDP (when evaluated using the true Q-function, $Q^k$). That is, for all $k$, we have that $\hat{\Delta}^k > \Delta^k$ under appropriate choices of $\alpha_1, \cdots, \alpha_k$. Empirically, this translates to generally negative (or positive with a small magnitude) values of $\hat{\Delta}^k$ for CQL, as shown in Figure~\ref{fig:delta_plots}(a) and (b). Note that it is sufficient for the empirical $\hat{\Delta}^k$ to be larger than $\Delta^k$, and not necessarily negative. 
% %%AK.5.26: Revisit this statement once.
% CQL generally maintains a negative value of $\hat{\Delta}^k$, and this difference is reflected in better and more stable final policy performance for CQL, as shown in Figure ??, even without a policy constraint. 

\section{Experimental Evaluation}
\label{sec:experiments}

%%SL.5.24: For the sake of space, I think we can cut this and the subsection headings, and things are still mostly readable.
%The goal of our empirical evaluation is to study and compare the performance of CQL to prior offline RL methods on a wide range of domains and dataset compositions. We evaluate CQL along  multiple axes: the complexity of the underlying control problem, the way that the offline data is collected, and with both low-dimensional state and high-dimensional image inputs. 

% Additionally, we also evaluate CQL as an off-policy evaluation (OPE) method on the high-dimensional MuJoCo benchmark tasks, which typically pose a major challenge for modern OPE methods~\citep{nachum2019dualdice}. 

% Finally, we aim to analyze different components of CQL, highlighting the significance and empirical effects of the different design choices discussed in Section~\ref{sec:framework}.

%\subsection{Offline RL in Continuous Control Domains}
We compare CQL to prior offline RL methods on a range of domains and dataset compositions, including continuous and discrete action spaces, state observations of varying dimensionality, and high-dimensional image inputs. We first evaluate actor-critic CQL, using CQL($\mathcal{H}$) from Algorithm~\ref{alg:practical_alg}, on continuous control datasets from the D4RL benchmark~\citep{d4rl}.
We compare to: prior offline RL methods that use a policy constraint -- BEAR~\citep{kumar2019stabilizing} and BRAC~\citep{wu2019behavior}; SAC~\citep{haarnoja}, an off-policy actor-critic method that we adapt to offline setting; and behavioral cloning (BC). {The code and instructions for reproducing our latest results in jax can be found at: \url{https://github.com/young-geng/JaxCQL}. Please refer to this repository and the latest D4RL for future comparisons. D4RL-v0 is deprecated.}
% We also ran AlgeaDICE~\citep{nachum2019algaedice}, a method that uses a state-marginal constraint, using code released by the authors, but were unable to get it to improve over a random initialization on any of the tasks, as learning was unstable and diverged quickly, and we provide these learning curves in Appendix~\ref{app:additional_results}.
%%SL.5.24: For AlgeaDICE, I still think these sentences are damn awkward. If you really want to have this, maybe rephrase more briefly: We also attempted to run AlgeaDICE~\citep{} on these tasks, but were unable to obtain reasonable results with the authors' code (more details in Appendix~\ref{app:additional_results}). [I would also recommend omitting this sentence in the arxiv and only including it in the submission copy for the reviewers!]
% done, removed
%%SL.5.11: Could we add a comparison to AWR? I feel like this would not only make the results more complete, but also help to "legitimize" the viability of AWR as an offline RL algorithm. I think many readers of the original paper might have missed that AWR is also a good offline RL method, and by not comparing to it here, we in some sense reinforce that notion.
%%AK.5.13: TODO in the D4RL paper


\begin{table*}[h]
\captionsetup{font=small}
\centering
%\scriptsize
\fontsize{9}{9}\selectfont

\begin{tabular}{l|r|r|r|r|r||r}
\hline
\textbf{Task Name} & \textbf{SAC} & \textbf{BC} & \textbf{BEAR} & \textbf{BRAC-p} & \textbf{BRAC-v} & \textbf{CQL($\mathcal{H}$)}\\ \hline
% halfcheetah-random &  30.5 & 2.1 & 25.5 & 23.5 & 28.1 & \textbf{35.4} \\
% hopper-random & \textbf{11.3} & 9.8 & 9.5 & \textbf{11.1} & \textbf{12.0} & \textbf{10.8}\\
% walker2d-random & 4.1 & 1.6 & \textbf{6.7} & 0.8 & 0.5 & \textbf{7.0}\\ \hline
halfcheetah-medium-v2 & -4.3 & 42.6 & 38.6 & {44.0} & {51} & {48.4 $\pm$ 0.3}\\
walker2d-medium-v2 & 0.9 & 75.3  & 33.2 & 72.7 & {81.3} & {82.0 $\pm$ 1.0}\\
hopper-medium-v2 & 0.8 & 52.9 & 47.6 & 31.2 & {86.6} & {71.8 $\pm$ 2.5}\\ \hline
% halfcheetah-expert & -1.9 & \textbf{107.0} & \textbf{108.2} & 3.8 & -1.1 & {104.8}\\
% hopper-expert & 0.7 & \textbf{109.0} & \textbf{110.3} & 6.6 & 3.7 & \textbf{109.9} \\
% walker2d-expert & -0.3 & \textbf{125.7} & 106.1 & -0.2 & -0.0 & \textbf{121.6} \\ \hline
halfcheetah-medium-expert-v2 & 1.8 & 55.2 & 51.7 & 43.8 & 44.0 & {87.3 $\pm$ 0.2}\\
walker2d-medium-expert-v2 & 1.9 & 52.5 & 10.8 & -0.3 & {111.6} & {106.1 $\pm$ 7.2}\\
hopper-medium-expert-v2 & 1.6 & 107.5 & 4.0 & 1.1 & 79.0 & {109.2 $\pm$ 3.6}\\ \hline
halfcheetah-medium-replay-v2 & -2.4 & 36.6 & 36.2 & {47.6} & {45.3} & {47.0 $\pm$ 0.2} \\
hopper-medium-replay-v2 & 3.5 & 18.1 & 25.3 & 0.7 & {96.2} & {93.8 $\pm$ 2.8}\\
walker2d-medium-replay-v2 & 1.9 & 26.0 & 10.8 & -0.3 & {85.0} & {86.2 $\pm$ 0.2}\\
\hline
\textbf{Total} & 5.7 & 466.7 & - & - & 680.0 & \textbf{731.8}\\
\hline
\end{tabular}
\vspace{0.05cm}
\caption{\label{table:mujoco}{\small Performance of CQL($\mathcal{H}$) and prior methods on gym domains from D4RL, on the normalized return metric, averaged over 3 seeds. Note that CQL performs similarly or better than the best prior method with simple datasets, and outperforms prior methods with complex distributions (``--medium-replay'', ``--medium-expert'').}}
\normalsize
\vspace{-10pt}
\end{table*}



% \begin{table*}[h]
% \captionsetup{font=small}
% \centering
% %\scriptsize
% \fontsize{8}{8}\selectfont

% \begin{tabular}{l|r|r|r|r|r||r}
% \hline
% \textbf{Task Name} & \textbf{SAC} & \textbf{BC} & \textbf{BEAR} & \textbf{BRAC-p} & \textbf{BRAC-v} & \textbf{CQL($\mathcal{H}$)}\\ \hline
% % halfcheetah-random &  30.5 & 2.1 & 25.5 & 23.5 & 28.1 & \textbf{35.4} \\
% % hopper-random & \textbf{11.3} & 9.8 & 9.5 & \textbf{11.1} & \textbf{12.0} & \textbf{10.8}\\
% % walker2d-random & 4.1 & 1.6 & \textbf{6.7} & 0.8 & 0.5 & \textbf{7.0}\\ \hline
% halfcheetah-medium-v2 & -4.3 & 42.6 & 38.6 & \textbf{44.0} & \textbf{51} & {46.1}\\
% walker2d-medium-v2 & 0.9 & 75.3  & 33.2 & 72.7 & \textbf{81.3} & {74.5}\\
% hopper-medium-v2 & 0.8 & 52.9 & 47.6 & 31.2 & \textbf{86.6} & {64.6}\\ \hline
% % halfcheetah-expert & -1.9 & \textbf{107.0} & \textbf{108.2} & 3.8 & -1.1 & {104.8}\\
% % hopper-expert & 0.7 & \textbf{109.0} & \textbf{110.3} & 6.6 & 3.7 & \textbf{109.9} \\
% % walker2d-expert & -0.3 & \textbf{125.7} & 106.1 & -0.2 & -0.0 & \textbf{121.6} \\ \hline
% halfcheetah-medium-expert-v2 & 1.8 & 55.2 & 51.7 & 43.8 & 44.0 & \textbf{87.3}\\
% walker2d-medium-expert-v2 & 1.9 & 52.5 & 10.8 & -0.3 & \textbf{111.6} & \textbf{109.9}\\
% hopper-medium-expert-v2 & 1.6 & 107.5 & 4.0 & 1.1 & 79.0 & \textbf{109.2}\\ \hline
% halfcheetah-medium-replay-v2 & -2.4 & 36.6 & 36.2 & \textbf{47.6} & \textbf{45.3} & \textbf{45.4} \\
% hopper-medium-replay-v2 & 3.5 & 18.1 & 25.3 & 0.7 & \textbf{96.2} & \textbf{92.3}\\
% walker2d-medium-replay-v2 & 1.9 & 26.0 & 10.8 & -0.3 & \textbf{85.0} & \textbf{83.7}\\
% \hline
% \textbf{Total} & 5.7 & 466.7 & - & - & 680.0 & \textbf{713.0}\\
% \hline
% \end{tabular}
% \vspace{-4pt}
% \caption{\label{table:mujoco}{\small Performance of CQL($\mathcal{H}$) and prior methods on gym domains from D4RL, on the normalized return metric, averaged over 3 seeds. Note that CQL performs similarly or better than the best prior method with simple datasets, and outperforms prior methods with complex distributions (``--medium-replay'', ``--random-expert'', ``--medium-expert'').}}
% \normalsize
% \vspace{-20pt}
% \end{table*}

% halfcheetah-random & -17.9 & 3502.0 & -2.1 & 2885.6 & 2641.0 & 3207.3 & \textbf{4524.7} \\
% halfcheetah-medium &  4196.4 & -808.6 & 4117.0 & 4508.7 & \textbf{5178.2} & \textbf{5365.3} & \textbf{5281.6} \\
% halfcheetah-mixed &  4492.1 & -581.3 & 3377.2 & 4211.3 & \textbf{5384.7} & \textbf{5413.8} & \textbf{5476.3}\\
% halfcheetah-medium-expert &  4169.4 & -55.7 &  7415.6 & 6132.5 & 5156.0 & 5342.4 & \textbf{9025.0}\\
% halfcheetah-random-expert &  -113.1 & 6298.7 & 7161.3 & 2768.8 & 3465.2 & -9.55 & \textbf{7951.9}\\ 
% \hline
% walker2d-random & 73.0 & 192.0 & 266.9 & 307.6 & 38.4 & 23.9 & \textbf{323.1}\\
% walker2d-medium & 304.8 & 44.2 & 881.7 & 1526.7 & 3341.1 & \textbf{3734.4} & \textbf{3801.4} \\
% walker2d-mixed & 518.6 & 87.8 & 384.6 & 495.3 & -11.5 & 44.5 & \textbf{1391.0} \\
% walker2d-medium-expert & 297.0 & -5.1 & 876.7 & 1193.6 & 141.7 & 3058.9 & \textbf{4348.1} \\
% walker2d-random-expert & 33.5 & 37.1 &  & 88.9 & 12.1 & 124.9 & \textbf{4245.8} \\
% \hline
% hopper-random & 299.4 & 347.7 & 274.7 & 289.5 & \textbf{341.0} & \textbf{370.5} & \textbf{355.2} \\
% hopper-medium & 923.5 & 5.7 & 1186.7 & 1527.9 & 994.8 & 1030.0 & \textbf{2403.1}\\
% hopper-mixed & 364.4 & 93.3 & 491.3 & 802.7 & 2.0 & 5.3 & \textbf{2267.9}\\
% hopper-medium-expert & \textbf{3621.2} & 32.9 & 2146.8 & 109.8 & 16.0 & 5.1 & \textbf{3699.1}\\
% \hline
% ant-random & & & & & & & 932.4\\
% ant-medium & & & & & & & 3325.4 \\
% ant-mixed & & & & & & & 2973.8\\
% ant-medium-expert & & & & & & & 3400.1 \\
% ant-random-expert & & & & & & & 2309.3\\
% \hline
% \begin{wrapfigure}{r}{0.15\textwidth}
%   \vspace{-20pt}
%   \begin{center}
%     % \includegraphics[width=0.17\textwidth]{chapters/cql/images/antmaze_all.png}
%     \includegraphics[width=0.97\linewidth]{chapters/cql/images/adroit_all.png}
%   \end{center}
%   \vspace{-21pt}
% \end{wrapfigure}
\textbf{Gym domains\footnote{An earlier version of the paper had results corresponding to \texttt{v0} versions of the D4RL benchmark from May 2020. Since then, D4RL has performed several modifications with respect to terminal/timeout handling and the latest release of D4RL is the v2 release. To facilitate comparisons on the latest version of the D4RL benchmark, we provide results with the \texttt{-v2} domains now.}} Results for the gym domains are shown in Table~\ref{table:mujoco}. The results for BEAR, BRAC, SAC, and BC are based on numbers reported by \citet{d4rl}. We find that across the board, on both the datasets generated by a single mediocre policy, marked as ``-medium'' and the datasets generated by mixing together experience from multiple policies (``-medium-replay'' and ``-medium-expert''), that are more likely to be encountered in practical problems, CQL outperforms prior methods.

% While CQL performs similarly to the best performing prior method when the offline dataset is constructed from a single RL policy, it significantly outperforms prior methods with complex data distributions, which are more indicative of real-world offline datasets~\citep{d4rl}.
%%SL.5.11: Well, if we're not going to show any of the AlgeaDICE results, maybe we shouldn't mention it before like this? We could add a sentence to the end of the first paragraph that describes all the prior methods and say something like: We also ran AlgeaDICE~\citep{} using code released by the authors, but were unable to get this method to improve over a random initialization on any of the tasks, as learning was unstable and diverged quickly.
% done
%%SL.5.11: In general, I think it would be best to break up the reported results into groups, and present them more slowly. For example, we could write the above paragraph something like this:
%The final performance of CQL($\mathcal{H}$) and each prior method on the D4RL tasks is shown in Table~\ref{table:something}. The first group of results is based on single-policy datasets for the MuJoCo gym benchmark environments: the ``-random'' and ``-medium'' datasets for HalfCheetah, Walker2D, and Hopper. These datsets are produced by either a random policy, or a medium-reward policy, respectively (see \citet{d4rl} for details). On these single-policy datasets, CQL matches or exceeds the best prior methods, but by a small margin. However, on the datasets that combine multiple policies (``-mixed,'' ``medium-expert,'' ``-random-expert''), prior methods generally perform substantially worse. Such mixed datasets are likely to be more common in real-world applications of offline RL, where training data might come from a variety of sources. CQL outperforms prior methods in these settings, on all three environments, in some cases by as much as 2-4x.
%%SL.5.11: You might want to also group the results in the tables/plots based on these groupings


% & halfcheetah-random & 100.0 & 2.1 & 30.5 & 25.5 & 23.5 & 28.1\\
% & walker2d-random & 100.0 & 1.6 & 4.1 & 6.7 & 0.8 & 0.5\\
% & hopper-random & 100.0 & 9.8 & 11.3 & 9.5 & 11.1 & 12.0\\
% & halfcheetah-medium & 100.0 & 36.1 & -4.3 & 38.6 & 44.0 & 45.5\\
% & walker2d-medium & 100.0 & 6.6 & 0.9 & 33.2 & 72.7 & 81.3\\
% & hopper-medium & 100.0 & 29.0 & 0.8 & 47.6 & 31.2 & 32.3\\
% & halfcheetah-medium-replay & 100.0 & 38.4 & -2.4 & 36.2 & 45.6 & 45.9\\
% & walker2d-medium-replay & 100.0 & 11.3 & 1.9 & 10.8 & -0.3 & 0.9\\
% & hopper-medium-replay & 100.0 & 11.8 & 3.5 & 25.3 & 0.7 & 0.8\\
% & halfcheetah-medium-expert & 100.0 & 35.8 & 1.8 & 51.7 & 43.8 & 45.3\\
% & walker2d-medium-expert & 100.0 & 6.4 & -0.1 & 26.0 & 3.1 & 66.6\\
% & hopper-medium-expert & 100.0 & 111.9 & 1.6 & 4.0 & 1.1 & 0.8\\
% %%%%%%%%%%%%%% TABLE %%%%%%%%%%%%%%%%%%%%%%
% \begin{table*}[h]
% \centering
% \small
% \begin{tabular}{l|r|r|r|r|r|r||r}
% \hline
% \textbf{Task Name} & \textbf{BC} & \textbf{SAC} & \textbf{BCQ} & \textbf{BEAR} & \textbf{BRAC-p} & \textbf{BRAC-v} & \textbf{CQL($\mathcal{H}$)}\\ \hline
% halfcheetah-random & -17.9 & 3502.0 & -2.1 & 2885.6 & 2641.0 & 3207.3 & \textbf{4524.7} \\
% halfcheetah-medium &  4196.4 & -808.6 & 4117.0 & 4508.7 & \textbf{5178.2} & \textbf{5365.3} & \textbf{5281.6} \\
% halfcheetah-mixed &  4492.1 & -581.3 & 3377.2 & 4211.3 & \textbf{5384.7} & \textbf{5413.8} & \textbf{5476.3}\\
% halfcheetah-medium-expert &  4169.4 & -55.7 &  7415.6 & 6132.5 & 5156.0 & 5342.4 & \textbf{9025.0}\\
% halfcheetah-random-expert &  -113.1 & 6298.7 & 7161.3 & 2768.8 & 3465.2 & -9.55 & \textbf{7951.9}\\ 
% \hline
% walker2d-random & 73.0 & 192.0 & 266.9 & 307.6 & 38.4 & 23.9 & \textbf{323.1}\\
% walker2d-medium & 304.8 & 44.2 & 881.7 & 1526.7 & 3341.1 & \textbf{3734.4} & \textbf{3801.4} \\
% walker2d-mixed & 518.6 & 87.8 & 384.6 & 495.3 & -11.5 & 44.5 & \textbf{1391.0} \\
% walker2d-medium-expert & 297.0 & -5.1 & 876.7 & 1193.6 & 141.7 & 3058.9 & \textbf{4348.1} \\
% walker2d-random-expert & 33.5 & 37.1 &  & 88.9 & 12.1 & 124.9 & \textbf{4245.8} \\
% \hline
% hopper-random & 299.4 & 347.7 & 274.7 & 289.5 & \textbf{341.0} & \textbf{370.5} & \textbf{355.2} \\
% hopper-medium & 923.5 & 5.7 & 1186.7 & 1527.9 & 994.8 & 1030.0 & \textbf{2403.1}\\
% hopper-mixed & 364.4 & 93.3 & 491.3 & 802.7 & 2.0 & 5.3 & \textbf{2267.9}\\
% hopper-medium-expert & \textbf{3621.2} & 32.9 & 2146.8 & 109.8 & 16.0 & 5.1 & \textbf{3699.1}\\
% \hline
% % ant-random & & & & & & & 932.4\\
% % ant-medium & & & & & & & 3325.4 \\
% % ant-mixed & & & & & & & 2973.8\\
% % ant-medium-expert & & & & & & & 3400.1 \\
% % ant-random-expert & & & & & & & 2309.3\\
% % \hline
% \end{tabular}
% \caption{{\footnotesize Performance of CQL and prior methods on gym MuJoCo tasks from D4RL. Observe that CQL performs similarly to the best prior method with simple datasets generated from a single RL policy, and greatly outperforms prior methods with mixed datasets.}}
% \label{table:mujoco}
% \normalsize
% \end{table*}
% %%SL.5.11: Maybe show some pictures of these tasks and the adroit and maze tasks, for readers that aren't familiar?
% % Yes I can add these, but i am concerned about too much space utilization
%%%%%%%%%%%%%%%%%%%%%%%%%%%%
% \begin{figure*}
% \centering
% \includegraphics[width=0.99\linewidth]{chapters/cql/images/mujoco_results_v1.png}
% \caption{{\footnotesize Performance of CQL and prior methods on gym MuJoCo tasks from D4RL reported as a normalized score. Observe that CQL performs similarly to the best prior method with simple datasets generated from a single RL policy, and greatly outperforms prior methods with mixed datasets.}}
% \end{figure*}
    % \includegraphics[width=0.9\linewidth]{chapters/cql/images/adept_franka_kitchen.png}

%% adroit tasks
\textbf{Adroit tasks.} The more complex Adroit~\citep{rajeswaran2018dapg} tasks in D4RL require controlling a 24-DoF robotic hand, using limited data from human demonstrations. These tasks are
substantially more difficult than the gym tasks in terms of both the dataset composition and high dimensionality. Prior offline RL methods generally struggle to learn meaningful behaviors on 
these tasks, and the strongest baseline is BC. As shown in Table~\ref{table:adroit_antmaze}, CQL variants are the only methods that improve over BC, attaining scores that are \textbf{2-9x} those of the next best offline RL method. CQL($\rho$) with $\rho = \hat{\policy}^{k-1}$ (the previous policy) outperforms CQL($\mathcal{H}$) on a number of these tasks, due to the higher action dimensionality resulting in higher variance for the CQL($\mathcal{H}$) importance weights. Both variants outperform prior methods.
%%AK.5.23: need to say something about the CQL(rho) variant doing better -- which is due to importance sampling, but we never really talk about importance sampling so far.

\begin{table}[H]
\captionsetup{font=small}
\vspace{-5pt}
\centering
%\scriptsize
\fontsize{8}{8}\selectfont
\begin{tabular}{l|l|r|r|r|r|r||r|r}
\hline
\textbf{Domain} & \textbf{Task Name} & \textbf{BC} & \textbf{SAC} & \textbf{BEAR} & \textbf{BRAC-p} & \textbf{BRAC-v} & \textbf{CQL($\mathcal{H}$)} & \textbf{CQL($\rho$)}\\ \hline
% \multirow{3}*{Maze2D}
% & maze2d-umaze &  3.8 & 88.2 & 3.4 & 4.7 & -14.6\\
% & maze2d-medium &  30.3 & 26.1 & 29.0 & 32.4 & 149.0\\
% & maze2d-large &  5.0 & -1.9 & 4.6 & 10.4 & 150.0\\
% \hline
\multirow{6}*{AntMaze}
& antmaze-umaze-v2 & 54.6 & 0.0 & {73.0} & 50.0 & 70.0 & \textbf{94.0} & - \\
& antmaze-umaze-diverse-v2  & 45.6 & 0.0 & 61.0 & 40.0 & 70.0 & {47.3} & - \\
& antmaze-medium-play-v2  & 0.0 & 0.0 & 0.0 & 0.0 & 0.0 & \textbf{62.4} & - \\
& antmaze-medium-diverse-v2  & 0.0 & 0.0 & 8.0 & 0.0 & 0.0 & \textbf{74.3}  & - \\
& antmaze-large-play-v2 & 0.0 & 0.0 & 0.0 & 0.0 & 0.0 & \textbf{34.2} & - \\
& antmaze-large-diverse-v2 & 0.0 & 0.0 & 0.0 & 0.0 & 0.0 & \textbf{40.7} & - \\
\hline
& \textbf{Total (antmazes)} & 100.2 & 0.0 & - & - & - & \textbf{352.9} & -\\
\hline
\multirow{8}*{Adroit}
& pen-human  & 34.4 & 6.3 & -1.0 & 8.1 & 0.6 & 37.5 & \textbf{55.8}\\
& hammer-human & 1.5 & 0.5 & 0.3 & 0.3 & 0.2 & \textbf{4.4} & {2.1}\\
& door-human & 0.5 & 3.9 & -0.3 & -0.3 & -0.3 & \textbf{9.9} & \textbf{9.1} \\
& relocate-human & 0.0 & 0.0 & -0.3 & -0.3 & -0.3 & 0.20 & \textbf{0.35}\\
& pen-cloned  & \textbf{56.9} & 23.5 & 26.5 & 1.6 & -2.5 & 39.2 & 40.3\\
& hammer-cloned & 0.8 & 0.2 & 0.3 & 0.3 & 0.3 & 2.1 & \textbf{5.7} \\
& door-cloned & -0.1 & 0.0 & -0.1 & -0.1 & -0.1 & 0.4 & \textbf{3.5}\\
& relocate-cloned & \textbf{-0.1} & -0.2 & -0.3 & -0.3 & -0.3 & \textbf{-0.1} & \textbf{-0.1}\\
\hline
\multirow{3}*{Kitchen}
& kitchen-complete & 33.8 & 15.0 & 0.0 & 0.0 & 0.0 & \textbf{43.8} & 31.3\\
& kitchen-partial & 33.8 & 0.0 & 13.1 & 0.0 & 0.0 & \textbf{49.8} & \textbf{50.1} \\
& kitchen-undirected & 47.5 & 2.5 & 47.2 & 0.0 & 0.0 & \textbf{51.0} & \textbf{52.4} \\ \hline
\end{tabular}
\vspace{0.1cm}
\caption{\label{table:adroit_antmaze}{Normalized scores of all methods on AntMaze, Adroit, and kitchen domains from D4RL, averaged across 4 seeds. On the harder mazes, CQL is the \textit{only} method that attains non-zero returns, and is the only method to outperform simple behavioral cloning on Adroit tasks with human demonstrations.
We observed that the CQL($\rho$) variant, which avoids importance weights, trains more stably, with no sudden fluctuations in policy performance over the course of training, on the higher-dimensional Adroit tasks.}}
\normalsize
\vspace{-22pt}
\end{table}

\textbf{AntMaze.\footnote{The results here use \texttt{antmaze-*-v2}. The original results from the 2020 version are shown in Appendix~\ref{sec:old_results}.}} 
These D4RL tasks require composing parts of suboptimal trajectories to form more optimal policies for reaching goals on a MuJoco Ant robot. 
Prior methods make some progress on the simpler U-maze, but only CQL is able to make meaningful progress  
%% adroit tasks
on the much harder medium and large mazes, outperforming prior methods by a wide margin.
%, achieving nearly \textbf{2-5x} higher return as compared to prior methods.

\begin{wrapfigure}{r}{0.15\textwidth}
  \vspace{-25pt}
  \begin{center}
  \includegraphics[width=0.99\linewidth]{chapters/cql/images/adept_franka_kitchen.png}
  \end{center}
  \vspace{-20pt}
\end{wrapfigure}
\textbf{Kitchen tasks.} Next, we evaluate CQL on the Franka kitchen domain~\citep{gupta2019relay} from D4RL~\citep{d4rl_repo}.
The goal is to control a 9-DoF robot to manipulate multiple objects (microwave, kettle, etc.) \textit{sequentially}, in a single episode to reach a desired configuration, with only sparse 0-1 completion reward for every object that attains the target configuration. These tasks are especially challenging, since they require composing parts of trajectories, precise long-horizon manipulation, and handling human-provided teleoperation data. As shown in Table~\ref{table:adroit_antmaze}, CQL outperforms prior methods in this setting, and is the only method that outperforms behavioral cloning, attaining over \textbf{40\%} success rate on all tasks.



% \begin{table*}[h]
% \centering
% \small
% \begin{tabular}{l|r|r|r|r|r|r|r|r}
% \hline
% \textbf{Task Name} & \textbf{SAC} & \textbf{BC} & \textbf{SAC-off} & \textbf{BEAR} & \textbf{BRAC-p} & \textbf{BRAC-v} & \textbf{CQL($\mathcal{H}$)} & \textbf{CQL($\rho$)}\\ \hline
% ant-umaze & 0.0 & 0.7 & 0.0 & 0.7 & 0.5 & 0.7 & \textbf{0.74} & \\
% ant-umaze-diverse & 0.0 & 0.6 & 0.0 & 0.6 & 0.4 & 0.7 & \textbf{0.79} & \\
% ant-medium-play & 0.0 & 0.0 & 0.0 & 0.0 & 0.0 & 0.0 & \textbf{0.54 } & \\
% ant-medium-diverse & 0.0 & 0.0 & 0.0 & 0.0 & 0.0 & 0.0 & \textbf{0.53} & \\
% ant-large-play & 0.0 & 0.0 & 0.0 & 0.0 & 0.0 & 0.0 & \textbf{0.20} & \\
% ant-large-diverse & 0.0 & 0.0 & 0.0 & 0.0 & 0.0 & 0.0 & \textbf{0.19} & \\
% \hline
% pen-human & 739.3 & 1121.9 & 284.8 & 66.3 & 339.0 & 114.7 & \textbf{1367.5} & \\
% hammer-human & -248.7 & -82.4 & -214.2 & -242.0 & -239.7 & -243.8 & \textbf{1098.1} & \\
% door-human & -61.8 & -41.7 & 57.2 & -66.4 & -66.5 & -66.4 & \textbf{273.0} & \\
% relocate-human & -13.7 & -5.6 & -4.5 & -18.9 & -19.7 & -19.7 & \textbf{5.7} & \\
% pen-cloned & \\
% hammer-cloned & \\
% door-cloned & \\
% relocate-cloned & \\
% \hline
% mini-kitchen-MKLS & \\
% kitchen-MKLS & \\
% kitchen-MKBL & \\
% \hline
% \end{tabular}
% \caption{{\footnotesize Performance of CQL and prior methods on AntMaze, Adroit and kitchen tasks from D4RL. Observe that CQL outperforms prior methods on the simpler U-shaped maze and is the \textit{only} method that is able to obtain non-zero performance on the harder, medium-sized maze. CQL is the only method that achieves a positive return on adroit tasks with human demonstrations.}}
% \label{table:adroit_antmaze}
% \normalsize
% \end{table*}
% \multirow{12}*{Adroit}
% & pen-human & 739.3 & 1121.9 & 284.8 & 66.3 & 339.0 & 114.7\\
% & pen-cloned & 739.3 & 1791.8 & 797.6 & 885.4 & 143.4 & 22.2\\
% & pen-expert & 739.3 & 2633.7 & 277.4 & 3254.1 & -7.8 & 6.4\\
% & hammer-human & -248.7 & -82.4 & -214.2 & -242.0 & -239.7 & -243.8\\
% & hammer-cloned & -248.7 & -175.1 & -244.1 & -241.1 & -236.7 & -236.9\\
% & hammer-expert & -248.7 & 16140.8 & 3019.5 & 16359.7 & -241.4 & -241.1\\
% & door-human & -61.8 & -41.7 & 57.2 & -66.4 & -66.5 & -66.4\\
% & door-cloned & -61.8 & -60.7 & -56.3 & -60.9 & -58.7 & -59.0\\
% & door-expert & -61.8 & 969.4 & 163.8 & 2980.1 & -66.4 & -66.6\\
% & relocate-human & -13.7 & -5.6 & -4.5 & -18.9 & -19.7 & -19.7\\
% & relocate-cloned & -13.7 & -10.1 & -16.1 & -17.6 & -19.8 & -19.4\\
% & relocate-expert & -13.7 & 4289.3 & -18.2 & 4173.8 & -20.6 & -21.4\\
% \hline
% \multirow{12}*{Gym}
% & halfcheetah-random & 12135.0 & -17.9 & 3502.0 & 2885.6 & 2641.0 & 3207.3\\
% & halfcheetah-medium & 12135.0 & 4196.4 & -808.6 & 4508.7 & 5178.2 & 5365.3\\
% & halfcheetah-mixed & 12135.0 & 4492.1 & -581.3 & 4211.3 & 5384.7 & 5413.8\\
% & halfcheetah-medium-expert & 12135.0 & 4169.4 & -55.7 & 6132.5 & 5156.0 & 5342.4\\
% & walker2d-random & 4592.3 & 73.0 & 192.0 & 307.6 & 38.4 & 23.9\\
% & walker2d-medium & 4592.3 & 304.8 & 44.2 & 1526.7 & 3341.1 & 3734.4\\
% & walker2d-mixed & 4592.3 & 518.6 & 87.8 & 495.3 & -11.5 & 44.5\\
% & walker2d-medium-expert & 4592.3 & 297.0 & -5.1 & 1193.6 & 141.7 & 3058.9\\
% & hopper-random & 3234.3 & 299.4 & 347.7 & 289.5 & 341.0 & 370.5\\
% & hopper-medium & 3234.3 & 923.5 & 5.7 & 1527.9 & 994.8 & 1030.0\\
% & hopper-mixed & 3234.3 & 364.4 & 93.3 & 802.7 & 2.0 & 5.3\\
% & hopper-medium-expert & 3234.3 & 3621.2 & 32.9 & 109.8 & 16.0 & 5.1\\
% \hline

%%SL.5.24: Consider deleting subsection headings in the whole experiments section. They really don't add much, and you can save some space.
%\subsection{Offline Image-Based RL on the Arcade Learning Environment}
\textbf{Offline RL on Atari games\footnote{Results for CQL on more Atari games, with varied dataset compositions can be found in Appendix F of DR3 (Kumar et al. ICLR 2022), available at the following arXiv URL: \url{https://arxiv.org/abs/2112.04716}}.} Lastly, we evaluate a discrete-action Q-learning variant of CQL (Algorithm~\ref{alg:practical_alg}) on offline, image-based Atari games~\citep{bellemare2013arcade}. We compare CQL to REM~\citep{agarwal2019optimistic} and QR-DQN~\citep{dabney2018distributional} on the five Atari tasks (Pong, Breakout, Qbert, Seaquest and Asterix) that are evaluated in detail by \citet{agarwal2019optimistic}, using the dataset released by the authors. 
% We were not able to reproduce the QR-DQN and REM results on Asterix using the official implementation and data from \citet{agarwal2019optimistic}, and no method was able to perform well on the provided data for this task, but we include the scores in Table~\ref{table:atari_reduced_size}.

\begin{figure*}[h]
\vspace{-12pt}
\begin{center}
  \includegraphics[width=0.23\linewidth]{chapters/cql/images/pong.pdf}  
  \includegraphics[width=.23\linewidth]{chapters/cql/images/breakout.pdf}  
  \includegraphics[width=.23\linewidth]{chapters/cql/images/qbert.pdf}  
  \includegraphics[width=.23\linewidth]{chapters/cql/images/seaquest.pdf}  
\end{center}
\vspace{-4pt}
\caption{\label{fig:cql_20m_atari}{\footnotesize Performance of CQL, QR-DQN and REM as a function of training steps (x-axis) in setting \textbf{(1)} when provided with only the first 20\% of the samples of an online DQN run. Note that CQL is able to learn stably on 3 out of 4 games, and its performance does not degrade as steeply as QR-DQN on Seaquest.}}
\vspace{-10pt}
\end{figure*}
Following the evaluation protocol of \citet{agarwal2019optimistic}, we evaluated on two types of datasets, both of which were generated from the DQN-replay dataset, released by~\citep{agarwal2019optimistic}:
\textbf{(1)} a dataset consisting of the first 20\% of the samples observed by an online DQN agent and 
\textbf{(2)} datasets consisting of only 1\%  and 10\% of all samples observed by an online DQN agent (Figures 6 and 7 in \citep{agarwal2019optimistic}). In setting \textbf{(1)}, shown in Figure~\ref{fig:cql_20m_atari}, CQL generally achieves similar or better performance throughout as QR-DQN and REM. When only using only 1\% or 10\% of the data, in setting \textbf{(2)} (Table~\ref{table:atari_reduced_size}),  CQL 
%AK: TODO (aviral): revisit this if we can't get seaquest to work -- Seaquest is almost maxed out, need to decide on what to say here?
\begin{wraptable}{r}{6.0cm}
\captionsetup{font=small}
    \centering
    \vspace{-7pt}
    \fontsize{8}{8}\selectfont
    \begin{tabular}{l|r|r||r}
    \hline
        \textbf{Task Name} & \textbf{QR-DQN} & \textbf{REM} & \textbf{CQL($\mathcal{H}$)} \\
        \hline
         Pong (1\%) & -13.8 & -6.9 & \textbf{19.3} \\
         Breakout & 7.9 & 11.0 & \textbf{61.1} \\
         Q*bert & 383.6 & 343.4 & \textbf{14012.0} \\
         Seaquest & 672.9 & 499.8 & \textbf{779.4} \\
         Asterix  & 166.3 & 386.5 & \textbf{592.4}\\
         \hline
         \hline
         Pong (10\%) & 15.1 & 8.9 & \textbf{18.5} \\
         Breakout & 151.2 & 86.7 & \textbf{269.3} \\
         Q*bert & 7091.3 & 8624.3 & \textbf{13855.6}\\
         Seaquest & 2984.8 & \textbf{3936.6} & 3674.1 \\
         Asterix & \textbf{189.2} & 75.1 & 156.3 \\
         \hline
    \end{tabular}
    \vspace{-4pt}
    \caption{{CQL, REM and QR-DQN in setting \textbf{(1)} with 1\% data (top), and 10\% data (bottom). CQL drastically outperforms prior methods with 1\% data, and usually attains better performance with 10\% data.}}
    \normalsize
    \label{table:atari_reduced_size}
    \vspace{-17pt}
\end{wraptable}
substantially outperforms REM and QR-DQN, especially in the harder 1\% condition, achieving \textbf{36x} and \textbf{6x} times the return of the best prior method on Q*bert and Breakout, respectively.
% \let\thefootnote\relax\footnote{{\small *Neither CQL nor any prior method, including REM, attains non-trivial performance on Asterix. When using exactly the same dataset and implementation provided by \citet{agarwal2019optimistic}, we were unable to reproduce the numbers reported in~\citep{agarwal2019optimistic} for QR-DQN and REM. }}

% \begin{table}[h]
% \captionsetup{font=small}
%     \centering
%     \vspace{-10pt}
%     \fontsize{8}{8}\selectfont
%     \begin{tabular}{l|r|r|r|||r|r|r}
%     \hline
%         \textbf{Task Name} & \textbf{QR-DQN} & \textbf{REM} & \textbf{CQL($\mathcal{H}$)}  & \textbf{QR-DQN} & \textbf{REM} & \textbf{CQL($\mathcal{H}$)}  \\
%         \hline
%          Pong & 3.0 & 6.0 & \textbf{19.3} & 16.7 & 16.8 & \textbf{18.5} \\
%          Breakout & 7.8 & 9.2 & \textbf{61.1} & 176.0 & 96.0 & \textbf{269.3} \\
%          Q*bert & 500.0 & 500.0 & \textbf{14012.0}& 11201.8 & 8070.2 & \textbf{13855.6}\\
%          Seaquest & 480.2 & 390.1 & \textbf{779.4} & 3300.0 & \textbf{4290.4} & 3674.1 \\
%          %%SL.5.30: Something is wrong! This footnote doesn't seem to actually show up in the paper!!
%          Asterix\footnote{No method (CQL or prior methods, REM and QR-DQN, were able to attain non-trivial performance on this game using the dataset provided by \citet{agarwal2019optimistic}} & 166.3 & 386.5 & \textbf{592.4} & 189.2 & 75.1 & 156.3 \\
%          \hline
%     \end{tabular}
%     \caption{{CQL, REM and QR-DQN in setting \textbf{(1)} with 1\% data (left), and 10\% data (right). CQL drastically outperforms prior methods with 1\% data, and usually attains better performance with 10\%.}}
%     \normalsize
%     \label{table:atari_reduced_size}
% \footnote{No method (CQL or prior methods, REM and QR-DQN, were able to attain non-trivial performance on this game using the dataset provided by \citet{agarwal2019optimistic}}
% \end{table}

\textbf{Analysis of CQL.} Finally, we perform empirical evaluation to verify that CQL indeed lower-bounds the value function, thus verifying Theorems~\ref{thm:cql_underestimates}, Appendix~\ref{thm:policy_eval_func_approx} empirically. To this end, we estimate the average value of the learned policy predicted by CQL, $\E_{\bs \sim \mathcal{D}}[\hat{V}^k(\bs)]$, and report the difference against the actual discounted return of the policy $\policy^{k}$ in Table~\ref{table:cql_lower_bound}. We also estimate these values for baselines, including the minimum predicted Q-value under an ensemble~\citep{haarnoja,fujimoto2018addressing}
%%SL.5.30: Maybe reference some papers that do this, like TD3 and REM?
of Q-functions with varying ensemble sizes, which is a standard technique to prevent overestimed Q-values~\citep{fujimoto2018addressing,haarnoja,hasselt2010double} and BEAR~\citep{kumar2019stabilizing}, a policy constraint method. The results show that CQL learns a lower bound for all three tasks, whereas the baselines are prone to overestimation. We also evaluate a variant of CQL that uses Equation~\ref{eqn:objective_1}, and observe that the resulting values are lower (that is, underestimate the true values) as compared to CQL($\mathcal{H}$). This provides empirical evidence that CQL($\mathcal{H}$) attains a tighter lower bound than the point-wise bound in Equation~\ref{eqn:objective_1}, as per Theorem~\ref{thm:cql_underestimates}. We also present an empirical analysis to show that Theorem~\ref{thm:gap_amplify}, that CQL is gap-expanding, holds in practice in Appendix~\ref{app:gap_amplify}, and present an ablation study on various design choices used in CQL in Appendix~\ref{app:additional_results}.

\begin{table}[h]
\captionsetup{font=small}
\centering
\vspace{-5pt}
    \fontsize{8}{8}\selectfont
    \begin{tabular}{l|r|r||r|r|r|r|r}
    \hline
        \textbf{Task Name} & \textbf{CQL($\mathcal{H})$} & \textbf{CQL (Eqn.~\ref{eqn:objective_1})} & \textbf{Ensemble(2)}  & \textbf{Ens.(4)} & \textbf{Ens.(10)} & \textbf{Ens.(20)} & \textbf{BEAR} \\
        \hline
        %%SL.5.30: bold the best result that is still a bound
        hopper-medium-expert & \textbf{-43.20} & -151.36 & 3.71e6 & 2.93e6 & 0.32e6 & 24.05e3 & 65.93 \\
        hopper-mixed & \textbf{-10.93} & -22.87 & 15.00e6 &  59.93e3 & 8.92e3 & 2.47e3 & 1399.46 \\
        hopper-medium & \textbf{-7.48} & -156.70 & 26.03e12 & 437.57e6 & 1.12e12 & 885e3 & 4.32 \\
        %%SL.5.30: I would recommend using scientific notation (e.g., e6 or $\times 10^6$) instead of "M" or "T"
        \hline
    \end{tabular}
    \caption{{Difference between predicted policy values and the true policy value for CQL, a variant of CQL that uses Equation~\ref{eqn:objective_1}, the minimum of an ensemble of varying sizes, and BEAR~\citep{kumar2019stabilizing} on three D4RL datasets. CQL is the only method that lower-bounds the actual return (i.e., has negative differences), and CQL($\mathcal{H})$ is much less conservative than CQL (Eqn.~\ref{eqn:objective_1}).}}
    \normalsize
    \vspace{-10pt}
    \label{table:cql_lower_bound}
\end{table}


%%AK.5.23: Commenting out this for now, discussion in the email. Basically we are not doing great here especially with the right way of using the Q-values for computing policy return. It is conservative, but it is too conservative in 4/8 cases (like the method outputs 153.4 as the return, but the actual return is 276). I think that if we believe that the other story is storng enough, we should probably not have it.
% \subsection{Off-Policy Evaluation Experiments}

% %%SL.5.11: Can we start off with a little transition to explain why we're talking about this? E.g., something like: The core of CQL is a Q-function learning method that learns a lower bound on the true Q-function. This lower bound can be used to learn a better policy, or simply to evaluate an existing policy. In this section, we study this latter setting, applying CQL to the off-policy evaluation problem.
% % done
% CQL learns a Q-function such that the expected value of the policy under this learned Q-function lower-bounds the true policy value. In this section, we evaluate the performance of CQL for evaluating the performance of a pre-specified policy using pre-specified offline datasets on three continuous control domains: Hopper, HalfCheetah and Walker2d. In accordance with the setup in prior work~\citep{Zhang2020GenDICE}, our task is to evaluate the return of an expert SAC policy using data generated from a mixture of suboptimal policies. Thus, we use the medium-expert and mixed datasets from \citep{d4rl} that contain a diverse mixture of suboptimal policies as the offline datasets.

% We compare CQL to state-of-the-art prior method, GenDICE~\citep{zhang2019generalized}, in Table ??. Note that CQL learns conservative return estimates, that lower bound the actual discounted return of the target policy, thus empirically validating the theoretical insights discussed in Section~\ref{sec:policy_eval}. In contrast,  (One line about OTHER METHODS once we have numbers) \ak{TODO: put table.}


\vspace{-9pt}
\section{Discussion}
% summary
\vspace{-9pt}
We proposed conservative Q-learning (CQL), an algorithmic framework for offline RL that learns a lower bound on the policy value.
%%SL: commenting out for length
%We show that the expected values of the Q-functions learned via CQL bound the expected values of the true Q-function. The resulting evaluation and learning algorithms are simple and easy to implement on top of existing Q-learning and actor-critic methods, using only a small modification to the Q-function objective.
Empirically, we demonstrate that CQL outperforms prior offline RL methods on a wide range of offline RL benchmark tasks, including complex control tasks and tasks with raw image observations. In many cases, the performance of CQL is substantially better than the best-performing prior methods, exceeding their final returns by 2-5x.
% implications
The simplicity and efficacy of CQL make it a promising choice for a wide range of real-world offline RL problems. However, a number of challenges remain. While we prove that CQL learns lower bounds on the Q-function in the tabular, linear, and a subset of non-linear function approximation cases, a rigorous theoretical analysis of CQL with deep neural nets, is left for future work. Additionally, offline RL methods are liable to suffer from overfitting in the same way as standard supervised methods, so another important challenge for future work is to devise simple and effective early stopping methods, analogous to validation error in supervised learning.
%We hope that the efficacy of CQL on complex datasets, which combine data from multiple sources (e.g., good and mediocre, real-world behavior policies), its performance with high-dimensional states and actions, and its simplicity and compactness will make it an appealing and promising choice for a wide range of real-world RL problems that must utilize previously collected datasets.

% limitations + future work
%While CQL empirically outperforms prior methods across a wide range of tasks, a number of technical questions still remain. While we prove that CQL learns lower bounds on the Q-function in the tabular and linear case, a rigorous theoretical analysis of CQL with non-linear function approximators, such as deep networks, is left for future work. Additionally, offline RL methods are liable to suffer from overfitting in the same way as standard supervised methods, so another important challenge for future work is to devise simple and effective early stopping methods, analogous to validation error in supervised learning. While these avenues for future promise to further improve the effectiveness of offline RL methods, we believe that CQL represents an important step toward offline RL algorithms that are simple, practical, and highly effective. Such algorithms, when scaled up and applied to large real-world datasets, have the potential to make it possible to bring RL to bear on a much wider range of problems than has been possible with traditional active RL methods.     

% \section*{Broader Impact}
% % very quick summary of the work
% Offline RL offers the promise to scale autonomous learning-based methods for decision-making to large-scale, real-world sequential decision making problems. Such methods can effectively leverage prior datasets without any further interaction, thus avoiding the exploration bottleneck and alleviating many of the safety and cost constraints associated with online reinforcement learning.
% In this work, we proposed conservative Q-learning (CQL), an algorithmic framework for offline reinforcement learning that learns a Q-function such that the expected policy value under this learned Q-function lower-bounds the actual policy value. This mitigates value function over-estimation issues due to distributional shift, which in practice are one of the major challenges in offline reinforcement learning. We analyzed algorithms derived from the CQL framework and demonstrated their performance empirically.

% %impacts in terms of applications where this could be used
% Our primary aim behind this work is to develop simple and effective offline RL algorithms, and we believe that CQL makes an important step in that direction. CQL can be applied directly to several problems of practical interest where large-scale datasets are abundant: autonomous driving, robotics, and software systems (such as recommender systems). We believe that a strong offline RL algorithm, coupled with highly expressive and powerful deep neural networks, will provide us the ability successfully apply end-to-end learning based approaches to such problems, providing considerable societal benefits. Of course, autonomous decision-making agents have a wide range of applications, and technology that enables more effective autonomous decision-making has both positive and negative societal effects. While effective autonomous decision-making can have considerable positive economic effects, it can also enable applications with complex implications in regard to privacy (e.g., in regard to autonomous agents on the web, recommendation agents, advertising, etc.), as well as complex economic effects due to changing economic conditions (e.g., changing job requirements, loss of jobs in some sectors and growth in others, etc.). Such implications apply broadly to technologies that enable automation and decision making, and are largely not unique to this specific work.

\vspace{-0.2cm}
\section*{Acknowledgements and Funding}
\vspace{-0.2cm}
We thank Mohammad Norouzi, Oleh Rybkin, Anton Raichuk, Avi Singh, Vitchyr Pong and anonymous reviewers from the Robotic AI and Learning Lab at UC Berkeley and NeurIPS for their feedback on an earlier version of this paper. We thank Rishabh Agarwal for help with the Atari QR-DQN/REM codebase and for sharing baseline results. This research was funded by the DARPA Assured Autonomy program, and compute support from Google and Amazon.


\bibliography{main.bib}
\bibliographystyle{plainnat}

\newpage
\appendix
\section{COMBO Proofs from Section~\ref{sec:combo_theory}}
\label{app:combo_proofs}

In this section, we provide proofs for theoretical results in Section~\ref{sec:combo_theory}. Before the proofs, we note that all statements are proven in the case of finite state space (i.e., $|\states| < \infty$) and finite action space (i.e., $|\actions| < \infty$) we define some commonly appearing notation symbols appearing in the proof: 
\begin{itemize}
\vspace{-5pt}
    \item $P_{\mdp}$ and $r_{\mdp}$ (or $P$ and $r$ with no subscript for notational simplicity) denote the dynamics and reward function of the actual MDP $\mdp$
    \vspace{-5pt}
    \item $P_{\mdpbar}$ and $r_{\mdpbar}$ denote the dynamics and reward of the empirical MDP $\mdpbar$ generated from the transitions in the dataset
    \vspace{-5pt}
    \item $P_{\mdphat}$ and $r_{\mdphat}$ denote the dynamics and reward of the MDP induced by the learned model $\mdphat$
\end{itemize}
\vspace{-5pt}
We also assume that whenever the cardinality of a particular state or state-action pair in the offline dataset $\data$, denoted by $|\mathcal{D}(\bs, \mathbf{a})|$, appears in the denominator, we assume it is non-zero. For any non-existent $(\bs, \mathbf{a}) \notin \data$, we can simply set $|\data(\bs, \mathbf{a})|$ to be a small value $< 1$, which prevents any bound from producing trivially $\infty$ values.

\subsection{A Useful Lemma and Its Proof}
\label{app:proof_lemma}

Before proving our main results, we first show that the penalty
term in equation \ref{eqn:combo_iterate} is positive in expectation. Such a positive penalty is important to combat any overestimation that may
arise as a result of using $\bellmanhat$.

\begin{lemma}[(Interpolation Lemma]
\label{thm:line_thm}
For any $f \in [0, 1]$, and any given $\rho(\bs, \mathbf{a}) \in \Delta^{|\states||\actions|}$, let $d_f$ be an f-interpolation of $\rho$ and $\data$, i.e., $d_f(\bs, \mathbf{a}) := f d(\bs, \mathbf{a}) + (1-f) \rho(\bs, \mathbf{a})$. For a given iteration $k$ of Equation~\ref{eqn:combo_iterate}, we restate the definition of the expected penalty under $\rho(\bs, \mathbf{a})$ in Eq.~\ref{eqn:expected_penalty}: 
\begin{equation*}
 \nu(\rho, f) := \E_{\bs, \mathbf{a} \sim \rho(\bs, \mathbf{a})}\left[\frac{\rho(\bs, \mathbf{a}) - d(\bs, \mathbf{a})}{d_f(\bs, \mathbf{a})} \right].
\end{equation*}
Then $\nu(\rho, f)$ satisfies, (1) $\nu(\rho, f) \geq 0,~~ \forall \rho, f$, (2) $\nu(\rho, f)$ is monotonically increasing in $f$ for a fixed $\rho$, and (3) $\nu(\rho, f) = 0$ iff $\forall~ \bs, \mathbf{a}, ~\rho(\bs, \mathbf{a}) = d(\bs, \mathbf{a}) \text{~or~} f = 0$. 
\end{lemma}
\begin{proof}
To prove this lemma, we use algebraic manipulation on the expression for quantity $\nu(\rho, f)$ and show that it is indeed positive and monotonically increasing in $f \in [0, 1]$.
\begin{align}
    \nu(\rho, f) &= \sum_{\bs, \mathbf{a}} \rho(\bs, \mathbf{a}) \left(\frac{\rho(\bs, \mathbf{a}) - d(\bs, \mathbf{a})}{f d(\bs, \mathbf{a}) + (1 - f) \rho(\bs, \mathbf{a})}\right)\nonumber \\
    &= \sum_{\bs, \mathbf{a}} \rho(\bs, \mathbf{a}) \left(\frac{\rho(\bs, \mathbf{a}) - d(\bs, \mathbf{a})}{\rho(\bs, \mathbf{a}) + f ( d(\bs, \mathbf{a}) - \rho(\bs, \mathbf{a}))}\right)\\
    \implies \frac{d \nu(\rho, f)}{d f} &= \sum_{\bs, \mathbf{a}} \rho(\bs, \mathbf{a}) \left(\rho(\bs, \mathbf{a}) - d(\bs, \mathbf{a})\right)^2 \cdot \left(\frac{1}{(\rho(\bs, \mathbf{a}) + f ( d(\bs, \mathbf{a}) - \rho(\bs, \mathbf{a}))}\right)^2 \geq 0\nonumber\\
    &~~~\forall f \in [0, 1].
\end{align}
Since the derivative of $\nu(\rho, f)$ with respect to $f$ is always positive, it is an increasing function of $f$ for a fixed $\rho$, and this proves the second part (2) of the Lemma. Using this property, we can show the part (1) of the Lemma as follows:
\begin{align}
    \forall f \in (0, 1],~ \nu(\rho, f) \geq \nu(\rho, 0) = \sum_{\bs, \mathbf{a}} \rho(\bs, \mathbf{a}) \frac{\rho(\bs, \mathbf{a}) - d(\bs, \mathbf{a})}{\rho(\bs, \mathbf{a})} &= \sum_{\bs, \mathbf{a}} \left( \rho(\bs, \mathbf{a}) - d(\bs, \mathbf{a}) \right)\nonumber\\
    &= 1 - 1 = 0.
\end{align}
Finally, to prove the third part (3) of this Lemma, note that when $f = 0$, $\nu(\rho, f) = 0$ (as shown above), and similarly by setting $\rho(\bs, \mathbf{a}) = d(\bs, \mathbf{a})$ note that we obtain $\nu(\rho, f) = 0$. To prove the only if side of (3), assume that $f \neq 0$ and $\rho(\bs, \mathbf{a}) \neq d(\bs, \mathbf{a})$ and we will show that in this case $\nu(\rho,f) \neq 0$. When $d(\bs, \mathbf{a}) \neq \rho(\bs, \mathbf{a})$, the derivative $\frac{d \nu(\rho,f)}{d f} > 0$ (i.e., strictly positive) and hence the function $\nu(\rho, f)$ is a strictly increasing function of $f$. Thus, in this case, $\nu(\rho, f) > 0 = \nu(\rho, 0)~ \forall f > 0$. Thus we have shown that if $\rho(\bs, \mathbf{a}) \neq d(\bs, \mathbf{a})$ and $f > 0$, $\nu(\rho, f) \neq 0$, which completes our proof for the only if side of (3). 
\end{proof}

\subsection{Proof of Proposition~\ref{thm:lower_bound}}
\label{app:proof_lower_bound}
Before proving this proposition, we provide a bound on the Bellman backup in the empirical MDP, $\bellman_{\mdpbar}$. To do so, we formally define the standard concentration properties of the reward and transition dynamics in the empirical MDP, $\mdpbar$, that we assume so as to prove Proposition~\ref{thm:line_thm}. Following prior work~\citep{osband2017posterior,jaksch2010near,kumar2020conservative}, we assume:
\begin{assumption}
\label{assumption:conc}
    $\forall~ \bs, \mathbf{a} \in \mdp$, the following relationships hold with high probability, $\geq 1 - \delta$
    \begin{equation*}
        |r_{\mdpbar}(\bs, \mathbf{a}) - r(\bs, \mathbf{a})| \leq \frac{C_{r, \delta}}{\sqrt{|\mathcal{D}(\bs, \mathbf{a})|}}, ~~~ ||P_{\mdpbar}(\bs'|\bs, \mathbf{a}) - P(\bs'|\bs, \mathbf{a})||_{1} \leq \frac{C_{P, \delta}}{\sqrt{|\mathcal{D}(\bs, \mathbf{a})|}}.
    \end{equation*}
\end{assumption}
Under this assumption and assuming that the reward function in the MDP, $r(\bs, \mathbf{a})$ is bounded, as $|r(\bs, \mathbf{a})| \leq R_{\max}$, we can bound the difference between the empirical Bellman operator, $\bellman_{\mdpbar}$ and the actual MDP, $\bellman_\mdp$,
\begin{align*}
    \left\vert\left({\bellman_{\mdpbar}}^\policy \hat{Q}^k \right) - \left({\bellman}^\policy_\mdp \hat{Q}^k \right)\right\vert &= \left\vert\left(r_{\mdpbar}(\bs, \mathbf{a}) - r_\mdp(\bs, \mathbf{a})\right)\right.\\
    &\left.+ \gamma \sum_{\bs'} \left({P}_{\mdpbar}(\bs'|\bs, \mathbf{a}) - P_\mdp(\bs'|\bs,\mathbf{a})\right) \E_{\policy(\mathbf{a}'|\bs')}\left[\hat{Q}^k(\bs' , \mathbf{a}')\right]\right\vert\\
    &\leq \left\vert r_{\mdpbar}(\bs, \mathbf{a}) - r_\mdp(\bs, \mathbf{a})\right\vert\\
    &+ \gamma \left\vert \sum_{\bs'} \left({P}_{\mdpbar}(\bs'|\bs, \mathbf{a}) - P_\mdp(\bs'|\bs,\mathbf{a})\right) \E_{\policy(\mathbf{a}'|\bs')}\left[\hat{Q}^k(\bs' , \mathbf{a}')\right]\right\vert\\
    &\leq \frac{C_{r, \delta} + \gamma C_{P, \delta} 2R_{\max} / (1 - \gamma)}{\sqrt{|\mathcal{D}(\bs, \mathbf{a})|}}. 
\end{align*}
Thus the overestimation due to sampling error in the empirical MDP, $\mdpbar$ is bounded as a function of a bigger constant, $C_{r, P, \delta}$ that can be expressed as a function of $C_{r, \delta}$ and $C_{P, \delta}$, and depends on $\delta$ via a $\sqrt{\log (1/\delta)}$ dependency. For the purposes of proving Proposition~\ref{thm:Q_bound}, we assume that:
\begin{equation}
\label{eqn:sampling_error}
    \forall \bs, \mathbf{a}, ~~\left\vert\left({\bellman_{\mdpbar}}^\policy \hat{Q}^k \right) - \left({\bellman}^\policy_\mdp \hat{Q}^k \right)\right\vert  \leq \frac{C_{r, T, \delta} R_{\max}}{(1 - \gamma) \sqrt{|\mathcal{D}(\bs, \mathbf{a})|}}.
\end{equation}

Next, we provide a bound on the error between the bellman backup induced by the learned dynamics model and the learned reward, $\bellman_{\mdphat}$, and the actual Bellman backup, $\bellman_{\mdp}$. To do so, we note that:
\begin{align}
    \left\vert\left({\bellman_{\mdphat}}^\policy \hat{Q}^k \right) - \left({\bellman}^\policy_\mdp \hat{Q}^k \right)\right\vert &= \left\vert\left(r_{\mdphat}(\bs, \mathbf{a}) - r_\mdp(\bs, \mathbf{a})\right)\right.\\
    &\left.+ \gamma \sum_{\bs'} \left({P}_{\mdphat}(\bs'|\bs, \mathbf{a}) - P_\mdp(\bs'|\bs,\mathbf{a})\right) \E_{\policy(\mathbf{a}'|\bs')}\left[\hat{Q}^k(\bs' , \mathbf{a}')\right]\right\vert \nonumber\\ 
    &\leq |r_{\mdphat}(\bs, \mathbf{a}) - r_\mdp(\bs, \mathbf{a})| + \gamma \frac{2 R_{\max}}{1 - \gamma} D(P, P_{\mdphat}),
    \label{eqn:model_error} 
\end{align}
where $D(P, P_{\mdphat})$ is the total-variation divergence between the learned dynamics model and the actual MDP. Now, we show that the asymptotic Q-function learned by COMBO lower-bounds the actual Q-function of any
policy $\pi$ with high probability for a large enough $\beta \geq 0$. We will use Equations~\ref{eqn:sampling_error} and \ref{eqn:model_error} to prove such a result.

\begin{proposition}[Asymptotic lower-bound]
\label{thm:Q_bound}
Let $P^\pi$ denote the Hadamard product of the dynamics $P$ and a given policy $\pi$ in the actual MDP and let $S^\pi := (I - \gamma P^\pi)^{-1}$. Let $D$ denote the total-variation divergence between two probability distributions. For any $\pi(\mathbf{a}|\bs)$, the Q-function obtained by recursively applying Equation~\ref{eqn:combo_iterate}, with $\hat{{\bellman}}^\pi = f \bellman_{\mdpbar}^\pi + (1 - f) \bellman_{\mdphat}^\pi$, with probability at least $1 - \delta$, results in $\hat{Q}^\pi$ that satisfies:
\begin{align*}
    \forall \bs, \mathbf{a},~ \hat{Q}^\pi(\bs, \mathbf{a}) \leq  Q^\pi(\bs, \mathbf{a}) &- \beta \cdot \left[ S^\pi \left[ \frac{\rho - d}{d_f} \right] \right](\bs, \mathbf{a}) + f \left[ S^\pi \left[ \frac{C_{r, T, \delta} R_{\max}}{(1 - \gamma) \sqrt{|\data|}} \right] \right](\bs, \mathbf{a})\\
    +&~ (1 - f) \left[ S^\pi \left[ |r - r_{\mdphat}| + \frac{ 2 \gamma  R_{\max}}{1 - \gamma} D(P, P_{\mdphat}) \right]  \right]\!\! (\bs, \mathbf{a}).
\end{align*}
\end{proposition}
\begin{proof}
We first note that the Bellman backup $\hat{\bellman}^\pi$ induces the following Q-function iterates as per Equation~\ref{eqn:combo_iterate},
\begin{align*}
    \hat{Q}^{k+1}(\bs, \mathbf{a}) &= \left(\hat{\bellman}^\pi \hat{Q}^k\right)(\bs, \mathbf{a}) - \beta \frac{\rho(\bs, \mathbf{a}) - d(\bs, \mathbf{a})}{d_f(\bs, \mathbf{a})}\\
    &=  f \left(\bellman^\pi_{\mdpbar} \hat{Q}^k \right) (\bs, \mathbf{a}) + (1 - f) \left(\bellman^\pi_{\mdphat} \hat{Q}^k \right) (\bs, \mathbf{a}) - \beta \frac{\rho(\bs, \mathbf{a}) - d(\bs, \mathbf{a})}{d_f(\bs, \mathbf{a})}\\
    &= \left(\bellman^\pi \hat{Q}^k\right)(\bs, \mathbf{a}) - \beta \frac{\rho(\bs, \mathbf{a}) - d(\bs, \mathbf{a})}{d_f(\bs, \mathbf{a})} + (1 - f) \left({\bellman_{\mdphat}}^\policy \hat{Q}^k - {\bellman}^\policy \hat{Q}^k \right)(\bs, \mathbf{a})\\
    &+ f  \left({\bellman_{\mdpbar}}^\policy \hat{Q}^k - {\bellman}^\policy \hat{Q}^k \right)(\bs, \mathbf{a})\\
   \forall \bs, \mathbf{a},~ \hat{Q}^{k+1} &\leq \left(\bellman^\pi \hat{Q}^k\right) - \beta \frac{\rho - d}{d_f} + (1 - f) \left[|r_{\mdphat} - r_\mdp| + \frac{2 \gamma R_{\max}}{1 - \gamma} D(P, P_{\mdphat}) \right] + f \frac{C_{r, T, \delta} R_{\max}}{(1 - \gamma) \sqrt{|\data|}} 
\end{align*}
Since the RHS upper bounds the Q-function pointwise for each $(\bs, \mathbf{a})$, the fixed point of the Bellman iteration process will be pointwise smaller than the fixed point of the Q-function found by solving for the RHS via equality. Thus, we get that
\begin{align*}
    \hat{Q}^\pi(\bs, \mathbf{a}) &\leq \underbrace{ S^\pi r_{\mdp}}_{= Q^\pi(\bs, \mathbf{a})} -\beta \left[ S^\pi \left[ \frac{\rho - d}{d_f} \right] \right](\bs, \mathbf{a}) +~ f \left[ S^\pi \left[ \frac{C_{r, T, \delta} R_{\max}}{(1 - \gamma) \sqrt{|\data|}} \right] \right](\bs, \mathbf{a})\\
    &+~ (1 - f) \left[ S^\pi \left[ |r - r_{\mdphat}| + \frac{ 2 \gamma  R_{\max}}{1 - \gamma} D(P, P_{\mdphat}) \right]  \right]\!\! (\bs, \mathbf{a}),  
\end{align*}
which completes the proof of this proposition.
\end{proof}

Next, we use the result and proof technique from Proposition~\ref{thm:Q_bound} to prove Corollary~\ref{thm:lower_bound}, that in expectation under the initial state-distribution, the expected Q-value is indeed a lower-bound. 

\begin{corollary}[Corollary~\ref{thm:lower_bound} restated]
For a sufficiently large $\beta$, we have a lower-bound that
$\E_{\bs \sim \mu_0, \mathbf{a} \sim \policy(\cdot|\bs)}[\hat{Q}^\pi(\bs, \mathbf{a})] \leq \E_{\bs \sim \mu_0, \mathbf{a} \sim \policy(\cdot|\bs)}[Q^\pi(\bs, \mathbf{a})]$, 
where $\mu_0(\bs)$ is the initial state distribution. 
Furthermore, when $\epsilon_{\text{s}}$ is small, such as in the large sample regime; or when the model bias $\epsilon_{\text{m}}$ is small, a small $\beta$ is sufficient along with an appropriate choice of $f$.
\end{corollary}

\begin{proof}
To prove this corollary, we note a slightly different variant of Proposition~\ref{thm:Q_bound}. To observe this, we will deviate from the proof of Proposition~\ref{thm:Q_bound} slightly and will aim to express the inequality using $\bellman_{\mdphat}$, the Bellman operator defined by the learned model and the reward function. Denoting $(I - \gamma P_{\mdphat})^{-1}$ as $S_{\mdphat}^\pi$, doing this will intuitively allow us to obtain $\beta \left(\mu(\bs) \policy(\mathbf{a}|\bs)\right)^T \left(S_{\mdphat}^\pi \left[\frac{\rho - d}{d_f} \right]\right)(\bs, \mathbf{a})$ as the conservative penalty which can be controlled by choosing $\beta$ appropriately so as to nullify the potential overestimation caused due to other terms. Formally,
\begin{align*}
    \hat{Q}^{k+1}(\bs, \mathbf{a}) &= \left(\hat{\bellman}^\pi \hat{Q}^k\right)(\bs, \mathbf{a}) - \beta \frac{\rho(\bs, \mathbf{a}) - d(\bs, \mathbf{a})}{d_f(\bs, \mathbf{a})} = \left(\bellman^\pi_{\mdphat} \hat{Q}^k \right)(\bs, \mathbf{a}) -  \beta \frac{\rho(\bs, \mathbf{a}) - d(\bs, \mathbf{a})}{d_f(\bs, \mathbf{a})}\\
    &+ f \underbrace{\left(\bellman^\pi_{\mdpbar} - \bellman^\pi_{\mdphat} \hat{Q}^k \right)(\bs, \mathbf{a})}_{:= \Delta(\bs, \mathbf{a})}
\end{align*}
By controlling $\Delta(\bs, \mathbf{a})$ using the pointwise triangle inequality:
\begin{equation}
    \forall \bs, \mathbf{a}, ~\left\vert \bellman^\pi_{\mdpbar} \hat{Q}^k - \bellman^\pi_{\mdphat} \hat{Q}^k \right\vert \leq \left\vert \bellman^\pi \hat{Q}^k - \bellman^\pi_{\mdphat} \hat{Q}^k \right\vert + \left\vert \bellman^\pi_{\mdpbar} \hat{Q}^k - \bellman^\pi \hat{Q}^k \right\vert,
\end{equation}
and then iterating the backup $\bellman^\pi_{\mdphat}$ to its fixed point and finally noting that $\rho(\bs, \mathbf{a}) = \left((\mu \cdot \pi)^T S^\pi_{\mdphat}\right)(\bs, \mathbf{a})$, we obtain:
\begin{equation}
    \E_{\mu, \pi}[\hat{Q}^\pi(\bs, \mathbf{a})] \leq \E_{\mu, \pi}[Q^\pi_{\mdphat}(\bs, \mathbf{a})] - \beta~ \E_{\rho(\bs, \mathbf{a})}\left[\frac{\rho(\bs, \mathbf{a}) - d(\bs, \mathbf{a})}{d_f(\bs, \mathbf{a})}\right] + \mathrm{terms~ independent~ of~} \beta.
\end{equation}
%%AK: there is one more term in the equation above, fit it in one line...
The terms marked as ``terms independent of $\beta$'' correspond to the additional positive error terms obtained by iterating $\left\vert \bellman^\pi \hat{Q}^k - \bellman^\pi_{\mdphat} \hat{Q}^k \right\vert$ and $\left\vert \bellman^\pi_{\mdpbar} \hat{Q}^k - \bellman^\pi \hat{Q}^k \right\vert$, which can be bounded similar to the proof of Proposition~\ref{thm:Q_bound} above. Now by replacing the model Q-function, $\E_{\mu, \pi}[Q^\pi_{\mdphat}(\bs, \mathbf{a})]$ with the actual Q-function, $\E_{\mu, \pi}[Q^\pi(\bs, \mathbf{a})]$ and adding an error term corresponding to model error to the bound, we obtain that:
\begin{equation}
\label{eqn:lower_bound_eqn}
    \E_{\mu, \pi}[\hat{Q}^\pi(\bs, \mathbf{a})] \leq \E_{\mu, \pi}[Q^\pi(\bs, \mathbf{a})] + \mathrm{terms~ independent~ of~} \beta - \beta~ \underbrace{\E_{\rho(\bs, \mathbf{a})}\left[\frac{\rho(\bs, \mathbf{a}) - d(\bs, \mathbf{a})}{d_f(\bs, \mathbf{a})}\right]}_{= \nu(\rho, f) > 0}.
\end{equation}
Hence, by choosing $\beta$ large enough, we obtain the desired lower bound guarantee. 
\end{proof}

\begin{remark}[\underline{\textbf{COMBO does not underestimate at every $\bs \in \mathcal{D}$ unlike CQL.}}]
\label{remak:tighter_lower_bound}
Before concluding this section, we discuss how the bound obtained by COMBO (Equation~\ref{eqn:lower_bound_eqn}) is tighter than CQL. CQL learns a Q-function such that the value of the policy under the resulting Q-function lower-bounds the true value function at each state $\bs \in \mathcal{D}$ individually (in the absence of no sampling error), i.e., $\forall \bs \in \mathcal{D}, \hat{V}^\pi_{\text{CQL}}(\bs) \leq V^\pi(\bs)$, whereas the bound in COMBO is only valid in expectation of the value function over the initial state distribution, i.e., $\E_{\bs \sim \mu_0(\bs)}[\hat{V}^\pi_{\text{COMBO}}(\bs)] \leq \E_{\bs \sim \mu_0(\bs)}[V^\pi(\bs)]$, and the value function at a given state may not be a lower-bound. For instance, COMBO can overestimate the value of a state more frequent in the dataset distribution $d(\bs, \mathbf{a})$ but not so frequent in the $\rho(\bs, \mathbf{a})$ marginal distribution of the policy under the learned model $\mdphat$. To see this more formally, note that the expected penalty added in the effective Bellman backup performed by COMBO (Equation~\ref{eqn:combo_iterate}), in expectation under the dataset distribution $d(\bs, \mathbf{a})$, $\widetilde{\nu}(\rho, d, f)$ is actually \textbf{\textit{negative}}:
\begin{align*}
    \widetilde{\nu}(\rho, d, f) = \sum_{\bs, \mathbf{a}} d(\bs, \mathbf{a}) \frac{\rho(\bs, \mathbf{a}) - d(\bs, \mathbf{a})}{d_f(\bs, \mathbf{a})} = - \sum_{\bs, \mathbf{a}} d(\bs, \mathbf{a}) \frac{d(\bs, \mathbf{a}) - \rho(\bs, \mathbf{a})}{f d(\bs, \mathbf{a}) + (1 - f) \rho(\bs, \mathbf{a})} < 0,
\end{align*}
where the final inequality follows via a direct application of the proof of Lemma~\ref{thm:line_thm}. Thus, COMBO actually \emph{overestimates} the values at atleast some states (in the dataset) unlike CQL.   
\end{remark}

\subsection{Proof of Proposition~\ref{prop:less_conservative}}
\label{app:proof_less_conservative}

In this section, we will provide a proof for Proposition~\ref{prop:less_conservative}, and show that the COMBO can be less conservative in terms of the estimated value. To recall, let $\Delta^\pi_\text{COMBO} := \E_{\bs, \mathbf{a} \sim d_{\mdpbar}(\bs), \pi(\mathbf{a}|\bs)}\left[\hat{Q}^\pi(\bs, \mathbf{a} \right]$ and let $\Delta^\pi_\text{CQL} := \E_{\bs, \mathbf{a} \sim d_{\mdpbar}, \pi(\mathbf{a}|\bs)} \left[\hat{Q}^\pi_\text{CQL}(\bs, \mathbf{a}) \right]$. From \citet{kumar2020conservative}, we obtain that $\hat{Q}^\pi_\text{CQL}(\bs, \mathbf{a}) := Q^\pi(\bs, \mathbf{a}) - \beta \frac{\pi(\mathbf{a}|\bs) - \pi_\beta(\mathbf{a}|\bs)}{\pi_\beta(\mathbf{a}|\bs)}$. We shall derive the condition for the real data fraction $f=1$ for COMBO, thus making sure that $d_f(\bs) = d^{\pi_\beta}(\bs)$. To derive the condition when $\Delta^\pi_\text{COMBO} \geq \Delta^\pi_\text{CQL}$, we note the following simplifications:
\begin{align}
    & \Delta^\pi_\text{COMBO} \geq \Delta^\pi_\text{CQL} \\
    \implies & \sum_{\bs, \mathbf{a}} d_{\mdpbar}(\bs) \pi(\mathbf{a}|\bs) \hat{Q}^\pi(\bs, \mathbf{a}) \geq \sum_{\bs, \mathbf{a}} d_{\mdpbar}(\bs) \pi(\mathbf{a}|\bs) \hat{Q}^\pi_\text{CQL}(\bs, \mathbf{a}) \\
    \label{eqn:cql_vs_combo_terms}
    \implies & \beta \sum_{\bs, \mathbf{a}} d_{\mdpbar}(\bs)\pi(\mathbf{a}|\bs) \left( \frac{\rho(\bs, \mathbf{a}) - d^{\pi_\beta}(\bs) \pi_\beta(\mathbf{a}|\bs)}{d^{\pi_\beta}(\bs) \pi_\beta(\mathbf{a}|\bs)} \right) \leq \beta \sum_{\bs, \mathbf{a}} d_{\mdpbar}(\bs)\pi(\mathbf{a}|\bs) \left(\frac{\pi(\mathbf{a}|\bs) - \pi_\beta(\mathbf{a}|\bs)}{\pi_\beta(\mathbf{a}|\bs)} \right).
\end{align}
Now, in the expression on the left-hand side, we add and subtract $d^{\pi_\beta}(\bs) \pi(\mathbf{a}|\bs)$ from the numerator inside the paranthesis.
\begin{align}
    & \sum_{\bs, \mathbf{a}} d_{\mdpbar}(\bs, \mathbf{a}) \left( \frac{\rho(\bs, \mathbf{a}) - d^{\pi_\beta}(\bs) \pi_\beta(\mathbf{a}|\bs)}{d^{\pi_\beta}(\bs) \pi_\beta(\mathbf{a}|\bs)} \right)\\
    &= \sum_{\bs, \mathbf{a}} d_{\mdpbar}(\bs, \mathbf{a}) \left( \frac{\rho(\bs, \mathbf{a}) - d^{\pi_\beta}(\bs) \pi(\mathbf{a}|\bs) + d^{\pi_\beta}(\bs) \pi(\mathbf{a}|\bs) - d^{\pi_\beta}(\bs) \pi_\beta(\mathbf{a}|\bs)}{d^{\pi_\beta}(\bs) \pi_\beta(\mathbf{a}|\bs)} \right)\\
    &= \underbrace{\sum_{\bs, \mathbf{a}} d_{\mdpbar}(\bs, \mathbf{a}) \frac{\pi(\mathbf{a}|\bs) - \pi_\beta(\mathbf{a}|\bs)}{\pi_\beta(\mathbf{a}|\bs)}}_{(1)} + \sum_{\bs, \mathbf{a}} d_{\mdpbar}(\bs, \mathbf{a}) \cdot \frac{\rho(\bs) - d^{\pi_\beta}(\bs)}{d^{\pi_\beta}(\bs)} \cdot \frac{\pi(\mathbf{a}|\bs)}{\pi_\beta(\mathbf{a}|\bs)}
\end{align}
The term marked $(1)$ is identical to the CQL term that appears on the right in Equation~\ref{eqn:cql_vs_combo_terms}. Thus the inequality in Equation~\ref{eqn:cql_vs_combo_terms} is satisfied when the second term above is negative. To show this, first note that $d^{\pi_\beta}(\bs) = d_{\mdpbar}(\bs)$ which results in a cancellation. Finally, re-arranging the second term into expectations gives us the desired result. An analogous condition can be derived when $f \neq 1$, but we omit that derivation as it will be hard to interpret terms appear in the final inequality.

\subsection{Proof of Proposition~\ref{thm:policy_improvement}}
\label{app:proof_policy_improvement}

To prove the policy improvement result in Proposition~\ref{thm:policy_improvement}, we first observe that using Equation~\ref{eqn:combo_iterate} for Bellman backups amounts to finding a policy that maximizes the return of the policy in the a modified ``f-interpolant'' MDP which admits the Bellman backup $\bellmanhat^\pi$, and is induced by a linear interpolation of backups in the empirical MDP $\mdpbar$ and the MDP induced by a dynamics model $\mdphat$ and the return of a policy $\pi$ in this effective f-interpolant MDP is denoted by $J(\mdpbar, \mdphat, f, \pi)$. Alongside this, the return is penalized by the conservative penalty where $\rho^\pi$ denotes the marginal state-action distribution of policy $\pi$ in the learned model $\mdphat$. 
\begin{equation}
    \hat{J}(f, \pi) = J(\mdpbar, \mdphat, f, \pi)  - \beta \frac{\nu(\rho^\pi, f)}{1 - \gamma}.
\label{eqn:penalized_objective}
\end{equation}
We will require bounds on the return of a policy $\pi$ in this f-interpolant MDP, $J(\mdpbar, \mdphat, f, \pi)$, which we first prove separately as Lemma~\ref{lemma:interpolant_regular_bound} below and then move to the proof of Proposition~\ref{thm:policy_improvement}.

\begin{lemma}[Bound on return in f-interpolant MDP]
\label{lemma:interpolant_regular_bound}
For any two MDPs, $\mdp_1$ and $\mdp_2$, with the same state-space, action-space and discount factor, and for a given fraction $f \in [0, 1]$, define the f-interpolant MDP $\mdp_f$ as the MDP on the same state-space, action-space and with the same discount as the MDP with dynamics: $P_{\mdp_f} := f P_{\mdp_1} + (1 - f) P_{\mdp_2}$ and reward function: $r_{\mdp_f} := f r_{\mdp_1} + (1 - f) r_{\mdp_2}$. Then, given any auxiliary MDP, $\mdp$, the return of any policy $\pi$ in $\mdp_f$, $J(\pi, \mdp_f)$, also denoted by $J(\mdp_1, \mdp_2, f, \pi)$, lies in the interval:
\begin{equation*}
    \big[ J(\pi, \mdp) - \alpha,~~ J(\pi, \mdp)+ \alpha \big], \text{~~~~~~~~~~~~where~} \alpha \text{~is given by:~}
\end{equation*}
\begin{align}
    \alpha &= \frac{2 \gamma (1 - f)}{(1 - \gamma)^2} R_{\max} D \left(P_{\mdp_2}, P_{\mdp}\right) + \frac{\gamma f}{1 - \gamma} \left\vert \E_{d^\pi_{\mdp} \pi} \left[ \left(P^\pi_{\mdp} - P^\pi_{\mdp_1}\right) Q^\pi_{\mdp} \right]\right\vert  \nonumber\\
   & + \frac{f}{1 - \gamma} \E_{\bs, \mathbf{a} \sim d^\pi_{\mdp} \pi}[|r_{\mdp_1}(\bs, \mathbf{a}) - r_{\mdp}(\bs, \mathbf{a})|] + \frac{1 - f}{1 - \gamma} \E_{\bs, \mathbf{a} \sim d^\pi_{\mdp} \pi}[|r_{\mdp_2}(\bs, \mathbf{a}) - r_{\mdp}(\bs, \mathbf{a})|].  \label{eqn:alpha_expr}
\end{align}
\end{lemma}
\begin{proof}
To prove this lemma, we note two general inequalities. First, note that for a fixed transition dynamics, say $P$, the return decomposes linearly in the components of the reward as the expected return is linear in the reward function:
\begin{equation*}
    J(P, r_{\mdp_f}) = J(P, f r_{\mdp_1} + (1 - f) r_{\mdp_2}) = f J (P, r_{\mdp_1}) + (1 - f) J(P, r_{\mdp_2}).  
\end{equation*}
As a result, we can bound $J(P, r_{\mdp_f})$ using $J(P, r)$ for a new reward function $r$ of the auxiliary MDP, $\mdp$, as follows
\begin{align*}
     J(P, r_{\mdp_f}) &= J(P, f r_{\mdp_1} + (1 - f) r_{\mdp_2}) = J (P, r + f (r_{\mdp_1} - r) + (1 -f) (r_{\mdp_2} - r)\\
     &= J(P, r) + f J(P, r_{\mdp_1} - r) + (1 - f) J(P, r_{\mdp_2} - r)\\
     &= J(P, r) + \frac{f}{1 - \gamma} \E_{\bs, \mathbf{a} \sim d^\pi_{\mdp}(\bs) \pi(\mathbf{a}|\bs)}\left[ r_{\mdp_1}(\bs, \mathbf{a}) - r(\bs, \mathbf{a}) \right]\\
     &+ \frac{1 - f}{1 - \gamma} \E_{\bs, \mathbf{a} \sim d^\pi_{\mdp}(\bs) \pi(\mathbf{a}|\bs)} \left[ r_{\mdp_2}(\bs, \mathbf{a}) - r(\bs, \mathbf{a}) \right].
\end{align*}
Second, note that for a given reward function, $r$, but a linear combination of dynamics, the following bound holds:
\begin{align*}
    J(P_{\mdp_f}, r) &= J(f P_{\mdp_1} + (1 - f) P_{\mdp_2}, r)\\
    &= J ( P_{\mdp} +  f( P_{\mdp_1} - P_{\mdp}) + (1 - f) (P_{\mdp_2} - P_{\mdp}), r)\\ 
    &= J (P_{\mdp}, r) - \frac{\gamma (1 - f)}{1 - \gamma} \E_{\bs, \mathbf{a} \sim d^\pi_{\mdp}(\bs) \pi(\mathbf{a}|\bs)} \left[ \left(P^\pi_{\mdp_2} - P^\pi_{\mdp}\right) Q^\pi_{\mdp}  \right]\\
    &- \frac{\gamma f}{1 - \gamma} \E_{\bs, \mathbf{a} \sim d^\pi_{\mdp}(\bs) \pi(\mathbf{a}|\bs)} \left[ \left(P^\pi_{\mdp} - P^\pi_{\mdp_1}\right) Q^\pi_{\mdp}  \right]\\
    &\in \left[ J( P_{\mdp}, r) ~\pm~ \left(\frac{\gamma f}{(1 - \gamma)} \left\vert \E_{\bs, \mathbf{a} \sim d^\pi_{\mdp}(\bs) \pi(\mathbf{a}|\bs)}\left[ \left(P^\pi_{\mdp} - P^\pi_{\mdp_1}\right) Q^\pi_{\mdp} \right] \right\vert\right.\right.\\
    &\left.\left.+ \frac{2 \gamma (1 -f) R_{\max}}{(1 - \gamma)^2} D(P_{\mdp_2}, P_{\mdp}) \right) \right].
    % &\in \left[J (P_{\mdp_1}, r) ~~\pm~~ \frac{\gamma (1 -f) R_{\max}}{(1 - \gamma)^2} D(P_{\mdp}, P_{\mdp_2}) ~\pm~ (1 - f) \frac{\gamma}{1 - \gamma}  \E_{\bs, \mathbf{a} \sim d^\pi_{\mdp_1}(\bs) \pi(\mathbf{a}|\bs)} \left[ \left(P^\pi_{\mdp} - P^\pi_{\mdp_1}\right) Q^\pi  \right] \right]. 
\end{align*}
To observe the third equality, we utilize the result on the difference between returns of a policy $\pi$ on two different MDPs, $P_{\mdp_1}$ and $P_{\mdp_f}$ from \citet{ajksbook} (Chapter 2, Lemma 2.2, Simulation Lemma), and additionally incorporate the auxiliary MDP $\mdp$ in the expression via addition and subtraction in the previous (second) step. In the fourth step, we finally bound one term that corresponds to the learned model via the total-variation divergence $D(P_{\mdp_2}, P_{\mdp})$ and the other term corresponding to the empirical MDP $\mdpbar$ is left in its expectation form to be bounded later. 

Using the above bounds on return for reward-mixtures and dynamics-mixtures, proving this lemma is straightforward:
\begin{align*}
    & J(\mdp_1, \mdp_2, f, \pi) := J(P_{\mdp_f}, f r_{\mdp_1} + (1 - f) r_{\mdp_2}) = J(f P_{\mdp_1} + (1 -f) P_{\mdp_2}, r_{\mdp_f})\\
    &\in \left[ J(P_{\mdp_f}, r_{\mdp}) ~\pm\right.\\
    &\left.~ \underbrace{\left(\frac{f}{1 - \gamma} \E_{\bs, \mathbf{a} \sim d^\pi_{\mdp} \pi}[|r_{\mdp_1}(\bs, \mathbf{a}) - r_{\mdp}(\bs, \mathbf{a})|] + \frac{1 - f}{1 - \gamma} \E_{\bs, \mathbf{a} \sim d^\pi_{\mdp} \pi}[|r_{\mdp_2}(\bs, \mathbf{a}) - r_{\mdp}(\bs, \mathbf{a})|] \right)}_{:= \Delta_R} \right],
    % ~\pm~ \left(\frac{2 \gamma f (1 - f)}{(1 - \gamma)^2} R_{\max} D \left(P_{\mdp_2}, P_{\mdp}\right) + \frac{2 \gamma f (1 - f)}{1 - \gamma} \E_{d^\pi_{\mdp_1}} \left\vert \left[ \left(P^\pi_{\mdp} - P^\pi_{\mdp_1}\right) Q^\pi  \right] \right\vert \right) \right],
\end{align*}
where the second step holds via linear decomposition of the return of $\pi$ in $\mdp_f$ with respect to the reward interpolation, and bounding the terms that appear in the reward difference. For convenience, we refer to these offset terms due to the reward as $\Delta_R$. For the final part of this proof, we bound $J(P_{\mdp_f}, r_{\mdp})$ in terms of the return on the actual MDP, $J(P_{\mdp}, r_{\mdp})$, using the inequality proved above that provides intervals for mixture dynamics but a fixed reward function. Thus, the overall bound is given by $J(\pi, \mdp_f) \in [J(\pi, \mdp) - \alpha, J(\pi, \mdp) + \alpha]$, where $\alpha$ is given by:
\begin{align}
\label{eqn:alpha_expr_repeat}
    \alpha = \frac{2 \gamma (1 - f)}{(1 - \gamma)^2} & R_{\max} D \left(P_{\mdp_2}, P_{\mdp}\right) + \frac{\gamma f}{1 - \gamma} \left\vert \E_{d^\pi_{\mdp} \pi} \left[ \left(P^\pi_{\mdp} - P^\pi_{\mdp_1}\right) Q^\pi_{\mdp} \right]\right\vert + \Delta_R.
\end{align}
This concludes the proof of this lemma.
\end{proof}



Finally, we prove Theorem~\ref{thm:policy_improvement} that shows how policy optimization with respect to $\hat{J}(f, \pi)$ affects the performance in the actual MD by using Equation~\ref{eqn:penalized_objective} and building on the  analysis of pure model-free algorithms from \citet{kumar2020conservative}. We restate a more complete statement of the theorem below and present the constants at the end of the proof. 

\begin{theorem}[Formal version of Proposition~\ref{thm:policy_improvement}]
Let $\hat{\pi}_{\text{out}}(\mathbf{a}|\bs)$ be the policy obtained by COMBO.
%%CF: Would be nice to have this definition outside of the theorem so that the theorem is shorter/simpler
Then, the policy ${\pi}_{\text{out}}(\mathbf{a}|\bs)$ is a $\zeta$-safe policy improvement over ${\behavior}$ in the actual MDP $\mdp$, i.e., $J({\pi}_{\text{out}}, \mdp) \geq J({\behavior}, \mdp) - \zeta$, with probability at least $1 - \delta$, where $\zeta$ is given by (where $\rho^\beta(\bs, \mathbf{a}) := d^\behavior_{\mdphat}(\bs, \mathbf{a})$):
\begin{align*}
&\mathcal{O}\left(\frac{\gamma f}{(1 - \gamma)^2}\right) {\left[ \E_{\bs \sim d^{\pi_{\text{out}}}_{\mdp}}\left[ \sqrt{\frac{|\actions|}{|\data(\bs)|} (\mathrm{D}_{\text{CQL}}({\pi}_{\text{out}}, \behavior) + 1)} \right] \right]}\\
&+ \mathcal{O}\left(\frac{\gamma (1 - f)}{(1 - \gamma)^2}\right) {\mathrm{D_{TV}}(P_{\mdp}, P_{\mdphat})} - \beta \frac{\nu(\rho^{\pi_{\text{out}}}, f) - \nu(\rho^\beta, f)}{(1 - \gamma)}.
    % &- \underbrace{\left({J}(\mdpbar, \mdphat, f, \pi) - {J}(\mdpbar, \mdphat, f, \behavior) \right)}_{:= (3),~~ \geq \beta \frac{\nu(\rho, f)}{(1 - \gamma)}} 
\end{align*}
\end{theorem}

\begin{proof}
We first note that since policy improvement is not being performed in the same MDP, $\mdp$ as the f-interpolant MDP, $\mdp_f$, we need to upper and lower bound the amount of improvement occurring in the actual MDP due to the f-interpolant MDP. As a result our first is to relate $J(\pi, \mdp)$ and $J(\pi, \mdp_f) := J(\mdpbar, \mdphat, f, \pi)$ for any given policy $\pi$.

\textbf{Step 1: Bounding the return in the actual MDP due to optimization in the f-interpolant MDP.} By directly applying Lemma~\ref{lemma:interpolant_regular_bound} stated and proved previously, we obtain the following upper and lower-bounds on the return of a policy $\pi$:
\begin{equation*}
    J(\mdpbar, \mdphat, f, \pi) \in \left[ J(\pi, \mdp) - \alpha,~~ J(\pi, \mdp) + \alpha \right],
\end{equation*}
where $\alpha$ is shown in Equation~\ref{eqn:alpha_expr}. As a result, we just need to bound the terms appearing the expression of $\alpha$ to obtain a bound on the return differences. We first note that the terms in the expression for $\alpha$ are of two types: \textbf{(1)} terms that depend only on the reward function differences (captured in $\Delta_R$ in Equation~\ref{eqn:alpha_expr_repeat}), and \textbf{(2)} terms that depend on the dynamics (the other two terms in Equation~\ref{eqn:alpha_expr_repeat}). 

To bound $\Delta_R$, we simply appeal to concentration inequalities on reward (Assumption~\ref{assumption:conc}), and bound $\Delta_R$ as:
\begin{align*}
\Delta_R &:= \frac{f}{1 - \gamma} \E_{\bs, \mathbf{a} \sim d^\pi_{\mdp} \pi}[|r_{\mdp_1}(\bs, \mathbf{a}) - r_{\mdp}(\bs, \mathbf{a})|] + \frac{1 - f}{1 - \gamma} \E_{\bs, \mathbf{a} \sim d^\pi_{\mdp} \pi}[|r_{\mdp_2}(\bs, \mathbf{a}) - r_{\mdp}(\bs, \mathbf{a})|]\\
&\leq \frac{C_{r, \delta}}{1 - \gamma} \E_{\bs, \mathbf{a} \sim d^\pi_{\mdp}\pi} \left[\frac{1}{\sqrt{D(\bs, \mathbf{a})}}\right] + \frac{1}{1 - \gamma} ||R_{\mdp} - R_{\mdphat}|| := \Delta_R^u.
\end{align*}
Note that both of these terms are of the order of $\mathcal{O}(1/ (1 - \gamma))$ and hence they don't figure in the informal bound in Theorem~\ref{thm:policy_improvement} in the main text, as these are dominated by terms that grow quadratically with the horizon.
% First, we use algebraic manipulation to obtain the following decompositionof the difference in the return of $\pi_{\text{out}}$ and $\pi_\beta$ in the actual MDP, $\mdp$:
% \begin{align*}
%     J(\pi_{\text{out}}, \mdp) - J(\behavior, \mdp) &= f \left(J(\pi_{\text{out}}, \mdp) - J(\pi_{\text{out}}, \mdpbar) \right) + (1 - f) \left(J(\pi_{\text{out}}, \mdp) - J(\pi_{\text{out}}, \mdphat) \right)~~~ \text{(a): policy difference}\\
%     &+ f (J(\pi_{\text{out}}, \mdpbar) - J(\behavior, \mdpbar)) + (1 - f) \left(J(\pi_{\text{out}}, \mdphat) - J(\behavior, \mdphat) \right)~~~\text{(b): policy improvement} \\
%     &+ f \left(J(\behavior, \mdpbar) - J(\behavior, \mdp) \right) + (1 - f) \left(J(\behavior, \mdphat) - J(\behavior, \mdp) \right)~~~~~~ \text{(c): behavior difference}
% \end{align*}
% Terms (a) and (c) correspond to a weighted sum of the difference in the return estimates of the policies in the empirical MDP $\mdpbar$ and the actual MDP $\mdp$ and the model-induced MDP $\mdphat$, and the actual MDP $\mdp$. 
To bound the remaining terms in the expression for $\alpha$, we utilize a result directly from \citet{kumar2020conservative} for the empirical MDP, $\mdpbar$, which holds for any policy $\pi(\mathbf{a}|\bs)$, as shown below.
\begin{align*}
   &\frac{\gamma}{(1 - \gamma)} \left\vert \E_{\bs, \mathbf{a} \sim d^\pi_{\mdp}(\bs) \pi(\mathbf{a}|\bs)}\left[ \left(P^\pi_{\mdp} - P^\pi_{\mdp_1}\right) Q^\pi_{\mdp} \right] \right\vert \\
   &\leq \frac{2 \gamma R_{\max} C_{P, \delta}}{(1 - \gamma)^2} \mathbb{E}_{\bs \sim d^{\policy}_{\mdpbar}(\bs)}\left[ \frac{\sqrt{|\mathcal{A}|}}{\sqrt{|\mathcal{D}(\bs)|}} \sqrt{ D_{\text{CQL}}(\policy, \behavior)(\bs) + 1} \right].
    %%AK: technically I think this is a naive result and certainly the MOReL paper was not the first one to come up with this... so unclear if we should be citing it for this result...
\end{align*}

\textbf{Step 2: Incorporate policy improvement in the f-inrerpolant MDP.} Now we incorporate the improvement of policy $\pi_{\text{out}}$ over the policy $\behavior$ on a weighted mixture of $\mdphat$ and $\mdpbar$. In what follows, we derive a lower-bound on this improvement by using the fact that policy $\pi_{\text{out}}$ is obtained by maximizing $\hat{J}(f, \pi)$ from Equation~\ref{eqn:penalized_objective}. As a direct consequence of Equation~\ref{eqn:penalized_objective}, we note that 
\begin{equation}
\label{eqn:improvement_expanded}
    \hat{J}(f, \pi_{\text{out}}) =  J(\mdpbar, \mdphat, f, \pi_{\text{out}}) - \beta \frac{\nu(\rho^\pi, f)}{1 - \gamma} \geq \hat{J}(f, \behavior) =  J(\mdpbar, \mdphat, f, \behavior) - \beta {\frac{\nu(\rho^\beta, f)}{1 - \gamma}}
\end{equation}
% Now, observe that we can both upper and lower-bound $J(\mdpbar, \mdphat, f, \pi)$ in terms of the return of policy $\pi$, individually in each MDP, $\mdpbar$ and $\mdphat$. We state this result more formally in Lemma~\ref{lemma:interpolant_regular_bound}.

% Next, we will use the upper bound on $J(\mdpbar, \mdphat, f, \pi)$ from Lemma~\ref{lemma:interpolant_regular_bound} for policy $\pi = \pi_{\text{out}}$ and a lower-bound on $J(\mdpbar, \mdphat, f, \pi)$ for policy $\pi = \behavior$, in the case when the auxiliary MDP is given by $\mdp$ (the actual MDP) to replace the expressions for $J(\mdpbar, \mdphat, f, \pi_{\text{out}})$ and $J(\mdpbar, \mdphat, f, \behavior)$ in the improvement equation~\ref{eqn:improvement_expanded}. Thus using Lemma~\ref{lemma:interpolant_regular_bound} we obtain the following inequality:
Following \textbf{Step 1}, we will use the upper bound on $J(\mdpbar, \mdphat, f, \pi)$ for policy $\pi = \pi_{\text{out}}$ and a lower-bound on $J(\mdpbar, \mdphat, f, \pi)$ for policy $\pi = \behavior$ and obtain the following inequality:
\begin{align*}
    J(\pi_{\text{out}}, \mdp) - \beta \frac{\nu(\rho^\pi, f)}{1 - \gamma} ~&\geq~ \Big\{ J(\behavior, \mdp) - \beta \frac{\nu(\rho^\beta, f)}{1 - \gamma}
    - \frac{4 \gamma (1 - f) R_{\max}}{(1 - \gamma)^2} D(P_{\mdp}, P_{\mdphat}) \\ 
    &- \underbrace{\frac{2 \gamma f}{(1 - \gamma)}\left\vert\E_{d^{\pi_{\text{out}}}_{\mdp}} \left[ \left(P^{\pi_{\text{out}}}_{\mdp} - P^{\pi_{\text{out}}}_{\mdpbar}\right) Q^{\pi_{\text{out}}}_{\mdp}  \right] \right\vert}_{:= (*)}\nonumber\\
    &- \underbrace{\frac{4 \gamma R_{\max} C_{P, \delta} f}{(1 - \gamma)^2} \E_{\bs \sim d^\behavior_{\mdp}}\left[ \sqrt{\frac{|\actions|}{|\data(\bs)|}}\right]}_{:= (\wedge)} - \Delta_R^u \Big\}.
\end{align*}
The term marked by $(*)$ in the above expression can be upper bounded by the concentration properties of the dynamics as done in Step 1 in this proof: 
\begin{align}
\label{eqn:bound_mdp_mdphat}
    (*) \leq \frac{4 \gamma f C_{P, \delta} R_{\max}}{(1 - \gamma)^2} \mathbb{E}_{\bs \sim d^{{\pi_{\text{out}}}}_{\mdp}(\bs)}\left[ \frac{\sqrt{|\mathcal{A}|}}{\sqrt{|\mathcal{D}(\bs)|}} \sqrt{ D_{\text{CQL}}({\pi_{\text{out}}}, \behavior)(\bs) + 1} \right]. 
\end{align}
Finally, using Equation~\ref{eqn:bound_mdp_mdphat}, we can lower-bound the policy return difference as:
\begin{align*}
\begin{small}
    J(\pi_{\text{out}}, \mdp) - J(\behavior, \mdp) \geq \beta \frac{\nu(\rho^\pi, f)}{1 - \gamma} - \beta \frac{\nu(\rho^\beta, f)}{1 - \gamma} - \frac{4 \gamma (1 -f) R_{\max}}{(1 - \gamma)^2} D(P_{\mdp}, P_{\mdphat}) - (*) - \Delta_R^u.
\end{small}
\end{align*}
Plugging the bounds for terms (a), (b) and (c) in the expression for $\zeta$ where $J(\pi_{\text{out}}, \mdp) - J(\behavior, \mdp) \geq \zeta$, we obtain:
\begin{align}
\zeta &= \left({\frac{4f \gamma R_{\max} C_{P, \delta}}{(1 - \gamma)^2}} \right)\mathbb{E}_{\bs \sim d^{\policy_{\text{out}}}_{\mdp}(\bs)}\left[ \frac{\sqrt{|\mathcal{A}|}}{\sqrt{|\mathcal{D}(\bs)|}} \sqrt{ D_{\text{CQL}}(\policy_{\text{out}}, \behavior)(\bs) + 1} \right]  + (\wedge) - \Delta_R^u \nonumber\\
\label{eqn:zeta_expression}
&~~~~~~~~~~~~+ \frac{4 (1 -f) \gamma R_{\max}}{(1 - \gamma)^2} D(P_{\mdp}, P_{\mdphat}) - \beta \frac{\nu(\rho^\pi, f)}{1 - \gamma} + \beta \frac{\nu(\rho^\beta, f)}{1 - \gamma}.
\end{align}
\end{proof}

\begin{remark}[\underline{\textbf{Interpretation of Proposition~\ref{thm:policy_improvement}}}] 
\label{remark:remark1}
Now we will interpret the theoretical expression for $\zeta$ in Equation~\ref{eqn:zeta_expression}, and discuss the scenarios when it is \emph{negative}. When the expression for $\zeta$ is negative, the policy $\pi_{\text{out}}$ is an improvement over $\behavior$ in the original MDP, $\mdp$. 

\begin{itemize}
    \item First note that we have never used the fact that the learned model $P_{\mdphat}$ is close to the actual MDP, $P_{\mdp}$ on the states visited by the behavior policy $\behavior$ in our analysis. We will use this fact now: in practical scenarios, $\nu(\rho^\beta, f)$ is expected to be smaller than $\nu(\rho^\pi, f)$, since $\nu(\rho^\beta, f)$ is directly controlled by the difference and density ratio of $\rho^\beta(\bs, \mathbf{a})$ and $d(\bs, \mathbf{a})$: $\nu(\rho^\beta, f) \leq \nu(\rho^\beta, f=1) = \sum_{\bs, \mathbf{a}} d^\behavior_{\mdphat}(\bs, \mathbf{a}) \left(d^\behavior_{\mdphat}(\bs, \mathbf{a})/d^\behavior_{\mdpbar}(\bs, \mathbf{a}) - 1\right)^2$ by Lemma~\ref{thm:line_thm} which is expected to be small for the behavior policy $\behavior$ in cases when the behavior policy marginal in the empirical MDP, $d^\behavior_{\mdpbar}(\bs, \mathbf{a})$, is broad. This is a direct consequence of the fact that the learned dynamics integrated with the policy under the learned model: $P_{\mdphat}^\behavior$ is closer to its counterpart in the empirical MDP:  $P_{\mdpbar}^\behavior$ for $\behavior$. Note that this is not true for any other policy besides the behavior policy that performs several counterfactual actions in a rollout and deviates from the data. For such a learned policy $\pi$, we incur an extra error which depends on the importance ratio of policy densities, compounded over the horizon and manifests as the $D_{\mathrm{CQL}}$ term (similar to Equation~\ref{eqn:bound_mdp_mdphat}, or Lemma D.4.1 in \citet{kumar2020conservative}). Thus, in practice, we argue that we are interested in situations where $\nu(\rho^\pi, f) > \nu(\rho^\beta, f)$, in which case by increasing $\beta$, we can make the expression for $\zeta$ in Equation~\ref{eqn:zeta_expression} negative, allowing for policy improvement.
    \item In addition, note that when $f$ is close to 1, the bound reverts to a standard model-free policy improvement bound and when $f$ is close to 0, the bound reverts to a typical model-based policy improvement bound. In scenarios with high sampling error (i.e. smaller $|\mathcal{D}(\bs)|$), if we can learn a good model, i.e., $D(P_{\mdp}, P_{\mdphat})$ is small, we can attain policy improvement better than model-free methods by relying on the learned model by setting $f$ closer to 0. A similar argument can be made in reverse for handling cases when learning an accurate dynamics model is hard. 
\end{itemize}
\end{remark}

% \begin{theorem}[Upper bound on $\nu(\rho, f)$]
% If the distributions $\rho(\bs, \mathbf{a})$ and $d(\bs, \mathbf{a})$ are such that $\sum_{\bs, \mathbf{a}} (\rho(\bs, \mathbf{a}) - d(\bs, \mathbf{a}))^2 \leq \varepsilon$, then the value of $\nu(\rho, f) \leq $. 
% \end{theorem}
% \begin{proof}
% To obtain a bound on $\nu(\rho, f)$, we solve the following optimization problem over $\rho$:
% \begin{align*}
%     \max_{\rho}&~~~ \nu(\rho, f):= \sum_{\bs, \mathbf{a}} \rho(\bs, \mathbf{a}) \frac{\rho(\bs, \mathbf{a}) - d(\bs, \mathbf{a})}{f d (\bs, \mathbf{a}) + (1 - f) \rho(\bs, \mathbf{a})}\\
%     &\text{s.t.}~~ \sum_{\bs, \mathbf{a}} (\rho(\bs, \mathbf{a}) - d(\bs, \mathbf{a}))^2 \leq \varepsilon, ~~ \sum_{\bs, \mathbf{a}} \rho(\bs, \mathbf{a}) = 1, ~~ \rho(\bs, \mathbf{a}) \geq 0.
% \end{align*}
% We first note that for any optimal $\rho=\rho^*$, the objective value is largest for $f = 1$, and thus, solving the above optimization problem for $f=1$ gives an upper bound on the objective value. Converting the problem for $f=1$ to a minimization problem and writing out the Lagrangian for optimization, we obtain:
% \begin{multline}
%     \mathcal{L}(\rho; \lambda, \alpha, \eta) = -\sum_{\bs, \mathbf{a}} d(\bs, \mathbf{a}) \frac{\rho(\bs, \mathbf{a})}{d(\bs, \mathbf{a})}  \left( \frac{\rho(\bs, \mathbf{a})}{d(\bs, \mathbf{a})} - 1 \right) + \lambda \left(\sum_{\bs, \mathbf{a}} d(\bs, \mathbf{a})^2 \left(\frac{\rho(\bs, \mathbf{a})}{d(\bs, \mathbf{a})} - 1 \right)^2 - \varepsilon \right) \\ - \eta \left(\sum_{\bs, \mathbf{a}} d(\bs, \mathbf{a}) \frac{\rho(\bs, \mathbf{a})}{d(\bs, \mathbf{a})} - 1 \right) - \sum_{\bs, \mathbf{a}} \alpha(\bs, \mathbf{a}) \frac{\rho(\bs, \mathbf{a})}{d(\bs, \mathbf{a})}.
% \end{multline}
% Noting the change of variable transformation: $w(\bs, \mathbf{a}) := \frac{\rho(\bs, \mathbf{a})}{d(\bs, \mathbf{a})} - 1$, we obtain the following optimization problem:
% \begin{equation*}
%     \mathcal{L}(w; \lambda, \alpha, \eta) = -\sum_{\bs, \mathbf{a}} d(\bs, \mathbf{a}) w(\bs, \mathbf{a})^2 + \lambda \left( \sum_{\bs, \mathbf{a}} d(\bs, \mathbf{a})^2 w(\bs, \mathbf{a})^2 - \varepsilon \right) - \eta \sum_{\bs, \mathbf{a}} d(\bs, \mathbf{a}) w(\bs, \mathbf{a}) - \sum_{\bs, \mathbf{a}} \alpha(\bs, \mathbf{a}) (w(\bs, \mathbf{a}) + 1).
% \end{equation*}
% Taking the derivative with respect to $w(\bs, \mathbf{a})$ and utilizing KKT conditions we obtain
% \begin{align}
% &- 2 d(\bs, \mathbf{a}) w(\bs, \mathbf{a}) + 2 \lambda d(\bs, \mathbf{a})^2 w(\bs, \mathbf{a}) - \eta d(\bs, \mathbf{a}) - \alpha(\bs, \mathbf{a}) = 0   \label{eq:grad}\\
% &\lambda \left( \sum_{\bs, \mathbf{a}} d(\bs, \mathbf{a})^2 w(\bs, \mathbf{a})^2 - \varepsilon \right) = 0.  \label{eq:slack1}\\
% & \alpha(\bs, \mathbf{a}) (w(\bs, \mathbf{a}) + 1) = 0 ~~ \forall \bs, \mathbf{a}.  \label{eq:slack2}
% \end{align}
% Multiplying Equation~\ref{eq:grad} by $w(\bs, \mathbf{a})$ and adding both LHS and RHS over $(\bs, \mathbf{a})$ we obtain:
% \begin{equation}
%     \label{eq:temp_add}
%     - 2 \sum_{\bs, \mathbf{a}} d(\bs, \mathbf{a}) w(\bs, \mathbf{a})^2 + 2 \underbrace{\lambda \sum_{\bs, \mathbf{a}} d(\bs, \mathbf{a})^2 w(\bs, \mathbf{a})^2}_{= \lambda \varepsilon} - \underbrace{\eta \sum_{\bs, \mathbf{a}} d(\bs, \mathbf{a}) w(\bs, \mathbf{a})}_{= \eta \times 0 = 0} = \sum_{\bs, \mathbf{a}} \alpha(\bs, \mathbf{a}) w(\bs, \mathbf{a}),
% \end{equation}
% and similarly, adding Equation~\ref{eq:grad} over $(\bs, \mathbf{a})$ we get:
% \begin{equation}
%     \label{eq:simple_add}
%     - 2 \underbrace{\sum_{\bs, \mathbf{a}} d(\bs, \mathbf{a}) w(\bs, \mathbf{a})}_{= 0} + 2 \lambda \sum_{\bs, \mathbf{a}} d(\bs, \mathbf{a})^2 w(\bs, \mathbf{a}) - \eta = \sum_{\bs, \mathbf{a}} \alpha(\bs, \mathbf{a}).
% \end{equation}
% Finally, from Equation~\ref{eq:grad}, we get that the value of $w(\bs, \mathbf{a})$ is given by:
% \begin{equation*}
%     w(\bs, \mathbf{a}) = \frac{\eta d(\bs, \mathbf{a}) + \alpha (\bs, \mathbf{a})}{2 \lambda d(\bs, \mathbf{a})^2 - 2 d(\bs, \mathbf{a})}
% \end{equation*}
% Adding Equations~\ref{eq:temp_add} and \ref{eq:simple_add}, we obtain:
% \begin{equation*}
%     2 \sum_{\bs, \mathbf{a}} d(\bs, \mathbf{a}) w(\bs, \mathbf{a}) \left[\lambda d(\bs, \mathbf{a}) - w(\bs, \mathbf{a}) \right] + \lambda \varepsilon - \eta = 0. 
% \end{equation*}
% \end{proof}

\section{Experimental Details for COMBO}
\label{app:details}

In this section, we include all details of our empirical evaluations of COMBO.

\subsection{Practical algorithm implementation details}
\label{app:combo_details}

\paragraph{Model training.}

In the setting where the observation space is low-dimensional, as mentioned in Section~\ref{sec:combo},  we represent the model as a probabilistic neural network that outputs a Gaussian distribution over the next state and reward given the current state and action: $$\widehat{T}_\theta(\bs_{t+1}, r| \bs, \mathbf{a}) = \mathcal{N}(\mu_\theta(\bs_t, \mathbf{a}_t), \Sigma_\theta(\bs_t, \mathbf{a}_t)).$$ We train an ensemble of $7$ such dynamics models following \cite{janner2019mbpo} and pick the best $5$ models based on the validation prediction error on a held-out set that contains $1000$ transitions in the offline dataset $\data$. During model rollouts, we randomly pick one dynamics model from the best $5$ models. Each model in the ensemble is represented as a 4-layer feedforward neural network with $200$ hidden units. For the generalization experiments in Section~\ref{sec:generalization_exps}, we additionally use a two-head architecture to output the mean and variance after the last hidden layer following \cite{yu2020mopo}.

In the image-based setting, we follow \citet{Rafailov2020LOMPO} and use a variational model with the following components:

\begin{gather}
\begin{aligned}
&\text{Image encoder:} && \mathbf{h}_t=E_\theta(\bo_t) \\
&\text{Inference model:} && \bs_t \sim q_\theta(\bs_t|\mathbf{h}_t, \bs_{t-1}, \mathbf{a}_{t-1})\\
&\text{Latent transition model:} &&\bs_t \sim \widehat{T}_\theta(\bs_t| \bs_{t-1}, \mathbf{a}_{t-1})\\
&\text{Reward predictor:} && r_t \sim p_\theta(r_t|\bs_t) \\
&\text{Image decoder:} && \bo_t \sim D_\theta(\bo_t|\bs_t).
\label{eq:latent_model}
\end{aligned}
\end{gather}%

We train the model using the evidence lower bound:

$$\max_{\theta}\sum_{\tau=0}^{T-1}\Big[\mathbb{E}_{q_{\theta}}[\log D_{\theta}(\bo_{\tau+1}|\bs_{\tau+1})]\Big]-\mathbb{E}_{q_{\theta}}\Big[D_{KL}[q_{\theta}(\bo_{\tau+1}, \bs_{\tau+1}|\bs_{\tau}, \mathbf{a}_{\tau})\|\widehat{T}_{\theta_{\tau}}(\bs_{\tau+1}, a_{\tau+1})]\Big]$$

At each step $\tau$ we sample a latent forward model $\widehat{T}_{\theta_{\tau}}$ from a fixed set of $K$ models $[\widehat{T}_{\theta_1},\ldots, \widehat{T}_{\theta_K}]$. For the encoder $E_{\theta}$ we use a convolutional neural network with kernel size 4 and stride 2. For the Walker environment we use 4 layers, while the Door Opening task has 5 layers. The $D_{\theta}$ is a transposed convolutional network with stride 2 and kernel sizes $[5,5,6,6]$ and $[5,5,5,6,6]$ respectively. The inference network has a two-level structure similar to \citet{Hafner2019PlanNet} with a deterministic path using a GRU cell with 256 units and a stochastic path implemented as a conditional diagonal Gaussian with 128 units. We only train an ensemble of stochastic forward models, which are also implemented as conditional diagonal Gaussians.


\paragraph{Policy Optimization.} We sample a batch size of $256$ transitions for the critic and policy learning. We set $f = 0.5$, which means we sample $50\%$ of the batch of transitions from $\data$ and another $50\%$ from $\data_\text{model}$. The equal split between the offline data and the model rollouts strikes the balance between conservatism and generalization in our experiments as shown in our experimental results in Section~\ref{sec:combo_exp}. We represent the Q-networks and policy as 3-layer feedforward neural networks with $256$ hidden units.

For the choice of $\rho(\bs,\mathbf{a})$ in Equation~\ref{eq:implicit_update}, we can obtain the Q-values that lower-bound the true value of the learned policy $\pi$ by setting $\rho(\bs,\mathbf{a}) = d^\policy_{\mdphat} (\bs) \pi(\mathbf{a} | \bs)$. However, as discussed in \cite{kumar2020conservative}, computing $\pi$ by alternating the full off-policy evaluation for the policy $\hat{\pi}^k$ at each iteration $k$ and one step of policy improvement is computationally expensive. Instead, following \cite{kumar2020conservative}, we pick a particular distribution $\psi(\mathbf{a}|\bs)$ that approximates the the policy that maximizes the Q-function at the current iteration and set $\rho(\bs,\mathbf{a}) = d^\policy_{\mdphat} (\bs) \psi(\mathbf{a} | \bs)$. We formulate the new objective as follows:
\begin{small}
\begin{align}
    \hat{Q}^{k+1} \leftarrow& \arg\min_{Q}\beta\left(\E_{\bs \sim d^\policy_{\mdphat} (\bs), \mathbf{a}\sim \psi(\mathbf{a} | \bs)}\!\left[Q(\bs,\mathbf{a})\right]-\E_{\bs, \mathbf{a} \sim \data}\left[Q(\bs,\mathbf{a})\right]\right)\nonumber\\
    &+ \frac{1}{2}\E_{\bs, \mathbf{a}, \bs' \sim d_f}\left[ \left(Q(\bs, \mathbf{a}) - \widehat{\bellman}^\policy\hat{Q}^k(\bs, \mathbf{a}))\right)^2 \right] + \mathcal{R}(\psi),
    \label{eq:combo_update_practical}
\end{align}
\end{small}
where $\mathcal{R}(\psi)$ is a regularizer on $\psi$. In practice, we pick $\mathcal{R}(\psi)$ to be the $-D_\text{KL}(\psi(\mathbf{a}|\bs)\|\text{Unif}(\mathbf{a}))$ and under such a regularization, the first term in Equation~\ref{eq:combo_update_practical} corresponds to computing softmax of the Q-values at any state $\bs$ as follows:
\begin{small}
\begin{align}
    \hat{Q}^{k+1} \leftarrow& \arg\min_{Q}\max_\psi\beta\left(\E_{\bs \sim d^\policy_{\mdphat} (\bs)}\!\left[\log\sum_\mathbf{a} Q(\bs,\mathbf{a})\right]-\E_{\bs, \mathbf{a} \sim \data}\left[Q(\bs,\mathbf{a})\right]\right) \nonumber\\
    &+ \frac{1}{2}\E_{\bs, \mathbf{a}, \bs' \sim d_f}\left[ \left(Q(\bs, \mathbf{a}) - \widehat{\bellman}^\policy\hat{Q}^k(\bs, \mathbf{a}))\right)^2 \right].
    \label{eq:combo_logsumexp}
\end{align}
\end{small}
We estimate the \texttt{log-sum-exp} term in Equation~\ref{eq:combo_logsumexp} by sampling $10$ actions at every state $\bs$ in the batch from a uniform policy $\text{Unif}(\mathbf{a})$ and the current learned policy $\pi(\mathbf{a}|\bs)$ with importance sampling following \cite{kumar2020conservative}.

\subsection{Hyperparameter Selection for COMBO}
\label{app:hyperparameter}

\neurips{In this section, we discuss the hyperparameters that we use for COMBO. In the D4RL and generalization experiments, our method are built upon the implementation of MOPO provided at: \url{https://github.com/tianheyu927/mopo}. The hyperparameters used in COMBO that relates to the backbone RL algorithm SAC such as twin Q-functions and number of gradient steps follow from those used in MOPO with the exception of smaller critic and policy learning rates, which we will discuss below. In the image-based domains, COMBO is built upon LOMPO without any changes to the parameters used there. For the evaluation of COMBO, we follow the evaluation protocol in D4RL~\citep{fu2020d4rl} and a variety of prior offline RL works~\citep{kumar2020conservative,yu2020mopo,kidambi2020morel} and report the normalized score of the smooth undiscounted averaged return over $3$ random seeds for all environments except \texttt{sawyer-door-close} and \texttt{sawyer-door} where we report the average success rate over $3$ random seeds.}

\neurips{We now list the additional hyperparameters as follows.
\begin{itemize}
    \item \textbf{Rollout length $h$.} We perform a short-horizon model rollouts in COMBO similar to \citet{yu2020mopo} and \citet{Rafailov2020LOMPO}. For the D4RL experiments and generalization experiments, we followed the defaults used in MOPO and used $h = 1$ for walker2d and \texttt{sawyer-door-close}, $h=5$ for hopper, halfcheetah and \texttt{halfcheetah-jump}, and $h=25$ for \texttt{ant-angle}. In the image-based domain we used rollout length of $h=5$ for both the the \texttt{walker-walk} and \texttt{sawyer-door-open} environments following the same hyperparameters used in \citet{Rafailov2020LOMPO}.
    \item \textbf{Q-function and policy learning rates.} On state-based domains, we searched over $\{1e-4, 3e-4\}$ for the Q-function learning rate and $\{1e-5, 3e-5, 1e-4\}$ for the policy learning rate. 
    We found that $3e-4$ for the Q-function learning rate (also used previously in \citet{kumar2020conservative}) and $1e-4$ for the policy learning rate (also recommended previously in \citet{kumar2020conservative} for gym domains) work well for almost all domains except that on walker2d where a smaller Q-function learning rate of $1e-4$ and a correspondingly smaller policy learning rate of $1e-5$ works the best. In the image-based domains, we followed the defaults from prior work \citep{Rafailov2020LOMPO} and used $3e-4$ for both the policy and Q-function.
    
    \item \textbf{Conservative coefficient $\beta$.} 
    % As noted in our theoretical results in Lemma~\ref{thm:line_thm}, the amount of conservatism depends on the choice of fraction $f$ and $\rho(\bs, \mathbf{a})$. In principle, we only need to control one of these factors, $\rho$, $f$, $\beta$ to obtain the right degree of conservatism. Since we do not alter $f$ and $\rho(\bs, \mathbf{a})$ for different quality datasets (see Appendix~\ref{app:combo_details} for our choice of $f$; $\rho$ was chosen based on model-prediction error as discussed next) we instead choose values of $\beta$ for different dataset types.
    We searched over $\{0.5, 1.0, 5.0\}$ for $\beta$, which correspond to low conservatism, medium conservatism and high conservatism.  A larger $\beta$ would be desirable in more narrow dataset distributions with lower-coverage of the state-action space that propagates error in a backup whereas a smaller $\beta$ is desirable with diverse dataset distributions. On the D4RL experiments, we found that $\beta = 0.5$ works well for halfcheetah agnostic of dataset quality, while on hopper and walker2d, we found that the more ``narrow'' dataset distributions: medium and medium-expert datasets work best with larger $\beta = 5.0$ whereas more ``diverse'' dataset distributions: random and medium-replay datasets work best with smaller $\beta$ ($\beta = 0.5$ for walker2d and $\beta = 1.0$ for hopper) which is consistent with the intuition. 
    % An intuitive explanation would be that on medium and medium-expert datasets where the data distribution is narrow, we need to be more conservative and hence large $\beta$ while on random and medium-replay datasets where the distribution is diverse and cover most of the state space, we require less conservatism. 
    On generalization experiments, $\beta = 1.0$ works best for all environments. In the image-domains we use $\beta=0.5$ for the medium-replay \texttt{walker-walk} task and and $\beta=1.0$ for all other domains, which again is in accordance with the impact of $\beta$ on performance.
    
    
    \item \textbf{Choice of $\rho(\bs,\mathbf{a})$.} We first decouple $\rho(\bs,\mathbf{a}) = \rho(\bs)\rho(\mathbf{a}|\bs)$ for convenience. As discussed in Appendix~\ref{app:combo_details}, we use $\rho(\mathbf{a}|\bs)$ as the soft-maximum of the Q-values and estimated with \texttt{log-sum-exp}. For $\rho(\bs)$, we searched over $\{d^\policy_{\mdphat}, \rho(\bs)=d_f\}$.  We found that $d^\policy_{\mdphat}$ works better the hopper task in D4RL while $d_f$ is better for the rest of the environments. For the remaining domains, we found $\rho(\bs)=d_f$ works well.
    
    
    \item \textbf{Choice of $\mu(\mathbf{a}|\bs)$.} For the rollout policy $\mu$, we searched $\{\text{Unif}(\mathbf{a}), \pi(\mathbf{a}|\bs)\}$, i.e. the set that contains a random policy and a current learned policy. We found that $\mu(\mathbf{a}|\bs) = \text{Unif}(\mathbf{a})$ works well on the hopper task in D4RL and also in the $\texttt{ant-angle}$ generalization experiment. For the remaining state-based environments, we discovered that $\mu(\mathbf{a}|\bs) = \pi(\mathbf{a}|\bs)$ excels. In the image-based domain, we found that $\mu(\mathbf{a}|\bs) = \text{Unif}(\mathbf{a})$ works well in the \texttt{walker-walk} domain and  $\mu(\mathbf{a}|\bs) = \pi(\mathbf{a}|\bs)$ is better for the \texttt{sawyer-door} environment. 
    % Similar to the choice of $\rho(\bs)$, 
    We observed that
    $\mu(\mathbf{a}|\bs) = \text{Unif}(\mathbf{a})$ behaves less conservatively and is suitable to tasks where dynamics models can be learned fairly precisely.
    \item \textbf{Choice of Backup.} Following CQL~\citep{kumar2020conservative}, we use the standard deterministic backup for COMBO.
    \item \textbf{Choice of $f$.} For the ratio between model rollouts and offline data $f$, we searched $\{0.5, 0.8\}$. We found that $f = 0.8$ works well on the medium and medium-expert in the walker2d task in D4RL. For the remaining tasks, we find $f = 0.5$ works well.
\end{itemize}}

\subsection{Automatic Hyperparameter Selection Rule for COMBO}

It is common in prior work on offline RL to select various hyperparameters using online policy rollouts~\citep{yu2020mopo,kidambi2020morel,argenson2020model,lee2021representation}. Requiring online rollouts to tune hyperparameters contradicts the main aim of offline RL, which is to learn entirely from offline data. Therefore, we attempted to devise an automated rule for tuning important hyperparameters such as $\beta$ and $f$ in a fully offline manner in COMBO. We search over a discrete set of hyperparameters for each task as dicussed above, and use the value of the regularization term $\mathbb{E}_{\mathbf{s}, \mathbf{a} \sim \rho(\mathbf{s},\mathbf{a})}\!\left[Q(\mathbf{s},\mathbf{a})\right]\!-\!\mathbb{E}_{\mathbf{s}, \mathbf{a} \sim \data}\!\left[Q(\mathbf{s},\mathbf{a})\right]$ (shown in Eq.~\ref{eq:implicit_update}) to evaluate the hyperparameters. This automated rule picks the hyperparameter setting which achieves the lowest regularization objective, which indicates that the Q-values on unseen model-predicted state-action tuples are not overestimated.
%%CF.9.30: The ICLR AC also wanted to see a discussion of how this offline selection scheme compared to prior methods for offline selection. Maybe discuss this somewhere? (perhaps in the appendix if space is short?)
%%TY.10.1: I discussed this in Appendix B.2.
%%SL.10.2: I slightly rephrased the paragraph above in a way that hopefully further avoids potential misunderstandings.

Below, we provide additional experimental validation showing the effiacy of this automatic hyperparameter selection rule from above. As shown in Table~\ref{tab:beta_selection},~\ref{tab:mu_selection}, ~\ref{tab:rho_selection} and \ref{tab:f_selection}, the above proposed automatic hyperparameter selection rule is able to pick the hyperparameters $\beta$, $\mu(\mathbf{a}|\bs)$, $\rho(\bs)$ and $f$ and  that correspond to the best policy performance based on the regularization value.

\begin{table}[ht]
    \centering
    \scriptsize
    \resizebox{1.0\textwidth}{!}{\begin{tabular}{l|r|r|r|r|}
    \toprule
    Task & $\beta=0.5$ & $\beta=0.5$ & $\beta=5.0$ & $\beta=5.0$\\
 & performance & regularizer value & performance & regularizer value\\
 \midrule
halfcheetah-medium &  \textbf{54.2}  & \textbf{-778.6}  & 40.8  & -236.8  \\
halfcheetah-medium-replay &  \textbf{55.1} & \textbf{28.9} & 9.3 & 283.9\\ 
halfcheetah-medium-expert & 89.4 & 189.8 & \textbf{90.0}  & \textbf{6.5}\\
hopper-medium      &  75.0  & -740.7  &\textbf{97.2}  & \textbf{-2035.9}\\
hopper-medium-replay & \textbf{89.5} & \textbf{37.7} & 28.3       & 107.2\\
hopper-medium-expert & \textbf{111.1}       & \textbf{-705.6}    & 75.3 &       -64.1\\
walker2d-medium        &  1.9  & 51.5  & \textbf{81.9}  & \textbf{-1991.2}\\
walker2d-medium-replay & \textbf{56.0}       & \textbf{-157.9}    & 27.0       & 53.6\\
walker2d-medium-expert & 10.3       & -788.3    &\textbf{103.3}       & \textbf{-3891.4}\\
    \bottomrule
\end{tabular}}
\caption{\footnotesize We include our automatic hyperparameter selection rule of $\beta$ on a set of representative D4RL environments. We show the policy performance (bold with the higher number) and the regularizer value (bold with the lower number). Lower regularizer value consistently corresponds to the higher policy return, suggesting the effectiveness of our automatic selection rule.}
\label{tab:beta_selection}
\end{table}

\begin{table}[ht]
    \centering
    \scriptsize
    \resizebox{1.0\textwidth}{!}{\begin{tabular}{l|r|r|r|r|}
    \toprule
    Task & $\mu(\mathbf{a}|\bs)=\text{Unif}(\mathbf{a})$ & $\mu(\mathbf{a}|\bs)=\text{Unif}(\mathbf{a})$            &$\mu(\mathbf{a}|\bs)=\pi(\mathbf{a}|\bs)$&$\mu(\mathbf{a}|\bs)=\pi(\mathbf{a}|\bs)$\\
 & performance & regularizer value & performance & regularizer value\\
 \midrule
hopper-medium        & \textbf{97.2}  & \textbf{-2035.9} &  52.6  & -14.9  \\
walker2d-medium        &  7.9  & -106.8  & \textbf{81.9}  & \textbf{-1991.2} \\
    \bottomrule
    \end{tabular}}
    \caption{\footnotesize We include our automatic hyperparameter selection rule of $\mu(\mathbf{a}|\bs)$ on the medium datasets in the hopper and walker2d environments from D4RL. We follow the same convention defined in Table~\ref{tab:beta_selection} and find that our automatic selection rule can effectively select $\mu$ offline.}
    \label{tab:mu_selection}
\end{table}

\begin{table}[ht]
    \centering
    \scriptsize
    \resizebox{0.9\textwidth}{!}{\begin{tabular}{l|r|r|r|r|}
    \toprule
    Task & $\rho(\bs) = d^\pi_{\hat{\mathcal{M}}} $&$\rho(\bs) = d^\pi_{\hat{\mathcal{M}}}$            &$\rho(\bs) = d_f$&$\rho(\bs) = d_f$\\
 & performance & regularizer value & performance & regularizer value\\
 \midrule
hopper-medium        & \textbf{97.2}  & \textbf{-2035.9} &  56.0  & -6.0  \\
walker2d-medium        &  1.8  & 14617.4  & \textbf{81.9}  & \textbf{-1991.2} \\
    \bottomrule
    \end{tabular}}
    \caption{\footnotesize We include our automatic hyperparameter selection rule of $\rho(\bs)$ on the medium datasets in the hopper and walker2d environments from D4RL. We follow the same convention defined in Table~\ref{tab:beta_selection} and find that our automatic selection rule can effectively select $\rho$ offline.}
    \label{tab:rho_selection}
\end{table}

\begin{table}[ht]
    \centering
    \scriptsize
    \resizebox{0.9\textwidth}{!}{\begin{tabular}{l|r|r|r|r|}
    \toprule
    Task & $f = 0.5 $&$f = 0.5$            &$f = 0.8$&$f = 0.8$\\
 & performance & regularizer value & performance & regularizer value\\
 \midrule
hopper-medium        & \textbf{97.2}  & \textbf{-2035.9} &  93.8  & -21.3  \\
walker2d-medium        &  70.9  & -1707.0  & \textbf{81.9}  & \textbf{-1991.2} \\
    \bottomrule
    \end{tabular}}
    \caption{\footnotesize We include our automatic hyperparameter selection rule of $f$ on the medium datasets in the hopper and walker2d environments from D4RL. We follow the same convention defined in Table~\ref{tab:beta_selection} and find that our automatic selection rule can effectively select $f$ offline.}
    \label{tab:f_selection}
\end{table}

\subsection{Details of generalization environments}
\label{app:ood_details}

For \texttt{halfcheetah-jump} and \texttt{ant-angle}, we follow the same environment used in MOPO. For \texttt{sawyer-door-close}, we train the \texttt{sawyer-door} environment in \url{https://github.com/rlworkgroup/metaworld} with dense rewards for opening the door until convergence. We collect $50000$ transitions with half of the data collected by the final expert policy and a policy that reaches the performance of about half the expert level performance. We relabel the reward such that the reward is $1$ when the door is fully closed and $0$ otherwise. Hence, the offline RL agent is required to learn the behavior that is different from the behavior policy in a sparse reward setting. We provide the datasets in the following anonymous link\footnote{The datasets of the generalization environments are available at the following link: \url{https://drive.google.com/file/d/1pn6dS5OgPQVp_ivGws-tmWdZoU7m_LvC/view?usp=sharing}.}.

\subsection{Details of image-based environments}
\label{app:image_details}

\begin{figure}[ht]
    \centering
    \includegraphics[width=0.25\textwidth]{chapters/combo/walker_task.png}
    \includegraphics[width=0.25\textwidth]{chapters/combo/dooropen_task.png}
    \vspace{-0.2cm}
    \caption{\footnotesize Our image-based environments: The observations are $64\times 64$ and $128\times 128$ raw RGB images for the \texttt{walker-walk} and \texttt{sawyer-door} tasks respectively. The \texttt{sawyer-door-close} environment used in in Section~\ref{sec:generalization_exps} also uses the \texttt{sawyer-door} environment.}
    \label{fig:visual}
\end{figure}


We visualize our image-based environments in Figure~\ref{fig:visual}. We use the standard \texttt{walker-walk} environment from \citet{tassa2018deepmind} with $64\times64$ pixel observations and an action repeat of 2. Datasets were constructed the same way as \citet{fu2020d4rl} with 200 trajectories each. For the \texttt{sawyer-door} we use $128\times128$ pixel observations. The medium-expert dataset contains 1000 rollouts (with a rollout length of 50 steps) covering the state distribution from grasping the door handle to opening the door. The expert dataset contains 1000 trajectories samples from a fully trained (stochastic) policy. The data was obtained from the training process of a stochastic SAC policy using dense reward function as defined in \citet{yu2020metaworld}. However, we relabel the rewards, so an agent receives a reward of 1 when the door is fully open and 0 otherwise. This aims to evaluate offline-RL performance in a sparse-reward setting. All the datasets are from \citep{Rafailov2020LOMPO}.


\section{Comparing COMBO to the Naive Combination of CQL and MBPO}
\label{app:cql_mbpo}

\iclr{In this section, we stress the distinction between COMBO and a direct combination of two previous methods CQL and MBPO (denoted as CQL + MBPO). CQL+MBPO performs Q-value regularization using CQL while expanding the offline data with MBPO-style model rollouts. While COMBO utilizes Q-value regularization similar to CQL, the effect is very different. CQL only penalizes the Q-value on unseen actions on the states observed in the dataset whereas COMBO penalizes Q-values on states generated by the learned model while maximizing Q values on state-action tuples in the dataset. Additionally, COMBO also utilizes MBPO-style model rollouts for also augmenting samples for training Q-functions.

To empirically demonstrate the consequences of this distinction, CQL + MBPO performs quite a bit worse than COMBO on generalization experiments (Section~\ref{sec:generalization_exps}) as shown in Table~\ref{tbl:cql_mbpo}. The results are averaged across 6 random seeds ($\pm$ denotes 95\%-confidence interval of the various runs). This suggests that carefully considering the state distribution, as done in COMBO, is crucial.}

\begin{table}[ht]
    \centering
    \scriptsize
    \resizebox{0.7\textwidth}{!}{\begin{tabular}{l|r|r|r|r|}
    \toprule 
    %
    %
    %
    \textbf{Environment} & \stackanchor{\textbf{Batch}}{\textbf{Mean}} & \stackanchor{\textbf{Batch}}{\textbf{Max}} & \stackanchor{\textbf{COMBO}}{\textbf{(Ours)}} & \textbf{CQL+MBPO}\\ \midrule
    halfcheetah-jump & -1022.6 & 1808.6 & \textbf{5392.7}$\pm$575.5 & 4053.4$\pm$176.9\\
    ant-angle & 866.7 & 2311.9 & \textbf{2764.8}$\pm$43.6 & 809.2$\pm$135.4\\
    sawyer-door-close & 5\% & 100\% & \textbf{100}\%$\pm$0.0\% & 62.7\%$\pm$24.8\%\\
    \bottomrule
    \end{tabular}}
    \vspace{-0.2cm}
    \caption{
    \footnotesize Comparison between COMBO and CQL+MBPO on tasks that require out-of-distribution generalization. Results are in average returns of \texttt{halfcheetah-jump} and \texttt{ant-angle} and average success rate of \texttt{sawyer-door-close}. All results are averaged over 6 random seeds, $\pm$ the $95\%$-confidence interval.
    }
    \vspace{-0.3cm}
    \label{tbl:cql_mbpo}
    \normalsize
    \end{table}
    

% \subsection{Computation Complexity}

% For the D4RL and generalization experiments, COMBO is trained on a single NVIDIA GeForce RTX 2080 Ti for one day. For the image-based experiments, we utilized a single NVIDIA GeForce RTX 2070. We trained the \texttt{walker-walk} tasks for a day and the \texttt{sawyer-door-open} tasks for about two days.

% \subsection{License of datasets}

% We acknowledge that all datasets used in this paper use the MIT license.

% % \vspace{1cm}
% \section{Empirical Evidence on Challenges of Uncertainty Quantification}
% \label{app:uq}

% \begin{figure}[t]
%     \centering
%     \includegraphics[width=0.47\linewidth]{halfcheetah_medium_corr_var_ood.png}
%     \includegraphics[width=0.47\linewidth]{halfcheetah_medium_corr_lip_ens_ood.png}
%     \includegraphics[width=0.47\linewidth]{hopper_medium_corr_var_ood.png}
%     \includegraphics[width=0.47\linewidth]{hopper_medium_corr_lip_ens_ood.png}
%     \includegraphics[width=0.47\linewidth]{walker_medium_corr_var_ood.png}
%     \includegraphics[width=0.47\linewidth]{walker_medium_corr_lip_ens_ood.png}
%     \vspace{-0.2cm}
%     \caption{\footnotesize
%     %
%     We visualize the correlation between the model error and two uncertainty quantification methods maximum learned variance over the ensemble (left column) and variance of the model prediction over the ensemble (right column) on three D4RL medium datasets (from the top row to the bottom row: halfcheetah, hopper and walker) where MOPO performs poorly compared to model-free methods. We show that \textbf{Max Var} tends to be overly conservative and overestimating the model error while \textbf{Ens. Var} is on the opposite. Such visualizations corroborate that uncertainty quantification is challenging with deep neural networks and could lead to poor performance in model-based offline RL. In the meantime, COMBO addresses this issue by removing the burden of performing uncertainty quantification.}
%     \label{fig:uq}
%     \vspace{-0.3cm}
% \end{figure}

% In this section, we perform empirical evaluations to show that uncertainty quantification with deep neural networks, especially in the setting of dynamics model learning, is challenging and could cause problems with uncertainty-based model-based offline RL methods such as MOReL~\citep{kidambi2020morel} and MOPO~\citep{yu2020mopo}. In our evaluations, we consider two uncertainty quantification methods, maximum learned variance over the ensemble (denoted as \textbf{Max Var}) $\max_{i=1,\dots,N}\|\Sigma^i_\theta(\bs,\mathbf{a})\|_\text{F}$ (used in MOPO) and the variance of the model prediction over the ensemble (denoted as \textbf{Ens. Var}) $\max_{i=1,\dots,N}\|\mu^i_\theta(\bs,\mathbf{a}) - \frac{1}{N}\sum_{j=1}^N\mu^j_\theta(\bs,\mathbf{a})\|_2$ (used in MOPO and MOReL) where we use an ensemble of $N$ probabilistic dynamics models $\{\widehat{T}^i_\theta(\bs_{t+1}, r| \bs, \mathbf{a}) = \mathcal{N}(\mu^i_\theta(\bs_t, \mathbf{a}_t), \Sigma^i_\theta(\bs_t, \mathbf{a}_t))\}_{i=1}^N$.

% As shown in Table~\ref{tbl:d4rl}, MOPO performs underwhelmingly on medium datasets in the D4RL datasets where the dataset is collected with a single policy and hence with relatively narrow data coverage of the whole state space. To empirically analyze the poor performance of MOPO on those datasets, we visualize the correlation between the true model error and two uncertainty quantification methods \textbf{Max Var} and \textbf{Ens. Var}. We normalize both the model error and the uncertainty estimates to be within scale $[0, 1]$. As shown in Figure~\ref{fig:uq}, on all three medium datasets, \textbf{Max Var} tends to be overly conservative and \textbf{Ens. Var} behaves too optimistic to correctly quantify the true model error, suggesting that uncertainty estimation used by MOPO is not accurate and might be the major factor that results in its poor performance. Meanwhile, COMBO circumvents challenging uncertainty quantification problem and achieves much better performances on those medium datasets, indicating the effectiveness and the robustness of the method.

\end{document}
