% \counterwithin{figure}{section}
% \counterwithin{table}{section}
% \counterwithin{equation}{section}

\section{Additional Visualizations and Experiments for \drmethodname}

In this section, we provide visualizations and diagnostic experiments evaluating various  aspects of feature co-adaptation and the \drmethodname\ regularizer. We first provide more empirical evidence showing the presence of feature co-adaptation in modern deep offline RL algorithms. We will also visualize \drmethodname\, inspired from the implicit regularizer term in TD-learning alleviates rank collapse discussed in \citet{kumar2021implicit}. We will compare the efficacies of the explicit regularizer induced for different choices of the noise covariance matrix $M$ (Equation~\ref{eqn:regularizer}), understand the effect of dropping the stop gradient term in ou practical regularizer and finally, perform diagnostic experiments visualizing if the Q-networks learned with DR3 resemble more like neural networks trained via supervised learning, measured in terms of sensitivity and robustness to layer reinitialization~\citep{zhang2019all}.

\vspace{-0.05in}
\subsection{More Empirical Evidence of Feature Co-Adaptation}
\label{app:more_evidence_coadaptation}
In this section, we provide more empirical evidence demonstrating the existence of the feature co-adaptation issue in modern offline RL algorithms such as DQN and CQL. As shown below in Figure~\ref{fig:dot_product_increases_five_games}, while the average dataset Q-value for both CQL and DQN exhibit a flatline trend, the dot product similarity for consecutive state-action tuples generally continues to increase throughout training and does not flatline. While DQN eventually diverges in Seaquest, the dot products increase with more gradient steps even before divergence starts to appear.  

\begin{figure}[H]
    \centering
    \vspace{-0.1in}
    \includegraphics[width=0.99\textwidth]{chapters/dr3/figures_iclr/figure1_dotproduct_value.pdf}
    \includegraphics[width=0.99\textwidth]{chapters/dr3/figures_iclr/figure1_q_vals.pdf}
    \vspace{-5pt}
    \caption{\footnotesize{\label{fig:dot_product_increases_five_games} \textbf{Demonstrating feature co-adaptation on five Atari games with standard offline DQN and CQL, averaged over 3 seeds.} Observe that the feature dot products continue to rise with more training for both CQL and DQN, indicating the presence of co-adaptation. On the other hand, the average Q-values exhibit a converged trend, except on Seaquest. Further, note that the dot products continue to increase for CQL even though CQL explicitly corrects for out-of-distribution action inputs. }}
    \vspace{-0.1in}
\end{figure}

\vspace{-0.1in}
\subsection{{Layer-wise structure of a Q-network trained with DR3}}
\label{app:layerwise}
\vspace{-0.05in}
To understand if DR3 indeed makes Q-networks behave as if they were trained via supervised learning, utilizing the empirical analysis tools from \citet{zhang2019all}, we test the robustness/sensitivity of each layer in the learned network to re-initialization, while keeping the other layers fixed. This tests if a particular layer is \emph{critical} to the predictions of the learned neural network and enables us to reason about generalization properties~\citep{zhang2019all,chatterji2019intriguing}. We ran CQL and REM and saved all the intermediate checkpoints. Then, as shown in Figure~\ref{fig:robustness}, we first loaded a checkpoint ($x$-axis), and computed policy performance (shaded color; colorbar)  by re-initializing a given layer ($y$-axis) of the network to its initialization value before training for the same run.

Note in Figure~\ref{fig:robustness}, that while almost all layers are absolutely critical for the base CQL algorithm, 
% and most layers, including the last layer are critical for the REM algorithm~(see \figref{}), 
utilizing DR3 substantially reduces sensitivity to the latter layers in the Q-network over the course of training. This is similar to what \citet{zhang2019all} observed for supervised learning, where the initial layers of a network were the most critical, and the latter layers primarily performed near-random transformations without affecting the performance of the network. This indicates that utilizing DR3 alters the internal layers of a Q-network trained with TD to behave closer to supervised learning.

\begin{figure}[H]
    \centering
    % \begin{subfigure}[c]
    %     \centering
    \includegraphics[width=0.612\textwidth]{chapters/dr3/figures/robustness_figures/layer_robustness_cql.pdf}
    % ~\hline~
    \vspace{0.4cm}
    \includegraphics[width=0.622\textwidth]{chapters/dr3/figures/robustness_figures/rem_robustness_final.pdf}
    \vspace{-5pt}
    \caption{\footnotesize{\label{fig:robustness} \textbf{CQL vs CQL + DR3 and REM vs REM + DR3.} Average robustness of the learned Q-function to re-initialization of all layers to different checkpoints over the course of training created based on the protocol from \citet{zhang2019all}. The colors in the heatmap indicate performance of the reinitialized checkpoint, normalized w.r.t. the checkpoint without any change to layers. Note that while CQL and REM are more sensitive (i.e., less robust) to reinitialization of all the layers especially the last layer, CQL + DR3 and REM + DR3 behave closer to supervised learning, in the sense that they are more robust to reinitialization of layers of the network, especially the last layer.}}
    % \end{subfigure}
    % \caption{\textbf{REM vs REM + \drmethodname}. Average re-initialization robustness to different checkpoints of all layers for REM~(left) and REM + \drmethodname~(right). A score of 1 corresponds to same performance as the trained checkpoint while a score of 0 correponds to performance of a randomly initialized model. 
    % Following the protocol in \citep{zhang2019all}, ..}
    % % \begin{subfigure}[c]
    %     \centering
    %     \includegraphics[width=0.47\textwidth]{chapters/dr3/figures/robustness_figures/cql_robustness.pdf}
    %      \includegraphics[width=0.47\textwidth]{chapters/dr3/figures/robustness_figures/dr3_cql_robustness.pdf}
    %      \vspace{-10pt}
    %     \caption{CQL vs CQL + \drmethodname}
    % % \end{subfigure}
    \vspace{-0.3cm}
\end{figure}



% \vspace{-0.15cm}
% \subsection{\rebuttal{Results on MuJoCo Domains}}
% \label{app:mujoco}
% \rebuttal{In this section, we provide the results of applying DR3 on the MuJoCo tasks shown in Figure A.5. (Appendix A) of \citet{kumar2021implicit}. To briefly describe the setup, in these tasks we train on the three gym tasks (Hopper-v2, Ant-v2, Walker2d-v2) using 20\% of the offline data, uniformly subsampled from the run of an online SAC agent, mimicking the setup from \citet{kumar2021implicit}. Rather than retraining an SAC agent to collect data, we subsampled the Gym-MuJoCo \texttt{*-full-replay-v2} replay buffers from the latest D4RL~\citep{fu2020d4rl}. In these cases we plot the $\mathrm{srank}_\delta$ values, the feature dot products and the corresponding performance values with and without the DR3 regularizer for 4M steps (\citet{kumar2021implicit} showed their plots for just under 4M steps) in Figure~\ref{fig:mujoco_results_from_iup}.}

% \begin{figure}[t]
%     \centering
%     \vspace{-5pt}
%     \includegraphics[width=0.89\textwidth]{chapters/dr3/rebuttal/offline_dr3_mujoco_envs_rebuttal.pdf.pdf}
%     \includegraphics[width=0.89\textwidth]{chapters/dr3/rebuttal/offline_dr3_mujoco_envs_rebuttal_feat_dot.pdf_final.pdf}
%     \includegraphics[width=0.89\textwidth]{chapters/dr3/rebuttal/offline_dr3_mujoco_envs_rebuttal_ranks.pdf.pdf}
%     \vspace{-5pt}
%     \caption{\footnotesize{\label{fig:mujoco_results_from_iup} \rebuttal{\textbf{Comparison of CQL and CQL + DR3 on the offline MuJoCo Gym domains}, mimicking the setup of \citet{kumar2021implicit}. The data is generated by randomly sampling 20\% of the transitions of the D4RL~\citep{fu2020d4rl} full-replay-v2 datasets, which are collected via the run of an online SAC agent. The performance is shown in terms of the D4RL normalized score, where 0.0 denotes the performance of a random policy and 100.0 denotes the performance of an expert online SAC policy. Observe that adding DR3 stabilizes the performance on Hopper, and prevents performance collapse on Walker2d and Ant. In addition note that the srank values attained by CQL + DR3 is higher than base CQL and more importantly, the feature dot products are much smaller for CQL + DR3 compared to CQL.}}}
%     \vspace{-0.1cm}
% \end{figure}

% \rebuttal{Observe in Figure~\ref{fig:mujoco_results_from_iup}, that while the standard CQL algorithm performs poorly and suffers from performance degradation within about 1M-1.5M steps for Walker2d and Ant, CQL + DR3 is able to prevent the performance degradation and trains stably. Base CQL demonstrates oscillatory performance on Hopper, but CQL + DR3 stabilizes the performace of CQL. This indicates that DR3 is effective on MuJoCo domains, and prevents the instabilities with CQL.}

% \rebuttal{For details, the weight on the CQL regularizer in this case is equal to 5.0 across all the tasks, and weight on the DR3 regularizer is 0.01. We also attempted to tune the CQL coefficient for the baseline CQL algorithm within $\{1.0, 2.0, 5.0, 10.0, 20.0\}$ to see if it address the performance degradation issues, but did not find any difference in the collapsing behavior of base CQL. Our CQL baseline is therefore well-tuned, and DR3 improves the performance over this baseline.}

\subsection{Rank Collapse is Alleviated With DR3}
\label{app:rank_collapse_is_gone}

\begin{figure}[H]
    \centering
    % \vspace{-5pt}
    \includegraphics[width=0.99\textwidth]{chapters/dr3/figures_iclr/rank_trends_dr3_cql.pdf}
    \vspace{-5pt}
    \caption{\footnotesize{\label{fig:rank_trends_5_games} \textbf{Comparing the feature ranks for CQL and CQL + DR3.} Observe that utilizing DR3 successfully alleviates the rank collapse issue noted in prior work without explicitly correcting for it.}}
    % \vspace{-0.5cm}
\end{figure}
Prior work~\citep{kumar2021implicit} has shown that implicit regularization in TD-learning can lead to a feature rank collapse phenomenon in the Q-function, which hinders the Q-function from using its full representational capacity. Such a phenomenon is absent in supervised learning, where the feature rank does not collapse. Since DR3 is inspired by mitigating the effects of the term in the implicit regularizer (Equation~\ref{eqn:regularizer}) that only appears in the case of TD-learning, we wish to understand if utilizing DR3 also alleviates rank collapse. To do so, we compute the effective rank $\mathrm{srank}_\delta(\phi)$ metric of the features learned by Q-functions trained via CQL and CQL with DR3 explicit regularizer. As shown in Figure~\ref{fig:rank_trends_5_games}, for the case of five Atari games, utilizing DR3 alleviates the rank collapse issue completely \rebuttal{(i.e., the ranks do not collapse to very small values when CQL + DR3 is trained for long). We do not claim that the ranks with DR3 are necessarily higher, and infact as we show below, a higher srank of features may not always imply a better solution.} The fact that DR3 can prevent rank collapse is potentially surprising, because no term in the practical DR3 regularizer explicitly aims to increase rank: feature dot products can be made smaller while retaining low ranks by simply rescaling the feature vectors. But, as we observe, utilizing DR3 enables learning features \rebuttal{that do not exhibit collapsed ranks}, thus \rebuttal{we hypothesize} that correcting for appropriate terms in $\mathrm{R}_\mathrm{TD}(\theta)$ can address some of the previously observed pathologies in TD-learning. 


\begin{figure}[H]
    \centering
    \vspace{-10pt}
    \includegraphics[width=0.99\textwidth]{chapters/dr3/rebuttal/figure_analysis_dqn_for_rank_collapse_appendix_rebuttal_0.1_final.pdf}
    \includegraphics[width=0.99\textwidth]{chapters/dr3/rebuttal/srank_phi_dqn_rebuttal_200_v4.pdf}
    \vspace{-5pt}
    \caption{\footnotesize{\label{fig:rank_collapse_test_for_dqn} \rebuttal{\textbf{Performance and $\mathrm{srank}$ values for DQN and DQN + DR3.} Observe that the srank values increase for DQN + DR3, while they collapse for DQN on Asterix, Seaquest and SpaceInvaders with more training. Thus, DQN + DR3 does not suffer from a sudden rank collapse. However, a higher srank does not imply a better return, and so while initially DQN does have a high rank, DQN + DR3 performs superiorly.}}}
    \vspace{-0.3cm}
\end{figure}

\rebuttal{We now investigate the feature ranks of a Q-network trained when DR3 is applied in conjunction with a standard DQN and REM~\citep{agarwal2019optimistic} on the Atari domains. We plot the values of $\mathrm{srank}_\delta(\phi)$, the feature dot products and the performance of the algorithm for DQN in Figure~\ref{fig:rank_collapse_test_for_dqn} and for REM in Figure~\ref{fig:rank_collapse_test_for_rem}. In the case of DQN, we find that unlike the base DQN algorithm for which feature rank does begin to collapse with more training, the srank for DQN + DR3 is increasing. We also note that DQN + DR3 attains a better performance compared to DQN, throughout training. }

\begin{figure}[H]
    \centering
    \vspace{-5pt}
    \includegraphics[width=0.99\textwidth]{chapters/dr3/rebuttal/figure_analysis_rem_for_rank_collapse_appendix_rebuttal_0.001_final.pdf}
    \includegraphics[width=0.99\textwidth]{chapters/dr3/rebuttal/srank_phi_rem_rebuttal_final_200_v4.pdf}
    \vspace{-5pt}
    \caption{\footnotesize{\label{fig:rank_collapse_test_for_rem} \rebuttal{\textbf{Comparing the performance and $\mathrm{srank}$ values for REM and REM + DR3.} Observe that while REM + DR3 outperforms REM, the $\mathrm{srank}$ values attained by REM are much larger than REM + DR3, and none of these ranks have collapsed. Thus, while REM + DR3 maintains non-collapsed features, for the case of REM, it reduces the value of $\mathrm{srank}$ and attains better performance. This does not contradict the observations from \citet{kumar2021implicit} as we discuss in the text.}}}
\end{figure}

\rebuttal{However, we note that the opposite trend is true for the case of REM: while REM + DR3 attains a better performance than REM, adding DR3 leads to a reduction in the $\mathrm{srank}$ value compared to base REM. At a first glance, this might seem contradicting \citet{kumar2021implicit}, but this is not the case: to our understanding, \citet{kumar2021implicit} establish a correlation between extremely low rank values (i.e., rank collapse) and poor performance, but this does not mean that all high rank features will lead to good performance. We suspect that since REM trains a multi-headed Q-function with shared features and randomized target values, it is able to preserve high-rank features, but this need not mean that these features are useful. In fact, as shown in Figure~\ref{fig:rem_dot_products}, we find that the base REM algorithm does exhibit feature co-adaptation. This case is an example where the srank metric from \citet{kumar2021implicit} may not indicate poor performance.}  

\begin{figure}[H]
    \centering
    \vspace{-10pt}
    \includegraphics[width=0.99\textwidth]{chapters/dr3/rebuttal/rebuttal_rem_dotproduct_value_final_v2.pdf}
    \vspace{-5pt}
    \caption{\footnotesize{\label{fig:rem_dot_products} \rebuttal{\textbf{Feature dot products for REM and REM + DR3 on log scale.} REM does suffer from feature co-adaptation despite high-rank features.}}}
    \vspace{-0.3cm}
\end{figure}


\vspace{-0.1cm}
\subsection{Induced Implicit Regularizer: Theory And Practice}
\label{app:theory_practice_gap}
\vspace{-0.1cm}

\begin{table}[t]
    \centering
\fontsize{8}{8}\selectfont
    \centering
    \vspace{-0.05cm}
    \caption{\footnotesize{Normalized interquartile mean performance with 95\% stratified bootstrap CIs~\citep{agarwal2021precipice} across 17 Atari games of REM, REM +  $\Delta'(\Phi)$ (Stop gradient in DR3), REM + \drmethodname\ after 6.5M gradient steps for the 1\% setting and 12.5M gradient steps for the 5\%, 10\% settings. Observe that REM + $\Delta'(\phi)$ also improves over the base REM method significantly, by about 130\%, even though $\Delta'(\phi)$ is generally comparable and somewhat worse than the DR3 regularizer used in the main paper.}}
    \label{tab:rem_phi_res}
\begin{tabular}{lcccc}
\toprule
% \multirow{2}{*}{\textbf{Data}}  & \multicolumn{4}{c}{\textbf{Stability performance}} \\
Data &  REM & REM + $\Delta'(\Phi)$ & REM+\drmethodname \\
\midrule
1\%   &  4.0~\ss{(3.3, 4.8)} & 15.0~\ss{(13.4, 16.6)}
& 16.5~\ss{(14.5, 18.6)}  \\
\midrule
5\%   & 25.9~\ss{(23.4, 28.8)} & 55.5~\ss{(50.8, 59.8)} &  60.2~\ss({55.8, 65.1}) \\
\midrule
10\%  & 53.3~\ss{(51.4, 55.3)} & 67.7~\ss{(64.7, 71.3)} & 73.8~\ss{(69.3, 78)} \\
\bottomrule
\vspace{-0.5cm}
\end{tabular}
\end{table}
In this section, we compare the performance of our practical DR3 regularizer to the regularizers (Equation~\ref{eqn:regularizer}) obtained for different choices of $M$, such as $M$ induced by noise, studied in previous work, and also evaluate the effect of dropping the stop gradient function from the practical version of our regularizer.

\textbf{Empirically comparing the explicit regularizers for different noise covariance matrices, $M$.} The theoretically derived regularizer (Equation~\ref{eqn:regularizer}) suggests that for a given choice of $M$, the following equivalent of feature dot products should increase over the course of training: 

\begin{equation}
\label{eqn:general_M}
    \Delta_M(\theta):= \sum_{\bs, \mathbf{a} \in \mathcal{D}} \mathrm{trace}\left[\Sigma^*_M \nabla Q(\bs, \mathbf{a}) \nabla Q(\bs', \mathbf{a}')^\top \right].~~~~~~ \text{(Generalized dot products)}
\end{equation}
We evaluate the efficacy of the explicit regularizer that penalizes the generalized dot products, $\Delta_M(\theta)$, in improving the performance of the policy, \rebuttal{with the goal of identifying if our practical method performs similar to this regularizer on generalized dot products.}. While $\Sigma^*_M$ must be explicitly computed by running fixed point iteration for every parameter iterate $\theta$ found during TD-learning -- which makes this method significantly computationally expensive\footnote{\rebuttal{In our implementation, we run 20 steps of the fixed-point computation of $\Sigma$ as shown in Theorem~\ref{thm:implicit_noise_reg} for each gradient step on the Q-function, and this increases the runtime to about 8 days for 50 iterations on a P100 GPU.}}, we evaluated it on five Atari games for \rebuttal{50 $\times$ 62.5k gradient steps as a proof of concept}. As shown in Figure~\ref{fig:explicit_m_choices}, the DR3 penalty with the choice of $M$ which corresponds to label noise, and the dot product DR3 penalty, which is our main practical approach in this paper generally perform similarly on these domains, attaining almost identical learning curves on \textbf{4/5 games}, and clearly improving over the base algorithm. This hints at the possibility of utilizing other noise covariance matrices to derive an explicit regularizer. \rebuttal{Deriving more computationally efficient versions of the regularizer for a general $M$ and identifying the best choice of $M$ are subject to future work.}

\begin{figure}[H]
    \centering
    \includegraphics[width=0.99\textwidth]{chapters/dr3/figures_iclr/figure_analysis_different_dr3_penalty.pdf}
    \vspace{-5pt}
    \caption{\footnotesize{\label{fig:explicit_m_choices} \textbf{Comparing the performance of explicit penalties for two different choices of the covariance matrix $M$.} Observe that in all the five games the DR3 regularizer derived for the choice of $M$ from \citet{blanc2020implicit} also leads to a substantial increase in performance over the base algorithm, and in four of five games, DR3 (label-noise) works just as well as DR3.}}
    \vspace{-0.25cm}
\end{figure}

\textbf{Effect of stop gradient.} Finally, we investigate the effect of utilizing a stop gradient in the DR3 regularizer. We run a variant of DR3: $\Delta'(\phi) = \sum_{\bs, \mathbf{a}, \bs'} \phi(\bs, \mathbf{a})^\top [[\phi(\bs', \mathbf{a}')]]$, with the stop gradient on the second term $(\bs', \mathbf{a}')$ and present a comparison to the one without the stop gradient in Table~\ref{tab:rem_phi_res} for REM as the base offline method, averaged over 17 games. Note that this version of DR3, with the stop gradient, also improves upon the baseline offline RL method (i.e., REM) by \textbf{130\%}. While this performs largely similar, but somewhat worse than the complete version without the stop gradient, these results do indicate that utilizing $\Delta'(\phi)$ can also lead to significant gains in performance.


\subsection{\rebuttal{Understanding Feature Co-Adaptation Some More}}
\vspace{-0.2cm}
\rebuttal{In this section, we present some more empirical evidence to understand feature co-adaptation. The three factors we wish to study are: \textbf{(1)} the effect of target update frequency on feature co-adaptation; \textbf{(2)} understand the trend in normalized similarities and compare these to the trend in dot products; and \textbf{(3)} understand the effect of out-on-sample actions in TD-learning and compare it to  offline SARSA on a simpler gridworld domain. We answer these questions one by one via experiments aiming to verify each hypothesis.}


\begin{figure}[H]
    \centering
    \vspace{-0.3cm}
    \includegraphics[width=0.3\linewidth]{chapters/dr3/rebuttal/cql_alpha0_target_delay_coadaptation.pdf}
    \includegraphics[width=0.3\linewidth]{chapters/dr3/rebuttal/cql_alpha_1_target_delay_coadaptation.pdf}
    \vspace{-0.3cm}
    \caption{\label{fig:target_update_frequency} \footnotesize{\rebuttal{\textbf{Comparing the feature dot products for various target update delays}, where a smaller $N$ implies a faster update and a larger $N$ corresponds to a slower target update. Observe that while slower updates to the target network may reduce co-adaptation, very slow target updates may still lead to excesssive co-adaptation.}}} 
    \vspace{-0.3cm}
\end{figure}

\subsubsection{\rebuttal{Effect of Target Update Frequency on Feature Co-Adaptation}}
\rebuttal{We studied the effect of target update frequency on feature co-adaptation, on some gridworld domains from \citet{fu2019diagnosing}. We utilized the \texttt{grid16smoothobs} environment, where the goal of the agent is to navigate from the center of a $16 \times 16$ gridworld maze to one of its corners while avoiding obstacles and ``lava'' cells. The observations provided to the RL algorithm are given by a high-dimensional random transformation of the $(x, y)$ coordinates, smoothed over neighboring cells in the gridworld. We sampled an offline dataset of 256 transitions and trained a Q-network with two hidden layers of size $(1024, 1024)$ via fitted Q-iteration (FQI)~\citep{Riedmiller2005}.}


\rebuttal{We evaluated the feature dot products for Q-functions trained with a varying target update frequencies, given generically as: updating the target network using a hard target update once per $N$ gradient steps, where $N$ takes on values $N=5, 10, 50, 100, 200, 500$, and present the results in Figure~\ref{fig:target_update_frequency} (left), averaged over 3 random seeds. Thee feature dot products initially decrease from $N=5$ to $N=10$, because the target network is updated slower, but then starts to rapidly increase when when the target network is slowed down further to $N=50$ and $N=200$ in one case and $N=500$ in the other case.} \rebuttal{We also evaluated the feature dot products when using CQL as the base offline RL algorithm. As shown in Figure~\ref{fig:target_update_frequency} (right), while CQL does reduce the absolute range of the feature dot products, slow target updates with $N=500$ still lead to the highest feature dot products as training progresses.}

\rebuttal{\textbf{Takeaway:} While it is intuitive to think that a slower target network might alleviate co-adaptation, we see that this is not the case empirically with both FQI and CQL, suggesting a deeper question that is an interesting avenue for future study.}


\subsubsection{\rebuttal{Gridworld Experiments Comparing TD-learning And Offline SARSA}}
\label{app:exact_behavior_policy}
\rebuttal{To supplement the analysis in Section~\ref{app:problem_more}, we ran some experiments in the gridworld domains from \citet{fu2019diagnosing}. In this case, we used the \texttt{grid16smoothsparse} and \texttt{grid16randomsparse} domains, which present challenging navigation tasks in a maze under a 0-1 sparse reward signal, provided at the end of the trajectory. Additionally, the observations available to the offline RL agent do not consist of the raw $(x, y)$ locations of the agent in the maze, but rather high-dimensional randomly chosen transformations of $(x, y)$ in the case of \texttt{grid16randomsparse}, which are additionally smoothed locally around a particular state to obtain \texttt{grid16smoothsparse}.}

Since our goal is to compare feature co-adaptation in TD-learning and offline SARSA, we consider a case where we evaluate a ``mixed'' behavior policy that chooses the optimal action with a probability of 0.7 at a given state, and chooses a random, suboptimal action with 0.3. We then generate a dataset of size $256$ transitions and train offline SARSA and TD-learning on this data. While SARSA backups the next action observed in the offline dataset, TD-learning computes a full expectation of the Q-function $\E_{\mathbf{a}' \sim \pi_\beta(\cdot|\bs')}\left[Q(\bs', \mathbf{a}')\right]$ under the behavior policy for computing Bellman backup targets. The behavior policy is fully known to the TD-learning agent. Our Q-network consists of two hidden layers of size $(1024, 1024)$ as before.

\begin{wrapfigure}{r}{0.6\textwidth}
    \centering
    \vspace{-0.5cm}
    \includegraphics[width=0.47\linewidth]{chapters/dr3/rebuttal/smoothsparse_td_vs_q.pdf}
    \includegraphics[width=0.47\linewidth]{chapters/dr3/rebuttal/randomsparse_td_vs_q.pdf}
    \vspace{-0.3cm}
    \caption{\label{fig:out_of_sample_actions} \footnotesize{\rebuttal{\textbf{Comparing the feature dot products for TD-learning and offline SARSA}, used to compute the value of the behavior policy using a dataset of size 256 on two gridworld domains. Observe that the feature dot products are higher in the case of TD-learning compared to offline SARSA.}}} 
    % \label{fig:1d_mdp}
    % \vspace{-0.5cm}
\end{wrapfigure}
We present the trends in the feature dot products for TD-learning and offline SARSA in Figure~\ref{fig:out_of_sample_actions}, averaged over three seeds. Observe that the trends in the dot product values for TD-learning and offline SARSA closely follow each other for the initial few gradient steps, soon, the dot products in TD-learning start growing faster. In contrast, the dot products for SARSA either saturate or start decreasing. The only difference between TD-learning and SARSA is the set of actions used to compute Bellman targets -- while the actions used for computing Bellman backup targets in SARSA are in-sample actions and are observed in the dataset, the actions used by TD-learning may be out-of-sample, but are still within the distribution of the data-generating behavior policy. This supports our empirical evidence in the main paper showing that out-of-sample actions can lead to feature co-adaptation.

\subsubsection{\rebuttal{Feature Co-Adaptation and Normalized Feature Similarities}}
\rebuttal{Note that we characterized feature co-adaptation via the dot products of features. In this section, we explore the trends in other notions of similarity, such as cosine similarity between $\phi(\bs, \mathbf{a})$ and $\phi(\bs', \mathbf{a}')$ which measures the dot product of feature vectors at consecutive state-action tuples after normalization. Formally,}
\rebuttal{
\begin{align*}
 \text{cos} (\phi(\bs, \mathbf{a}), \phi(\bs', \mathbf{a}')) := \frac{\phi(\bs, \mathbf{a})^\top \phi(\bs', \mathbf{a}')}{||\phi(\bs, \mathbf{a})||_2 \cdot ||\phi(\bs', \mathbf{a}')||_2}.
\end{align*}
}
\rebuttal{\!We plot the trend in the cosine similarity with and without DR3 for five Atari games in Figure~\ref{fig:atari_cosine} with CQL, DQN and REM, and for the three MuJoCo tasks studied in Appendix~\ref{app:mujoco} in Figure~\ref{fig:mujoco_cosine}. We find that the cosine similarity is generally very high on the Atari domains, close to 1, and not indicative of performance degradation. On the Ant and Walker2d MuJoCo domains, we find that the cosine similarity first rises up close to 1 and roughly saturates there. On the Hopper domain, the cosine similarity even decreases over training. However we observe that the feature dot products are increasing for all the domains. Applying DR3 in both cases improves performance (as shown in earlier Appendix), and generally gives rise to reduced cosine similarity values, though it can also increase the cosine similarity values occasionally. Furthermore, even when the cosine similarities were decreasing for the base algorithm (e.g., in the case of Hopper), addition of DR3 reduced the feature dot products and helped improve performance. This indicates that both the norm and directional alignment are contributors to the co-adaptation issue, which is what DR3 aims to fix and independently directional alignment does not indicate poor performance.}

\begin{figure}[H]
    \centering
    \includegraphics[width=0.99\textwidth]{chapters/dr3/rebuttal/rebuttal_dqn_cosine_sim_value_final_rebuttal.pdf}
    \includegraphics[width=0.99\textwidth]{chapters/dr3/rebuttal/rem_cosine.png}
    \includegraphics[width=0.99\linewidth]{chapters/dr3/rebuttal/rebuttal_cql_cosine_sim_value_final_rebuttal.pdf}
    \vspace{-5pt}
    \caption{\footnotesize{\label{fig:atari_cosine} \rebuttal{\textbf{Cosine similarities of DQN, DQN + DR3, REM, REM + DR3 and CQL, CQL + DR3.} Note that DQN, REM and CQL attain close to 1 cosine similarities, and addition of DR3 does reduce the cosine similarities of consecutive state-action features.}}}
    \vspace{-0.25cm}
\end{figure}

\begin{figure}[t]
    \centering
    \includegraphics[width=0.9\textwidth]{chapters/dr3/rebuttal/offline_dr3_mujoco_envs_rebuttal_feat_cosine.pdf_Final.pdf}
    \vspace{-5pt}
    \caption{\footnotesize{\label{fig:mujoco_cosine} \rebuttal{\textbf{Cosine similarities of CQL and CQL + DR3 on MuJoCo domains.} Note that the cosine similarities of CQL grow to 1 and roughly stabilize for Ant and Walker2d, but start decreasing for Hopper. This happens despite the oscillatory trends in performance of CQL on Hopper~\ref{fig:mujoco_results_from_iup}. This means that a low cosine similarity need not imply poor performance, and DR3 can improve performance even when cosine similarity of base CQL is decreasing. We also notice that DR3 does actually reduce cosine similarity.}}}
    \vspace{-0.25cm}
\end{figure}


\subsection{\rebuttal{Stability of DR3 From a Good Solution}}
\label{app:cql_stability}
\rebuttal{In this appendix, we study the trend of CQL + DR3 when starting learning from a good initialization, which was studied in Figure~\ref{fig:stability}. As shown in Figure~\ref{fig:cql_stability}, while the performance for baseline CQL degrades significantly (from 5000 at initialization on Asterix, performance degrades to $\sim$2000 by 100 iterations for base CQL), whereas the performance of DR3 only moves from 5000 to $\sim$4300. A similar trend holds for Breakout. This means that the addition of DR3 does stabilize the learning relative to the baseline algorithm. Please note that we are not claiming that DR3 is unequivocally stable, but that improves stability relative to the base method.}

% \begin{figure}[t]
%     \centering
%     \includegraphics[width=0.45\textwidth]{chapters/dr3/rebuttal/figure_analysis_stability_with_without_dr3_cql.pdf} ~\vline~
%     \includegraphics[width=0.45\textwidth]{chapters/dr3/rebuttal/figure_analysis_dot_product_stability_cql_vs_cql_dr3.pdf}
%     \vspace{-5pt}
%     \caption{\footnotesize{\label{fig:cql_stability} \rebuttal{\textbf{Running CQL + DR3 and CQL in the setup of Figure~\ref{fig:stability} to evaluate the stability of CQL + DR3 when starting training from a good solution.} Observe that the performance of base CQL decays quickly from the good solution, but CQL + DR3 is \emph{relatively} more stable. Additionally, the feature dot products for DR3 are much smaller compared to CQL.}}}
%     \vspace{-0.25cm}
% \end{figure}

\begin{figure}[H]
    \centering
    \includegraphics[width=0.99\textwidth]{chapters/dr3/rebuttal/figure_analysis_stability_five_games_final_cql_dr3.pdf}
    \vspace{-5pt}
    \caption{\footnotesize{\label{fig:cql_stability} \rebuttal{\textbf{Running CQL + DR3 and CQL in the setup of Figure~\ref{fig:stability} to evaluate the stability of CQL + DR3 when starting training from a good solution.} Observe that the performance of base CQL decays quickly from the good solution, but CQL + DR3 is \emph{relatively} more stable.}}}
    \vspace{-0.25cm}
\end{figure}

\begin{figure}[H]
    \centering
    \includegraphics[width=0.99\textwidth]{chapters/dr3/rebuttal/figure_analysis_stability_five_games_final_dqn_dr3.pdf}
    \vspace{-5pt}
    \caption{\footnotesize{\label{fig:dqn_stability} \rebuttal{\textbf{Running DQN + DR3 and DQN in the setup of Figure~\ref{fig:stability} to evaluate the stability of DQN + DR3 when starting training from a good solution.} Observe that the performance of base DQN decays quickly from the good solution, but DQN + DR3 is \emph{relatively} more stable.}}}
    \vspace{-0.25cm}
\end{figure}


\iffalse

\vspace{-0.1cm}
\subsection{\rebuttal{Statistical Significance of DR3 and Franka Kitchen Results}}
\label{app:significance}

\begin{table}[h]
    % \begin{table}[t]
    % % \vspace{-0.8cm}
    \vspace{-0.05in}
    \fontsize{10}{8}\selectfont
    \centering
    \captionof{table}{\footnotesize{\textbf{Performance of CQL, CQL + \drmethodname\ after 2M gradient steps on the Franka Kitchen domains} averaged over 3 seeds. This is training for \textbf{6x} longer compared to CQL defaults. Observe that CQL + \drmethodname\ outperforms CQL at 2M steps, indicating is efficacy in preventing long term performance degradation.
    }}
    \label{tab:cql_kitchen}
    \vspace{-0.1in}
    \begin{tabular}{@{}lrr@{}}
    \toprule
    {\textbf{D4RL Task}} & CQL & CQL + \drmethodname \\
    \midrule
    % \texttt{kitchen-mixed} & 14.6 $\pm$ 20.5 & \textbf{37.0 $\pm$ 8.0} \\
    % \texttt{kitchen-partial} & 29.6 $\pm$ 19.6 & \textbf{43.5 $\pm$ 1.9}  \\
    % \texttt{kitchen-complete} & 22.3 $\pm$ 17.5 & 24.8 $\pm$ 15.3 \\
    \texttt{kitchen-mixed} & 27.67 $\pm$ 12.66 & \textbf{37.00 $\pm$ 11.53} \\
    \texttt{kitchen-partial} & 20.67 $\pm$ 15.57 & \textbf{40.67 $\pm$ 4.04}  \\
    \texttt{kitchen-complete} & 28.00 $\pm$ 14.73 & \textbf{38.67 $\pm$ 6.66} \\
    \bottomrule
    \end{tabular}
    % \vspace{cm}
\end{table}
We present the results comparing CQL and CQL+DR3 on the Franka Kitchen tasks from D4RL in Table~\ref{tab:cql_kitchen}. Observe that CQL+DR3 outperforms CQL, and to test the statistical significance of these results, we analyze the probability of improvement of CQL+DR3 over CQL next.

% \vspace{-0.5cm}
\begin{figure}[H]
    \centering
    \vspace{-0.15cm}
    \includegraphics[width=0.4\linewidth]{chapters/dr3/rebuttal/statistical_significance.png}
    \vspace{-0.2cm}
    \caption{\label{fig:significance} \footnotesize{\rebuttal{\textbf{Statistical significance of the results of CQL + DR3 vs CQL (Table~\ref{tab:cql_d4rl}) as measured by average probability of improvement~\citep{agarwal2021precipice},} with stratified bootstrap confidence intervals for this statistic. Since the lower CI for this statistic is $> 0.5$, CQL + DR3 \textbf{significantly} improves over base CQL, and since the mean and upper CI are $\geq 0.75$, this improvement is also \textbf{meaningful}.}}} 
    \vspace{-0.4cm}
\end{figure}

\rebuttal{In order to assess the statistical significance of our D4RL Antmaze and Kitchen results, we follow the recommendedations by \citet{agarwal2021precipice} for comparing deep RL algorithms considering their statistical uncertainties. Specifically, we computed the average probability of improvement~\citep{agarwal2021precipice} of CQL + DR3 over CQL on the antmaze and kitchen domains, and we find that DR3 \textbf{does} significantly and meaningfully improve over CQL on both the Kitchen and AntMaze domains. Before presenting the results, let us first describe the metric we compute.}

% \vspace{-0.1cm}
\rebuttal{\textbf{Probability of improvement and statistical significance}. For two given algorithms $\mathsf{Alg}_1$ and $\mathsf{Alg}_2$, and runs $X_{k,1}, X_{k,2}, \cdots, X_{k,m}$ from $\mathsf{Alg}_1$ and runs $Y_{k,1}, Y_{k,2}, \cdots, Y_{k,n}$ from $\mathsf{Alg}_2$ on task $k$, the probability of improvement of $\mathsf{Alg}_1$ over $\mathsf{Alg}_2$ is given by $P(\mathsf{Alg}_1 > \mathsf{Alg}_2) = \frac{1}{K} \sum_{k=1}^{K} P(\mathsf{Alg}^k_1 > \mathsf{Alg}^k_2)$. The probability of improvement on a given task $k$,   $P(\mathsf{Alg}^k_1 > \mathsf{Alg}^k_2)$ is computed using the Mann-Whitney U-statistic and is given by:}
\vspace{-0.2cm}
\rebuttal{
\begin{align*}
   P(\mathsf{Alg}^k_1 > \mathsf{Alg}^k_2) = \frac{1}{M N} \sum_{i=1}^{M}\sum_{j=1}^{N}S(X_{k,i}, Y_{k, j})\quad \text{where}\quad
S(x,y)={\begin{cases}1,&{\text{if }}y<x,\\{\tfrac {1}{2}},&{\text{if }}y=x,\\0,&{\text{if }}y>x.\end{cases}}\label{eq:prob_improve}
\end{align*}
\vspace{-0.05cm}
}


\rebuttal{\!$\mathsf{Alg}_1$ leads to statistically \emph{significant} improve over $\mathsf{Alg}_2$ if the lower CI for $\mathsf{P} (\mathsf{Alg}_1 > \mathsf{Alg}_2)$ is larger than 0.5. Per the Neyman-Pearson statistical testing criterion in \citet{bouthillier2021accounting}, $\mathsf{Alg}_1$ leads to statistically \emph{meaningful} improvement over $\mathsf{Alg}_2$  if the upper confidence interval~(CI) of $\mathsf{P} (\mathsf{Alg}_1 > \mathsf{Alg}_2)$ is larger than 0.75.} 

\rebuttal{Figure~\ref{fig:significance} presents the value of $\mathsf{P} (\text{CQL + DR3} > \text{CQL})$ on the AntMaze and Kitchen domains at 2M gradient steps along with the 95\% CI for this statistic. DR3 improves over CQL on the AntMaze domains with probability \textbf{0.83} with 95\% CI (0.7, 0.96) and on the Kitchen domains with probability \textbf{0.8} with 95\% CI (0.6, 1.0). These values pass the criterion of being both statistically significant and meaningful per the above definitions, implying that DR3 does significantly and meaningfully improve upon CQL on these domains.}

\fi

%we aim to answer these questions by showing that implicit regularization induced during TD-learning is the primary cause of this co-adaptation phenomenon, and it can lead to several undesirable consequences. 
%%AK: I still kept SARSA for now, but we can remove it if needed. I removed statements saying SARSA fixes all issues, and I called it a proxy method to indicate that it is not good

\vspace{-0.2cm}
\section{Related Works}
\vspace{-0.2cm}
\label{sec:extended_related}
In this section, we briefly review some extended related works, and in particular, try to connect feature co-adaptation and implicit regularization to various interesting results pertaining to RL lower-bounds with function approximation and self-supervised learning.

\subsection{Brief Summary of Related Work}
\label{sec:related}
\vspace{-5pt}
Prior analyses of the learning dynamics in RL has focused primarily on analyzing error propagation in tabular or linear settings~\citep[\eg][]{chen2019information,duan2020minimax,xie2020q, wang2021what,wang2021instabilities,farahmand2010error,de2002alp}, understanding instabilities in deep RL~\citep{achiam2019towards,bengio2020interference,kumar2020discor,van2018deep} and deriving weighted TD updates that enjoy convergence guarantees~\citep{maei09nonlineargtd,mahmood2015emphatic,sutton16emphatic}, but these methods do not reason about implicit regularization or any form of representation learning. \citet{ghosh2020representations} focuses on understanding the stability of TD-learning in underparameterized linear settings, whereas our focus is on the overparameterized setting, when optimizing TD error and learning representations via SGD.  \citet{kumar2021implicit} studies the learning dynamics of Q-learning and observes that the rank of the feature matrix, $\Phi$, drops during training. While this observation is related, our analysis characterizes the implicit preference of learning towards feature co-adaptation (Theorem~\ref{thm:implicit_noise_reg}) on out-of-sample actions as the primary culprit for aliasing. Additionally, while the goal of our work is not to increase $\srank(\Phi)$, utilizing \methodname\ not only outperforms the $\srank(\Phi)$ penalty in \citet{kumar2021implicit} by more than \textbf{100\%}, but it also alleviates rank collapse, with no apparent term that explicitly enforces high rank values. Somewhat related to DR3, \citet{durugkar2018td,pohlen2018observe} heuristically constrain gradients of TD to prevent changes in target Q-values to prevent divergence. Contrary to such heuristic approaches,  DR3 is inspired from a theoretical model of implicit regularization, and does not prevent changes in target values, but rather reduces feature dot products.

\subsection{Extended Related Work}
\textbf{Lower-bounds for offline RL.} \citet{zanette2020exponential} identifies hard instances for offline TD learning of linear value functions when the provided features are ``aliased''. Note that this work does not consider feature learning or implicit regularization, but their  hardness result relies heavily on the fact the given linear features are aliased in a special sense. Aliased features utilized in the hard instance inhibit learning along certain dimensions of the feature space with TD-style updates, necessitating an exponential sample size for near-accurate value estimation, even under strong coverage assumptions. A combination of \citet{zanette2020exponential}'s argument, which provides a hard instance given aliased features, and our analysis, which studies the emergence of co-adapted/similar features in the offline deep RL setting, could imply that the co-adaptation can lead to failure modes from the hard instance, even on standard offline RL problems, when provided with limited data.

\textbf{Connections to self-supervised learning (SSL).}  Several modern self-supervised learning methods~\citep{grill2020bootstrap,chen2020exploring} can be viewwed as utilizing some form of bootstrapping where different augmentations of the same input ($\bx + \text{Aug}_1, \bx + \text{Aug}_2$) serve as consecutive state-action tuples that appear on two sides of the backup. If we may extrapolate our reasoning of feature co-adaptation to this setting, it would suggest that performing noisy updates on a self-supervised bootstrapping loss will give us feature representations that are highly similar for consecutive state-action tuples, i.e., the representations for $\phi(\bx + \text{Aug}_1)^\top \phi(\bx + \text{Aug}_2)$ will be high. Intuitively, an easy way for obtaining high feature dot products is for $\phi(\cdot)$ to capture only that information in $\cdot$, which is agnostic to data augmentation, thus giving rise to features that are invariant to transformations. This aligns with what has been shown in self-supervised learning~\citep{tian2020understanding,tian2021understanding}. Another interesting point to note is that while such an explanation would indicate that highly co-adapted features are beneficial in SSL, such features can be adverse in value-based RL as discussed in Section~\ref{sec:problem}. 

\textbf{Preventing divergence in deep TD-learning.} Finally, we discuss \citet{achiam2019towards} which proposes to pre-condition the TD-update using the inverse the neural tangent kernel~\citep{ntk} matrix so that the TD-update is always a contraction, for every $\theta_k$ found during TD-learning. Intuitively, this can be overly restrictive in several cases: we do not need to ensure that TD always contracts, but that is eventually stabilizes at good solution over long periods of running noisy TD updates, Our implicit regularizer (Equation`\ref{eqn:regularizer}) derives this condition, and our theoretically-inspired \methodname\ regularizer shows that empirically, it suffices to penalize the dot product similarity in practice.   

% A number of prior works have empirically studied several issues in deep RL: interference~\citep{achiam2019towards,bengio2020interference}, impact of data distributions~\citep{fu2019diagnosing,kumar2020discor,du2019distributioncheck}, inability to use big networks~\citep{bjorck2021towards,sinha2020d2rl,ota2021training} and non-stationarity~\citep{igl2020impact,fedus2020catastrophic}. Convergence of deep RL has also been shown under idealistic assumptions~\citep{yang2020theoretical,cai2019neural,zhang2020can,xu2019finite}. However, these analyses generally focus on orthogonal factors and do not analyze any form of implicit regularization effects in TD learning. While not a direct focus of our work, several methods have been proposed to handle distributional shift in offline RL (see \citep{levine2020offline} for a review) and our method, \methodname\ can be applied on many of these methods as shown in our experiments in Section~\ref{sec:experiments}.

%%SL.9.29: can probably cut out most of this section, really only the discussion of kumar2021implicit is important
% \citet{kumar2021implicit} studies the learning dynamics of Q-learning and observes that the rank of the feature matrix, $\Phi$, drops during training. While this observation is related, our analysis characterizes the implicit preference of learning towards feature co-adaptation (Theorem~\ref{thm:implicit_noise_reg}) on out-of-sample actions as the primary culprit for aliasing and rank collapse. Additionally, utilizing \methodname\ not only outperforms the $\srank(\Phi)$ penalty in \citet{kumar2021implicit} by more than \textbf{100\%}, but it also alleviates rank collapse, with no apparent term that explicitly increases rank. 
% %%AK: not sure if we want to discuss lyle2021?
% % Prior work \citep{lyle2021effect} has also studied the impact of auxilliary tasks on feature learning, which is orthogonal to the problem we study. 
% \citet{zanette2020exponential} identifies hard instances for offline TD learning of linear value functions when the provided features are ``aliased''. Aliased features may inhibit learning along certain dimensions of the feature space necessitating an exponential sample size even under strong coverage assumptions. A combination of \citet{zanette2020exponential}'s argument and our analysis would imply that the co-adaptation caused by implicit regularization can lead to failure modes from the hard instance. Additionally, \citet{durugkar2018td,pohlen2018observe} attempt to constrain gradients of TD heuristically to encourage more stationary updates by preventing the target values from changing. Finally, a similar co-adaptation also arises due to implicit regularization in self-supervised learning (SSL) methods based on bootstrapping~\citep{grill2020bootstrap,chen2020exploring} allowing the network to learn representations invariant to spurious correlations~\citep{tian2021understanding,tian2020understanding}. While this is beneficial in SSL, it can be adverse in value-based RL as discussed in Section~\ref{sec:problem}. 


\vspace{-0.25cm}
\section{Proof of Theorem~\ref{thm:implicit_noise_reg}}
\vspace{-0.25cm}
\label{app:proofs}

In this section, we will derive our implicit regularizer $R_\mathrm{TD}(\theta)$ that emerges when performing TD updates with a stochastic noise model with covariance matrix $M$. We first introduce our notation that we will use throughout the proof, then present our assumptions and finally derive the regularizer. Our proof utilizes the analysis techniques from \citet{blanc2020implicit} and \citet{damian2021label}, which analyze label-noise SGD for supervised learning, however key modifications need to be made to their arguments to account for non-symmetric matrices that emerge in TD learning. As a result, the form of the resulting regularizer is very different. To keep the proof concise, we will appeal to lemmas from these prior works which will allow us to bound certain concentration terms. 

\subsection{Notation}
The noisy TD-learning update for training the Q-function is given by:
\begin{align}
    \!\!\!\theta_{k+1} = \theta_k - \eta \underbrace{\left( \sum_i \nabla_\theta Q(\bs_i, \ba_i) \left(Q_\theta(\bs_i, \ba_i)\!- \!(r_i\!+\!\gamma {Q}_{\theta}(\bs'_i, \ba'_i))\right) \right)}_{:= g(\theta)}\!+\!\eta \varepsilon_k,  ~~~~ \varepsilon_k \sim \mathcal{N}(0, M)
% \label{eq:td_update_appendix}
\end{align}
% \vspace{-0.5cm}
where $g(\theta)$ denotes the parameter update. Note that $g(\theta)$ is not a full gradient of a scalar objective, but it is a form of a ``pseudo''-gradient or ``semi''-gradient. Let $\varepsilon_k$ denote an  i.i.d.random noise that is added to each update. This noise is sampled from a zero-mean Gaussian random variable with covariance matrix $M$, i.e., $\mathcal{N}(0, M)$. 

Let $\theta^*$ denote a point in the parameter space such that in the vicinity of $\theta^*$, $g(\theta) \leq \mathscr{C}$, for a small enough $\mathscr{C}$. Let $G(\theta)$ denote the derivative of $g(\theta)$ w.r.t. $\theta$: $G(\theta) = \nabla_\theta g(\theta)$ and let $\nabla G(\theta)$ denote the third-order tensor $\nabla^2_\theta g(\theta)$. For notation clarity, let $G = G(\theta^*), \nabla G = \nabla G (\theta^*)$. Let $e_i$ denote the signed TD error for a given transition $(\bs_i, \ba_i, \bs'_i) \in \mathcal{D}$ at $\theta^*$: 
\begin{align}
e_i = Q_{\theta^*}(\bs_i, \ba_i) - (r_i + \gamma Q_{\theta^*}(\bs'_i, \ba'_i)).
\end{align}
Since $\theta^*$ is a fixed point of the training TD error, $e_i = 0$. Following \citet{blanc2020implicit}, we will assume that the learning rate in gradient descent, $\eta$, is small and we will ignore terms that scale as $\mathcal{O}(\eta^{1 + \delta})$, for $\delta > 0$. Our proof will rely on using a reference Ornstein-Uhlenbeck (OU) process which the TD parameter iterates will be compared to. Let $\zeta_k$ denote the $k$-th iterate of an OU process, which is defined as:
\vspace{-0.1in}
\begin{equation}
    \label{eqn:ou_process}
    \zeta_{k+1} = (I - \eta G) \zeta_k + \eta \varepsilon_k, ~~~ \varepsilon_k \sim \mathcal{N}(0, M)
\end{equation}
We will drop $\theta$ from $\nabla_\theta$ to indicate that the gradient is being computed at $\theta^*$, and drop $(\bs_i, \ba_i)$ from $Q(\bs_i, \ba_i)$ and instead represent it as $Q_i$ for brevity; we will represent $Q(\bs'_i, \ba'_i)$ as $Q'_i$. We assume that $\nabla^2 Q_i$ is $\mathscr{L}_2$-Lipschitz and $\nabla^3 Q_i$ is $\mathscr{L}_3$-Lipschitz throughout the parameter space $\Theta$.

\subsection{Proof Strategy} 
For a given point $\theta^*$ to be an attractive fixed point of TD-learning, our proof strategy would be to derive the condition under which it mimics a given OU noise process, which as we will show stays close to the parameter $\theta^*$. This condition would then be interpreted as the gradient of a ``induced'' implicit regularizer. If the point $\theta^*$ is not a stationary point of this regularizer, we will show that the movement $\theta$ is large when running the noisy TD updates, indicating that the regularizer, at least in part guides the dynamics of TD-learning. To show this, we would write out the gradient update, isolate some terms that will give rise to the implicit regularizer, and bound the remaining terms using contraction and concentration arguments. The contraction arguments largely follow prior work (though with key exceptions in handling contraction with asymmetric and complex eigenvalue matrices), while the form of the implicit regularizer is different. Finally, we will interpret the resulting update over large timescales to show that learning is indeed guided by the implicit regularizer.   

\subsection{Assumptions and Conditions}
Next, we present some key assumptions we will need for the proof. Our first assumption is that the matrix $G \in \mathbb{R}^{d \times d}$ is of maximal rank possible, which is equal to the number of datapoints $n$ and $n \ll d$, the dimensionality of the parameter space. Crucially, this assumption do not imply that $G$ is of full rank -- it cannot be, because we are in the overparameterized regime. 
\begin{assumption}[$G$ spans an $n$-dimensional basis.]
\label{assumption:psd}
Assume that the matrix $G$ spans $n$-possible directions in the parameter space and hence, attains the maximal possible rank it can.
\end{assumption}

The second condition we require is that the matrices $\sum_i \nabla Q_i \nabla Q_i^\top$ and $M$ share the same $n$-dimensional basis as matrix $G$:
\begin{assumption}
\label{assumption:shared_basis}
$\sum_i \nabla Q_i \nabla Q_i^\top$, $M$, and $G$ span identical $n$-dimensional subspaces.
\end{assumption}
This is a technical condition that is required. If this condition is not met, as we will show the learning dynamics of noisy TD will not be a contraction in certain direction in the parameter space and TD-learning will not stabilize at such a solution $\theta^*$. We will utilize a stronger version of this statement for TD-learning to converge, and we will discuss this shortly.

\subsection{Lemmas Used In The Proof}
Next, we present some lemmas that would be useful for proving the theoretical result.  

\begin{lemma}[Expressions for the first and and second-order derivatives of $g(\theta)$.]
\label{lemma:useful}
The following definitions and expansions apply to our proof:
\begin{align*}
    G(\theta^*) &= \sum_{i} \nabla^2 Q_i e_i + \sum_i \nabla Q_i (\nabla Q_i - \gamma \nabla Q'_i)^\top\\
    \nabla G (\theta^*) [\bv, \bv] &= 2 \sum_i \nabla^2 Q_i \bv \bv^\top (\nabla Q_i - \gamma \nabla Q'_i) + \sum_i \mathrm{tr}\left((\nabla^2 Q_i - \gamma \nabla^2 Q'_i) \bv \bv^\top\right) \nabla Q_i + \nabla^3 Q_i e_i  
\end{align*}
\end{lemma}
Lemma~\ref{lemma:useful} presents a decomposition of the matrix $G$ and the directional derivative of the third order tensor $\nabla G [\bv, \bv]$ in directions $\bv$ and $\bv$, which will appear in the Taylor expansion layer. Note that at $\theta^*$ since $e_i = 0$, the first term in $G(\theta^*)$ and the third term in $\nabla G(\theta^*)[\bv, \bv]$ vanish.
Lemma~\ref{lemma:covariance_noise} derives a fixed-point recursion for the covariance matrix of the total noise accumulated in the OU-process with covariance matrix $M$ and this will appear in our proof.  

\begin{lemma}[Covariance of the random noise process $\zeta_k$]
\label{lemma:covariance_noise}
Let $\zeta_{k}$ denote the OU process satisfying: $\zeta_{k+1} = (I - \eta G) \zeta_k + \eta \varepsilon_k$, where $\varepsilon_k \sim \mathcal{N}(0, M)$, where $M \succcurlyeq 0$. Then, $\zeta_{k+1} \sim \mathcal{N}(0, \Sigma)$, where $\Sigma$ satisfies the discrete Lyapunov equation: 
\begin{equation*}
    \Sigma^*_M = (I - \eta G) \Sigma^*_M (I - \eta G)^\top + \eta^2 M.
\end{equation*}
\end{lemma}
\begin{proof}
For the OU process, $\zeta_{k+1} = (I - \eta G) \zeta_k + {\eta} \varepsilon_k$, since $\varepsilon_k$ is a Gaussian random variable, by induction so is $\zeta_{k+1}$, and therefore the covariance matrix of $\zeta_{k+1}$ is given by:
% \vspace{-0.1in}
\begin{align}
    \Sigma_{k+1} := (I - \eta G) \Sigma_k (I - \eta G^\top) + \eta^2 M.
\end{align}
% \vspace{-0.1in}
Solving for the fixed point for $\Sigma_k$ gives the desired expression.
\end{proof}

In our proofs, we will require the following contraction lemmas to tightly bound the magnitude of some zero-mean terms that will appear in the noisy TD update under certain scenarios. Unlike the analysis in \citet{damian2021label} and \citet{blanc2020implicit} for supervised learning with label noise, where the contraction terms like $(I - \eta G)^k G$ are bounded by $\approx \frac{1}{k \eta}$ intuitively because $I - \eta G$ is a contraction in the subspace spanned by matrix $G$. However, this is not true for TD-learning directly since terms like $(I - \eta G)^k S$ appear for a different matrix $S$. Therefore, TD-learning will diverge from $\theta^*$ unless matrices $G$ and $M$ have their corresponding eigenvectors assigned to the top eigenvalues be approximately ``aligned''. We formalize this definition next, and then provide a proof of the concentration guarantee. 

\begin{definition}[$(\omega, C_0)$-alignment]
\label{def:alignment}
Given a positive semidefinite matrix $A$, let $A = U_A \Lambda_A U_A^\top$ denote its eigendecomposition. Without loss of generality assume that the eiegenvalues are arranged in decreasing order, i.e., $\forall i > j, \Lambda_A(i) \leq \Lambda_A(j)$. Given another matrix $B$, let $B = U_B \Lambda_B U_B^H$ denote its complex eigendecomposition, where eigenvalues in $\Lambda_B$ are arranged in decreasing order of their complex magnitudes, i.e., $\forall i > j, |\Lambda_B(i)| \leq |\Lambda_B(j)|$. Then the matrix pair $(A, B)$ is said to be $(\omega, C_0)$-aligned if $|U_B^H(i) U_A(i)| \leq \omega$ and if $\forall~ i, \Lambda_A(i) \leq C_0 |\Lambda_B(i)|$ for a constant $C_0$. 
\end{definition}
If two matrices are $(\omega, C_0)$-aligned, this means that the corresponding eigenvectors when arranged in decreasing order of eigenvalue magnitude roughly align with each other, per the definition of alignment above. This condition would be crucial while deriving the implicit regularizer as it will quantify the rate of contraction of certain terms that define the neighborhood that the iterates of noisy TD-learning will lie in with high probability. We will operate in the setting when the matrix $G$ and $\sum_i \nabla Q_i \nabla Q_i^\top$ are $(\omega, C_0)$-aligned with each other, and matrix $M$ and $G$ are also $(\omega, C_0)$-aligned (note that we can consider $\omega', C'_0)$, which will not change our bounds and therefore we go for less notational clutter). Next we utilize this notion of alignment to show a particular contraction bound that extends the weak contraction bound in \citet{damian2021label}.  

\begin{lemma}
\label{lemma:contraction}
Assume we are given a matrix $G$ such that $|\lambda_i(I - \eta G)| \leq \rho_0 < 1$ for all $\lambda_i$ such that $\lambda_i \neq 0$. Let $G = U \Lambda U^H$ be the complex eigenvalue decomposition of $G$ (since almost every matrix is complex-diagonalizable). For a positive semi-definite matrix $S$ that is $(\omega, C_0)$-aligned with $G$, if $S = U_S \Lambda_S U_S^\top$ is its eigenvalue decomposition, the following contraction bound holds: 
\begin{align*}
    ||(I - \eta G)^k S||  = \mathcal{O}\left(\frac{\omega C_0}{\eta k}\right)
\end{align*}
\end{lemma}
\begin{proof}
To prove this statement, we can expand $(I - \eta G)$ using its eigenvalue decomposition only in the subspace that is jointly shared by $G$ and $M$, and then utilize the definition of $\omega$-alignment to bound the terms.
\begin{align}
    ||(I - \eta G)^k S|| &= ||(I - \eta U \Lambda U^H)^k U_S \Lambda_S U_S^\top ||\\
    &= \left\vert \left\vert (U U^H - \eta U \Lambda_U U^H)^k U_S \Lambda_S U_S^\top \right\vert \right\vert \\
    &= \leftnorm U \left(I - \eta \Lambda \right)^k U^H U_S \Lambda_S U_S^\top \rightnorm\\
    &\leq \omega \cdot ||\left(I - \eta \Lambda\right)^k|| \cdot \Lambda_S \\
    & \leq \omega \cdot C_0 \cdot \left( \max_i~~ |1 - \eta \Lambda (i)|^k |\Lambda(i)|\right)
\end{align}
Now we need to solve for the inner maximization term. When $\Lambda(i)$ is not complex for any $i$, the term above is $\lesssim 1/\eta k$ by applying the result from \citet{damian2021label}, but when $\Lambda(i)$ is complex, this bound can only hold under certain conditions. To note when this quantity is bounded, we expand $|1 - \eta x|^k$ for some complex number $x = r (\cos \theta + \iota \sin \theta) $:
\begin{align}
|1 - \eta x|^k &= \left\vert\left(1 - \eta r \cos \theta \right) + \iota \eta r \sin \theta \right\vert \\
&= \left[\sqrt{\left(1 - \eta r \cos \theta\right)^2 + \eta^2 r^2 \sin^2 \theta}\right]^k = \left(1 +\eta^2 r^2 - 2 \eta r \cos \theta\right)^{k/2}\\
\implies |1 - \eta x|^k |x| &= \left(1 +\eta^2 r^2 - 2 \eta r \cos \theta\right)^{k/2} r\\
& \lesssim \frac{1}{\eta k}~~~~\text{if}~ \eta \leq \min_{i} \frac{\mathrm{Re}(\Lambda(i))}{|\Lambda(i)|} ~~~~~\text{and}~~~~ \infty~~\text{otherwise}.
\end{align}
Plugging back the above expression in the bound above completes the proof.
\end{proof}

The proof of Lemma~\ref{lemma:contraction} indicates that unless the learning rate $\eta$ and the matrix $G$ are such that the $|\lambda_i(I - \eta G)| \leq \rho < 1$ in directions spanned by matrix $S$, such an expression may not converge. This is expected since the matrix $I - \eta G$  will not contract in directions of non-zero eigenvalues if the real part $r \cos \theta$ is negative or zero. Additionally, we note that under Definition~\ref{def:alignment}, we can extend several weak-contraction bounds from \citet{damian2021label} (Lemmas 9-14 in \citet{damian2021label}) to our setting. 

Next, Lemma~\ref{lemma:bounded_noise} shows that the OU noise iterates are bounded with high probability when Definition~\ref{def:alignment} holds:
\begin{lemma}[$\zeta_k$ is bounded with high probability]
\label{lemma:bounded_noise}
With probability atleast $1 - \delta$ and under Definition~\ref{def:alignment}, $||\zeta_k|| \leq n \omega \sqrt{\eta C_0} \log \frac{1}{\delta} = \mathcal{O}(\sqrt{\eta})$. 
\end{lemma}
\begin{proof}
To prove this lemma, we first bound the trace of the covariance matrix $\Sigma_{k+1}$ and then apply high probability bounds on the Martingale norm concentration. The trace of the covariance matrix $\Sigma_{k+1}$ can be bounded as follows (all the equations below are restricted to the dimensions of non-zero eigenvalues of $G$):
\begin{align}
    \mathrm{tr}\left[\Sigma_{k+1} \right] &= \sum_{j \leq k} \mathrm{tr}\left[(I - \eta G)^j M (I - \eta G^\top)^j \right]\\
    &= \sum_{j \leq k} \mathrm{tr}\left[ (U U^H - \eta U \Lambda U^H)^j M (U U^H - \eta U \Lambda U^H)^j \right]\\
    &= \sum_{j \leq k} \mathrm{tr}\left[ U (I - \eta \Lambda)^j U^H U_M \Lambda_M U_M^\top U (I - \eta \Lambda)^j U^H \right]\\
    &= \sum_{j \leq k} n \omega^2 C_0 \mathrm{tr}\left[ |I - \eta \Lambda|^j \cdot |\Lambda| \cdot |I - \eta \Lambda|^j \right]\\
    &\leq n \omega^2 C_0 \sum_{j \leq k} n \cdot \max_\lambda (|1 - \eta \lambda|^{2j} \cdot  |\lambda|) \leq {\eta n^2 C_0 \omega^2}
\end{align}
Now, we can apply Corollary 1 from \citet{damian2021label} to obtain a bound on $||\zeta_k||$ as with high probability, atleast $1 - \delta$, $||\zeta_k|| \leq \sqrt{2 \mathrm{tr}(\Sigma) \log \frac{1}{\delta}} = n \omega \sqrt{\eta C_0} \log \frac{1}{\delta}$.
\end{proof}

\subsection{Main Proof of Theorem~\ref{thm:implicit_noise_reg}}
In this section, we present the main proof of Theorem~\ref{thm:implicit_noise_reg}. The proof involves two components: \textbf{(1)} the part where we derive the regularizer, and \textbf{(2)} bounding additional terms via concentration inequalities. Part \textbf{(1)} is specific to TD-learning, while a lot of the machinery for part \textbf{(2)} is directly taken from prior work~\citep{damian2021label} and \citet{blanc2020implicit}. We focus on part \textbf{(1)} here.

Our strategy is to analyze the learning dynamics of noisy TD updates that originate at $\theta^*$. In a small neighborhood around $\theta^*$, we can expand the noisy TD update (Equation~\ref{eq:td_update}) using Taylor's expansion around $\theta^*$ which gives:
\begin{align}
    \label{eqn:nu_k_app}
    &\theta_{k+1} = \theta_k - \eta g(\theta_k)+ \eta \varepsilon_k, ~~ \varepsilon_k \sim \mathcal{N}(0, M)\\
    \implies &\theta_{k+1} = \theta_k - \eta \left( g + G (\theta_k - \theta^*) - \frac{\eta}{2} G [\theta_k - \theta^*, \theta_k - \theta^*] \right) + \eta \varepsilon_k + \mathcal{O}(\eta ||\theta_k - \theta^*||^3).
\end{align}
Denoting $\nu_k := \theta_k - \theta^*$, using the fact that $||g(\theta^*)|| \leq \mathscr{C}$, we find that $\nu_k$ can be written as:
\begin{align}
\label{eqn:nu_k_appe}
    \nu_{k+1} &= (I - \eta G) \nu_k + \varepsilon_k + \frac{\eta}{2} G [\nu_k, \nu_k] + \mathcal{O}(\eta ||\nu_k||^3 + \eta\mathscr{C})
\end{align}
Since the OU process $\zeta_k$ stays in the vicinity of the point $\theta^*$, and follows a similar recursion to the one above, our goal would be to design a regularizer so that Equation~\ref{eqn:nu_k_appe} closely follows the OU process. Thus, we would want to bound the difference between the variable $\nu_k$ and the variable $\zeta_k$, denoted as $r_k$ to be within a small neighborhood:
\begin{align*}
r_{k+1} = \nu_{k+1} - \zeta_{k+1} = (I - \eta G) \underbracket{(\nu_k - \zeta_k)}_{r_k} + \frac{1}{2} G [\nu_k, \nu_k] + \mathcal{O}(\eta ||\nu_k||^3 + \eta \mathscr{C}). 
\end{align*}
We can write down an expression for $r_k$ summing over all the terms:
\begin{equation}
\label{eqn:r_k}
    r_{k+1} = - \underbracket{\frac{\eta}{2} \sum_{j \leq k} (I - \eta G)^{k - j} \nabla G [\nu_k, \nu_k]}_{\text{term (a)}} + \underbracket{\sum_{j \leq k} (I - \eta G)^j \left[\mathcal{O}(\eta ||\nu_k||^3 + \eta \mathscr{C}) \right]}_{\text{term (b)}}.
\end{equation}
Term (a) in the above equation is the one that can induce a displacement in $r_k$ as $k$ increases and would be used to derive the regularizer, whereas term (b) primarily consists of terms that concentrate to $0$. We first analyze term (a) and then we will analyze the concentration terms later. 

To analyze term (a), note that the term $\nabla G [\nu_k, \nu_k]$, by Lemma~\ref{lemma:useful}, only depends on $\nu_k$ via the covariance matrix $\nu_k \nu_k^\top$. So we will partition this term into two terms: \textbf{(i)} a term that utilizes the asymptotic covariance matrix of the OU process and \textbf{(ii)} errors due to a finite $k$ and stochasticity that will concentrate.
\begin{align}
    2 \times \text{(a)}~ &= \eta  \sum_{j \leq k} (I - \eta G)^{k - j} \nabla G [\nu_k, \nu_k]\\
    \label{eqn:random1}
    &= \sum_{j \leq k} (I - \eta G)^{k - j} \nabla G [\zeta^*, \zeta^*] + \sum_{j \leq k} (I - \eta G)^{k - j} \nabla G ([\nu_k, \nu_k] - [\zeta^*, \zeta^*]),
\end{align}
The first term is a ``bias'' term and doesn't concentrate to $0$, and will give rise to the regularizer. We can break this term using Lemma~\ref{lemma:useful} as:
\begin{align}
\label{eqn:b_9}
   \!\!\!\!\!\!\!\!\!\!\!\!\nabla G [\zeta^*, \zeta^*] =& 2 \sum_i \nabla^2 Q_i \Sigma^*_M (\nabla Q_i - \gamma \nabla Q'_i) + \sum_i \mathrm{tr}\left[(\nabla^2 Q_i - \gamma \nabla^2 Q'_i) \Sigma^*_M \right] \nabla Q_i 
\end{align}
The regularizer $R_\mathrm{TD}(\theta)$ is the function such that:
\begin{align}
    \label{eqn:derive_regularizer}
    \nabla_\theta R_\mathrm{TD}(\theta) & = \sum_i \nabla^2 Q_i \Sigma^*_M (\nabla Q_i - \gamma \nabla Q'_i)\\
    \label{eqn:regularizer_fn_app}
    \implies R_\mathrm{TD}(\theta) &= \sum_i \nabla Q_i \Sigma^*_M \nabla Q_i^\top - \gamma \sum_i \mathrm{trace}\left(\Sigma^*_M \nabla Q_i [[\nabla Q'_i]]^\top\right),
\end{align}
where $[[\cdot]]$ denotes the stop gradient operator. If the point $\theta^*$ is a stationary point of the regularizer $R_\mathrm{TD}(\theta)$, then Equations~\ref{eqn:derive_regularizer} and \ref{eqn:regularizer_fn_app} imply that the first term of Equation~\ref{eqn:b_9} must be 0. Therefore in this case to show that $\theta^*$ is attractive, we need to show that the other terms in Equations~\ref{eqn:b_9}, \ref{eqn:random1} and term (b) in Equation~\ref{eqn:r_k} concentrate around $0$ and are bounded in magnitude. The remaining part of the proof shown in Appendix~\ref{app:concentrating} provides these details, but we first summarize the main takeaways in the proof to conclude the argument.

\subsection{Summary of the Argument}
We will show how to concentrate terms in Equation~\ref{eqn:derive_regularizer} besides the regularizer largely following the techniques from prior work, but we first summarize the entire proof. The overall update to the vector $r_k$ which measures the displacement between the parameter vector $\theta_k - \theta^*$ and the OU-process $\zeta_k$ can be written as follows, and it is governed by the derivative of the implicit regularizer (modulo error terms):
\begin{equation}
\label{Eqn:rk_final}
    r_{k+1} = - \frac{\eta}{2} \sum_{j \leq k} (I - \eta G)^{k - j} \nabla_\theta R_\mathrm{TD}(\theta^*) + \mathcal{O}\left(\sqrt{\eta t} \cdot \mathrm{poly}(\mathscr{C}, \mathscr{L}_2, \mathscr{L}_3, \omega, C_0)\right).   
\end{equation}
An important detail to note here is that since the regularizer consists of $\Sigma^*_M$ and the size of $\Sigma^*_M$ (i.e, its eigenvalues), as shown in Lemma~\ref{lemma:bounded_noise} depends on one factor of $\eta$. So, effectively the first term in Equation~\ref{Eqn:rk_final} does depend on two factors of $\eta$. Using Equation~\ref{Eqn:rk_final}, we can write the deviation between $\theta^*$ and $\theta_k$ as:
\begin{align}
\nu_{k+1} &= \zeta_{k+1} - \frac{\eta}{2} \sum_{j \leq k} (I - \eta G)^{k - j} \nabla_\theta R_\mathrm{TD}(\theta^*) + \mathcal{O}\left(\sqrt{\eta t} \cdot \mathrm{poly}(\mathscr{C}, \mathscr{L}_2, \mathscr{L}_3, \omega, C_0)\right).
\end{align}
The OU process $\zeta_k$ converges to $\theta^*$ in the subspace spanned by $G$, since the condition $\rho(I - \eta G) < 1$ is active in this subspace (if the condition that $\rho(I - \eta G) < 1$ in the subspace spanned by $G$ is not true, then as \citet{ghosh2020representations} show, TD can diverge). Now, given $G$ satisfies this spectral radius condition, $\zeta_k$ would converge to $\theta^*$ within a timescale of $\mathcal{O}\left(\frac{1}{\eta} \right)$ within this subspace, which as \citet{blanc2020implicit} put it is the strength of the ``mean-reversion'' term. On the remaining directions (note that $d \gg n$), the dynamics is guided by the regularizer, although with a smaller weight of $\eta^2$. 

% Thus if the point $\theta^*$ is not the stationary point of the regularizer $R_\mathrm{TD}(\theta)$, then we will see that the vector $r_k$ will increase linearly in $k$ as $k$ increases. Note the other conditions which are still required: \textbf{(1)} we require the spectral radius $\rho(I - \eta G) \leq 1$, with $\rho(I - \eta G) \leq \rho_0 = 1$ in directions with non-zero eigenvalues of $G$, and \textbf{(2)} $(\omega, C_0)$-contraction to ensure that various terms besides the regularizer (which is like a ``bias'' term) contract at a $\sqrt{1/\eta}$ rate. If condition \textbf{(1)} is not true, then TD learning will diverge, whereas if condition \textbf{(2)} is not true, then the learning dynamics would be dominated by \todo{fill}.     

\subsection{Additional Proof Details: Concentrating Other Terms}
\label{app:concentrating}
We first concentrate the terms in Equation~\ref{eqn:b_9}. The cumulative effect of the second term in Equation~\ref{eqn:b_9} is given by:
\begin{align}
    &\eta \sum_{j \leq k} (I - \eta G)^{j-k} \nabla Q_i \mathrm{tr}\left[ (\nabla^2 Q_i - \gamma \nabla^2 Q'_i) \Sigma^*_M \right]\\ 
    &\leq \eta \sum_{j \leq k} (I - \eta G)^{j-k} \nabla Q_i \cdot \mathcal{O}\left( \mathscr{L}_2 (1+ \gamma) \sigma \right) \leq \mathcal{O}\left( \eta \sqrt{\frac{k}{\eta}} \omega_0 C_0  \mathscr{L}_2 (1 + \gamma) \sigma \right), 
\end{align}
which follows from the fact that $\nabla^2 Q_i$ is $\mathscr{L}_2$-Lipschitz, and using Lemma~\ref{lemma:contraction} for contracting the remaining terms.

Next, we turn to concentrating the second term in Equation~\ref{eqn:random1}. This term corresponds to the contribution of difference between the empirical covariance matrix  $\nu_k \nu_k^\top$ and the asymptotic covariance matrix $\zeta^* \zeta^{* \top}$. We expand this term below using the form of $G$ from Lemma~\ref{lemma:useful}, and bound it one by one.
\begin{align}
    &\sum_{j \leq k} (I - \eta G)^{k - j} \nabla G ([\nu_k, \nu_k] - [\zeta^*, \zeta^*])\\
    & = \sum_{j \leq k} \sum_i (I - \eta G)^{k-j} \nabla^2 Q_i \left(\nu_k \nu_k - \zeta^* \zeta^{* \top}\right) (\nabla Q_i - \gamma \nabla Q'_i) + \mathcal{O}\left(\sqrt{\eta k} \omega_0 C_0 \mathscr{L}_2 (1 + \gamma) \sigma \right)
    \label{eqn:remaining}
\end{align}
Now, we note that the term $\Delta_{k+1} := \nu_{k+1} \nu_{k+1}^\top - \zeta^* \zeta^{* \top}$ can itself be written as a recursion:
\begin{align}
    \Delta_{k+1} &= (I - \eta G) (\Delta_k) (I - \eta G)^\top + \underbracket{(I - \eta G) \zeta_k \varepsilon^\top  + \varepsilon \zeta_k^\top (I - \eta G)^\top}_{A_k} + \underbracket{\varepsilon \varepsilon^\top - \eta M}_{B_k}     
\end{align}
Expanding the term $\Delta_{k+1}$ in terms of a summation over $k$, and plugging it into the expression from Equation~\ref{eqn:remaining} we get
\begin{align}
\label{eqn:remaining2}
    \sum_{i} \sum_{j \leq k} (I - \eta G)^{k-j} & \nabla^2 Q_i (I - \eta G)^j \Delta_0 (I - \eta G^\top)^j \\ 
    + \sum_i &~ \sum_{j \leq k} \sum_{p \leq j} (I - \eta G)^{k-j} \nabla^2 Q_i (I - \eta G)^{j-p-1} (A_p + B_p) (I - \eta G^\top)^{j-p-1} \nonumber
\end{align}
Now by noting that if $G$ and $\nabla Q_i$ are $(\omega, C_0)$-aligned, then so are $G^\top$ and $\nabla Q_i$, we can finish the proof by repeating the calculations used by \citet{damian2021label} (Appendix B, Equations 67-73) to bound the terms in Equation~\ref{eqn:remaining2} by $\mathcal{O}(\sqrt{\eta k})$, but with an additional factor of $\omega^2 C_0^2$.  

\textbf{Term (b) in Equation~\ref{eqn:r_k}.} When $\mathscr{C}$ is small enough, we can bound the term (b) using $\mathcal{O}(\sqrt{\eta k})$, similar to \citet{damian2021label}.  




\section{Proof of Proposition~\ref{thm:co_adapted_features_are_bad}}
\label{app:new_thm}
In this section, we will prove Proposition~\ref{thm:co_adapted_features_are_bad}. First, we refer to Proposition 3.1 in \citet{ghosh2020representations}, which shows that TD-learning is stable and converges if and only if the matrix $M_\phi = \Phi^\top (\Phi - \gamma \Phi')$ has eigenvalues with all positive real entries. Now note that if, 
\begin{align}
    \sum_{\bs, \mathbf{a}} \phi(\bs, \mathbf{a})^\top \phi(\bs, \mathbf{a}) &\leq \gamma \sum_{\bs, \mathbf{a}, \bs'} \phi(\bs', \mathbf{a}')^\top \phi(\bs, \mathbf{a})\\
    \implies \mathrm{trace} \left(\Phi^\top \Phi\right) &\leq \gamma \mathrm{trace}\left(\Phi^\top \Phi'\right)\\
    \implies \mathrm{trace}\left[\Phi^\top \left(\Phi - \gamma \Phi'\right) \right] \leq 0. 
\end{align}
Since the trace of a real matrix is the sum of real components of eigenvalues, if for a given matrix $M$, $\mathrm{trace}(M) \leq 0$, then there exists atleast one eigenvalue $\lambda_i$ such that $\mathrm{Re}(\lambda_i) \leq 0$. If $\lambda_i < 0$, then the learning dynamics of TD would diverge, while if $\lambda_i = 0$ for all $i$, then learning will not contract towards the TD fixed point. This concludes the proof of this result.


\section{Experimental Details of Applying \drmethodname}
\label{app:additional_background}

In this section, we discuss the practical experimental details and hyperparameters in applying our method, \drmethodname\ to various offline RL methods. We first discuss an overview of the offline RL methods we considered in this paper, and then provide a discussion of hyperparameters for \drmethodname.

\subsection{Background on Various Offline RL Algorithms}
\label{app:details_algo}

In this paper, we consider four base offline RL algorithms that we apply DR3 on. These methods are detailed below: 

\textbf{REM}. Random ensemble mixture~\citep{agarwal2019optimistic} is an uncertainty-based offline RL algorithm uses multiple parameterized Q-functions to estimate the Q-values. During the Bellman backup, REM computes a random convex combination of the target Q-values and then trains the Q-function to match this randomized target estimate. The randomized target value estimate provides a robust estimate of target values, and delays unlearning and performance degradation that we typically see with standard DQN-style algorithms in the offline setting. {For instantiating REM}, we follow the instantiation provided by the authors and instantiate a multi-headed Q-function with 200 heads, each of which serves as an estimate of the target value. These multiple heads branch off the last-but-one layer features of the base Q-network.
The objective for REM is given by:
\begin{equation}
\!\!\!\!\!\!\!\!\!\min_{\theta} \expected_{\bs, \mathbf{a}, r, \bs' \sim \mathcal{D}} \left[ \expected_{\alpha_{1}, \dots, \alpha_{K} \sim {\Delta}} \left[ \ld \left(
\sum_{k} \alpha_{k} \Qt^{k}(\bs, \mathbf{a}) - r - \gamma\max_{\mathbf{a}'} \sum_{k} \alpha_{k}\Qtp^{k}(\bs', \mathbf{a}') \right) \right]\right] \label{eq:sqn}
\end{equation} where $l_\lambda$ denotes the Huber loss while $P_\Delta$ denotes the probability distribution over the standard (K − 1)-simplex.

\textbf{CQL}. Conservative Q-learning~\citep{kumar2020conservative} is an offline RL algorithm that learns a conservative value function such that the estimated performance of the policy under this learned value function lower-bounds its true value. CQL modifies the Q-function training to incorporate a term that minimizes the overestimated Q-values in expectation, while maximizing the Q-values observed in the dataset, in addition to standard TD error. This CQL regularizer is typically multiplied by a coefficient $\alpha$, and we pick $\alpha=0.1$ for all our Atari experiments following \citet{kumar2021implicit} and $\alpha=5.0$ for all our kitchen and antmaze D4RL experiments. Using $\overline{y}_k(\bs, \mathbf{a})$ to denote the target values computed via the Bellman backup (we use actor-critic backup for D4RL experiments and the $\max_{\mathbf{a}'}$ backup for standard Q-learning in our Atari experiments following \citet{kumar2020conservative}), the objective for training CQL is given by: 
\begin{equation*}
    \!\!\small{\min_{Q} \alpha \left(\E_{\bs \sim \mathcal{D}}\left[\log \sum_{\mathbf{a}} \exp(Q(\bs, \mathbf{a}))\right] - \E_{\bs, \mathbf{a} \sim \mathcal{D}}\left[Q(\bs, \mathbf{a})\right] \right)\! +\! \frac{1}{2}\! \E_{\bs, \mathbf{a}, \bs' \sim \mathcal{D}}\left[\left(Q(\bs, \mathbf{a}) - \overline{y}_k(\bs, \mathbf{a}) \right)^2 \right]}.
\end{equation*}
For Atari, we utilize the standard convolutional neural network from \citet{agarwal2019optimistic,kumar2021implicit} with 3 convolutional layers borrowed from the nature DQN network and then a hidden feedforward layer of size $512$.


\textbf{COG}. COG~\citep{singh2020cog} is an algorithmic framework for utilizing large, unlabeled datasets of diverse behavior to learn generalizable policies via offline RL. Similar to real-world scenarios where large unlabeled datasets are available alongside limited task-specific data, the agent is provided with two types of datasets. The task-specific dataset consists of behavior relevant for the task, but the prior dataset can consist of a number of random or scripted behaviors being executed in the same environment/setting. The goal in this task is to actually stitch together relevant and overlapping parts of different trajectories to obtain a good policy that can work from a new initial condition that was not seen in a trajectory that actually achieved the reward. COG utilizes CQL as the base offline RL algorithm, and following \citet{singh2020cog}, we fix the hyperparameter $\alpha=1.0$ in the CQL part for both base COG and COG + DR3. All other hyperparameters including network sizes, etc are kept fixed as the prior work~\citet{singh2020cog} as well.    


\vspace{-0.2cm}
\subsection{Tasks and Environments Used}
\label{app:tasks}
\vspace{-0.2cm}

{\bf Atari 2600 games used}. For all our experiments, we used the same set of 17 games utilized by \citet{kumar2021implicit} to test rank collapse. In the case of Atari, we used the 5 standard games (\textsc{Asterix}, \textsc{Qbert}, \textsc{Pong}, \textsc{Seaquest}, \textsc{Breakout}) for tuning the hyperparameters, a strategy followed by several prior works~\citep{gulcehre2020rl,agarwal2019optimistic,kumar2021implicit}. The 17 games we test on are: \ \textsc{Asterix}, \textsc{Qbert}, \textsc{Pong}, \textsc{Seaquest}, \textsc{Breakout}, \textsc{Double Dunk}, \textsc{James Bond}, \textsc{Ms. Pacman}, \textsc{Space Invaders}, \textsc{Zaxxon}, \textsc{Wizard of Wor}, \textsc{Yars' Revenge}, \textsc{Enduro}, \textsc{Road Runner}, \textsc{BeamRider}, \textsc{Demon Attack}, \textsc{Ice Hockey}.

Following \citet{agarwal2021precipice}, we report interquartile mean~(IQM) normalized scores across all runs as mean scores can be dominated by performance on a few outlier tasks while median is independent of performance on all except 1 task -- zero score on half of the tasks would not affect the median. IQM which corresponds to 25\% trimmed mean and considers the performance on middle 50\% of the runs. IQM interpolates between mean and median, which correspond to 0\% and almost 50\% trimmed means across runs.

% {\bf D4RL tasks used.} For our experiments on D4RL, we utilize the Gym-MuJoCo-v0 environments for evaluating BRAC, since BRAC performed somewhat reasonably on these domains~\citep{fu2020d4rl}, whereas we use the harder AntMaze and Franka Kitchen domains for evaluating CQL, since these domains are challenging for CQL~\citep{kumar2020conservative}.

{\bf Robotic manipulation tasks from COG~\citep{singh2020cog}.} These tasks consist of a 6-DoF WidowX robot, placed in front of two drawers and a larger variety of objects. The robot can open or close a drawer, grasp objects from inside the drawer or on the table, and place them anywhere in the scene. The task here consists of taking an object out of a drawer. A reward of +1 is obtained when the object has been taken out, and zero otherwise. There are two variants of this domain: \textbf{(1)} in the first variant, the drawer starts out closed, the top drawer starts out open (which blocks the handle for the lower drawer), and an object starts out in front of the closed drawer,
which must be moved out of the way before opening, and \textbf{(2)} in the second variant, the drawer is blocked by an object, and this object must be removed before the drawer can be opened and the target object can be grasped from the drawer. The prior data for this environment is collected from a collection of scripted randomized policies. These policies are capable of opening and closing both drawers with 40-50\% success rates, can grasp objects in the scene with about a 70\% success rate, and place those objects at random places in the scene (with a slight bias for putting them in the tray). 


\begin{table*}[t]
\small
\caption{\textbf{Hyperparameters used by the offline RL Atari agents in our experiments.} Following \citet{agarwal2019optimistic}, the Atari environments used by us are stochastic due to sticky actions, \ie\ there is a 25\% chance at every time step that the environment will execute the agents previous action again, instead of the new action commanded. We report offline training results with same hyperparameters over 5 random seeds of the offline dataset, game simulator and network initialization.} 
\centering
\begin{tabular}{lrr}
\toprule
Hyperparameter & \multicolumn{2}{r}{Setting (for both variations)} \\
\midrule
Sticky actions && Yes        \\
Sticky action probability && 0.25\\
Grey-scaling && True \\
Observation down-sampling && (84, 84) \\
Frames stacked && 4 \\
Frame skip~(Action repetitions) && 4 \\
Reward clipping && [-1, 1] \\
Terminal condition && Game Over \\
Max frames per episode && 108K \\
Discount factor && 0.99 \\
Mini-batch size && 32 \\
Target network update period & \multicolumn{2}{r}{every 2000 updates} \\
Training environment steps per iteration && 250K \\
Update period every && 4 environment steps \\
Evaluation $\epsilon$ && 0.001 \\
Evaluation steps per iteration && 125K \\
$Q$-network: channels && 32, 64, 64 \\
$Q$-network: filter size && $8\times8$, $4\times4$, $3\times3$\\
$Q$-network: stride && 4, 2, 1\\
$Q$-network: hidden units && 512 \\
\bottomrule
\end{tabular}
\label{table:hyperparams_atari}
\end{table*}

\vspace{-0.2cm}
\subsection{The DR3 Regularizer Coefficient}
\label{app:tuning_dr3}
\vspace{-0.2cm}
We utilize identical hyperparameters of the base offline RL algorithms when DR3 is used, where the base hyper-parameters correspond to the ones provided in the corresponding publications. DR3 requires us to tune the additional coefficient $c_0$, that weights the DR3 explicit regularizer term. In order to find this value on our domains, we followed the tuning strategy typically followed on Atari, where we evaluated four different values of $c_0 \in \{0.001, 0.01, 0.03, 0.3\}$ on 5 games (\textsc{Asterix}, \textsc{Seaquest}, \textsc{Breakout}, \textsc{Pong} and \textsc{SpaceInvaders}) on the 5\% replay dataset settings, picked $c_0$ that wprked best on just these domains, and used it to report performance on all 17 games, across all dataset settings (1\% replay and 10\% initial replay) in Section~\ref{sec:experiments}. This protocol is standard in Atari and has been used previously in \citet{agarwal2019optimistic,gulcehre2020rl,kumar2021implicit} in the context of offline RL. The value of the coefficient found using this strategy was $c_0 = 0.001$ for REM and $c_0 = 0.03$ for CQL.




% For CQL on D4RL, we ran DR3 with multiple values of $c_0 \in \{0.0001, 0.001, 0.01, 0.5, 1.0, 10.0\}$, and picked the smallest value of $c_0$ which did not lead to eventually divergent (either negatively diverging or positively diverging) Q-values, in average. For the antmaze domains, this corresponded to $c_0=0.001$ and for the FrankaKitchen domains, this corresponded to $c_0=1.0$. 

% \section{\drmethodname\ Address Rank Collapse~\citep{kumar2021implicit} With Bootstrapping}
% \label{app:extra_results}
% In this section, we provide empirical evidence to show that utilizing \drmethodname\ alleviates the rank collapse issue pointed out by \citet{kumar2021implicit}, without explicitly aiming to address it. As shown in Figure~\ref{fig:iup_is_fixed}, we plot the effective rank $\mathrm{srank}(\Phi)$~\citep{kumar2021implicit} of the last-but-one layer representations of the Q-function. The effective rank of a matrix $\bM \in  \mathbb{R}^n \times d, n > d$, for a given threshold $\delta$ is given by: $\mathrm{srank}_\delta(\bM) = \min \{k: \frac{\sum_{i=1}^k \sigma_i(\bM)}{\sum_{i=1}^d \sigma_i(\bM)} \geq 1 - \delta \}$, where $\{\sigma_i(\bM)\}$ denotes the singular values of $\bM$ arranged in decreasing order. While it has been noted that bootstrapping can lead to rank collapse of these feature representations in the sense that the effective rank of features decreases, we find that DR3 addresses this issue allowing the Q-function to use its complete representational capacity as shown below.  

% \begin{figure}[ht]
% \small \begin{center}
% % \vspace{-5pt}
% \includegraphics[width=0.7\linewidth]{figures/rank_trends_dr3_dqn.pdf}
% % \includegraphics[width=\linewidth]{figures/pickplace_open_grasp.png}
% % \vspace{-20pt}
% \caption{Trend of effective rank, $\mathrm{srank}(\Phi)$ of features $\Phi$ learned by the Q-function when trained with TD error (red, ``Without DR3'') and with TD error + \drmethodname\ (blue, ``With DR3'') on three Atari games using the 5\% replay dataset~\citep{agarwal2019optimistic} in the offline RL regime. Note that the usage of \drmethodname\ clearly alleviates rank collapse, without actually explicitly correcting for the rank.}
% \label{fig:iup_is_fixed}
% \end{center}
% \end{figure}
% \section{Extra Results}
% \label{app;extra_results}

% \subsection{Atari Results}
% \label{app:atari_results}
% ABCD 



\section{Full Results for DR3}
\label{app:full_results}

% In this section, we present the results obtained by running DR3 on the Atari and D4RL domains which were not discussed in the main paper due to lack of space. We first understand the effect of applying DR3 on BRAC~\citep{wu2019behavior}, which was missing from the main paper, and then present the per-game Atari results. 

\begin{table}[H]
\small
\centering
 \caption{\small{Normalized interquartile mean~ (IQM) final performance (last iteration return) of CQL, CQL + \drmethodname, REM and REM + \drmethodname\ after 6.5M gradient steps for the 1\% setting and 12.5M gradient steps for the 5\%, 10\% settings. Intervals in brackets show 95\% CIs computed using stratified percentile bootstrap~\citep{agarwal2021precipice}}}.
 \label{tab:cql_res_median}
    \vspace{0.1cm}
\begin{tabular}{ccccc}
\toprule
% \multirow{2}{*}{\textbf{Data}} &  \multicolumn{4}{c|}{\textbf{Last iteration performance}} & \multicolumn{4}{c}{\textbf{Stability performance}} \\
Data & CQL &  CQL + \drmethodname & REM & REM + \drmethodname \\
\midrule
1\% & 44.4~\ss{(31.0, 54.3)} & \textbf{61.6}~\ss{(39.1, 71.5)} & 0.0~\ss{(-0.7, 0.1)} & \textbf{13.1}~\ss{(9.9, 18.3)} \\
\midrule
5\%  & 89.6~\ss{(67.9, 98.1)} & \textbf{100.2}~\ss{(90.6, 102.7)} & 3.9~\ss{(3.1, 7.6)} & \textbf{74.8}~\ss{(59.6, 84.4)} \\
\midrule
10\%  & 57.4~\ss{(53.2, 62.4)} &  \textbf{67.0}~\ss{(62.8, 73.0)} & 24.9~\ss{(15.0, 29.1)} &  \textbf{72.4}~\ss{(65.7, 81.7)} \\
\bottomrule
% \vspace{-10pt}
\end{tabular}
\end{table}


% \begin{table}[H]
% \fontsize{9}{9}
% \centering
% \caption{\textbf{Performance of \drmethodname\ when applied in conjunction with BRAC~\citep{wu2019behavior}.} Note that DR3 attains a larger final performance (at the end of 2M steps of training) as well as a higher average performance (i.e. stability score) across all iterations of training. %\todo{add bold}
% }
% \label{tab:brac}
% \vspace{0.2cm}
% \begin{tabular}{ccccc}
% \toprule
% \multirow{2}{*}{Task} & \multicolumn{2}{c}{Average Performance across Iterations}   & \multicolumn{2}{c}{Final Performance} \\
% & BRAC & BRAC + \drmethodname & BRAC & BRAC + \drmethodname \\
% \midrule
% %('True', '0.1', '2')
% halfcheetah-expert-v0 & 1.7 $\pm$ 1.9 & 49.9 $\pm$ 16.7  & 2.1 $\pm$ 3.3 & 71.5 $\pm$ 24.9 \\
% halfcheetah-medium-v0 & 43.5 $\pm$ 0.2 & 43.2 $\pm$ 0.2  & 45.1 $\pm$ 0.8 & 44.9 $\pm$ 0.6 \\
% halfcheetah-medium-expert-v0 & 17.0 $\pm$ 5.4 & 6.0 $\pm$ 5.5  & 24.8 $\pm$ 9.3 & 6.7 $\pm$ 7.3 \\
% halfcheetah-random-v0 & 24.4 $\pm$ 0.4 & 18.4 $\pm$ 0.3  & 24.9 $\pm$ 0.8 & 18.2 $\pm$ 1.0 \\
% halfcheetah-medium-replay-v0 & 44.9 $\pm$ 0.3 & 44.1 $\pm$ 0.4  & 45.0 $\pm$ 1.4 & 44.9 $\pm$ 0.5 \\
% hopper-expert-v0 & 15.7 $\pm$ 1.5 & 21.8 $\pm$ 3.2  & 16.6 $\pm$ 6.0 & 20.8 $\pm$ 5.3 \\
% hopper-medium-v0 & 32.8 $\pm$ 1.4 & 46.3 $\pm$ 7.1  & 36.2 $\pm$ 1.7 & 58.3 $\pm$ 13.7 \\
% hopper-medium-expert-v0 & 40.2 $\pm$ 5.7 & 37.0 $\pm$ 2.9  & 31.7 $\pm$ 11.8 & 21.8 $\pm$ 4.9 \\
% hopper-random-v0 & 11.7 $\pm$ 0.0 & 11.2 $\pm$ 0.0  & 12.2 $\pm$ 0.0 & 11.1 $\pm$ 0.0 \\
% hopper-medium-replay-v0 & 31.6 $\pm$ 0.3 & 30.3 $\pm$ 0.8  & 31.3 $\pm$ 1.2 & 36.1 $\pm$ 5.7 \\
% walker2d-expert-v0 & 25.5 $\pm$ 14.4 & 33.6 $\pm$ 11.8  & 54.0 $\pm$ 31.0 & 60.6 $\pm$ 20.2 \\
% walker2d-medium-v0 & 81.3 $\pm$ 0.3 & 80.8 $\pm$ 0.2  & 83.8 $\pm$ 0.2 & 83.4 $\pm$ 0.3 \\
% walker2d-medium-expert-v0 & 5.8 $\pm$ 5.2 & 6.4 $\pm$ 3.4  & 22.4 $\pm$ 22.0 & 39.5 $\pm$ 23.3 \\
% walker2d-random-v0 & 1.4 $\pm$ 0.8 & 1.7 $\pm$ 0.9  & 0.0 $\pm$ 0.1 & 2.9 $\pm$ 2.1 \\
% walker2d-medium-replay-v0 & 26.1 $\pm$ 6.4 & 47.4 $\pm$ 4.1  & 11.7 $\pm$ 7.0 & 38.7 $\pm$ 9.6 \\
% % \midrule
% % Median Normalized Performance & 25.5 & \textbf{33.6} & 24.9 & \textbf{38.7} \\
% % Mean Normalized Performance & 26.9 & \textbf{31.9} & 29.5 & \textbf{37.3} \\
% \bottomrule
% \end{tabular}
% \end{table}

\begin{table*}[H]
\small
 \caption{\small{Normalized median final performance (last iteration return) and mediann average performance (our metric for stability) of CQL, CQL + \drmethodname, REM and REM + \drmethodname\ after 6.5M gradient steps for the 1\% setting and 12.5M gradient steps for the 5\%, 10\% settings. Intervals in brackets show 95\% CIs computed using stratified percentile bootstrap~\citep{agarwal2021precipice}}}.
 \label{tab:cql_res_median}
    \vspace{0.1cm}
\begin{tabular}{c|cccc|cccc}
\toprule
\multirow{2}{*}{\textbf{Data}} &  \multicolumn{4}{c|}{\textbf{Last iteration performance}} & \multicolumn{4}{c}{\textbf{Stability performance}} \\
& CQL & CQL+\drmethodname & REM & REM+\drmethodname & CQL & CQL+\drmethodname & REM & REM+\drmethodname \\
\midrule
1\% & 44.4~\ss{(31.0, 54.3)} & 61.6~\ss{(39.1, 71.5)} & 0.0~\ss{(-0.7, 0.1)} & \textbf{13.1}~\ss{(9.9, 18.3)} & 43.6~\ss{(36.4, 52.7)} & 56.3~\ss{(46.9, 70.3)} & 4.1~\ss{(2.9, 4.9)} & \textbf{18.1}~\ss{(11.3, 22.5)}\\
\midrule
5\%  & 89.6~\ss{(67.9, 98.1)} & 100.2~\ss{(90.6, 102.7)} & 3.9~\ss{(3.1, 7.6)} & \textbf{74.8}~\ss{(59.6, 84.4)} & 85.8~\ss{(77.3, 95.8)} & \textbf{107.6}~\ss{(105.4, 109.5)} & 28.7~\ss{(20.4, 30.0)} & \textbf{60.5}~\ss{(55.1, 65.5)} \\
\midrule
10\%  & 57.4~\ss{(53.2, 62.4)} &  \textbf{67.0}~\ss{(62.8, 73.0)} & 24.9~\ss{(15.0, 29.1)} &  \textbf{72.4}~\ss{(65.7, 81.7)} & 53.6~\ss{(51.9, 56.5)} & \textbf{71.5}~\ss{(66.5, 73.9)} & 49.4~\ss{(47.7, 54.1)} & \textbf{63.9}~\ss{(67.1, 73.9)} \\
\bottomrule
\vspace{-10pt}
\end{tabular}
\end{table*}

\iffalse

\begin{table}[H]
\centering
    \caption{\textbf{Mean evaluation returns per Atari game across 5 runs with standard deviations for 1\% dataset}. The coefficient for \drmethodname\ is 0.03 with a CQL coefficient of 1.0.
    The average performance is computed over 20 checkpoints spaced uniformly over training for 100 iterations where 1 iteration corresponds to 62,500 gradient updates.}
    \label{tab:cql_dqn_1}
    \vspace{0.2cm}
\resizebox{0.99\textwidth}{!}{\begin{tabular}{ccccc}
\toprule
\multirow{2}{*}{Game} & \multicolumn{2}{c}{Final Performance}   & \multicolumn{2}{c}{Average Performance across Iterations} \\
& CQL & CQL + \drmethodname & CQL & CQL + \drmethodname \\
\midrule
Asterix       &      656.9 $\pm$ 91.0 &      821.4 $\pm$ 75.1 &      650.2 $\pm$ 65.3 &      814.1 $\pm$ 25.1 \\
Breakout      &        23.9 $\pm$ 3.8 &        32.0 $\pm$ 3.2 &        23.8 $\pm$ 0.5 &        32.8 $\pm$ 3.1 \\
Pong          &        16.7 $\pm$ 1.7 &        14.2 $\pm$ 3.3 &        15.7 $\pm$ 2.0 &        15.1 $\pm$ 2.3 \\
Seaquest      &      449.0 $\pm$ 11.0 &      446.6 $\pm$ 26.9 &      474.5 $\pm$ 30.3 &      456.1 $\pm$ 17.0 \\
Qbert         &   8033.8 $\pm$ 1513.2 &    9162.7 $\pm$ 993.6 &    7980.0 $\pm$ 379.9 &    9000.7 $\pm$ 225.2 \\
SpaceInvaders &     386.0 $\pm$ 123.2 &      351.9 $\pm$ 77.1 &      371.7 $\pm$ 47.5 &      440.6 $\pm$ 29.6 \\
Zaxxon        &     829.4 $\pm$ 813.3 &    1757.4 $\pm$ 879.4 &     834.6 $\pm$ 504.0 &    1634.0 $\pm$ 673.9 \\
YarsRevenge   &  11848.2 $\pm$ 2977.7 &  16011.3 $\pm$ 1409.0 &  15077.9 $\pm$ 1301.9 &   17741.6 $\pm$ 613.6 \\
RoadRunner    &  37000.7 $\pm$ 1148.5 &  24928.7 $\pm$ 7484.5 &   35899.9 $\pm$ 653.1 &  32063.3 $\pm$ 1011.4 \\
MsPacman      &    1869.8 $\pm$ 167.2 &    2245.7 $\pm$ 193.8 &     1991.9 $\pm$ 55.1 &     2224.1 $\pm$ 80.8 \\
BeamRider     &      780.3 $\pm$ 64.5 &      617.9 $\pm$ 25.1 &      782.0 $\pm$ 36.1 &      619.9 $\pm$ 20.9 \\
Jamesbond     &     558.5 $\pm$ 124.8 &     460.5 $\pm$ 102.0 &     524.6 $\pm$ 118.5 &      484.2 $\pm$ 89.4 \\
Enduro        &      198.4 $\pm$ 34.2 &      253.5 $\pm$ 14.2 &      259.8 $\pm$ 16.4 &      276.1 $\pm$ 16.9 \\
WizardOfWor   &     771.1 $\pm$ 358.2 &     904.6 $\pm$ 343.7 &     833.7 $\pm$ 168.4 &     935.2 $\pm$ 174.4 \\
IceHockey     &        -8.7 $\pm$ 1.3 &        -7.8 $\pm$ 0.9 &        -8.8 $\pm$ 0.9 &        -7.9 $\pm$ 0.7 \\
DoubleDunk    &       -15.1 $\pm$ 1.9 &       -14.0 $\pm$ 2.8 &       -15.3 $\pm$ 0.9 &       -14.5 $\pm$ 1.0 \\
DemonAttack   &    1970.2 $\pm$ 161.3 &      386.2 $\pm$ 75.3 &    1338.8 $\pm$ 298.4 &      414.0 $\pm$ 46.0 \\
\bottomrule
\end{tabular}}
\end{table}



\begin{table}[H]
\centering
    \caption{\textbf{Mean evaluation returns per Atari game across 5 runs with standard deviations for 5\% dataset.} The coefficient for \drmethodname\ is 0.03 with a CQL coefficient of 0.1.
    The average performance is computed over 20 checkpoints spaced uniformly over training for 200 iterations where 1 iteration corresponds to 62,500 gradient updates.}
    \label{tab:cql_dqn_5}
    \vspace{0.2cm}
    \resizebox{0.99\textwidth}{!}{\begin{tabular}{ccccc}
\toprule
\multirow{2}{*}{Game} & \multicolumn{2}{c}{Final Performance}   & \multicolumn{2}{c}{Average Performance across Iterations} \\
& CQL & CQL + \drmethodname & CQL & CQL + \drmethodname \\
\midrule
Asterix       &    1798.2 $\pm$ 168.6 &    3318.5 $\pm$ 301.7 &     1812.7 $\pm$ 64.0 &    3790.5 $\pm$ 218.0 \\
Breakout      &       94.1 $\pm$ 44.4 &      166.0 $\pm$ 23.1 &      105.1 $\pm$ 10.4 &       196.5 $\pm$ 4.4 \\
Pong          &        13.1 $\pm$ 4.2 &        17.9 $\pm$ 1.1 &        15.2 $\pm$ 1.3 &        17.4 $\pm$ 1.2 \\
Seaquest      &    1815.9 $\pm$ 722.8 &    2030.7 $\pm$ 822.8 &    1382.3 $\pm$ 258.1 &    3722.3 $\pm$ 969.5 \\
Qbert         &  10595.7 $\pm$ 1648.5 &   9605.6 $\pm$ 1593.5 &    9552.0 $\pm$ 925.6 &   10830.7 $\pm$ 783.1 \\
SpaceInvaders &      758.9 $\pm$ 56.9 &    1214.6 $\pm$ 281.8 &      662.0 $\pm$ 58.1 &     1323.7 $\pm$ 94.4 \\
Zaxxon        &   1501.0 $\pm$ 1165.7 &    4250.1 $\pm$ 626.2 &    1508.8 $\pm$ 437.5 &    3556.5 $\pm$ 531.3 \\
YarsRevenge   &  24036.7 $\pm$ 3370.6 &  17124.7 $\pm$ 2125.6 &  22733.1 $\pm$ 1175.3 &  18339.8 $\pm$ 1299.7 \\
RoadRunner    &  40728.4 $\pm$ 3318.9 &  38432.6 $\pm$ 1539.7 &   42338.4 $\pm$ 471.4 &  41260.2 $\pm$ 1008.6 \\
MsPacman      &    2975.9 $\pm$ 522.1 &    2790.6 $\pm$ 353.1 &    2923.6 $\pm$ 251.3 &    3101.2 $\pm$ 381.6 \\
BeamRider     &    1897.6 $\pm$ 473.7 &      785.8 $\pm$ 43.5 &    2218.5 $\pm$ 242.4 &      775.9 $\pm$ 12.5 \\
Jamesbond     &      108.8 $\pm$ 49.1 &       96.8 $\pm$ 43.2 &        76.5 $\pm$ 4.6 &      106.1 $\pm$ 34.8 \\
Enduro        &     764.3 $\pm$ 168.7 &      938.5 $\pm$ 63.9 &      797.7 $\pm$ 47.8 &      923.2 $\pm$ 40.3 \\
WizardOfWor   &     943.2 $\pm$ 380.3 &     612.0 $\pm$ 343.3 &    1004.3 $\pm$ 314.7 &    1007.4 $\pm$ 313.2 \\
IceHockey     &       -17.3 $\pm$ 0.6 &       -15.0 $\pm$ 0.7 &       -16.6 $\pm$ 0.5 &       -12.0 $\pm$ 0.3 \\
DoubleDunk    &       -18.1 $\pm$ 1.5 &       -16.2 $\pm$ 1.7 &       -17.3 $\pm$ 1.0 &       -16.0 $\pm$ 1.6 \\
DemonAttack   &    4055.8 $\pm$ 499.7 &   8517.4 $\pm$ 1065.9 &    4062.4 $\pm$ 465.8 &    8396.7 $\pm$ 689.4 \\
\bottomrule
\end{tabular}}
\end{table}

\begin{table}[H]
\centering
    \caption{\textbf{Mean returns per Atari game across 5 runs with standard deviations for initial 10\% dataset.} The coefficient for \drmethodname\ is 0.03 with a CQL coefficient of 0.1. The average performance is computed over 20 checkpoints spaced uniformly over training for 200 iterations.}
    \label{tab:cql_dqn_10}
    \vspace{0.2cm}
    \resizebox{0.99\textwidth}{!}{\begin{tabular}{ccccc}
\toprule
\multirow{2}{*}{Game} & \multicolumn{2}{c}{Final Performance}   & \multicolumn{2}{c}{Average Performance across Iterations} \\
& CQL & CQL + \drmethodname & CQL & CQL + \drmethodname \\
\midrule
Asterix       &    2803.9 $\pm$ 294.6 &    3906.2 $\pm$ 521.3 &    2903.2 $\pm$ 217.7 &    4692.2 $\pm$ 377.0 \\
Breakout      &        64.7 $\pm$ 7.3 &        70.8 $\pm$ 5.5 &        65.6 $\pm$ 5.7 &        75.4 $\pm$ 6.0 \\
Pong          &         5.3 $\pm$ 6.8 &         5.5 $\pm$ 6.2 &         7.3 $\pm$ 5.0 &         8.1 $\pm$ 5.2 \\
Seaquest      &     222.3 $\pm$ 219.5 &    1313.0 $\pm$ 220.0 &     704.9 $\pm$ 254.5 &    1327.9 $\pm$ 250.0 \\
Qbert         &    4803.2 $\pm$ 489.5 &   5395.3 $\pm$ 1003.6 &    4492.5 $\pm$ 240.8 &    4708.5 $\pm$ 463.0 \\
SpaceInvaders &     704.9 $\pm$ 121.5 &      938.1 $\pm$ 80.3 &      737.8 $\pm$ 23.8 &      902.1 $\pm$ 60.0 \\
Zaxxon        &     231.6 $\pm$ 450.9 &     836.8 $\pm$ 434.7 &     394.4 $\pm$ 385.1 &     725.7 $\pm$ 370.3 \\
YarsRevenge   &  13076.2 $\pm$ 2427.0 &  12413.9 $\pm$ 2869.7 &   12493.2 $\pm$ 543.6 &  12395.6 $\pm$ 1044.2 \\
RoadRunner    &  45063.5 $\pm$ 1749.7 &  45336.9 $\pm$ 1366.7 &  45522.7 $\pm$ 1068.1 &   44808.0 $\pm$ 911.7 \\
MsPacman      &    2459.5 $\pm$ 381.3 &    2427.5 $\pm$ 191.3 &    2528.1 $\pm$ 149.2 &    2488.3 $\pm$ 109.8 \\
BeamRider     &    4200.7 $\pm$ 470.2 &    3468.0 $\pm$ 238.0 &     4729.5 $\pm$ 94.8 &    3344.3 $\pm$ 289.0 \\
Jamesbond     &       84.6 $\pm$ 25.4 &       89.7 $\pm$ 15.6 &      108.7 $\pm$ 34.1 &      111.7 $\pm$ 10.9 \\
Enduro        &     946.7 $\pm$ 289.7 &     1160.2 $\pm$ 81.5 &     1013.9 $\pm$ 29.7 &     1136.2 $\pm$ 32.5 \\
WizardOfWor   &     520.4 $\pm$ 451.2 &     764.7 $\pm$ 250.0 &     499.8 $\pm$ 238.5 &     792.2 $\pm$ 101.3 \\
IceHockey     &       -18.1 $\pm$ 0.7 &       -16.0 $\pm$ 1.3 &       -17.6 $\pm$ 0.5 &       -15.2 $\pm$ 1.0 \\
DoubleDunk    &       -21.2 $\pm$ 1.1 &       -20.6 $\pm$ 1.0 &       -20.6 $\pm$ 0.3 &       -19.7 $\pm$ 0.5 \\
DemonAttack   &    4145.2 $\pm$ 400.6 &    7152.9 $\pm$ 723.2 &    4839.4 $\pm$ 586.7 &    7278.5 $\pm$ 701.3 \\
\bottomrule
\end{tabular}}
\end{table}

\begin{table}[H]
\centering
    \caption{\textbf{Mean returns per Atari game across 5 runs with standard deviations for 1\% dataset.} The coefficient for \drmethodname\ is 0.001 while we use a multi-headed REM with 200 Q-heads~\citep{agarwal2019optimistic}. The average performance is computed over 20 checkpoints spaced uniformly over training for 100 iterations.}
    \label{tab:rem_dqn_1}
    \vspace{0.2cm}
    \resizebox{0.99\textwidth}{!}{\begin{tabular}{ccccc}
\toprule
\multirow{2}{*}{Game} & \multicolumn{2}{c}{Final Performance}   & \multicolumn{2}{c}{Average Performance across Iterations} \\
& REM & REM + \drmethodname & REM & REM + \drmethodname \\
\midrule
Asterix       &     240.4 $\pm$ 29.1 &     405.7 $\pm$ 46.5 &     304.4 $\pm$ 9.3 &      413.7 $\pm$ 39.6 \\
Breakout      &        0.7 $\pm$ 0.7 &       14.3 $\pm$ 2.8 &       6.3 $\pm$ 1.0 &        10.3 $\pm$ 1.1 \\
Pong          &      -14.2 $\pm$ 1.7 &       -7.7 $\pm$ 6.3 &     -14.1 $\pm$ 2.2 &       -15.3 $\pm$ 3.0 \\
Seaquest      &      81.0 $\pm$ 78.5 &    293.3 $\pm$ 191.5 &    246.6 $\pm$ 49.5 &     489.9 $\pm$ 128.6 \\
Qbert         &    239.6 $\pm$ 133.2 &    436.3 $\pm$ 111.5 &    255.5 $\pm$ 76.0 &     471.0 $\pm$ 116.5 \\
SpaceInvaders &     152.8 $\pm$ 27.5 &     206.6 $\pm$ 77.6 &     188.6 $\pm$ 5.8 &      262.7 $\pm$ 22.4 \\
Zaxxon        &    534.9 $\pm$ 731.3 &  2596.4 $\pm$ 1726.4 &  1807.9 $\pm$ 478.2 &     707.7 $\pm$ 577.4 \\
YarsRevenge   &  1452.6 $\pm$ 1631.0 &   5480.2 $\pm$ 962.3 &  4018.8 $\pm$ 987.8 &    7352.0 $\pm$ 574.7 \\
RoadRunner    &        0.0 $\pm$ 0.0 &  3872.9 $\pm$ 1616.4 &  1601.2 $\pm$ 637.9 &  14231.9 $\pm$ 2406.0 \\
MsPacman      &    698.8 $\pm$ 129.5 &   1275.1 $\pm$ 345.6 &    690.4 $\pm$ 69.7 &      860.4 $\pm$ 57.1 \\
BeamRider     &     703.0 $\pm$ 97.4 &     522.9 $\pm$ 42.2 &    745.5 $\pm$ 30.7 &      592.2 $\pm$ 27.7 \\
Jamesbond     &      41.0 $\pm$ 27.0 &     157.6 $\pm$ 65.0 &     53.3 $\pm$ 12.1 &       88.8 $\pm$ 27.2 \\
Enduro        &        0.5 $\pm$ 0.4 &     132.4 $\pm$ 16.1 &      21.7 $\pm$ 4.0 &      197.5 $\pm$ 19.1 \\
WizardOfWor   &    362.5 $\pm$ 321.8 &   1663.7 $\pm$ 417.8 &   552.1 $\pm$ 253.1 &    1460.8 $\pm$ 194.8 \\
IceHockey     &      -16.7 $\pm$ 0.9 &       -9.1 $\pm$ 5.1 &     -12.1 $\pm$ 0.8 &        -4.8 $\pm$ 1.8 \\
DoubleDunk    &      -21.8 $\pm$ 1.0 &      -17.6 $\pm$ 1.5 &     -20.4 $\pm$ 0.6 &       -17.1 $\pm$ 1.6 \\
DemonAttack   &     102.0 $\pm$ 17.3 &     162.0 $\pm$ 34.7 &    124.0 $\pm$ 10.7 &      145.6 $\pm$ 27.2 \\
\bottomrule
\end{tabular}}
\end{table}

\begin{table}[H]
\centering
    \caption{\textbf{Mean returns per Atari game across 5 runs with standard deviations for the 5\% dataset.} The coefficient for \drmethodname\ is 0.001 while we use a multi-headed REM with 200 Q-heads~\citep{agarwal2019optimistic}. The average performance is computed over 20 checkpoints spaced uniformly over training for 200 iterations.}
    \label{tab:rem_dqn_5}
    \vspace{0.2cm}
    \resizebox{0.99\textwidth}{!}{\begin{tabular}{ccccc}
\toprule
\multirow{2}{*}{Game} & \multicolumn{2}{c}{Final Performance}   & \multicolumn{2}{c}{Average Performance across Iterations} \\
& REM & REM + \drmethodname & REM & REM + \drmethodname \\
\midrule
Asterix       &      876.8 $\pm$ 201.1 &    2317.0 $\pm$ 838.1 &      958.9 $\pm$ 50.9 &    1252.6 $\pm$ 395.1 \\
Breakout      &         15.2 $\pm$ 4.9 &        33.4 $\pm$ 4.0 &        16.3 $\pm$ 3.4 &        17.7 $\pm$ 2.4 \\
Pong          &          7.5 $\pm$ 5.2 &        -0.7 $\pm$ 9.9 &        -4.7 $\pm$ 3.0 &       -12.0 $\pm$ 3.2 \\
Seaquest      &     1276.0 $\pm$ 417.3 &   2753.6 $\pm$ 1119.7 &    1484.3 $\pm$ 367.7 &    1602.0 $\pm$ 603.7 \\
Qbert         &    2421.4 $\pm$ 1841.8 &   7417.0 $\pm$ 2106.7 &    1330.7 $\pm$ 431.0 &    4045.8 $\pm$ 898.9 \\
SpaceInvaders &       431.5 $\pm$ 23.3 &      443.5 $\pm$ 67.4 &      349.5 $\pm$ 22.6 &      362.1 $\pm$ 33.6 \\
Zaxxon        &     6738.2 $\pm$ 966.6 &   1609.7 $\pm$ 1814.1 &    3630.7 $\pm$ 751.4 &     346.1 $\pm$ 512.1 \\
YarsRevenge   &   14454.2 $\pm$ 1644.4 &  16930.4 $\pm$ 2625.8 &  14628.3 $\pm$ 1945.1 &  12936.5 $\pm$ 1286.0 \\
RoadRunner    &  15570.9 $\pm$ 12795.6 &  46601.6 $\pm$ 2617.2 &  22740.3 $\pm$ 1977.2 &  33554.1 $\pm$ 1880.4 \\
MsPacman      &     1272.2 $\pm$ 215.3 &    2303.1 $\pm$ 202.7 &    1147.7 $\pm$ 126.1 &    1438.7 $\pm$ 140.4 \\
BeamRider     &     1922.5 $\pm$ 589.1 &      674.8 $\pm$ 21.4 &      886.9 $\pm$ 82.1 &      698.3 $\pm$ 21.5 \\
Jamesbond     &       189.6 $\pm$ 77.0 &      130.5 $\pm$ 45.7 &       120.2 $\pm$ 9.3 &       88.6 $\pm$ 41.5 \\
Enduro        &       172.7 $\pm$ 55.9 &     583.9 $\pm$ 108.7 &      236.8 $\pm$ 11.3 &      457.7 $\pm$ 39.3 \\
WizardOfWor   &      838.4 $\pm$ 670.0 &    2661.6 $\pm$ 371.4 &     1281.3 $\pm$ 66.7 &    1863.7 $\pm$ 261.2 \\
IceHockey     &         -9.7 $\pm$ 4.2 &        -6.5 $\pm$ 3.1 &        -8.1 $\pm$ 0.7 &        -4.1 $\pm$ 1.5 \\
DoubleDunk    &        -18.4 $\pm$ 0.9 &       -17.6 $\pm$ 2.6 &       -19.6 $\pm$ 1.0 &       -17.8 $\pm$ 1.9 \\
DemonAttack   &      507.7 $\pm$ 120.1 &   5602.3 $\pm$ 1855.5 &     581.6 $\pm$ 207.0 &    1452.3 $\pm$ 765.0 \\
\bottomrule
\end{tabular}}
\end{table}

\begin{table}[H]
\centering
    \caption{\textbf{Mean returns per Atari game across 5 runs with standard deviations for initial 10\% dataset.} The coefficient for \drmethodname\ is 0.001 while we use a multi-headed REM with 200 Q-heads~\citep{agarwal2019optimistic}. The average performance is computed over 20 checkpoints spaced uniformly over training for 200 iterations.}
    \label{tab:rem_dqn_10}
    \vspace{0.2cm}
    \resizebox{0.99\textwidth}{!}{\begin{tabular}{ccccc}
\toprule
\multirow{2}{*}{Game} & \multicolumn{2}{c}{Final Performance}   & \multicolumn{2}{c}{Average Performance across Iterations} \\
& REM & REM + \drmethodname & REM & REM + \drmethodname \\
\midrule
Asterix       &    2254.7 $\pm$ 403.6 &    5122.9 $\pm$ 328.9 &   2684.6 $\pm$ 184.4 &    3432.1 $\pm$ 257.5 \\
Breakout      &       81.2 $\pm$ 13.9 &       96.8 $\pm$ 21.2 &       63.5 $\pm$ 4.6 &        62.4 $\pm$ 6.1 \\
Pong          &         8.8 $\pm$ 3.1 &        7.6 $\pm$ 11.1 &        2.6 $\pm$ 2.1 &        -2.5 $\pm$ 5.6 \\
Seaquest      &    1540.2 $\pm$ 354.6 &     981.3 $\pm$ 605.9 &   1029.5 $\pm$ 260.6 &     836.2 $\pm$ 234.3 \\
Qbert         &    4330.7 $\pm$ 250.2 &    4126.2 $\pm$ 495.7 &   3478.0 $\pm$ 248.0 &    3494.7 $\pm$ 380.3 \\
SpaceInvaders &      895.2 $\pm$ 68.3 &      799.0 $\pm$ 28.3 &     699.7 $\pm$ 31.4 &      653.1 $\pm$ 21.5 \\
Zaxxon        &     950.7 $\pm$ 897.4 &         0.0 $\pm$ 0.0 &    490.2 $\pm$ 306.6 &         0.0 $\pm$ 0.0 \\
YarsRevenge   &  10913.1 $\pm$ 1519.1 &  11924.8 $\pm$ 2413.8 &  11508.5 $\pm$ 290.0 &  10977.7 $\pm$ 1026.9 \\
RoadRunner    &  45521.7 $\pm$ 2502.1 &  49129.4 $\pm$ 1887.9 &  37997.4 $\pm$ 638.6 &  41995.2 $\pm$ 1482.1 \\
MsPacman      &    2177.4 $\pm$ 393.0 &    2268.8 $\pm$ 455.0 &   1930.5 $\pm$ 141.7 &    2126.6 $\pm$ 147.6 \\
BeamRider     &    2921.7 $\pm$ 308.7 &    4154.9 $\pm$ 357.2 &   3727.5 $\pm$ 304.3 &     2871.0 $\pm$ 44.3 \\
Jamesbond     &      197.8 $\pm$ 73.8 &     149.3 $\pm$ 304.5 &    149.0 $\pm$ 120.5 &      83.3 $\pm$ 162.4 \\
Enduro        &     529.5 $\pm$ 200.7 &      832.5 $\pm$ 65.5 &     584.6 $\pm$ 85.3 &      801.8 $\pm$ 39.3 \\
WizardOfWor   &     606.5 $\pm$ 823.2 &     920.0 $\pm$ 497.0 &    838.3 $\pm$ 343.7 &     926.3 $\pm$ 318.5 \\
IceHockey     &        -4.3 $\pm$ 0.6 &        -5.9 $\pm$ 5.1 &       -7.0 $\pm$ 1.1 &        -5.4 $\pm$ 3.7 \\
DoubleDunk    &       -17.7 $\pm$ 3.9 &       -19.5 $\pm$ 2.5 &      -16.9 $\pm$ 0.5 &       -16.7 $\pm$ 1.0 \\
DemonAttack   &   6097.9 $\pm$ 1251.3 &   9674.7 $\pm$ 1600.6 &   4649.1 $\pm$ 514.6 &    5141.9 $\pm$ 361.4 \\
\bottomrule
\end{tabular}}
\end{table}




\begin{table}[H]
\centering
    \caption{Average returns across 5 runs for the random agent and the average performance of the trajectories in the DQN~(Nature) dataset. For Atari normalized scores reported in the paper, the random agent is assigned a score of 0 while the average DQN replay is assigned a score of 100. Note that the random agent scores are also evaluated on Atari 2600 games with sticky actions.}
    \label{tab:random_dqn_scores}
    \vspace{0.2cm}
    \resizebox{0.6\textwidth}{!}{\begin{tabular}{ccc}
\toprule
Game &  Random &  Average DQN-Replay \\
\midrule
Asterix       &   279.1 &              3185.2 \\
Breakout      &     1.3 &               104.9 \\
Pong          &   -20.3 &                14.5 \\
Seaquest      &    81.8 &              1597.4 \\
Qbert         &   155.0 &              8249.7 \\
SpaceInvaders &   149.5 &              1529.8 \\
Zaxxon        &    10.6 &              1854.1 \\
YarsRevenge   &  3147.7 &             21015.0 \\
RoadRunner    &    15.5 &             38352.3 \\
MsPacman      &   248.0 &              3108.8 \\
BeamRider     &   362.0 &              4576.4 \\
Jamesbond     &    27.6 &               560.3 \\
Enduro        &     0.0 &               671.9 \\
WizardOfWor   &   686.6 &              1128.5 \\
IceHockey     &    -9.8 &                -8.5 \\
DoubleDunk    &   -18.4 &               -11.3 \\
DemonAttack   &   166.0 &              4407.5 \\
\bottomrule
\end{tabular}}
\end{table}



\fi

% \begin{table*}[t]
% \fontsize{9}{9}\selectfont
%     \centering
%     \vspace{-0.1cm}
%     \caption{\small{Normalized final performance (last iteration return) and average performance (our metric for stability) of CQL, CQL + \drmethodname, REM and REM + \drmethodname\ after 6.5M gradient steps for the 1\% setting and 12.5M gradient steps for the 5\%, 10\% settings. Individual performance for all 17 games are provided in the Tables~\ref{tab:cql_dqn_1}-\ref{tab:rem_dqn_10}. \drmethodname\ improves the performance of both CQL and REM.} As recommended by \citet{agarwal2021precipice}, we report interquartile mean~(IQM) performance that trims 50\% of outlier runs and computes the mean of the remaining runs.}
%     \label{tab:cql_res}
%     \vspace{0.1cm}
% \begin{tabular}{c|cccc|cccc}
% \toprule
% \multirow{2}{*}{\textbf{Data}} &  \multicolumn{4}{c|}{\textbf{Last iteration performance}} & \multicolumn{4}{c}{\textbf{Stability performance}} \\
% & CQL & CQL+\drmethodname & REM & REM+\drmethodname & CQL & CQL+\drmethodname & REM & REM+\drmethodname \\
% \midrule
% 1\%  & 44.0~\ss{(38.1, 49.8)} & \textbf{50.9}~\ss{(45.8, 56.3)} & -0.1~\ss{(-0.7, 0.6)} & \textbf{13.4}~\ss{(11.1, 16.4)} & 43.7~\ss{(39.6, 48.6)} & \textbf{56.9}~\ss{(52.5, 61.2)} & 4.0~\ss{(3.3, 4.8)} & \textbf{16.5}~\ss{(14.5, 18.6)}  \\
% \midrule
% 5\%  & 74.1~\ss{(67.3, 81.3)} & \textbf{93.6}~\ss{(88.1, 99.2)} &  6.4~\ss{(5, 7.9)} & \textbf{67.5}~\ss{(62.5, 73.3)} &  78.1~\ss{(74.5, 82.4)} & \textbf{105.7}~\ss{(101.9, 110.9)} & 25.9~\ss{(23.4, 28.8)} & \textbf{60.2}~\ss({55.8, 65.1}) \\
% \midrule
% 10\% & 54.6~\ss{(49.9, 59.7)}  & \textbf{67.3}~\ss{(63.6, 72.0)} & 23.3~\ss{(19.7, 26.9)} & \textbf{77.1}~\ss{(71.8, 84.2)} & 59.3~\ss{(56.4, 61.9)} & \textbf{65.8}~\ss{(63.3, 68.3)} & 53.3~\ss{(51.4, 55.3)} & \textbf{73.8}~\ss{(69.3, 78)} \\
% \bottomrule
% \vspace{-10pt}
% \end{tabular}
% \end{table*}

% \subsection{\rebuttal{Per-Game Learning Curves for Atari Games}}
% \label{per_game_figures}


% \begin{figure}[H]
%     \centering
%     \vspace{-0.3cm}
%     \includegraphics[width=0.85\textwidth]{chapters/dr3/rebuttal/cql_dr3_5m.pdf}
%     \vspace{-0.2cm}
%     \caption{\footnotesize{\label{fig:cql_5_percent_all_games} \rebuttal{\textbf{Per-game learning curves of CQL and CQL + DR3 on the 5\% uniform replay dataset, for which the normalized average learning curve is shown in Figure~\ref{fig:atari_all_combined}.} Note that CQL + DR3 attains a higher performance than CQL for a majority of games, and rises up to a higher peak. }}}
% \end{figure}

% \begin{figure}[H]
%     \centering
%     \vspace{-5pt}
%     \includegraphics[width=0.85\textwidth]{chapters/dr3/rebuttal/rem_vs_rem_dr3_5p.pdf}
%     \vspace{-5pt}
%     \caption{\footnotesize{\label{fig:rem_5_percent_all_games} \rebuttal{\textbf{Per-game learning curves of REM and REM + DR3 on the 5\% uniform replay dataset, for which the normalized average learning curve is shown in Figure~\ref{fig:atari_all_combined}.} Note that REM + DR3 attains a higher performance than REM for a majority of games. }}}
%     \vspace{-0.5cm}
% \end{figure}


% \subsection{\rebuttal{Dot Product Similarities For CQL+DR3 and REM+DR3 on 17 Games}}
% \begin{figure}[H]
%     \centering
%     \vspace{-0.3cm}
%     \includegraphics[width=0.85\textwidth]{chapters/dr3/rebuttal/cql_vs_dr3_dot_products.pdf}
%     \vspace{-0.2cm}
%     \caption{\footnotesize{\label{fig:cql_5_percent_all_games_dot_products} \rebuttal{\textbf{Per-game feature dot products (\underline{in log scale}) of CQL and CQL + DR3 on the 5\% uniform replay dataset}. Note that CQL + DR3 attains a smaller value of the feature dot product.}}}
% \end{figure}

% \begin{figure}[H]
%     \centering
%     \vspace{-5pt}
%     \includegraphics[width=0.85\textwidth]{chapters/dr3/rebuttal/rem_vs_dr3_dot_products.pdf}
%     \vspace{-5pt}
%     \caption{\footnotesize{\label{fig:rem_5_percent_all_games_dot_products} \rebuttal{\textbf{Per-game feature dot products (\underline{in log scale}) of REM and REM + DR3 on the 5\% uniform replay dataset} Note that REM + DR3 attains a higher performance than REM for a majority of games. Note that the dot products for REM+DR3 stabilize are small, and decreases for a majority of the training steps for a number of games, or stabilize at a small value. }}}
%     \vspace{-0.5cm}
% \end{figure}



