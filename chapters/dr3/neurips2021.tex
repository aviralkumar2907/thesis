\documentclass{article}

% if you need to pass options to natbib, use, e.g.:
%     \PassOptionsToPackage{numbers, compress}{natbib}
% before loading neurips_2020

% ready for submission
% \usepackage{neurips_2020}

% to compile a preprint version, e.g., for submission to arXiv, add add the
% [preprint] option:
%     \usepackage[preprint]{neurips_2020}

% to compile a camera-ready version, add the [final] option, e.g.:
%     \usepackage[final]{neurips_2020}

% to avoid loading the natbib package, add option nonatbib:
\usepackage[nonatbib]{neurips_2021}
\usepackage[numbers]{natbib}

\usepackage{microtype}
\usepackage{graphicx}
\usepackage{subfigure}
\usepackage{booktabs} % for professional tables
\usepackage{multirow}
\usepackage{wrapfig}
\usepackage{mathtools}


% hyperref makes hyperlinks in the resulting PDF.
% If your build breaks (sometimes temporarily if a hyperlink spans a page)
% please comment out the following usepackage line and replace
% \usepackage{icml2021} with \usepackage[nohyperref]{icml2021} above.
\usepackage{amsthm,amsmath}

% Attempt to make hyperref and algorithmic work together better:
\newcommand{\theHalgorithm}{\arabic{algorithm}}

\usepackage[utf8]{inputenc} % allow utf-8 input
\usepackage[T1]{fontenc}    % use 8-bit T1 fonts
\usepackage{url,xcolor}            % simple URL typesetting
\usepackage[colorlinks=true,allcolors=blue]{hyperref}
\usepackage{booktabs}       % professional-quality tables
\usepackage{amsfonts}       % blackboard math symbols
\usepackage{nicefrac}       % compact symbols for 1/2, etc.
\usepackage{microtype}      % microtypography
\usepackage{amsmath}


%% editing comment
\newcommand{\cmt}[1]{{\footnotesize\textcolor{red}{#1}}}
\newcommand{\cmto}[1]{{\footnotesize\textcolor{orange}{#1}}}
\newcommand{\note}[1]{\cmt{Note: #1}}
\newcommand{\todo}[1]{\cmt{TO-DO: #1}}
\newcommand{\question}[1]{\cmto{Question: #1}}
\newcommand{\sergey}[1]{{\footnotesize\textcolor{blue}{Sergey: #1}}}
\newcommand{\ak}[1]{{\textcolor{black}{#1}}}
\newcommand{\edits}[1]{\textcolor{blue}{#1}}
\newcommand{\editsred}[1]{\textcolor{red}{#1}}
\newcommand{\editsp}[1]{\textcolor{purple}{#1}}
\newcommand{\editsv}[1]{\textcolor{magenta}{#1}}


\newcommand{\methodname}{Cal-QL}
\newcommand{\aliasingproblemname}{bootstrapping aliasing}
\newcommand{\Aliasingproblemname}{Bootstrapping aliasing}
\newcommand{\AliasingProblemName}{Bootstrapping Aliasing}
\newcommand{\simnorm}{\mathrm{sim}_{\mathrm{n}}^\pi}
\newcommand{\simunnorm}{\mathrm{sim}_{\mathrm{u}}^\pi}

%% abbreviations
\newcommand{\x}{\mathbf{x}}
\newcommand{\z}{\mathbf{z}}
\newcommand{\y}{\mathbf{y}}
\newcommand{\w}{\mathbf{w}}
\newcommand{\data}{\mathcal{D}}

\newcommand{\etal}{{et~al.}\ }
\newcommand{\eg}{e.g.\ }
\newcommand{\ie}{i.e.\ }
\newcommand{\nth}{\text{th}}
\newcommand{\pr}{^\prime}
\newcommand{\tr}{^\mathrm{T}}
\newcommand{\inv}{^{-1}}
\newcommand{\pinv}{^{\dagger}}
\newcommand{\real}{\mathbb{R}}
\newcommand{\gauss}{\mathcal{N}}
\newcommand{\norm}[1]{\left|#1\right|}
\newcommand{\trace}{\text{tr}}

%% specifics for the paper
\newcommand{\reward}{r}
\newcommand{\policy}{\pi}
\newcommand{\mdp}{\mathcal{M}}
\newcommand{\states}{\mathcal{S}}
\newcommand{\actions}{\mathcal{A}}
\newcommand{\observations}{\mathcal{O}}
\newcommand{\transitions}{T}
\newcommand{\initstate}{d_0}
\newcommand{\freq}{d}
\newcommand{\obsfunc}{E}
\newcommand{\initial}{\mathcal{I}}
\newcommand{\horizon}{H}
\newcommand{\rewardevent}{R}
\newcommand{\probr}{p_\rewardevent}
\newcommand{\metareward}{\bar{\reward}}
\newcommand{\discount}{\gamma}
\newcommand{\behavior}{{\pi_\beta}}
\newcommand{\bellman}{\mathcal{B}}
\newcommand{\qparams}{\phi}
\newcommand{\qparamset}{\Phi}
\newcommand{\qset}{\mathcal{Q}}
\newcommand{\batch}{B}
\newcommand{\qfeat}{\mathbf{f}}
\newcommand{\Qfeat}{\mathbf{F}}
\newcommand{\hatbehavior}{\hat{\pi}_\beta}

\newcommand{\traj}{\tau}

\newcommand{\pihi}{\pi^{\text{hi}}}
\newcommand{\pilo}{\pi^{\text{lo}}}
\newcommand{\ah}{\mathbf{w}}

\newcommand{\proj}{\Pi}

\newcommand{\loss}{\mathcal{L}}
\newcommand{\eye}{\mathbf{I}}

\newcommand{\model}{\hat{p}}
\newcommand{\mhat}{\hat{\mathcal{M}}}
\newcommand{\mdphat}{\widehat{\mathcal{M}}}
\newcommand{\mdpbar}{\overline{\mathcal{M}}}

\newcommand{\pimix}{\pi_{\text{mix}}}

\newcommand{\pib}{\bar{\pi}}
\newcommand{\epspi}{\epsilon_{\pi}}
\newcommand{\epsmodel}{\epsilon_{m}}

\newcommand{\return}{\mathcal{R}}

%% math
\newcommand{\cY}{\mathcal{Y}}
\newcommand{\cX}{\mathcal{X}}
\newcommand{\en}{\mathcal{E}}
\newcommand{\bu}{\mathbf{u}}
\newcommand{\bv}{\mathbf{v}}
\newcommand{\be}{\mathbf{e}}
\newcommand{\by}{\mathbf{y}}
\newcommand{\bx}{\mathbf{x}}
\newcommand{\bz}{\mathbf{z}}
\newcommand{\bw}{\mathbf{w}}
\newcommand{\bo}{\mathbf{o}}
\newcommand{\bs}{\mathbf{s}}
\newcommand{\ba}{\mathbf{a}}
\newcommand{\bM}{\mathbf{M}}
\newcommand{\ot}{\bo_t}
\newcommand{\st}{\bs_t}
\newcommand{\at}{\ba_t}
\newcommand{\op}{\mathcal{O}}
\newcommand{\opt}{\op_t}
\newcommand{\kl}{D_\text{KL}}
\newcommand{\tv}{D_\text{TV}}
\newcommand{\ent}{\mathcal{H}}
\newcommand{\bG}{\mathbf{G}}
\newcommand{\byk}{\mathbf{y_k}}
\newcommand{\bI}{\mathbf{I}}
\newcommand{\bg}{\mathbf{g}}
\newcommand{\bV}{\mathbf{V}}
\newcommand{\bD}{\mathbf{D}}
\newcommand{\bR}{\mathbf{R}}
\newcommand{\bQ}{\mathbf{Q}}
\newcommand{\bA}{\mathbf{A}}
\newcommand{\bN}{\mathbf{N}}
\newcommand{\bS}{\mathbf{S}}
\newcommand{\bW}{\mathbf{W}}
\newcommand{\bU}{\mathbf{U}}
\newcommand{\bO}{\mathbf{O}}

\newcommand{\bzhi}{\bz^\text{hi}}

\newcommand{\expected}{\mathbb{E}}
\newcommand{\E}{\mathbb{E}}
\newcommand{\srank}{\text{srank}}
\newcommand{\rank}{\text{rank}}
\newcommand{\deepnet}{\bW_N(k, t) \bW_\phi(k, t)}
\newcommand{\features}{\bW_\phi(k, t)}
\newcommand{\stateactioni}{[\bs_i; \ba_i]}
\newcommand{\diag}{\text{\textbf{diag}}}

\def\thetaP{\theta^{\prime}}
\def\cf{\emph{c.f.}\ }
\def\vs{\emph{vs}.\ }
\def\etc{\emph{etc.}\ }
\def\Eqref#1{Equation~\ref{#1}}

\newenvironment{repeatedthm}[1]{\@begintheorem{#1}{\unskip}}{\@endtheorem}
\newcommand{\algname}{COMBO\xspace}

\newtheorem{theorem}{Theorem}[section]
\newtheorem{sketchtheorem}{Sketch Theorem}[section]
\newtheorem{lemma}[theorem]{Lemma}
\newtheorem{corollary}[theorem]{Corollary}
\newtheorem{proposition}[theorem]{Proposition}
\newtheorem{definition}[theorem]{Definition}
\newtheorem{conjecture}[theorem]{Conjecture}
\newtheorem{problem}[theorem]{Problem}
\newtheorem{formulation}[theorem]{Formulation}
\newtheorem{claim}[theorem]{Claim}
\newtheorem{remark}[theorem]{Remark}
\newtheorem{example}[theorem]{Example}
\newtheorem{assumption}[theorem]{Assumption}
\newtheorem{exercise}[theorem]{Exercise}


\newcommand{\indep}{\rotatebox[origin=c]{90}{$\models$}}

\renewcommand{\mathbf}{\boldsymbol}

\newcommand{\conv}{\circledast}
\newcommand{\mb}{\mathbf}
\newcommand{\mc}{\mathcal}
\newcommand{\mf}{\mathfrak}
\newcommand{\md}{\mathds}
\newcommand{\bb}{\mathbb}
\newcommand{\msf}{\mathsf}
\newcommand{\mcr}{\mathscr}
\newcommand{\magnitude}[1]{ \left| #1 \right| }
\newcommand{\set}[1]{\left\{ #1 \right\}}
\newcommand{\condset}[2]{ \left\{ #1 \;\middle|\; #2 \right\} }


\newcommand{\reals}{\bb R}
\newcommand{\proj}{\mathrm{proj}}

\newcommand{\eps}{\varepsilon}
\newcommand{\R}{\reals}
\newcommand{\Cp}{\bb C}
\newcommand{\Z}{\bb Z}
\newcommand{\N}{\bb N}
\newcommand{\Sp}{\bb S}
\newcommand{\Ba}{\bb B}
\newcommand{\indicator}[1]{\mathbbm 1\left\{#1\right\}}
\renewcommand{\P}{\mathbb{P}}
\newcommand{\rvline}{\hspace*{-\arraycolsep}\vline\hspace*{-\arraycolsep}}
\makeatletter
\def\Ddots{\mathinner{\mkern1mu\raise\p@
\vbox{\kern7\p@\hbox{.}}\mkern2mu
\raise4\p@\hbox{.}\mkern2mu\raise7\p@\hbox{.}\mkern1mu}}
\makeatother
% to declare new operator
% \DeclareMathOperator{\xxx}{xxx}

%% Other definitions

\newcommand{\event}{\mc E}

\newcommand{\e}{\mathrm{e}}
\newcommand{\im}{\mathrm{i}}
\newcommand{\rconcave}{r_\fgecap}
\newcommand{\Lconcave}{\mc L^\fgecap}
\newcommand{\rconvex}{r_\fgecup}
\newcommand{\Rconvex}{R_\fgecup}
\newcommand{\Lconvex}{\mc L^\fgecup}

\newcommand{\wh}{\widehat}
\newcommand{\wt}{\widetilde}
\newcommand{\ol}{\overline}


\newcommand{\betaconcave}{\beta_\fgecap}
\newcommand{\betagrad}{\beta_{\mathrm{grad}}}

\newcommand{\norm}[2]{\left\| #1 \right\|_{#2}}
\newcommand{\abs}[1]{\left| #1 \right|}
\newcommand{\row}[1]{\text{row}\left( #1 \right)}
\newcommand{\innerprod}[2]{\left\langle #1,  #2 \right\rangle}
\newcommand{\prob}[1]{\bb P\left[ #1 \right]}
\newcommand{\expect}[1]{\bb E\left[ #1 \right]}
\newcommand{\function}[2]{#1 \left(#2\right}
\newcommand{\integral}[4]{\int_{#1}^{#2}\; #3\; #4}
\newcommand{\paren}[1]{\left( #1 \right)}
\newcommand{\brac}[1]{\left[ #1 \right]}
\newcommand{\Brac}[1]{\left\{ #1 \right\}}

% Adding new defs here
\newcommand{\moff}{m_\mr{off}}
\newcommand{\mon}{m_\mr{on}}
\newcommand{\EmpiricalOffline}{\wh{\bb E}_{\mc D^\nu_h}}
\newcommand{\EmpiricalOnline}{\wh{\bb E}_{\mc D^\tau_h}}
\newcommand{\Deltaoff}{\Delta_\mr{off}}
\newcommand{\Deltaon}{\Delta_\mr{on}}
\newcommand{\Vmax}{V_{\max}}
\newcommand{\regret}{\mr{Reg}}
\newcommand{\regreton}{\mr{Sub}_{\mr {on}}}
\newcommand{\regretoff}{\mr{Sub}_{\mr {off}}}
\newcommand{\Doff}{\mc D_\mr{off}}
\newcommand{\Don}{\mc D_\mr{on}}
\newcommand{\piref}{\pi_\mr{ref}}

\newcommand{\nt}[1]{{\color{purple}{\bf [Next: #1]}}}
\newcommand{\here}{{\color{purple}{\bf [Writing here]}}}
% \newcommand{\todo}{{\color{purple}{\bf [TODO]}}}
\newcommand{\sz}[1]{{\color{blue}{\bf [Simon: #1]}}}
% \newcommand{\note}[1]{{\color{red}{\bf [note: #1]}}}
\newcommand{\mr}{\mathrm}
\newcommand{\sym}{\mathrm{Sym}}
\newcommand{\sks}{\mathrm{Skew}}
\newcommand{\inprod}[2]{\langle#1,#2\rangle}
\newcommand{\parans}[1]{\left(#1\right}
\newcommand{\clip}{\msf{clipped}}
\newcommand{\beha}{\msf b}
\numberwithin{equation}{section}

% \def \endprf{\hfill {\vrule height6pt width6pt depth0pt}\medskip}

\newcommand{\cdsmethodname}{CDS}
\newcommand{\udsmethodname}{UDS}
\newcommand{\ptrmethodname}{PTR~}
\newcommand{\primemethodname}{PRIME~}
\newcommand{\arxiv}[1] {{\color{black} #1}}

% \newenvironment{proof}{\noindent {\bf Proof} }{\endprf\par}

% \newcommand{\qed}{{\unskip\nobreak\hfil\penalty50\hskip2em\vadjust{}
%            \nobreak\hfil$\Box$\parfillskip=0pt\finalhyphendemerits=0\par}}


\newcommand\myworries[1]{\textcolor{red}{#1}}
\newcommand\running[1]{\textcolor{blue}{#1}}

\newcommand{\benchl}[1]{{\scriptsize\textsf{#1}}\normalsize\xspace}
\newcommand{\bench}[1]{{\fontsize{8.5}{10}\selectfont\textsf{#1}}\normalsize\xspace}
\newcommand{\code}[1]{{\fontsize{8.5}{1}\selectfont{\tt #1}}\xspace}
\newcommand{\codebold}[1]{{\fontsize{8.5}{1}\selectfont{\tt \textbf{#1}}}\xspace}
\newcommand{\xx}[1]{\textcolor{black}{#1}}
\newcommand{\xxs}[1]{\textcolor{black}{\scriptsize\textsf{#1}}}
\newcommand{\supertiny}[1]{\fontsize{5}{4}\selectfont{#1}}
\newcommand{\xxred}[1]{\textcolor{red}{\textsf{#1}}}
% \DeclareMathOperator{\EX}{\mathbb{\hat{E}}}% expected value
% \makeatletter
% \newcommand\footnoteref[1]{\protected@xdef\@thefnmark{\ref{#1}}\@footnotemark}
% \makeatother
% \usepackage{scrextend}

% \deffootnote[1em]{1em}{1em}{\textsuperscript{\thefootnotemark}\,}
% \newcolumntype{P}[1]{>{\centering\arraybackslash}p{#1}}
% \newcommand\mycommfont[1]{\footnotesize\ttfamily\textcolor{blue}{#1}}

\newcommand\aviral[1]{\textcolor{red}{aviralkumar@: #1}}
\newcommand\ayazdan[1]{\textcolor{red}{ayazdan@: #1}}
\newcommand\sv[1]{\textcolor{red}{SV@: #1}}
\newcommand\fix[1]{\textcolor{green}{#1}}

\newcommand{\round}[1]{\ensuremath{\lfloor#1\rceil}}

\newcommand{\tgray}[1]{\colorbox{lightgray}{\textbf{#1}}}
\newcommand{\finalcheck}[1]{\textcolor{green}{#1}}

\newcommand{\review}[1]{#1}

\newcommand{\niparagraph}[1]{\vspace{2pt}\noindent\textbf{#1}}

\usepackage{amsmath}
\usepackage{amssymb}
\usepackage{mathtools}
\usepackage{natbib}
\usepackage{enumerate}
\usepackage{tikz}
\usepackage{graphicx}

\DeclarePairedDelimiter\abs{\lvert}{\rvert}%
\DeclarePairedDelimiter\norm{\lVert}{\rVert}%
\DeclarePairedDelimiter\ceil{\lceil}{\rceil}
\DeclarePairedDelimiter\floor{\lfloor}{\rfloor}

\newcommand{\TODO}[1]{\textcolor{red}{TODO: #1}}
\newcommand{\textdiff}[1]{\textcolor{red}{#1}}
\newcommand{\citemissing}{\textcolor{red}{(cite?)}}

\newcommand{\normt}[1]{\left\lVert#1\right\rVert_2}
\newcommand{\normtmu}[1]{\left\lVert#1\right\rVert_{2, \mu}}
\newcommand{\normm}[1]{\left\lVert#1\right\rVert}
\newcommand{\norminf}[1]{\left\lVert#1\right\rVert_\infty}
\newcommand{\normtt}[1]{\left\lVert#1\right\rVert^2_2}

\newcommand{\half}{\frac{1}{2}}
\newcommand{\fourth}{\frac{1}{4}}
\newcommand{\vect}[1]{\overrightafrrow{\textbf{#1}}}
\newcommand{\phat}{\hat{p}}
\newcommand{\KL}[2]{D_{KL}(#1||#2)}
\newcommand{\TV}[2]{D_{TV}(#1||#2)}
\newcommand{\ind}[1]{1[#1]}
\newcommand{\pardiv}[1]{\frac{\partial}{\partial #1}}
\newcommand{\parHess}[1]{\frac{\partial^2}{\partial #1 ^2}}
\newcommand{\Xhat}{{\hat{X}}}
\newcommand{\xhat}{{\hat{x}}}
\newcommand{\defeq}{\mathrel{\stackrel{\makebox[0pt]{\mbox{\normalfont\tiny def}}}{=}}}

\newcommand{\argmax}[1]{\underset{#1}{\textrm{argmax}}\ }
\newcommand{\argmin}[1]{\underset{#1}{\textrm{argmin}}\ }
\newcommand{\grad}[1]{\nabla{#1}}
\newcommand{\innerp}[2]{\langle{#1,#2}\rangle}
\newcommand{\Hess}[1]{\nabla^2{#1}}
\newcommand{\EXP}[1]{\text{exp}\{#1\}}

% debug q
\newcommand{\Proj}{\Pi}
\newcommand{\Projmu}{\Pi_\mu}
\newcommand{\trans}{T}
\newcommand{\backup}{\mathcal{T}}
\newcommand{\Qclass}{\mathcal{Q}}
\newcommand{\ReplayBuffer}{\mathcal{B}}
\newcommand{\ltwonorm}{L_2}
\newcommand{\lpnorm}{L_p}
\newcommand{\linfnorm}{L_\infty}

\newcommand{\UniformVec}{U[-1,1]^{32}}

\usepackage{amsmath,amssymb,amsthm}
\usepackage{algorithm,algorithmic}
\usepackage{mathtools}
\newtheorem{claim}{Claim}
\newtheorem{assumption}{Assumption}
\newtheorem{theorem}{Theorem}[section]
\newtheorem{lemma}[theorem]{Lemma}
\newtheorem{corollary}[theorem]{Corollary}
\newtheorem{remark}[theorem]{Remark}
\newtheorem{proposition}[theorem]{Proposition}
\newtheorem{definition}{Definition}
\renewcommand\theassumption{A\arabic{assumption}}


\def\shownotes{1} 
\ifnum\shownotes=1
\newcommand{\authnote}[2]{{[#1: #2]}}
\else 
\newcommand{\authnote}[2]{{}}
\fi
\newcommand{\tnote}[1]{{\color{orange}\authnote{TM}{#1}}}

\title{Value-Based Deep Reinforcement Learning Requires Explicit Regularization}
%%AK: Somehow aliasing doesnt seem to fit well here.... Is there a different word we could use?
%%SL.5.13: I feel like we can find a more inspiring title. Right now there are two "weak" words in the beginning: "Mitigating" and "Excessive" -- the former suggests "a partial and incomplete solution" and the latter suggests "something that is not too bad, but problematic in large quantities." Together this kind of sounds like it's an incomplete solution to something that is not a big problem. I think we need a stronger title. A few potential ideas:
% Stable Representations for Offline Reinforcement Learning
% Offline Reinforcement Learning with Feature Stabilization
% Offline Reinforcement Learning without Feature Aliasing (or, more generally Offline Reinforcement learning without [something bad])
% any other ideas?
%%AK.5.16: We can also do the form of [Methodname]: Stable offline RL without Feature aliasing or something like that?
%%SL.5.17: It's a little bit of a reach, but I think we can afford to drop "value-based" in the title and just write "Deep Reinforcement Learning Requires Explicit Regularization". I think that will be a great public title. That said, for the submission, one important thing is that we really should try to get a more theory-focused reviewer (or should we?), in which case we could add something like:
% Deep Reinforcement Learning Requires Explicit Regularization: a Theoretical and Empirical Analysis of Implicit and Explicit Regularization in Offline RL (it's horribly long, but perhaps the word "theoretical" will index into the right reviewer pool)

% The \author macro works with any number of authors. There are two commands
% used to separate the names and addresses of multiple authors: \And and \AND.
%
% Using \And between authors leaves it to LaTeX to determine where to break the
% lines. Using \AND forces a line break at that point. So, if LaTeX puts 3 of 4
% authors names on the first line, and the last on the second line, try using
% \AND instead of \And before the third author name.

\author{%
  David S.~Hippocampus\thanks{Use footnote for providing further information
    about author (webpage, alternative address)---\emph{not} for acknowledging
    funding agencies.} \\
  Department of Computer Science\\
  Cranberry-Lemon University\\
  Pittsburgh, PA 15213 \\
  \texttt{hippo@cs.cranberry-lemon.edu} \\
  % examples of more authors
  % \And
  % Coauthor \\
  % Affiliation \\
  % Address \\
  % \texttt{email} \\
  % \AND
  % Coauthor \\
  % Affiliation \\
  % Address \\
  % \texttt{email} \\
  % \And
  % Coauthor \\
  % Affiliation \\
  % Address \\
  % \texttt{email} \\
  % \And
  % Coauthor \\
  % Affiliation \\
  % Address \\
  % \texttt{email} \\
}

\begin{document}

\maketitle

\begin{abstract}


While deep reinforcement learning (RL) methods present an appealing approach to sequential decision making, such methods are often difficult to use in practice due to their instability. What accounts for this instability? In this paper, we focus on the specific case of offline deep RL, and show that value-based offline deep RL methods can learn degenerate features. In supervised learning, deep networks enjoy the benefits of implicit regularization, which leads to effective generalization despite overparameterization. We show that, in the case of deep RL, implicit regularization in offline RL can instead lead to degenerate features.
Specifically, features learned by the Q-network at state-action tuples appearing on both sides of the Bellman update ``co-adapt'' to each other, giving rise to poor
solutions. We show that this holds both in theory and practice. In fact, even when initializing a network close to an optimal solution, feature co-adaptation
can lead the model away from this optimum. To address the adverse impacts of the co-adaptation induced by the implicit regularization, we propose a simple and effective explicit regularizer, \methodname, that  minimizes similarity of learned features of the Q-network at state-tuples appearing on both sides of the Bellman update. Empirically when combined with several existing offline RL methods, \methodname\ improves both performance and stability on Atari 2600 games, D4RL domains, and robotic manipulation from images.
\end{abstract}

%%SL.5.22: It would be good if we could say something more specific than "poor features" ("degenerate" might be a better word if we can't think of anything better)
%%SL.5.22: Same comment, would be good if we can say something other than "poor"
%%SL.5.22: I don't think we said anything about gradient descent prior to this -- perhaps remove mention of gradient descent here?

\vspace{-0.1cm}
\section{Introduction}
\vspace{-0.1cm}
Deep neural networks are overparameterized, with billions of parameters, which in principle should leave them vulnerable to overfitting. Despite this, supervised learning with deep networks still learn representations that generalize  well. A widely held consensus is that deep nets find simple solutions that generalize due to various \emph{implicit} regularization effects~\citep{blanc2020implicit,woodworth2020kernel,arora2018optimization,gunasekar2017implicit,wei2019regularization,li2019towards}. We may surmise that using deep neural nets in reinforcement learning~(RL) will work well for the same reason, learning effective representations that generalize due to such implicit regularization effects. But is this actually the case for value functions trained via bootstrapping? 

In this paper, we argue that, while implicit regularization leads to effective representations in supervised deep learning, it may lead to poor learned representations when training overparameterized deep network value functions. 
In order to rule out confounding effects from exploration and non-stationary data distributions, we focus on the offline RL setting -- where deep value networks must be trained from a static dataset of experience.
There is already evidence that value functions trained via bootstrapping learn poor representations: value functions trained with offline deep RL eventually degrade in performance~\citep{agarwal2019optimistic, kumar2021implicit} and this degradation is correlated with the emergence of low-rank features
in the value network~\citep{kumar2021implicit}.
Our goal is to understand the underlying cause of the emergence of poor representations during bootstrapping and develop a potential solution. Building on the theoretical framework developed by \citet{blanc2020implicit,damian2021label}, we characterize the implicit regularizer that arises when training deep value functions with TD learning. The form of this implicit regularizer implies that TD-learning would co-adapt feature representations at state-action tuples that appear on either side of a Bellman backup.

We show that this theoretically predicted aliasing phenomenon manifests in practice as feature \textbf{co-adaptation}, where the features of consecutive state-action tuples learned by the Q-value network become very similar in terms of their dot product~(\Secref{sec:problem}). This co-adaptation co-occurs with oscillatory learning dynamics, and training runs that exhibit feature co-adaptation typically converge to poorly performing solutions. Even when Q-values are not overestimated, prolonged training in offline RL can result in performance degradation as feature co-adaptation increases. 
To mitigate this co-adaptation issue, which arises as a result of implicit regularization, we propose an \emph{explicit regularizer} that we call \methodname~(\Secref{sec:method}).
%%SL.9.29: This is a relatively minor thing, but when you introduce the name DR3, can you actually say what it stands for?
%%AK: I was trying to cook up a full form, but it seems like the reason why we put it this way was "DR3: Deep Reinforcement Learning Requires Explicit Regularization", but maybe this is not the best thing to call a method?
While exactly estimating and cancelling the effects of the theoretically derived implicit regularizer is computationally difficult, \methodname\ provides a simple and tractable theoretically-inspired approximation that mitigates the issues discussed above. In practice, \methodname\ amounts to regularizing the features at consecutive state-action pairs to be dissimilar in terms of their dot-product similarity. Empirically, we find that \methodname\ prevents previously noted pathologies such as feature rank collapse~\citep{kumar2021implicit},  gives methods that train for longer and improves performance relative to the base offline RL method employed in practice.
% Empirically, we find that \methodname\ allows neural network Q-functions to use their full representational capacity, as measured by the rank of the learned features, and \textcolor{red}{enables the use of larger, more expressive neural networks}, 
% %%SL.9.29: How do we test it allows them to use their full capacity? Perhaps we should remove this claim, since it doesn't seem like we have any evidence for it.
% %%AK: rank of learned features is higher? 
% %%SL.10.27: I'm still concerned that this statement may not be backed up by evidence. Do we actually show that this *allows* using larger networks (i.e., larger net + DR3 = good, but larger net - DR3 = bad?) I think we would need something like that to back up this statement
% giving rise to methods that can train for longer without degradation and thus reach a better solution~(\Secref{sec:experiments}).

Our first contribution is the derivation of the implicit regularizer that arises when training deep net value functions via TD learning, and an empirical demonstration that it manifests as \emph{feature co-adaptation} in the offline deep RL setting.
%, which results in highly similar feature representations for state-action tuples at consecutive time steps. 
Feature co-adaptation accounts at least in part for some of the challenges of offline deep RL, including degradation of performance with prolonged training. Second, we propose a simple and effective \emph{explicit} regularizer for offline value-based RL, \methodname, which minimizes the feature similarity between state-action pairs appearing in a bootstrapping update. \methodname\ is inspired by the theoretical derivation of the implicit regularizer, it alleviates co-adaptation and can be easily combined with modern offline RL methods, such as REM~\citep{agarwal2019optimistic}, CQL~\citep{kumar2020conservative}, and BRAC~\citep{wu2019behavior}. Empirically, using \methodname\ in conjunction with existing offline RL methods provides about \textbf{60\%} performance improvement on the harder D4RL~\citep{fu2020d4rl} tasks, and \textbf{160\%} and \textbf{25\%} stability gains for REM and CQL, respectively, on offline RL tasks in 17 Atari 2600 games. Additionally, we observe large improvements on image-based robotic manipulation tasks~\citep{singh2020cog}.


\section{Preliminaries}
\vspace{-0.15cm}
\label{sec:prelim}

\textbf{Offline RL.} Standard RL considers a Markov decision process (MDP), $\mdp =(\states, \actions, P, \gamma, R)$, where $\states$ and $\actions$ denote the state and action spaces respectively, $P(\bs' | \bs, \mathbf{a})$ denotes the dynamics, $\gamma \in [0, 1)$ is the discount factor, and $R$ correspond to the reward function. Offline RL tackles the problem of learning a policy $\pi(\mathbf{a}|\bs)$ from a static dataset with $\mathcal{D}$, generated by a behavior policy $\pi_\beta(\mathbf{a}|\bs)$.


\textbf{Data sharing in offline RL.} Data sharing has been considered in the multi-task offline RL setting where there is a static multi-task dataset with $\mathcal{D} = \cup_{i=1}^N \mathcal{D}_i$ where $N$ is the number of tasks. Prior works~\citep{kalashnikov2021mt,eysenbach2020rewriting,yu2021conservative} show that sharing data from different tasks to task $i$ to be conducive. To do so, these prior methods assume access to the functional form of the reward $r_i$. This is a strong assumption in practice, as it necessitates access to a functional (programmatic) form for the reward function. In offline RL, it might be desirable to simply label the reward function by hand, but then the algorithm does not have access to the functional form of the reward, and all unlabeled data also needs to be labeled by hand for use with such methods. Our aim in this paper is to utilize unlabeled data without any reward labels at all.
If however functional access to the reward \emph{is} available, a simple strategy is to na\"ively share data across all tasks, which we refer to as Sharing All. Formally, Sharing All defines the dataset of transitions relabeled from task $j$ to task $i$ as $\mathcal{D}_{j \rightarrow i}$ and the method can be then defined as
    $\mathcal{D}^\mathrm{eff}_i := \mathcal{D}_i \cup ( \cup_{j \neq i} \mathcal{D}_{j \rightarrow i})$,
where $\mathcal{D}^\mathrm{eff}_i$ denotes the effective dataset for task $i$. Therefore, the policy optimization objective in Sharing All can be written as follows:
\begin{equation*}
     \forall i \in [N], ~~\pi^*(\mathbf{a}|\bs, i) := \arg \max_{\pi}~~ J_{\mathcal{D}^\mathrm{eff}_i}(\pi) - \alpha D(\pi, \pi^\mathrm{eff}_\beta),
\end{equation*}
where $\pi_\beta^\mathrm{eff}(\mathbf{a}|\bs, i)$ is the effective behavior policy for task $i$ denoted as $\pi_\beta^\mathrm{eff}(\mathbf{a}|\bs, i) := |\mathcal{D}^\mathrm{eff}_i(\bs, \mathbf{a})| / |\mathcal{D}^\mathrm{eff}_i(\bs)|$, $J_{\mathcal{D}^\mathrm{eff}_i}(\pi)$ denotes the average return of policy $\pi$ in the empirical MDP induced by the effective dataset, and $D(\pi, \pi^\mathrm{eff}_\beta)$ denotes a divergence measure (e.g., KL-divergence~\citep{jaques2019way,wu2019behavior}, fisher divergence~\citep{kostrikov2021offline}, MMD distance~\citep{kumar2019stabilizing} or $D_{\text{CQL}}$ from conservative Q-values~\citep{kumar2020conservative}) between the learned policy $\pi$ and the effective behavior policy $\pi_\beta^\mathrm{eff}$. Note that conservative Q-values refer to the Q-value for a given policy corresponding to a modified reward function $r(\bs, \mathbf{a}) - \alpha \pi(\mathbf{a}|\bs) \cdot (\pi(\mathbf{a}|\bs) / \pi_\beta(\mathbf{a}|\bs) - 1)$, computed on the empirical MDP. We also note that Sharing All can be easily adapted to the single-task setting where there is only one target task with labeled data $\mathcal{D}_\text{L}$ and unlabeled prior data $\mathcal{D}_\text{U}$. While data sharing tends to show promising results, it requires the assumption of the access to the functional form of the reward function. We instead focus on the data sharing problem where we do not make such an assumption and instead, only have the reward labels for originally commanded task, which we will discuss in the following section.




% \vspace{-0.2cm}
% \subsection{Implicit Regularization in Deep RL via TD-Learning}
% \vspace{-0.2cm}
While the ``deadly-triad''~\citep{suttonrlbook} suggests that training value function approximators with bootstrapping off-policy can lead to divergence, modern deep RL algorithms have been able to successfully combine these properties~\citep{Hasselt2018DeepRL}. However, making too many TD updates to the Q-function in offline deep RL is known to sometimes lead to performance degradation and unlearning, even for otherwise effective modern algorithms~\citep{fu2019diagnosing, fedus2020revisiting,agarwal2019optimistic,kumar2021implicit}. Such unlearning is not typically observed when training overparameterized models via supervised learning, so what about TD learning is responsible for it? We show that one possible explanation behind this pathology is the implicit regularization induced by minimizing TD error on a deep Q-network. Our theoretical results suggest that this implicit regularization ``co-adapts'' the representations of state-action pairs that appear in a Bellman backup (we will define this more precisely below).
Empirically, this typically manifests as ``co-adapted'' features for consecutive state-action tuples, even with specialized TD-learning algorithms that account for distributional shift, and this in turn leads to poor final performance both in theory and in practice. We first provide empirical evidence of this co-adaptation phenomenon in Section~\ref{app:problem_more} (additional evidence in Appendix~\ref{app:more_evidence_coadaptation}) and then theoretically characterize the implicit regularization in TD learning, and discuss how it can explain the co-adaptation phenomenon in Section~\ref{sec:dr3_theory}.


\begin{figure}[t]
    \centering
    \vspace{-5pt}
    \includegraphics[width=0.67\linewidth]{chapters/dr3/figures_iclr/final_plot.pdf}~\vline~\vline~
    \includegraphics[width=0.32\linewidth]{chapters/dr3/figures_iclr/final_dqn_fig.pdf}
    \vspace{-0.3cm}
    \caption{\small{Feature dot-products $\phi(\bs, \mathbf{a})^\top \phi(\bs', \mathbf{a}')$ increase during training when backing up from \emph{out-of-sample} but in-distribution actions (\textbf{TD-learning}: left, \textbf{Q-learning}: right), though the average Q-value converges and stays relatively constant. Using only seen state-action pairs for backups (\textbf{offline SARSA}) or not performing Bellman backups (i.e., \textbf{supervised regression}) avoids this issue, with stable and relatively low dot products. \textit{Left}: TD-learning with high feature dot products eventually destabilizes and produces incorrect Q-values, \textit{Right}: DQN attains extremely large feature dot products, despite a relatively stable trend in Q-values.}}  
    \label{fig:dot_products}
    \vspace{-0.3cm}
\end{figure}


\vspace{-0.2cm}
\subsection{Feature Co-Adaptation And Implicit Regularization}
\label{app:problem_more}
\vspace{-0.2cm}

In this section, we empirically identify a \emph{feature co-adaptation} phenomenon that appears when training value functions via bootstrapping, where the feature representations of consecutive state-action pairs exhibit a large value of the dot product $\phi(\bs, \mathbf{a})^\top \phi(\bs', \mathbf{a}')$. Note that feature co-adaptation may arise because of high cosine similarity or because of high feature norms. Feature co-adaptation appears even when there is no explicit objective to increase feature similarity.

\textbf{Experimental setup.} We ran supervised regression and three variants of approximate dynamic programming (ADP)
on an offline dataset consisting of 1\% of uniformly-sampled data from the replay buffer of DQN on two Atari games, previously used in \citet{agarwal2019optimistic}. First, for comparison, we trained a Q-function via \textbf{supervised regression} to Monte-Carlo~(MC) return estimates on the offline dataset to estimate the value of the behavior policy. Then, we trained variants of ADP which differ in the selection procedure for the action $\mathbf{a}'$ that appears in the target value in the TD-error. The \textbf{offline SARSA} variant aims to estimate the value of the behavior policy, $Q^{\pi_\beta}$, and sets $\mathbf{a}'$ to the actual action observed at the next time step in the dataset, such that $(\bs', \mathbf{a}') \in \mathcal{D}$. The \textbf{TD-learning} variant also aims to estimate the value of the behavior policy, but utilizes the expectation of the target Q-value over actions $\mathbf{a}'$ sampled from the behavior policy $\pi_\beta$, $\mathbf{a}' \sim \pi_\beta(\cdot|\bs')$. We do not have access to the functional form of $\pi_\beta$ for the experiment shown in Figure~\ref{fig:dot_products} since the dataset corresponds to the behavior policy induced by the replay buffer of an online DQN, so we train a model for this policy using supervised learning. However, we see similar results comparing \textbf{offline SARSA} and \textbf{TD-learning} on a gridworld domain where we can access the exact functional form of the behavior policy in Appendix~\ref{app:exact_behavior_policy}. %Unlike SARSA, this action $\mathbf{a}'$ may be different from the one in $\mathcal{D}$, though it comes from the same distribution.
All of the methods so far estimate $Q^{\pi_\beta}$ using different target value estimators.
We also train \textbf{Q-learning}, which chooses the action $\mathbf{a}'$ to maximize the learned Q-function. While Q-learning learns a different Q-function, we can still compare the relative stability of these methods to gain intuition about the learning dynamics. %To measure feature co-adaptation, we track the average dot product between the learned features at consecutive state-action tuples, $\text{sim}(\bs, \mathbf{a}, \bs', \mathbf{a}') := \phi(\bs, \mathbf{a})^\top \phi(\bs', \mathbf{a}')$. 
In addition to feature dot products $\phi(\bs, \mathbf{a})^\top \phi(\bs', \mathbf{a}')$, we also track the average prediction of the Q-network over the dataset to measure whether the predictions diverge or are stable in expectation.

%%AK: also need to figure out an arrangement for figures in the paper
\textbf{Observing feature co-adaptation empirically.} As shown in Figure~\ref{fig:dot_products} (right), the average dot product (top row) between features at consecutive state-action tuples continuously increases for both Q-learning and TD-learning (after enough gradient steps), whereas it flatlines and converges to a small value for supervised regression. We might at first think that this is simply a case of Q-learning failing to converge. However, the bottom row shows that the average Q-values do in fact converge to a stable value. Despite this, the optimizer drives the network towards higher feature dot products. There is no explicit term in the TD error objective that encourages this behavior, 
% and in fact the TD error relatively stays flat \textcolor{red}{(Figure ??)} during training, 
indicating the presence of some implicit regularization phenomenon. This \emph{implicit} preference towards maximizing the dot products of features at consecutive state-action tuples is what we call ``feature co-adaptation.''

\textbf{When does feature co-adaptation emerge?} Observe in Figure~\ref{fig:dot_products} (right) that the feature dot products for offline SARSA converge quickly and are relatively flat, similarly to supervised regression. This indicates that utilizing a bootstrapped update alone is not responsible for the increasing dot-products and instability, because while offline SARSA uses backups, it behaves similarly to supervised MC regression. Unlike offline SARSA, feature co-adaptation emerges for TD-learning, which is surprising as TD-learning also aims to estimate the value of the behavior policy, and hence should match offline SARSA in expectation. The key difference is that while offline SARSA always utilizes actions $\mathbf{a}'$ observed in the training dataset for the backup, TD-learning may utilize potentially unseen actions $\mathbf{a}'$ in the backup, even though these actions $\mathbf{a}' \sim \pi_\beta(\cdot|\bs')$ are \emph{within} the distribution of the data-generating policy. This suggests that utilizing \textbf{out-of-sample} actions in the Bellman backup, even when they are not out-of-distribution, critically alters the learning dynamics. This is distinct from the more common observation in offline RL, which attributes training challenges to out-of-distribution actions, but not out-of-sample actions. The theoretical model developed in Section~\ref{sec:dr3_theory} will provide an explanation for this observation with a discussion about how feature co-adaption caused due to out-of-sample actions can be detrimental in offline RL. 

% While action $\mathbf{a}'$ can be out-of-distribution in the case of Q-learning, this action is sampled from the data-generating distribution of the behavior policy for TD-learning and is hence, not out-of-distribution in this case.

%%AK: does it feel like a jump here, or is it fine?
% The only difference between offline SARSA and Q-learning is whether unseen actions are used to compute the Bellman target\gjt{I don't think that is an accurate statement. The action selection mechanism is different.}:
%%SL.9.17: Yeah, this is correct. Maybe it would be better to mention the TD thing as well in the first paragraph, rather than introducing it for the first time in this paragraph below? That would avoid one very likely criticism you would otherwise get about the comparison between SARSA and Q-learning being non-sensical because they are just learning totally different things
% while $(\bs', \mathbf{a}')$ used in Q-learning may be unobserved in $\mathcal{D}$, the $(\bs', \mathbf{a}')$ tuples used in SARSA appear in the dataset, i.e., $(\bs', \mathbf{a}') \in \mathcal{D}$÷.
%%SL.9.17: some readers won't understand why
% Thus, we might wonder if the use of OOD actions in the backup is the primary culprit for the difference in the observed trends. 

% To understand if this is the case, in Figure~\ref{fig:dot_products} (left), we compare offline SARSA to another variant that we call \textbf{TD-learning},
% %%AK: need to rename TD-learning to something else?
% which utilizes potentially unseen but in-distribution actions sampled from the behavior policy for the Bellman update, such that $\mathbf{a}' \sim \pi_\beta(\cdot|\bs')$. Like offline SARSA, TD-learning also aims to estimate the value of the behavior policy $Q^{\pi_\beta}$, but it differs from offline SARSA in that the actions $\mathbf{a}'$ are sampled from the data-generating distribution of the behavior policy. Thus, while they are not \emph{out-of-distribution}, they are also generally not actions that were seen in the dataset $\mathcal{D}$, in contrast to offline SARSA.
% %%SL.9.17: This paragraph is already really hard to read, adding a footnote that creates another branch point for the reader creates an overwhelming cognitive load. Try to merge this footnote into the paragraph, and generally just shorter, more concise, and more digestible sentences.
% % to compute the value of the behavior policy $Q^{\pi_\beta}$. \textbf{Offline SARSA} and \textbf{TD-learning} should match in expectation. 
% As expected, we observe in Figure~\ref{fig:dot_products} (left) that the Q-values predicted by TD-learning match those learned by \textbf{offline SARSA}. However, in contrast to \textbf{offline SARSA}, we find that the dot product similarity for TD-learning increases after enough gradient steps, and this is accompanied by instability in the Q-values. This trend is absent in offline SARSA. This suggests that utilizing out-of-sample actions (even when they are not OOD) in the Bellman backup critically alters the learning dynamics. The theoretical model developed in Section~\ref{sec:theory} provides an explanation for this observation. 
%%AK: one concern: we overload "TD-learning" to mean generic bootstrapping algorithms as well as the "TD learning" baseline we plot. Should we change one of them to something else? Like calling TD-learning generically as bootstrapping and TD learning baseline as FQE.

%%SL.9.17: You can probably shorten this paragraph if you want to save some space
 
% What is the implicit mechanism causing co-adaptation, and why does it occur only when using out-of-sample state-action tuples for the backup? This issue is not corrected by existing offline RL methods, which only avoid out-of-distribution actions~\citep{levine2020offline}, so how does co-adaptation affect offline RL performance? In the next section, we will show how an implicit regularization effect that is studied as a potential benefit of SGD in supervised deep learning can explain this phenomenon, and how this leads to a number of issues in the RL setting.

% \textbf{How do Bellman backups with out-of-sample actions behave?} 
% As shown in Figure~\ref{fig:dot_products} (left), with \textbf{TD-learning}, the dot products steadily increase during training though the average Q-value predictions are roughly constant over the this phase. These Q-values on average match those learned by \textbf{offline SARSA} and \textbf{supervised regression},
% %%SL.7.13: Supervised regression doesn't appear in the left plots!
% but the dot products for \textbf{SARSA} and supervised learning remain flat. In contrast, with \textbf{TD-learning}, the dot products increase after too many gradient steps, and this is accompanied by instability in the Q-values, which then diverge. Note that the only difference between \textbf{TD-learning} and \textbf{SARSA} is the resampling of the target value action, where \textbf{TD-learning} uses a new (out-of-sample) action from the same distribution, while \textbf{SARSA} uses the dataset action.
% %these two approaches quickly converge over the course of training unlike \textbf{TD-learning}. With more training, we find that \textbf{TD-learning} exhibits unstable Q-values, which eventually diverge, whereas the predictions for both \textbf{offline SARSA} and \textbf{supervised regression} converge and do not destabilize with more training.
% % Notably, with \textbf{TD-learning}, the dot-products steadily increase during training and the Q-values eventually diverge, whereas both the dot-products and Q-values of \textbf{offline SARSA} and \textbf{supervised regression} quickly converge and remain stable throughout training (Figure 2). 
% These observations indicate that Bellman backups alone are not responsible for the increasing dot-products and instability because \textbf{offline SARSA} uses Bellman backups, but behaves similarly to \textbf{supervised regression}.
% %%SL.7.13: Again, this is not apparent from the figure, because supervised regression is not shown on the left side!
% On the other hand, \textbf{TD-learning}, which uses out-of-sample actions in the Bellman backup, exhibits increasing dot-products, which eventually ends in  instability, suggesting that out-of-sample actions critically alter the learning dynamics. The theoretical model developed in Section~\ref{sec:theory} provides an explanation for this observation.

% Next, we train two methods that estimate the value of the behavior policy that generated the dataset via dynamic programming. The first approach, which we refer to as  standard \textbf{TD-learning}, estimates $Q^{\pi_\beta}$ by minimizing TD-error $\mathcal{L}_\mathrm{TD}(\theta)$ (Equation in Section~\ref{sec:background}), 
% % $\sum_{\bs, \mathbf{a}, \bs' \in \mathcal{D}, \mathbf{a}' \sim {\pi}_\beta(\cdot | \bs')} \left(R(\bs, \mathbf{a}) + \gamma \mathbf{a}r{Q}_\theta(\bs', \mathbf{a}') - Q_\theta(\bs, \mathbf{a}) \right)^2$, 
% where the action $\mathbf{a}' \sim {\pi}_\beta(\cdot|\bs)$ is sampled from the behavior policy and can be \textit{out-of-sample} for the training dataset, i.e., $(\bs', \mathbf{a}') \notin \mathcal{D}$, but is \emph{in-distribution}. \aviral{This version is implemented by first learning a model of the behavior policy using supervised classification.} The second approach, which we refer to as \textbf{offline SARSA}, performs Bellman backups from the exact $(\bs', \mathbf{a}')$ tuple observed in the dataset $\mathcal{D}$ (i.e., $\mathbf{a}'$ appearing in $\mathcal{L}_\mathrm{TD}(\theta)$ is observed in $\mathcal{D}$).
% % minimizing $\sum_{\bs, \mathbf{a}, \bs', \mathbf{a}' \in \mathcal{D}} \left(R(\bs, \mathbf{a}) + \gamma \mathbf{a}r{Q}_\theta(\bs', \mathbf{a}') - Q_\theta(\bs, \mathbf{a}) \right)^2$). 
% Offline SARSA is the same as TD-learning in expectation. However, TD-learning uses potentially out-of-sample actions\footnote{Note that these actions are sampled from the data generating distribution, hence are not out-of-distribution, but may be absent from the training dataset.} in the backup.   As we will show, utilizing out-of-sample state-action tuples in the backup plays a critical role in the learning dynamics.
% Finally, we also train standard \textbf{Q-learning}, which chooses actions $\mathbf{a}'$ that maximize the learned Q-function. Such actions are also out-of-sample.  


%%SL.7.13: My high-level comment on this section is the following: Right now, your analysis launches into the out-of-sample action discussion right away, without adequately explaining feature co-adaptation. This feels like it's putting the cart before the horse -- before the reader has fully appreciated or even understood what co-adaptation means, they are confronted with this somewhat nuanced discussion about out-of-sample actions. Maybe we should have a couple of sentences before this that basically say "notice how the feature dot products tend to go up right around the same time that things get unstable, isn't that funny? this is what we call feature co-adaptation, let's see what kinds of methods have this issue, and what kinds don't" -- this could be written in just a few sentences, and it should come *before* the discussion of out-of-sample actions, which is secondary to this.


% \begin{figure}[t]
%     \centering
%     \vspace{-5pt}
%     \includegraphics[width=0.45\linewidth]{figures/figure1_dotproduct_dot_products_final.pdf}\\
%     ~~~\includegraphics[width=0.45\linewidth]{section3_figs/figure1_dotproduct_q_values (1).pdf}
%     \vspace{-0.24cm}
%     %%SL.5.22: Let's not label it "MC", that's not very informative. Call it "supervised" instead.
%     %%AK: TODO for me, will change it. 
%     %%AK: change labels here
%     \caption{\small{Dot-product $\text{sim}(\bs, \mathbf{a}, \bs', \mathbf{a}')$ increases through training when backing up from out-of-sample actions though the average Q-value stays relatively constant, whereas utilizing seen state-action pairs for backups or supervised learning exhibit near-constant dot product similarities as well.}}  
%     \label{fig:dot_products}
%     \vspace{-0.6cm}
% \end{figure}
% \textbf{How do Bellman backups with out-of-sample actions behave?} 
% As shown in Figure~\ref{fig:dot_products} (left), with \textbf{TD-learning}, the dot products steadily increase during training though the average Q-value predictions are roughly constant over the this phase. These Q-values on average match those learned by \textbf{offline SARSA} and \textbf{supervised regression},
% %%SL.7.13: Supervised regression doesn't appear in the left plots!
% but the dot products for \textbf{SARSA} and supervised learning remain flat. In contrast, with \textbf{TD-learning}, the dot products increase after too many gradient steps, and this is accompanied by instability in the Q-values, which then diverge. Note that the only difference between \textbf{TD-learning} and \textbf{SARSA} is the resampling of the target value action, where \textbf{TD-learning} uses a new (out-of-sample) action from the same distribution, while \textbf{SARSA} uses the dataset action.
% %these two approaches quickly converge over the course of training unlike \textbf{TD-learning}. With more training, we find that \textbf{TD-learning} exhibits unstable Q-values, which eventually diverge, whereas the predictions for both \textbf{offline SARSA} and \textbf{supervised regression} converge and do not destabilize with more training.
% % Notably, with \textbf{TD-learning}, the dot-products steadily increase during training and the Q-values eventually diverge, whereas both the dot-products and Q-values of \textbf{offline SARSA} and \textbf{supervised regression} quickly converge and remain stable throughout training (Figure 2). 
% These observations indicate that Bellman backups alone are not responsible for the increasing dot-products and instability because \textbf{offline SARSA} uses Bellman backups, but behaves similarly to \textbf{supervised regression}.
% %%SL.7.13: Again, this is not apparent from the figure, because supervised regression is not shown on the left side!
% On the other hand, \textbf{TD-learning}, which uses out-of-sample actions in the Bellman backup, exhibits increasing dot-products, which eventually ends in  instability, suggesting that out-of-sample actions critically alter the learning dynamics. The theoretical model developed in Section~\ref{sec:theory} provides an explanation for this observation.

%To discern the differences between TD-learning and supervised learning, we first note in Figure~\ref{fig:dot_products} (right) that both the Q-value predictions and dot-product values quickly stabilize for \textbf{supervised regression} (shown in red). A similar convergent behavior in both dot products and learned Q-values is observed when utilizing in-sample actions in the case of \textbf{offline SARSA}, even though SARSA uses Bellman backups (Figure~\ref{fig:dot_products}, ``in-sample''). Next, we turn to case of \textbf{TD-learning}, which utilizes out-of-sample actions. We would expect TD-learning to converge to the same solution as offline SARSA, and we indeed observe similar Q-values in average (compare red vs blue lines in Figure~\ref{fig:dot_products} (left)), however, crucially the feature dot product values are much larger for TD-learning. In fact, the feature dot products in TD-learning continually increase with more training (Figure~\ref{fig:dot_products}) while it stabilizes for offline SARSA. This indicates that even though the Q-function outputs similar values, performing Bellman backups with out-of-sample actions gives rise to increasing dot-product values. Furthermore, as shown in Figure~\ref{fig:dot_products} (middle), once the values of dot-products are large enough (after more training steps), training further leads to an unstable trend in Q-values.


% \vspace{-5pt}

\vspace{-0.2cm}
\subsection{Theoretically Characterizing Implicit Regularization in TD-Learning}
\label{sec:theory} 
\vspace{-0.2cm}
Why does feature co-adaptation emerge in TD-learning and what do \emph{out-of-sample} actions have to do with it? To answer this question, we theoretically characterize the implicit regularization effects in TD-learning. We analyze the learning dynamics of TD learning in the overparameterized regime, where there are many different parameter vectors $\theta$ that fully minimize the training set temporal difference error. We base our analysis of TD learning on the analysis of implicit regularization in supervised learning, previously developed by \citet{blanc2020implicit,damian2021label}.

\textbf{Background.} When training an overparameterized $f_\theta(\bx)$ via supervised regression using the squared loss, denoted by $L$, many different values of $\theta$ will satisfy $L(\theta)=0$ on the training set due to overparameterization, but \citet{blanc2020implicit} show that the dynamics of stochastic gradient descent will only find fixed points $\theta^*$ that additionally satisfy a condition which can be expressed as $\nabla_\theta R(\theta^*) = 0$, along certain directions (that we will describe shortly). This function $R(\theta)$ is referred to as the implicit regularizer. The noisy gradient updates analyzed in this model have the form:  
\vspace{-0.05in}
\begin{equation}
\label{eq:gradient_update}
    \theta_{k+1} \leftarrow \theta_k - \eta \nabla_\theta L(\theta) + \eta \varepsilon_k, ~~ \varepsilon_k \sim \mathcal{N}(0, M).
\end{equation}
\vspace{-0.05in}
\citet{blanc2020implicit} and \citet{damian2021label} show that some common SGD techniques fall into this framework, for example, when the regression targets in supervised learning are corrupted with $\mathcal{N}(0, 1)$ label noise, then the resulting $M = \sum_{i=1}^{|\mathcal{D}|} \nabla_\theta f_\theta(\bx_i) \nabla_\theta f_\theta(\bx_i)^\top$ and the induced implicit regularizer $R$ is given by $R(\theta) = \sum_{i}^{|\mathcal{D}|} ||\nabla_\theta f_\theta(\bx_i)||_2^2$. Any solution $\theta^*$ found by Equation~\ref{eq:gradient_update} must satisfy $\nabla_\theta R(\theta^*) = 0$ along directions $\bv \in \mathbb{R}^{|\theta|}$ which lie in the null space of the Hessian of the loss $\nabla^2_\theta L(\theta^*)$ at $\theta^*$,  $\bv \in \text{Null}(\nabla^2_\theta L(\theta^*))$. The intuition behind the implicit regularization effect is that along such directions in the parameter space, the Hessian is unable to contract $\theta_k$ when running noisy gradient updates (Equation~\ref{eq:gradient_update}). Therefore, the only condition that the noisy gradient updates converge/stabilize at $\theta^*$ is given by the condition that $\nabla R(\theta^*) = 0$. This model corroborates findings~\citep{mulayoff2020unique, damian2021label} about the solutions from SGD, which motivates our use. 

\paragraph{Our setup.} Following this framework, we analyze the fixed points of noisy TD-learning. We consider noisy pseudo-gradient (or semi-gradient) TD updates with a general noise covariance $M$:
\vspace{-0.05in}
\begin{align}
    \theta_{k+1} = \theta_k - \eta \underbrace{\left( \sum_i \nabla_\theta Q(\bs_i, \mathbf{a}_i) \left(Q_\theta(\bs_i, \mathbf{a}_i)\!- \!(r_i\!+\!\gamma {Q}_{\theta}(\bs'_i, \mathbf{a}'_i))\right) \right)}_{:= g(\theta)} +\eta \varepsilon_k,  ~~ \varepsilon_k \sim \mathcal{N}(0, M)
\label{eq:td_update}
\end{align}
We use a deterministic policy $\mathbf{a}'_i = \pi(\bs'_i)$ to simplify exposition. Following \citet{damian2021label}, we can set the noise model $M$ as $M = \sum_i \nabla_\theta Q(\bs_i, \mathbf{a}_i) \nabla_\theta Q(\bs_i, \mathbf{a}_i)^\top$, or utilize a different choice of $M$, but we will derive the general form first.  Let $\theta^*$ denote a stationary point of the training TD error, such that the pseudo-gradient
$g(\theta^*) = 0$. Further, we denote the derivative of $g(\theta)$ w.r.t. $\theta$ as the matrix $G(\theta) \in \mathbb{R}^{|\theta| \times |\theta|}$, and refer to it as the \emph{pseudo-Hessian}: although $G(\theta)$ is not actually the second derivative of any well-defined objective, since TD updates are not proper gradient updates, as we will see it will play a similar role to the Hessian in gradient descent. For brevity, define $G = G(\theta^*)$, $g = g(\theta^*)$, $\nabla G = \nabla_\theta G(\theta^*) \in \mathbb{R}^{|\theta| \times |\theta| \times |\theta|}$, and let $\lambda_i(P)$ denote the $i$-th eigenvalue of matrix $P$, when arranged in decreasing order of its (complex) magnitude $|\lambda_i(P)|$ (note that eigenvalues can be complex for non-symmetric matrices that we encounter here). 

\paragraph{Assumptions.} To simplify analysis, we assume that matrices $G$ and $M$ (i.e., the noise covariance matrix) span the same $n$-dimensional basis in $d$-dimensional space, where $d$ is the number of parameters and $n$ is the number of datapoints, and $n \ll d$ due to overparameterization. We also require $\theta^*$ to satisfy a technical criterion that requires approximate alignment between the eigenspaces of $G$ and the gradient of the Q-function, without which noisy TD may not be stable at $\theta^*$. We summarize all the assumptions in Appendix~\ref{app:proofs}, and present the resulting regularizer below. 

\begin{tcolorbox}[colback=blue!6!white,colframe=black,boxsep=0pt,top=-3pt,bottom=2pt]
\vspace{2mm}
\begin{theorem}[Implicit regularizer at TD fixed points]
\label{thm:implicit_noise_reg}
Under the assumptions so far, a fixed point of TD-learning,  $\theta^*$, where $Q_{\theta^*}(\bs_i, \mathbf{a}_i) = r_i + \gamma Q_{\theta^*}(\bs'_i, \mathbf{a}'_i)$ for every $(\bs_i, \mathbf{a}_i, \bs'_i) \in \mathcal{D}$ is stable (atttractive) if: \textbf{(1)} it satisfies $\mathrm{Re}(\lambda_i(G)) \geq 0, \forall i$ and $\mathrm{Re}(\lambda_i(G)) > 0$ if $|\mathrm{Imag}(\lambda_i(G))| > 0$, and \textbf{(2)} along directions $\bv \in \mathbb{R}^{\text{dim}(\theta)}, \bv \in \text{Null}(G)$, $\theta^*$ is the stationary point of the implicit regularizer:
\vspace{-0.2cm}
\begin{align}
\label{eqn:regularizer}
\!\!\!\!R_\mathrm{TD}(\theta)\!\!&=\!\!\underbrace{\eta \sum_{i=1}^{|\mathcal{D}|} \nabla Q_\theta(\bs_i, \mathbf{a}_i)^\top \Sigma^{*}_M \nabla Q_\theta(\bs_i, \mathbf{a}_i)}_{\text{implicit regularizer for noisy GD}}
\!\!\!-\!\!\!\underbrace{\eta \gamma \sum_{i=1}^{|\mathcal{D}|} \mathrm{tr}\left(\left[\left[\nabla Q_\theta(\bs'_i, \mathbf{a}'_i)^\top\right]\right]^\top \Sigma_M^* \nabla Q_\theta(\bs_i, \mathbf{a}_i)  \right)}_{\text{additional term in TD learning}}
%%SL.10.27: would it be better to write the trace term as gradQ^T sigma gradQ (since tr(sigma gradQ gradQ^T) = tr(gradQ^T sigma gradQ))? that might make the similarity (but subtle difference) between the two terms more apparent...
    % R_\mathrm{TD}(\theta) = \sum_{i=1}^{|\mathcal{D}|} \mathrm{trace}\left[ \Sigma^{* \top}_M \nabla_\theta Q_\theta(\bs_i, \mathbf{a}_i) \left( \nabla_\theta Q_\theta(\bs_i, \mathbf{a}_i) - \gamma \texttt{Stop}(\nabla_\theta Q_\theta(\bs'_i, \mathbf{a}'_i)) \right)^\top \right].  
    %%SL.9.29: I think it might be clearer if you write this as a difference of two terms, because then the first term will look like the supervised R(theta), but the second one will look clearly weird. The first term then becomes gradQ^T Sigma^T gradQ (and you don't need trace), which is very intuitive (also, Sigma is symmetric, right? so is the transpose symbol there necessary?). Additionally, consider using a less jarring symbol for Stop, so that the equation is typeset so that it looks more like -grad Q [gradQ^T] or something and the stop part is unobtrusive -- that would make it easier for readers to get intuition for what this equation means. Currently it's extremely hard to understand from looking at it.
\end{align}
% \vspace{-0.1in}
where $(\bs_i, \mathbf{a}_i)$ and $(\bs'_i, \mathbf{a}'_i)$ denote state-action pairs that appear together in a Bellman update, $[[\square]]$ denotes the stop-gradient function, which does not pass partial gradients w.r.t. $\theta$ into $\square$. $\Sigma^*_M$ is the fixed point of the  discrete Lyapunov equation: $$\Sigma^*_M := (I - \eta G) \Sigma^*_M (I - \eta G)^\top + \eta^2 M.$$
\end{theorem}
\vspace{1mm}
\end{tcolorbox}
A proof of Theorem~\ref{thm:implicit_noise_reg} is provided in Appendix~\ref{app:proofs}. Next, we explain the intuition behind this result and provide a proof sketch. To derive the induced implicit regularizer for a stable fixed point $\theta^*$ of TD error, we study the learning dynamics of noisy TD learning (Equation~\ref{eq:td_update}) initialized at $\theta^*$, and derive conditions under which this noisy update would stay close to $\theta^*$ with multiple updates.  This gives rise to the two conditions shown in Theorem~\ref{thm:implicit_noise_reg} which can be understood as controlling stability in mutually exclusive directions in the parameter space. If condition \textbf{(1)} is not satisfied, then even under-parameterized TD will diverge away from $\theta^*$, since $I - \eta G$ would be a non-contraction as the spectral radius, $\rho(I - \eta G) \geq 1$ in that case. Thus, $\theta_k - \theta^*$ will grow or not decrease in some direction. When \textbf{(1)} is satisfied for all directions in the parameter space, there are still directions where both the real and imaginary parts of the eigenvalue $\lambda_i(G)$ are $0$ due to overparameterization\footnote{To see why this is the case, note that $\text{rank}(G) \leq |\mathcal{D}| \ll \text{dim}(\theta)$, and so some eigenvalues of $G$ are $0$.}. 
In such directions, learning is governed by the projection of the noise under the tensor  $\nabla G$,
%%SL.12.5: Are we making that mistake here again where we call the TD pseudo-gradient a derivative? I would recommend being *very* careful about the term derivative. Basically, don't call something a derivative if it's not really a derivative. You can introduce some term for it like pseudo-gradient, but it's very important to clearly distinguish between things that are actual derivatives or gradients, and the TD update. Try to avoid mixing the terminology, but it's OK to define some shorthand term like pseudo-gradient
%%AK: changed it here
which appears in the Taylor expansion of $\theta_k - \theta^*$ around the point $\theta^*$:
\begin{align}
    \label{eqn:nu_k}
    &\theta_{k+1} = \theta_k - \eta \left(g + G (\theta_k - \theta^*) + \frac{1}{2} \nabla G [\theta_k -\theta^*, \theta_k - \theta^*] \right) + \varepsilon_k, ~~ \varepsilon_k \sim \mathcal{N}(0, M)\\
    \implies &\nu_{k+1} = (I - \eta G )\nu_k  - \frac{\eta}{2} \nabla G [\nu_k, \nu_k] + \varepsilon_k,
    \label{eqn:nu_k_actual}
\end{align}
where we reparameterize in terms of $\nu_k := \theta_k - \theta^*$. The proof shows that $\theta^*$ is stable if it is a stationary point of the implicit regularizer $R_\mathrm{TD}$ (condition \textbf{(2)}), which ensures that total noise (i.e., accumulated $\varepsilon_k$ over iterations $k$) accumulated by $\nabla G$ does not lead to a large deviation in $\nu_k$ in directions where $I - \eta G$ does not contract. 
% The total noise accumulated with such noisy backups (Equation~\ref{eqn:nu_k}) admits the covariance matrix $\Sigma^*_M$, and the derivative of the regularizer $\nabla_\theta R_\mathrm{TD}(\theta)$ at an attractive $\theta^*$ must be zero to prevent any deviation in $\nu_k$ in directions where learning is controlled by \textbf{(2)}.   

% To derive the induced implicit regularizer for a stable fixed point $\theta^*$ of TD error, we study the learning dynamics of noisy TD learning (Equation~\ref{eq:td_update}) initialized at $\theta^*$, and derive conditions under which this noisy update would stay close to $\theta^*$ with multiple updates. In this case, we can utilize Taylor expansion around $\theta^*$ to track the evolution of the difference between $\theta_k$ and $\theta^*$, denoted as $\nu_k = \theta_k - \theta^*$ (roughly; modulo some terms we will discuss in Appendix~\ref{app:proofs}), as shown in Equation~\ref{eqn:nu_k}:
% \begin{align}
%     \label{eqn:nu_k}
%     &\theta_{k+1} = \theta_k - \eta \left(g + G (\theta_k - \theta^*) + \frac{1}{2} \nabla G [\theta_k -\theta^*, \theta_k - \theta^*] \right) + \varepsilon_k, ~~ \varepsilon_k \sim \mathcal{N}(0, M)\\
%     \implies &\nu_{k+1} = (I - \eta G )\nu_k  - \frac{\eta}{2} \nabla G [\nu_k, \nu_k] + \varepsilon_k.
%     \label{eqn:nu_k_actual}
% \end{align}
% Our theoretical result derives the implicit regularizer by characterizing conditions on $\theta^*$ that are necessary for the stability of the learning dynamics in Equation~\ref{eqn:nu_k_actual} in the neighborhood around $\theta^*$. 

\paragraph{Interpretation of Theorem~\ref{thm:implicit_noise_reg}.} While the choice of the noise model $M$ will change the form of the implicit regularizer, in practice, the form of $M$ is not known as this corresponds to the noise induced via SGD. We can consider choices of $M$ for interpretation, but Theorem~\ref{thm:implicit_noise_reg} is easy to qualitatively interpret for $M$ such that $\Sigma^*_M = I$. In this case, we find that the implicit preference towards local minima of $R_\mathrm{TD}(\theta)$ can explain feature co-adaptation. In this case, the regularizer is simpler:
\begin{align*}
    R_\mathrm{TD}(\theta) := \sum_i ||\nabla Q_\theta(\bs_i, \mathbf{a}_i) ||_2^2 - \gamma \nabla Q_\theta(\bs_i, \mathbf{a}_i) \nabla [[Q_\theta(\bs'_i, \mathbf{a}'_i)]].
\end{align*}
The first term is equal to the squared per-datapoint gradient norm, which is same as the implicit regularizer in supervised learning obtained by \citet{blanc2020implicit,damian2021label} with label noise. However, $R_\mathrm{TD}(\theta)$ additionally includes a second term that is equal to the dot product of the gradient of the Q-function at the current and next states, $\nabla_\theta Q_\theta(\bs_i, \mathbf{a}_i)^\top \nabla_\theta Q_\theta(\bs'_i, \mathbf{a}'_i)$, and thus this term is effectively \emph{maximized}. When restricted to the last-layer parameters of a neural network,
this term is equal to the dot product of the features at consecutive state-action tuples: $\sum_i \nabla_\theta Q_\theta(\bs_i, \mathbf{a}_i)^\top \nabla_\theta Q_\theta (\bs'_i, \mathbf{a}'_i) = \sum_i \phi(\bs_i, \mathbf{a}_i)^\top \phi(\bs'_i, \mathbf{a}'_i)$. The tendency to maximize this quantity to attain a local minimizer of the implicit regularizer corroborates the empirical findings of increased dot product in Section~\ref{app:problem_more}. 

\paragraph{Explaining the difference between utilizing seen and unseen actions in the backup.} If all state-action pairs $(\bs'_i, \mathbf{a}'_i)$ appearing on the right-hand-side of the Bellman update also appear in the dataset $\mathcal{D}$, as in the case of offline SARSA (Figure~\ref{fig:dot_products}), the preference to increase dot products will be balanced by the affinity to reduce gradient norm (first term of $R_\mathrm{TD}(\theta)$ when $\Sigma^*_M = I$): for example, for offline SARSA, when $(\bs'_i, \mathbf{a}'_i)$ are permutations of $(\bs_i, \mathbf{a}_i)$, $R_\mathrm{TD}$ is lower bounded by $(1 - \gamma) \sum_i ||\nabla_\theta Q_\theta(\bx_i)||_2^2$ and hence minimizing $R_\mathrm{TD}(\theta)$ would minimize the feature norm instead of maximizing dot products. This also corresponds to the implicit regularizer we would obtain when training Q-functions via supervised learning and hence, our analysis predicts that offline SARSA with in-sample actions (i.e., when $(\bs', \mathbf{a}') \in \mathcal{D}$) would behave similarly to supervised regression. 

However, the regularizer behaves very differently when unseen state-action pairs $(\bs'_i, \mathbf{a}'_i)$ appear only on the right-hand-side of the backup. This happens with any algorithm where $\mathbf{a}'$ is not the dataset action, which is the case for all deep RL algorithms that compute target values by selecting $\mathbf{a}'$ according to the current policy. In this case, we expect the dot product of gradients at $(\bs, \mathbf{a})$ and $(\bs', \mathbf{a}')$ to be large at any attractive fixed point, since this minimizes $R_\mathrm{TD}(\theta)$. This is precisely a form of co-adaptation: \textit{gradients at out-of-sample state-action tuples are highly similar to gradients at observed state-action pairs measured by the dot product}. This observation is also supported by the analysis in Section~\ref{app:problem_more}. Finally, note that the choice of $M$ is a modelling assumption, and to derive our explicit regularizer, later in the paper, we will make a simplifying choice of $M$. However, we also empirically verify that a different choice of $M$, given by label noise, works well.
%%%%%%%%%%%%%%%%%%%%%%%%%%%%%%%%%%%%%%%%%%%%%%%%%%%%%%%%%%


\begin{figure}[t]
    \centering
    % \vspace{-18pt}
    \includegraphics[width=0.99\linewidth]{chapters/dr3/figures_iclr/return_degrades.pdf}
    % \includegraphics[width=0.49\linewidth]{figures/figure3_neurips_stability_cql_dot_product_both_return.pdf}
    \vspace{-0.3cm}
    \caption{\small{Even when current offline RL algorithms are initialized at a high-performing checkpoint that attains small feature dot products, feature dot products increase with further training and the performance degrades.}}  
    %%SL.7.13: I don't really understand the implication of the last part ("note that the values of the TD error and the overall training loss for either algorithm are generally low and decrease in many cases")
    \label{fig:stability}
    \vspace{-0.3cm}
\end{figure}



\paragraph{Why is implicit regularization detrimental to policy performance?}
To answer this question, we present theoretical and empirical evidence that illustrates the adverse effects of this implicit regularizer. Empirically, we ran two algorithms, DQN and CQL, initialized from a high-performing Q-function checkpoint,
%%SL.10.27: This part is going to really throw off some readers. Of course it's not a stable point for TD, because you didn't learn it with TD! Why do we expect the solution found by one method (in this case DR3) to be stable for another method? That doesn't illustrate that TD is bad, just that DR3 changes the fixed point (which it should). But perhaps if you don't want to go into detail about this issue, you could somehow sweep the "obtained using DR3" bit under the rug to avoid distracting the reader?
%%AK: agreed, removed
which attains relatively small feature dot products (i.e., the second term of $R_\mathrm{TD}(\theta)$ is small). Our goal is to see if TD updates starting from such a ``good'' initialization still stay around it or diverge to poorer solutions. Our theoretical analysis in Section~\ref{sec:theory} would predict that TD learning would destabilize from such a solution, since it would not be a stable fixed point. Indeed, as shown in Figure~\ref{fig:stability}, the policy immediately degrades, and the the dot-product similarities start to increase. This even happens with CQL, which explicitly corrects for distributional shift confounds, implying that the performance drop cannot be directly explained by the typical out-of-distribution action explanations. To investigate the reasons behind this drop, we also measured the training loss function values for these algorithms (i.e., TD error for DQN and TD error + CQL regularizer for CQL) and find in Figure~\ref{fig:stability} that the loss values are generally small for both CQL and DQN. This indicates that the preference to increase dot products is not explained by an inability to minimize TD error. 
%%SL.12.5: I don't understand what "implicit phenomenon" means or how the loss being low indicates this
In Appendix~\ref{app:cql_stability}, we show that this drop in performance when starting from good solutions can be effectively mitigated with our proposed \methodname\ explicit regularizer for both DQN and CQL. Thus we find that not only standard TD learning degrades from a good solution in favor of increasing feature dot products, but keeping small dot products enables these algorithms to remain stable near the good solution.
%%SL.12.5: I slightly tweaked the above sentence, but it can be badly misunderstood as saying that basically our entire evaluation of DR3 has been offloaded into that appendix.

To motivate why co-adapted features can lead to poor performance in TD-learning, we study the convergence of linear TD-learning on co-adapted features. Our theoretical result characterizes a lower bound on the feature dot products in terms of the feature norms for state-action pairs in the dataset $\mathcal{D}$, which if satisfied, will inhibit  convergence: 
\begin{tcolorbox}[colback=blue!6!white,colframe=black,boxsep=0pt,top=-3pt,bottom=2pt]
\vspace{2mm}
\begin{proposition}[TD-learning on co-adapted features]
\label{thm:co_adapted_features_are_bad}
Assume that the features $\Phi = [\phi(\bs, \mathbf{a})]_{\bs, \mathbf{a}}$ are used for linear TD-learning. Then, if 
$$\sum_{\bs, \mathbf{a}, \bs' \in \mathcal{D}} \phi(\bs, \mathbf{a})^\top \phi(\bs', \mathbf{a}') \geq \frac{1}{\gamma} \sum_{\bs, \mathbf{a} \in \mathcal{D}} \phi(\bs, \mathbf{a})^\top \phi(\bs, \mathbf{a}),$$ 
linear TD-learning using features $\Phi$ will not converge. 
\end{proposition}
\end{tcolorbox}
A proof of Proposition~\ref{thm:co_adapted_features_are_bad} is provided in Appendix~\ref{app:new_thm} and it relies on a stability analysis of linear TD. While features change during training for TD-learning with neural networks, and arguably linear TD is a simple model to study consequences of co-adapted features, even in this simple linear setting, Proposition~\ref{thm:co_adapted_features_are_bad} indicates that TD-learning may be non-convergent as a result of co-adaptation.

% \textbf{Comparison to implicit regularization in TD learning with linear function approximation.} Running stochastic gradient descent in overparameterized linear regression finds solutions with the smallest $\ell_2$ norm, which is often regarded as the implicit regularizer. Based on this observation, one might wonder how our derived implicit regularizer relates to minimum norm solutions attained by gradient descent in overparameterized linear TD learning. The implicit regularizer we obtain in  Equation~\ref{eqn:regularizer} would be a constant, independent of the parameter vector $\theta$ for linear TD learning. Thus our regularization specifically captures the effect of SGD on non-linear function approximators, which are absent when studying linear function approximation. 

% \textbf{Takeaways.} We summarize the key takeaways from our theoretical analysis now. \textbf{(1)} \emph{The implicit regularizer at TD fixed points} is shown in Equation~\ref{eqn:regularizer}. The first term corresponds to the regularizer for SGD in supervised learning, while the second term that is unique to TD and leads to an (undesirable) increase in gradient or feature dot products; \textbf{(2)}\emph{Out-of-sample actions exacerbate the implicit regularization effect,} since feature dot products can be easily increased when out-of-sample actions, which do not appear in the dataset, are used to compute Bellman targets; \textbf{(3)} The implicit regularizer in Equation~\ref{eqn:regularizer} is induced via a mechanism unique to non-linear Q-functions, different from overparameterized, linear TD-learning.
% \vspace{-0.15cm}

% \textbf{Theoretically,} we characterize the adverse effects of this implicit regularization by examining the trend in the worst-case error incurred by value-function learning on top of the learned features as feature dot-products increase.


\iffalse
We will first derive the implicit regularization effect induced due to noise in stochastic gradient descent. Following prior work~\citep{blanc2020implicit} in supervised learning, we will model this noise as additive Gaussian noise added to the regression targets. \citet{blanc2020implicit} has identified that in supervised learning, SGD with label noise
%%SL.5.26: is the "label noise" part even significant? almost any supervised learning problem assumes label noise
diverges from a solution $\theta^*$ corresponding to a loss function $L(\theta)$, if and only if it is not a first-order stationary point of:
%%SL.5.26: I found the above sentence pretty hard to parse, can we state this in a simpler way?
\begin{equation*}
    \min_\theta~~ R(\theta) ~~~ \text{s.t.}~~~ L(\theta) = 0,
\end{equation*}
where $R(\theta)$ is an implicit regularizer given by the mean squared L2-norm of the gradient of the learned function with respect to its parameter $\theta$.
%%SL.5.26: add an equation for this
Our goal in this section is to derive the corresponding regularizer for TD-learning, $R_\mathrm{TD}(\theta)$, under similar assumptions, and then analyze its effect on the solution found via TD-learning.  

We first set up the notation. Let $Q_\theta(\bs, \mathbf{a})$ denote the Q-network; for brevity, we will use $\bx$ as shorthand for $(\bs, \mathbf{a})$, such that $\bx := (\bs, \mathbf{a})$ is the input to $Q_\theta$. Given a dataset $\data = \{(\bs_i, \mathbf{a}_i, r_i, \bs'_i)\}_{[n]}$ and following prior work,
%%SL.5.26: add citation
we will assume that the Q-function minimizes  mean-squared TD-error with added label-noise, $\hat{L}(\theta)$, such that \textcolor{red}{George, can we move the label noise to gradient noise?}
\begin{equation*}
%%SL.5.26: reverse the order here, have \hat{L} come first, then \ell
    \ell_\theta(i) = \frac{1}{2} \left(Q_\theta(\bx_i) - \mathrm{StopGrad}\left[r_i + \gamma Q_\theta(\bx'_i) \right] - \epsilon_i \right)^2; ~~~\epsilon_i \sim \mathcal{N}(0, \sigma^2); ~~~ \hat{L}(\theta) := \frac{1}{n}\sum_{i=1}^n \ell_\theta(i),
\end{equation*}
where $\mathrm{StopGrad}$ denotes the stop-gradient operation typically used in TD-learning, $\epsilon_i$ is random Gaussian scalar noise, and $(\bx_i, \bx'_i)$ represent state-action pairs appearing together in a Bellman backup.
%%SL.5.26: make it more explicit what \bx'_i is
TD-learning would then minimize $\hat{L}(\theta)$ via gradient descent. We will assume that $Q_\theta(\bx_i)$,  $\nabla_\theta Q_\theta(\bx_i)$ and $\nabla^2_\theta Q_\theta(\bx_i)$ are all Lipschitz with some coefficients. 


%%SL.5.26: Can we abstract away some of this derivation into a theorem statement and move the derivation to an appendix? This would help to get the paper under the length limit.
To identify the implicit regularizer, we build on the approach of \citet{blanc2020implicit} for supervised learning. We assume that we initialize the parameter $\theta$ to $\theta_0 = \theta^*$, which is an optimal Q-function, satisfying \emph{all} the Bellman consistency equations on all the states in the MDP, not just the states in the dataset.
%%SL.5.26: Do we need to assume that we *initialize* there? Can we just say that we are analyzing the optimum or something? Is this really the assumption that prior work makes?
Now, we will run gradient descent on the loss $\hat{L}(\theta)$ with a sufficiently small learning rate $\eta$, starting from $\theta_0$. This results in the iterates shown below in Equation~\ref{eqn:gradient_descent}. We will then bound the divergence between the $k$-th gradient iterate $\theta_k$ and $\theta^*$, and derive the condition on $\theta^*$ that allows this divergence $||\theta_k - \theta^*||_2$ to be small. This condition will specify the implicit regularizer $R_\mathrm{TD}(\theta^*)$ at a stable optimum $\theta^*$. The gradient descent equation in Equation~\ref{eqn:gradient_descent} can be simplified as shown below in Equations~\ref{eqn:gradient_descent_simplified} and \ref{eqn:grad_descent2}:
\begin{align}
\label{eqn:gradient_descent}
    \theta_{k+1} &:= \theta_k - \eta \nabla_\theta \hat{L}(\theta_k).\\
    \label{eqn:gradient_descent_simplified}
     \theta_{k+1} &= \theta_k - \eta \sum_i \nabla_\theta Q_\theta(\bx_i) \left(Q_\theta(\bx_i) - (r_i + \gamma Q_\theta(\bx'_i)) - \epsilon_i \right)\\
     \label{eqn:grad_descent2}
    \implies \theta_{k+1} &= \theta_k - \eta \underbrace{\sum_{i} \nabla_\theta Q_\theta(\bx_i) \left[Q_\theta(\bx_i) - (r_i + \gamma Q_\theta(\bx'_i)) \right]}_{\text{(a)} := \nabla_\theta L(\theta_k)} + \eta  \underbrace{\sum_i \nabla_\theta Q_\theta(\bx_i) \epsilon_i.}_{\text{(b)} \sim \mathcal{N}(0, \sigma^2 \nabla_\theta Q_\theta \nabla_\theta Q_\theta^\top).}
\end{align}
Note in Equation~\ref{eqn:grad_descent2} that, due to the additive nature of label noise, the parameters $\theta_k$ evolve based on the gradients of the original TD-error loss function \emph{without} noise ($L(\theta)$, term (a)), along with a data-dependent noise (term (b)) sampled from $\mathcal{N}(0, \sigma^2 \nabla Q_\theta \nabla Q_\theta^\top)$. Next, since $\theta_k$ is initialized in the local neighborhood of $\theta^*$, we can rewrite Equation~\ref{eqn:grad_descent2} in terms of $\nabla^2 L := \nabla^2_\theta L(\theta)|_{\theta^*}$, $\nabla^3 L := \nabla^3_\theta L(\theta)|_{\theta^*}$ and $M := \nabla_\theta Q_\theta \nabla_\theta Q_\theta^\top|_{\theta^*}$ by using Taylor expansion around $\theta^*$. Also note that since $\theta^*$ is a global optimum, $\nabla_\theta L(\theta^*) = 0$. Denoting $\nu_k = \theta_k - \theta^*$, we obtain ($\varepsilon_k \sim \mathcal{N}(0, \sigma^2 \nabla Q_\theta \nabla Q_\theta^\top)$):
%%SL.5.26: why are there two sets of parens on that last equation?
\begin{align}
    \label{eqn:\nu_k}
    \theta_{k+1} ~&= \theta_k - \eta \left( \nabla L + \nabla^2 L (\theta_k - \theta^*) + \frac{1}{2} \nabla^3 L (\theta_k - \theta^*, \theta_k - \theta^*) \right) + \varepsilon_k, ~~~~~~\\
    \label{eqn:recursive_v}
    \implies \nu_{k+1} ~&= (I - \eta \nabla^2 L )\nu_k  - \frac{\eta}{2} \nabla^3 L (\nu_k, \nu_k) + \varepsilon_k, ~~~~ \varepsilon_k \sim \mathcal{N}(0, \eta \sigma^2 M).
\end{align}
From Equation~\ref{eqn:recursive_v} we can make a few observations. First, the distance between $\theta_k$ and $\theta^*$, $||\nu_k||_2$ decreases at a rate proportional to $(I - \eta \nabla^2 L)$. $\nabla^2 L$ only spans certain directions due to the overparameterized nature of the landscape. Thus $\nu_k$ will not contract on along every direction, and the noise $\varepsilon_k$ in each iteration can compound through powers of $(I - \eta \nabla^2 L)$ leading to an increased in the value of $\nu_{k}$,
%%SL.5.26: presumably it makes it increase under some conditions on that matrix?
thus making $\theta_k$ diverge from $\theta^*$. However, if the starting $\theta^*$ is such that the third derivative $\nabla^3 L (\nu_k, \nu_k)$ can compensate for any potential increase in the value of $\nu_k$, then $\nu_{k} \rightarrow 0$ as $k \rightarrow \infty$. Theorem~\ref{thm:implicit_noise_reg} formalizes this to obtain an expression for the resulting implicit regularizer. A proof for Theorem~\ref{thm:implicit_noise_reg} can be found in Appendix ??.
\begin{theorem}[Informal, Implicit regularization at optimal TD-solutions]
\label{thm:implicit_noise_reg}
Assuming notations
%%SL.5.26: "Assuming notations" seems like a weird phrase
and conditions of label-noise gradient descent on TD error discussed so far, a global minimizer of TD error $\theta^*$ on the dataset $\mathcal{D}$ is a stable optimum if and only if it minimizes implicit regularizer $R_\mathrm{TD}(\theta)$ given by:
\begin{equation}
\label{eqn:regularizer}
    R_\mathrm{TD}(\theta) = \mathrm{tr} \left(\sum_{i=1}^n \nabla_\theta Q_\theta(\bx_i) \left( \nabla_\theta Q_\theta(\bx_i) - \gamma \nabla_\theta Q_\theta(\bx'_i) \right)^\top \right).
\end{equation}
\end{theorem}
\textbf{Interpretation of Theorem~\ref{thm:implicit_noise_reg}.} Theorem~\ref{thm:implicit_noise_reg} indicates that out of all possible global minimizers $\theta^*$ of the TD error
%%SL.5.26: this is a fairly basic point, but I think when we introduce the notion of implicit regularization earlier, it might help to expand on what this "out of all possible minimizers" thing means. E.g., something like this: When training overparameterized functions, such as deep networks, multiple different parameter vectors $\theta$ will minimize $L(\theta)$. But not all of these minimizers will generalize equally well. \citet{someone} proposes that the minimizer found via SGD will satisfy the following constrained optimization problem: [stuff], where $R(\theta)$ is an \emph{implicit} regularizer that arises from the structure of SGD with overparameterized models. Intuitively, $R(\theta)$ causes SGD to prefer simpler (and therefore more generalizable) solutions, even when we might otherwise expect overparameterized models to be liable to overfit. This model has been put forward as one explanation for the effective generalization of overparameterized deep networks~\citep{stuff}. [this could go at the top of Sec 3.2.1 for example]
on the training dataset, gradient descent with label noise will only stabilize at a minimizer of $R_\mathrm{TD}(\theta)$. $R_\mathrm{TD}$ consists of two types of terms: positive terms equal to gradient norms, $\sum_i ||\nabla_\theta Q_\theta(\bx_i)||^2_2$, and additional negative terms equal to the expected dot-product of gradients at consecutive states, $\sum_\theta Q_\theta(\bx_i)^\top \nabla_\theta Q_\theta(\bx'_i)$.
%%SL.5.26: Is it completely obvious where these come from? I realize this is just algebra, but we could spoon-feed it to the reader more by writing out the distributed equation so that these terms actually show up.
If all state-action pairs $\bx'_i$ appearing on the right-hand-side of the Bellman update also appear in the dataset $\mathcal{D}$, as in the case of SARSA (Figure~\ref{fig:dot_products}), they would also contribute to the gradient norm term, and the value of $R_\mathrm{TD}$ will be bounded below.  We can show that in this case the optimal $\theta^*$ would effectively minimize $(1 - \gamma) \sum_i ||\nabla_\theta Q_\theta(\bx_i)||_2^2$.
%%SL.5.26: can we make these statements a bit more formal? e.g., have a corollary or something for the special case where all x' are in the dataset, and another one for the case where that is not true?
This corresponds to a regularizer one would obtain when training the Q-network via pure supervised learning~\citep{blanc2020implicit}, indicating that SARSA is regularized in similar ways as supervised regression. 
%%AK: is there consensus on this aspect?

%%AK: I am sure the para below doesnt have a great flow. If someone has any suggestions to make it more dramatic, it will be great -- this is the key RL explanation part of this math.
On the other hand, if a state-action pair $\bx'_i$ only appears on the right-hand-side of the backup, \ie, $(\bs', \mathbf{a}')$ correspond to out-of-sample state-action pairs,
%%SL.5.26: remind the reader why this would be the case, like this: However, the regularizer behaves very differently when some state-action pairs $\bx'_i$ only appear on the right-hand-side of the backup. This happens with any algorithm where $\mathbf{a}'$ is not the dataset action, which is the case for all actor-critic and Q-learning algorithms that compute target values by selecting $\mathbf{a}'$ according to the current policy. Note that $(\bs',\mathbf{a}')$ does not need to be \emph{out-of-distribution}, merely \emph{out-of-sample}, and hence this would even be the case when evaluating $\pi_\beta(\mathbf{a}'|\bs')$ with samples from $\pi_\beta(\mathbf{a}'|\bs')$ for the target value actions.
then the implicit regularizer, $R_\mathrm{TD}(\theta)$ is minimized only at solutions $\theta^*$ for which gradients at $\bx_i$ and the corresponding $\bx'_i$ are very similar in terms of dot-product.
%%SL.5.26: some pretty informal statements here, can we summarize this more precisely in a formal corollary?
This is the co-adaptation phenomenon observed in Section~\ref{sec:dr3_analysis}: gradients at out-of-sample state-action pairs $\bx'_i$ are extremely similar to the gradients at $\bx_i$ at the resulting optimum.
%%SL.5.26: gradients are similar? or features are similar?
Section~\ref{sec:dr3_analysis} instead demonstrated the co-adaptation phenomenon for penultimate layer features, which are also the gradients with respect to the parameters of the last layer, since these values are cheap to compute.
%%SL.5.26: put the above in a footnote, phrase like this: One discrepancy is that Section~\ref{sec:analysis} analyzes dot products between the last-layer features, whereas this derivation focuses on dot products between gradients. Note, however, that the last-layer features are the gradients of the weights in the last layer. [maybe allude to some NTK stuff?]. Alternatively, given that you use "gradients" and "features" interchangeably below, you could also put some discussion like this in the main text, but earlier: Note that this regularizer concerns the \emph{gradients} of the model. However, regularization of gradients and regularization of features are closely related~\citep{something} -- for example, the gradients of the last-layer weights are equal to the penultimate layer features for networks with linear readout layers.
Thus, we have shown that stochastic gradient descent on the TD-error tends to stabilize only at solutions that exhibit highly co-adapted features between out-of-sample points used for the backup and the points in the dataset. \textcolor{red}{also add when it will diverge; tr(...) cant be < 0, else we wont contract.}
\fi


\iffalse
\subsubsection{Implicit Regularization of Neural Network Architectures}
%%SL.5.26: Change the title -- the point is not that it's architecture specific, but that it is focused on overparameterization and min norm.

In the previous section, we showed, that agnostic of the neural network architecture, implicit regularization arising out of the stochasticity in SGD produces Q-functions with highly co-adapted features. In this section, we analyze a different kind of implicit regularization originating from the inductive bias and invariance of the neural network architecture.
%%SL.5.26: This seems like a really strange way to "sell" this analysis -- it's basically saying that before we analyzed the general case, now we'll make some more (unrealistic) assumptions and show that the results still hold under these additional (unrealistic) assumptions. That's a very strange statement. The fact that this analysis is architecture-specific is not really a plus. Can we come up with a better way to motivate why we have this analysis? A reasonable way to go could be something like: In this section, we will show that a similarly deleterious implicit regularization effect in TD-learning can be derived from a very different model of implicit regularization with overparameterized deep networks, based on minimum norm solutions. [and then at the end of this subsection, you can conclude with something like: We've shown that two very different models of implicit regularization in deep nets proposed in prior work~\citep{} both lead to the same conclusion in the TD-learning setting: the very same implicit regularization effects that promote effective generalization in supervised deep learning lead to learning of features that can fail to distinguish between successive state-action tuples in the TD-learning setting.]
For this analysis, we study a 2-layer wide ReLU network using ideas from prior works~\citep{blanc2020implicit,savarese2019infinite,wei2019regularization}.
%%SL.5.26: can we rephrase the above sentence and talk about how we adopt a similar model of implicit regularization as prior work?
For simplicity, we assume that the input space of the Q-function is 1-dimensional, which can be attained by mapping state-action pairs $(\bs_i, \mathbf{a}_i)$ to a one-dimensional representation $\bx_i \in \mathbb{R}$, though our argument can potentially be generalized to higher-dimensional inputs analogous to \citet{}.
%%SL.5.26: Whoa, that seems like a crazy assumption. Is that really how prior work does it?? I mean, I can see how this can be made to work, but it's an enormous leap.
%%AK: cite the savarese paper 2 from 2019 that uses radon transform

To begin, we define a canonical 2-layer ReLU network with 1D inputs: $Q_\theta(\bx_i) = \sum_i \bw_i \sigma(\mathbf{a}_i \bx_i + \bb_i) + \bd_i$, where $\sigma(\cdot) = \max(\cdot, 0)$ is the ReLU function. It is well known~\citep{wei2019regularization,savarese2019infinite} that minimizing $L(\theta) = \sum_i (Q_\theta(\bx_i) - y_i)^2$ for input-output pairs $(\bx_i, y_i)$ on a 2-layer ReLU network of sufficient width produces a solution that satisfies the following optimization (left):
%%AK: The para above is not talking about TD-learning but optimizations already use TD, so I need to fix that.
%%SL.5.26: yeah, this is an issue -- above says y, below says r + gamma*Q
\begin{equation}
\begin{aligned}
    &\text{\textbf{Neural Network parameterization}} \\
    \min_{\bw, \mathbf{a}, \bb, \bd}~~& ||\bw||_1 \\
    \text{s.t.}~&~ \forall~ \bx_i \in \mathcal{D},~~ Q_\theta(\bx_i) = r_i + \gamma \mathbf{a}r{Q}_\theta(\bx'_i) 
    \label{eqn:relu_nets}
\end{aligned}
\;~~ \vline\;
\begin{aligned}
    &\text{\textbf{Function space parameterization}} \\
    ~~~\min_{Q}~~& \int_{-\infty}^{\infty} |Q''(\bx)| d \bx \\
    \text{s.t.}~~&~ \forall \bx_i \in \mathcal{D}, ~~ Q(\bx_i) = r_i + \gamma \mathbf{a}r{Q}(\bx'_i).
\end{aligned}
\end{equation}
The optimization on the left can be converted to a function space parameterization (Equation~\ref{eqn:relu_nets}, right) that minimizes the second-derivative of the function $Q''(\bx)$ w.r.t. the input $\bx \in \mathbb{R}$ while fitting the data. This amounts to fitting the smoothest possible function that attains zero Bellman error.
%%SL.5.26: Let's just introduced the supervised model *first*, and only then talk about Bellman error, otherwise we are throwing more at the reader than they can reasonably handle.
In supervised learning this gives rise to a piecewise linear function with kinks at the points from the dataset, $\bx_i \in \mathcal{D}$ (Theorem 3.1 in \citet{savarese2019infinite}). In our case, we are interested in answering the following question: Among all solutions with zero Bellman error on the training data, do co-adapted solutions attain maximum smoothness? 
%%SL.5.26: That doesn't seem like the right way to pose the question, because "maximum smoothness" is a vague and imprecise term. Let's rather refer to the formal statement of the objective.

To answer this question, we first set up some notation. Let $(\bx_i, \bx'_i)$ be the representations of state-action tuples that appear together in a Bellman update.
%%SL.5.26: already defined this
Without loss of generality, let $\{\bx_i\}_{[n]} \in \mathcal{D}$ be ordered as $\bx_1 < \bx_2 < \cdots < \bx_N$. We further make a locality assumption on the consecutive tuples:
\begin{assumption}[State-action pairs on two sides of a Bellman backup are close to each other]
\label{assumption:state_next_state_are_close}
If $(\bx_i, \bx'_i)$ appear on the two sides of the Bellman equation, and if $\bx_{i-1}$ and $\bx_{i+1}$ be the left and right neighbors of $\bx_i$ observed in $\mathcal{D}$, then, we assume $|\bx_i - \bx'_i| \leq \min \left(|\bx_i - \bx_{i-1}|, |\bx_{i+1} - \bx_i| \right)$. 
\end{assumption}
Assumption~\ref{assumption:state_next_state_are_close} encodes the notion that the next state,
%%SL.5.26: Don't overload terminology, you're using "state" to mean two different things. This will be perceived as (intentionally) deceptive, because the consecutive actions are *not* similar.
$\bx'_i$, is closer to the current state $\bx_i$ compared to the neighbors of $\bx_i$ found in the dataset. This is reasonable to assume in practice, where states and next states generally do not differ from each other by a huge amount, often changing by a few pixels.
%%SL.5.26: I disagree with the above statement -- why the heck would \bx'_i be closer to \bx_i than \bx_{i+1}? Presumably the next action of pi_beta is at least as close to the current action of pi_beta than the next action of pi. Maybe you mis-stated something above and intended to write something else? But as written this just doesn't seem plausible.
%%AK: is there a better, simpler assumption? And can I cite something to say that the states- next states are closeby?
Under Assumption~\ref{assumption:state_next_state_are_close}, we show the following result:
%%AK: refine theorem statement and verify edge cases once (a lot of the edge cases are measure zero, so maybe just almost surely will eliminate time?)
\begin{theorem}[Informal, ReLU networks attain additional kinks
%%SL.5.26: let's avoid informal statements in theorems ("kinks")
and learn incorrect Q-values]
\label{thm:relu_nets_kinks}
Assuming~\ref{assumption:state_next_state_are_close} and other notation from this section, the optimal solution for Equation~\ref{eqn:relu_nets} (right) consists of kinks
%%SL.5.26: that doesn't seem like a precise theorem statement...
at $\mathcal{D} \cup \{\bx'_1, \cdots, \bx'_n\}$. Moreover, the Q-values at any $\bx_j$ for which $\bx'_j \notin \mathcal{D}$ will be incorrect.
%%AK: TODO: refine the theorem to make it more formal
\end{theorem}
%%AK: TODO; also add SARSA discussion?
When fitting supervised targets to a deep ReLU network, the resulting solution is a piecewise linear function, with pieces intersecting only at the datapoints $\bx_1, \cdots, \bx_n$. On the other hand, Theorem~\ref{thm:relu_nets_kinks} shows that the optimal solution that satisfies Bellman equations will have additional kinks on out-of-sample state-action tuples used for the backup, i.e., $\{\bx'_1, \cdots, \bx'_n\} \mathbf{a}ckslash \mathcal{D}$. These kinks will alter the values learned at several other state-action pairs, while still satisfying Bellman constraints on the training data and being more smooth as measured by the integral of the second derivative of the function. This is a form of co-adaptation: implicit regularization towards smooth functions in a ReLU network makes predictions at $\bx$ and $\bx'$ coupled together so as to maximize smoothness. Having seen that implicit regularization stemming from a variety of factors including noisy updates and neural network architectures when combined with TD error and gradient descent and its direct relationship with feature co-adaptation, the next section aims to devise an explicit regularizer that can tackle these adverse effects of implicit regularization.   
\fi


\iffalse
\subsection{Consequences of Feature Co-Adaptation in TD-Learning}
Having seen that TD-learning co-adapts features at unseen actions to features at state-action tuples in the dataset, we now study its consequences. We will show that this co-adaptation can prevent learning high-frequency information in the Q-function crucial for control and destabalize learning, even when initialized in the vicinity of a good solution.

\begin{wrapfigure}{r}{0.4\textwidth}
    \centering
    \vspace{-22pt}
    \includegraphics[width=0.48\linewidth]{chapters/dr3/section3_figs/dynamics.pdf}
    \includegraphics[width=0.48\linewidth]{chapters/dr3/section3_figs/optimal_q.pdf}
    \includegraphics[width=0.48\linewidth]{chapters/dr3/section3_figs/supervised_q.pdf}
    \includegraphics[width=0.48\linewidth]{chapters/dr3/section3_figs/td_learning.pdf}
    \vspace{-0.21cm}
    \caption{\small{\textbf{TD-learning fails to represent high frequency changes in Q-functions much more than supervised learning.} On the 1D MDP (dynamics, top-left), TD learning ignores the high-frequency components of the Q-function leading to worse action selection compared to supervised regression.}} 
    \label{fig:1d_mdp}
    \vspace{-0.6cm}
\end{wrapfigure}
%%AK: Tengyu had an interesting suggestion here: represent the action choices of each function via a shaded interval on the the number line, using blue for a_0 and red for a_1. The number of switches will determine the complexity of the Q-function, and this will show that TD has 4 switches, supervised has 8 and actual has 12 or 13 switches.
\textbf{Inability to model high-frequency components of Q-function.} Prior work has noted that Q-functions can be highly non-smooth even when reward functions and dynamics are relatively simple~\citep{dong2020expressivity}.
%%AK: does gamma models also note something related to complexity of Q vs reward via the discount profile stuff?
Would feature co-adaptation lead the Q-function to ignore certain high-frequency components in the objective?
As a didactic example of this phenomenon, we utilize a 2-action MDP with a 1-D state space $\mathcal{S} \in [0, 1]$ from~\citep{dong2020expressivity}. This MDP exhibits a piecewise linear dynamics (Fig.~\ref{fig:1d_mdp}, top-left) and an identity reward function $r(\bs, \mathbf{a}) = \bs$ with two actions $\mathbf{a} \in \{0, 1\}$. The optimal Q-function exhibits high-frequency changes (Fig.~\ref{fig:1d_mdp}, top-right). We find that running Q-learning (Fig.~\ref{fig:1d_mdp}, bottom-right)
attains Q-functions that completely ignore these high-frequency changes. This often makes the resulting policy choose the worse action of the two possible actions. On the contrary, a supervised projection (Fig.~\ref{fig:1d_mdp}, bottom-left) of the Q-function does capture many of these high frequency shifts in the Q-function. To formalize this didactic example, we prove the following result showing that high-frequency components of the Q-function are not modelled as a direct consequence of Theorem~\ref{thm:aliasing_exists}. The proof and a complete statement for Theorem~\ref{thm:num_pieces} can be found in Appendix ??.

%%AK: this will have some conditions on dynamics too, maybe we mention that as a detail in the appendix? But if it is unclear, maybe we should add it here, and explain it here...
\begin{theorem}[Informal]
Assume that the state-space $\mathcal{S}$ of the MDP is given by the 1-D number line, and that the groundtruth Q-function is (approximately) piecewise linear in the state $s$ with $N^*$ pieces. Denote the (approximate) number of linear pieces in a learned infinite-capacity ReLU Q-network via direct supervised regression as $N_{\mathrm{Sup}}$ and via TD-learning as $N_{\mathrm{TD}}$. Then: $N_{\mathrm{TD}} \leq N_{\mathrm{Sup}} \leq N^*$.     
\label{thm:num_pieces}
\end{theorem}
%%AK: In the proof of this theorem, we show this for approximate number of pieces, which is the integral of the second derivative of the function w.rt.t the input over the input space, but I think going into details would just hurt understanding here.

%%AK: I am a little unsure about the following part, in the sense if we should have it or not? This might seem obvious to some extent? 
% \textbf{Severe co-adaptation renders distributional shift corrections ineffective.} To test whether offline RL corrections alleviate feature co-adaptation, we performed a controlled experiment -- we constrained offline RL regularizers to only control the last linear layer of the Q-network, whereas bootstrapping was allowed to train the entire network. While we might expect that offline RL methods may still be effective by adapting the last weight layer to the features, contrary to this expectation, as shown in Figure ??, we find that all such variants (denoted as CQL($\phi$)) perform extremely poorly compared to the complete offline RL method. Further note the inability to minimize the regularizer corresponding to distributional shift and an increased dot-product similarity, $\Delta(\bs, \mathbf{a}, \bs', \mathbf{a}')$. This indicates that feature co-adaptation can lead to failure of offline RL methods. \textcolor{red}{add new figure for this}
%%AK: If we do keep this, maybe we should also add some line to justify why offline RL methods are still sensitive -- this is because they are not exactly doing SARSA?

\textbf{Lack of stability near ``good'' Q-function solutions.}
%%SL.5.22: instead of using a vague term like "good" with scare quotes, can we just say optimal? And what does "lack of stability" mean? Maybe just state it directly: Implicit regularization can prevent convergence, even when initializing at an optimal solution. [or something like that]
Finally, we study if co-adaptation of features cause the TD-learning process to diverge away, even when initialized favorably in the vicinity of a good Q-function (e.g., one obtained via supervised regression to MC returns or one obtained via online RL). Running Q-learning from such a favorable initialization eventually produces solutions that perform poorly as shown in Figure ?? below. Moreover, \textcolor{red}{say something about ranks here}. Indeed, in accordance with Theorem~\ref{thm:aliasing_exists}, the feature co-adaption phenomenon drives learning towards solutions with lower $\srank(\bM_{\mathrm{TD}}(\phi))$ values, giving rise to poor performance. \textcolor{red}{add figure, theorem}      
\fi

%%SL.5.17: what are Bellman constraints?
% are only enforced approximately (i.e., when the TD error
% %%SL.5.17: was the TD error ever defined?
% is not exactly 0), the implicit regularization towards minimum $||\bw||_2$ norm solutions will lead to the Q-function ignoring high-frequency components. That is, if the true Q-function changes dramatically from one state to the next, low TD-error solutions will fail to represent these changes.
%%SL.5.17: I don't see why the above theorem indicates that this is true
%%AK: add a worst-case theorem as discussed with George today?
%%SL.5.17: maybe we should put this didactic example into a separate \textbf{} with more setup, instead of presenting this as a kind of footnote -- as-is, I think many people will not really understand it


%%AK: Check if we can take this paper's "peicewise linear theory" and convert it to a policy improvement bound differentiating between TD and supervised, as opposed to just fitting Q*-values?

% \subsection{Consequences of Overly Regularized Representations in TD Learning}
% Having seen aliasing of features on state-action tuples used for bootstrapping emerge as one pathological consequence of overly regularized representations in TD-learning compared to supervised learning, we next ask the following question we study the impact of over-regularization have on the performance of TD-learning. In particular, we ask: do TD-learning algorithms find generalizing and stable optima? To answer these questions, we consider a simple scenario where learning is initialized from a 


% \textbf{Abstract model.} Our abstract model captures feature learning as making a discrete selection among $K$ different feature vector candidates, $\{\Phi_1, \Phi_2, \cdots, \Phi_K\}$, $\Phi_i \in \mathbb{R}^{|\data|\times d}$,
%%SL.5.13: It would be way easier to understand if we could get continuous domains, and then just frame this as an optimization over \Phi (i.e., optimization over \Phi corresponds to selecting the best \Phi \in [some set]), that way we don't have to have this "discrete selection" business and could just say that it's part of the optimization process.
% and then training a linear layer $\bw \in \mathbb{R}^{d}$ to obtain the Q-function.
%%SL.5.13: One way you could phrase is this: Our abstract model of the learning process separates the neural network into two parts: a representation $\Phi$ and a weight vector $\bw \in ...$, such that the full model is given by $\Phi(..)^T \bw$ (i.e., $\bw$ corresponds to the last linear layer). The learning process is framed as a \emph{bilevel} optimization problem, where the weights $\bw$ are chosen subject to a constraint that the learning process chooses the optimal features $\Phi \in [set]$ for $\bw$ (or something like that)
% To mimic the overparameterized nature of neural networks, we assume that we operate in the overparameterized regime with $n < d$.
%%SL.5.13: This is kind of weird -- usually the last layer features are not that high dimensional, but the model parameters are. Are you sure we shouldn't look for some way to "NTK-ify" this? Perhaps a better way to frame this is that we are in the NTK regime where the choice of Phi corresponds to the choice of NTK (i.e., it's not fixed, as is more common in this analysis), while bw corresponds to the neural net parameters? That would justify the overparameterized regime and make this less weird.
% Assume that the initial value of the weight vector $\bw^{(0)} = 0$. We then write out the minimum-norm optimization problem shown below in Equation~\ref{eqn:min_norm} that attains the same solution as the optimal solution found by minimizing training TD error in this model, and characterize the properties of features $\Phi_K^*$ that are selected to obtain the minimum-norm solution. \textcolor{red}{more assumptions?}     
%%SL.5.13: It won't be clear to some people what min-norm has to do with neural net training, can you cite something to justify this?
% \begin{align}
%     \min_{\bw, \boldm_i \in \{0^d, 1^d \}}~~& ||\bw||_2^2 \nonumber\\
%     \text{s.t.}~~&~ \mathbf{a}r{\Phi}^\top \mathbf{a}r{\Phi} \bw = \mathbf{a}r{\Phi}^\top R + \gamma \mathbf{a}r{\Phi}^\top P^\pi \mathbf{a}r{\Phi} \bw, ~~ \mathbf{a}r{\Phi} = \left[\Phi_1, \cdots, \Phi_k\right] \otimes [\boldm_1, \cdots, \boldm_K]
% \label{eqn:min_norm}
% \end{align}
%%SL.5.13: ouch, this \bm_i is... difficult to parse
% Our first result characterizes the feature representation $\mathbf{a}r{\Phi}$ -- equal to one of $\Phi_i$ selected based on the learned masks $\bm_i$ -- that satisfies the Bellman consistency condition but also minimize the implicit regularizer, $||\bw||_2^2$,
% %%SL.5.13: where does this implicit regularizer come from?
% and use this to depict the existence of this phenomenon.


\iffalse

\section{Representation Regularization in Offline Q-Learning}
\label{sec:problem}
%%SL.5.13: See my comment on the title about "excessive" (maybe we call it Implicit Over-Regularization?). That said, this again sounds *extremely* similar to IUP, to the point where the section title alone could lead many readers to suspect this is just a direct copy of the IUP paper.
%%AK: yeah I agree. I am a little unsure what to call it, besides maybe admitting that this is similar IUP in high-level motivation but not low-level technical details. Do you think that's doable? My rationale was that right now readers might have the impression that we are trying to do something like IUP but also trying to distinguish it from IUP, without making clear what our contribution is and what's already there. Perhaps just saying something like "Fine-grained analysis" or something that explicitly quantifies the extent of this contribution is different? Avoiding that might just create questions. What do you think?
%%AK: the title sounds lika having a good connotation, is there a bad word for "regularization" that is not just "over-regularization" or "aliasing"?

% In this section, we study the mechanism by which implicit regularization effects are induced in offline Q-learning, and discuss how these effects can lead to pathological issues such as overly aliased representations and convergence to poor solutions. These aliasing properties exist even when learning is initialized from good solutions that do not exhibit this aliasing, and can make the learning eventually diverge. We first provide an empirical analysis of this phenomenon and then theoretically formalize these observations in a simple abstract model of learning dynamics of Q-learning.
Offline RL algorithms discussed previously are unstable and suffer from hyperparameter tuning challenges. A simple choice such as the number of training steps can be game-changing -- too few gradient steps will of course give rise to underfitted Q-functions, but perhaps surprisingly, too many gradient steps also lead to poor performance (Figure~\ref{fig:atari_5_percent}, Figure 2 in \citep{kumar2021implicit}). This phenomenon resembles statistical overfitting at first, however, it is actually underfitting caused due to excessive representational regularization of training deep networks with TD error that manifests as aliased features. While this issue has been broadly noted in previous work~\citep{kumar2021implicit}, in this section we will provide a fine-grained analysis of this phenomenon first empirically and then theoretically. In Section~\ref{sec:dr3_method}, we will then discuss a simple regularization scheme that can mitigate this issue. \textcolor{red}{TODO}  
%%SL.5.13: Maybe a somewhat more forceful lead-in could look like this:
% While the offline RL algorithms discussed in the previous section mitigate the worst challenges of offline RL~\citep{bear}, effectively using such methods in practice still requires extensive hyperparameter tuning. A particularly delicate choice is the number of gradient steps to take on the offline dataset -- too few gradient steps obviously produce underfitted, suboptimal value functions. But surprisingly, too many gradient steps also often result in poor performance, as illustrated in Figure ??. What is the reason for this performance collapse? While this phenomenon initially resembles overfitting, it is in fact an instance of \emph{underfitting}: although deep networks trained with SGD should provide a very good fit in standard supervised settings, we will argue that training with TD backups introduces a pathological over-regularization effect, induces excessive aliasing, and greatly constrains the expressive power of the resulting features. We first analyze this empirically, and then present a theoretical analysis. In Section ??, we will discuss how a simple regularization scheme can mitigate this issue.
%%AK: I havent added this fully, and instead cited IUP for the basic "noting" of this phenomenon and then said that we provide finegrained analysis of it. But I can change it. 

\iffalse
\subsection{Empirical Analysis}
\label{sec:empirical_analysis}
\begin{wrapfigure}{r}{0.49\textwidth}
    \vspace{-47pt}
    \centering
    \includegraphics[width=\linewidth]{chapters/dr3/atari/perf_3_games.pdf}
    \includegraphics[width=\linewidth]{chapters/dr3/atari/unnorm_3_games.pdf}
    \includegraphics[width=\linewidth]{chapters/dr3/atari/norm_3_games.pdf}
    \vspace{-0.65cm}
    \caption{\small{Performance of offline DQN and BC on 5\% DQN replay dataset~\citep{agarwal2019optimistic} (top), normalized similarity scores for DQN and BC (bottom). As compared to DQN, $\simnorm$ (cosine similarity) decays significantly faster for BC. 
    \textcolor{red}{Remove the middle row that's not useful. Reduce to 2 games. Also add SARSA}}
    } 
    \label{fig:atari_3_games}
    \vspace{-0.4cm}
\end{wrapfigure}
%% The first para is saying aliasing exists in very simple language
%%SL.5.13: I think before we dive into why it happens, can we just show what the problem is? E.g., show some learning curves where performance peaks and drops, and explain what's going on. Only then talk about *why* it happens
%%AK: sorry for bringing this up again. But I feel like that would make it like iUP basically. I have cited this issue of what happens using a figure later in the paper in the para above as well as citing figure from IUP. We can put a wrapfig in the accompanying text above, but it feels a little copied if we spend the technical section on this. Maybe this is not a good choice. What do you think? 
As noted in~\citep{kumar2021implicit}, one of the most notably visible impacts of representation regularization is the pathological feature aliasing phenomenon that arises with more training. While this aliasing issue has been previously quantified via a collapse in the rank of the feature matrix $\Phi$, this evidence does not shed light on the exact mechanism by which aliasing happens -- even standard supervised learning would exhibit a drop in rank($\Phi$); though not in enormous amounts. % one closing line on how as a result it is less clear how to measure aliasing and connection to bootstrapping empirically.   
%%AK: tried to add a comparison against 

\textcolor{red}{This first part of Section 3.1 will likely be removed in favor of a didactic example} \textit{How can we connect the degree of aliasing to bootstrapping?} Since the difference between bootstrapping and standard supervised learning is primarily that the Q-function at $(\bs, \mathbf{a}) \in \mathcal{D}$ is trained  with targets generated using Q-values at $(\bs', \mathbf{a}')$ instead of fixed targets, excessive similarity between $\phi(\bs, \mathbf{a})$ and $\phi(\bs', \mathbf{a}')$ leads to highly coupled Q-values on the two sides of the Bellman update, which can lead to issues such as overestimation and divergence~\citep{durugkar2018td}. Hence, it is informative to measure the similarities in representations $\phi(\bs, \mathbf{a})$ and $\phi(\bs', \mathbf{a}')$. We measure two notions of similarity: \textbf{(1)} we measure the cosine similarity between $\phi(\bs, \mathbf{a})$ and $\phi(\bs', \mathbf{a}')$, and \textbf{(2)} we measure an aggregate    

Observe in  Figure~\ref{fig:atari_3_games} that perhaps surprisingly the similarity between $\phi(\bs, \mathbf{a})$ and $\phi(\bs', \mathbf{a}')$ decreases and saturates at low values with behavior cloning (BC),
%%SL.5.13: This feels like a non-sequitur -- you're comparing representations at two different states, why does it matter that BC is trying to match the behavior policy?
On the other hand, DQN, which is trying to actually improve upon the behavior policy, essentially aliases
%%SL.5.13: There is no evidence of aliasing, just of high dot product (which is not the same)
feature representations at $(\bs, \mathbf{a})$ and $(\bs, \mathbf{a}')$, giving rise to very high similarity values.
%%SL.5.13: Without more context about what's going on, I would say at this point that this is probably due to the OOD actions problem you mentioned before, which DQN does nothing to fix. Additionally, I think it's very likely that many reviewers at this point woudl complain that it's non-sensical to compare BC (which learns policies) with DQN (which learns Q-functions).
This indicates that, compared to supervised learning (e.g., BC), implicit regularization effects in deep Q-learning have a tendency to alias predictions at states and corresponding next states. \textcolor{red}{Would be good to show this with SARSA vs MC: that way we can make a stronger statement like: Note that while both SARSA and MC returns are essentially computing the same quantity and differ only in the nature of implicit regularization induced. This difference makes a huge difference -- in one case, representations at consecutive states are essentially completely aliased, while supervised learning is able to disentangle representations.}
%%SL.5.13: Overall, I think this paragraph is rather problematic. If you want to explain this part, it would be good to really slow it way down and walk the reader through it much more slowly, otherwise so many of the choices in the above paragraph come across as ad hoc, making the conclusions unconvincing.

\fi

\subsection{A Didactic Example}
\label{sec:empirical_analysis}

As noted in~\citep{kumar2021implicit}, one of the most notably visible impacts of representation regularization is the pathological feature aliasing phenomenon that arises with more training. While this aliasing issue has been previously quantified via a collapse in the rank of the feature matrix $\Phi$, this evidence does not shed light on the exact mechanism by which aliasing happens -- even standard supervised learning would exhibit a drop in rank($\Phi$) with more training, and while prior work shows that bootstrapping exacerbates it empirically, it is unclear how exactly this amplification happens. In this section, we describe the intuition behind this mechanism with a didactic example of a 2-action, 1D line MDP~\citep{dong2020expressivity} with a piece-wise linear deterministic dynamics function, $P(\bs'|\bs, \mathbf{a}) = \mathbb{I}(\bs' = f(\bs, \mathbf{a}))$ shown in Figure ??. The reward $r(\bs, \mathbf{a})$ at any state is the value of the state itself, i.e., $r(\bs, \mathbf{a}) = \bs$.

The optimal Q-function for this MDP is shown in Figure ??, and 3-layer deep ReLU network Q-functions estimators learned via supervised regression and TD-learning on the identical finite dataset are shown respectively in Figures ?? and ??. While neither supervised regression nor TD learning can learn the complete structure of the optimal Q-function, TD learning fails to represent important high-frequency components of the Q-function (marked in yellow), leaning a ``simple'', smooth Q-function. Since it fails to model the changes in the Q-function well, the resulting policy often chooses the worse action. Quantitatively, the policy extracted from such a TD Q-function is worse than that extracted from the supervised Q-function at more than half the states.  
%%AK: todo: mark in yellow via keynote

%% The next para is saying aliasing is undesirable
\textbf{Why do we observe overly smooth Q-functions in the didactic example when trained with TD learning?}  While excessive aliasing of internal representations in the neural network is expected to generally lead to poor performance, aliasing between $\phi(\bs, \mathbf{a})$ and $\phi(\bs', \mathbf{a}')$ is especially detrimental when learning with Bellman backups. Intuitively, since Bellman backups train features such that the difference of Q-values, $Q(\bs, \mathbf{a}) - \gamma Q(\bs', \mathbf{a}')$ matches the reward function, $r(\bs, \mathbf{a})$, only on a finite number of state-action tuples seen in the dataset, the features $\phi(\bs, \mathbf{a})$ and $\phi(\bs', \mathbf{a}')$ can learn to only be sufficiently different to predict the reward, thereby achieving low TD error and may be excessively regularized otherwise, thus not capturing long-term structure in the Q-function. Put in other words, there are many possible assignments of weights to a function approximator that could give rise to equally low TD error at the cost of varying degrees of aliasing or regularization.

%%SL.5.13: This feels really hand-wavy. I'm also not sure I agree with this argument -- after all, how would it be any different if there *wasn't* aliasing? Wouldn't you still get a good fit between the difference and reward? This kind of a makes a non-falsifiable statement.
%%AK: this figure is like the example in the MB vs MF paper, but with Bellman backups run on it.

\begin{wrapfigure}{r}{0.5\textwidth}
    \centering
    \includegraphics[width=\linewidth]{chapters/dr3/atari/cql_on_bootstrapping_feat.pdf}
    \includegraphics[width=\linewidth]{chapters/dr3/atari/cql_losses_bootstrapped_feat.pdf}
    \includegraphics[width=\linewidth]{chapters/dr3/atari/sim_s_ns_cql_on_bootstrapping_feat.pdf}
    \vspace{-0.65cm}
    \caption{\small{CQL($\phi$), trained using 5\% DQN replay dataset, that learns on features trained solely via bootstrapping where the CQL regularizer $\mathcal{R}(\theta)$ only updates the linear weights of the Q-network. Different values of $\alpha_R$ correspond to different strengths of conservative regularization. We also show standard CQL~(red) for comparison.}} 
    \label{fig:atari_3_games_cql_bootstrap}
    \vspace{-0.6cm}
\end{wrapfigure}
When excessive aliasing is induced by such a mechanism, even modern offline RL methods that are meant to prevent against distributional shift
%%SL.5.13: It's unclear what preventing distributional shift has to do with this
are rendered ineffective. To demonstrate this empirically, we trained a modified version of CQL~\citep{kumar2020conservative}, CQL$(\phi)$, that learns on features $\phi(\bs, \mathbf{a})$ solely trained via bootstrapping, and the CQL regularizer is allowed to only update the last linear layer weights. As shown in Figure~\ref{fig:atari_3_games_cql_bootstrap}, no strength of conservative regularization is able to minimize out-of-distribution Q-values resulting in higher values of CQL loss and significantly worse performance as compared to CQL. This indicates that, no matter what, aliased representations can significantly hamper the efficacy of offline RL methods.
%%SL.5.13: This experiment seems weird. You said (and showed before) that features learned with bootstrapping are bad, and it seems like now you are saying if you take those features and retrain, it's still bad, but that's not surprising. I think the subtlety here is that you are also showing that the CQL regularizer is ineffective, but that seems obvious? And it also requires a degree of familiarity with CQL to understand, that the reader might not have. Maybe we can do away with this paragraph?

%%AK: this is the experiment where we initialize the Q-function from a good checkpoint and show it still performs poorly so it is reasoning about the stability aspect.
Finally, to demonstrate the detrimental extent of this implicit regularization on stability of the offline RL algorithm, we perform a controlled experiment where Q-learning is initialized from a ``good'' Q-function that doesn't exhibit aliasing.
%%SL.5.13: where does this come from?
As shown in Figure ??,
%%AK: TODO(AK): add figure!
more learning iterations with modern offline RL algorithms can still drive the algorithm away from this good solution towards more aliased and poor performing solutions. This shows that aliasing caused due to the implicit regularization of training does not just affect the peak performance of an algorithm, but also plays a significant role in destabilizing algorithms when they reach their peak performance.  

\subsection{Theoretical Analysis of Implicit Regularization in Offline Deep Q-Learning}
\label{sec:theory_evidence}
%%SL.5.13: Calling this "implicit regularization" seems premature -- all we showed is that features get larger dot products, which doesn't mean there is some sort of "implicit regularization" going on
In this section, we formalize our empirical observations from Section~\ref{sec:empirical_analysis} and provide a theoretical analysis of the implicit regularization issue. We aim to answer two questions: \textbf{(1)} How do implicit regularization effects in TD learning affect the the aliasing of representations at consecutive states used in the Bellman update? and, \textbf{(2)} How does excessive aliasing affect performance of the algorithm? To answer these questions, we first introduce a simple abstract model of neural network behavior
%%SL.5.13: Rephrase as something like: we first introduce a simple abstract model of neural network training dynamics in value-based RL, and then use this model to analyze the effect of repeated SGD updates on the TD objective. [or something like that]
that allows us to answer these questions.

% abstract model
%% AK: TODO (AK): Also check if we can generalize this to arbitrary continous domains
%%SL.5.13: In its current form, I'm a bit nervous about this version of the theory. I think the SGD implicit regularization version is more convincing and makes fewer arbitrary choices. I do think this version could be made better if we can get rid of the discrete set though. Would be good to get Tengyu's take on it too...
\textbf{Abstract model.} Our abstract model captures feature learning as making a discrete selection among $K$ different feature vector candidates, $\{\Phi_1, \Phi_2, \cdots, \Phi_K\}$, $\Phi_i \in \mathbb{R}^{|\data|\times d}$,
%%SL.5.13: It would be way easier to understand if we could get continuous domains, and then just frame this as an optimization over \Phi (i.e., optimization over \Phi corresponds to selecting the best \Phi \in [some set]), that way we don't have to have this "discrete selection" business and could just say that it's part of the optimization process.
and then training a linear layer $\bw \in \mathbb{R}^{d}$ to obtain the Q-function.
%%SL.5.13: One way you could phrase is this: Our abstract model of the learning process separates the neural network into two parts: a representation $\Phi$ and a weight vector $\bw \in ...$, such that the full model is given by $\Phi(..)^T \bw$ (i.e., $\bw$ corresponds to the last linear layer). The learning process is framed as a \emph{bilevel} optimization problem, where the weights $\bw$ are chosen subject to a constraint that the learning process chooses the optimal features $\Phi \in [set]$ for $\bw$ (or something like that)
To mimic the overparameterized nature of neural networks, we assume that we operate in the overparameterized regime with $n < d$.
%%SL.5.13: This is kind of weird -- usually the last layer features are not that high dimensional, but the model parameters are. Are you sure we shouldn't look for some way to "NTK-ify" this? Perhaps a better way to frame this is that we are in the NTK regime where the choice of Phi corresponds to the choice of NTK (i.e., it's not fixed, as is more common in this analysis), while bw corresponds to the neural net parameters? That would justify the overparameterized regime and make this less weird.
Assume that the initial value of the weight vector $\bw^{(0)} = 0$. We then write out the minimum-norm optimization problem shown below in Equation~\ref{eqn:min_norm} that attains the same solution as the optimal solution found by minimizing training TD error in this model, and characterize the properties of features $\Phi_K^*$ that are selected to obtain the minimum-norm solution. \textcolor{red}{more assumptions?}     
%%SL.5.13: It won't be clear to some people what min-norm has to do with neural net training, can you cite something to justify this?
\begin{align}
    \min_{\bw, \boldm_i \in \{0^d, 1^d \}}~~& ||\bw||_2^2 \nonumber\\
    \text{s.t.}~~&~ \mathbf{a}r{\Phi}^\top \mathbf{a}r{\Phi} \bw = \mathbf{a}r{\Phi}^\top R + \gamma \mathbf{a}r{\Phi}^\top P^\pi \mathbf{a}r{\Phi} \bw, ~~ \mathbf{a}r{\Phi} = \left[\Phi_1, \cdots, \Phi_k\right] \otimes [\boldm_1, \cdots, \boldm_K]
\label{eqn:min_norm}
\end{align}
%%SL.5.13: ouch, this \bm_i is... difficult to parse
Our first result characterizes the feature representation $\mathbf{a}r{\Phi}$ -- equal to one of $\Phi_i$ selected based on the learned masks $\bm_i$ -- that satisfies the Bellman consistency condition but also minimize the implicit regularizer, $||\bw||_2^2$,
%%SL.5.13: where does this implicit regularizer come from?
and use this to depict the existence of this phenomenon.

\begin{theorem}
\label{thm:aliasing_exists}
Let the singular value decomposition of $\Phi_i$ be given as $\Phi_i = \bU_i \Sigma_i \bV_i^\top$ and $\bw^{(*)}, \boldm^{(*)}$ minimize the objective in Equation~\ref{eqn:min_norm}. Assume that the reward vector lies in the column space of $\Phi_i$, $\forall i \in [K]$, i.e., $\exists~ y_i, R = \Phi_i y_i $.  Then, $\mathbf{a}r{\Phi}$ is such that:
\begin{equation*}
    \mathbf{a}r{\Phi} := \arg \min_{i}~ \big|\big| \Sigma_i^{-1} \left( \bU_i^T (I - \gamma P^\pi) \bU_i \right)^{-1} \Sigma_i y_i\big|\big|_2^2.
\end{equation*}
Thus, the resulting $\mathbf{a}r{\Phi}$ satisfies: $\mathrm{srank}\left(\mathbf{a}r{\Phi}^\top (\mathbf{a}r{\Phi} - \gamma P^\pi \mathbf{a}r{\Phi}) \right) \leq \mathrm{srank}\left(\Phi_i^\top (\Phi_i - \gamma P^\pi \Phi_i) \right)~ \forall i$, which quantifies the existence of aliasing between representations at consecutive states in TD-learning.
\end{theorem}
%%SL.5.13: It's not clear what this last sentence means ("which quantifies the existence of aliasing between representations at consecutive states in TD-learning") -- can we state the implication of this theorem more precisely. As written, it's also not clear what this theorem has to do with SGD (I guess it's the min-norm part?).

%%SL.5.13: It might also help with clarity to explain why this problem *doesn't* happen in the supervised learning case

A proof of Theorem~\ref{thm:aliasing_exists} can be found in the Appendix ??. The main consequence of this result is a characterization of the learned features at optimal TD solutions in our abstract model in terms of the effective rank~\citep{kumar2021implicit} of the matrix $\bM(\Phi) := \Phi^\top (\Phi - \gamma P^\pi \Phi)$. A low rank of $\bM(\Phi)$ for a given rank of $\Phi$ intuitively indicates that the basis of the difference in features at consecutive states, $\Phi - \gamma P^\pi \Phi$, heavily lies
%%SL.5.13: try to avoid hand-wavy language ("heavily lies") and state what you mean more precisely
in the null space of the feature matrix $\Phi$, as a result of which the weight vector $\bw$ will be updated only in a few directions allowed by both $\Phi$ and $\Phi-\gamma P^\pi \Phi$.indicating that only a partial set of features will actually be used for learning.
%%SL.5.13: something is malformed above ("indicating that")
To empirically verify the existence of such an aliasing phenomenon, following the procedure outlined in \citep{kumar2021implicit}, we measure the effective rank of $\bM(\Phi)$ and observe in Figure ?? that this matrix indeed has extremely low rank when training with TD backups, as compared to supervised regression.
%% AK: this supervised regression is BC. Should we also do this for something else?

%%AK: maybe write the stuff below as a theorem?
Another interesting consequence of Theorem~\ref{thm:aliasing_exists} is the effect of the ``simplicity'' of the reward function on feature aliasing. We define simplicity by the number of non-zero components in the vector $y_i$.
%%SL.5.13: maybe we should avoid ad hoc definitions like this, and try to just state this more plainly and directly?
As an extreme case, note that if the vector $y_i$ has all zeros, except a single 1 entry, the optimal $\mathbf{a}r{\Phi}$ is expected to induce $\bM(\Phi)$ with a much lower rank compared to when a significantly more number of values of $y_i$ are non-zero (as shown in Appendix ??).
%%SL.5.13: this seems imprecise ("much lower")
This means that when the reward function $R$ actually non-trivially combines the singular vectors of $\Phi$ -- which we refer to as a ``complex'' reward function -- the effective aliasing
%%SL.5.13: I think if you're going to use the term "aliasing" like this, it needs to be formally defined. Aliasing means that two things are indistinguishable (not similar, but indistinguishable). The word is being used in a different way here.
is little less than when it does not. We indeed observe this behavior in practice, as shown in Figure ?? in Section~\ref{sec:empirical_analysis}, and our abstract model sheds light on how this implicit regularization effect in TD-learning is exacerbated in scenarios where reward functions can be expressed using very few components of the feature matrix $\Phi$.
%%SL.5.13: how do you know if it can be expressed using very few components?
%%SL.5.13: I think I understand what you are trying to say in the above paragraph, but we need to find a cleaner and more concise way of saying it. Maybe what we can say is something like -- consider the projection of the reward function onto the column-space [?] of Phi, with coefficients ??. If these coefficients are sparse, we would expect [??] (try to be precise!)...

To conclude our analysis for question \textbf{(1)}, we finally note that an analogous result in supervised learning would indicate no existence of any implicit bias that preferentially aliases feature representations at consecutive states. While prior work \citep{kumar2021implicit} has also generally shown the compounding effect of implicit regularization towards low-rank $\Phi$ in TD-learning, our analysis explicitly identifies the structure of aliasing induced by this implicit regularization: the rank of the matrix $\bM(\phi)$ drops, leading to aliasing at consecutive states.
%%SL.5.13: It's good to have a paragraph like the one above, but it addresses two things simultaneously, and doesn't do either very well: the supervised learning bit seems to have no punchline (so... is this a contradiction? if not, why not?); the bit about IUP doesn't clearly state how what you are showing is different from IUP.

%%SL.5.13: Given how long-winded the above is, maybe consider a subsection heading for (2) or something (or at least paragraph heading)
Next, we answer question \textbf{(2)} regarding the detrimental impacts of aliasing. 
\textcolor{red}{Need to finish this bit -- there are some options we can go: (1) We can show that there exist MDPs with simple reward functions and complex Q-functions, where such an implicit regularizer will cause the MDP to learn overly smooth Q-functions. (2) We can show that even when initialized close to a good solution, this implicit regularizer will drive the model towards picking features that are the most aliased, in which case we do not even stabilize near a good solution even if we reach it. (3) We could show that distributional shift correction on top of aliased features will not work, similar to what we had before the ICML deadline. Which of these option(s) should we prefer?}

\fi

\section{Out-of-Distribution Actions in Q-Learning}
\label{sec:Problem Description}

% Q-learning and other ADP methods which rely on iterating the Bellman backup operator are particularly susceptible to out-of-distribution inputs, because any errors incurred on these inputs can be propagated to neighbor states via the backup and keep compounding over iterations of the algorithm. Unfortunately, error on a single state can propagate to other states and can potentially cause inaccurate predictions across the entire Q-function. As we will show, these inaccuracies do affect the performance of off-policy algorithms in practice.

When Q-learning and off-policy actor-critic algorithms are used with static off-policy data, it's common to see returns improve at first and then deteriorate, or even deteriorate right from the start, as shown in Figure \ref{fig:divergence}. At first glance, this resembles overfitting, but increasing the size of the static dataset does not rectify the problem, suggesting the issue is more complex.
%When running Q-learning on a static off-policy dataset, we often find that the performance of the algorithm is poor and the performance doesn't change drastically through training (e.g., results for the na\"{i}ve RL method in Figure \ref{fig:divergence}) whereas the Q-values usually diverge over the course of training. However, unlike in supervised learning, increasing the size of the static off-policy dataset does not rectify the problem, suggesting the situation is more complex.
%These results suggest a form of overfitting, however, the situation is more complex than supervised learning. 
%as the ground truth performance curves resemble validation error curves during overfitting in supervised learning. 
%However, this interpretation does not tell the whole story: 
%while we can test for overfitting using the Bellman error, early stopping on Bellman error is ineffective [either cite something or add an experiment on this to the appendix].
% . , suggesting that simple overfitting is an inadequate explanation. 
\begin{wrapfigure}{r}{0.5\textwidth}
\vspace{-10pt}
\begin{center}
    \includegraphics[width=0.48\linewidth]{images/cheetah_divergence.pdf}
    ~
    \includegraphics[width=0.48\linewidth]{images/cheetah_divergence_q_val.pdf}
  \end{center}
 \vspace{-10pt}
 %%SL.5.22: Very important: the y-axes are not labeled right now, and it took me a while to figure out which plot was showing what. What is log(Q)? I guess you're trying to show that the right plot has Bellman error (?), while the left has performance? A couple more things: (1) always put space before ( (you often omit this space) (2) consider a caption like this (once the figures are labeled more clearly): Off-policy learning with SAC on HalfCheetah-v2 for different dataset sizes ($n$). The performance (left) does not correlate with $n$, while the Q-values (right) diverge or saturate at values far from the actual return.
  \caption{Performance of SAC on HalfCheetah-v2 with off-policy expert data w.r.t. number of training samples ($n$). Note the large discrepancy between returns (which are negative) and logarithm of Q-values (which converge to large positive values or diverge) that is not solved with additional samples.} 
 \vspace{-15pt}
 \label{fig:divergence}
\end{wrapfigure}

We can understand the source of instability by examining the form of the Bellman backup. Although minimizing the mean squared Bellman error corresponds to a supervised regression problem, the targets for this regression are themselves derived from the current Q-function estimate. The targets are calculated by maximizing the approximate $Q$-function with respect to the action at the next state. However, the $Q$-function estimator is only reliable on inputs from the same distribution as its training set. As a result, na\"{i}vely maximizing the value may evaluate the $\hat{Q}$ estimator on actions that lie far outside of the training distribution, resulting in pathological values that incur large error. We refer to these actions as out-of-distribution (OOD) actions, and we call errors due to OOD actions \textit{boostrapping errors}. This is because not only do they produce inaccurate values on the states where the backup is computed, these errors will propagate on subsequent Bellman backups. 
If $\valerr_k(s) = |Q_k(s,a) - Q^*(s,a)|$ denotes the total error at iteration $k$ of Q-learning and $\projerr_k(s, a) = |Q_k(s,a) - \backup Q_{k-1}(s,a)|$ denote the current Bellman error, we can write $\valerr_k(s) \le \projerr_k(s,a) + \gamma \max_{a'} E_{s'}[\valerr_{k-1}(s',a')]$. This means errors from $(s', a')$ are discounted, then accumulated into $Q(s,a)$ in addition to new projection errors $\projerr_k(s, a)$ being introduced on the current iteration. $\projerr$ is expected to be high on OOD states and actions
%\TODO{gjt: explain this} 
as errors at these states-action pairs are never minimized during training.
%Furthermore, these errors are propagated to other states through the backup operator.

To mitigate bootstrapping errors, we can restrict the policy/actor to ensure that they output actions that lie in the support of training distribution. 
% \TODO{Isn't that just batch constrained Q-learning? Just restricting the actions to those in the training set. -- changed to distribution: addressed}. 
This is distinct from previous work (e.g.,~\citep{fujimoto2018off}) which constrains the \emph{distribution} of the learned policy to be close to the behavior policy, similarly to behavioral cloning~\cite{Schaal99isimitation}.
While this is sufficient to ensure that actions lie in the training set with high probability, it is overly restrictive. For example, if the behavior policy is close to uniform, the learned policy will behave randomly, resulting in poor performance, even when the data is sufficient to learn a strong policy (see Figure~\ref{fig:gridworld}.
for an illustration). The key distinction is that we restrict the support of the learned policy, but not the probabilities of the actions within the support.
Restricting the actions reduces bootstrapping error as the Q-function is no more queried on OOD actions. However, it may also prevent the algorithm from converging to the optimal $Q^*$. In the next subsection, we theoretically analyze this tradeoff.


%In order to formally analyze this problem, we perform an error propagation analysis of Q-learning on the lines of Approximate Value Iteration (AVI)~\cite{munos2003errorapi} and Approximate Policy Iteration (API)~\cite{bruno2015approximate}. Let $Q_1, \cdots, Q_K$ be the value-function iterates and $\pi_1, \cdots,\pi_k$ be the policy-iterates generated when performing actor-critic based Q-learning, which a special case of API. We can express the \emph{policy evaluation error} at iteration $k$ as $\valerr_k(s, a) = |Q_k(s, a) - Q^\pi(s, a)|$, and the \emph{projection error} as $\projerr_k(s, a) = |Q_k(s, a) - \Tpi Q_{k-1}(s, a)|$. 
%Then, we have $\valerr_k(s, a) \le \delta_k(s, a) + \gamma E_{s', \pi}[\valerr_{k-1}(s', a')]$ (see Appendix~\ref{app:error_prop} for details). In other words, approximation errors $\projerr_k(s, a)$ are introduced during the projection step, discounted, and propagated to neighboring states via the backup operator. Understanding the source of the errors and controlling them is key to producing a stable algorithm.

%When using Q-function values on actions that greedily maximize the value at the next state $s'$ ($\max_{a'} Q(s', a')$) as target values for Q-learning, the maximizing action at $s'$ can potentially be very different from the distribution of actions at state $s'$ defined by dataset distribution $\dataset$. Such actions that are very unlikely to have been sampled from the dataset distribution are called out-of-distribution (OOD) actions. 
%%SL.5.20: Can we formally define what that means, instead of just saying they are called this? -- i don't think so that we can formally define OOD in general, without going into some hypothesis testing thing.
%As neural nets are known to produce inaccurate results when queried on out-of-distribution inputs -- adversarial examples~\citep{goodfellow2015advexamples} are a well-known example of this phenomenon, Q-values corresponding to OOD actions are not accurate and reliable. This also means that using such Q-values for Bellman backups tends to destabilize Q-value estimates. Empirically, we find that OOD actions are a major source of error that arise in Q-learning style ADP methods with static-datasets. The error accumulated in the Q-function due to backups from OOD actions is called \textit{boostrapping error}. In Figure~\ref{fig:gridworld}, the top row demonstrates how error can propagate between states due to the bootstrapping process. We next propose to restrict the policy $\pi$ during policy improvement step and while computing the backup, so as to limit the amount of errors incurred due to OOD actions, which we discuss in the following sections.


%%%%%%%%%%%%%%%%%%%%%%%%%% OLD: Monday 7:04 pm
% In this section, we describe how errors occurring in the Q-function due to bootstrapping errors from certain
% %%SL.5.20: This somewhat contradicts what we wrote in the related work section -- we said the Fujimoto analysis is at the level of sets, while ours is on distributions, but now we are talking about sets too?
% actions -- which we call out-of-distribution actions -- can accumulate and hurt the performance of off-policy algorithms in practice to a major extent. We start by revisiting the study of error propagation in ADP methods.
% %%SL.5.20: I think it takes us way too long to get to the point here. Can we have a more focused opening paragraph that specifically talks about what we'll be analyzing and doing?

% %%SL.5.20: Is there any way we can move this discussion to related work? I think it breaks the flow to have a little "mini related work" at the top of the technical section. Especially after such a lengthy setup, many readers will get annoyed and wonder when you'll get around to telling them what you actually do.
% We analyse off-policy static-dataset Q-learning algorithms as specific instances of approximate value iteration (AVI)~\citep{munos2003errorapi} and approximate policy iteration (API)~\citep{bruno2015approximate}. 
% %%SL.5.20: Do you really need all this API and AVI stuff? Why not keep it simple and just discuss Q-learning?

% %%SL.5.20: Maybe we should have two subsections here -- a 4.1 that starts here and explains the problem, and a 4.2 (current 4.1) that explains the solution.

% Errors encountered during Q-learning
% %%SL.5.20: Just say Q-learning...
% %\TODO{not defined yet} 
% propagate 
% between neighbor states $s'$ and $s$ when performing a Bellman backup.
% %%SL.5.20: I feel like the above sentence is just a really long-winded way to say "The Bellman backup results in compounding errors." Try to rephrase sentences to be more concise, avoid unnecessary words.
% To formalize this, let $Q_1, \cdots, Q_K$ be the value-function iterates and $\pi_1, \cdots,\pi_k$ be the policy-iterates generated when performing actor-critic based Q-learning. We can express the \emph{policy evaluation error} at iteration $k$ as $\valerr_k(s, a) = |Q_k(s, a) - Q^\pi(s, a)|$, and the \emph{projection error} as $\projerr_k(s, a) = |Q_k(s, a) - \Tpi Q_{k-1}(s, a)|$. 
% Then, we have $\valerr_k(s, a) \le \delta_k(s, a) + \gamma E_{s', \pi}[\valerr_{k-1}(s', a')]$ (see Appendix~\ref{app:error_prop} for details).
% % \TODO{all for a fixed $\pi$, how is this related to Q learning}
% % \TODO{define $V_k$, etc.}
% In other words, approximation errors $\projerr_k(s, a)$ are introduced during the projection step, discounted, and propagated to neighboring states via the backup operator. Understanding the source of the errors and controlling them is key to producing a stable algorithm.
% %%SL.5.20: This seems like a reasonable statement, but I think readers will be lost at this point about where you are going. Maybe you can preface the above paragraph by saying something like this: When training on off-policy data, we often see actual policy performance improving briefly and then deteriorating sharply (see, e.g., results for the na\"{i}ve RL method in Figure ???). One might at first surmise that these issues are a form of overfitting, as the ground truth performance curves resemble validation error curves during overfitting in supervised learning. However, this interpretation does not tell the whole story: while we can test for overfitting using the Bellman error, early stopping on Bellman error is ineffective [either cite something or add an experiment on this to the appendix]. Furthermore, even a very large off-policy training set does not avoid this degradation problem, suggesting that simple overfitting is an inadequate explanation. We can obtain a better explanation from examining the form of the Bellman backup. Note that minimizing the Bellman error corresponds to a supervised regression problem, but the targets for this regression are themselves derived from the current Q-function estimate. We argue that it is these estimates themselves that are the sources of the degradation: they are calculated by finding the action that maximizes the value at the next state. However, the value estimate is obtained from a function approximator, and this function approximator is only reliable on inputs from the same distribution as its training set. Since Q-learning only trains the Q-function via regression on state-action tuples seen in the training data, the actions that maximize the target value might lie very far outside of the training distribution, and therefore might incur very large error. We can formally analyze this source of error as follows. [and then talk about delta etc]

% These errors $\valerr_k$ arise from a multitude of sources and include function approximation error, sampling error, distribution shift error, and bootstrapping error. When learning from static, off-policy datasets, sampling error is largely uncontrollable, and it is hard to provide guarantees about function approximation error with deep neural nets.
% %%SL.5.20: I think this is misleading. While it's true that we cannot provide guarantees, in general function approximation error with large function approximators is low. You can reference our debugging paper for this.
% When training Q-functions using $(s, a, r, s')$ tuples from the dataset, the fixed point iteration scheme
% %%SL.5.20: Which fixed point iteration scheme?
% queries and uses the Q-function value on actions that greedily maximize the value at the next state $s'$ ($\max_{a'} Q(s', a')$). However, the maximizing action at $s'$ can potentially be very different from the distribution of actions at state $s'$ defined by dataset distribution $\dataset$. Such actions that are very unlikely to have been sampled from the dataset distribution are called out-of-distribution (OOD) actions.
% %%SL.5.20: Can we formally define what that means, instead of just saying they are called this?
% Neural nets are known to produce inaccurate results when queried on out-of-distribution inputs -- adversarial examples~\citep{goodfellow2015advexamples} are a well-known example of this phenomenon.
% %%SL.5.20: That's a good explanation
% Regressing to the target-value computed using such OOD actions coupled with the $\max$ step can lead to an accumulation of error in the Q-function by virtue of error propagation.
% %%SL.5.20: Now it's again unclear whether you are talking about the same phenomenon, or something else. Basically, the logical connection between the sentence "Neural nets.." and "Regressing.." is missing.
% %\TODO{is $\pidata$ a single policy or set? -- set}.
% The problem is further exacerbated by the fact that updates are only made to Q-values of state-action pairs present in $\dataset$, so even though the Bellman backup queries the Q-function for OOD actions, it never \emph{trains} it on those actions. As deep neural-net function approximators can often generalize in undesired and unpredictable ways, such backups can destabilize learning and can lead to divergence in Q-functions.

% Empirically, we find that OOD actions are a major source of error that arise in Q-learning style ADP methods with static-datasets. The error accumulated in the Q-function due to backups from OOD actions is called \textit{boostrapping error}. In Figure~\ref{fig:gridworld}, the top row demonstrates how error can propagate between states due to the bootstrapping process.
% %%SL.5.20: Can you add your quantative "badness" experiments, at least to an appendix, and reference it here

% We next propose to restrict the policy $\pi$ during policy improvement step and while computing the backup, so as to limit the amount of errors incurred due to OOD actions, which we discuss in the following sections.
%%SL.5.20: These are two separate things: one thing is to restrict the policy, the other is to restrict the backup. Should we more explicitly separate these things?
%%%%%%%%%%%%%%%%%%%%%%%%%%%%%%%%%%%%%%%%%

%%%%%%%%%%%%%%%%%%%%%%%% OLD: Sunday 05/19 9pm%%%%%%%%%%%%%%%
% When training Q-functions using $(s, a, r, s')$ tuples, the fixed point iteration scheme queries and uses the Q-function value on actions that greedily maximize the value at the next state $s'$ ($\max_{a'} Q(s', a')$) \TODO{this may be an appropriate time to talk about adverserial examples as the actor does end up exploiting inaccuracies in Q}. However, the maximizing action at $s'$ can potentially be very different from the distribution of actions defined by the behaviour policy $\pidata(\cdot|s')$ \TODO{is $\pidata$ a single policy or set?}. Moreover, updates are only made to Q-values of state-action pairs present in $\dataset$. As deep neural-net function approximators can often generalize in undesired and unpredictable ways, regressing to the target-value computed using out-of-distribution actions coupled with the $\max$ step can lead to an accumulation of error in the Q-function. We refer to this problem as the out-of-distribution actions problem in Q-learning.

% \TODO{I suggest rewriting this section as: 1) Errors in Q functions are bad in ADP because they propogate, 2) general error prop result, 3) where do errors come from, 4) OOD errors are a big problem}

% In this section, we describe the out-of-distribution action problem and motivate a solution via constraining action choices in the Bellman backup operator.
% To get started, it is important to mention that, machine learning algorithms are known to produce inaccurate results when queried on out-of-distribution inputs -- adversarial examples~\citep{goodfellow2015advexamples} are a well-known example of this phenomenon. \TODO{gjt: unclear why this is important to mention there, the following is relevant to all sources of error in the learned Q (which you state could come from many sources later).}
% Q-learning and other ADP methods which rely on iterating the Bellman backup operator are particularly susceptible to out-of-distribution inputs, because any errors incurred on these inputs can be propagated to neighbor states via the backup and keep compounding over iterations of the algorithm. Unfortunate error on a single state can potentially cause inaccurate predictions across the entire Q-function. As we will show, these inaccuracies do affect the performance of off-policy algorithms in practice. 

% %In this section we will describe the problem we are aiming to solve. 
% When training Q-functions using $(s, a, r, s')$ tuples, the fixed point iteration scheme queries and uses the Q-function value on actions that greedily maximize the value at the next state $s'$ ($\max_{a'} Q(s', a')$) \TODO{this may be an appropriate time to talk about adverserial examples as the actor does end up exploiting inaccuracies in Q}. However, the maximizing action at $s'$ can potentially be very different from the distribution of actions defined by the behaviour policy $\pidata(\cdot|s')$ \TODO{is $\pidata$ a single policy or set?}. Moreover, updates are only made to Q-values of state-action pairs present in $\dataset$. As deep neural-net function approximators can often generalize in undesired and unpredictable ways, regressing to the target-value computed using out-of-distribution actions coupled with the $\max$ step can lead to an accumulation of error in the Q-function. We refer to this problem as the out-of-distribution actions problem in Q-learning.
% %%SL.5.15: Make sure you introduce your terminology! While it may be obvious what \pi_{data} means, it's still better to define.
% %for that state. 
% %%SL.5.15: I would suggest defining new commands for commonly used symbols (like \pi_{data}) so that it's easy to change and you don't have to constantly type these things. That also makes it convenient to properly use \mathbf for vectors, etc. Learning how to use macros like that in latex is important for good typesetting and flexible refactoring of symbols.
% %%SL.5.15: Somewhere there is a "missing punchline" in the above paragraph. Can you add a sentence at the end about what's the point? Don't leave that implicit.
% %%SL.5.11: can we write this in terms of distributions rather than sets? this is important, because we care about things being out-of-distribution, not just out-of-sample -- that's an important distinction from regular overfitting
% %Further,
% %%SL.5.15: I'm not sure if "Also" is the right word with which to begin this transition, can you think of a way to change this transition to reflect the logical progression of the argument?
% %with fixed datasets ADP algorithms would perform fixed point updates on only those transitions that are present in $\dataset$, so Q-values for state-action pairs which are out of the dataset distribution are never updated. This results in inaccurate values for $Q(s, a)~ \forall~ (s, a) ~~\text{s.t.}~~ \rho_{\dataset}(s, a) \leq \varepsilon$. The fixed point backup for Q-learning would still end up querying and backing up these incorrect Q-values without updating them ever during training. This leads to an accumulation of error as more and more steps of ADP are performed, making Q-learning in batched, off-policy
% %%SL.5.15: Off-policy is not synonymous with batched
% %settings highly sensitive to initialization and prone to instabilities. This often manifests as diverging Q-functions or excessively overestimated Q-value estimates~\cite{fujimoto18addressing},
% %%SL.5.15: maybe not the right citation for this? there must be many works that point this out
% %and as we will show, is a serious problem with most batched Q-learning applications. We call this problem Out-of-distribution Actions Problem. This problem exists in Q-learning as there is no implicit normalization mechanism on Q-functions as opposed to policies which are probability distributions and must integrate to 1, hence, learning Q-values for good actions by performing dynamic programming doesn't ensure other Q-values are correct.
% %%SL.5.15: I don't think we need to capitalize this.
% %Next, we analyze this problem from the perspective of error-propagation, which has been used in the analysis of approximate dynamic programming methods. Then we discuss what implications our analysis has in deigning a practical algorithm for reducing bootstrapping error.
% %%SL.5.15: why is it significant that this area is well studied?

% %%SL.5.15: Perhaps for next section, you mean to create a subsection rather than a new top-level section? I also think that the current opening here is not very informative. It doesn't really say what the rest of this section will be discussing.

% %%SL.5.11: somehow it's not obvious to me what point above sentence is trying to make -- are you saying that only target value matters and not current value? if so, try to make that explicit

% %%SL.5.11: allude to some evidence we will show of this?


% %%SL.5.11: rewrite above w/o using "counterfactual" somehow, it seems really confusing

% %%SL.5.11: very run-on sentence


% %%SL.5.15: I got this far on 5/15, will continue tomorrow
% %\subsection{Error Propagation and the out-of-distribution action problem}

% In order to formalize the out-of-distribution actions problem, we turn to the study of error propagation in ADP, which provides machinery for understanding how approximation errors propagate through a repeated bootstrapping process. We analyse off-policy static-dataset Q-learning algorithms as specific instances of Approximate Value Iteration(AVI)~\citep{munos2003errorapi} and Approximate Policy Iteration(API)~\citep{bruno2015approximate}. \TODO{maybe mention 1 line about what API and AVI do}

% Errors encountered during API or AVI 
% %\TODO{not defined yet} propagate 
% between neighbor states $s'$ to $s$ when an action is selected which results in the agent visiting $s'$. We can express the \emph{policy evaluation error} at iteration $k$ as $\valerr_k(s) = |V_k(s) - V^\pi(s)|$, and the \emph{projection error} as $\projerr_k(s) = |V_k(s) - \Tpi V_{k-1}(s)|$. 
% Then, we have $\valerr_k(s) \le \delta_k(s) + \gamma E_{\pi}[\valerr_{k-1}(s')]$ (see Appendix~\ref{app:error_prop} for details).
% % \TODO{all for a fixed $\pi$, how is this related to Q learning}
% \TODO{define $V_k$, etc.}
% In other words, approximation errors are introduced during the projection step ($\projerr_k(s)$), discounted, and propagated to neighboring states via the backup operator. Understanding the source of the errors and controlling them is key to producing a stable algorithm.

% In the case of static datasets, we empirically find that the major source of this error is a result of using out-of-distribution actions \TODO{define what an OOD action is} to compute the backup. Figure~\ref{fig:gridworld} demonstrates the accumulation of such error in the case of gridworlds. \TODO{one more sentence here}

% We propose to carefully restrict the policy $\pi$ during policy improvement and while computing the backup, so as to limit the amount of errors incurred due to OOD actions, which we discuss in the following sections.
% \TODO{so far, only talking about approx error, not specific to OOD error}
%%%%%%%%%%%%%%%%%%%%%%%%%%%%%%%%%%%%%%%%%%%%%%%%%%%%%%%

%\subsection{Set-Constrained Backups}
%\label{sec:set_constrained_backup}
%In order to formally analyze this problem, we perform an error propagation analysis of Q-learning on the lines of Approximate Value Iteration (AVI)~\cite{munos2003errorapi} and Approximate Policy Iteration (API)~\cite{bruno2015approximate}. Let $Q_1, \cdots, Q_K$ be the value-function iterates and $\pi_1, \cdots,\pi_k$ be the policy-iterates generated when performing actor-critic based Q-learning, which a special case of API. We can express the \emph{policy evaluation error} at iteration $k$ as $\valerr_k(s, a) = |Q_k(s, a) - Q^\pi(s, a)|$, and the \emph{projection error} as $\projerr_k(s, a) = |Q_k(s, a) - \Tpi Q_{k-1}(s, a)|$. 
%Then, we have $\valerr_k(s, a) \le \delta_k(s, a) + \gamma E_{s', \pi}[\valerr_{k-1}(s', a')]$ (see Appendix~\ref{app:error_prop} for details). In other words, approximation errors $\projerr_k(s, a)$ are introduced during the projection step, discounted, and propagated to neighboring states via the backup operator. Understanding the source of the errors and controlling them is key to producing a stable algorithm.


%%SL.5.20: Try to be careful to distinguish this from Fujimoto -- all this stuff that treats "out of distribution" as a set will come across as rather disappointing.
%We believe this to be an effective strategy because while it is difficult to prevent errors from arising (i.e. due to lack of data, or other approximation errors), we can prevent the propagation of errors through better action selection. How should we select actions? We should select actions from all policies that could have possibly generated that action -- policies for which the sampled action lies in their high-confidence support-set.

% First, we introduce the notion of set-constrained Bellman operators,
% %%SL.5.20: This kind of makes it sound like it's basically a copy of the Fujimoto paper. Can we stop talking about sets? This would also go a lot better if before introducing this, you provide a bit of discussion that motivates what is coming next.
% which restricts the set of policies in the maximization.%, then describe how this reduces error propagation compares to using an unmodified Bellman backups and finally utilize them to build a practical algorithm. 
% \begin{definition}[Set-constrained operators]
% Given a set of policies $\Pi$ 
% %, we define the set-constrained policy improvement operator as:
% %$\greedyPi(Q, s) = \argmax{\pi \in \Pi}~ \expec_{a \sim \pi}[Q(s, a)]$,
% %and 
% , the set-constrained backup operator is:
% \[ \TPi Q(s, a) \coloneqq \expec \left[ R(s, a) + \gamma \max_{\pi \in \Pi} \expec_{\pi(a' | s')T(s' | s, a)}\left[Q(s', a') \right] \right]. \]
% \end{definition}
% Although this is a general definition, we are interested in the case where $\Pi = \{ \pi ~|~ \pi( a | s) = 0 \text{ whenever } \beta( a | s) < \epsilon \}$, where $\beta$ is the behavior policy (i.e., the set of policies that have support in the probable regions of the behavior policy). Restricting the backup operator may prevent the algorithm from converging to the optimal Q-function. To capture this, we define a \emph{suboptimality constant}, which measures how far $\pi^*$ is from $\Pi$. %Specifically, it is the difference between the expected return when following a maximizing policy in $\Pi$ and $\pi^*$ for one step and then following $\pi^*$ thereafter.
% \begin{definition}[Suboptimality constant]
% The suboptimality constant is defined as:
% \[ \alpha(\Pi, s, a) = |\TPi Q^*(s, a) - T^* Q^*(s, a)|. \]
% \end{definition}
% Note that if $\pi^*$ is contained within $\Pi$, then the suboptimality constant is 0.

% \TODO{Justin: change from here}
% We can quantify the suboptimality of \emph{set-constrained Q-learning}. For clarity, we present results using $\ell^\infty$-norm bounds in the main text, which demonstrate the intuitions underlying the method, and refer readers to Appendix~\ref{app:error_prop} for more practically applicable $\ell^p$-norm bounds, as well as analogous results for set-constrained API. 
%%SL.5.20: I think some readers will feel a bit lost here, because it's not completely obvious what the significance of these results are. Perhaps it would be better if you explain the method first (this is kind of implied -- just constrain the policy to the set -- but it should be made explicit).

% \begin{theorem}[Set-constrained error propagation (API)]
% Suppose we run approximate value iteration with the set-constrained backup operator $\TPi$ and obtain a sequence of Q-functions $Q_0, Q_1, ... Q_K$ and a sequence of policies $\pi_1, \cdots, \pi_K$ and denote $Q^{\pi_i}$ as the true Q-function for policy $\pi_i$. Let $\delta_k(s, a) = |Q_k(s, a) - \TPi Q_{k-1}(s, a)|$, and $\valerr(s, a) = |Q_k(s, a) - Q^*(s, a)|$. Then,
% {$$\valerr_k(s, a) \le \projerr_k(s, a) + \alpha(\Pi, s, a) + \gamma \max_{\pi \in \Pi} \expec_{\pi, s'}[\valerr_{k-1}(s', a')], $$}
% and 
%\TODO{this bound actually doesn't illustrate error reduction - probably need Lp norm bounds}
%$$ Q^* - Q^{\pi_k} \leq  $$
% \[\lim_{k \to \infty} \norminf{V_k - V^*} \le \frac{2\gamma }{(1-\gamma)^2}\left(\max_{s, k} \delta_k(s) + \alpha(\Pi,s)\right). \]
% \end{theorem}
% \begin{proof}
% See Appendix~\ref{app:error_prop}.
% \end{proof}

% Our strategy for reducing error propagation is to restrict the policy improvement and the Bellman Backup operators from selecting actions which may query the Q-function on out-of-distribution actions. First, we introduce the notion of set-constrained Bellman operators,
% %%SL.5.20: This kind of makes it sound like it's basically a copy of the Fujimoto paper. Can we stop talking about sets? This would also go a lot better if before introducing this, you provide a bit of discussion that motivates what is coming next.
% which restricts the set of policies in the maximization, then describe how this reduces error propagation compares to using an unmodified Bellman backups and finally utilize them to build a practical algorithm.  \begin{definition}[Set-constrained operators]
% Given a set of policies $\Pi$ 
% %, we define the set-constrained policy improvement operator as:
% %$\greedyPi(Q, s) = \argmax{\pi \in \Pi}~ \expec_{a \sim \pi}[Q(s, a)]$,
% %and 
% , the set-constrained backup operator is:
% \[ \TPi Q(s, a) \coloneqq \expec \left[ R(s, a) + \gamma \max_{\pi \in \Pi} \expec_{\pi(a' | s')T(s' | s, a)}\left[Q(s', a') \right] \right]. \]
% \end{definition}
% Although this is a general definition, as we will see, we are interested in the case where $\Pi = \{ \pi ~|~ \pi( a | s) = 0 \text{ whenever } \beta( a | s) < \epsilon \}$, where $\beta$ is the behavior policy (i.e., the set of policies that have support in the probable regions of the behavior policy). Restricting the backup operator may prevent the algorithm from converging to the optimal Q-function. To capture this, we define a \emph{suboptimality constant}, which measures how far $\pi^*$ is from $\Pi$. %Specifically, it is the difference between the expected return when following a maximizing policy in $\Pi$ and $\pi^*$ for one step and then following $\pi^*$ thereafter.
% \begin{definition}[Suboptimality constant]
% The suboptimality constant is defined as:
% \[ \alpha(\Pi, s, a) = |\TPi Q^*(s, a) - T^* Q^*(s, a)|. \]
% \end{definition}
% Note that if $\pi^*$ is contained within $\Pi$, then the suboptimality constant is 0. A natural question now is whether we can ever improve upon constraining the set $\Pi$ to just be the behaviour policy $\beta(a|s)$. The answer to this is yes, and this can be explained via the following bound.

% \begin{proposition}
% \TODO{Justin: define symbols}
% Suppose we run \TODO{policy evaluation?} using $\pi$. Let the projection error be bounded each step as $|V_k - T^{\pi}V_{k-1}| \le \epsilon$. Then,
% \[
% \lim_{k \to \infty} \rho_0 |V_k - V^*| \le \rho^{\pi} \epsilon + |\rho^\pi - \rho^{\pi^*}|\Rmax
% \]
% \end{proposition}
% \begin{proof}
% See Appendix~\ref{app:error_prop}.
% \end{proof}

%The above bound tells us that backing up from the a policy that differs from the behaviour policy, in a lot of cases, can bring about a larger decrease in the suboptimality bias than increase in the amount of error incurred. In the next subsection, we see how to choose such a policy.   

\subsection{Distribution-Constrained Backups}
\label{sec:dist_constrained}
In this section, we describe a backup that restricts the set of policies in maximization, and provide performance bounds which depend on the choice of policy set chosen. Then, we motivate why using the support of the data is a reasonable choice for constructing the constraint set. We begin with the definition of a distribution-constrained operator, which only performs backups according to a restricted set of policies:
\begin{definition}[Distribution-constrained operators]
Given a set of policies $\Pi$ 
%, we define the set-constrained policy improvement operator as:
%$\greedyPi(Q, s) = \argmax{\pi \in \Pi}~ \expec_{a \sim \pi}[Q(s, a)]$,
%and 
, the distribution-constrained backup operator is:
\[ \TPi Q(s, a) \coloneqq \expec \big[ R(s, a) + \gamma \max_{\pi \in \Pi} \expec_{\pi(a' | s')T(s' | s, a)}\left[Q(s', a') \right] \big] \]
\end{definition}

To analyze the behavior of this backup, we first quantify two sources of error. The first is a \emph{suboptimality bias} -- the optimal policy may not exist inside the chosen policy set and thus a suboptimal solution will be found. The second arises from distribution shift between the training distribution and the policies used for backups -- this formalizes the notion of OOD actions. %and states.
To capture suboptimality in the final solution due to this restriction, we define a \emph{suboptimality constant}, which measures how far $\pi^*$ is from $\Pi$. 
\begin{definition}[Suboptimality constant]
The suboptimality constant is defined as:
\[ \alpha(\Pi) = \max_{s,a} |\TPi Q^*(s, a) - T^* Q^*(s, a)|. \]
\end{definition}
The best possible value of of this constant is $\alpha(\Pi) = 0$, which corresponds to the case when $\pi^*$ is contained within $\Pi$. Next, we define a concentrability coefficient~\citep{munos2005erroravi} which quantifies a ratio of how far the policy's visitation distribution is from the data. This constant captures the notion of the degree to which states and actions are out-of-distribution.
\begin{assumption}[Concentrability]
Let $\rhoinit$ denote the initial state distribution, and $\mu$ denote the distribution of the training data over $\mathcal{S} \times \mathcal{A}$. For all $\pi_1, ... \pi_k, \pi_{k+1}$ such that $\pi_i \in \Pi$, and define the operator $A^{\pi}$ as: $A^{\pi}: \mathcal{S} \rightarrow \mathcal{S} \times \mathcal{A}$, then, assume there exists coefficients $c(m)$ such that:
\[
\rhoinit P^{\pi_1}P^{\pi_2}...P^{\pi_k} A^{\pi_{k+1}} \le c(k) \mu
\]
Correspondingly, the concentrability coefficient $C(\Pi)$ is defined as
\[
C(\Pi) \defeq (1-\gamma)^2\sum_{k=1}^\infty k\gamma^{k-1}c(k)
\]
\end{assumption}
To provide some intuition on the value of $C(\Pi)$, we can see that if $\mu$ was generated by some policy $\pi$, and $\Pi = \{\pi\}$ is a singleton set, then we would have $C(\Pi)=1$, which is the smallest possible value. However, if $\Pi$ contained policies significantly far from $\pi$, the value could be potentially large, in the limit, tending to $\infty$ if support of $\Pi$ is not contained in $\Pi$. With $C(\Pi)$ and $\alpha(\Pi)$ defined, we can now bound the performance of approximate distribution-constrained Q-iteration:
\begin{theorem}
\label{thm:avi_bound}
Suppose we run approximate distribution-constrained Q-iteration with a set constrained backup $\TPi$. Assume that $\delta(s,a) \ge \max_k |Q_k(s,a) - \TPi Q_{k-1}(s,a)|$ bounds the Bellman error. Then,
\[\lim_{k \to \infty} \expec_{\rhoinit, \pi something}[|Q_k(s,a) - Q^*(s,a)|] \le 
\frac{\gamma}{(1-\gamma)^2}\left[ C(\Pi)\expec_\mu[\projerr(s,a)] + \alpha(\Pi) \right]
\]
\end{theorem}
\begin{proof} See Appendix~\ref{app:error_prop}, Thm.~\ref{thm:avi_bound_proof} \end{proof}

This bound formalizes the tradeoff between keeping policies chosen during backups close to the data (captured by $C(\Pi)$) and keeping the set $\Pi$ large enough to capture well-performing policies (captured by $\alpha(\Pi)$). When we expand the set of policies $\Pi$, we are increasing $C(\Pi)$ but decreasing $\alpha(\Pi)$. An example of this tradeoff, and how a careful choice of $\Pi$ can yield superior results, is given in a toy gridworld example in Fig.~\ref{fig:gridworld}, where we visualize errors accumulating during distribution-constrained Q-iteration for different choices of $\Pi$. 

Finally, we motivate the use of support sets to construct $\Pi$. We are interested in the case where $\Pi_\epsilon = \{ \pi ~|~ \pi( a | s) = 0 \text{ whenever } \beta( a | s) < \epsilon \}$, where $\beta$ is the behavior policy (i.e., $\Pi$ is the set of policies that have support in the probable regions of the behavior policy). 

Why should we use support in order to construct $\Pi$ in the distribution-constrained operator? Keeping policies in the support of the data distribution is a reasonable choice as it allows us to bound the concentrability coefficient:
\begin{theorem}
\label{thm:conc_coeff_bound}
Assume the data distribution $\mu$ is generated by a policy $\beta$, such that $\mu(s,a) = d_\beta(s,a)$. Let us define $\Pi_\epsilon = \{ \pi ~|~ \pi( a | s) = 0 \text{ whenever } \beta( a | s) < \epsilon \}$. Then, the concentrability coefficient is bounded as:
\[
C(\Pi_\epsilon) \le \TODO{Justin}
\]
\end{theorem}
\begin{proof} See Appendix~\ref{app:error_prop} \end{proof}

Thus, using support sets gives us a single lever, $\epsilon$, which simultaneously trades off the value of $C(\Pi)$ and $\alpha(\Pi)$. Not only can we provide theoretical guarantees, we will see in our experiments (Sec.~\ref{sec:experiments}) that constructing $\Pi$ in this way provides a simple and effective method for implementing distribution-constrained algorithms.

\begin{figure}
    \centering
    \includegraphics[width=0.9\textwidth]{images/gridworld}
    \caption{Visualized error propagation for various choices of the constraint set $\Pi$
    - unconstrained (AVI,
    %%SL.5.22: is AVI ever defined?
    \TODO{Justin: define AVI, or lets call it Q-learning}
    top row), support-constrained (middle),
    and constraining to the behaviour policy (policy-evaluation, bottom). Dark values represent high error and light values represent low error. The task (leftmost image) is to reach the bottom-left corner from the top-left, but the behaviour policy (visualized as arrows in the task image, support shown in black on the support set image) travels to the bottom-right with a small amount of $\epsilon$-greedy exploration. Standard AVI propagates large errors from the low-data regime into the high-data regime, leading to inaccurate value estimates. Policy-evaluation reduces error propagation from low-data regimes but introduces significant suboptimality bias as the data policy is not optimal. A carefully chosen support-constrained backup strikes a balance between these two extremes by confining error propagation to the low-data region while introducing minimal suboptimality bias.}
    \label{fig:gridworld} 
\end{figure}

% \subsection{Choosing Backup Policies for OOD Action Error Reduction}
% \label{sec:choosing_policies}
% Argument in Sec.~\ref{sec:tradeoff} tells us that, with a careful selection of the policy under which the target value is computed, the overall error of value estimates from the optimal value function $\|V^* - V_k\|$ can be reduced. How should we search for a policy that minimizes the overall error? Our choice is to backup from policies which maintain high-support over the action set of the data.
% %%SL.5.22: I think it's not obvious to readers that "policy for the backup" means the distribution over the actions under which the target value is calculated. -- addressed

% To justify this choice,
% %%SL.5.22: What choice? -- choice of backing up from any policy that maintains high support over data.
% we note that the error analysis relies on being able to quantify $\delta_k(s, a)$ (the per-state-action bellman error) for OOD actions. Outside of the support of the data distribution, it is hard to provide guarantees on $\delta_k$. However, when $a$ lies inside the support of the training distribution for a given state $s$, high-capacity function approximators trained with supervised learning are expected to produce a bounded error, given enough samples.
% %Even if they don't produce bounded error on such in-support inputs, techniques such as Prioritized Replay~\cite{Schaul2016PrioritizedER} can be employed to ensure bounded error on all in-support inputs. 
% %Furthermore, often the quantity of interest is the Bellman error weighted by the inverse density of the behaviour policy~\cite{antos07fitted}, which depends only on the support of the behaviour policy and this error metric is the equal for two policies provided they share the same support.
% Therefore, backing up from all actions that have non-negligible support under the training distribution is sufficient (but not necessary) to prevent error accumulation. Hence, we restrict the set $\Pi$
% %%SL.5.22: Did we define \Pi before? since we cut the set backup operator stuff, now this is much harder to follow. Maybe we can bring it back (but call it something else)?
% of policies used for distribution-constrained backups to the set of policies that are supported on the probable regions of the behaviour policy. That is, $\Pi = \{ \pi | \pi( a | s) = 0 \text{ whenever } \beta( a | s) < \epsilon \}$, where $\beta$ is the behavior policy (i.e., the set of policies that have support in the probable regions of the behavior policy). This means that we are allowed to backup from any action distribution supported over the support of the behaviour policy. Previous work~\cite{fujimoto2018off} restricts the choice of actions to be a distribution close to the behaviour policy. 

%%SL.5.22: I don't really understand what the above paragraph is saying. Read literally, it seems to say "prior work does something similar, and in the worst case we are equally bad." That's not very satisfying. Maybe just delete this paragraph, or rephrase if that's not what you meant?
%Now, explain why this does a good job of balancing the terms. Next, we explain how this bound motivates the use of set-constrained backups to reduce accumulation of bootstrapping error. \TODO{explanation about $\delta1$ goes here} -- addressed -- removed this paragraph


% we need to determine how to formulate the appropriate constraint and how to implement so as to back up only values of policies in $\Pi$.
% %%SL.5.20: Rephrase. In order to develop a practical algorithm based on the set-constrained backup, we need to determine how to formulate the appropriate constraint and how to implement so as to back up only values of policies in $\Pi$.
% Intuitively, we would like $\Pi(s)$ for a particular state $s$ to contain only those policies that permit actions within the support of the dataset distribution. Instead of inferring $\Pi$, we use a notion of divergence between the uniform distribution over the support-set of the current policy and the current policy for optimization.  

% %%SL.5.20: Rephrase. Intuitively, we would like $\Pi(s)$ for a particular state $s$ to contain only those policies that permit actions withi
% In order confidence support set perform the $\max$ on the high-over actions from only these policies, we need to define a tractable objective. Instead of inferring the set of policies $\Pi$ we rather resort to specifying a notion of divergence between the set $\mathcal{A}_\varepsilon^\dataset$ and the current policy, $\operatorname{Divergence}(\mathcal{A}^{\mathcal{D}}_{\varepsilon}(s), \pi)$ thereby fitting the problem of inferring $\Pi$ in an optimization setup.
% %%SL.5.20: I don't really understand the above sentence. Try rewriting it to be clearer?
% Next, we move on to presenting our method, which we call \emph{bootstrap error accumulation reduction} (BEAR).


%%%%%%%%%%%%%%%%%%%%%%%%% OLD: Monday 7:30pm%%%%%%%%%%%%%%%
 % \subsection{Reducing Error Propagation via Set-Constrained Backups}
% Next, we explain how this bound motivates the use of set-constrained backups to reduce accumulation of errors. Suppose we are in an off-policy setting training from a static dataset. Let us denote the set of states and actions within the support of the data as $\mathcal{B}_1$ and the set of states outside of support as $\mathcal{B}_2 = (\mathcal{S} \times \mathcal{A}) - \mathcal{B}_1$. 

% When running Q-learning methods, we typically use supervised learning to minimize Bellman error - stanard supervised learning methods will allow us to control the expected loss on $\mathcal{B}_1$, but not on states over $\mathcal{B}_2$. Suppose that we incur a maximum error of $\delta_1$ on in-distribution states within $\mathcal{B}_1$ and a maximum error of $\delta_2$ on out-of-distribution states within $\mathcal{B}_2$.

% % Set-constrained backups will only be successful at error reduction if the overall error incurred by using set-constrained backups is lower in magnitude than the error incurred when using unrestricted backups. We therefore compare bootstrapping error and error incurred due to the set-constrained backups next.

% %Note that intermediate steps in AVI are all supervised learning problems. When using high-capacity function approximators for these supervised learning problems, we are better able to control the projection error $\delta_k$ induced due to bootstrapping from state-action pairs which lie in the high-confidence support set of the training distribution,  whereas the errors incurred due to backups from state-action pairs outside of this distribution are largely unbounded.
% %%SL.5.20: I noticed that nowhere in this analysis do you actually introduce any notation for the training distribution, and instead you use this set notation $\data$ all the time. As I wrote above, this is very problematic. If one of our central claims is that our analysis treats distributions better than Fujimoto, this isn't going to fly. Can you introduce notation for the training *distribution* (i.e., p(something)) and use that throughout?
% %This is because we can, in principle, overfit
% %%SL.5.20: overfit is the wrong word
% %to the state-action pairs lying in this set
% %%SL.5.20: stop using set
% %with powerful function approximators. However, when we bootstrap from actions that are not present 
% %in the high-confidence support of the train distribution, we back up values that can't be controlled.
% %%SL.5.20: use a different term than "can't be controlled" (we back up values that may have unbounded error)
% %To formalize this, let $\mathcal{A}^{\mathcal{D}}_{\varepsilon}(s) \defeq \{ a \in \mathcal{A} |~ \exists \pi \in \Pi,~ \pi(a|s) \geq \varepsilon \}$ 
% %\TODO{sometimes $\Pi$ is a set, sometimes it's a policy}
% %denote the set of action samples
% %%SL.5.20: the set of actions for state $s$? (maybe also remind the reader why we only worry about actions and not states)
% %that will most likely be sampled from the policies in the set $\Pi$. Let the error incurred in the value estimate at states $s$ reached when backing up from $s'$ using action $a \in \mathcal{A}^{\mathcal{D}}_{\varepsilon}(s')$ be $\delta^1(s)$ and let $\delta^2(s)$ be the error when backing up from $s'$ by using $a \notin \mathcal{A}^{\mathcal{D}}_{\varepsilon}(s')$, we can reasonably expect $\delta^1 < \delta^2$. 
% %%SL.5.20: be consistent with ', i.e., use a' for action in s' and a for action in s

% Using set-constrained AVI will allow us to recover an error bound of (using $\norm{\cdot}_{\infty,\mathcal{B}}$ to denote the maximum over a set $\mathcal{B}$):
%  $\lim_{k \to \infty} \norm{V_k - V^*}_{\infty, \mathcal{B}_1} \le \frac{2\gamma }{(1-\gamma)^2}\left(\max_{s \in , k} \delta^1_k(s) + \alpha(\Pi,s)\right) $
% In contrast, naively running unconstrained AVI will guarantee a bound of:
% $\lim_{k \to \infty} \norminf{V_k - V^*} \le \frac{2\gamma }{(1-\gamma)^2}\left(\max_{s, k} \delta^2_k(s)\right)$. 
% %\TODO{again, does this hold w/ fixed dataset?}
% %%SL.5.20: Can you state this as a theorem with a proof in the appendix? Generally tidying up the above paragraph would be a good idea, this is a very important paragraph and it's currently rather messy.

% Therefore, set-constrained algorithms enable us to accumulate error at the more favorable $\delta^1$ rate
% %$\mathcal{S} \times \mathcal{A}^{\mathcal{D}}_{\varepsilon}$,
% %$\mathcal{S} \times \mathcal{A} -\mathcal{S} \times \mathcal{A}^{\mathcal{D}}_{\varepsilon}$
% , at the cost of introducing additional suboptimality bias. Thus, it is beneficial to use set-constrained backups when $\delta^1 + \alpha < \delta^2$. This is a complex trade-off, which depends on the training data, the underlying MDP, and the function approximator. However, in general $\delta^2$ can be arbitrarily high and exceptionally difficult to control, especially with the use of function approximators such as neural networks, as it uses function-approximation outputs at \emph{state-action pairs at which we have little to no data}. 
% In addition, when our training data is close to optimal, we can expect a small suboptimality bias. Thus, we expect set-constrained algorithms to provide significantly better error bounds many practical scenarios.
% %%SL.5.20: Generally a good idea for several of the other coauthors to take a pass over the previous two paragraphs and tighten them up, deleting anything that is unnecessary. These two paragraphs are crucial for the paper to make sense, and esp the last para is currently way too long-winded.


% % %%% SEEMS TOO SUDDENLY COMING
% % In practice, we are better able to control the projection error $\delta$ on states within our training data than those outside. To formalize this tradeoff, assume we partition the state-action space into two sets, $\mathcal{B}_1$, the in-data set, and $\mathcal{B}_2 = (\mathcal{S} \times \mathcal{A}) - \mathcal{B}_1$, the out-of-data set. We assume that we can control projection error as $\delta^1$ on states within $\mathcal{B}_1$ and $\delta^2$ on $\mathcal{B}_2$, with $\delta^1 < \delta^2$.



% %Justin, it is better if you use this version of algorithm for error propagation 
% %$$
% %\begin{aligned} v_{k}=\left(T_{\pi_{k}}\right)^{m} v_{k-1} & \text { (evaluation step) } \\ \pi_{k+1}=\mathcal{G}\left[\left(T_{\pi_{k}}\right)^{m} v_{k-1}\right] &(\text { greedy step }) \end{aligned}
% %$$

% % Let $\mathcal{A}^{\mathcal{D}}_{\varepsilon}(s) \defeq \{ a \in \mathcal{A} | \pi_{data}(a|s) \geq \varepsilon \}$, and $\mathcal{A}^{\pi_{\theta}}_{\varepsilon}(s) \defeq \{ a \in \mathcal{A} | \pi(a|s) \geq \varepsilon \}$ denote the $\varepsilon$-high confidence support sets of the behaviour policy $\pi_{data}$ and the current actor $\pi$ at state $s$. Actions sampled from $\pi_{data}$ and $\pi$ are most likely to come from these sets respectively, and hence actions from these sets will be used to evaluate expectations or greedy maximum for backing up while performing ADP. If $\operatorname{Divergence}(\mathcal{A}^{\mathcal{D}}_{\varepsilon}(s), \mathcal{A}^{\pi_theta}_{\varepsilon}(s))$ is high, then we end up backing up from actions which haven't been updated, for which our Q-function could give arbitrarily overestimated/underestimated values. Also, note that as there is no implicit normalization mechanism on Q-functions as opposed to policies which are probability distributions and must integrate to 1, there is no check on the values of the Q-function approximator. 

% Now, we pose and answer one key question before presenting our approach to the problem. Given a dataset $\dataset$, how do we constrain our backups to only use policies from $\Pi$?
% %%SL.5.20: Rephrase. In order to develop a practical algorithm based on the set-constrained backup, we need to determine how to formulate the appropriate constraint and how to implement so as to back up only values of policies in $\Pi$.
% The possible candidates for $\Pi$ at a particular state $s$ is the set of policies for which the observed action $a$ lies in the high-confidence support.
% %%SL.5.20: Rephrase. Intuitively, we would like $\Pi(s)$ for a particular state $s$ to contain only those policies that permit actions withi
% In order tconfidence support set $equation$.o perform the $\max$ n the high-over actions from only these policies, we need to define a tractable objective. Instead of inferring the set of policies $\Pi$ we rather resort to specifying a notion of divergence between the set $\mathcal{A}_\varepsilon^\dataset$ and the current policy, $\operatorname{Divergence}(\mathcal{A}^{\mathcal{D}}_{\varepsilon}(s), \pi)$ thereby fitting the problem of inferring $\Pi$ in an optimization setup.
% %%SL.5.20: I don't really understand the above sentence. Try rewriting it to be clearer?
% Next, we move on to presenting our method, which we call \emph{bootstrap error accumulation reduction} (BEAR). 


% Section~\ref{sec:set_constrained_backup} presents a quantitative argument for the restricting the action distribution as an approach to solving the out-of-distribution actions problem in ADP by invoking tools from error propagation. In this section, extend that analysis to motivate the design for a practical algorithm for bootstrapping error reduction. We start by noting there is no direct way that the \TODO{(for Justin): Can we refer to the error-prop stuff under this name -- "abstract error model"} presented above can be instantiated in practice, as it involves the quantities $\delta^1$ and $\delta^2$ which are intractable and can only be measured in retrospect. Moreover, the choice of function approximator strongly affects these errors and theoretical properties of neural net function approximators are not fully understood. However, we can use the error propagation to motivate two design choices for the practical algorithm -- namely, (1) constraining to the support of the dataset action-distribution, (2) correcting the policy improvement step in addition to the policy evaluation step.
% %AK. (Question to Sergey:) Do you think (2) is relevent enough to be mentioned?
 
% Firstly, our definition of BEAR-backup restricts the max-backup to a subset of policies $\Pi$. In practical situations, the set of policies $\Pi$ is observed in the form of an action sample $a$ for each state $s$. Possible candidates for $\pi \in \Pi$ thus include policies for which $a$ belongs to the high-support region. \TODO{complete thia argument} 
% Alongside this, both steps of approximate policy iteration -- evaluation and improvement -- are supervised learning problems in themselves, being solved in practice by using powerful function-approximators such that in principle it is possible to model outputs for high-support datapoints of the training distribution to a high-enough degree of precision which prevents error propagation and accumulation from the set of all the high-confidence support actions. Hence, the out-of-distribution actions problem, in practice, manifests as an out-of-high-confidence support problem - i.e. usage of actions that are less likely than a particular chosen threshold, say, $\varepsilon$ under the training distribution for backups can accumulate a lot of error, but actions show high-enough support are good irrespective of the exact proportions of their appearance in the $\dataset$. 
% %AK.05.15: Question to Sergey: What do you think about the above paragraph? Does it seem too vague, and what's a better way of writing it?
% To formalize this notion, let $\mathcal{A}^{\mathcal{D}}_{\varepsilon}(s) \defeq \{ a \in \mathcal{A} | \Pi(a|s) \geq \varepsilon \}$ denote the $\varepsilon$-high confidence support set of the behaviour policy $\Pi$. Actions sampled from $\Pi$ are most likely to belong to these sets respectively, and hence actions from these sets will be used to evaluate expectations or greedy maximum for backing up while performing ADP. If $\operatorname{Divergence}(\mathcal{A}^{\mathcal{D}}_{\varepsilon}(s), \pi_\theta)$ is high, then we end up backing up from actions which haven't been updated, for which our Q-function could output arbitrarily overestimated/underestimated values.

%%%%%%%%%%%%%%%%%%%%%%%%%%%%%%%% COMMENT HERE
%AK.5.15: Note to Sergey: currently this section is written assuming theory corresponding to approximate policy iteration and not approximate value iteration as is currently present in Section 4.
% Secondly, the two components of $\delta^2$, $\epsilon_k$ (error arising due to approximate policy evaluation), and $\epsilon'_k$ (error arising due to approximate greedy maximization) can be analysed in worst case analysis. If we assume that the dataset $\mathcal{D}$ consists of $N$ samples, and the VC-dimension~\cite{} of the policy class $\Pi$ is denoted bby $h$, then the worst-case error accumulation (apart from intrinsic bellman error) in $\epsilon_k$ is $\mathcal{O}\big(V_{max} \sqrt{\frac{\log (N/\delta) }{N} \big)}$, whereas worst case $\epsilon'_k$ is $\mathcal{O} \big( V_{max} \sqrt{\log (N/\delta)} + V_{max} \sqrt{\frac{h \log (N/h) }{N}} \big)$. For more details, we refer the reader to Lemmas 11 and 12 in \cite{bruno2015approximate}. These bounds suggest that the worst case error incurred in the greedy maximization step is higher than the evaluation step for sufficiently rich function class of policies. Hence, in the worst case, errors in policy improvement can compound fast.
%%%%%%%%%%%%%%%%%%%%%%%%%%%%%%%%%%%%%%%%%%%%%%%%%

% From a practical algorithmic standpoint, the discussion above suggests
% that the actor update in the policy-improvement step be made more constrained, especially to the distribution $\Pi$ with less error compounding then. This also naturally leads to backups with restricted action distributions. Secondly, in practice, the algorithm should always stay in the high confidence support of the set of policies that generated the data. We use these insights to develop an algorithm which we present next.

% Let us revisit the error bounds from the abstract error model for the case of Classification Based Modified Policy Iteration (CBMPI)(\cite{bruno2015approximate}) -- which has a similar structure as modern Actor-Critic algorithms. Theorem 8 in \cite{bruno2015approximate} quantifies the error in the approximate policy iteration scheme defined by the abstract model. We revisit the theorem to gain insights about the problem. Let $\epsilon_k$ and $\epsilon'_k$ be instantaneous worst-case errors incurred in the evaluation step and greedy maximization step respectively. That is,
% $\epsilon_k \defeq v_k -\left(T_{\pi_{k}}\right)^{m} v_{k-1}$ and $\epsilon'_k \defeq \max_{\pi'} T_{\pi'} v_{k-1} - T_{\pi_k} v_{k-1}$. Then, the $\rho$-weighted $p$-th norm of the overall error, $v_* - v_{\pi_k}$ satisfies:
% \begin{multline*}
%     \|v* - v_{\pi_k}\|_{p, \rho} \leq 2 \gamma^{m} \sum_{i=1}^{k-2} \frac{\gamma^{i}}{1-\gamma}\left(\mathcal{C}_{q}^{i, i+1, m}\right)^{\frac{1}{p}}\left\|\epsilon_{k-i-1}\right\|_{p q^{\prime}, \mu}+\sum_{i=0}^{k-1} \frac{\gamma^{i}}{1-\gamma}\left(\mathcal{C}_{q}^{i, i+1,0}\right)^{\frac{1}{p}}\left\|\epsilon_{k-i}^{\prime}\right\|_{p q^{\prime}, \mu}+g(k)
% \end{multline*}

% where $\mathcal{C}_{q}^{l, k, m}$ is a concentrability coefficient and is a function of the MDP, with the property that $\mathcal{C}_{q}^{l, k, m} \geq \mathcal{C}_{q}^{l', k, m}$ for $l \leq l'$. (For more details we refer the reader to \cite{bruno2015approximate}). The concentrability coefficient is a function of the MDP, and hence cannot be controlled. It can therefore be argued that the major contribution in this error term comes from the second term, due to $\epsilon'$, as $m \geq 1$. This implies that imperfect policy improvement step is a major component of bootstrapping error. In modern deep RL settings, this means that optimization of the actor/policy towards regions with imperfect Q-values can be disastrous for the algorithm. Hence, one logical starting point for our approach is to constrain the set of actions we backup from. 

% \section{Error Propagation in Actor-Critic vs Q-Learning Algorithms}
% AVI:
% $$
% \left\|V^{*}-V^{\pi_{K}}\right\|_{p, \rho} \leq \frac{2 \gamma}{(1-\gamma)^{2}}\left[\inf _{r \in[0,1]} C_{V I, \rho, \nu}^{\frac{1}{2 p}}(K ; r) \mathcal{E}^{\frac{1}{2 p}}\left(\varepsilon_{0}, \ldots, \varepsilon_{K-1} ; r\right)+\frac{2}{1-\gamma} \gamma^{\frac{K}{p}} R_{\max }\right]
% $$

% API:
% $$
% \left\|Q^{*}-Q^{\pi_{K}}\right\|_{p, \rho} \leq \frac{2 \gamma}{(1-\gamma)^{2}}\left[\inf _{r \in[0,1]} C_{P(B R A E), \rho, \nu}^{\frac{1}{2 p}}(K ; r) \mathcal{E}^{\frac{1}{2 p}}\left(\varepsilon_{0}, \ldots, \varepsilon_{K-1} ; r\right)+\gamma^{\frac{K}{p}-1} R_{\max }\right]
% $$


\section{Bootstrapping Error Accumulation Reduction (BEAR)}
\label{sec:bear}

We now propose a practical actor-critic style algorithm that uses distribution-constrained backups
%%SL.5.22: Did you define set-constrained backups? I think probably the best thing would be to simply bring back the definition.
to reduce accumulation of bootstrapping error.
Our model has two main components. We use ensembles of Q-functions to provide a conservative estimate of the Q-function which is used in policy improvement, and a constraint which will be used for searching over the set of policies $\Pi$, which share the same support as the behaviour policy. Both of these components will appear as modifications of the policy improvement step in actor-critic style algorithms.

We use an ensemble of Q-functions $\hat{Q}_1, \cdots, \hat{Q}_K$ to compute a conservative estimate of the Q-values: $\frac{1}{K} \sum_{i=1}^K \hat{Q}_i (s, a) - \lambda \sqrt{\operatorname{var}_k \hat{Q}_k(s, a)}$, where $\lambda \in \mathbb{R}^+$ is a hyperparameter. %We use this value as a conservative estimate of the Q-function. This can be derived using Cantelli's inequality. 
Then, the policy is updated to maximize the conservative estimate of the Q-values: $$ \pi_{k+1}(s) := \max_{\pi \in \Pi} E_{a \sim \pi(\cdot|s)} [\hat{Q}_{k}(s, a)] - \lambda \sqrt{ \operatorname{var_k}E_{a \sim \pi(\cdot |s) }[\hat{Q}_k(s, a)]}.$$



% Let $\mathcal{F}_t$ be the sigma-algebra generated by the training procedure until iteration $t$, and let $\operatorname{var}_{t} \hat{Q}(s,a) := \mathbb{E}[(\hat{Q}_t(s, a) - \mathbb{E}[(\hat{Q}_t(s, a) | \mathcal{F}_t))^2|\mathcal{F}_t]$
%%SL.5.20: use mbox. And for clarity, it might be good to indicate what the expectation is over (and use [ instead of ( for E so that parens don't get cluttered). Also, what is up with this (s,a) hanging out at the end? do you mean to put (s,a) inside (after \hat{Q})?
% denote the variance of the Q-function $\hat{Q}_t$, at time $t$ during training. Then, for each state-action pair $(s, a)$, 
% ${Pr (\hat{Q}_t \geq \mathbb{E}(\hat{Q}_t|\mathcal{F}_t) + \sqrt{\frac{(1 - \delta) \operatorname{var}_{t} \hat{Q}_t }{\delta}})  \leq \delta}$
%%SL.5.20: can you state in words what this means for the purpose of this section? also, rhetoric-wise, amybe better state as a theorem (it's kind of obvious, but still) and then after say that this is easy to show via Cantelli's inequality or something?

%%SL.5.20: It's not clear what the concentration bound is actually used for.

 %In the above concentration bound, $\mathbb{E}(\hat{Q}_t|\mathcal{F}_t)$ refers to the true Q-value, which can be obtained given no stochasticity in the procedure.


%%SL.5.20: The logical thread here is broken. What are you doing with set divergence? State the issue first, then th e resolution, else it's hard for the reader to follow.
In practice, the behaviour policy $\beta$ is unknown, so we need an approximate way to constrain $\pi$ to $\Pi$. We define a differentiable constraint that approximately constrains $\pi$ to $\Pi$, and then approximately solve the constrained optimization problem via dual gradient descent.  We use the sampled version of maximum mean discrepancy (MMD)~\cite{gretton2012kernel}
%%SL.5.22: Alg names are not capitalized unless they contain proper nouns, put a space after the words and before open paren (I fixed it above, but this issue happens often, please take this comment into account) -- Thanks for pointing this out!
between the unknown behaviour policy $\beta$ and the actor $\pi$ because it can be estimated based solely on samples from the distributions. Given samples $x_1, \cdots, x_n \sim P$ and $y_1, \cdots, y_m \sim Q$, the sampled MMD between $P$ and $Q$ is given by:\\
$$\operatorname{MMD}^2(\{x_1, \cdots, x_n\}, \{y_1, \cdots, y_m\}) = \frac{1}{n^2} \sum_{i, i'} k(x_i, x_{i'}) - \frac{2}{nm} \sum_{i, j} k(x_i, y_j) + \frac{1}{m^2} \sum_{j, j'} k(y_j, y_{j'}).
$$
Here, $k(\cdot, \cdot)$ is any universal kernel. In our experiments, we find both Laplacian and Gaussian kernels work well.
%As the $\operatorname{MMD}$ distance does not depend on the density function of either distribution, minimizing it using samples is a reasonable proxy for enforcing that $Q$ lies inside the support of $P$. This is because, 
Empirically we find that, in the low sample regime, the sampled MMD between $P$ and $Q$ is similar to the MMD between a uniform distribution over $P$'s support and $Q$ (See Appendix~\ref{} for numerical simulations justifying this approach). %We provide some empirical evidence to justify this choice in the appendix using numerical simulations on gaussian distributions.

% and hence, we parameterize the set $\mathcal{A}^{\mathcal{D}}_{\varepsilon}(s)$ as a distribution $\pi_{set}(a|s)$ such that $\mathcal{A}(s) := \mathcal{A}^{\pi_{set}}_{\varepsilon}(s) := \{a \in \mathcal{A} | \pi_{set}(a|s) \geq \varepsilon \}$, in other words, $\mathcal{A}(s)$ is the high-confidence support set of the distribution $\pi_{set}$, and we train for a parametric $\pi_{set}$.
%%SL.5.20: I don't actually understand at this point what you are doing. Are you optimizing a neural net that denotes \pi_set? or something else?

% \paragraph{Deriving the update:} Let $\hat{Q}_k$ be the Q-function at the k-th step of the algorithm. Actor-critic Q-learning algorithms maintain a parameterized policy, $\pi_k$ that is updated towards the maximizing the Q-function.
% %-- $\pi_{k+1}(s) := \max_{\pi \in \Delta_{|S|}} E_{a \sim \pi(\cdot|s)} [\hat{Q}_{k}(s, a)]$. 
% In order to reduce the number of moving parts, we let the actor in this case serve both its regular function of maximizing the Q-function while also constraining the action distribution close to $\mathcal{A}^\dataset_\varepsilon$, which is the the task of $\pi_{set}$. We use the bound derived on Q-values to update the policy in the direction of maximizing a conservative estimate of the true Q-value -- $$ \pi_{k+1}(s) := \max_{\pi \in \Delta_{|S|}} E_{a \sim \pi(\cdot|s)} [\hat{Q}_{k}(s, a)] - \lambda \sqrt{ \operatorname{var_k}E_{a \sim \pi(\cdot |s) }[\hat{Q}_k(s, a)]}$$
% %TODO{may want to mention that this amounts to subtracting a constant times the std, which sounds reasonable}
% We still need to account for the problem of specifying support divergence. In order to enforce this constraint, we use a measure of support matching between the training distribution $\Pi$ and the policy $\pi(\cdot|s)$, which we choose to be a sampled version of the Maximum Mean Discrepancy(MMD) Distance between $\Pi$ and the actor $\pi$. Sampled MMD distance between two probability distributions $P$ and $Q$ is given by, $\operatorname{MMD}(P, Q)$, where $x_1, \cdots, x_n \sim P$ and $y_1, \cdots, y_m \sim Q$ is given by:\\
% $$\operatorname{MMD}^2(\{x_1, \cdots, x_n\}, \{y_1, \cdots, y_m\}) = \frac{1}{n^2} \sum_{i, i'} k(x_i, x_{i'}) - \frac{2}{nm} \sum_{i, j} k(x_i, y_j) + \frac{1}{m^2} \sum_{j, j'} k(y_j, y_{j'})
% $$
% When the number of samples $n$ is an intermediate number (4-10), the above sampled objective can also be approximately considered as a distance between a uniform distribution over the high confidence support set of the distribution $P$ and the distribution $Q$ -- therefore, if trained perfectly, $Q$ should have the same support as $P$. That is, $\operatorname{MMD}(P, Q)$ is a reasonable proxy for $\operatorname{MMD}(\mathcal{U}(\mathcal{A}_{\varepsilon}(P)), Q)$. 
% %\TODO{what does it mean MMD between a set and distribution}
% The expression for $\operatorname{MMD}$ does not use the density function of either distribution, thereby making it suited as an approximate way of support matching.

Putting it all together, the overall optimization problem in the policy improvement step becomes
\begin{multline}
    \label{eqn:policy_update}
   \pi_{k+1}(s) := \max_{\pi \in \Delta_{|S|}} E_{a \sim \pi(\cdot|s)} [\hat{Q}_{k}(s, a)] - \lambda \sqrt{ \operatorname{var_k}E_{a \sim \pi(\cdot |s) }[\hat{Q}_k(s, a)]}\\
   \text{~~s.t.~~} \mathbb{E}_{s \sim \mathcal{D}} [\operatorname{MMD}(\mathcal{D}(s), \pi(\cdot|s))] \leq \varepsilon
\end{multline}
where $\varepsilon$ is an approximately chosen threshold. We choose a threshold of $\varepsilon=0.05$ in all our experiments. 
% We use an ensemble of $M$ Q-functions, $\{Q_{\theta_i} \}_{i=1}^M$ trained on the same data starting from different initializations for modeling $\operatorname{var}(\hat{Q}|\mathcal{F}_{1:t})$, using sample variance of the ensemble. 
The Algorithm is summarized in Algorithm~\ref{algo:bear_ql}.
% $\operatorname{var}(\hat{Q}_k(s, a)) \approx \frac{1}{M} \sum_{i=1}^{M} (\hat{Q}_{\theta_i, k}(s, a) - \bar{Q}_{\theta, k}(s, a))^2$, where $\bar{Q}_{\theta, k}(s, a) = \frac{1}{M} \sum_{i=1}^{M} \hat{Q}_{\theta_i, k}(s, a)$ is the sample mean of the ensemble. 

%AK.05.15: Note to Sergey: this is the actor-critic version, optional depends on results.
% Another variant of the above approach can be where this single policy improvement step can be decomposed into two decoupled steps -- (1) Learning a policy $\pi_{set}$, whose high-confidence set defines the support set $\mathcal{A}_{\varepsilon}(s)$ at a state $s$, by minimizing the sampling error in $\hat{Q}_k$ and accounting for the deviation from the dataset, and then, (2) Learning to maximize the expected Q-function $\hat{Q}_k$ on this set $\mathcal{A}_{\varepsilon}(s)$, in practice obtained by sampling from $\pi_{set}$. In practice, we found using Equation~\ref{eqn:policy_update} working better than the latter approach and hence, we stick to this formulation for our experiments. The overall algorithm is summarized in Algorithm~\ref{alg:q_learning}, and the actor-critic version is described in Algorithm~\ref{alg:actor_critic}.   

\begin{algorithm}[H]

\small
\caption{BEAR Q-Learning}
\label{alg:q_learning}
\begin{algorithmic}[1]
    \INPUT: Dataset $\mathcal{D}$, target network update rate $\tau$, mini-batch size $N$, sampled actions for MMD $n$, minimum $\lambda$
    \STATE Initialize Q-ensemble $\{Q_{\theta_i} \}_{i=1}^{K}$, actor $\pi_{\phi}$, Lagrange multiplier $\alpha$, target networks $\{ Q_{\theta'_i} \}_{i=1}^K$, and a target actor $\pi_{\phi'}$, with $\phi' \leftarrow \phi, \theta'_i \leftarrow \theta_i$
    \FOR{$t$ in \{1, \dots, N\}}
        \STATE Sample mini-batch of transitions $(s, a, r, s') \sim \mathcal{D}$\\
        \textbf{Q-update:}
            \STATE Sample $p$ action samples, $\{a_i \sim \pi_{\phi'}(\cdot|s')\}_{i=1}^p$
            \STATE Define $y = \max_{a_i} [ \lambda \min_{j=1,..,K} Q_{\theta'_j}(s', a_i) + (1 - \lambda) \max_{j=1,..,K} Q_{\theta'_j}(s', a_i)]$
            \STATE $\forall i, \theta_i \leftarrow \arg \min_{\theta_i} (Q_{\theta_i}(s, a) - (r + \gamma y(s, a)))^2$\\
        \textbf{Policy-update:}
        \STATE Sample actions $\{ \hat{a}_i \sim \pi_{\phi}(\cdot | s) \}_{i=1}^{m}$ and $\{ a_j \sim \mathcal{D}(s)\}_{j=1}^{n}$, $n$ preferably an intermediate integer(1-10)
        \STATE Update $\phi$, $\alpha$ by minimizing Equation~\ref{eqn:policy_update} by using dual gradient descent with Lagrange multiplier $\alpha$
        \STATE \textbf{Update Target Networks: } $\theta'_i \leftarrow \tau \theta_i + (1 - \tau)\theta'_i$; $\phi' \leftarrow \tau \phi + (1 -\tau) \phi'$ 
    \ENDFOR
\end{algorithmic}
\label{algo:bear_ql}
\end{algorithm}

To summarize Algorithm~\ref{algo:bear_ql}: The actor is updated towards maximizing the Q-function while still being forced to remain in the valid search space defined by $\Pi$. The Q-function uses actions sampled from the actor to then perform set-constrained Q-learning, over a reduced set of policies. The maximization step in the actor-update empirically helps, but can be coupled with maximization in Step 5. Similar to \cite{fujimoto2018off} we use a soft-minimum to compute target values for updating Q-functions. Implementation and other details are present in Appendix ?.
%%SL.5.22: Remember to fill this in.

% \begin{algorithm}[H]
% \small
% \caption{BEAR Actor-Critic}
% \label{alg:actor_critic}
% \begin{algorithmic}[1]
%     \INPUT: Dataset $\mathcal{D}$, target network update rate $\tau$, mini-batch size $N$, sampled actions for MMD $n$, minimum $\lambda$, policy gradient clipping constants $\beta_1, \beta_2; \beta_1 \leq \beta_2$, MMD threshold constant $\varepsilon$
%     \STATE Initialize Q-ensemble $\{Q_{\theta_i} \}_{i=1}^{M}$, actor $\pi_{\phi}$, set-determining policy $\pi_{set}$, Lagrange multiplier $\alpha$, target networks $\{ Q_{\theta'_i} \}_{i=1}^M$, and a target actor $\pi_{\phi'}$, with $\phi' \leftarrow \phi, \theta'_i \leftarrow \theta_i$
%     \FOR{$t$ in \{1, \dots, N\}}
%         \STATE Sample mini-batch of transitions $(s, a, r, s') \sim \mathcal{D}$\\
%         \textbf{Q-update:}
%             \STATE Sample $m$ action samples, $\{a_i \sim \pi_{\phi'}(\cdot|s')\}_{i=1}^n$
%             \STATE Define $y = \frac{1}{m} \sum_{a_i} [ \lambda \min_{j=1,..,M} Q_{\theta'_j}(s', a_i) + (1 - \lambda) \max_{j=1,..,M} Q_{\theta'_j}(s', a_i)]$
%             \STATE $\forall i, \theta_i \leftarrow \arg \min_{\theta_i} (Q_{\theta_i}(s, a) - (r + \gamma y))^2$\\
%         \textbf{Set-update and Actor-update:}
%         \STATE Sample actions $A_1(s) \equiv \{ \hat{a}_i \sim \pi_{set}(\cdot | s) \}_{i=1}^{m}$ and $A_2(s) \equiv \{ a_j \sim \mathcal{D}(s)\}_{j=1}^{n}$, $n << m$
%         \STATE Update $\pi_{set}, \alpha$: $$ \pi_{set}, \alpha \leftarrow \arg \min_{\pi_{set}} \max_{\alpha \geq 0} \sqrt{\frac{(1 - \delta) \operatorname{var_k}E_{a \sim \pi_{set}(\cdot |s) }[\hat{Q}_k(s, a)]}{\delta}} + \alpha \mathbb{E}_{s \sim \mathcal{D}} ([\operatorname{MMD}(A_1, A_2)] -  \varepsilon) $$
%         \STATE Update $\phi$ using Importance Sampled Policy Gradient: 
%         $$ \pi_{\phi} \leftarrow  \max_{\pi_{\phi}} \mathbb{E}_{s \sim \mathcal{D}} \mathbb{E}_{a \sim \pi_{set}(\cdot|s)} \Big( \Big[ \frac{\pi_\phi(a|s)}{\pi_{set}(a|s)} \Big]_{\beta_1}^{\beta_2} Q(s, a) \Big)$$
%         \STATE \textbf{Update Target Networks: } $\theta'_i \leftarrow \tau \theta_i + (1 - \tau)\theta'_i$; $\phi' \leftarrow \tau \phi + (1 -\tau) \phi'$ 
%     \ENDFOR
% \end{algorithmic}
% \end{algorithm}


% Let $\bar{Q}(\cdot, \cdot)$ be the delayed target network, and $Q(\cdot, \cdot)$ be the current Q-function. Define $d_i$ be the the TD error for the $i^{th}$ datapoint.
% $$
% d_{i}(Q ; \bar{Q}, \pi)=R_{t}+\gamma \bar{Q}\left(s'_{i}, \pi_{set} \left(s'_i\right)\right)-Q\left(s_{i}, a_{i}\right)

% $$
% Further we define the empirical loss function by
% $$
% \hat{L}_{N}(Q ; \bar{Q}, \pi)=\frac{1}{N} \sum_{t=1}^{N} \frac{d_{t}^{2}(Q ; \bar{Q}, \pi_{set})}{\lambda(\mathcal{A})}
% $$
% where normalization $\lambda{\mathcal{A}}$ is introduced for mathematical convenience. Then, each policy evaluation step can be written as:  

% If we solely backup from actions present in our dataset, there is no way the algorithm can perform better than the policy that collected the data. The capacity of Q-learning and other ADP algorithms to ``stitch'' together performant sub-trajectories is lost. Hence, our method does allow the agent to backup from actions that occur outside the dataset, while still being constrained to not go farther away from the support of $\mathcal{D}$. In principle, a measure of distance from a given dataset can only be obtained using Bayesian Approaches (?). In practice, we use the variance of the ensemble as a measure to approximately quantify closeness to the support set. Our overall approach is described in the next paragraph.




% Our problem setting does not allow any interaction with the environment, and only lets us use the dataset $\mathcal{D}$. Since we see a limited subset of state-action pairs from the environment, the expected estimate of the Q-function conditioned on all training history in our case, $\mathbb{E}(\hat{Q}|\mathcal{F}_t)$, is biased. \TODO{aviral: finish this argument} 

% We train an ensemble of $N$ parametric Q-functions, $Q_{\theta_1}, \cdots, Q_{\theta_N}$ by using bootstrap masks on the data points of the dataset $\mathcal{D}$. This is done to simulate epistemic variance. To make sure that the actions chosen for backing up Q-functions are valid, we learn a set selection policy, $\pi_{set}$ -- a policy that can provide high densities to actions that don't propagate errors.   

\iffalse

\section{Theoretical Analysis of \methodname\ }%and \AliasingProblemName}
\label{sec:analysis}

\textcolor{red}{This section goes away} A major challenge with value based RL methods, especially in the offline setting, is bootstrapping error accumulation, which refers to how sampling error increases when the value function itself is used to produce regression targets for temporal difference~(TD) error minimization.
In this section, we illustrate the benefits of \methodname\ by showing that given a set of features $\phi(\bs, \ba)$, the policy estimation error according to the Q$^\pi$-function learned from these features depends on $\simunnorm(\bs, \ba, \bs'; \phi)$. This indicates that explicitly minimizing $\simunnorm$, as done by \methodname, should result in lower error accumulation and more accurate estimates of policy return, and therefore more effective offline RL.
% In this section, we theoretically analyze the benefits of \methodname, by illustrating that the value of $\simunnorm$ directly affects bootstrapping error accumulation. By controlling $\simunnorm$, \methodname\ can reduce error accumulation and produce more accurate value estimates, thus improving offline RL.
%%SL.2.3: Slightly tweaked the above, but we could make it even mroe explicit, such as: can effectively reduce error accumulation. When combined with conservative value learning methods like CQL~\citep{}, we show that Q-functions learned with \methodname\ lower bound the true Q-function, but represent a tighter bound than those obtained with CQL alone. [or something like that]

%%SL.2.3: If 5.1 is the only subsection in Sec 5, then I don't think we need the subsection heading, we could just make the subsection title be the section title

Our goal is to show that training the Q-function with \methodname, which minimizes unnormalized feature similarities, results in a tighter bound on the true policy return when combined with a conservative offline RL algorithm, such as CQL~\citep{kumar2020conservative}. Intuitively, a tighter bound is attained because \methodname\ reduces the degree to which bootstrapping error accumulates. 
% compounds in temporal difference (bootstrapping) backups with finite datasets.
%%SL.2.3: I wonder if it might help to briefly mention something about bootstrap error before this, by way of motivation -- e.g., something like: A major challenge with TD-based RL, especially in the offline setting, is boootstrap error accumulation, which refers to how error increases when the value function itself is used to produce regression targets for Bellman error minimization.
Since we are primarily interested in the relationship between feature similarities and policy return estimates,
%%SL.2.1: I don't get it -- isn't the whole point to analyze feature learning? How can we analyze a regularizer on the features if the features are constant? Or does "linear function approximation" in this case mean something other than "features are constant"?
%%AK.2.3: Agreed, but I don't think we can analyze feature learning in the first place at all as of now. This is just aiming to obtain a groundtruth estimate of what properties will make learning better. 
%%SL.2.3: OK, after reading this a bit more carefully, what I *think* you meant to say here is more like this: We will show that, given a set of feature $whatever$, the policy estimation error according to the Q-function learned from these features depends on $sim(whatever)$. This indicates that explicitly minimizing this quantity should result in lower error accumulation and more accurate estimates of policy return, and therefore more effective reinforcement learning.
our analysis will focus on the linear function approximation setting, where we will characterize the values of $\simunnorm$ that lead to better estimates.

We assume a notion of realizability on the features $\phi(\bs,\ba)$. While conventional realizability refers to the true Q-function being representable in terms of the provided features, we use a significantly weaker assumption, which we term \emph{conservative realizability}, which only requires that a conservative estimate of the Q-function be realizable under features $\phi(\bs, \ba)$ as is the case with offline RL methods that learn lower-bounds on the value function~(\eg CQL) or pessimistically modify the reward function~\citep{?}.  
\begin{definition}[Conservative realizability]
Assume that the learned Q-function for any policy $\pi$ is parameterized as a linear function of features $\phi$: $Q_\phi(\bs, \ba) = {\bw}_\pi^T \phi(\bs,\ba)$. Then features $\phi(\bs, \ba)$ are said to be conservatively realizable with degree $\alpha$ against the behavior policy $\behavior$ if, there exists a function $f(\pi, \behavior, \phi)$ such that, $\forall~ \pi, \bs, \ba,~ Q^\phi(\bs, \ba) \geq Q_\pi(\bs, \ba) - \alpha f(\pi, \behavior, \phi)(\bs, \ba)$, and f satisfies: $f(\behavior, \behavior, \phi) = 0$, $f(\pi, \behavior, \phi) > 0 ~\forall~ \pi \neq \behavior$.  
\end{definition}

\begin{theorem}[\methodname\ attains tighter bound on return]
\label{thm:return_bound}
Suppose the Q-function is given by $Q_\phi(\bs, \ba) = \bw^T \phi(\bs, \ba)$ and for any policy $\pi \in \Pi$, representation $\phi$ is conservatively realizable. 
Given an offline dataset $\data$ of size $N$, let $\Sigma_\data = \sum_{i=1}^N \phi(\bs_i, \ba_i) \phi(\bs_i, \ba_i)^T$ be the covariance matrix of the dataset features and assume $\sigma_{\min}(\Sigma_\data) := 1/c >0$. In this case, running off-policy evaluation for a given policy $\pi$ via fitted Q-evaluation~\citep{} yields $\hat{\bw}_\pi$ such that the optimal $\bw_\pi$ and $\hat{\bw}^\pi$ satisfy with high probability $\geq 1-\delta$ that
\begin{equation*}
    ||\bw_\pi - \hat{\bw}_\pi||_2 \leq \frac{C_{\delta, \gamma} c}{1 - \gamma c \left(\E_{\bs, \ba, \bs' \sim \data}[\simunnorm(\bs, \ba, \bs')] \right)},
%%AK.2.3: this should technically have an additional term that subtracts the function value f
\end{equation*}
where $C_{\delta, \gamma}$ is a constant depending on $\delta, \gamma$. Thus, the expected policy return under initial state-action distribution $\rho(\bs_0, \ba)$, $\hat{J}_\phi(\pi) := \E_{\bs_0 \sim \rho, \ba \sim \pi}[Q_\phi(\bs, \ba)]$, and the actual policy return, $J(\pi)$, satisfies
\begin{equation*}
    \left\vert \hat{J}_\phi(\pi) - J(\pi) \right\vert \leq ||\Phi_\rho||_2 ||\bw_\pi - \hat{\bw}_\pi||_2,
%%AK.1.31: see jamboard and decide whether to write this in terms of \hat{w}^\pi directly or the return of the policy. 
\end{equation*}
where $\phi_\rho$ denotes the expected feature vector on the initial state distribution $\bs_0 \sim \rho$.
\end{theorem}
This result clearly indicates that $\E_{\bs, \ba, \bs' \sim \mathcal{D}}[\simunnorm(\bs, \ba, \bs')]$ directly controls the amount of error accumulated in the weight vector $\hat{\bw}^\pi$ as a result of bootstrapping. Since $\hat{J}(\pi)$ is closer to $J(\pi)$ when the expected unnormalized similarity is small, optimizing features via \methodname\ to directly minimize this quantity controls the tightness or accuracy of the return estimate $\hat{J}(\pi)$. When applied on top of a conservative offline RL method, such as CQL, which satisfies the conservative realizability condition with $f=$, \methodname\ provides tighter return estimates by reducing the amount of error accumulated during bootstrapping,
%%AK.2.3: fill this in
%%AK.1.31: is it unclear how stable came into the picture?
%%SL.2.1: Yes, there is a missing link between "tighter bound" and "robust and stable" (also, try to avoid the word "robust" as it is easy to misunderstand to mean robust control -- i.e., robust to perturbations)

\fi

\iffalse

\subsection{Why Does \AliasingProblemName\ Happen?}
\label{sec:why_does_aliasing_happen}
%%AK.2.1: this section is not edited yet
%%SL.2.1: It's also unclear how this section relates to the previous one...

Finally, we aim to theoretically understand the cause behind the feature aliasing problem. It is evident from Figure ?? that feature aliasing only persists as long as bootstrapping (i.e., TD error) is used and supervised regression does not exhibit this issue. We will now show that feature aliasing is caused by \emph{compounding} of the implicit regularization of gradient descent towards producing simple ``min-norm'' solutions in non-linear networks when bootstrapping is used.
%%SL.1.26: The above sentence reads kind of backwards, maybe we can ease the reader into this by first explaining what this compounding is, and then what it has to do with feature aliasing
Similar to other works in supervised learning~\citep{savarese2019infinite}, operates in the setting with wide 2-layer ReLU networks.
%%SL.1.26: Above sentence is malformed somehow? Maybe a missing word or typo?
Here, the Q-function is given by:
\begin{equation}
    Q(\bs, \ba) = \sum_{i=1}^{n} w^i_2 \underbrace{\left[ [\bs, \ba]^T \bw^i_1 + b^i_1  \right]_{+}}_{:= \phi^i(\bs, \ba)} +~ b^i_2.
\label{eqn:relu_q}
%%SL.1.26: it's not immediately obvious why "here" the q-function is given by this -- are you just writing the equation for last layer features? it's kind of not clear where this comes from
\end{equation}
The $i$-th dimension of the features $\phi(\bs, \ba)$ is marked in Equation~\ref{eqn:relu_q}. Now, we define the ``min-norm'' solution for bootstrapping that is equivalent to the solution found by gradient descent on the TD error objective on a finite dataset, and the rest of our analysis will characterize properties of this optimization problem with bootstrapping.
%%SL.1.26: I'm actually having a lot of difficulty following the paragraph above or what it is trying to say. Perhaps it would help to add a bit more discussion for why we care about min-norm solutions, and what this has to do with gradient descent and bootstrapping
\begin{definition}[Min-norm solution]
\label{eqn:min_norm}
The min-norm solution for parameters $\bw_2$, $\bw_1$, $b_1$ and $b_2$ of the two-layer ReLU network in a given iteration of TD-learning for off-policy evaluation of a given policy $\pi$ is given by:
\begin{align*}
    \min_{\bw_1, \bw2, b_1, b_2}~~ & ||\bw_2||_1 \\
    \text{s.t.}~~ Q_\theta(\bs_i, \ba_i) &= r_i + \gamma \E_{\pi}[Q_{\bar{\theta}}(\bs'_i, \ba')]~~ \forall~ i \in [\data],\\
    & ||\bw^i_1||_2 = 1~~ \forall~ i \in [1,\cdots, n].
\end{align*}
\end{definition}
%%SL.1.26: this definition comes across as a bit arbitrary. Maybe a bit more context would help. Why is the norm of w_1 equal to 1? I can guess the intuition (I'm guessing you're trying to limit the Lipschitz constant of the features, so that any statement about the norm of the last-layer weights actually means something about the overall function), but this is not really explained, so many readers will see this as pretty ad hoc and wonder what this has to do with reality.
%%AK.1.24: also need to discuss early stopping somewhere -- else the datapoint thing is a constraint, and one could always fit it and something bad will may never happen. Also, check if we need to include the biases b_1 and b_2 in the optimization objective or not? 
Here, we  use $\bar{\theta}$ to denote the target Q-network parameters (which is equal to the original Q-network from the previous iteration but is held fixed). We will denote the features of this target network for a state-action tuple $(\bs, \ba)$ as $\bar{\phi}(\bs, \ba)$.
Having covered the optimization problem, we now prove in Theorem~\ref{thm:bootstrapping} that training with bootstrapping is the primary reason behind the feature aliasing problem. We show this under two separate scenarios, both of which are likely to arise in offline RL settings with TD-learning.
%%SL.1.26: Kind of unclear what any of this has to do with Definition 2
The first scenario arises when the magnitude of the Q-value, $Q_{\theta}(\bs, \ba)$ is large compared to the magnitude of the reward $r$. This is expected in scenarios with completely positive or negative reward functions and limited data, when the Q-function erroneously overestimates or underestimates intermediately during training.
%%SL.1.26: This sounds pretty ad hoc, can you state this more precisely? I think reward is positive is fine, but it looks ad-hoc if you bring it up informally here like this (but it's also not that important I think...)
The second scenario comes into effect when the policy $\pi$ is quite different
%%SL.1.26: again, this comes across as vague and imprecise, can you state this more precisely somehow?
from the behavior policy $\pi_\beta$, where target Q-values are evaluated on unseen state-action pairs, and they are not trained on.
This is common in several OPE problems, offline RL settings with limited data or when standard off-policy RL algorithms are used.
%%SL.1.26: I don't think it's more common in OPE problems, but it's fine to just say: this is often the case in offline RL, since the goal of offline RL methods is to learn policies that improve over the behavior policy, and therefore differ from it [or something like that]
We formally state these conditions as Scenarios~\ref{assumption:magnitude} and \ref{assumption:ood} and show in Theorem~\ref{thm:bootstrapping} that features are aligned under these conditions.

\begin{assumption}[Magnitude difference]
\label{assumption:magnitude}
Without loss of generality,
%%SL.1.26: maybe just bring in this assumption earlier (e.g., in the preliminaries) -- something like, in our analysis we will assume that the reward are always positive. This assumption can be made without loss of generality if the rewards are guaranteed to be finite, because...
assume that the reward function $r(\bs, \ba)$ is always positive. Then, we assume that $\gamma$ is sufficiently close to 1 and the current Q-function is such that 
\begin{equation*}
    Q_\theta = r + \gamma \hat{\gT}^\pi Q_\theta  ~\approx~ Q_\theta = \varepsilon + \gamma \hat{\gT}^\pi Q_\theta,~~~ \varepsilon \sim \gN(0, I).
%%AK.1.24: basically, we want to say that now the reward function doesnt matter, it is just as good as noise. The eps sampled from N(O, I) may appear a bit odd, we need not have it. I just put it to signify that the term is just noise.
\end{equation*}
\end{assumption}
%%SL.1.26: Perhaps for these assumptions, it would be a bit clearer to state the assumption first, then the intuition, instead of putting the intuition first like you did above -- otherwise the intuition coming before the definition makes it sound imprecise and a bit clumsy

\begin{assumption}[OOD actions]
\label{assumption:ood}
Assume that the policy $\pi(\ba|\bs)$ is such that for any state-action tuple $\bs \in \mathcal{D}, \ba \sim \pi(\cdot|\bs)$, $\pi_\beta(\ba|\bs) < \varepsilon$.  
%%SL.1.26: for *any*? meaning all actions have low probability? or just some (i.e., "there exists")? If this is for all actions, that seems like a pretty unusual situation
%%AK.1.24: compute the form for epsilon here.... it will depend on the dataset. Also need to relax the for all condition... can be done when analyzing under early stopping
\end{assumption}

\begin{theorem}
\label{thm:bootstrapping}
Assume at least one of scenarios~\ref{assumption:magnitude} and \ref{assumption:ood} holds. Let the feature representation of the target Q-network $Q_{\bar{\theta}}$ be denoted as $\bar{\phi}$. Then, solving the min-norm problem (Definition~\ref{eqn:min_norm}) produces $\phi$ such that, for all $(\bs, \ba, r, \bs') \in \mathcal{D}$, 
\begin{equation*}
     \simunnorm(\bs, \ba, \bs'; \phi) > \simunnorm(\bs, \ba, \bs'; \bar{\phi}).
\end{equation*}
\end{theorem}
%%SL.1.26: It's not clear to me what the significance of this is for two reasons: First, are we solving the min-norm problem? I guess you were going to argue that this is what we're doing because of bootstrap, but I don't see this made explicit anywhere. Second, why is having the cos^2 increase a bad thing? That does not prove that it will actually be large, or that anything bad will actually happen.
A proof can be found in Appendix ??. This key proof idea is that, whenever the TD error either uses target values from unseen state-action tuples or when the magnitude of Q-values is sufficient to overpower the reward signal, Problem~\ref{eqn:min_norm} is effectively under-constrained. In this case, an optimal solution will align state-action features, as doing this also minimizes $||\bw_2||_1$, which is the objective in Problem~\ref{eqn:min_norm}. Before moving on to our method, which attempts to mitigate the feature aliasing issue, we discuss how these results relate to recent work on lower bounds for offline RL.
%%AK.1.26: one thing that is missing here is that we do not talk about the recursiveness of the argument: since our theory so far depends on two scenarios, we need to show that once aliasing happens one of the two scenarios still remains, so that aliasing happens more again, and so on. But maybe that's hard. What do you guys think?
%%SL.1.26: Yeah, I see your point. I think maybe the bigger issue is that it's not clear why showing that current network aliasing between time steps is worse than target aliasing between time steps is actually equivalent to the original claims, which talk about aliasing between current and target networks. I think the reason is that a ~ beta and a' ~ pi, but this is not made explicit, and I think most readers won't get it. It's also not clear if proving that this increases like this actually proves that it gets bad (it might asymptote), can you make a more asymptotic argument somehow?

\fi

\section{Related Work}

\textbf{Offline RL.} Offline RL~\citep{ernst2005tree, riedmiller2005neural, LangeGR12, levine2020offline} has shown promise in domains such as robotic manipulation~\citep{kalashnikov2018scalable, mandlekar2020iris, Rafailov2020LOMPO,singh2020cog,kalashnikov2021mt}, NLP~\citep{jaques2019way,jaques2020human}, recommender systems \& advertising~\citep{strehl2010learning,garcin2014offline,charles2013counterfactual,theocharous2015ad,thomas2017predictive}, and healthcare~\citep{shortreed2011informing, Wang2018SupervisedRL}. The major challenge in offline RL is distribution shift~\citep{fujimoto2018off,kumar2019stabilizing,kumar2020conservative}, where the learned policy might generate out-of-distribution actions, resulting in erroneous value backups. Prior offline RL methods address this issue by regularizing the learned policy to be ``close`` to the behavior policy~\citep{fujimoto2018off,liu2020provably,jaques2019way,wu2019behavior, zhou2020plas,kumar2019stabilizing,siegel2020keep, peng2019advantage}, through variants of importance sampling~\citep{precup2001off, sutton2016emphatic, LiuSAB19, SwaminathanJ15, nachum2019algaedice}, via uncertainty quantification on Q-values~\citep{agarwal2020optimistic, kumar2019stabilizing, wu2019behavior, levine2020offline}, by learning conservative Q-functions~\citep{kumar2020conservative,kostrikov2021offline}, and with model-based training with a penalty on out-of-distribution states~\citep{kidambi2020morel, yu2020mopo,matsushima2020deployment,argenson2020model,swazinna2020overcoming,Rafailov2020LOMPO,lee2021representation,yu2021combo}. While current benchmarks in offline RL~\citep{fu2020d4rl,gulcehre2020rl} contain datasets that involve multi-task structure, existing offline RL methods do not leverage the shared structure of multiple tasks and instead train each individual task from scratch. In this paper, we exploit the shared structure in the offline multi-task setting and train a general policy that can acquire multiple skills.

%%CF.5.17: I would consider mentioning meta-RL methods somewhere, since they also address multi-task RL and especially since there are some that aren't conflicted I think (e.g. VariBAD, meta-Q-learning). Some of them even reuse data
%%TY.5.21: I think meta-RL methods might be a bit orthogonal since they aim for generalization to new tasks. I can cite some multi-task RL works that are less conflicted.
\textbf{Multi-task RL algorithms.} Multi-task RL algorithms~\citep{wilson2007multi,parisotto2015actor,teh2017distral,espeholt2018impala,hessel2019popart,yu2020gradient, xu2020knowledge, yang2020multi, kalashnikov2021mt,sodhani2021multi}
%%CF.5.17: there are online MTRL methods that are more recent than this. For example, there's one on soft modules from USC or UCSD. You can look at papers that cite PCGrad or meta-world and/or look on google scholar for more.
%%TY.5.21: added several more papers.
focus on solving multiple tasks jointly in an efficient way. While multi-task RL methods seem to provide a promising way to build general-purpose agents~\citep{kalashnikov2021mt}, prior works have observed major challenges in multi-task RL, in particular, the optimization challenge~\citep{hessel2019popart,schaul2019ray,yu2020gradient}.
Beyond the optimization challenge, how to perform effective representation learning via weight sharing is another major challenge in multi-task RL. Prior works have considered distilling per-task policies into a single policy that solves all tasks~\citep{rusu2015policy,teh2017distral,ghosh2017divide,xu2020knowledge}, separate shared and task-specific modules with theoretical guarantees~\citep{d2019sharing}, and incorporating additional supervision~\citep{sodhani2021multi}. Finally, sharing data across tasks emerges as a challenge in multi-task RL, especially in the off-policy setting, as na\"{i}vely sharing data across all tasks turns out to hurt performance in certain scenarios~\citep{kalashnikov2021mt}. Unlike most of these prior works, we focus on the offline setting where the challenges in data sharing are most relevant. Methods that study optimization and representation learning issues are complementary and can be readily combined with our approach.
%%CF.5.17: how does your method & analysis contrast with all of these methods? need to explicitly state what is different. (ie things like - we focus on the offline setting where some of these issues are less severe & just different; we focus on data sharing & methods that look at optimization & representation are complementary, something about the analysis contributing to our understanding in a complementary way, etc)
%%CF.5.17: I also wonder if we should include a comparison that runs only one of these methods to show some evidence that they don't solve the problem.
%%TY.5.21: Added the discussion above. We can also perform the empirical analysis on HIPI to see if it solves the problem.
% We will next survey methods in data sharing in multi-task off-policy RL.

%%CF.5.17: Other papers that do some form of data sharing:
% the REPAINT paper - includes a less naive data sharing approach
% model-based RL methods (e.g. visual foresight, some experiments in Danijar's papers, MBOLD) - these share everything
% Dave Held paper on goal-conditioned RL from images - not sure how they share
% probably other GCRL papers? eg Yevgen's actionable models, distributional planning networks, but probably others that aren't conflicted
%%TY.5.21: The REPAINT paper and model-based RL methods do not seem to study the multi-task problem, which might be less relevant? I added references to more GCRL papers.
%%CF.8.1: The model-based RL papers that I mention above *do* include some multi-task experiments (and share data across tasks, and often don't recompute rewards because often only the model is what is sharing cross-task data.) Also, both visual foresight and MBOLD operate in the fully offline setting.
\textbf{Data sharing in multi-task RL.} Prior works~\citep{andrychowicz2017hindsight,kaelbling1993learning,pong2018temporal,schaul2015universal,eysenbach2020rewriting,li2020generalized,kalashnikov2021mt,chebotar2021actionable} have found it effective to reuse data across tasks by recomputing the rewards of data collected for one task and using such relabeled data for other tasks, which effectively augments the amount of data available for learning each task and boosts performance. These methods perform relabeling either uniformly~\citep{kalashnikov2021mt} or based on metrics such as estimated Q-values~\citep{eysenbach2020rewriting,li2020generalized}, domain knowledge~\citep{kalashnikov2021mt}, the distance to states or images in goal-conditioned settings~\citep{andrychowicz2017hindsight,pong2018temporal,nair2018visual,liu2019competitive,sun2019policy,lin2019reinforcement,huang2019mapping,lynch2020grounding,yang2021bias,chebotar2021actionable}, \arxiv{and metric learning for robust inference in the offline meta-RL setting~\citep{li2019multi}. All of these methods either require online data collection and do not consider data sharing in a fully offline setting, or only consider offline goal-conditioned or meta-RL problems~\citep{chebotar2021actionable,li2019multi}.} \arxiv{While these prior works empirically find that data sharing helps, we believe that our analysis in Section~\ref{sec:analysis} provides the first analytical understanding of why and when data sharing can help in multi-task offline RL and why it hurts in some cases.} 
\arxiv{Specifically, our analysis reveals the effect of distributional shift introduced during data sharing, which is not taken into account by these prior works. Our proposed approach, CDS, tackles the challenge of distributional shift in data sharing by intelligently sharing data across tasks and improves multi-task performance by effectively trading off between the benefits of data sharing and the harms of excessive distributional shift.}
%%SL.8.1: I think this paragraph kind of buries the main point: it makes it sound like we are just (rather naively) extending the ideas from these past papers to the offline multi-task setting, which really undersells the contribution. It's not like we're just doing what they already did but extending beyond goals, we are actually addressing a challenge that these methods did not address (and indeed that they suffer from).
%%TY.8.1: I revised the above paragraph to say that our method addresses the challenge that prior works didn't address.
%%SL.5.15: I think it's important to expand the discussion of prior multi-task RL methods and better cover other methods that aim to understand why multi-task RL is hard, empirically observe that it's hard, and offer various solutions. Right now the above citations seem to focus more or less exclusively on "applications" of multi-task RL, whereas we need to survey prior work on analysis and solutions (maybe in a separate paragraph). This includes things like ray interference, pcgrad, and other papers you can find that cite those or are cited by them
%%TY.5.16: I added a paragraph discussing challenges in multi-task RL and then use the above paragraph to survey relabeling methods in the off-policy setting.


%\section{Experiments}
%\seclabel{experiments}
%\subsection{Amodal completion}
\subsection{Quantitative Evaluation}
\paragraph{Dataset:} For the purpose of amodal bounding box prediction, we need annotations for amodal bounding boxes (unlike visible bounding box annotations present in all standard detection datasets). We use the PASCAL 3D+~\cite{pascal3d} dataset which has approximate 3D models aligned to 12 rigid categories on PASCAL VOC~\cite{pascal-voc-2012} to generate these amodal bounding box annotations. It also contains additional annotations for images from ImageNet \cite{imagenet_cvpr09} for each of these categories (22k instances in total from ImageNet). For example, it has between 4 different models aligned to ``chair'' and 10 aligned to ``cars''. The different models primarily distinguish between subcategories (but might also be redundant). The 3D models in the dataset are first aligned coarsely to the object instances and then further refined using keypoint annotations. As a consequence, they correctly capture the amodal extent of the object and allow us to obtain amodal ground-truth.  We project the 3D model fitted per instance into the image, extract the binary mask of the projection and fit a tight bounding box around it which we treat as our amodal box (\figref{amodal}). We train our amodal box regressors on the detection training set of PASCAL VOC 2012 (\textit{det-train}) and the additional images from ImageNet for these 12 categories which have 3D models aligned in PASCAL 3D+ and test on the detection validation set (\textit{det-val}) from the PASCAL VOC 2012 dataset.

\paragraph{Experiments:} We benchmark our amodal bounding box predictor under two settings - going from ground truth visible bounding boxes to amodal boxes and in a detection setting where we predict amodal bounding boxes from noisy detection boxes. We compare against the baseline of using the modal bounding box itself as the amodal bounding box (\textit{modal bbox}) which is in fact the correct prediction for all untruncated instances. \tableref{gtBboxTable} summarizes our experiments in the former setting where we predict amodal boxes from visible ground truth boxes on various subsets of the dataset and report the mean IoU of our predicted amodal boxes with the ground truth amodal boxes generated from PASCAL 3D+. As expected, we obtain the greatest boost over the baseline for truncated instances. Interestingly, the class agnostic network performs as well the class specific one signaling that occlusion patterns span across classes and one can leverage these similarities to train a generic amodal box regressor. 

To test our amodal box predictor in a noisy setting, we apply it on bounding boxes predicted by the RCNN\cite{girshick2013rich} system from Girshick \etal. We assume a detection be correct if the RCNN bounding box has an IoU $> 0.5$ with the ground truth visible box \textit{and} the predicted amodal bounding box also has an IoU $> 0.5$ with the ground truth amodal box. We calculate the average precision for each class under the above definition of a ``correct'' detection and call it the Amodal $AP$ (or $AP^{am}$). \tableref{detectionTable} presents our $AP^{am}$ results on VOC2012 \textit{det-val}. As we can see again, the class agnostic and class specific systems perform very similarly. The notable improvement is only in a few classes (\eg diningtable and boat) where truncated/occluded instances dominate. Note that we do not rescore the RCNN detections using our amodal predictor and thus our performance is bounded by the detector performance. Moreover, the instances detected correctly by the detector tend to be cleaner ones and thus the baseline (\textit{modal bbox}) of using the detector box output as the amodal box also does reasonably well. Our RCNN detector is based on the VGG16 \cite{simonyan2014very} architecture and has a mean $AP$ of $57.0$ on the 12 rigid categories we consider.

% Latent positives table
\renewcommand{\arraystretch}{1.4}
\setlength{\tabcolsep}{6pt}
\begin{table}[htb!]
\centering
\resizebox{.5\linewidth}{!}{
\begin{tabular}{ccccc}
\toprule
& \textbf{all} & \textbf{trunc/occ} & \textbf{trunc} & \textbf{occ} \tabularnewline
\midrule

\textbf{modal bbox} & 0.66 & 0.59 & 0.52 & 0.64 \tabularnewline
\textbf{class specific} & 0.68 & 0.62 & 0.57 & 0.65 \tabularnewline
\textbf{class agnostic} & 0.68 & 0.62 & 0.56 & 0.65 \tabularnewline

\bottomrule
\end{tabular}}
\caption{Mean IoU of amodal boxes predicted from the visible bounding box on various subsets of the validation set in PASCAL VOC. Here \textit{occ} and \textit{trunc} refer to occluded and truncated instances respectively. The class specific and class agnostic methods refer to our variations of the training the amodal box regressors (see text for details) and modal bbox refers to the baseline of using the visible/modal bounding box itself as the predicted amodal bounding box.}
\tablelabel{gtBboxTable}
\end{table}

% Detection table
\renewcommand{\arraystretch}{1.4}
\begin{table*}[htb!]
\centering
\resizebox{\linewidth}{!}{
\begin{tabular}{ccccccccccccc|c}
\toprule
& \textbf{aero} & \textbf{bike} & \textbf{boat} & \textbf{bottle} & \textbf{bus} & \textbf{car} & \textbf{chair} & \textbf{table} & \textbf{mbike} & \textbf{sofa} & \textbf{train} & \textbf{tv} & \textbf{mean}\tabularnewline
\midrule

\textbf{modal bbox} & 70.0 & 66.2 & 23.9 & 35.1 & 76.4 & 57.7 & 28.9 & 24.2 & 68.3 & 45.8 & 58.1 & 59.6 & 51.2 \tabularnewline
\textbf{class specific} & 69.5 & 67.2 & \textbf{26.9} & 36.0 & \textbf{77.0} & \textbf{61.4} & \textbf{31.4} & 29.2 & \textbf{69.0} & \textbf{49.4} & \textbf{59.3} & 59.5 & \textbf{53.0} \tabularnewline
\textbf{class agnostic} & \textbf{70.0} & \textbf{67.5} & 26.8 & \textbf{36.3} & 76.8 & 61.3 & 31.1 & \textbf{30.9} & 68.9 & 48.4 & 58.6 & \textbf{59.6} & \textbf{53.0} \tabularnewline
%\midrule
% \textbf{RCNN} & 0.72 & 0.7 & 0.36 & 0.38 & 0.78 & 0.65 & 0.35 & 0.36 & 0.72 & 0.52 & 0.71 & %0.6 & 0.57 \tabularnewline
\bottomrule
\end{tabular}}
\caption{$AP^{am}$ for our amodal bounding box predictors on VOC 2012 \textit{det-val}. $AP^{am}$ is defined as the average precision when a detection is assumed to be correct only when both the modal and amodal bounding boxes have IoU $> 0.5$ with their corresponding ground truths.}
\tablelabel{detectionTable}
\end{table*}

Armed with amodal bounding boxes, we now show how we tackle the problem of inferring real world object sizes from images.



\vspace{-10pt}
\section{Discussion}
\label{sec:discussion}
\vspace{-10pt}
% Summary
This paper showed that implicit regularization of TD-updates leads to feature co-adaptation that gives rise to representations that fail to distinguish between consecutive state-action tuples that both appear in a Bellman backup. This regularization effect is exacerbated when out-of-sample %(and not out-of-distribution) 
state-action samples are used for the Bellman backup and it can lead to poor policy performance. We devise an explicit regularizer, \methodname\ that directly counteracts this by penalizing the dot product of the learned representations of the network and yields substantial improvements on stability and performance on a wide range of offline RL problems including robotic manipulation tasks, Atari games and continuous control D4RL tasks. Learning-based machine learning systems can have both positive and negative societal impacts  (e.g., loss of jobs in some areas). These implications also broadly apply to our work as any other work in reinforcement learning.  

% Future work
While DR3 shows substaintial performance improvements, there may be better approaches to mitigate the issues we observed. We believe that understanding the learning dynamics of deep Q-learning and the induced implicit regularization will lead to the development of more robust and stable deep RL algorithms. Furthermore, this understanding can help us to predict the instability issues in value-based RL methods in advance, which can inspire cross-validation and model selection strategies, an important, open challenge in offline RL~\citep{fu2021benchmarks}. 
% \textbf{Limitations:} 
% While DR3 improves performance on a number of tasks, it is not clear if it is the best possible approach to mitigate the issues we observe. Answering this question requires a deep understanding of the learning dynamics of Q-learning. 
% \textbf{Societal Impacts:} 
% Learning-based machine learning systems can have both positive (e.g., positive economic impact) and negative societal impacts  (e.g., loss of jobs in some areas). These implications also broadly apply to our work as any other work in reinforcement learning, and are largely not unique to this work.     

\bibliography{reference}{}
\bibliographystyle{plainnat}

%%%%%%%%%%%%%%%%%%%%%%%%%%%%%%%%%%%%%%%%%%%%%%%%%%%%%%%%%%%%
\section*{Checklist}

%%%% BEGIN INSTRUCTIONS %%%
%The checklist follows the references.  Please
%read the checklist guidelines carefully for information on how to answer these
%questions.  For each question, change the default \answerTODO{} to \answerYes{},
%\answerNo{}, or \answerNA{}.  You are strongly encouraged to include a {\bf
%justification to your answer}, either by referencing the appropriate section of
%your paper or providing a brief inline description.  For example:
%\begin{itemize}
%  \item Did you include the license to the code and datasets? \answerYes{See Section~\ref{gen_inst}.}
%  \item Did you include the license to the code and datasets? \answerNo{The code and the data are proprietary.}
%  \item Did you include the license to the code and datasets? \answerNA{}
%\end{itemize}
%Please do not modify the questions and only use the provided macros for your
%answers.  Note that the Checklist section does not count towards the page
%limit.  In your paper, please delete this instructions block and only keep the
%Checklist section heading above along with the questions/answers below.
%%%% END INSTRUCTIONS %%%

\begin{enumerate}

\item For all authors...
\begin{enumerate}
  \item Do the main claims made in the abstract and introduction accurately reflect the paper's contributions and scope?
    \answerYes{}
  \item Did you describe the limitations of your work?
    \answerYes{Please see Section~\ref{sec:discussion}.}
  \item Did you discuss any potential negative societal impacts of your work?
    \answerYes{Please see Section~\ref{sec:discussion}.}
  \item Have you read the ethics review guidelines and ensured that your paper conforms to them?
    \answerYes{}
\end{enumerate}

\item If you are including theoretical results...
\begin{enumerate}
  \item Did you state the full set of assumptions of all theoretical results?
    \answerYes{Assumptions are described in the main text and in Appendix~\ref{app:proofs}.}
	\item Did you include complete proofs of all theoretical results?
    \answerYes{Please see Appendix~\ref{app:proofs}.}
\end{enumerate}

\item If you ran experiments...
\begin{enumerate}
  \item Did you include the code, data, and instructions needed to reproduce the main experimental results (either in the supplemental material or as a URL)?
    \answerYes{}
  \item Did you specify all the training details (e.g., data splits, hyperparameters, how they were chosen)?
    \answer{Yes, please see Appendix~\ref{app:method_details}.}
	\item Did you report error bars (e.g., with respect to the random seed after running experiments multiple times)?
    \answerYes{}
	\item Did you include the total amount of compute and the type of resources used (e.g., type of GPUs, internal cluster, or cloud provider)?
    \answerYes{Please see Appendix~\ref{app:method_details}.}
\end{enumerate}

\item If you are using existing assets (e.g., code, data, models) or curating/releasing new assets...
\begin{enumerate}
  \item If your work uses existing assets, did you cite the creators?
    \answerYes{We cited the creators of datasets that we use.}
  \item Did you mention the license of the assets?
    \answerYes{Please check appendix~\ref{app:method_details}.}
  \item Did you include any new assets either in the supplemental material or as a URL?
    \answerNo{}
  \item Did you discuss whether and how consent was obtained from people whose data you're using/curating?
    \answerYes{The data we are using is open-sourced, so we could directly use it.}
  \item Did you discuss whether the data you are using/curating contains personally identifiable information or offensive content?
    \answerYes{Our data does not have personally identifiable information or offensive content.}
\end{enumerate}

\item If you used crowdsourcing or conducted research with human subjects...
\begin{enumerate}
  \item Did you include the full text of instructions given to participants and screenshots, if applicable?
    \answerNA{}
  \item Did you describe any potential participant risks, with links to Institutional Review Board (IRB) approvals, if applicable?
    \answerNA{}
  \item Did you include the estimated hourly wage paid to participants and the total amount spent on participant compensation?
    \answerNA{}
\end{enumerate}

\end{enumerate}

%%%%%%%%%%%%%%%%%%%%%%%%%%%%%%%%%%%%%%%%%%%%%%%%%%%%%%%%%%%%

%%%%%%%%%%%%%%%%%%%%%%%%%%%%%%%%%%%%%%%%%%%%%%%%%%%%%%%%%%%%%%%%%%%%%%%%%%%%%%%
%%%%%%%%%%%%%%%%%%%%%%%%%%%%%%%%%%%%%%%%%%%%%%%%%%%%%%%%%%%%%%%%%%%%%%%%%%%%%%%
% DELETE THIS PART. DO NOT PLACE CONTENT AFTER THE REFERENCES!
%%%%%%%%%%%%%%%%%%%%%%%%%%%%%%%%%%%%%%%%%%%%%%%%%%%%%%%%%%%%%%%%%%%%%%%%%%%%%%%
%%%%%%%%%%%%%%%%%%%%%%%%%%%%%%%%%%%%%%%%%%%%%%%%%%%%%%%%%%%%%%%%%%%%%%%%%%%%%%%
\appendix
\newpage
\section{COMBO Proofs from Section~\ref{sec:combo_theory}}
\label{app:combo_proofs}

In this section, we provide proofs for theoretical results in Section~\ref{sec:combo_theory}. Before the proofs, we note that all statements are proven in the case of finite state space (i.e., $|\states| < \infty$) and finite action space (i.e., $|\actions| < \infty$) we define some commonly appearing notation symbols appearing in the proof: 
\begin{itemize}
\vspace{-5pt}
    \item $P_{\mdp}$ and $r_{\mdp}$ (or $P$ and $r$ with no subscript for notational simplicity) denote the dynamics and reward function of the actual MDP $\mdp$
    \vspace{-5pt}
    \item $P_{\mdpbar}$ and $r_{\mdpbar}$ denote the dynamics and reward of the empirical MDP $\mdpbar$ generated from the transitions in the dataset
    \vspace{-5pt}
    \item $P_{\mdphat}$ and $r_{\mdphat}$ denote the dynamics and reward of the MDP induced by the learned model $\mdphat$
\end{itemize}
\vspace{-5pt}
We also assume that whenever the cardinality of a particular state or state-action pair in the offline dataset $\data$, denoted by $|\mathcal{D}(\bs, \mathbf{a})|$, appears in the denominator, we assume it is non-zero. For any non-existent $(\bs, \mathbf{a}) \notin \data$, we can simply set $|\data(\bs, \mathbf{a})|$ to be a small value $< 1$, which prevents any bound from producing trivially $\infty$ values.

\subsection{A Useful Lemma and Its Proof}
\label{app:proof_lemma}

Before proving our main results, we first show that the penalty
term in equation \ref{eqn:combo_iterate} is positive in expectation. Such a positive penalty is important to combat any overestimation that may
arise as a result of using $\bellmanhat$.

\begin{lemma}[(Interpolation Lemma]
\label{thm:line_thm}
For any $f \in [0, 1]$, and any given $\rho(\bs, \mathbf{a}) \in \Delta^{|\states||\actions|}$, let $d_f$ be an f-interpolation of $\rho$ and $\data$, i.e., $d_f(\bs, \mathbf{a}) := f d(\bs, \mathbf{a}) + (1-f) \rho(\bs, \mathbf{a})$. For a given iteration $k$ of Equation~\ref{eqn:combo_iterate}, we restate the definition of the expected penalty under $\rho(\bs, \mathbf{a})$ in Eq.~\ref{eqn:expected_penalty}: 
\begin{equation*}
 \nu(\rho, f) := \E_{\bs, \mathbf{a} \sim \rho(\bs, \mathbf{a})}\left[\frac{\rho(\bs, \mathbf{a}) - d(\bs, \mathbf{a})}{d_f(\bs, \mathbf{a})} \right].
\end{equation*}
Then $\nu(\rho, f)$ satisfies, (1) $\nu(\rho, f) \geq 0,~~ \forall \rho, f$, (2) $\nu(\rho, f)$ is monotonically increasing in $f$ for a fixed $\rho$, and (3) $\nu(\rho, f) = 0$ iff $\forall~ \bs, \mathbf{a}, ~\rho(\bs, \mathbf{a}) = d(\bs, \mathbf{a}) \text{~or~} f = 0$. 
\end{lemma}
\begin{proof}
To prove this lemma, we use algebraic manipulation on the expression for quantity $\nu(\rho, f)$ and show that it is indeed positive and monotonically increasing in $f \in [0, 1]$.
\begin{align}
    \nu(\rho, f) &= \sum_{\bs, \mathbf{a}} \rho(\bs, \mathbf{a}) \left(\frac{\rho(\bs, \mathbf{a}) - d(\bs, \mathbf{a})}{f d(\bs, \mathbf{a}) + (1 - f) \rho(\bs, \mathbf{a})}\right)\nonumber \\
    &= \sum_{\bs, \mathbf{a}} \rho(\bs, \mathbf{a}) \left(\frac{\rho(\bs, \mathbf{a}) - d(\bs, \mathbf{a})}{\rho(\bs, \mathbf{a}) + f ( d(\bs, \mathbf{a}) - \rho(\bs, \mathbf{a}))}\right)\\
    \implies \frac{d \nu(\rho, f)}{d f} &= \sum_{\bs, \mathbf{a}} \rho(\bs, \mathbf{a}) \left(\rho(\bs, \mathbf{a}) - d(\bs, \mathbf{a})\right)^2 \cdot \left(\frac{1}{(\rho(\bs, \mathbf{a}) + f ( d(\bs, \mathbf{a}) - \rho(\bs, \mathbf{a}))}\right)^2 \geq 0\nonumber\\
    &~~~\forall f \in [0, 1].
\end{align}
Since the derivative of $\nu(\rho, f)$ with respect to $f$ is always positive, it is an increasing function of $f$ for a fixed $\rho$, and this proves the second part (2) of the Lemma. Using this property, we can show the part (1) of the Lemma as follows:
\begin{align}
    \forall f \in (0, 1],~ \nu(\rho, f) \geq \nu(\rho, 0) = \sum_{\bs, \mathbf{a}} \rho(\bs, \mathbf{a}) \frac{\rho(\bs, \mathbf{a}) - d(\bs, \mathbf{a})}{\rho(\bs, \mathbf{a})} &= \sum_{\bs, \mathbf{a}} \left( \rho(\bs, \mathbf{a}) - d(\bs, \mathbf{a}) \right)\nonumber\\
    &= 1 - 1 = 0.
\end{align}
Finally, to prove the third part (3) of this Lemma, note that when $f = 0$, $\nu(\rho, f) = 0$ (as shown above), and similarly by setting $\rho(\bs, \mathbf{a}) = d(\bs, \mathbf{a})$ note that we obtain $\nu(\rho, f) = 0$. To prove the only if side of (3), assume that $f \neq 0$ and $\rho(\bs, \mathbf{a}) \neq d(\bs, \mathbf{a})$ and we will show that in this case $\nu(\rho,f) \neq 0$. When $d(\bs, \mathbf{a}) \neq \rho(\bs, \mathbf{a})$, the derivative $\frac{d \nu(\rho,f)}{d f} > 0$ (i.e., strictly positive) and hence the function $\nu(\rho, f)$ is a strictly increasing function of $f$. Thus, in this case, $\nu(\rho, f) > 0 = \nu(\rho, 0)~ \forall f > 0$. Thus we have shown that if $\rho(\bs, \mathbf{a}) \neq d(\bs, \mathbf{a})$ and $f > 0$, $\nu(\rho, f) \neq 0$, which completes our proof for the only if side of (3). 
\end{proof}

\subsection{Proof of Proposition~\ref{thm:lower_bound}}
\label{app:proof_lower_bound}
Before proving this proposition, we provide a bound on the Bellman backup in the empirical MDP, $\bellman_{\mdpbar}$. To do so, we formally define the standard concentration properties of the reward and transition dynamics in the empirical MDP, $\mdpbar$, that we assume so as to prove Proposition~\ref{thm:line_thm}. Following prior work~\citep{osband2017posterior,jaksch2010near,kumar2020conservative}, we assume:
\begin{assumption}
\label{assumption:conc}
    $\forall~ \bs, \mathbf{a} \in \mdp$, the following relationships hold with high probability, $\geq 1 - \delta$
    \begin{equation*}
        |r_{\mdpbar}(\bs, \mathbf{a}) - r(\bs, \mathbf{a})| \leq \frac{C_{r, \delta}}{\sqrt{|\mathcal{D}(\bs, \mathbf{a})|}}, ~~~ ||P_{\mdpbar}(\bs'|\bs, \mathbf{a}) - P(\bs'|\bs, \mathbf{a})||_{1} \leq \frac{C_{P, \delta}}{\sqrt{|\mathcal{D}(\bs, \mathbf{a})|}}.
    \end{equation*}
\end{assumption}
Under this assumption and assuming that the reward function in the MDP, $r(\bs, \mathbf{a})$ is bounded, as $|r(\bs, \mathbf{a})| \leq R_{\max}$, we can bound the difference between the empirical Bellman operator, $\bellman_{\mdpbar}$ and the actual MDP, $\bellman_\mdp$,
\begin{align*}
    \left\vert\left({\bellman_{\mdpbar}}^\policy \hat{Q}^k \right) - \left({\bellman}^\policy_\mdp \hat{Q}^k \right)\right\vert &= \left\vert\left(r_{\mdpbar}(\bs, \mathbf{a}) - r_\mdp(\bs, \mathbf{a})\right)\right.\\
    &\left.+ \gamma \sum_{\bs'} \left({P}_{\mdpbar}(\bs'|\bs, \mathbf{a}) - P_\mdp(\bs'|\bs,\mathbf{a})\right) \E_{\policy(\mathbf{a}'|\bs')}\left[\hat{Q}^k(\bs' , \mathbf{a}')\right]\right\vert\\
    &\leq \left\vert r_{\mdpbar}(\bs, \mathbf{a}) - r_\mdp(\bs, \mathbf{a})\right\vert\\
    &+ \gamma \left\vert \sum_{\bs'} \left({P}_{\mdpbar}(\bs'|\bs, \mathbf{a}) - P_\mdp(\bs'|\bs,\mathbf{a})\right) \E_{\policy(\mathbf{a}'|\bs')}\left[\hat{Q}^k(\bs' , \mathbf{a}')\right]\right\vert\\
    &\leq \frac{C_{r, \delta} + \gamma C_{P, \delta} 2R_{\max} / (1 - \gamma)}{\sqrt{|\mathcal{D}(\bs, \mathbf{a})|}}. 
\end{align*}
Thus the overestimation due to sampling error in the empirical MDP, $\mdpbar$ is bounded as a function of a bigger constant, $C_{r, P, \delta}$ that can be expressed as a function of $C_{r, \delta}$ and $C_{P, \delta}$, and depends on $\delta$ via a $\sqrt{\log (1/\delta)}$ dependency. For the purposes of proving Proposition~\ref{thm:Q_bound}, we assume that:
\begin{equation}
\label{eqn:sampling_error}
    \forall \bs, \mathbf{a}, ~~\left\vert\left({\bellman_{\mdpbar}}^\policy \hat{Q}^k \right) - \left({\bellman}^\policy_\mdp \hat{Q}^k \right)\right\vert  \leq \frac{C_{r, T, \delta} R_{\max}}{(1 - \gamma) \sqrt{|\mathcal{D}(\bs, \mathbf{a})|}}.
\end{equation}

Next, we provide a bound on the error between the bellman backup induced by the learned dynamics model and the learned reward, $\bellman_{\mdphat}$, and the actual Bellman backup, $\bellman_{\mdp}$. To do so, we note that:
\begin{align}
    \left\vert\left({\bellman_{\mdphat}}^\policy \hat{Q}^k \right) - \left({\bellman}^\policy_\mdp \hat{Q}^k \right)\right\vert &= \left\vert\left(r_{\mdphat}(\bs, \mathbf{a}) - r_\mdp(\bs, \mathbf{a})\right)\right.\\
    &\left.+ \gamma \sum_{\bs'} \left({P}_{\mdphat}(\bs'|\bs, \mathbf{a}) - P_\mdp(\bs'|\bs,\mathbf{a})\right) \E_{\policy(\mathbf{a}'|\bs')}\left[\hat{Q}^k(\bs' , \mathbf{a}')\right]\right\vert \nonumber\\ 
    &\leq |r_{\mdphat}(\bs, \mathbf{a}) - r_\mdp(\bs, \mathbf{a})| + \gamma \frac{2 R_{\max}}{1 - \gamma} D(P, P_{\mdphat}),
    \label{eqn:model_error} 
\end{align}
where $D(P, P_{\mdphat})$ is the total-variation divergence between the learned dynamics model and the actual MDP. Now, we show that the asymptotic Q-function learned by COMBO lower-bounds the actual Q-function of any
policy $\pi$ with high probability for a large enough $\beta \geq 0$. We will use Equations~\ref{eqn:sampling_error} and \ref{eqn:model_error} to prove such a result.

\begin{proposition}[Asymptotic lower-bound]
\label{thm:Q_bound}
Let $P^\pi$ denote the Hadamard product of the dynamics $P$ and a given policy $\pi$ in the actual MDP and let $S^\pi := (I - \gamma P^\pi)^{-1}$. Let $D$ denote the total-variation divergence between two probability distributions. For any $\pi(\mathbf{a}|\bs)$, the Q-function obtained by recursively applying Equation~\ref{eqn:combo_iterate}, with $\hat{{\bellman}}^\pi = f \bellman_{\mdpbar}^\pi + (1 - f) \bellman_{\mdphat}^\pi$, with probability at least $1 - \delta$, results in $\hat{Q}^\pi$ that satisfies:
\begin{align*}
    \forall \bs, \mathbf{a},~ \hat{Q}^\pi(\bs, \mathbf{a}) \leq  Q^\pi(\bs, \mathbf{a}) &- \beta \cdot \left[ S^\pi \left[ \frac{\rho - d}{d_f} \right] \right](\bs, \mathbf{a}) + f \left[ S^\pi \left[ \frac{C_{r, T, \delta} R_{\max}}{(1 - \gamma) \sqrt{|\data|}} \right] \right](\bs, \mathbf{a})\\
    +&~ (1 - f) \left[ S^\pi \left[ |r - r_{\mdphat}| + \frac{ 2 \gamma  R_{\max}}{1 - \gamma} D(P, P_{\mdphat}) \right]  \right]\!\! (\bs, \mathbf{a}).
\end{align*}
\end{proposition}
\begin{proof}
We first note that the Bellman backup $\hat{\bellman}^\pi$ induces the following Q-function iterates as per Equation~\ref{eqn:combo_iterate},
\begin{align*}
    \hat{Q}^{k+1}(\bs, \mathbf{a}) &= \left(\hat{\bellman}^\pi \hat{Q}^k\right)(\bs, \mathbf{a}) - \beta \frac{\rho(\bs, \mathbf{a}) - d(\bs, \mathbf{a})}{d_f(\bs, \mathbf{a})}\\
    &=  f \left(\bellman^\pi_{\mdpbar} \hat{Q}^k \right) (\bs, \mathbf{a}) + (1 - f) \left(\bellman^\pi_{\mdphat} \hat{Q}^k \right) (\bs, \mathbf{a}) - \beta \frac{\rho(\bs, \mathbf{a}) - d(\bs, \mathbf{a})}{d_f(\bs, \mathbf{a})}\\
    &= \left(\bellman^\pi \hat{Q}^k\right)(\bs, \mathbf{a}) - \beta \frac{\rho(\bs, \mathbf{a}) - d(\bs, \mathbf{a})}{d_f(\bs, \mathbf{a})} + (1 - f) \left({\bellman_{\mdphat}}^\policy \hat{Q}^k - {\bellman}^\policy \hat{Q}^k \right)(\bs, \mathbf{a})\\
    &+ f  \left({\bellman_{\mdpbar}}^\policy \hat{Q}^k - {\bellman}^\policy \hat{Q}^k \right)(\bs, \mathbf{a})\\
   \forall \bs, \mathbf{a},~ \hat{Q}^{k+1} &\leq \left(\bellman^\pi \hat{Q}^k\right) - \beta \frac{\rho - d}{d_f} + (1 - f) \left[|r_{\mdphat} - r_\mdp| + \frac{2 \gamma R_{\max}}{1 - \gamma} D(P, P_{\mdphat}) \right] + f \frac{C_{r, T, \delta} R_{\max}}{(1 - \gamma) \sqrt{|\data|}} 
\end{align*}
Since the RHS upper bounds the Q-function pointwise for each $(\bs, \mathbf{a})$, the fixed point of the Bellman iteration process will be pointwise smaller than the fixed point of the Q-function found by solving for the RHS via equality. Thus, we get that
\begin{align*}
    \hat{Q}^\pi(\bs, \mathbf{a}) &\leq \underbrace{ S^\pi r_{\mdp}}_{= Q^\pi(\bs, \mathbf{a})} -\beta \left[ S^\pi \left[ \frac{\rho - d}{d_f} \right] \right](\bs, \mathbf{a}) +~ f \left[ S^\pi \left[ \frac{C_{r, T, \delta} R_{\max}}{(1 - \gamma) \sqrt{|\data|}} \right] \right](\bs, \mathbf{a})\\
    &+~ (1 - f) \left[ S^\pi \left[ |r - r_{\mdphat}| + \frac{ 2 \gamma  R_{\max}}{1 - \gamma} D(P, P_{\mdphat}) \right]  \right]\!\! (\bs, \mathbf{a}),  
\end{align*}
which completes the proof of this proposition.
\end{proof}

Next, we use the result and proof technique from Proposition~\ref{thm:Q_bound} to prove Corollary~\ref{thm:lower_bound}, that in expectation under the initial state-distribution, the expected Q-value is indeed a lower-bound. 

\begin{corollary}[Corollary~\ref{thm:lower_bound} restated]
For a sufficiently large $\beta$, we have a lower-bound that
$\E_{\bs \sim \mu_0, \mathbf{a} \sim \policy(\cdot|\bs)}[\hat{Q}^\pi(\bs, \mathbf{a})] \leq \E_{\bs \sim \mu_0, \mathbf{a} \sim \policy(\cdot|\bs)}[Q^\pi(\bs, \mathbf{a})]$, 
where $\mu_0(\bs)$ is the initial state distribution. 
Furthermore, when $\epsilon_{\text{s}}$ is small, such as in the large sample regime; or when the model bias $\epsilon_{\text{m}}$ is small, a small $\beta$ is sufficient along with an appropriate choice of $f$.
\end{corollary}

\begin{proof}
To prove this corollary, we note a slightly different variant of Proposition~\ref{thm:Q_bound}. To observe this, we will deviate from the proof of Proposition~\ref{thm:Q_bound} slightly and will aim to express the inequality using $\bellman_{\mdphat}$, the Bellman operator defined by the learned model and the reward function. Denoting $(I - \gamma P_{\mdphat})^{-1}$ as $S_{\mdphat}^\pi$, doing this will intuitively allow us to obtain $\beta \left(\mu(\bs) \policy(\mathbf{a}|\bs)\right)^T \left(S_{\mdphat}^\pi \left[\frac{\rho - d}{d_f} \right]\right)(\bs, \mathbf{a})$ as the conservative penalty which can be controlled by choosing $\beta$ appropriately so as to nullify the potential overestimation caused due to other terms. Formally,
\begin{align*}
    \hat{Q}^{k+1}(\bs, \mathbf{a}) &= \left(\hat{\bellman}^\pi \hat{Q}^k\right)(\bs, \mathbf{a}) - \beta \frac{\rho(\bs, \mathbf{a}) - d(\bs, \mathbf{a})}{d_f(\bs, \mathbf{a})} = \left(\bellman^\pi_{\mdphat} \hat{Q}^k \right)(\bs, \mathbf{a}) -  \beta \frac{\rho(\bs, \mathbf{a}) - d(\bs, \mathbf{a})}{d_f(\bs, \mathbf{a})}\\
    &+ f \underbrace{\left(\bellman^\pi_{\mdpbar} - \bellman^\pi_{\mdphat} \hat{Q}^k \right)(\bs, \mathbf{a})}_{:= \Delta(\bs, \mathbf{a})}
\end{align*}
By controlling $\Delta(\bs, \mathbf{a})$ using the pointwise triangle inequality:
\begin{equation}
    \forall \bs, \mathbf{a}, ~\left\vert \bellman^\pi_{\mdpbar} \hat{Q}^k - \bellman^\pi_{\mdphat} \hat{Q}^k \right\vert \leq \left\vert \bellman^\pi \hat{Q}^k - \bellman^\pi_{\mdphat} \hat{Q}^k \right\vert + \left\vert \bellman^\pi_{\mdpbar} \hat{Q}^k - \bellman^\pi \hat{Q}^k \right\vert,
\end{equation}
and then iterating the backup $\bellman^\pi_{\mdphat}$ to its fixed point and finally noting that $\rho(\bs, \mathbf{a}) = \left((\mu \cdot \pi)^T S^\pi_{\mdphat}\right)(\bs, \mathbf{a})$, we obtain:
\begin{equation}
    \E_{\mu, \pi}[\hat{Q}^\pi(\bs, \mathbf{a})] \leq \E_{\mu, \pi}[Q^\pi_{\mdphat}(\bs, \mathbf{a})] - \beta~ \E_{\rho(\bs, \mathbf{a})}\left[\frac{\rho(\bs, \mathbf{a}) - d(\bs, \mathbf{a})}{d_f(\bs, \mathbf{a})}\right] + \mathrm{terms~ independent~ of~} \beta.
\end{equation}
%%AK: there is one more term in the equation above, fit it in one line...
The terms marked as ``terms independent of $\beta$'' correspond to the additional positive error terms obtained by iterating $\left\vert \bellman^\pi \hat{Q}^k - \bellman^\pi_{\mdphat} \hat{Q}^k \right\vert$ and $\left\vert \bellman^\pi_{\mdpbar} \hat{Q}^k - \bellman^\pi \hat{Q}^k \right\vert$, which can be bounded similar to the proof of Proposition~\ref{thm:Q_bound} above. Now by replacing the model Q-function, $\E_{\mu, \pi}[Q^\pi_{\mdphat}(\bs, \mathbf{a})]$ with the actual Q-function, $\E_{\mu, \pi}[Q^\pi(\bs, \mathbf{a})]$ and adding an error term corresponding to model error to the bound, we obtain that:
\begin{equation}
\label{eqn:lower_bound_eqn}
    \E_{\mu, \pi}[\hat{Q}^\pi(\bs, \mathbf{a})] \leq \E_{\mu, \pi}[Q^\pi(\bs, \mathbf{a})] + \mathrm{terms~ independent~ of~} \beta - \beta~ \underbrace{\E_{\rho(\bs, \mathbf{a})}\left[\frac{\rho(\bs, \mathbf{a}) - d(\bs, \mathbf{a})}{d_f(\bs, \mathbf{a})}\right]}_{= \nu(\rho, f) > 0}.
\end{equation}
Hence, by choosing $\beta$ large enough, we obtain the desired lower bound guarantee. 
\end{proof}

\begin{remark}[\underline{\textbf{COMBO does not underestimate at every $\bs \in \mathcal{D}$ unlike CQL.}}]
\label{remak:tighter_lower_bound}
Before concluding this section, we discuss how the bound obtained by COMBO (Equation~\ref{eqn:lower_bound_eqn}) is tighter than CQL. CQL learns a Q-function such that the value of the policy under the resulting Q-function lower-bounds the true value function at each state $\bs \in \mathcal{D}$ individually (in the absence of no sampling error), i.e., $\forall \bs \in \mathcal{D}, \hat{V}^\pi_{\text{CQL}}(\bs) \leq V^\pi(\bs)$, whereas the bound in COMBO is only valid in expectation of the value function over the initial state distribution, i.e., $\E_{\bs \sim \mu_0(\bs)}[\hat{V}^\pi_{\text{COMBO}}(\bs)] \leq \E_{\bs \sim \mu_0(\bs)}[V^\pi(\bs)]$, and the value function at a given state may not be a lower-bound. For instance, COMBO can overestimate the value of a state more frequent in the dataset distribution $d(\bs, \mathbf{a})$ but not so frequent in the $\rho(\bs, \mathbf{a})$ marginal distribution of the policy under the learned model $\mdphat$. To see this more formally, note that the expected penalty added in the effective Bellman backup performed by COMBO (Equation~\ref{eqn:combo_iterate}), in expectation under the dataset distribution $d(\bs, \mathbf{a})$, $\widetilde{\nu}(\rho, d, f)$ is actually \textbf{\textit{negative}}:
\begin{align*}
    \widetilde{\nu}(\rho, d, f) = \sum_{\bs, \mathbf{a}} d(\bs, \mathbf{a}) \frac{\rho(\bs, \mathbf{a}) - d(\bs, \mathbf{a})}{d_f(\bs, \mathbf{a})} = - \sum_{\bs, \mathbf{a}} d(\bs, \mathbf{a}) \frac{d(\bs, \mathbf{a}) - \rho(\bs, \mathbf{a})}{f d(\bs, \mathbf{a}) + (1 - f) \rho(\bs, \mathbf{a})} < 0,
\end{align*}
where the final inequality follows via a direct application of the proof of Lemma~\ref{thm:line_thm}. Thus, COMBO actually \emph{overestimates} the values at atleast some states (in the dataset) unlike CQL.   
\end{remark}

\subsection{Proof of Proposition~\ref{prop:less_conservative}}
\label{app:proof_less_conservative}

In this section, we will provide a proof for Proposition~\ref{prop:less_conservative}, and show that the COMBO can be less conservative in terms of the estimated value. To recall, let $\Delta^\pi_\text{COMBO} := \E_{\bs, \mathbf{a} \sim d_{\mdpbar}(\bs), \pi(\mathbf{a}|\bs)}\left[\hat{Q}^\pi(\bs, \mathbf{a} \right]$ and let $\Delta^\pi_\text{CQL} := \E_{\bs, \mathbf{a} \sim d_{\mdpbar}, \pi(\mathbf{a}|\bs)} \left[\hat{Q}^\pi_\text{CQL}(\bs, \mathbf{a}) \right]$. From \citet{kumar2020conservative}, we obtain that $\hat{Q}^\pi_\text{CQL}(\bs, \mathbf{a}) := Q^\pi(\bs, \mathbf{a}) - \beta \frac{\pi(\mathbf{a}|\bs) - \pi_\beta(\mathbf{a}|\bs)}{\pi_\beta(\mathbf{a}|\bs)}$. We shall derive the condition for the real data fraction $f=1$ for COMBO, thus making sure that $d_f(\bs) = d^{\pi_\beta}(\bs)$. To derive the condition when $\Delta^\pi_\text{COMBO} \geq \Delta^\pi_\text{CQL}$, we note the following simplifications:
\begin{align}
    & \Delta^\pi_\text{COMBO} \geq \Delta^\pi_\text{CQL} \\
    \implies & \sum_{\bs, \mathbf{a}} d_{\mdpbar}(\bs) \pi(\mathbf{a}|\bs) \hat{Q}^\pi(\bs, \mathbf{a}) \geq \sum_{\bs, \mathbf{a}} d_{\mdpbar}(\bs) \pi(\mathbf{a}|\bs) \hat{Q}^\pi_\text{CQL}(\bs, \mathbf{a}) \\
    \label{eqn:cql_vs_combo_terms}
    \implies & \beta \sum_{\bs, \mathbf{a}} d_{\mdpbar}(\bs)\pi(\mathbf{a}|\bs) \left( \frac{\rho(\bs, \mathbf{a}) - d^{\pi_\beta}(\bs) \pi_\beta(\mathbf{a}|\bs)}{d^{\pi_\beta}(\bs) \pi_\beta(\mathbf{a}|\bs)} \right) \leq \beta \sum_{\bs, \mathbf{a}} d_{\mdpbar}(\bs)\pi(\mathbf{a}|\bs) \left(\frac{\pi(\mathbf{a}|\bs) - \pi_\beta(\mathbf{a}|\bs)}{\pi_\beta(\mathbf{a}|\bs)} \right).
\end{align}
Now, in the expression on the left-hand side, we add and subtract $d^{\pi_\beta}(\bs) \pi(\mathbf{a}|\bs)$ from the numerator inside the paranthesis.
\begin{align}
    & \sum_{\bs, \mathbf{a}} d_{\mdpbar}(\bs, \mathbf{a}) \left( \frac{\rho(\bs, \mathbf{a}) - d^{\pi_\beta}(\bs) \pi_\beta(\mathbf{a}|\bs)}{d^{\pi_\beta}(\bs) \pi_\beta(\mathbf{a}|\bs)} \right)\\
    &= \sum_{\bs, \mathbf{a}} d_{\mdpbar}(\bs, \mathbf{a}) \left( \frac{\rho(\bs, \mathbf{a}) - d^{\pi_\beta}(\bs) \pi(\mathbf{a}|\bs) + d^{\pi_\beta}(\bs) \pi(\mathbf{a}|\bs) - d^{\pi_\beta}(\bs) \pi_\beta(\mathbf{a}|\bs)}{d^{\pi_\beta}(\bs) \pi_\beta(\mathbf{a}|\bs)} \right)\\
    &= \underbrace{\sum_{\bs, \mathbf{a}} d_{\mdpbar}(\bs, \mathbf{a}) \frac{\pi(\mathbf{a}|\bs) - \pi_\beta(\mathbf{a}|\bs)}{\pi_\beta(\mathbf{a}|\bs)}}_{(1)} + \sum_{\bs, \mathbf{a}} d_{\mdpbar}(\bs, \mathbf{a}) \cdot \frac{\rho(\bs) - d^{\pi_\beta}(\bs)}{d^{\pi_\beta}(\bs)} \cdot \frac{\pi(\mathbf{a}|\bs)}{\pi_\beta(\mathbf{a}|\bs)}
\end{align}
The term marked $(1)$ is identical to the CQL term that appears on the right in Equation~\ref{eqn:cql_vs_combo_terms}. Thus the inequality in Equation~\ref{eqn:cql_vs_combo_terms} is satisfied when the second term above is negative. To show this, first note that $d^{\pi_\beta}(\bs) = d_{\mdpbar}(\bs)$ which results in a cancellation. Finally, re-arranging the second term into expectations gives us the desired result. An analogous condition can be derived when $f \neq 1$, but we omit that derivation as it will be hard to interpret terms appear in the final inequality.

\subsection{Proof of Proposition~\ref{thm:policy_improvement}}
\label{app:proof_policy_improvement}

To prove the policy improvement result in Proposition~\ref{thm:policy_improvement}, we first observe that using Equation~\ref{eqn:combo_iterate} for Bellman backups amounts to finding a policy that maximizes the return of the policy in the a modified ``f-interpolant'' MDP which admits the Bellman backup $\bellmanhat^\pi$, and is induced by a linear interpolation of backups in the empirical MDP $\mdpbar$ and the MDP induced by a dynamics model $\mdphat$ and the return of a policy $\pi$ in this effective f-interpolant MDP is denoted by $J(\mdpbar, \mdphat, f, \pi)$. Alongside this, the return is penalized by the conservative penalty where $\rho^\pi$ denotes the marginal state-action distribution of policy $\pi$ in the learned model $\mdphat$. 
\begin{equation}
    \hat{J}(f, \pi) = J(\mdpbar, \mdphat, f, \pi)  - \beta \frac{\nu(\rho^\pi, f)}{1 - \gamma}.
\label{eqn:penalized_objective}
\end{equation}
We will require bounds on the return of a policy $\pi$ in this f-interpolant MDP, $J(\mdpbar, \mdphat, f, \pi)$, which we first prove separately as Lemma~\ref{lemma:interpolant_regular_bound} below and then move to the proof of Proposition~\ref{thm:policy_improvement}.

\begin{lemma}[Bound on return in f-interpolant MDP]
\label{lemma:interpolant_regular_bound}
For any two MDPs, $\mdp_1$ and $\mdp_2$, with the same state-space, action-space and discount factor, and for a given fraction $f \in [0, 1]$, define the f-interpolant MDP $\mdp_f$ as the MDP on the same state-space, action-space and with the same discount as the MDP with dynamics: $P_{\mdp_f} := f P_{\mdp_1} + (1 - f) P_{\mdp_2}$ and reward function: $r_{\mdp_f} := f r_{\mdp_1} + (1 - f) r_{\mdp_2}$. Then, given any auxiliary MDP, $\mdp$, the return of any policy $\pi$ in $\mdp_f$, $J(\pi, \mdp_f)$, also denoted by $J(\mdp_1, \mdp_2, f, \pi)$, lies in the interval:
\begin{equation*}
    \big[ J(\pi, \mdp) - \alpha,~~ J(\pi, \mdp)+ \alpha \big], \text{~~~~~~~~~~~~where~} \alpha \text{~is given by:~}
\end{equation*}
\begin{align}
    \alpha &= \frac{2 \gamma (1 - f)}{(1 - \gamma)^2} R_{\max} D \left(P_{\mdp_2}, P_{\mdp}\right) + \frac{\gamma f}{1 - \gamma} \left\vert \E_{d^\pi_{\mdp} \pi} \left[ \left(P^\pi_{\mdp} - P^\pi_{\mdp_1}\right) Q^\pi_{\mdp} \right]\right\vert  \nonumber\\
   & + \frac{f}{1 - \gamma} \E_{\bs, \mathbf{a} \sim d^\pi_{\mdp} \pi}[|r_{\mdp_1}(\bs, \mathbf{a}) - r_{\mdp}(\bs, \mathbf{a})|] + \frac{1 - f}{1 - \gamma} \E_{\bs, \mathbf{a} \sim d^\pi_{\mdp} \pi}[|r_{\mdp_2}(\bs, \mathbf{a}) - r_{\mdp}(\bs, \mathbf{a})|].  \label{eqn:alpha_expr}
\end{align}
\end{lemma}
\begin{proof}
To prove this lemma, we note two general inequalities. First, note that for a fixed transition dynamics, say $P$, the return decomposes linearly in the components of the reward as the expected return is linear in the reward function:
\begin{equation*}
    J(P, r_{\mdp_f}) = J(P, f r_{\mdp_1} + (1 - f) r_{\mdp_2}) = f J (P, r_{\mdp_1}) + (1 - f) J(P, r_{\mdp_2}).  
\end{equation*}
As a result, we can bound $J(P, r_{\mdp_f})$ using $J(P, r)$ for a new reward function $r$ of the auxiliary MDP, $\mdp$, as follows
\begin{align*}
     J(P, r_{\mdp_f}) &= J(P, f r_{\mdp_1} + (1 - f) r_{\mdp_2}) = J (P, r + f (r_{\mdp_1} - r) + (1 -f) (r_{\mdp_2} - r)\\
     &= J(P, r) + f J(P, r_{\mdp_1} - r) + (1 - f) J(P, r_{\mdp_2} - r)\\
     &= J(P, r) + \frac{f}{1 - \gamma} \E_{\bs, \mathbf{a} \sim d^\pi_{\mdp}(\bs) \pi(\mathbf{a}|\bs)}\left[ r_{\mdp_1}(\bs, \mathbf{a}) - r(\bs, \mathbf{a}) \right]\\
     &+ \frac{1 - f}{1 - \gamma} \E_{\bs, \mathbf{a} \sim d^\pi_{\mdp}(\bs) \pi(\mathbf{a}|\bs)} \left[ r_{\mdp_2}(\bs, \mathbf{a}) - r(\bs, \mathbf{a}) \right].
\end{align*}
Second, note that for a given reward function, $r$, but a linear combination of dynamics, the following bound holds:
\begin{align*}
    J(P_{\mdp_f}, r) &= J(f P_{\mdp_1} + (1 - f) P_{\mdp_2}, r)\\
    &= J ( P_{\mdp} +  f( P_{\mdp_1} - P_{\mdp}) + (1 - f) (P_{\mdp_2} - P_{\mdp}), r)\\ 
    &= J (P_{\mdp}, r) - \frac{\gamma (1 - f)}{1 - \gamma} \E_{\bs, \mathbf{a} \sim d^\pi_{\mdp}(\bs) \pi(\mathbf{a}|\bs)} \left[ \left(P^\pi_{\mdp_2} - P^\pi_{\mdp}\right) Q^\pi_{\mdp}  \right]\\
    &- \frac{\gamma f}{1 - \gamma} \E_{\bs, \mathbf{a} \sim d^\pi_{\mdp}(\bs) \pi(\mathbf{a}|\bs)} \left[ \left(P^\pi_{\mdp} - P^\pi_{\mdp_1}\right) Q^\pi_{\mdp}  \right]\\
    &\in \left[ J( P_{\mdp}, r) ~\pm~ \left(\frac{\gamma f}{(1 - \gamma)} \left\vert \E_{\bs, \mathbf{a} \sim d^\pi_{\mdp}(\bs) \pi(\mathbf{a}|\bs)}\left[ \left(P^\pi_{\mdp} - P^\pi_{\mdp_1}\right) Q^\pi_{\mdp} \right] \right\vert\right.\right.\\
    &\left.\left.+ \frac{2 \gamma (1 -f) R_{\max}}{(1 - \gamma)^2} D(P_{\mdp_2}, P_{\mdp}) \right) \right].
    % &\in \left[J (P_{\mdp_1}, r) ~~\pm~~ \frac{\gamma (1 -f) R_{\max}}{(1 - \gamma)^2} D(P_{\mdp}, P_{\mdp_2}) ~\pm~ (1 - f) \frac{\gamma}{1 - \gamma}  \E_{\bs, \mathbf{a} \sim d^\pi_{\mdp_1}(\bs) \pi(\mathbf{a}|\bs)} \left[ \left(P^\pi_{\mdp} - P^\pi_{\mdp_1}\right) Q^\pi  \right] \right]. 
\end{align*}
To observe the third equality, we utilize the result on the difference between returns of a policy $\pi$ on two different MDPs, $P_{\mdp_1}$ and $P_{\mdp_f}$ from \citet{ajksbook} (Chapter 2, Lemma 2.2, Simulation Lemma), and additionally incorporate the auxiliary MDP $\mdp$ in the expression via addition and subtraction in the previous (second) step. In the fourth step, we finally bound one term that corresponds to the learned model via the total-variation divergence $D(P_{\mdp_2}, P_{\mdp})$ and the other term corresponding to the empirical MDP $\mdpbar$ is left in its expectation form to be bounded later. 

Using the above bounds on return for reward-mixtures and dynamics-mixtures, proving this lemma is straightforward:
\begin{align*}
    & J(\mdp_1, \mdp_2, f, \pi) := J(P_{\mdp_f}, f r_{\mdp_1} + (1 - f) r_{\mdp_2}) = J(f P_{\mdp_1} + (1 -f) P_{\mdp_2}, r_{\mdp_f})\\
    &\in \left[ J(P_{\mdp_f}, r_{\mdp}) ~\pm\right.\\
    &\left.~ \underbrace{\left(\frac{f}{1 - \gamma} \E_{\bs, \mathbf{a} \sim d^\pi_{\mdp} \pi}[|r_{\mdp_1}(\bs, \mathbf{a}) - r_{\mdp}(\bs, \mathbf{a})|] + \frac{1 - f}{1 - \gamma} \E_{\bs, \mathbf{a} \sim d^\pi_{\mdp} \pi}[|r_{\mdp_2}(\bs, \mathbf{a}) - r_{\mdp}(\bs, \mathbf{a})|] \right)}_{:= \Delta_R} \right],
    % ~\pm~ \left(\frac{2 \gamma f (1 - f)}{(1 - \gamma)^2} R_{\max} D \left(P_{\mdp_2}, P_{\mdp}\right) + \frac{2 \gamma f (1 - f)}{1 - \gamma} \E_{d^\pi_{\mdp_1}} \left\vert \left[ \left(P^\pi_{\mdp} - P^\pi_{\mdp_1}\right) Q^\pi  \right] \right\vert \right) \right],
\end{align*}
where the second step holds via linear decomposition of the return of $\pi$ in $\mdp_f$ with respect to the reward interpolation, and bounding the terms that appear in the reward difference. For convenience, we refer to these offset terms due to the reward as $\Delta_R$. For the final part of this proof, we bound $J(P_{\mdp_f}, r_{\mdp})$ in terms of the return on the actual MDP, $J(P_{\mdp}, r_{\mdp})$, using the inequality proved above that provides intervals for mixture dynamics but a fixed reward function. Thus, the overall bound is given by $J(\pi, \mdp_f) \in [J(\pi, \mdp) - \alpha, J(\pi, \mdp) + \alpha]$, where $\alpha$ is given by:
\begin{align}
\label{eqn:alpha_expr_repeat}
    \alpha = \frac{2 \gamma (1 - f)}{(1 - \gamma)^2} & R_{\max} D \left(P_{\mdp_2}, P_{\mdp}\right) + \frac{\gamma f}{1 - \gamma} \left\vert \E_{d^\pi_{\mdp} \pi} \left[ \left(P^\pi_{\mdp} - P^\pi_{\mdp_1}\right) Q^\pi_{\mdp} \right]\right\vert + \Delta_R.
\end{align}
This concludes the proof of this lemma.
\end{proof}



Finally, we prove Theorem~\ref{thm:policy_improvement} that shows how policy optimization with respect to $\hat{J}(f, \pi)$ affects the performance in the actual MD by using Equation~\ref{eqn:penalized_objective} and building on the  analysis of pure model-free algorithms from \citet{kumar2020conservative}. We restate a more complete statement of the theorem below and present the constants at the end of the proof. 

\begin{theorem}[Formal version of Proposition~\ref{thm:policy_improvement}]
Let $\hat{\pi}_{\text{out}}(\mathbf{a}|\bs)$ be the policy obtained by COMBO.
%%CF: Would be nice to have this definition outside of the theorem so that the theorem is shorter/simpler
Then, the policy ${\pi}_{\text{out}}(\mathbf{a}|\bs)$ is a $\zeta$-safe policy improvement over ${\behavior}$ in the actual MDP $\mdp$, i.e., $J({\pi}_{\text{out}}, \mdp) \geq J({\behavior}, \mdp) - \zeta$, with probability at least $1 - \delta$, where $\zeta$ is given by (where $\rho^\beta(\bs, \mathbf{a}) := d^\behavior_{\mdphat}(\bs, \mathbf{a})$):
\begin{align*}
&\mathcal{O}\left(\frac{\gamma f}{(1 - \gamma)^2}\right) {\left[ \E_{\bs \sim d^{\pi_{\text{out}}}_{\mdp}}\left[ \sqrt{\frac{|\actions|}{|\data(\bs)|} (\mathrm{D}_{\text{CQL}}({\pi}_{\text{out}}, \behavior) + 1)} \right] \right]}\\
&+ \mathcal{O}\left(\frac{\gamma (1 - f)}{(1 - \gamma)^2}\right) {\mathrm{D_{TV}}(P_{\mdp}, P_{\mdphat})} - \beta \frac{\nu(\rho^{\pi_{\text{out}}}, f) - \nu(\rho^\beta, f)}{(1 - \gamma)}.
    % &- \underbrace{\left({J}(\mdpbar, \mdphat, f, \pi) - {J}(\mdpbar, \mdphat, f, \behavior) \right)}_{:= (3),~~ \geq \beta \frac{\nu(\rho, f)}{(1 - \gamma)}} 
\end{align*}
\end{theorem}

\begin{proof}
We first note that since policy improvement is not being performed in the same MDP, $\mdp$ as the f-interpolant MDP, $\mdp_f$, we need to upper and lower bound the amount of improvement occurring in the actual MDP due to the f-interpolant MDP. As a result our first is to relate $J(\pi, \mdp)$ and $J(\pi, \mdp_f) := J(\mdpbar, \mdphat, f, \pi)$ for any given policy $\pi$.

\textbf{Step 1: Bounding the return in the actual MDP due to optimization in the f-interpolant MDP.} By directly applying Lemma~\ref{lemma:interpolant_regular_bound} stated and proved previously, we obtain the following upper and lower-bounds on the return of a policy $\pi$:
\begin{equation*}
    J(\mdpbar, \mdphat, f, \pi) \in \left[ J(\pi, \mdp) - \alpha,~~ J(\pi, \mdp) + \alpha \right],
\end{equation*}
where $\alpha$ is shown in Equation~\ref{eqn:alpha_expr}. As a result, we just need to bound the terms appearing the expression of $\alpha$ to obtain a bound on the return differences. We first note that the terms in the expression for $\alpha$ are of two types: \textbf{(1)} terms that depend only on the reward function differences (captured in $\Delta_R$ in Equation~\ref{eqn:alpha_expr_repeat}), and \textbf{(2)} terms that depend on the dynamics (the other two terms in Equation~\ref{eqn:alpha_expr_repeat}). 

To bound $\Delta_R$, we simply appeal to concentration inequalities on reward (Assumption~\ref{assumption:conc}), and bound $\Delta_R$ as:
\begin{align*}
\Delta_R &:= \frac{f}{1 - \gamma} \E_{\bs, \mathbf{a} \sim d^\pi_{\mdp} \pi}[|r_{\mdp_1}(\bs, \mathbf{a}) - r_{\mdp}(\bs, \mathbf{a})|] + \frac{1 - f}{1 - \gamma} \E_{\bs, \mathbf{a} \sim d^\pi_{\mdp} \pi}[|r_{\mdp_2}(\bs, \mathbf{a}) - r_{\mdp}(\bs, \mathbf{a})|]\\
&\leq \frac{C_{r, \delta}}{1 - \gamma} \E_{\bs, \mathbf{a} \sim d^\pi_{\mdp}\pi} \left[\frac{1}{\sqrt{D(\bs, \mathbf{a})}}\right] + \frac{1}{1 - \gamma} ||R_{\mdp} - R_{\mdphat}|| := \Delta_R^u.
\end{align*}
Note that both of these terms are of the order of $\mathcal{O}(1/ (1 - \gamma))$ and hence they don't figure in the informal bound in Theorem~\ref{thm:policy_improvement} in the main text, as these are dominated by terms that grow quadratically with the horizon.
% First, we use algebraic manipulation to obtain the following decompositionof the difference in the return of $\pi_{\text{out}}$ and $\pi_\beta$ in the actual MDP, $\mdp$:
% \begin{align*}
%     J(\pi_{\text{out}}, \mdp) - J(\behavior, \mdp) &= f \left(J(\pi_{\text{out}}, \mdp) - J(\pi_{\text{out}}, \mdpbar) \right) + (1 - f) \left(J(\pi_{\text{out}}, \mdp) - J(\pi_{\text{out}}, \mdphat) \right)~~~ \text{(a): policy difference}\\
%     &+ f (J(\pi_{\text{out}}, \mdpbar) - J(\behavior, \mdpbar)) + (1 - f) \left(J(\pi_{\text{out}}, \mdphat) - J(\behavior, \mdphat) \right)~~~\text{(b): policy improvement} \\
%     &+ f \left(J(\behavior, \mdpbar) - J(\behavior, \mdp) \right) + (1 - f) \left(J(\behavior, \mdphat) - J(\behavior, \mdp) \right)~~~~~~ \text{(c): behavior difference}
% \end{align*}
% Terms (a) and (c) correspond to a weighted sum of the difference in the return estimates of the policies in the empirical MDP $\mdpbar$ and the actual MDP $\mdp$ and the model-induced MDP $\mdphat$, and the actual MDP $\mdp$. 
To bound the remaining terms in the expression for $\alpha$, we utilize a result directly from \citet{kumar2020conservative} for the empirical MDP, $\mdpbar$, which holds for any policy $\pi(\mathbf{a}|\bs)$, as shown below.
\begin{align*}
   &\frac{\gamma}{(1 - \gamma)} \left\vert \E_{\bs, \mathbf{a} \sim d^\pi_{\mdp}(\bs) \pi(\mathbf{a}|\bs)}\left[ \left(P^\pi_{\mdp} - P^\pi_{\mdp_1}\right) Q^\pi_{\mdp} \right] \right\vert \\
   &\leq \frac{2 \gamma R_{\max} C_{P, \delta}}{(1 - \gamma)^2} \mathbb{E}_{\bs \sim d^{\policy}_{\mdpbar}(\bs)}\left[ \frac{\sqrt{|\mathcal{A}|}}{\sqrt{|\mathcal{D}(\bs)|}} \sqrt{ D_{\text{CQL}}(\policy, \behavior)(\bs) + 1} \right].
    %%AK: technically I think this is a naive result and certainly the MOReL paper was not the first one to come up with this... so unclear if we should be citing it for this result...
\end{align*}

\textbf{Step 2: Incorporate policy improvement in the f-inrerpolant MDP.} Now we incorporate the improvement of policy $\pi_{\text{out}}$ over the policy $\behavior$ on a weighted mixture of $\mdphat$ and $\mdpbar$. In what follows, we derive a lower-bound on this improvement by using the fact that policy $\pi_{\text{out}}$ is obtained by maximizing $\hat{J}(f, \pi)$ from Equation~\ref{eqn:penalized_objective}. As a direct consequence of Equation~\ref{eqn:penalized_objective}, we note that 
\begin{equation}
\label{eqn:improvement_expanded}
    \hat{J}(f, \pi_{\text{out}}) =  J(\mdpbar, \mdphat, f, \pi_{\text{out}}) - \beta \frac{\nu(\rho^\pi, f)}{1 - \gamma} \geq \hat{J}(f, \behavior) =  J(\mdpbar, \mdphat, f, \behavior) - \beta {\frac{\nu(\rho^\beta, f)}{1 - \gamma}}
\end{equation}
% Now, observe that we can both upper and lower-bound $J(\mdpbar, \mdphat, f, \pi)$ in terms of the return of policy $\pi$, individually in each MDP, $\mdpbar$ and $\mdphat$. We state this result more formally in Lemma~\ref{lemma:interpolant_regular_bound}.

% Next, we will use the upper bound on $J(\mdpbar, \mdphat, f, \pi)$ from Lemma~\ref{lemma:interpolant_regular_bound} for policy $\pi = \pi_{\text{out}}$ and a lower-bound on $J(\mdpbar, \mdphat, f, \pi)$ for policy $\pi = \behavior$, in the case when the auxiliary MDP is given by $\mdp$ (the actual MDP) to replace the expressions for $J(\mdpbar, \mdphat, f, \pi_{\text{out}})$ and $J(\mdpbar, \mdphat, f, \behavior)$ in the improvement equation~\ref{eqn:improvement_expanded}. Thus using Lemma~\ref{lemma:interpolant_regular_bound} we obtain the following inequality:
Following \textbf{Step 1}, we will use the upper bound on $J(\mdpbar, \mdphat, f, \pi)$ for policy $\pi = \pi_{\text{out}}$ and a lower-bound on $J(\mdpbar, \mdphat, f, \pi)$ for policy $\pi = \behavior$ and obtain the following inequality:
\begin{align*}
    J(\pi_{\text{out}}, \mdp) - \beta \frac{\nu(\rho^\pi, f)}{1 - \gamma} ~&\geq~ \Big\{ J(\behavior, \mdp) - \beta \frac{\nu(\rho^\beta, f)}{1 - \gamma}
    - \frac{4 \gamma (1 - f) R_{\max}}{(1 - \gamma)^2} D(P_{\mdp}, P_{\mdphat}) \\ 
    &- \underbrace{\frac{2 \gamma f}{(1 - \gamma)}\left\vert\E_{d^{\pi_{\text{out}}}_{\mdp}} \left[ \left(P^{\pi_{\text{out}}}_{\mdp} - P^{\pi_{\text{out}}}_{\mdpbar}\right) Q^{\pi_{\text{out}}}_{\mdp}  \right] \right\vert}_{:= (*)}\nonumber\\
    &- \underbrace{\frac{4 \gamma R_{\max} C_{P, \delta} f}{(1 - \gamma)^2} \E_{\bs \sim d^\behavior_{\mdp}}\left[ \sqrt{\frac{|\actions|}{|\data(\bs)|}}\right]}_{:= (\wedge)} - \Delta_R^u \Big\}.
\end{align*}
The term marked by $(*)$ in the above expression can be upper bounded by the concentration properties of the dynamics as done in Step 1 in this proof: 
\begin{align}
\label{eqn:bound_mdp_mdphat}
    (*) \leq \frac{4 \gamma f C_{P, \delta} R_{\max}}{(1 - \gamma)^2} \mathbb{E}_{\bs \sim d^{{\pi_{\text{out}}}}_{\mdp}(\bs)}\left[ \frac{\sqrt{|\mathcal{A}|}}{\sqrt{|\mathcal{D}(\bs)|}} \sqrt{ D_{\text{CQL}}({\pi_{\text{out}}}, \behavior)(\bs) + 1} \right]. 
\end{align}
Finally, using Equation~\ref{eqn:bound_mdp_mdphat}, we can lower-bound the policy return difference as:
\begin{align*}
\begin{small}
    J(\pi_{\text{out}}, \mdp) - J(\behavior, \mdp) \geq \beta \frac{\nu(\rho^\pi, f)}{1 - \gamma} - \beta \frac{\nu(\rho^\beta, f)}{1 - \gamma} - \frac{4 \gamma (1 -f) R_{\max}}{(1 - \gamma)^2} D(P_{\mdp}, P_{\mdphat}) - (*) - \Delta_R^u.
\end{small}
\end{align*}
Plugging the bounds for terms (a), (b) and (c) in the expression for $\zeta$ where $J(\pi_{\text{out}}, \mdp) - J(\behavior, \mdp) \geq \zeta$, we obtain:
\begin{align}
\zeta &= \left({\frac{4f \gamma R_{\max} C_{P, \delta}}{(1 - \gamma)^2}} \right)\mathbb{E}_{\bs \sim d^{\policy_{\text{out}}}_{\mdp}(\bs)}\left[ \frac{\sqrt{|\mathcal{A}|}}{\sqrt{|\mathcal{D}(\bs)|}} \sqrt{ D_{\text{CQL}}(\policy_{\text{out}}, \behavior)(\bs) + 1} \right]  + (\wedge) - \Delta_R^u \nonumber\\
\label{eqn:zeta_expression}
&~~~~~~~~~~~~+ \frac{4 (1 -f) \gamma R_{\max}}{(1 - \gamma)^2} D(P_{\mdp}, P_{\mdphat}) - \beta \frac{\nu(\rho^\pi, f)}{1 - \gamma} + \beta \frac{\nu(\rho^\beta, f)}{1 - \gamma}.
\end{align}
\end{proof}

\begin{remark}[\underline{\textbf{Interpretation of Proposition~\ref{thm:policy_improvement}}}] 
\label{remark:remark1}
Now we will interpret the theoretical expression for $\zeta$ in Equation~\ref{eqn:zeta_expression}, and discuss the scenarios when it is \emph{negative}. When the expression for $\zeta$ is negative, the policy $\pi_{\text{out}}$ is an improvement over $\behavior$ in the original MDP, $\mdp$. 

\begin{itemize}
    \item First note that we have never used the fact that the learned model $P_{\mdphat}$ is close to the actual MDP, $P_{\mdp}$ on the states visited by the behavior policy $\behavior$ in our analysis. We will use this fact now: in practical scenarios, $\nu(\rho^\beta, f)$ is expected to be smaller than $\nu(\rho^\pi, f)$, since $\nu(\rho^\beta, f)$ is directly controlled by the difference and density ratio of $\rho^\beta(\bs, \mathbf{a})$ and $d(\bs, \mathbf{a})$: $\nu(\rho^\beta, f) \leq \nu(\rho^\beta, f=1) = \sum_{\bs, \mathbf{a}} d^\behavior_{\mdphat}(\bs, \mathbf{a}) \left(d^\behavior_{\mdphat}(\bs, \mathbf{a})/d^\behavior_{\mdpbar}(\bs, \mathbf{a}) - 1\right)^2$ by Lemma~\ref{thm:line_thm} which is expected to be small for the behavior policy $\behavior$ in cases when the behavior policy marginal in the empirical MDP, $d^\behavior_{\mdpbar}(\bs, \mathbf{a})$, is broad. This is a direct consequence of the fact that the learned dynamics integrated with the policy under the learned model: $P_{\mdphat}^\behavior$ is closer to its counterpart in the empirical MDP:  $P_{\mdpbar}^\behavior$ for $\behavior$. Note that this is not true for any other policy besides the behavior policy that performs several counterfactual actions in a rollout and deviates from the data. For such a learned policy $\pi$, we incur an extra error which depends on the importance ratio of policy densities, compounded over the horizon and manifests as the $D_{\mathrm{CQL}}$ term (similar to Equation~\ref{eqn:bound_mdp_mdphat}, or Lemma D.4.1 in \citet{kumar2020conservative}). Thus, in practice, we argue that we are interested in situations where $\nu(\rho^\pi, f) > \nu(\rho^\beta, f)$, in which case by increasing $\beta$, we can make the expression for $\zeta$ in Equation~\ref{eqn:zeta_expression} negative, allowing for policy improvement.
    \item In addition, note that when $f$ is close to 1, the bound reverts to a standard model-free policy improvement bound and when $f$ is close to 0, the bound reverts to a typical model-based policy improvement bound. In scenarios with high sampling error (i.e. smaller $|\mathcal{D}(\bs)|$), if we can learn a good model, i.e., $D(P_{\mdp}, P_{\mdphat})$ is small, we can attain policy improvement better than model-free methods by relying on the learned model by setting $f$ closer to 0. A similar argument can be made in reverse for handling cases when learning an accurate dynamics model is hard. 
\end{itemize}
\end{remark}

% \begin{theorem}[Upper bound on $\nu(\rho, f)$]
% If the distributions $\rho(\bs, \mathbf{a})$ and $d(\bs, \mathbf{a})$ are such that $\sum_{\bs, \mathbf{a}} (\rho(\bs, \mathbf{a}) - d(\bs, \mathbf{a}))^2 \leq \varepsilon$, then the value of $\nu(\rho, f) \leq $. 
% \end{theorem}
% \begin{proof}
% To obtain a bound on $\nu(\rho, f)$, we solve the following optimization problem over $\rho$:
% \begin{align*}
%     \max_{\rho}&~~~ \nu(\rho, f):= \sum_{\bs, \mathbf{a}} \rho(\bs, \mathbf{a}) \frac{\rho(\bs, \mathbf{a}) - d(\bs, \mathbf{a})}{f d (\bs, \mathbf{a}) + (1 - f) \rho(\bs, \mathbf{a})}\\
%     &\text{s.t.}~~ \sum_{\bs, \mathbf{a}} (\rho(\bs, \mathbf{a}) - d(\bs, \mathbf{a}))^2 \leq \varepsilon, ~~ \sum_{\bs, \mathbf{a}} \rho(\bs, \mathbf{a}) = 1, ~~ \rho(\bs, \mathbf{a}) \geq 0.
% \end{align*}
% We first note that for any optimal $\rho=\rho^*$, the objective value is largest for $f = 1$, and thus, solving the above optimization problem for $f=1$ gives an upper bound on the objective value. Converting the problem for $f=1$ to a minimization problem and writing out the Lagrangian for optimization, we obtain:
% \begin{multline}
%     \mathcal{L}(\rho; \lambda, \alpha, \eta) = -\sum_{\bs, \mathbf{a}} d(\bs, \mathbf{a}) \frac{\rho(\bs, \mathbf{a})}{d(\bs, \mathbf{a})}  \left( \frac{\rho(\bs, \mathbf{a})}{d(\bs, \mathbf{a})} - 1 \right) + \lambda \left(\sum_{\bs, \mathbf{a}} d(\bs, \mathbf{a})^2 \left(\frac{\rho(\bs, \mathbf{a})}{d(\bs, \mathbf{a})} - 1 \right)^2 - \varepsilon \right) \\ - \eta \left(\sum_{\bs, \mathbf{a}} d(\bs, \mathbf{a}) \frac{\rho(\bs, \mathbf{a})}{d(\bs, \mathbf{a})} - 1 \right) - \sum_{\bs, \mathbf{a}} \alpha(\bs, \mathbf{a}) \frac{\rho(\bs, \mathbf{a})}{d(\bs, \mathbf{a})}.
% \end{multline}
% Noting the change of variable transformation: $w(\bs, \mathbf{a}) := \frac{\rho(\bs, \mathbf{a})}{d(\bs, \mathbf{a})} - 1$, we obtain the following optimization problem:
% \begin{equation*}
%     \mathcal{L}(w; \lambda, \alpha, \eta) = -\sum_{\bs, \mathbf{a}} d(\bs, \mathbf{a}) w(\bs, \mathbf{a})^2 + \lambda \left( \sum_{\bs, \mathbf{a}} d(\bs, \mathbf{a})^2 w(\bs, \mathbf{a})^2 - \varepsilon \right) - \eta \sum_{\bs, \mathbf{a}} d(\bs, \mathbf{a}) w(\bs, \mathbf{a}) - \sum_{\bs, \mathbf{a}} \alpha(\bs, \mathbf{a}) (w(\bs, \mathbf{a}) + 1).
% \end{equation*}
% Taking the derivative with respect to $w(\bs, \mathbf{a})$ and utilizing KKT conditions we obtain
% \begin{align}
% &- 2 d(\bs, \mathbf{a}) w(\bs, \mathbf{a}) + 2 \lambda d(\bs, \mathbf{a})^2 w(\bs, \mathbf{a}) - \eta d(\bs, \mathbf{a}) - \alpha(\bs, \mathbf{a}) = 0   \label{eq:grad}\\
% &\lambda \left( \sum_{\bs, \mathbf{a}} d(\bs, \mathbf{a})^2 w(\bs, \mathbf{a})^2 - \varepsilon \right) = 0.  \label{eq:slack1}\\
% & \alpha(\bs, \mathbf{a}) (w(\bs, \mathbf{a}) + 1) = 0 ~~ \forall \bs, \mathbf{a}.  \label{eq:slack2}
% \end{align}
% Multiplying Equation~\ref{eq:grad} by $w(\bs, \mathbf{a})$ and adding both LHS and RHS over $(\bs, \mathbf{a})$ we obtain:
% \begin{equation}
%     \label{eq:temp_add}
%     - 2 \sum_{\bs, \mathbf{a}} d(\bs, \mathbf{a}) w(\bs, \mathbf{a})^2 + 2 \underbrace{\lambda \sum_{\bs, \mathbf{a}} d(\bs, \mathbf{a})^2 w(\bs, \mathbf{a})^2}_{= \lambda \varepsilon} - \underbrace{\eta \sum_{\bs, \mathbf{a}} d(\bs, \mathbf{a}) w(\bs, \mathbf{a})}_{= \eta \times 0 = 0} = \sum_{\bs, \mathbf{a}} \alpha(\bs, \mathbf{a}) w(\bs, \mathbf{a}),
% \end{equation}
% and similarly, adding Equation~\ref{eq:grad} over $(\bs, \mathbf{a})$ we get:
% \begin{equation}
%     \label{eq:simple_add}
%     - 2 \underbrace{\sum_{\bs, \mathbf{a}} d(\bs, \mathbf{a}) w(\bs, \mathbf{a})}_{= 0} + 2 \lambda \sum_{\bs, \mathbf{a}} d(\bs, \mathbf{a})^2 w(\bs, \mathbf{a}) - \eta = \sum_{\bs, \mathbf{a}} \alpha(\bs, \mathbf{a}).
% \end{equation}
% Finally, from Equation~\ref{eq:grad}, we get that the value of $w(\bs, \mathbf{a})$ is given by:
% \begin{equation*}
%     w(\bs, \mathbf{a}) = \frac{\eta d(\bs, \mathbf{a}) + \alpha (\bs, \mathbf{a})}{2 \lambda d(\bs, \mathbf{a})^2 - 2 d(\bs, \mathbf{a})}
% \end{equation*}
% Adding Equations~\ref{eq:temp_add} and \ref{eq:simple_add}, we obtain:
% \begin{equation*}
%     2 \sum_{\bs, \mathbf{a}} d(\bs, \mathbf{a}) w(\bs, \mathbf{a}) \left[\lambda d(\bs, \mathbf{a}) - w(\bs, \mathbf{a}) \right] + \lambda \varepsilon - \eta = 0. 
% \end{equation*}
% \end{proof}

\section{Experimental Details for COMBO}
\label{app:details}

In this section, we include all details of our empirical evaluations of COMBO.

\subsection{Practical algorithm implementation details}
\label{app:combo_details}

\paragraph{Model training.}

In the setting where the observation space is low-dimensional, as mentioned in Section~\ref{sec:combo},  we represent the model as a probabilistic neural network that outputs a Gaussian distribution over the next state and reward given the current state and action: $$\widehat{T}_\theta(\bs_{t+1}, r| \bs, \mathbf{a}) = \mathcal{N}(\mu_\theta(\bs_t, \mathbf{a}_t), \Sigma_\theta(\bs_t, \mathbf{a}_t)).$$ We train an ensemble of $7$ such dynamics models following \cite{janner2019mbpo} and pick the best $5$ models based on the validation prediction error on a held-out set that contains $1000$ transitions in the offline dataset $\data$. During model rollouts, we randomly pick one dynamics model from the best $5$ models. Each model in the ensemble is represented as a 4-layer feedforward neural network with $200$ hidden units. For the generalization experiments in Section~\ref{sec:generalization_exps}, we additionally use a two-head architecture to output the mean and variance after the last hidden layer following \cite{yu2020mopo}.

In the image-based setting, we follow \citet{Rafailov2020LOMPO} and use a variational model with the following components:

\begin{gather}
\begin{aligned}
&\text{Image encoder:} && \mathbf{h}_t=E_\theta(\bo_t) \\
&\text{Inference model:} && \bs_t \sim q_\theta(\bs_t|\mathbf{h}_t, \bs_{t-1}, \mathbf{a}_{t-1})\\
&\text{Latent transition model:} &&\bs_t \sim \widehat{T}_\theta(\bs_t| \bs_{t-1}, \mathbf{a}_{t-1})\\
&\text{Reward predictor:} && r_t \sim p_\theta(r_t|\bs_t) \\
&\text{Image decoder:} && \bo_t \sim D_\theta(\bo_t|\bs_t).
\label{eq:latent_model}
\end{aligned}
\end{gather}%

We train the model using the evidence lower bound:

$$\max_{\theta}\sum_{\tau=0}^{T-1}\Big[\mathbb{E}_{q_{\theta}}[\log D_{\theta}(\bo_{\tau+1}|\bs_{\tau+1})]\Big]-\mathbb{E}_{q_{\theta}}\Big[D_{KL}[q_{\theta}(\bo_{\tau+1}, \bs_{\tau+1}|\bs_{\tau}, \mathbf{a}_{\tau})\|\widehat{T}_{\theta_{\tau}}(\bs_{\tau+1}, a_{\tau+1})]\Big]$$

At each step $\tau$ we sample a latent forward model $\widehat{T}_{\theta_{\tau}}$ from a fixed set of $K$ models $[\widehat{T}_{\theta_1},\ldots, \widehat{T}_{\theta_K}]$. For the encoder $E_{\theta}$ we use a convolutional neural network with kernel size 4 and stride 2. For the Walker environment we use 4 layers, while the Door Opening task has 5 layers. The $D_{\theta}$ is a transposed convolutional network with stride 2 and kernel sizes $[5,5,6,6]$ and $[5,5,5,6,6]$ respectively. The inference network has a two-level structure similar to \citet{Hafner2019PlanNet} with a deterministic path using a GRU cell with 256 units and a stochastic path implemented as a conditional diagonal Gaussian with 128 units. We only train an ensemble of stochastic forward models, which are also implemented as conditional diagonal Gaussians.


\paragraph{Policy Optimization.} We sample a batch size of $256$ transitions for the critic and policy learning. We set $f = 0.5$, which means we sample $50\%$ of the batch of transitions from $\data$ and another $50\%$ from $\data_\text{model}$. The equal split between the offline data and the model rollouts strikes the balance between conservatism and generalization in our experiments as shown in our experimental results in Section~\ref{sec:combo_exp}. We represent the Q-networks and policy as 3-layer feedforward neural networks with $256$ hidden units.

For the choice of $\rho(\bs,\mathbf{a})$ in Equation~\ref{eq:implicit_update}, we can obtain the Q-values that lower-bound the true value of the learned policy $\pi$ by setting $\rho(\bs,\mathbf{a}) = d^\policy_{\mdphat} (\bs) \pi(\mathbf{a} | \bs)$. However, as discussed in \cite{kumar2020conservative}, computing $\pi$ by alternating the full off-policy evaluation for the policy $\hat{\pi}^k$ at each iteration $k$ and one step of policy improvement is computationally expensive. Instead, following \cite{kumar2020conservative}, we pick a particular distribution $\psi(\mathbf{a}|\bs)$ that approximates the the policy that maximizes the Q-function at the current iteration and set $\rho(\bs,\mathbf{a}) = d^\policy_{\mdphat} (\bs) \psi(\mathbf{a} | \bs)$. We formulate the new objective as follows:
\begin{small}
\begin{align}
    \hat{Q}^{k+1} \leftarrow& \arg\min_{Q}\beta\left(\E_{\bs \sim d^\policy_{\mdphat} (\bs), \mathbf{a}\sim \psi(\mathbf{a} | \bs)}\!\left[Q(\bs,\mathbf{a})\right]-\E_{\bs, \mathbf{a} \sim \data}\left[Q(\bs,\mathbf{a})\right]\right)\nonumber\\
    &+ \frac{1}{2}\E_{\bs, \mathbf{a}, \bs' \sim d_f}\left[ \left(Q(\bs, \mathbf{a}) - \widehat{\bellman}^\policy\hat{Q}^k(\bs, \mathbf{a}))\right)^2 \right] + \mathcal{R}(\psi),
    \label{eq:combo_update_practical}
\end{align}
\end{small}
where $\mathcal{R}(\psi)$ is a regularizer on $\psi$. In practice, we pick $\mathcal{R}(\psi)$ to be the $-D_\text{KL}(\psi(\mathbf{a}|\bs)\|\text{Unif}(\mathbf{a}))$ and under such a regularization, the first term in Equation~\ref{eq:combo_update_practical} corresponds to computing softmax of the Q-values at any state $\bs$ as follows:
\begin{small}
\begin{align}
    \hat{Q}^{k+1} \leftarrow& \arg\min_{Q}\max_\psi\beta\left(\E_{\bs \sim d^\policy_{\mdphat} (\bs)}\!\left[\log\sum_\mathbf{a} Q(\bs,\mathbf{a})\right]-\E_{\bs, \mathbf{a} \sim \data}\left[Q(\bs,\mathbf{a})\right]\right) \nonumber\\
    &+ \frac{1}{2}\E_{\bs, \mathbf{a}, \bs' \sim d_f}\left[ \left(Q(\bs, \mathbf{a}) - \widehat{\bellman}^\policy\hat{Q}^k(\bs, \mathbf{a}))\right)^2 \right].
    \label{eq:combo_logsumexp}
\end{align}
\end{small}
We estimate the \texttt{log-sum-exp} term in Equation~\ref{eq:combo_logsumexp} by sampling $10$ actions at every state $\bs$ in the batch from a uniform policy $\text{Unif}(\mathbf{a})$ and the current learned policy $\pi(\mathbf{a}|\bs)$ with importance sampling following \cite{kumar2020conservative}.

\subsection{Hyperparameter Selection for COMBO}
\label{app:hyperparameter}

\neurips{In this section, we discuss the hyperparameters that we use for COMBO. In the D4RL and generalization experiments, our method are built upon the implementation of MOPO provided at: \url{https://github.com/tianheyu927/mopo}. The hyperparameters used in COMBO that relates to the backbone RL algorithm SAC such as twin Q-functions and number of gradient steps follow from those used in MOPO with the exception of smaller critic and policy learning rates, which we will discuss below. In the image-based domains, COMBO is built upon LOMPO without any changes to the parameters used there. For the evaluation of COMBO, we follow the evaluation protocol in D4RL~\citep{fu2020d4rl} and a variety of prior offline RL works~\citep{kumar2020conservative,yu2020mopo,kidambi2020morel} and report the normalized score of the smooth undiscounted averaged return over $3$ random seeds for all environments except \texttt{sawyer-door-close} and \texttt{sawyer-door} where we report the average success rate over $3$ random seeds.}

\neurips{We now list the additional hyperparameters as follows.
\begin{itemize}
    \item \textbf{Rollout length $h$.} We perform a short-horizon model rollouts in COMBO similar to \citet{yu2020mopo} and \citet{Rafailov2020LOMPO}. For the D4RL experiments and generalization experiments, we followed the defaults used in MOPO and used $h = 1$ for walker2d and \texttt{sawyer-door-close}, $h=5$ for hopper, halfcheetah and \texttt{halfcheetah-jump}, and $h=25$ for \texttt{ant-angle}. In the image-based domain we used rollout length of $h=5$ for both the the \texttt{walker-walk} and \texttt{sawyer-door-open} environments following the same hyperparameters used in \citet{Rafailov2020LOMPO}.
    \item \textbf{Q-function and policy learning rates.} On state-based domains, we searched over $\{1e-4, 3e-4\}$ for the Q-function learning rate and $\{1e-5, 3e-5, 1e-4\}$ for the policy learning rate. 
    We found that $3e-4$ for the Q-function learning rate (also used previously in \citet{kumar2020conservative}) and $1e-4$ for the policy learning rate (also recommended previously in \citet{kumar2020conservative} for gym domains) work well for almost all domains except that on walker2d where a smaller Q-function learning rate of $1e-4$ and a correspondingly smaller policy learning rate of $1e-5$ works the best. In the image-based domains, we followed the defaults from prior work \citep{Rafailov2020LOMPO} and used $3e-4$ for both the policy and Q-function.
    
    \item \textbf{Conservative coefficient $\beta$.} 
    % As noted in our theoretical results in Lemma~\ref{thm:line_thm}, the amount of conservatism depends on the choice of fraction $f$ and $\rho(\bs, \mathbf{a})$. In principle, we only need to control one of these factors, $\rho$, $f$, $\beta$ to obtain the right degree of conservatism. Since we do not alter $f$ and $\rho(\bs, \mathbf{a})$ for different quality datasets (see Appendix~\ref{app:combo_details} for our choice of $f$; $\rho$ was chosen based on model-prediction error as discussed next) we instead choose values of $\beta$ for different dataset types.
    We searched over $\{0.5, 1.0, 5.0\}$ for $\beta$, which correspond to low conservatism, medium conservatism and high conservatism.  A larger $\beta$ would be desirable in more narrow dataset distributions with lower-coverage of the state-action space that propagates error in a backup whereas a smaller $\beta$ is desirable with diverse dataset distributions. On the D4RL experiments, we found that $\beta = 0.5$ works well for halfcheetah agnostic of dataset quality, while on hopper and walker2d, we found that the more ``narrow'' dataset distributions: medium and medium-expert datasets work best with larger $\beta = 5.0$ whereas more ``diverse'' dataset distributions: random and medium-replay datasets work best with smaller $\beta$ ($\beta = 0.5$ for walker2d and $\beta = 1.0$ for hopper) which is consistent with the intuition. 
    % An intuitive explanation would be that on medium and medium-expert datasets where the data distribution is narrow, we need to be more conservative and hence large $\beta$ while on random and medium-replay datasets where the distribution is diverse and cover most of the state space, we require less conservatism. 
    On generalization experiments, $\beta = 1.0$ works best for all environments. In the image-domains we use $\beta=0.5$ for the medium-replay \texttt{walker-walk} task and and $\beta=1.0$ for all other domains, which again is in accordance with the impact of $\beta$ on performance.
    
    
    \item \textbf{Choice of $\rho(\bs,\mathbf{a})$.} We first decouple $\rho(\bs,\mathbf{a}) = \rho(\bs)\rho(\mathbf{a}|\bs)$ for convenience. As discussed in Appendix~\ref{app:combo_details}, we use $\rho(\mathbf{a}|\bs)$ as the soft-maximum of the Q-values and estimated with \texttt{log-sum-exp}. For $\rho(\bs)$, we searched over $\{d^\policy_{\mdphat}, \rho(\bs)=d_f\}$.  We found that $d^\policy_{\mdphat}$ works better the hopper task in D4RL while $d_f$ is better for the rest of the environments. For the remaining domains, we found $\rho(\bs)=d_f$ works well.
    
    
    \item \textbf{Choice of $\mu(\mathbf{a}|\bs)$.} For the rollout policy $\mu$, we searched $\{\text{Unif}(\mathbf{a}), \pi(\mathbf{a}|\bs)\}$, i.e. the set that contains a random policy and a current learned policy. We found that $\mu(\mathbf{a}|\bs) = \text{Unif}(\mathbf{a})$ works well on the hopper task in D4RL and also in the $\texttt{ant-angle}$ generalization experiment. For the remaining state-based environments, we discovered that $\mu(\mathbf{a}|\bs) = \pi(\mathbf{a}|\bs)$ excels. In the image-based domain, we found that $\mu(\mathbf{a}|\bs) = \text{Unif}(\mathbf{a})$ works well in the \texttt{walker-walk} domain and  $\mu(\mathbf{a}|\bs) = \pi(\mathbf{a}|\bs)$ is better for the \texttt{sawyer-door} environment. 
    % Similar to the choice of $\rho(\bs)$, 
    We observed that
    $\mu(\mathbf{a}|\bs) = \text{Unif}(\mathbf{a})$ behaves less conservatively and is suitable to tasks where dynamics models can be learned fairly precisely.
    \item \textbf{Choice of Backup.} Following CQL~\citep{kumar2020conservative}, we use the standard deterministic backup for COMBO.
    \item \textbf{Choice of $f$.} For the ratio between model rollouts and offline data $f$, we searched $\{0.5, 0.8\}$. We found that $f = 0.8$ works well on the medium and medium-expert in the walker2d task in D4RL. For the remaining tasks, we find $f = 0.5$ works well.
\end{itemize}}

\subsection{Automatic Hyperparameter Selection Rule for COMBO}

It is common in prior work on offline RL to select various hyperparameters using online policy rollouts~\citep{yu2020mopo,kidambi2020morel,argenson2020model,lee2021representation}. Requiring online rollouts to tune hyperparameters contradicts the main aim of offline RL, which is to learn entirely from offline data. Therefore, we attempted to devise an automated rule for tuning important hyperparameters such as $\beta$ and $f$ in a fully offline manner in COMBO. We search over a discrete set of hyperparameters for each task as dicussed above, and use the value of the regularization term $\mathbb{E}_{\mathbf{s}, \mathbf{a} \sim \rho(\mathbf{s},\mathbf{a})}\!\left[Q(\mathbf{s},\mathbf{a})\right]\!-\!\mathbb{E}_{\mathbf{s}, \mathbf{a} \sim \data}\!\left[Q(\mathbf{s},\mathbf{a})\right]$ (shown in Eq.~\ref{eq:implicit_update}) to evaluate the hyperparameters. This automated rule picks the hyperparameter setting which achieves the lowest regularization objective, which indicates that the Q-values on unseen model-predicted state-action tuples are not overestimated.
%%CF.9.30: The ICLR AC also wanted to see a discussion of how this offline selection scheme compared to prior methods for offline selection. Maybe discuss this somewhere? (perhaps in the appendix if space is short?)
%%TY.10.1: I discussed this in Appendix B.2.
%%SL.10.2: I slightly rephrased the paragraph above in a way that hopefully further avoids potential misunderstandings.

Below, we provide additional experimental validation showing the effiacy of this automatic hyperparameter selection rule from above. As shown in Table~\ref{tab:beta_selection},~\ref{tab:mu_selection}, ~\ref{tab:rho_selection} and \ref{tab:f_selection}, the above proposed automatic hyperparameter selection rule is able to pick the hyperparameters $\beta$, $\mu(\mathbf{a}|\bs)$, $\rho(\bs)$ and $f$ and  that correspond to the best policy performance based on the regularization value.

\begin{table}[ht]
    \centering
    \scriptsize
    \resizebox{1.0\textwidth}{!}{\begin{tabular}{l|r|r|r|r|}
    \toprule
    Task & $\beta=0.5$ & $\beta=0.5$ & $\beta=5.0$ & $\beta=5.0$\\
 & performance & regularizer value & performance & regularizer value\\
 \midrule
halfcheetah-medium &  \textbf{54.2}  & \textbf{-778.6}  & 40.8  & -236.8  \\
halfcheetah-medium-replay &  \textbf{55.1} & \textbf{28.9} & 9.3 & 283.9\\ 
halfcheetah-medium-expert & 89.4 & 189.8 & \textbf{90.0}  & \textbf{6.5}\\
hopper-medium      &  75.0  & -740.7  &\textbf{97.2}  & \textbf{-2035.9}\\
hopper-medium-replay & \textbf{89.5} & \textbf{37.7} & 28.3       & 107.2\\
hopper-medium-expert & \textbf{111.1}       & \textbf{-705.6}    & 75.3 &       -64.1\\
walker2d-medium        &  1.9  & 51.5  & \textbf{81.9}  & \textbf{-1991.2}\\
walker2d-medium-replay & \textbf{56.0}       & \textbf{-157.9}    & 27.0       & 53.6\\
walker2d-medium-expert & 10.3       & -788.3    &\textbf{103.3}       & \textbf{-3891.4}\\
    \bottomrule
\end{tabular}}
\caption{\footnotesize We include our automatic hyperparameter selection rule of $\beta$ on a set of representative D4RL environments. We show the policy performance (bold with the higher number) and the regularizer value (bold with the lower number). Lower regularizer value consistently corresponds to the higher policy return, suggesting the effectiveness of our automatic selection rule.}
\label{tab:beta_selection}
\end{table}

\begin{table}[ht]
    \centering
    \scriptsize
    \resizebox{1.0\textwidth}{!}{\begin{tabular}{l|r|r|r|r|}
    \toprule
    Task & $\mu(\mathbf{a}|\bs)=\text{Unif}(\mathbf{a})$ & $\mu(\mathbf{a}|\bs)=\text{Unif}(\mathbf{a})$            &$\mu(\mathbf{a}|\bs)=\pi(\mathbf{a}|\bs)$&$\mu(\mathbf{a}|\bs)=\pi(\mathbf{a}|\bs)$\\
 & performance & regularizer value & performance & regularizer value\\
 \midrule
hopper-medium        & \textbf{97.2}  & \textbf{-2035.9} &  52.6  & -14.9  \\
walker2d-medium        &  7.9  & -106.8  & \textbf{81.9}  & \textbf{-1991.2} \\
    \bottomrule
    \end{tabular}}
    \caption{\footnotesize We include our automatic hyperparameter selection rule of $\mu(\mathbf{a}|\bs)$ on the medium datasets in the hopper and walker2d environments from D4RL. We follow the same convention defined in Table~\ref{tab:beta_selection} and find that our automatic selection rule can effectively select $\mu$ offline.}
    \label{tab:mu_selection}
\end{table}

\begin{table}[ht]
    \centering
    \scriptsize
    \resizebox{0.9\textwidth}{!}{\begin{tabular}{l|r|r|r|r|}
    \toprule
    Task & $\rho(\bs) = d^\pi_{\hat{\mathcal{M}}} $&$\rho(\bs) = d^\pi_{\hat{\mathcal{M}}}$            &$\rho(\bs) = d_f$&$\rho(\bs) = d_f$\\
 & performance & regularizer value & performance & regularizer value\\
 \midrule
hopper-medium        & \textbf{97.2}  & \textbf{-2035.9} &  56.0  & -6.0  \\
walker2d-medium        &  1.8  & 14617.4  & \textbf{81.9}  & \textbf{-1991.2} \\
    \bottomrule
    \end{tabular}}
    \caption{\footnotesize We include our automatic hyperparameter selection rule of $\rho(\bs)$ on the medium datasets in the hopper and walker2d environments from D4RL. We follow the same convention defined in Table~\ref{tab:beta_selection} and find that our automatic selection rule can effectively select $\rho$ offline.}
    \label{tab:rho_selection}
\end{table}

\begin{table}[ht]
    \centering
    \scriptsize
    \resizebox{0.9\textwidth}{!}{\begin{tabular}{l|r|r|r|r|}
    \toprule
    Task & $f = 0.5 $&$f = 0.5$            &$f = 0.8$&$f = 0.8$\\
 & performance & regularizer value & performance & regularizer value\\
 \midrule
hopper-medium        & \textbf{97.2}  & \textbf{-2035.9} &  93.8  & -21.3  \\
walker2d-medium        &  70.9  & -1707.0  & \textbf{81.9}  & \textbf{-1991.2} \\
    \bottomrule
    \end{tabular}}
    \caption{\footnotesize We include our automatic hyperparameter selection rule of $f$ on the medium datasets in the hopper and walker2d environments from D4RL. We follow the same convention defined in Table~\ref{tab:beta_selection} and find that our automatic selection rule can effectively select $f$ offline.}
    \label{tab:f_selection}
\end{table}

\subsection{Details of generalization environments}
\label{app:ood_details}

For \texttt{halfcheetah-jump} and \texttt{ant-angle}, we follow the same environment used in MOPO. For \texttt{sawyer-door-close}, we train the \texttt{sawyer-door} environment in \url{https://github.com/rlworkgroup/metaworld} with dense rewards for opening the door until convergence. We collect $50000$ transitions with half of the data collected by the final expert policy and a policy that reaches the performance of about half the expert level performance. We relabel the reward such that the reward is $1$ when the door is fully closed and $0$ otherwise. Hence, the offline RL agent is required to learn the behavior that is different from the behavior policy in a sparse reward setting. We provide the datasets in the following anonymous link\footnote{The datasets of the generalization environments are available at the following link: \url{https://drive.google.com/file/d/1pn6dS5OgPQVp_ivGws-tmWdZoU7m_LvC/view?usp=sharing}.}.

\subsection{Details of image-based environments}
\label{app:image_details}

\begin{figure}[ht]
    \centering
    \includegraphics[width=0.25\textwidth]{chapters/combo/walker_task.png}
    \includegraphics[width=0.25\textwidth]{chapters/combo/dooropen_task.png}
    \vspace{-0.2cm}
    \caption{\footnotesize Our image-based environments: The observations are $64\times 64$ and $128\times 128$ raw RGB images for the \texttt{walker-walk} and \texttt{sawyer-door} tasks respectively. The \texttt{sawyer-door-close} environment used in in Section~\ref{sec:generalization_exps} also uses the \texttt{sawyer-door} environment.}
    \label{fig:visual}
\end{figure}


We visualize our image-based environments in Figure~\ref{fig:visual}. We use the standard \texttt{walker-walk} environment from \citet{tassa2018deepmind} with $64\times64$ pixel observations and an action repeat of 2. Datasets were constructed the same way as \citet{fu2020d4rl} with 200 trajectories each. For the \texttt{sawyer-door} we use $128\times128$ pixel observations. The medium-expert dataset contains 1000 rollouts (with a rollout length of 50 steps) covering the state distribution from grasping the door handle to opening the door. The expert dataset contains 1000 trajectories samples from a fully trained (stochastic) policy. The data was obtained from the training process of a stochastic SAC policy using dense reward function as defined in \citet{yu2020metaworld}. However, we relabel the rewards, so an agent receives a reward of 1 when the door is fully open and 0 otherwise. This aims to evaluate offline-RL performance in a sparse-reward setting. All the datasets are from \citep{Rafailov2020LOMPO}.


\section{Comparing COMBO to the Naive Combination of CQL and MBPO}
\label{app:cql_mbpo}

\iclr{In this section, we stress the distinction between COMBO and a direct combination of two previous methods CQL and MBPO (denoted as CQL + MBPO). CQL+MBPO performs Q-value regularization using CQL while expanding the offline data with MBPO-style model rollouts. While COMBO utilizes Q-value regularization similar to CQL, the effect is very different. CQL only penalizes the Q-value on unseen actions on the states observed in the dataset whereas COMBO penalizes Q-values on states generated by the learned model while maximizing Q values on state-action tuples in the dataset. Additionally, COMBO also utilizes MBPO-style model rollouts for also augmenting samples for training Q-functions.

To empirically demonstrate the consequences of this distinction, CQL + MBPO performs quite a bit worse than COMBO on generalization experiments (Section~\ref{sec:generalization_exps}) as shown in Table~\ref{tbl:cql_mbpo}. The results are averaged across 6 random seeds ($\pm$ denotes 95\%-confidence interval of the various runs). This suggests that carefully considering the state distribution, as done in COMBO, is crucial.}

\begin{table}[ht]
    \centering
    \scriptsize
    \resizebox{0.7\textwidth}{!}{\begin{tabular}{l|r|r|r|r|}
    \toprule 
    %
    %
    %
    \textbf{Environment} & \stackanchor{\textbf{Batch}}{\textbf{Mean}} & \stackanchor{\textbf{Batch}}{\textbf{Max}} & \stackanchor{\textbf{COMBO}}{\textbf{(Ours)}} & \textbf{CQL+MBPO}\\ \midrule
    halfcheetah-jump & -1022.6 & 1808.6 & \textbf{5392.7}$\pm$575.5 & 4053.4$\pm$176.9\\
    ant-angle & 866.7 & 2311.9 & \textbf{2764.8}$\pm$43.6 & 809.2$\pm$135.4\\
    sawyer-door-close & 5\% & 100\% & \textbf{100}\%$\pm$0.0\% & 62.7\%$\pm$24.8\%\\
    \bottomrule
    \end{tabular}}
    \vspace{-0.2cm}
    \caption{
    \footnotesize Comparison between COMBO and CQL+MBPO on tasks that require out-of-distribution generalization. Results are in average returns of \texttt{halfcheetah-jump} and \texttt{ant-angle} and average success rate of \texttt{sawyer-door-close}. All results are averaged over 6 random seeds, $\pm$ the $95\%$-confidence interval.
    }
    \vspace{-0.3cm}
    \label{tbl:cql_mbpo}
    \normalsize
    \end{table}
    

% \subsection{Computation Complexity}

% For the D4RL and generalization experiments, COMBO is trained on a single NVIDIA GeForce RTX 2080 Ti for one day. For the image-based experiments, we utilized a single NVIDIA GeForce RTX 2070. We trained the \texttt{walker-walk} tasks for a day and the \texttt{sawyer-door-open} tasks for about two days.

% \subsection{License of datasets}

% We acknowledge that all datasets used in this paper use the MIT license.

% % \vspace{1cm}
% \section{Empirical Evidence on Challenges of Uncertainty Quantification}
% \label{app:uq}

% \begin{figure}[t]
%     \centering
%     \includegraphics[width=0.47\linewidth]{halfcheetah_medium_corr_var_ood.png}
%     \includegraphics[width=0.47\linewidth]{halfcheetah_medium_corr_lip_ens_ood.png}
%     \includegraphics[width=0.47\linewidth]{hopper_medium_corr_var_ood.png}
%     \includegraphics[width=0.47\linewidth]{hopper_medium_corr_lip_ens_ood.png}
%     \includegraphics[width=0.47\linewidth]{walker_medium_corr_var_ood.png}
%     \includegraphics[width=0.47\linewidth]{walker_medium_corr_lip_ens_ood.png}
%     \vspace{-0.2cm}
%     \caption{\footnotesize
%     %
%     We visualize the correlation between the model error and two uncertainty quantification methods maximum learned variance over the ensemble (left column) and variance of the model prediction over the ensemble (right column) on three D4RL medium datasets (from the top row to the bottom row: halfcheetah, hopper and walker) where MOPO performs poorly compared to model-free methods. We show that \textbf{Max Var} tends to be overly conservative and overestimating the model error while \textbf{Ens. Var} is on the opposite. Such visualizations corroborate that uncertainty quantification is challenging with deep neural networks and could lead to poor performance in model-based offline RL. In the meantime, COMBO addresses this issue by removing the burden of performing uncertainty quantification.}
%     \label{fig:uq}
%     \vspace{-0.3cm}
% \end{figure}

% In this section, we perform empirical evaluations to show that uncertainty quantification with deep neural networks, especially in the setting of dynamics model learning, is challenging and could cause problems with uncertainty-based model-based offline RL methods such as MOReL~\citep{kidambi2020morel} and MOPO~\citep{yu2020mopo}. In our evaluations, we consider two uncertainty quantification methods, maximum learned variance over the ensemble (denoted as \textbf{Max Var}) $\max_{i=1,\dots,N}\|\Sigma^i_\theta(\bs,\mathbf{a})\|_\text{F}$ (used in MOPO) and the variance of the model prediction over the ensemble (denoted as \textbf{Ens. Var}) $\max_{i=1,\dots,N}\|\mu^i_\theta(\bs,\mathbf{a}) - \frac{1}{N}\sum_{j=1}^N\mu^j_\theta(\bs,\mathbf{a})\|_2$ (used in MOPO and MOReL) where we use an ensemble of $N$ probabilistic dynamics models $\{\widehat{T}^i_\theta(\bs_{t+1}, r| \bs, \mathbf{a}) = \mathcal{N}(\mu^i_\theta(\bs_t, \mathbf{a}_t), \Sigma^i_\theta(\bs_t, \mathbf{a}_t))\}_{i=1}^N$.

% As shown in Table~\ref{tbl:d4rl}, MOPO performs underwhelmingly on medium datasets in the D4RL datasets where the dataset is collected with a single policy and hence with relatively narrow data coverage of the whole state space. To empirically analyze the poor performance of MOPO on those datasets, we visualize the correlation between the true model error and two uncertainty quantification methods \textbf{Max Var} and \textbf{Ens. Var}. We normalize both the model error and the uncertainty estimates to be within scale $[0, 1]$. As shown in Figure~\ref{fig:uq}, on all three medium datasets, \textbf{Max Var} tends to be overly conservative and \textbf{Ens. Var} behaves too optimistic to correctly quantify the true model error, suggesting that uncertainty estimation used by MOPO is not accurate and might be the major factor that results in its poor performance. Meanwhile, COMBO circumvents challenging uncertainty quantification problem and achieves much better performances on those medium datasets, indicating the effectiveness and the robustness of the method.



\end{document}