\documentclass{article}

% if you need to pass options to natbib, use, e.g.:
%     \PassOptionsToPackage{numbers, compress}{natbib}
% before loading neurips_2021

% ready for submission
\PassOptionsToPackage{numbers, compress}{natbib}
\usepackage{neurips_2021}

% to compile a preprint version, e.g., for submission to arXiv, add add the
% [preprint] option:
%     \usepackage[preprint]{neurips_2021}

% to compile a camera-ready version, add the [final] option, e.g.:
%     \usepackage[final]{neurips_2021}

% to avoid loading the natbib package, add option nonatbib:
%    \usepackage[nonatbib]{neurips_2021}

\usepackage[utf8]{inputenc} % allow utf-8 input
\usepackage[T1]{fontenc}    % use 8-bit T1 fonts
\usepackage{amsthm}
\usepackage[colorlinks=true,linkcolor=blue,citecolor=blue]{hyperref}
\newtheorem{proposition}{Proposition}[section]
\newtheorem{lemma}{Lemma}[section]
\usepackage{url}            % simple URL typesetting
\usepackage{booktabs}       % professional-quality tables
\usepackage{amsfonts}       % blackboard math symbols
\usepackage{nicefrac}       % compact symbols for 1/2, etc.
\usepackage{microtype}      % microtypography
\usepackage{xcolor}         % colors
\usepackage{fleqn, tabularx}
\usepackage{multirow}
\usepackage[export]{adjustbox}
\usepackage{amsmath}
\usepackage{colortbl}
\usepackage{wrapfig}
\usepackage{algorithm}
\usepackage[noend]{algorithmic}
\usepackage{xcolor}

\definecolor{Gray}{gray}{0.9}


\makeatletter
\newcommand{\mybox}{%
    \collectbox{%
        \setlength{\fboxsep}{1pt}%
        \fbox{\BOXCONTENT}%
    }%
}
\makeatother
\newcommand{\CC}{\cellcolor{Gray}}

%% editing comment
\newcommand{\cmt}[1]{{\footnotesize\textcolor{red}{#1}}}
\newcommand{\cmto}[1]{{\footnotesize\textcolor{orange}{#1}}}
\newcommand{\note}[1]{\cmt{Note: #1}}
\newcommand{\todo}[1]{\cmt{TO-DO: #1}}
\newcommand{\question}[1]{\cmto{Question: #1}}
\newcommand{\sergey}[1]{{\footnotesize\textcolor{blue}{Sergey: #1}}}
\newcommand{\ak}[1]{{\textcolor{black}{#1}}}
\newcommand{\edits}[1]{\textcolor{blue}{#1}}
\newcommand{\editsred}[1]{\textcolor{red}{#1}}
\newcommand{\editsp}[1]{\textcolor{purple}{#1}}
\newcommand{\editsv}[1]{\textcolor{magenta}{#1}}


\newcommand{\methodname}{Cal-QL}
\newcommand{\aliasingproblemname}{bootstrapping aliasing}
\newcommand{\Aliasingproblemname}{Bootstrapping aliasing}
\newcommand{\AliasingProblemName}{Bootstrapping Aliasing}
\newcommand{\simnorm}{\mathrm{sim}_{\mathrm{n}}^\pi}
\newcommand{\simunnorm}{\mathrm{sim}_{\mathrm{u}}^\pi}

%% abbreviations
\newcommand{\x}{\mathbf{x}}
\newcommand{\z}{\mathbf{z}}
\newcommand{\y}{\mathbf{y}}
\newcommand{\w}{\mathbf{w}}
\newcommand{\data}{\mathcal{D}}

\newcommand{\etal}{{et~al.}\ }
\newcommand{\eg}{e.g.\ }
\newcommand{\ie}{i.e.\ }
\newcommand{\nth}{\text{th}}
\newcommand{\pr}{^\prime}
\newcommand{\tr}{^\mathrm{T}}
\newcommand{\inv}{^{-1}}
\newcommand{\pinv}{^{\dagger}}
\newcommand{\real}{\mathbb{R}}
\newcommand{\gauss}{\mathcal{N}}
\newcommand{\norm}[1]{\left|#1\right|}
\newcommand{\trace}{\text{tr}}

%% specifics for the paper
\newcommand{\reward}{r}
\newcommand{\policy}{\pi}
\newcommand{\mdp}{\mathcal{M}}
\newcommand{\states}{\mathcal{S}}
\newcommand{\actions}{\mathcal{A}}
\newcommand{\observations}{\mathcal{O}}
\newcommand{\transitions}{T}
\newcommand{\initstate}{d_0}
\newcommand{\freq}{d}
\newcommand{\obsfunc}{E}
\newcommand{\initial}{\mathcal{I}}
\newcommand{\horizon}{H}
\newcommand{\rewardevent}{R}
\newcommand{\probr}{p_\rewardevent}
\newcommand{\metareward}{\bar{\reward}}
\newcommand{\discount}{\gamma}
\newcommand{\behavior}{{\pi_\beta}}
\newcommand{\bellman}{\mathcal{B}}
\newcommand{\qparams}{\phi}
\newcommand{\qparamset}{\Phi}
\newcommand{\qset}{\mathcal{Q}}
\newcommand{\batch}{B}
\newcommand{\qfeat}{\mathbf{f}}
\newcommand{\Qfeat}{\mathbf{F}}
\newcommand{\hatbehavior}{\hat{\pi}_\beta}

\newcommand{\traj}{\tau}

\newcommand{\pihi}{\pi^{\text{hi}}}
\newcommand{\pilo}{\pi^{\text{lo}}}
\newcommand{\ah}{\mathbf{w}}

\newcommand{\proj}{\Pi}

\newcommand{\loss}{\mathcal{L}}
\newcommand{\eye}{\mathbf{I}}

\newcommand{\model}{\hat{p}}
\newcommand{\mhat}{\hat{\mathcal{M}}}
\newcommand{\mdphat}{\widehat{\mathcal{M}}}
\newcommand{\mdpbar}{\overline{\mathcal{M}}}

\newcommand{\pimix}{\pi_{\text{mix}}}

\newcommand{\pib}{\bar{\pi}}
\newcommand{\epspi}{\epsilon_{\pi}}
\newcommand{\epsmodel}{\epsilon_{m}}

\newcommand{\return}{\mathcal{R}}

%% math
\newcommand{\cY}{\mathcal{Y}}
\newcommand{\cX}{\mathcal{X}}
\newcommand{\en}{\mathcal{E}}
\newcommand{\bu}{\mathbf{u}}
\newcommand{\bv}{\mathbf{v}}
\newcommand{\be}{\mathbf{e}}
\newcommand{\by}{\mathbf{y}}
\newcommand{\bx}{\mathbf{x}}
\newcommand{\bz}{\mathbf{z}}
\newcommand{\bw}{\mathbf{w}}
\newcommand{\bo}{\mathbf{o}}
\newcommand{\bs}{\mathbf{s}}
\newcommand{\ba}{\mathbf{a}}
\newcommand{\bM}{\mathbf{M}}
\newcommand{\ot}{\bo_t}
\newcommand{\st}{\bs_t}
\newcommand{\at}{\ba_t}
\newcommand{\op}{\mathcal{O}}
\newcommand{\opt}{\op_t}
\newcommand{\kl}{D_\text{KL}}
\newcommand{\tv}{D_\text{TV}}
\newcommand{\ent}{\mathcal{H}}
\newcommand{\bG}{\mathbf{G}}
\newcommand{\byk}{\mathbf{y_k}}
\newcommand{\bI}{\mathbf{I}}
\newcommand{\bg}{\mathbf{g}}
\newcommand{\bV}{\mathbf{V}}
\newcommand{\bD}{\mathbf{D}}
\newcommand{\bR}{\mathbf{R}}
\newcommand{\bQ}{\mathbf{Q}}
\newcommand{\bA}{\mathbf{A}}
\newcommand{\bN}{\mathbf{N}}
\newcommand{\bS}{\mathbf{S}}
\newcommand{\bW}{\mathbf{W}}
\newcommand{\bU}{\mathbf{U}}
\newcommand{\bO}{\mathbf{O}}

\newcommand{\bzhi}{\bz^\text{hi}}

\newcommand{\expected}{\mathbb{E}}
\newcommand{\E}{\mathbb{E}}
\newcommand{\srank}{\text{srank}}
\newcommand{\rank}{\text{rank}}
\newcommand{\deepnet}{\bW_N(k, t) \bW_\phi(k, t)}
\newcommand{\features}{\bW_\phi(k, t)}
\newcommand{\stateactioni}{[\bs_i; \ba_i]}
\newcommand{\diag}{\text{\textbf{diag}}}

\def\thetaP{\theta^{\prime}}
\def\cf{\emph{c.f.}\ }
\def\vs{\emph{vs}.\ }
\def\etc{\emph{etc.}\ }
\def\Eqref#1{Equation~\ref{#1}}

\newenvironment{repeatedthm}[1]{\@begintheorem{#1}{\unskip}}{\@endtheorem}
\newcommand{\algname}{COMBO\xspace}

\newtheorem{theorem}{Theorem}[section]
\newtheorem{sketchtheorem}{Sketch Theorem}[section]
\newtheorem{lemma}[theorem]{Lemma}
\newtheorem{corollary}[theorem]{Corollary}
\newtheorem{proposition}[theorem]{Proposition}
\newtheorem{definition}[theorem]{Definition}
\newtheorem{conjecture}[theorem]{Conjecture}
\newtheorem{problem}[theorem]{Problem}
\newtheorem{formulation}[theorem]{Formulation}
\newtheorem{claim}[theorem]{Claim}
\newtheorem{remark}[theorem]{Remark}
\newtheorem{example}[theorem]{Example}
\newtheorem{assumption}[theorem]{Assumption}
\newtheorem{exercise}[theorem]{Exercise}


\newcommand{\indep}{\rotatebox[origin=c]{90}{$\models$}}

\renewcommand{\mathbf}{\boldsymbol}

\newcommand{\conv}{\circledast}
\newcommand{\mb}{\mathbf}
\newcommand{\mc}{\mathcal}
\newcommand{\mf}{\mathfrak}
\newcommand{\md}{\mathds}
\newcommand{\bb}{\mathbb}
\newcommand{\msf}{\mathsf}
\newcommand{\mcr}{\mathscr}
\newcommand{\magnitude}[1]{ \left| #1 \right| }
\newcommand{\set}[1]{\left\{ #1 \right\}}
\newcommand{\condset}[2]{ \left\{ #1 \;\middle|\; #2 \right\} }


\newcommand{\reals}{\bb R}
\newcommand{\proj}{\mathrm{proj}}

\newcommand{\eps}{\varepsilon}
\newcommand{\R}{\reals}
\newcommand{\Cp}{\bb C}
\newcommand{\Z}{\bb Z}
\newcommand{\N}{\bb N}
\newcommand{\Sp}{\bb S}
\newcommand{\Ba}{\bb B}
\newcommand{\indicator}[1]{\mathbbm 1\left\{#1\right\}}
\renewcommand{\P}{\mathbb{P}}
\newcommand{\rvline}{\hspace*{-\arraycolsep}\vline\hspace*{-\arraycolsep}}
\makeatletter
\def\Ddots{\mathinner{\mkern1mu\raise\p@
\vbox{\kern7\p@\hbox{.}}\mkern2mu
\raise4\p@\hbox{.}\mkern2mu\raise7\p@\hbox{.}\mkern1mu}}
\makeatother
% to declare new operator
% \DeclareMathOperator{\xxx}{xxx}

%% Other definitions

\newcommand{\event}{\mc E}

\newcommand{\e}{\mathrm{e}}
\newcommand{\im}{\mathrm{i}}
\newcommand{\rconcave}{r_\fgecap}
\newcommand{\Lconcave}{\mc L^\fgecap}
\newcommand{\rconvex}{r_\fgecup}
\newcommand{\Rconvex}{R_\fgecup}
\newcommand{\Lconvex}{\mc L^\fgecup}

\newcommand{\wh}{\widehat}
\newcommand{\wt}{\widetilde}
\newcommand{\ol}{\overline}


\newcommand{\betaconcave}{\beta_\fgecap}
\newcommand{\betagrad}{\beta_{\mathrm{grad}}}

\newcommand{\norm}[2]{\left\| #1 \right\|_{#2}}
\newcommand{\abs}[1]{\left| #1 \right|}
\newcommand{\row}[1]{\text{row}\left( #1 \right)}
\newcommand{\innerprod}[2]{\left\langle #1,  #2 \right\rangle}
\newcommand{\prob}[1]{\bb P\left[ #1 \right]}
\newcommand{\expect}[1]{\bb E\left[ #1 \right]}
\newcommand{\function}[2]{#1 \left(#2\right}
\newcommand{\integral}[4]{\int_{#1}^{#2}\; #3\; #4}
\newcommand{\paren}[1]{\left( #1 \right)}
\newcommand{\brac}[1]{\left[ #1 \right]}
\newcommand{\Brac}[1]{\left\{ #1 \right\}}

% Adding new defs here
\newcommand{\moff}{m_\mr{off}}
\newcommand{\mon}{m_\mr{on}}
\newcommand{\EmpiricalOffline}{\wh{\bb E}_{\mc D^\nu_h}}
\newcommand{\EmpiricalOnline}{\wh{\bb E}_{\mc D^\tau_h}}
\newcommand{\Deltaoff}{\Delta_\mr{off}}
\newcommand{\Deltaon}{\Delta_\mr{on}}
\newcommand{\Vmax}{V_{\max}}
\newcommand{\regret}{\mr{Reg}}
\newcommand{\regreton}{\mr{Sub}_{\mr {on}}}
\newcommand{\regretoff}{\mr{Sub}_{\mr {off}}}
\newcommand{\Doff}{\mc D_\mr{off}}
\newcommand{\Don}{\mc D_\mr{on}}
\newcommand{\piref}{\pi_\mr{ref}}

\newcommand{\nt}[1]{{\color{purple}{\bf [Next: #1]}}}
\newcommand{\here}{{\color{purple}{\bf [Writing here]}}}
% \newcommand{\todo}{{\color{purple}{\bf [TODO]}}}
\newcommand{\sz}[1]{{\color{blue}{\bf [Simon: #1]}}}
% \newcommand{\note}[1]{{\color{red}{\bf [note: #1]}}}
\newcommand{\mr}{\mathrm}
\newcommand{\sym}{\mathrm{Sym}}
\newcommand{\sks}{\mathrm{Skew}}
\newcommand{\inprod}[2]{\langle#1,#2\rangle}
\newcommand{\parans}[1]{\left(#1\right}
\newcommand{\clip}{\msf{clipped}}
\newcommand{\beha}{\msf b}
\numberwithin{equation}{section}

% \def \endprf{\hfill {\vrule height6pt width6pt depth0pt}\medskip}

\newcommand{\cdsmethodname}{CDS}
\newcommand{\udsmethodname}{UDS}
\newcommand{\ptrmethodname}{PTR~}
\newcommand{\primemethodname}{PRIME~}
\newcommand{\arxiv}[1] {{\color{black} #1}}

% \newenvironment{proof}{\noindent {\bf Proof} }{\endprf\par}

% \newcommand{\qed}{{\unskip\nobreak\hfil\penalty50\hskip2em\vadjust{}
%            \nobreak\hfil$\Box$\parfillskip=0pt\finalhyphendemerits=0\par}}


\newcommand\myworries[1]{\textcolor{red}{#1}}
\newcommand\running[1]{\textcolor{blue}{#1}}

\newcommand{\benchl}[1]{{\scriptsize\textsf{#1}}\normalsize\xspace}
\newcommand{\bench}[1]{{\fontsize{8.5}{10}\selectfont\textsf{#1}}\normalsize\xspace}
\newcommand{\code}[1]{{\fontsize{8.5}{1}\selectfont{\tt #1}}\xspace}
\newcommand{\codebold}[1]{{\fontsize{8.5}{1}\selectfont{\tt \textbf{#1}}}\xspace}
\newcommand{\xx}[1]{\textcolor{black}{#1}}
\newcommand{\xxs}[1]{\textcolor{black}{\scriptsize\textsf{#1}}}
\newcommand{\supertiny}[1]{\fontsize{5}{4}\selectfont{#1}}
\newcommand{\xxred}[1]{\textcolor{red}{\textsf{#1}}}
% \DeclareMathOperator{\EX}{\mathbb{\hat{E}}}% expected value
% \makeatletter
% \newcommand\footnoteref[1]{\protected@xdef\@thefnmark{\ref{#1}}\@footnotemark}
% \makeatother
% \usepackage{scrextend}

% \deffootnote[1em]{1em}{1em}{\textsuperscript{\thefootnotemark}\,}
% \newcolumntype{P}[1]{>{\centering\arraybackslash}p{#1}}
% \newcommand\mycommfont[1]{\footnotesize\ttfamily\textcolor{blue}{#1}}

\newcommand\aviral[1]{\textcolor{red}{aviralkumar@: #1}}
\newcommand\ayazdan[1]{\textcolor{red}{ayazdan@: #1}}
\newcommand\sv[1]{\textcolor{red}{SV@: #1}}
\newcommand\fix[1]{\textcolor{green}{#1}}

\newcommand{\round}[1]{\ensuremath{\lfloor#1\rceil}}

\newcommand{\tgray}[1]{\colorbox{lightgray}{\textbf{#1}}}
\newcommand{\finalcheck}[1]{\textcolor{green}{#1}}

\newcommand{\review}[1]{#1}

\newcommand{\niparagraph}[1]{\vspace{2pt}\noindent\textbf{#1}}


\usepackage{titlesec}
\titlespacing\section{0pt}{0pt plus 2pt minus 2pt}{0pt plus 2pt minus 2pt}
\titlespacing\subsection{0pt}{3pt plus 4pt minus 2pt}{0pt plus 2pt minus 2pt}
\titlespacing\subsubsection{0pt}{3pt plus 4pt minus 2pt}{0pt plus 2pt minus 2pt}

\title{Multi-Task Offline Reinforcement Learning with Conservative Data Sharing}
%%CF.5.17: The current title (Multi-Task Offline Reinforcement Learning with Conservative Data Sharing) seems pretty good to me.
% KY: ideas for better titles?
%%SL.5.15: Multi-Task Offline Reinforcement Learning with Optimal Relabeling? There is something appealing about this paper being known as "the multi-task offline RL paper" :)
% The \author macro works with any number of authors. There are two commands
% used to separate the names and addresses of multiple authors: \And and \AND.
%
% Using \And between authors leaves it to LaTeX to determine where to break the
% lines. Using \AND forces a line break at that point. So, if LaTeX puts 3 of 4
% authors names on the first line, and the last on the second line, try using
% \AND instead of \And before the third author name.

\author{%
  David S.~Hippocampus\thanks{Use footnote for providing further information
    about author (webpage, alternative address)---\emph{not} for acknowledging
    funding agencies.} \\
  Department of Computer Science\\
  Cranberry-Lemon University\\
  Pittsburgh, PA 15213 \\
  \texttt{hippo@cs.cranberry-lemon.edu} \\
  % examples of more authors
  % \And
  % Coauthor \\
  % Affiliation \\
  % Address \\
  % \texttt{email} \\
  % \AND
  % Coauthor \\
  % Affiliation \\
  % Address \\
  % \texttt{email} \\
  % \And
  % Coauthor \\
  % Affiliation \\
  % Address \\
  % \texttt{email} \\
  % \And
  % Coauthor \\
  % Affiliation \\
  % Address \\
  % \texttt{email} \\
}

\begin{document}

\maketitle

\begin{abstract}
Offline reinforcement learning (RL) algorithms have shown promising results in domains where abundant pre-collected data is available. However, prior methods focus on solving individual problems from scratch with an offline dataset without considering how an offline RL agent can acquire multiple skills. We argue that a natural use case of offline RL is in settings where we can pool large amounts of data collected in a number of different scenarios for solving various tasks, and utilize all this data to learn strategies for all the tasks more effectively rather than training each one in isolation. To this end, we study the offline multi-task RL problem, with the goal of devising data-sharing strategies for effectively learning behaviors across all of the tasks. While it is possible to share all data across all tasks, we find that this simple strategy can actually exacerbate the distributional shift between the learned policy and the dataset, which in turn can lead to very poor performance.
To address this challenge, we develop a simple technique for data-sharing in multi-task offline RL that routes data based on the improvement over the task-specific data. We call this approach conservative data sharing (\methodname), and it can be applied with any single-task offline RL method. On a range of challenging multi-task locomotion, navigation, and image-based robotic manipulation problems, \methodname\ achieves the best or comparable performance compared to prior offline multi-task RL methods and previously proposed online multi-task data sharing approaches.  
\end{abstract}

\vspace{-0.1cm}
\section{Introduction}
\vspace{-0.1cm}
Deep neural networks are overparameterized, with billions of parameters, which in principle should leave them vulnerable to overfitting. Despite this, supervised learning with deep networks still learn representations that generalize  well. A widely held consensus is that deep nets find simple solutions that generalize due to various \emph{implicit} regularization effects~\citep{blanc2020implicit,woodworth2020kernel,arora2018optimization,gunasekar2017implicit,wei2019regularization,li2019towards}. We may surmise that using deep neural nets in reinforcement learning~(RL) will work well for the same reason, learning effective representations that generalize due to such implicit regularization effects. But is this actually the case for value functions trained via bootstrapping? 

In this paper, we argue that, while implicit regularization leads to effective representations in supervised deep learning, it may lead to poor learned representations when training overparameterized deep network value functions. 
In order to rule out confounding effects from exploration and non-stationary data distributions, we focus on the offline RL setting -- where deep value networks must be trained from a static dataset of experience.
There is already evidence that value functions trained via bootstrapping learn poor representations: value functions trained with offline deep RL eventually degrade in performance~\citep{agarwal2019optimistic, kumar2021implicit} and this degradation is correlated with the emergence of low-rank features
in the value network~\citep{kumar2021implicit}.
Our goal is to understand the underlying cause of the emergence of poor representations during bootstrapping and develop a potential solution. Building on the theoretical framework developed by \citet{blanc2020implicit,damian2021label}, we characterize the implicit regularizer that arises when training deep value functions with TD learning. The form of this implicit regularizer implies that TD-learning would co-adapt feature representations at state-action tuples that appear on either side of a Bellman backup.

We show that this theoretically predicted aliasing phenomenon manifests in practice as feature \textbf{co-adaptation}, where the features of consecutive state-action tuples learned by the Q-value network become very similar in terms of their dot product~(\Secref{sec:problem}). This co-adaptation co-occurs with oscillatory learning dynamics, and training runs that exhibit feature co-adaptation typically converge to poorly performing solutions. Even when Q-values are not overestimated, prolonged training in offline RL can result in performance degradation as feature co-adaptation increases. 
To mitigate this co-adaptation issue, which arises as a result of implicit regularization, we propose an \emph{explicit regularizer} that we call \methodname~(\Secref{sec:method}).
%%SL.9.29: This is a relatively minor thing, but when you introduce the name DR3, can you actually say what it stands for?
%%AK: I was trying to cook up a full form, but it seems like the reason why we put it this way was "DR3: Deep Reinforcement Learning Requires Explicit Regularization", but maybe this is not the best thing to call a method?
While exactly estimating and cancelling the effects of the theoretically derived implicit regularizer is computationally difficult, \methodname\ provides a simple and tractable theoretically-inspired approximation that mitigates the issues discussed above. In practice, \methodname\ amounts to regularizing the features at consecutive state-action pairs to be dissimilar in terms of their dot-product similarity. Empirically, we find that \methodname\ prevents previously noted pathologies such as feature rank collapse~\citep{kumar2021implicit},  gives methods that train for longer and improves performance relative to the base offline RL method employed in practice.
% Empirically, we find that \methodname\ allows neural network Q-functions to use their full representational capacity, as measured by the rank of the learned features, and \textcolor{red}{enables the use of larger, more expressive neural networks}, 
% %%SL.9.29: How do we test it allows them to use their full capacity? Perhaps we should remove this claim, since it doesn't seem like we have any evidence for it.
% %%AK: rank of learned features is higher? 
% %%SL.10.27: I'm still concerned that this statement may not be backed up by evidence. Do we actually show that this *allows* using larger networks (i.e., larger net + DR3 = good, but larger net - DR3 = bad?) I think we would need something like that to back up this statement
% giving rise to methods that can train for longer without degradation and thus reach a better solution~(\Secref{sec:experiments}).

Our first contribution is the derivation of the implicit regularizer that arises when training deep net value functions via TD learning, and an empirical demonstration that it manifests as \emph{feature co-adaptation} in the offline deep RL setting.
%, which results in highly similar feature representations for state-action tuples at consecutive time steps. 
Feature co-adaptation accounts at least in part for some of the challenges of offline deep RL, including degradation of performance with prolonged training. Second, we propose a simple and effective \emph{explicit} regularizer for offline value-based RL, \methodname, which minimizes the feature similarity between state-action pairs appearing in a bootstrapping update. \methodname\ is inspired by the theoretical derivation of the implicit regularizer, it alleviates co-adaptation and can be easily combined with modern offline RL methods, such as REM~\citep{agarwal2019optimistic}, CQL~\citep{kumar2020conservative}, and BRAC~\citep{wu2019behavior}. Empirically, using \methodname\ in conjunction with existing offline RL methods provides about \textbf{60\%} performance improvement on the harder D4RL~\citep{fu2020d4rl} tasks, and \textbf{160\%} and \textbf{25\%} stability gains for REM and CQL, respectively, on offline RL tasks in 17 Atari 2600 games. Additionally, we observe large improvements on image-based robotic manipulation tasks~\citep{singh2020cog}.


\section{Related Work}

\textbf{Offline RL.} Offline RL~\citep{ernst2005tree, riedmiller2005neural, LangeGR12, levine2020offline} has shown promise in domains such as robotic manipulation~\citep{kalashnikov2018scalable, mandlekar2020iris, Rafailov2020LOMPO,singh2020cog,kalashnikov2021mt}, NLP~\citep{jaques2019way,jaques2020human}, recommender systems \& advertising~\citep{strehl2010learning,garcin2014offline,charles2013counterfactual,theocharous2015ad,thomas2017predictive}, and healthcare~\citep{shortreed2011informing, Wang2018SupervisedRL}. The major challenge in offline RL is distribution shift~\citep{fujimoto2018off,kumar2019stabilizing,kumar2020conservative}, where the learned policy might generate out-of-distribution actions, resulting in erroneous value backups. Prior offline RL methods address this issue by regularizing the learned policy to be ``close`` to the behavior policy~\citep{fujimoto2018off,liu2020provably,jaques2019way,wu2019behavior, zhou2020plas,kumar2019stabilizing,siegel2020keep, peng2019advantage}, through variants of importance sampling~\citep{precup2001off, sutton2016emphatic, LiuSAB19, SwaminathanJ15, nachum2019algaedice}, via uncertainty quantification on Q-values~\citep{agarwal2020optimistic, kumar2019stabilizing, wu2019behavior, levine2020offline}, by learning conservative Q-functions~\citep{kumar2020conservative,kostrikov2021offline}, and with model-based training with a penalty on out-of-distribution states~\citep{kidambi2020morel, yu2020mopo,matsushima2020deployment,argenson2020model,swazinna2020overcoming,Rafailov2020LOMPO,lee2021representation,yu2021combo}. While current benchmarks in offline RL~\citep{fu2020d4rl,gulcehre2020rl} contain datasets that involve multi-task structure, existing offline RL methods do not leverage the shared structure of multiple tasks and instead train each individual task from scratch. In this paper, we exploit the shared structure in the offline multi-task setting and train a general policy that can acquire multiple skills.

%%CF.5.17: I would consider mentioning meta-RL methods somewhere, since they also address multi-task RL and especially since there are some that aren't conflicted I think (e.g. VariBAD, meta-Q-learning). Some of them even reuse data
%%TY.5.21: I think meta-RL methods might be a bit orthogonal since they aim for generalization to new tasks. I can cite some multi-task RL works that are less conflicted.
\textbf{Multi-task RL algorithms.} Multi-task RL algorithms~\citep{wilson2007multi,parisotto2015actor,teh2017distral,espeholt2018impala,hessel2019popart,yu2020gradient, xu2020knowledge, yang2020multi, kalashnikov2021mt,sodhani2021multi}
%%CF.5.17: there are online MTRL methods that are more recent than this. For example, there's one on soft modules from USC or UCSD. You can look at papers that cite PCGrad or meta-world and/or look on google scholar for more.
%%TY.5.21: added several more papers.
focus on solving multiple tasks jointly in an efficient way. While multi-task RL methods seem to provide a promising way to build general-purpose agents~\citep{kalashnikov2021mt}, prior works have observed major challenges in multi-task RL, in particular, the optimization challenge~\citep{hessel2019popart,schaul2019ray,yu2020gradient}.
Beyond the optimization challenge, how to perform effective representation learning via weight sharing is another major challenge in multi-task RL. Prior works have considered distilling per-task policies into a single policy that solves all tasks~\citep{rusu2015policy,teh2017distral,ghosh2017divide,xu2020knowledge}, separate shared and task-specific modules with theoretical guarantees~\citep{d2019sharing}, and incorporating additional supervision~\citep{sodhani2021multi}. Finally, sharing data across tasks emerges as a challenge in multi-task RL, especially in the off-policy setting, as na\"{i}vely sharing data across all tasks turns out to hurt performance in certain scenarios~\citep{kalashnikov2021mt}. Unlike most of these prior works, we focus on the offline setting where the challenges in data sharing are most relevant. Methods that study optimization and representation learning issues are complementary and can be readily combined with our approach.
%%CF.5.17: how does your method & analysis contrast with all of these methods? need to explicitly state what is different. (ie things like - we focus on the offline setting where some of these issues are less severe & just different; we focus on data sharing & methods that look at optimization & representation are complementary, something about the analysis contributing to our understanding in a complementary way, etc)
%%CF.5.17: I also wonder if we should include a comparison that runs only one of these methods to show some evidence that they don't solve the problem.
%%TY.5.21: Added the discussion above. We can also perform the empirical analysis on HIPI to see if it solves the problem.
% We will next survey methods in data sharing in multi-task off-policy RL.

%%CF.5.17: Other papers that do some form of data sharing:
% the REPAINT paper - includes a less naive data sharing approach
% model-based RL methods (e.g. visual foresight, some experiments in Danijar's papers, MBOLD) - these share everything
% Dave Held paper on goal-conditioned RL from images - not sure how they share
% probably other GCRL papers? eg Yevgen's actionable models, distributional planning networks, but probably others that aren't conflicted
%%TY.5.21: The REPAINT paper and model-based RL methods do not seem to study the multi-task problem, which might be less relevant? I added references to more GCRL papers.
%%CF.8.1: The model-based RL papers that I mention above *do* include some multi-task experiments (and share data across tasks, and often don't recompute rewards because often only the model is what is sharing cross-task data.) Also, both visual foresight and MBOLD operate in the fully offline setting.
\textbf{Data sharing in multi-task RL.} Prior works~\citep{andrychowicz2017hindsight,kaelbling1993learning,pong2018temporal,schaul2015universal,eysenbach2020rewriting,li2020generalized,kalashnikov2021mt,chebotar2021actionable} have found it effective to reuse data across tasks by recomputing the rewards of data collected for one task and using such relabeled data for other tasks, which effectively augments the amount of data available for learning each task and boosts performance. These methods perform relabeling either uniformly~\citep{kalashnikov2021mt} or based on metrics such as estimated Q-values~\citep{eysenbach2020rewriting,li2020generalized}, domain knowledge~\citep{kalashnikov2021mt}, the distance to states or images in goal-conditioned settings~\citep{andrychowicz2017hindsight,pong2018temporal,nair2018visual,liu2019competitive,sun2019policy,lin2019reinforcement,huang2019mapping,lynch2020grounding,yang2021bias,chebotar2021actionable}, \arxiv{and metric learning for robust inference in the offline meta-RL setting~\citep{li2019multi}. All of these methods either require online data collection and do not consider data sharing in a fully offline setting, or only consider offline goal-conditioned or meta-RL problems~\citep{chebotar2021actionable,li2019multi}.} \arxiv{While these prior works empirically find that data sharing helps, we believe that our analysis in Section~\ref{sec:analysis} provides the first analytical understanding of why and when data sharing can help in multi-task offline RL and why it hurts in some cases.} 
\arxiv{Specifically, our analysis reveals the effect of distributional shift introduced during data sharing, which is not taken into account by these prior works. Our proposed approach, CDS, tackles the challenge of distributional shift in data sharing by intelligently sharing data across tasks and improves multi-task performance by effectively trading off between the benefits of data sharing and the harms of excessive distributional shift.}
%%SL.8.1: I think this paragraph kind of buries the main point: it makes it sound like we are just (rather naively) extending the ideas from these past papers to the offline multi-task setting, which really undersells the contribution. It's not like we're just doing what they already did but extending beyond goals, we are actually addressing a challenge that these methods did not address (and indeed that they suffer from).
%%TY.8.1: I revised the above paragraph to say that our method addresses the challenge that prior works didn't address.
%%SL.5.15: I think it's important to expand the discussion of prior multi-task RL methods and better cover other methods that aim to understand why multi-task RL is hard, empirically observe that it's hard, and offer various solutions. Right now the above citations seem to focus more or less exclusively on "applications" of multi-task RL, whereas we need to survey prior work on analysis and solutions (maybe in a separate paragraph). This includes things like ray interference, pcgrad, and other papers you can find that cite those or are cited by them
%%TY.5.16: I added a paragraph discussing challenges in multi-task RL and then use the above paragraph to survey relabeling methods in the off-policy setting.


\section{Preliminaries}
\vspace{-0.15cm}
\label{sec:prelim}

\textbf{Offline RL.} Standard RL considers a Markov decision process (MDP), $\mdp =(\states, \actions, P, \gamma, R)$, where $\states$ and $\actions$ denote the state and action spaces respectively, $P(\bs' | \bs, \mathbf{a})$ denotes the dynamics, $\gamma \in [0, 1)$ is the discount factor, and $R$ correspond to the reward function. Offline RL tackles the problem of learning a policy $\pi(\mathbf{a}|\bs)$ from a static dataset with $\mathcal{D}$, generated by a behavior policy $\pi_\beta(\mathbf{a}|\bs)$.


\textbf{Data sharing in offline RL.} Data sharing has been considered in the multi-task offline RL setting where there is a static multi-task dataset with $\mathcal{D} = \cup_{i=1}^N \mathcal{D}_i$ where $N$ is the number of tasks. Prior works~\citep{kalashnikov2021mt,eysenbach2020rewriting,yu2021conservative} show that sharing data from different tasks to task $i$ to be conducive. To do so, these prior methods assume access to the functional form of the reward $r_i$. This is a strong assumption in practice, as it necessitates access to a functional (programmatic) form for the reward function. In offline RL, it might be desirable to simply label the reward function by hand, but then the algorithm does not have access to the functional form of the reward, and all unlabeled data also needs to be labeled by hand for use with such methods. Our aim in this paper is to utilize unlabeled data without any reward labels at all.
If however functional access to the reward \emph{is} available, a simple strategy is to na\"ively share data across all tasks, which we refer to as Sharing All. Formally, Sharing All defines the dataset of transitions relabeled from task $j$ to task $i$ as $\mathcal{D}_{j \rightarrow i}$ and the method can be then defined as
    $\mathcal{D}^\mathrm{eff}_i := \mathcal{D}_i \cup ( \cup_{j \neq i} \mathcal{D}_{j \rightarrow i})$,
where $\mathcal{D}^\mathrm{eff}_i$ denotes the effective dataset for task $i$. Therefore, the policy optimization objective in Sharing All can be written as follows:
\begin{equation*}
     \forall i \in [N], ~~\pi^*(\mathbf{a}|\bs, i) := \arg \max_{\pi}~~ J_{\mathcal{D}^\mathrm{eff}_i}(\pi) - \alpha D(\pi, \pi^\mathrm{eff}_\beta),
\end{equation*}
where $\pi_\beta^\mathrm{eff}(\mathbf{a}|\bs, i)$ is the effective behavior policy for task $i$ denoted as $\pi_\beta^\mathrm{eff}(\mathbf{a}|\bs, i) := |\mathcal{D}^\mathrm{eff}_i(\bs, \mathbf{a})| / |\mathcal{D}^\mathrm{eff}_i(\bs)|$, $J_{\mathcal{D}^\mathrm{eff}_i}(\pi)$ denotes the average return of policy $\pi$ in the empirical MDP induced by the effective dataset, and $D(\pi, \pi^\mathrm{eff}_\beta)$ denotes a divergence measure (e.g., KL-divergence~\citep{jaques2019way,wu2019behavior}, fisher divergence~\citep{kostrikov2021offline}, MMD distance~\citep{kumar2019stabilizing} or $D_{\text{CQL}}$ from conservative Q-values~\citep{kumar2020conservative}) between the learned policy $\pi$ and the effective behavior policy $\pi_\beta^\mathrm{eff}$. Note that conservative Q-values refer to the Q-value for a given policy corresponding to a modified reward function $r(\bs, \mathbf{a}) - \alpha \pi(\mathbf{a}|\bs) \cdot (\pi(\mathbf{a}|\bs) / \pi_\beta(\mathbf{a}|\bs) - 1)$, computed on the empirical MDP. We also note that Sharing All can be easily adapted to the single-task setting where there is only one target task with labeled data $\mathcal{D}_\text{L}$ and unlabeled prior data $\mathcal{D}_\text{U}$. While data sharing tends to show promising results, it requires the assumption of the access to the functional form of the reward function. We instead focus on the data sharing problem where we do not make such an assumption and instead, only have the reward labels for originally commanded task, which we will discuss in the following section.




\section{Out-of-Distribution Actions in Q-Learning}
\label{sec:Problem Description}

% Q-learning and other ADP methods which rely on iterating the Bellman backup operator are particularly susceptible to out-of-distribution inputs, because any errors incurred on these inputs can be propagated to neighbor states via the backup and keep compounding over iterations of the algorithm. Unfortunately, error on a single state can propagate to other states and can potentially cause inaccurate predictions across the entire Q-function. As we will show, these inaccuracies do affect the performance of off-policy algorithms in practice.

When Q-learning and off-policy actor-critic algorithms are used with static off-policy data, it's common to see returns improve at first and then deteriorate, or even deteriorate right from the start, as shown in Figure \ref{fig:divergence}. At first glance, this resembles overfitting, but increasing the size of the static dataset does not rectify the problem, suggesting the issue is more complex.
%When running Q-learning on a static off-policy dataset, we often find that the performance of the algorithm is poor and the performance doesn't change drastically through training (e.g., results for the na\"{i}ve RL method in Figure \ref{fig:divergence}) whereas the Q-values usually diverge over the course of training. However, unlike in supervised learning, increasing the size of the static off-policy dataset does not rectify the problem, suggesting the situation is more complex.
%These results suggest a form of overfitting, however, the situation is more complex than supervised learning. 
%as the ground truth performance curves resemble validation error curves during overfitting in supervised learning. 
%However, this interpretation does not tell the whole story: 
%while we can test for overfitting using the Bellman error, early stopping on Bellman error is ineffective [either cite something or add an experiment on this to the appendix].
% . , suggesting that simple overfitting is an inadequate explanation. 
\begin{wrapfigure}{r}{0.5\textwidth}
\vspace{-10pt}
\begin{center}
    \includegraphics[width=0.48\linewidth]{images/cheetah_divergence.pdf}
    ~
    \includegraphics[width=0.48\linewidth]{images/cheetah_divergence_q_val.pdf}
  \end{center}
 \vspace{-10pt}
 %%SL.5.22: Very important: the y-axes are not labeled right now, and it took me a while to figure out which plot was showing what. What is log(Q)? I guess you're trying to show that the right plot has Bellman error (?), while the left has performance? A couple more things: (1) always put space before ( (you often omit this space) (2) consider a caption like this (once the figures are labeled more clearly): Off-policy learning with SAC on HalfCheetah-v2 for different dataset sizes ($n$). The performance (left) does not correlate with $n$, while the Q-values (right) diverge or saturate at values far from the actual return.
  \caption{Performance of SAC on HalfCheetah-v2 with off-policy expert data w.r.t. number of training samples ($n$). Note the large discrepancy between returns (which are negative) and logarithm of Q-values (which converge to large positive values or diverge) that is not solved with additional samples.} 
 \vspace{-15pt}
 \label{fig:divergence}
\end{wrapfigure}

We can understand the source of instability by examining the form of the Bellman backup. Although minimizing the mean squared Bellman error corresponds to a supervised regression problem, the targets for this regression are themselves derived from the current Q-function estimate. The targets are calculated by maximizing the approximate $Q$-function with respect to the action at the next state. However, the $Q$-function estimator is only reliable on inputs from the same distribution as its training set. As a result, na\"{i}vely maximizing the value may evaluate the $\hat{Q}$ estimator on actions that lie far outside of the training distribution, resulting in pathological values that incur large error. We refer to these actions as out-of-distribution (OOD) actions, and we call errors due to OOD actions \textit{boostrapping errors}. This is because not only do they produce inaccurate values on the states where the backup is computed, these errors will propagate on subsequent Bellman backups. 
If $\valerr_k(s) = |Q_k(s,a) - Q^*(s,a)|$ denotes the total error at iteration $k$ of Q-learning and $\projerr_k(s, a) = |Q_k(s,a) - \backup Q_{k-1}(s,a)|$ denote the current Bellman error, we can write $\valerr_k(s) \le \projerr_k(s,a) + \gamma \max_{a'} E_{s'}[\valerr_{k-1}(s',a')]$. This means errors from $(s', a')$ are discounted, then accumulated into $Q(s,a)$ in addition to new projection errors $\projerr_k(s, a)$ being introduced on the current iteration. $\projerr$ is expected to be high on OOD states and actions
%\TODO{gjt: explain this} 
as errors at these states-action pairs are never minimized during training.
%Furthermore, these errors are propagated to other states through the backup operator.

To mitigate bootstrapping errors, we can restrict the policy/actor to ensure that they output actions that lie in the support of training distribution. 
% \TODO{Isn't that just batch constrained Q-learning? Just restricting the actions to those in the training set. -- changed to distribution: addressed}. 
This is distinct from previous work (e.g.,~\citep{fujimoto2018off}) which constrains the \emph{distribution} of the learned policy to be close to the behavior policy, similarly to behavioral cloning~\cite{Schaal99isimitation}.
While this is sufficient to ensure that actions lie in the training set with high probability, it is overly restrictive. For example, if the behavior policy is close to uniform, the learned policy will behave randomly, resulting in poor performance, even when the data is sufficient to learn a strong policy (see Figure~\ref{fig:gridworld}.
for an illustration). The key distinction is that we restrict the support of the learned policy, but not the probabilities of the actions within the support.
Restricting the actions reduces bootstrapping error as the Q-function is no more queried on OOD actions. However, it may also prevent the algorithm from converging to the optimal $Q^*$. In the next subsection, we theoretically analyze this tradeoff.


%In order to formally analyze this problem, we perform an error propagation analysis of Q-learning on the lines of Approximate Value Iteration (AVI)~\cite{munos2003errorapi} and Approximate Policy Iteration (API)~\cite{bruno2015approximate}. Let $Q_1, \cdots, Q_K$ be the value-function iterates and $\pi_1, \cdots,\pi_k$ be the policy-iterates generated when performing actor-critic based Q-learning, which a special case of API. We can express the \emph{policy evaluation error} at iteration $k$ as $\valerr_k(s, a) = |Q_k(s, a) - Q^\pi(s, a)|$, and the \emph{projection error} as $\projerr_k(s, a) = |Q_k(s, a) - \Tpi Q_{k-1}(s, a)|$. 
%Then, we have $\valerr_k(s, a) \le \delta_k(s, a) + \gamma E_{s', \pi}[\valerr_{k-1}(s', a')]$ (see Appendix~\ref{app:error_prop} for details). In other words, approximation errors $\projerr_k(s, a)$ are introduced during the projection step, discounted, and propagated to neighboring states via the backup operator. Understanding the source of the errors and controlling them is key to producing a stable algorithm.

%When using Q-function values on actions that greedily maximize the value at the next state $s'$ ($\max_{a'} Q(s', a')$) as target values for Q-learning, the maximizing action at $s'$ can potentially be very different from the distribution of actions at state $s'$ defined by dataset distribution $\dataset$. Such actions that are very unlikely to have been sampled from the dataset distribution are called out-of-distribution (OOD) actions. 
%%SL.5.20: Can we formally define what that means, instead of just saying they are called this? -- i don't think so that we can formally define OOD in general, without going into some hypothesis testing thing.
%As neural nets are known to produce inaccurate results when queried on out-of-distribution inputs -- adversarial examples~\citep{goodfellow2015advexamples} are a well-known example of this phenomenon, Q-values corresponding to OOD actions are not accurate and reliable. This also means that using such Q-values for Bellman backups tends to destabilize Q-value estimates. Empirically, we find that OOD actions are a major source of error that arise in Q-learning style ADP methods with static-datasets. The error accumulated in the Q-function due to backups from OOD actions is called \textit{boostrapping error}. In Figure~\ref{fig:gridworld}, the top row demonstrates how error can propagate between states due to the bootstrapping process. We next propose to restrict the policy $\pi$ during policy improvement step and while computing the backup, so as to limit the amount of errors incurred due to OOD actions, which we discuss in the following sections.


%%%%%%%%%%%%%%%%%%%%%%%%%% OLD: Monday 7:04 pm
% In this section, we describe how errors occurring in the Q-function due to bootstrapping errors from certain
% %%SL.5.20: This somewhat contradicts what we wrote in the related work section -- we said the Fujimoto analysis is at the level of sets, while ours is on distributions, but now we are talking about sets too?
% actions -- which we call out-of-distribution actions -- can accumulate and hurt the performance of off-policy algorithms in practice to a major extent. We start by revisiting the study of error propagation in ADP methods.
% %%SL.5.20: I think it takes us way too long to get to the point here. Can we have a more focused opening paragraph that specifically talks about what we'll be analyzing and doing?

% %%SL.5.20: Is there any way we can move this discussion to related work? I think it breaks the flow to have a little "mini related work" at the top of the technical section. Especially after such a lengthy setup, many readers will get annoyed and wonder when you'll get around to telling them what you actually do.
% We analyse off-policy static-dataset Q-learning algorithms as specific instances of approximate value iteration (AVI)~\citep{munos2003errorapi} and approximate policy iteration (API)~\citep{bruno2015approximate}. 
% %%SL.5.20: Do you really need all this API and AVI stuff? Why not keep it simple and just discuss Q-learning?

% %%SL.5.20: Maybe we should have two subsections here -- a 4.1 that starts here and explains the problem, and a 4.2 (current 4.1) that explains the solution.

% Errors encountered during Q-learning
% %%SL.5.20: Just say Q-learning...
% %\TODO{not defined yet} 
% propagate 
% between neighbor states $s'$ and $s$ when performing a Bellman backup.
% %%SL.5.20: I feel like the above sentence is just a really long-winded way to say "The Bellman backup results in compounding errors." Try to rephrase sentences to be more concise, avoid unnecessary words.
% To formalize this, let $Q_1, \cdots, Q_K$ be the value-function iterates and $\pi_1, \cdots,\pi_k$ be the policy-iterates generated when performing actor-critic based Q-learning. We can express the \emph{policy evaluation error} at iteration $k$ as $\valerr_k(s, a) = |Q_k(s, a) - Q^\pi(s, a)|$, and the \emph{projection error} as $\projerr_k(s, a) = |Q_k(s, a) - \Tpi Q_{k-1}(s, a)|$. 
% Then, we have $\valerr_k(s, a) \le \delta_k(s, a) + \gamma E_{s', \pi}[\valerr_{k-1}(s', a')]$ (see Appendix~\ref{app:error_prop} for details).
% % \TODO{all for a fixed $\pi$, how is this related to Q learning}
% % \TODO{define $V_k$, etc.}
% In other words, approximation errors $\projerr_k(s, a)$ are introduced during the projection step, discounted, and propagated to neighboring states via the backup operator. Understanding the source of the errors and controlling them is key to producing a stable algorithm.
% %%SL.5.20: This seems like a reasonable statement, but I think readers will be lost at this point about where you are going. Maybe you can preface the above paragraph by saying something like this: When training on off-policy data, we often see actual policy performance improving briefly and then deteriorating sharply (see, e.g., results for the na\"{i}ve RL method in Figure ???). One might at first surmise that these issues are a form of overfitting, as the ground truth performance curves resemble validation error curves during overfitting in supervised learning. However, this interpretation does not tell the whole story: while we can test for overfitting using the Bellman error, early stopping on Bellman error is ineffective [either cite something or add an experiment on this to the appendix]. Furthermore, even a very large off-policy training set does not avoid this degradation problem, suggesting that simple overfitting is an inadequate explanation. We can obtain a better explanation from examining the form of the Bellman backup. Note that minimizing the Bellman error corresponds to a supervised regression problem, but the targets for this regression are themselves derived from the current Q-function estimate. We argue that it is these estimates themselves that are the sources of the degradation: they are calculated by finding the action that maximizes the value at the next state. However, the value estimate is obtained from a function approximator, and this function approximator is only reliable on inputs from the same distribution as its training set. Since Q-learning only trains the Q-function via regression on state-action tuples seen in the training data, the actions that maximize the target value might lie very far outside of the training distribution, and therefore might incur very large error. We can formally analyze this source of error as follows. [and then talk about delta etc]

% These errors $\valerr_k$ arise from a multitude of sources and include function approximation error, sampling error, distribution shift error, and bootstrapping error. When learning from static, off-policy datasets, sampling error is largely uncontrollable, and it is hard to provide guarantees about function approximation error with deep neural nets.
% %%SL.5.20: I think this is misleading. While it's true that we cannot provide guarantees, in general function approximation error with large function approximators is low. You can reference our debugging paper for this.
% When training Q-functions using $(s, a, r, s')$ tuples from the dataset, the fixed point iteration scheme
% %%SL.5.20: Which fixed point iteration scheme?
% queries and uses the Q-function value on actions that greedily maximize the value at the next state $s'$ ($\max_{a'} Q(s', a')$). However, the maximizing action at $s'$ can potentially be very different from the distribution of actions at state $s'$ defined by dataset distribution $\dataset$. Such actions that are very unlikely to have been sampled from the dataset distribution are called out-of-distribution (OOD) actions.
% %%SL.5.20: Can we formally define what that means, instead of just saying they are called this?
% Neural nets are known to produce inaccurate results when queried on out-of-distribution inputs -- adversarial examples~\citep{goodfellow2015advexamples} are a well-known example of this phenomenon.
% %%SL.5.20: That's a good explanation
% Regressing to the target-value computed using such OOD actions coupled with the $\max$ step can lead to an accumulation of error in the Q-function by virtue of error propagation.
% %%SL.5.20: Now it's again unclear whether you are talking about the same phenomenon, or something else. Basically, the logical connection between the sentence "Neural nets.." and "Regressing.." is missing.
% %\TODO{is $\pidata$ a single policy or set? -- set}.
% The problem is further exacerbated by the fact that updates are only made to Q-values of state-action pairs present in $\dataset$, so even though the Bellman backup queries the Q-function for OOD actions, it never \emph{trains} it on those actions. As deep neural-net function approximators can often generalize in undesired and unpredictable ways, such backups can destabilize learning and can lead to divergence in Q-functions.

% Empirically, we find that OOD actions are a major source of error that arise in Q-learning style ADP methods with static-datasets. The error accumulated in the Q-function due to backups from OOD actions is called \textit{boostrapping error}. In Figure~\ref{fig:gridworld}, the top row demonstrates how error can propagate between states due to the bootstrapping process.
% %%SL.5.20: Can you add your quantative "badness" experiments, at least to an appendix, and reference it here

% We next propose to restrict the policy $\pi$ during policy improvement step and while computing the backup, so as to limit the amount of errors incurred due to OOD actions, which we discuss in the following sections.
%%SL.5.20: These are two separate things: one thing is to restrict the policy, the other is to restrict the backup. Should we more explicitly separate these things?
%%%%%%%%%%%%%%%%%%%%%%%%%%%%%%%%%%%%%%%%%

%%%%%%%%%%%%%%%%%%%%%%%% OLD: Sunday 05/19 9pm%%%%%%%%%%%%%%%
% When training Q-functions using $(s, a, r, s')$ tuples, the fixed point iteration scheme queries and uses the Q-function value on actions that greedily maximize the value at the next state $s'$ ($\max_{a'} Q(s', a')$) \TODO{this may be an appropriate time to talk about adverserial examples as the actor does end up exploiting inaccuracies in Q}. However, the maximizing action at $s'$ can potentially be very different from the distribution of actions defined by the behaviour policy $\pidata(\cdot|s')$ \TODO{is $\pidata$ a single policy or set?}. Moreover, updates are only made to Q-values of state-action pairs present in $\dataset$. As deep neural-net function approximators can often generalize in undesired and unpredictable ways, regressing to the target-value computed using out-of-distribution actions coupled with the $\max$ step can lead to an accumulation of error in the Q-function. We refer to this problem as the out-of-distribution actions problem in Q-learning.

% \TODO{I suggest rewriting this section as: 1) Errors in Q functions are bad in ADP because they propogate, 2) general error prop result, 3) where do errors come from, 4) OOD errors are a big problem}

% In this section, we describe the out-of-distribution action problem and motivate a solution via constraining action choices in the Bellman backup operator.
% To get started, it is important to mention that, machine learning algorithms are known to produce inaccurate results when queried on out-of-distribution inputs -- adversarial examples~\citep{goodfellow2015advexamples} are a well-known example of this phenomenon. \TODO{gjt: unclear why this is important to mention there, the following is relevant to all sources of error in the learned Q (which you state could come from many sources later).}
% Q-learning and other ADP methods which rely on iterating the Bellman backup operator are particularly susceptible to out-of-distribution inputs, because any errors incurred on these inputs can be propagated to neighbor states via the backup and keep compounding over iterations of the algorithm. Unfortunate error on a single state can potentially cause inaccurate predictions across the entire Q-function. As we will show, these inaccuracies do affect the performance of off-policy algorithms in practice. 

% %In this section we will describe the problem we are aiming to solve. 
% When training Q-functions using $(s, a, r, s')$ tuples, the fixed point iteration scheme queries and uses the Q-function value on actions that greedily maximize the value at the next state $s'$ ($\max_{a'} Q(s', a')$) \TODO{this may be an appropriate time to talk about adverserial examples as the actor does end up exploiting inaccuracies in Q}. However, the maximizing action at $s'$ can potentially be very different from the distribution of actions defined by the behaviour policy $\pidata(\cdot|s')$ \TODO{is $\pidata$ a single policy or set?}. Moreover, updates are only made to Q-values of state-action pairs present in $\dataset$. As deep neural-net function approximators can often generalize in undesired and unpredictable ways, regressing to the target-value computed using out-of-distribution actions coupled with the $\max$ step can lead to an accumulation of error in the Q-function. We refer to this problem as the out-of-distribution actions problem in Q-learning.
% %%SL.5.15: Make sure you introduce your terminology! While it may be obvious what \pi_{data} means, it's still better to define.
% %for that state. 
% %%SL.5.15: I would suggest defining new commands for commonly used symbols (like \pi_{data}) so that it's easy to change and you don't have to constantly type these things. That also makes it convenient to properly use \mathbf for vectors, etc. Learning how to use macros like that in latex is important for good typesetting and flexible refactoring of symbols.
% %%SL.5.15: Somewhere there is a "missing punchline" in the above paragraph. Can you add a sentence at the end about what's the point? Don't leave that implicit.
% %%SL.5.11: can we write this in terms of distributions rather than sets? this is important, because we care about things being out-of-distribution, not just out-of-sample -- that's an important distinction from regular overfitting
% %Further,
% %%SL.5.15: I'm not sure if "Also" is the right word with which to begin this transition, can you think of a way to change this transition to reflect the logical progression of the argument?
% %with fixed datasets ADP algorithms would perform fixed point updates on only those transitions that are present in $\dataset$, so Q-values for state-action pairs which are out of the dataset distribution are never updated. This results in inaccurate values for $Q(s, a)~ \forall~ (s, a) ~~\text{s.t.}~~ \rho_{\dataset}(s, a) \leq \varepsilon$. The fixed point backup for Q-learning would still end up querying and backing up these incorrect Q-values without updating them ever during training. This leads to an accumulation of error as more and more steps of ADP are performed, making Q-learning in batched, off-policy
% %%SL.5.15: Off-policy is not synonymous with batched
% %settings highly sensitive to initialization and prone to instabilities. This often manifests as diverging Q-functions or excessively overestimated Q-value estimates~\cite{fujimoto18addressing},
% %%SL.5.15: maybe not the right citation for this? there must be many works that point this out
% %and as we will show, is a serious problem with most batched Q-learning applications. We call this problem Out-of-distribution Actions Problem. This problem exists in Q-learning as there is no implicit normalization mechanism on Q-functions as opposed to policies which are probability distributions and must integrate to 1, hence, learning Q-values for good actions by performing dynamic programming doesn't ensure other Q-values are correct.
% %%SL.5.15: I don't think we need to capitalize this.
% %Next, we analyze this problem from the perspective of error-propagation, which has been used in the analysis of approximate dynamic programming methods. Then we discuss what implications our analysis has in deigning a practical algorithm for reducing bootstrapping error.
% %%SL.5.15: why is it significant that this area is well studied?

% %%SL.5.15: Perhaps for next section, you mean to create a subsection rather than a new top-level section? I also think that the current opening here is not very informative. It doesn't really say what the rest of this section will be discussing.

% %%SL.5.11: somehow it's not obvious to me what point above sentence is trying to make -- are you saying that only target value matters and not current value? if so, try to make that explicit

% %%SL.5.11: allude to some evidence we will show of this?


% %%SL.5.11: rewrite above w/o using "counterfactual" somehow, it seems really confusing

% %%SL.5.11: very run-on sentence


% %%SL.5.15: I got this far on 5/15, will continue tomorrow
% %\subsection{Error Propagation and the out-of-distribution action problem}

% In order to formalize the out-of-distribution actions problem, we turn to the study of error propagation in ADP, which provides machinery for understanding how approximation errors propagate through a repeated bootstrapping process. We analyse off-policy static-dataset Q-learning algorithms as specific instances of Approximate Value Iteration(AVI)~\citep{munos2003errorapi} and Approximate Policy Iteration(API)~\citep{bruno2015approximate}. \TODO{maybe mention 1 line about what API and AVI do}

% Errors encountered during API or AVI 
% %\TODO{not defined yet} propagate 
% between neighbor states $s'$ to $s$ when an action is selected which results in the agent visiting $s'$. We can express the \emph{policy evaluation error} at iteration $k$ as $\valerr_k(s) = |V_k(s) - V^\pi(s)|$, and the \emph{projection error} as $\projerr_k(s) = |V_k(s) - \Tpi V_{k-1}(s)|$. 
% Then, we have $\valerr_k(s) \le \delta_k(s) + \gamma E_{\pi}[\valerr_{k-1}(s')]$ (see Appendix~\ref{app:error_prop} for details).
% % \TODO{all for a fixed $\pi$, how is this related to Q learning}
% \TODO{define $V_k$, etc.}
% In other words, approximation errors are introduced during the projection step ($\projerr_k(s)$), discounted, and propagated to neighboring states via the backup operator. Understanding the source of the errors and controlling them is key to producing a stable algorithm.

% In the case of static datasets, we empirically find that the major source of this error is a result of using out-of-distribution actions \TODO{define what an OOD action is} to compute the backup. Figure~\ref{fig:gridworld} demonstrates the accumulation of such error in the case of gridworlds. \TODO{one more sentence here}

% We propose to carefully restrict the policy $\pi$ during policy improvement and while computing the backup, so as to limit the amount of errors incurred due to OOD actions, which we discuss in the following sections.
% \TODO{so far, only talking about approx error, not specific to OOD error}
%%%%%%%%%%%%%%%%%%%%%%%%%%%%%%%%%%%%%%%%%%%%%%%%%%%%%%%

%\subsection{Set-Constrained Backups}
%\label{sec:set_constrained_backup}
%In order to formally analyze this problem, we perform an error propagation analysis of Q-learning on the lines of Approximate Value Iteration (AVI)~\cite{munos2003errorapi} and Approximate Policy Iteration (API)~\cite{bruno2015approximate}. Let $Q_1, \cdots, Q_K$ be the value-function iterates and $\pi_1, \cdots,\pi_k$ be the policy-iterates generated when performing actor-critic based Q-learning, which a special case of API. We can express the \emph{policy evaluation error} at iteration $k$ as $\valerr_k(s, a) = |Q_k(s, a) - Q^\pi(s, a)|$, and the \emph{projection error} as $\projerr_k(s, a) = |Q_k(s, a) - \Tpi Q_{k-1}(s, a)|$. 
%Then, we have $\valerr_k(s, a) \le \delta_k(s, a) + \gamma E_{s', \pi}[\valerr_{k-1}(s', a')]$ (see Appendix~\ref{app:error_prop} for details). In other words, approximation errors $\projerr_k(s, a)$ are introduced during the projection step, discounted, and propagated to neighboring states via the backup operator. Understanding the source of the errors and controlling them is key to producing a stable algorithm.


%%SL.5.20: Try to be careful to distinguish this from Fujimoto -- all this stuff that treats "out of distribution" as a set will come across as rather disappointing.
%We believe this to be an effective strategy because while it is difficult to prevent errors from arising (i.e. due to lack of data, or other approximation errors), we can prevent the propagation of errors through better action selection. How should we select actions? We should select actions from all policies that could have possibly generated that action -- policies for which the sampled action lies in their high-confidence support-set.

% First, we introduce the notion of set-constrained Bellman operators,
% %%SL.5.20: This kind of makes it sound like it's basically a copy of the Fujimoto paper. Can we stop talking about sets? This would also go a lot better if before introducing this, you provide a bit of discussion that motivates what is coming next.
% which restricts the set of policies in the maximization.%, then describe how this reduces error propagation compares to using an unmodified Bellman backups and finally utilize them to build a practical algorithm. 
% \begin{definition}[Set-constrained operators]
% Given a set of policies $\Pi$ 
% %, we define the set-constrained policy improvement operator as:
% %$\greedyPi(Q, s) = \argmax{\pi \in \Pi}~ \expec_{a \sim \pi}[Q(s, a)]$,
% %and 
% , the set-constrained backup operator is:
% \[ \TPi Q(s, a) \coloneqq \expec \left[ R(s, a) + \gamma \max_{\pi \in \Pi} \expec_{\pi(a' | s')T(s' | s, a)}\left[Q(s', a') \right] \right]. \]
% \end{definition}
% Although this is a general definition, we are interested in the case where $\Pi = \{ \pi ~|~ \pi( a | s) = 0 \text{ whenever } \beta( a | s) < \epsilon \}$, where $\beta$ is the behavior policy (i.e., the set of policies that have support in the probable regions of the behavior policy). Restricting the backup operator may prevent the algorithm from converging to the optimal Q-function. To capture this, we define a \emph{suboptimality constant}, which measures how far $\pi^*$ is from $\Pi$. %Specifically, it is the difference between the expected return when following a maximizing policy in $\Pi$ and $\pi^*$ for one step and then following $\pi^*$ thereafter.
% \begin{definition}[Suboptimality constant]
% The suboptimality constant is defined as:
% \[ \alpha(\Pi, s, a) = |\TPi Q^*(s, a) - T^* Q^*(s, a)|. \]
% \end{definition}
% Note that if $\pi^*$ is contained within $\Pi$, then the suboptimality constant is 0.

% \TODO{Justin: change from here}
% We can quantify the suboptimality of \emph{set-constrained Q-learning}. For clarity, we present results using $\ell^\infty$-norm bounds in the main text, which demonstrate the intuitions underlying the method, and refer readers to Appendix~\ref{app:error_prop} for more practically applicable $\ell^p$-norm bounds, as well as analogous results for set-constrained API. 
%%SL.5.20: I think some readers will feel a bit lost here, because it's not completely obvious what the significance of these results are. Perhaps it would be better if you explain the method first (this is kind of implied -- just constrain the policy to the set -- but it should be made explicit).

% \begin{theorem}[Set-constrained error propagation (API)]
% Suppose we run approximate value iteration with the set-constrained backup operator $\TPi$ and obtain a sequence of Q-functions $Q_0, Q_1, ... Q_K$ and a sequence of policies $\pi_1, \cdots, \pi_K$ and denote $Q^{\pi_i}$ as the true Q-function for policy $\pi_i$. Let $\delta_k(s, a) = |Q_k(s, a) - \TPi Q_{k-1}(s, a)|$, and $\valerr(s, a) = |Q_k(s, a) - Q^*(s, a)|$. Then,
% {$$\valerr_k(s, a) \le \projerr_k(s, a) + \alpha(\Pi, s, a) + \gamma \max_{\pi \in \Pi} \expec_{\pi, s'}[\valerr_{k-1}(s', a')], $$}
% and 
%\TODO{this bound actually doesn't illustrate error reduction - probably need Lp norm bounds}
%$$ Q^* - Q^{\pi_k} \leq  $$
% \[\lim_{k \to \infty} \norminf{V_k - V^*} \le \frac{2\gamma }{(1-\gamma)^2}\left(\max_{s, k} \delta_k(s) + \alpha(\Pi,s)\right). \]
% \end{theorem}
% \begin{proof}
% See Appendix~\ref{app:error_prop}.
% \end{proof}

% Our strategy for reducing error propagation is to restrict the policy improvement and the Bellman Backup operators from selecting actions which may query the Q-function on out-of-distribution actions. First, we introduce the notion of set-constrained Bellman operators,
% %%SL.5.20: This kind of makes it sound like it's basically a copy of the Fujimoto paper. Can we stop talking about sets? This would also go a lot better if before introducing this, you provide a bit of discussion that motivates what is coming next.
% which restricts the set of policies in the maximization, then describe how this reduces error propagation compares to using an unmodified Bellman backups and finally utilize them to build a practical algorithm.  \begin{definition}[Set-constrained operators]
% Given a set of policies $\Pi$ 
% %, we define the set-constrained policy improvement operator as:
% %$\greedyPi(Q, s) = \argmax{\pi \in \Pi}~ \expec_{a \sim \pi}[Q(s, a)]$,
% %and 
% , the set-constrained backup operator is:
% \[ \TPi Q(s, a) \coloneqq \expec \left[ R(s, a) + \gamma \max_{\pi \in \Pi} \expec_{\pi(a' | s')T(s' | s, a)}\left[Q(s', a') \right] \right]. \]
% \end{definition}
% Although this is a general definition, as we will see, we are interested in the case where $\Pi = \{ \pi ~|~ \pi( a | s) = 0 \text{ whenever } \beta( a | s) < \epsilon \}$, where $\beta$ is the behavior policy (i.e., the set of policies that have support in the probable regions of the behavior policy). Restricting the backup operator may prevent the algorithm from converging to the optimal Q-function. To capture this, we define a \emph{suboptimality constant}, which measures how far $\pi^*$ is from $\Pi$. %Specifically, it is the difference between the expected return when following a maximizing policy in $\Pi$ and $\pi^*$ for one step and then following $\pi^*$ thereafter.
% \begin{definition}[Suboptimality constant]
% The suboptimality constant is defined as:
% \[ \alpha(\Pi, s, a) = |\TPi Q^*(s, a) - T^* Q^*(s, a)|. \]
% \end{definition}
% Note that if $\pi^*$ is contained within $\Pi$, then the suboptimality constant is 0. A natural question now is whether we can ever improve upon constraining the set $\Pi$ to just be the behaviour policy $\beta(a|s)$. The answer to this is yes, and this can be explained via the following bound.

% \begin{proposition}
% \TODO{Justin: define symbols}
% Suppose we run \TODO{policy evaluation?} using $\pi$. Let the projection error be bounded each step as $|V_k - T^{\pi}V_{k-1}| \le \epsilon$. Then,
% \[
% \lim_{k \to \infty} \rho_0 |V_k - V^*| \le \rho^{\pi} \epsilon + |\rho^\pi - \rho^{\pi^*}|\Rmax
% \]
% \end{proposition}
% \begin{proof}
% See Appendix~\ref{app:error_prop}.
% \end{proof}

%The above bound tells us that backing up from the a policy that differs from the behaviour policy, in a lot of cases, can bring about a larger decrease in the suboptimality bias than increase in the amount of error incurred. In the next subsection, we see how to choose such a policy.   

\subsection{Distribution-Constrained Backups}
\label{sec:dist_constrained}
In this section, we describe a backup that restricts the set of policies in maximization, and provide performance bounds which depend on the choice of policy set chosen. Then, we motivate why using the support of the data is a reasonable choice for constructing the constraint set. We begin with the definition of a distribution-constrained operator, which only performs backups according to a restricted set of policies:
\begin{definition}[Distribution-constrained operators]
Given a set of policies $\Pi$ 
%, we define the set-constrained policy improvement operator as:
%$\greedyPi(Q, s) = \argmax{\pi \in \Pi}~ \expec_{a \sim \pi}[Q(s, a)]$,
%and 
, the distribution-constrained backup operator is:
\[ \TPi Q(s, a) \coloneqq \expec \big[ R(s, a) + \gamma \max_{\pi \in \Pi} \expec_{\pi(a' | s')T(s' | s, a)}\left[Q(s', a') \right] \big] \]
\end{definition}

To analyze the behavior of this backup, we first quantify two sources of error. The first is a \emph{suboptimality bias} -- the optimal policy may not exist inside the chosen policy set and thus a suboptimal solution will be found. The second arises from distribution shift between the training distribution and the policies used for backups -- this formalizes the notion of OOD actions. %and states.
To capture suboptimality in the final solution due to this restriction, we define a \emph{suboptimality constant}, which measures how far $\pi^*$ is from $\Pi$. 
\begin{definition}[Suboptimality constant]
The suboptimality constant is defined as:
\[ \alpha(\Pi) = \max_{s,a} |\TPi Q^*(s, a) - T^* Q^*(s, a)|. \]
\end{definition}
The best possible value of of this constant is $\alpha(\Pi) = 0$, which corresponds to the case when $\pi^*$ is contained within $\Pi$. Next, we define a concentrability coefficient~\citep{munos2005erroravi} which quantifies a ratio of how far the policy's visitation distribution is from the data. This constant captures the notion of the degree to which states and actions are out-of-distribution.
\begin{assumption}[Concentrability]
Let $\rhoinit$ denote the initial state distribution, and $\mu$ denote the distribution of the training data over $\mathcal{S} \times \mathcal{A}$. For all $\pi_1, ... \pi_k, \pi_{k+1}$ such that $\pi_i \in \Pi$, and define the operator $A^{\pi}$ as: $A^{\pi}: \mathcal{S} \rightarrow \mathcal{S} \times \mathcal{A}$, then, assume there exists coefficients $c(m)$ such that:
\[
\rhoinit P^{\pi_1}P^{\pi_2}...P^{\pi_k} A^{\pi_{k+1}} \le c(k) \mu
\]
Correspondingly, the concentrability coefficient $C(\Pi)$ is defined as
\[
C(\Pi) \defeq (1-\gamma)^2\sum_{k=1}^\infty k\gamma^{k-1}c(k)
\]
\end{assumption}
To provide some intuition on the value of $C(\Pi)$, we can see that if $\mu$ was generated by some policy $\pi$, and $\Pi = \{\pi\}$ is a singleton set, then we would have $C(\Pi)=1$, which is the smallest possible value. However, if $\Pi$ contained policies significantly far from $\pi$, the value could be potentially large, in the limit, tending to $\infty$ if support of $\Pi$ is not contained in $\Pi$. With $C(\Pi)$ and $\alpha(\Pi)$ defined, we can now bound the performance of approximate distribution-constrained Q-iteration:
\begin{theorem}
\label{thm:avi_bound}
Suppose we run approximate distribution-constrained Q-iteration with a set constrained backup $\TPi$. Assume that $\delta(s,a) \ge \max_k |Q_k(s,a) - \TPi Q_{k-1}(s,a)|$ bounds the Bellman error. Then,
\[\lim_{k \to \infty} \expec_{\rhoinit, \pi something}[|Q_k(s,a) - Q^*(s,a)|] \le 
\frac{\gamma}{(1-\gamma)^2}\left[ C(\Pi)\expec_\mu[\projerr(s,a)] + \alpha(\Pi) \right]
\]
\end{theorem}
\begin{proof} See Appendix~\ref{app:error_prop}, Thm.~\ref{thm:avi_bound_proof} \end{proof}

This bound formalizes the tradeoff between keeping policies chosen during backups close to the data (captured by $C(\Pi)$) and keeping the set $\Pi$ large enough to capture well-performing policies (captured by $\alpha(\Pi)$). When we expand the set of policies $\Pi$, we are increasing $C(\Pi)$ but decreasing $\alpha(\Pi)$. An example of this tradeoff, and how a careful choice of $\Pi$ can yield superior results, is given in a toy gridworld example in Fig.~\ref{fig:gridworld}, where we visualize errors accumulating during distribution-constrained Q-iteration for different choices of $\Pi$. 

Finally, we motivate the use of support sets to construct $\Pi$. We are interested in the case where $\Pi_\epsilon = \{ \pi ~|~ \pi( a | s) = 0 \text{ whenever } \beta( a | s) < \epsilon \}$, where $\beta$ is the behavior policy (i.e., $\Pi$ is the set of policies that have support in the probable regions of the behavior policy). 

Why should we use support in order to construct $\Pi$ in the distribution-constrained operator? Keeping policies in the support of the data distribution is a reasonable choice as it allows us to bound the concentrability coefficient:
\begin{theorem}
\label{thm:conc_coeff_bound}
Assume the data distribution $\mu$ is generated by a policy $\beta$, such that $\mu(s,a) = d_\beta(s,a)$. Let us define $\Pi_\epsilon = \{ \pi ~|~ \pi( a | s) = 0 \text{ whenever } \beta( a | s) < \epsilon \}$. Then, the concentrability coefficient is bounded as:
\[
C(\Pi_\epsilon) \le \TODO{Justin}
\]
\end{theorem}
\begin{proof} See Appendix~\ref{app:error_prop} \end{proof}

Thus, using support sets gives us a single lever, $\epsilon$, which simultaneously trades off the value of $C(\Pi)$ and $\alpha(\Pi)$. Not only can we provide theoretical guarantees, we will see in our experiments (Sec.~\ref{sec:experiments}) that constructing $\Pi$ in this way provides a simple and effective method for implementing distribution-constrained algorithms.

\begin{figure}
    \centering
    \includegraphics[width=0.9\textwidth]{images/gridworld}
    \caption{Visualized error propagation for various choices of the constraint set $\Pi$
    - unconstrained (AVI,
    %%SL.5.22: is AVI ever defined?
    \TODO{Justin: define AVI, or lets call it Q-learning}
    top row), support-constrained (middle),
    and constraining to the behaviour policy (policy-evaluation, bottom). Dark values represent high error and light values represent low error. The task (leftmost image) is to reach the bottom-left corner from the top-left, but the behaviour policy (visualized as arrows in the task image, support shown in black on the support set image) travels to the bottom-right with a small amount of $\epsilon$-greedy exploration. Standard AVI propagates large errors from the low-data regime into the high-data regime, leading to inaccurate value estimates. Policy-evaluation reduces error propagation from low-data regimes but introduces significant suboptimality bias as the data policy is not optimal. A carefully chosen support-constrained backup strikes a balance between these two extremes by confining error propagation to the low-data region while introducing minimal suboptimality bias.}
    \label{fig:gridworld} 
\end{figure}

% \subsection{Choosing Backup Policies for OOD Action Error Reduction}
% \label{sec:choosing_policies}
% Argument in Sec.~\ref{sec:tradeoff} tells us that, with a careful selection of the policy under which the target value is computed, the overall error of value estimates from the optimal value function $\|V^* - V_k\|$ can be reduced. How should we search for a policy that minimizes the overall error? Our choice is to backup from policies which maintain high-support over the action set of the data.
% %%SL.5.22: I think it's not obvious to readers that "policy for the backup" means the distribution over the actions under which the target value is calculated. -- addressed

% To justify this choice,
% %%SL.5.22: What choice? -- choice of backing up from any policy that maintains high support over data.
% we note that the error analysis relies on being able to quantify $\delta_k(s, a)$ (the per-state-action bellman error) for OOD actions. Outside of the support of the data distribution, it is hard to provide guarantees on $\delta_k$. However, when $a$ lies inside the support of the training distribution for a given state $s$, high-capacity function approximators trained with supervised learning are expected to produce a bounded error, given enough samples.
% %Even if they don't produce bounded error on such in-support inputs, techniques such as Prioritized Replay~\cite{Schaul2016PrioritizedER} can be employed to ensure bounded error on all in-support inputs. 
% %Furthermore, often the quantity of interest is the Bellman error weighted by the inverse density of the behaviour policy~\cite{antos07fitted}, which depends only on the support of the behaviour policy and this error metric is the equal for two policies provided they share the same support.
% Therefore, backing up from all actions that have non-negligible support under the training distribution is sufficient (but not necessary) to prevent error accumulation. Hence, we restrict the set $\Pi$
% %%SL.5.22: Did we define \Pi before? since we cut the set backup operator stuff, now this is much harder to follow. Maybe we can bring it back (but call it something else)?
% of policies used for distribution-constrained backups to the set of policies that are supported on the probable regions of the behaviour policy. That is, $\Pi = \{ \pi | \pi( a | s) = 0 \text{ whenever } \beta( a | s) < \epsilon \}$, where $\beta$ is the behavior policy (i.e., the set of policies that have support in the probable regions of the behavior policy). This means that we are allowed to backup from any action distribution supported over the support of the behaviour policy. Previous work~\cite{fujimoto2018off} restricts the choice of actions to be a distribution close to the behaviour policy. 

%%SL.5.22: I don't really understand what the above paragraph is saying. Read literally, it seems to say "prior work does something similar, and in the worst case we are equally bad." That's not very satisfying. Maybe just delete this paragraph, or rephrase if that's not what you meant?
%Now, explain why this does a good job of balancing the terms. Next, we explain how this bound motivates the use of set-constrained backups to reduce accumulation of bootstrapping error. \TODO{explanation about $\delta1$ goes here} -- addressed -- removed this paragraph


% we need to determine how to formulate the appropriate constraint and how to implement so as to back up only values of policies in $\Pi$.
% %%SL.5.20: Rephrase. In order to develop a practical algorithm based on the set-constrained backup, we need to determine how to formulate the appropriate constraint and how to implement so as to back up only values of policies in $\Pi$.
% Intuitively, we would like $\Pi(s)$ for a particular state $s$ to contain only those policies that permit actions within the support of the dataset distribution. Instead of inferring $\Pi$, we use a notion of divergence between the uniform distribution over the support-set of the current policy and the current policy for optimization.  

% %%SL.5.20: Rephrase. Intuitively, we would like $\Pi(s)$ for a particular state $s$ to contain only those policies that permit actions withi
% In order confidence support set perform the $\max$ on the high-over actions from only these policies, we need to define a tractable objective. Instead of inferring the set of policies $\Pi$ we rather resort to specifying a notion of divergence between the set $\mathcal{A}_\varepsilon^\dataset$ and the current policy, $\operatorname{Divergence}(\mathcal{A}^{\mathcal{D}}_{\varepsilon}(s), \pi)$ thereby fitting the problem of inferring $\Pi$ in an optimization setup.
% %%SL.5.20: I don't really understand the above sentence. Try rewriting it to be clearer?
% Next, we move on to presenting our method, which we call \emph{bootstrap error accumulation reduction} (BEAR).


%%%%%%%%%%%%%%%%%%%%%%%%% OLD: Monday 7:30pm%%%%%%%%%%%%%%%
 % \subsection{Reducing Error Propagation via Set-Constrained Backups}
% Next, we explain how this bound motivates the use of set-constrained backups to reduce accumulation of errors. Suppose we are in an off-policy setting training from a static dataset. Let us denote the set of states and actions within the support of the data as $\mathcal{B}_1$ and the set of states outside of support as $\mathcal{B}_2 = (\mathcal{S} \times \mathcal{A}) - \mathcal{B}_1$. 

% When running Q-learning methods, we typically use supervised learning to minimize Bellman error - stanard supervised learning methods will allow us to control the expected loss on $\mathcal{B}_1$, but not on states over $\mathcal{B}_2$. Suppose that we incur a maximum error of $\delta_1$ on in-distribution states within $\mathcal{B}_1$ and a maximum error of $\delta_2$ on out-of-distribution states within $\mathcal{B}_2$.

% % Set-constrained backups will only be successful at error reduction if the overall error incurred by using set-constrained backups is lower in magnitude than the error incurred when using unrestricted backups. We therefore compare bootstrapping error and error incurred due to the set-constrained backups next.

% %Note that intermediate steps in AVI are all supervised learning problems. When using high-capacity function approximators for these supervised learning problems, we are better able to control the projection error $\delta_k$ induced due to bootstrapping from state-action pairs which lie in the high-confidence support set of the training distribution,  whereas the errors incurred due to backups from state-action pairs outside of this distribution are largely unbounded.
% %%SL.5.20: I noticed that nowhere in this analysis do you actually introduce any notation for the training distribution, and instead you use this set notation $\data$ all the time. As I wrote above, this is very problematic. If one of our central claims is that our analysis treats distributions better than Fujimoto, this isn't going to fly. Can you introduce notation for the training *distribution* (i.e., p(something)) and use that throughout?
% %This is because we can, in principle, overfit
% %%SL.5.20: overfit is the wrong word
% %to the state-action pairs lying in this set
% %%SL.5.20: stop using set
% %with powerful function approximators. However, when we bootstrap from actions that are not present 
% %in the high-confidence support of the train distribution, we back up values that can't be controlled.
% %%SL.5.20: use a different term than "can't be controlled" (we back up values that may have unbounded error)
% %To formalize this, let $\mathcal{A}^{\mathcal{D}}_{\varepsilon}(s) \defeq \{ a \in \mathcal{A} |~ \exists \pi \in \Pi,~ \pi(a|s) \geq \varepsilon \}$ 
% %\TODO{sometimes $\Pi$ is a set, sometimes it's a policy}
% %denote the set of action samples
% %%SL.5.20: the set of actions for state $s$? (maybe also remind the reader why we only worry about actions and not states)
% %that will most likely be sampled from the policies in the set $\Pi$. Let the error incurred in the value estimate at states $s$ reached when backing up from $s'$ using action $a \in \mathcal{A}^{\mathcal{D}}_{\varepsilon}(s')$ be $\delta^1(s)$ and let $\delta^2(s)$ be the error when backing up from $s'$ by using $a \notin \mathcal{A}^{\mathcal{D}}_{\varepsilon}(s')$, we can reasonably expect $\delta^1 < \delta^2$. 
% %%SL.5.20: be consistent with ', i.e., use a' for action in s' and a for action in s

% Using set-constrained AVI will allow us to recover an error bound of (using $\norm{\cdot}_{\infty,\mathcal{B}}$ to denote the maximum over a set $\mathcal{B}$):
%  $\lim_{k \to \infty} \norm{V_k - V^*}_{\infty, \mathcal{B}_1} \le \frac{2\gamma }{(1-\gamma)^2}\left(\max_{s \in , k} \delta^1_k(s) + \alpha(\Pi,s)\right) $
% In contrast, naively running unconstrained AVI will guarantee a bound of:
% $\lim_{k \to \infty} \norminf{V_k - V^*} \le \frac{2\gamma }{(1-\gamma)^2}\left(\max_{s, k} \delta^2_k(s)\right)$. 
% %\TODO{again, does this hold w/ fixed dataset?}
% %%SL.5.20: Can you state this as a theorem with a proof in the appendix? Generally tidying up the above paragraph would be a good idea, this is a very important paragraph and it's currently rather messy.

% Therefore, set-constrained algorithms enable us to accumulate error at the more favorable $\delta^1$ rate
% %$\mathcal{S} \times \mathcal{A}^{\mathcal{D}}_{\varepsilon}$,
% %$\mathcal{S} \times \mathcal{A} -\mathcal{S} \times \mathcal{A}^{\mathcal{D}}_{\varepsilon}$
% , at the cost of introducing additional suboptimality bias. Thus, it is beneficial to use set-constrained backups when $\delta^1 + \alpha < \delta^2$. This is a complex trade-off, which depends on the training data, the underlying MDP, and the function approximator. However, in general $\delta^2$ can be arbitrarily high and exceptionally difficult to control, especially with the use of function approximators such as neural networks, as it uses function-approximation outputs at \emph{state-action pairs at which we have little to no data}. 
% In addition, when our training data is close to optimal, we can expect a small suboptimality bias. Thus, we expect set-constrained algorithms to provide significantly better error bounds many practical scenarios.
% %%SL.5.20: Generally a good idea for several of the other coauthors to take a pass over the previous two paragraphs and tighten them up, deleting anything that is unnecessary. These two paragraphs are crucial for the paper to make sense, and esp the last para is currently way too long-winded.


% % %%% SEEMS TOO SUDDENLY COMING
% % In practice, we are better able to control the projection error $\delta$ on states within our training data than those outside. To formalize this tradeoff, assume we partition the state-action space into two sets, $\mathcal{B}_1$, the in-data set, and $\mathcal{B}_2 = (\mathcal{S} \times \mathcal{A}) - \mathcal{B}_1$, the out-of-data set. We assume that we can control projection error as $\delta^1$ on states within $\mathcal{B}_1$ and $\delta^2$ on $\mathcal{B}_2$, with $\delta^1 < \delta^2$.



% %Justin, it is better if you use this version of algorithm for error propagation 
% %$$
% %\begin{aligned} v_{k}=\left(T_{\pi_{k}}\right)^{m} v_{k-1} & \text { (evaluation step) } \\ \pi_{k+1}=\mathcal{G}\left[\left(T_{\pi_{k}}\right)^{m} v_{k-1}\right] &(\text { greedy step }) \end{aligned}
% %$$

% % Let $\mathcal{A}^{\mathcal{D}}_{\varepsilon}(s) \defeq \{ a \in \mathcal{A} | \pi_{data}(a|s) \geq \varepsilon \}$, and $\mathcal{A}^{\pi_{\theta}}_{\varepsilon}(s) \defeq \{ a \in \mathcal{A} | \pi(a|s) \geq \varepsilon \}$ denote the $\varepsilon$-high confidence support sets of the behaviour policy $\pi_{data}$ and the current actor $\pi$ at state $s$. Actions sampled from $\pi_{data}$ and $\pi$ are most likely to come from these sets respectively, and hence actions from these sets will be used to evaluate expectations or greedy maximum for backing up while performing ADP. If $\operatorname{Divergence}(\mathcal{A}^{\mathcal{D}}_{\varepsilon}(s), \mathcal{A}^{\pi_theta}_{\varepsilon}(s))$ is high, then we end up backing up from actions which haven't been updated, for which our Q-function could give arbitrarily overestimated/underestimated values. Also, note that as there is no implicit normalization mechanism on Q-functions as opposed to policies which are probability distributions and must integrate to 1, there is no check on the values of the Q-function approximator. 

% Now, we pose and answer one key question before presenting our approach to the problem. Given a dataset $\dataset$, how do we constrain our backups to only use policies from $\Pi$?
% %%SL.5.20: Rephrase. In order to develop a practical algorithm based on the set-constrained backup, we need to determine how to formulate the appropriate constraint and how to implement so as to back up only values of policies in $\Pi$.
% The possible candidates for $\Pi$ at a particular state $s$ is the set of policies for which the observed action $a$ lies in the high-confidence support.
% %%SL.5.20: Rephrase. Intuitively, we would like $\Pi(s)$ for a particular state $s$ to contain only those policies that permit actions withi
% In order tconfidence support set $equation$.o perform the $\max$ n the high-over actions from only these policies, we need to define a tractable objective. Instead of inferring the set of policies $\Pi$ we rather resort to specifying a notion of divergence between the set $\mathcal{A}_\varepsilon^\dataset$ and the current policy, $\operatorname{Divergence}(\mathcal{A}^{\mathcal{D}}_{\varepsilon}(s), \pi)$ thereby fitting the problem of inferring $\Pi$ in an optimization setup.
% %%SL.5.20: I don't really understand the above sentence. Try rewriting it to be clearer?
% Next, we move on to presenting our method, which we call \emph{bootstrap error accumulation reduction} (BEAR). 


% Section~\ref{sec:set_constrained_backup} presents a quantitative argument for the restricting the action distribution as an approach to solving the out-of-distribution actions problem in ADP by invoking tools from error propagation. In this section, extend that analysis to motivate the design for a practical algorithm for bootstrapping error reduction. We start by noting there is no direct way that the \TODO{(for Justin): Can we refer to the error-prop stuff under this name -- "abstract error model"} presented above can be instantiated in practice, as it involves the quantities $\delta^1$ and $\delta^2$ which are intractable and can only be measured in retrospect. Moreover, the choice of function approximator strongly affects these errors and theoretical properties of neural net function approximators are not fully understood. However, we can use the error propagation to motivate two design choices for the practical algorithm -- namely, (1) constraining to the support of the dataset action-distribution, (2) correcting the policy improvement step in addition to the policy evaluation step.
% %AK. (Question to Sergey:) Do you think (2) is relevent enough to be mentioned?
 
% Firstly, our definition of BEAR-backup restricts the max-backup to a subset of policies $\Pi$. In practical situations, the set of policies $\Pi$ is observed in the form of an action sample $a$ for each state $s$. Possible candidates for $\pi \in \Pi$ thus include policies for which $a$ belongs to the high-support region. \TODO{complete thia argument} 
% Alongside this, both steps of approximate policy iteration -- evaluation and improvement -- are supervised learning problems in themselves, being solved in practice by using powerful function-approximators such that in principle it is possible to model outputs for high-support datapoints of the training distribution to a high-enough degree of precision which prevents error propagation and accumulation from the set of all the high-confidence support actions. Hence, the out-of-distribution actions problem, in practice, manifests as an out-of-high-confidence support problem - i.e. usage of actions that are less likely than a particular chosen threshold, say, $\varepsilon$ under the training distribution for backups can accumulate a lot of error, but actions show high-enough support are good irrespective of the exact proportions of their appearance in the $\dataset$. 
% %AK.05.15: Question to Sergey: What do you think about the above paragraph? Does it seem too vague, and what's a better way of writing it?
% To formalize this notion, let $\mathcal{A}^{\mathcal{D}}_{\varepsilon}(s) \defeq \{ a \in \mathcal{A} | \Pi(a|s) \geq \varepsilon \}$ denote the $\varepsilon$-high confidence support set of the behaviour policy $\Pi$. Actions sampled from $\Pi$ are most likely to belong to these sets respectively, and hence actions from these sets will be used to evaluate expectations or greedy maximum for backing up while performing ADP. If $\operatorname{Divergence}(\mathcal{A}^{\mathcal{D}}_{\varepsilon}(s), \pi_\theta)$ is high, then we end up backing up from actions which haven't been updated, for which our Q-function could output arbitrarily overestimated/underestimated values.

%%%%%%%%%%%%%%%%%%%%%%%%%%%%%%%% COMMENT HERE
%AK.5.15: Note to Sergey: currently this section is written assuming theory corresponding to approximate policy iteration and not approximate value iteration as is currently present in Section 4.
% Secondly, the two components of $\delta^2$, $\epsilon_k$ (error arising due to approximate policy evaluation), and $\epsilon'_k$ (error arising due to approximate greedy maximization) can be analysed in worst case analysis. If we assume that the dataset $\mathcal{D}$ consists of $N$ samples, and the VC-dimension~\cite{} of the policy class $\Pi$ is denoted bby $h$, then the worst-case error accumulation (apart from intrinsic bellman error) in $\epsilon_k$ is $\mathcal{O}\big(V_{max} \sqrt{\frac{\log (N/\delta) }{N} \big)}$, whereas worst case $\epsilon'_k$ is $\mathcal{O} \big( V_{max} \sqrt{\log (N/\delta)} + V_{max} \sqrt{\frac{h \log (N/h) }{N}} \big)$. For more details, we refer the reader to Lemmas 11 and 12 in \cite{bruno2015approximate}. These bounds suggest that the worst case error incurred in the greedy maximization step is higher than the evaluation step for sufficiently rich function class of policies. Hence, in the worst case, errors in policy improvement can compound fast.
%%%%%%%%%%%%%%%%%%%%%%%%%%%%%%%%%%%%%%%%%%%%%%%%%

% From a practical algorithmic standpoint, the discussion above suggests
% that the actor update in the policy-improvement step be made more constrained, especially to the distribution $\Pi$ with less error compounding then. This also naturally leads to backups with restricted action distributions. Secondly, in practice, the algorithm should always stay in the high confidence support of the set of policies that generated the data. We use these insights to develop an algorithm which we present next.

% Let us revisit the error bounds from the abstract error model for the case of Classification Based Modified Policy Iteration (CBMPI)(\cite{bruno2015approximate}) -- which has a similar structure as modern Actor-Critic algorithms. Theorem 8 in \cite{bruno2015approximate} quantifies the error in the approximate policy iteration scheme defined by the abstract model. We revisit the theorem to gain insights about the problem. Let $\epsilon_k$ and $\epsilon'_k$ be instantaneous worst-case errors incurred in the evaluation step and greedy maximization step respectively. That is,
% $\epsilon_k \defeq v_k -\left(T_{\pi_{k}}\right)^{m} v_{k-1}$ and $\epsilon'_k \defeq \max_{\pi'} T_{\pi'} v_{k-1} - T_{\pi_k} v_{k-1}$. Then, the $\rho$-weighted $p$-th norm of the overall error, $v_* - v_{\pi_k}$ satisfies:
% \begin{multline*}
%     \|v* - v_{\pi_k}\|_{p, \rho} \leq 2 \gamma^{m} \sum_{i=1}^{k-2} \frac{\gamma^{i}}{1-\gamma}\left(\mathcal{C}_{q}^{i, i+1, m}\right)^{\frac{1}{p}}\left\|\epsilon_{k-i-1}\right\|_{p q^{\prime}, \mu}+\sum_{i=0}^{k-1} \frac{\gamma^{i}}{1-\gamma}\left(\mathcal{C}_{q}^{i, i+1,0}\right)^{\frac{1}{p}}\left\|\epsilon_{k-i}^{\prime}\right\|_{p q^{\prime}, \mu}+g(k)
% \end{multline*}

% where $\mathcal{C}_{q}^{l, k, m}$ is a concentrability coefficient and is a function of the MDP, with the property that $\mathcal{C}_{q}^{l, k, m} \geq \mathcal{C}_{q}^{l', k, m}$ for $l \leq l'$. (For more details we refer the reader to \cite{bruno2015approximate}). The concentrability coefficient is a function of the MDP, and hence cannot be controlled. It can therefore be argued that the major contribution in this error term comes from the second term, due to $\epsilon'$, as $m \geq 1$. This implies that imperfect policy improvement step is a major component of bootstrapping error. In modern deep RL settings, this means that optimization of the actor/policy towards regions with imperfect Q-values can be disastrous for the algorithm. Hence, one logical starting point for our approach is to constrain the set of actions we backup from. 

% \section{Error Propagation in Actor-Critic vs Q-Learning Algorithms}
% AVI:
% $$
% \left\|V^{*}-V^{\pi_{K}}\right\|_{p, \rho} \leq \frac{2 \gamma}{(1-\gamma)^{2}}\left[\inf _{r \in[0,1]} C_{V I, \rho, \nu}^{\frac{1}{2 p}}(K ; r) \mathcal{E}^{\frac{1}{2 p}}\left(\varepsilon_{0}, \ldots, \varepsilon_{K-1} ; r\right)+\frac{2}{1-\gamma} \gamma^{\frac{K}{p}} R_{\max }\right]
% $$

% API:
% $$
% \left\|Q^{*}-Q^{\pi_{K}}\right\|_{p, \rho} \leq \frac{2 \gamma}{(1-\gamma)^{2}}\left[\inf _{r \in[0,1]} C_{P(B R A E), \rho, \nu}^{\frac{1}{2 p}}(K ; r) \mathcal{E}^{\frac{1}{2 p}}\left(\varepsilon_{0}, \ldots, \varepsilon_{K-1} ; r\right)+\gamma^{\frac{K}{p}-1} R_{\max }\right]
% $$


\section{Bootstrapping Error Accumulation Reduction (BEAR)}
\label{sec:bear}

We now propose a practical actor-critic style algorithm that uses distribution-constrained backups
%%SL.5.22: Did you define set-constrained backups? I think probably the best thing would be to simply bring back the definition.
to reduce accumulation of bootstrapping error.
Our model has two main components. We use ensembles of Q-functions to provide a conservative estimate of the Q-function which is used in policy improvement, and a constraint which will be used for searching over the set of policies $\Pi$, which share the same support as the behaviour policy. Both of these components will appear as modifications of the policy improvement step in actor-critic style algorithms.

We use an ensemble of Q-functions $\hat{Q}_1, \cdots, \hat{Q}_K$ to compute a conservative estimate of the Q-values: $\frac{1}{K} \sum_{i=1}^K \hat{Q}_i (s, a) - \lambda \sqrt{\operatorname{var}_k \hat{Q}_k(s, a)}$, where $\lambda \in \mathbb{R}^+$ is a hyperparameter. %We use this value as a conservative estimate of the Q-function. This can be derived using Cantelli's inequality. 
Then, the policy is updated to maximize the conservative estimate of the Q-values: $$ \pi_{k+1}(s) := \max_{\pi \in \Pi} E_{a \sim \pi(\cdot|s)} [\hat{Q}_{k}(s, a)] - \lambda \sqrt{ \operatorname{var_k}E_{a \sim \pi(\cdot |s) }[\hat{Q}_k(s, a)]}.$$



% Let $\mathcal{F}_t$ be the sigma-algebra generated by the training procedure until iteration $t$, and let $\operatorname{var}_{t} \hat{Q}(s,a) := \mathbb{E}[(\hat{Q}_t(s, a) - \mathbb{E}[(\hat{Q}_t(s, a) | \mathcal{F}_t))^2|\mathcal{F}_t]$
%%SL.5.20: use mbox. And for clarity, it might be good to indicate what the expectation is over (and use [ instead of ( for E so that parens don't get cluttered). Also, what is up with this (s,a) hanging out at the end? do you mean to put (s,a) inside (after \hat{Q})?
% denote the variance of the Q-function $\hat{Q}_t$, at time $t$ during training. Then, for each state-action pair $(s, a)$, 
% ${Pr (\hat{Q}_t \geq \mathbb{E}(\hat{Q}_t|\mathcal{F}_t) + \sqrt{\frac{(1 - \delta) \operatorname{var}_{t} \hat{Q}_t }{\delta}})  \leq \delta}$
%%SL.5.20: can you state in words what this means for the purpose of this section? also, rhetoric-wise, amybe better state as a theorem (it's kind of obvious, but still) and then after say that this is easy to show via Cantelli's inequality or something?

%%SL.5.20: It's not clear what the concentration bound is actually used for.

 %In the above concentration bound, $\mathbb{E}(\hat{Q}_t|\mathcal{F}_t)$ refers to the true Q-value, which can be obtained given no stochasticity in the procedure.


%%SL.5.20: The logical thread here is broken. What are you doing with set divergence? State the issue first, then th e resolution, else it's hard for the reader to follow.
In practice, the behaviour policy $\beta$ is unknown, so we need an approximate way to constrain $\pi$ to $\Pi$. We define a differentiable constraint that approximately constrains $\pi$ to $\Pi$, and then approximately solve the constrained optimization problem via dual gradient descent.  We use the sampled version of maximum mean discrepancy (MMD)~\cite{gretton2012kernel}
%%SL.5.22: Alg names are not capitalized unless they contain proper nouns, put a space after the words and before open paren (I fixed it above, but this issue happens often, please take this comment into account) -- Thanks for pointing this out!
between the unknown behaviour policy $\beta$ and the actor $\pi$ because it can be estimated based solely on samples from the distributions. Given samples $x_1, \cdots, x_n \sim P$ and $y_1, \cdots, y_m \sim Q$, the sampled MMD between $P$ and $Q$ is given by:\\
$$\operatorname{MMD}^2(\{x_1, \cdots, x_n\}, \{y_1, \cdots, y_m\}) = \frac{1}{n^2} \sum_{i, i'} k(x_i, x_{i'}) - \frac{2}{nm} \sum_{i, j} k(x_i, y_j) + \frac{1}{m^2} \sum_{j, j'} k(y_j, y_{j'}).
$$
Here, $k(\cdot, \cdot)$ is any universal kernel. In our experiments, we find both Laplacian and Gaussian kernels work well.
%As the $\operatorname{MMD}$ distance does not depend on the density function of either distribution, minimizing it using samples is a reasonable proxy for enforcing that $Q$ lies inside the support of $P$. This is because, 
Empirically we find that, in the low sample regime, the sampled MMD between $P$ and $Q$ is similar to the MMD between a uniform distribution over $P$'s support and $Q$ (See Appendix~\ref{} for numerical simulations justifying this approach). %We provide some empirical evidence to justify this choice in the appendix using numerical simulations on gaussian distributions.

% and hence, we parameterize the set $\mathcal{A}^{\mathcal{D}}_{\varepsilon}(s)$ as a distribution $\pi_{set}(a|s)$ such that $\mathcal{A}(s) := \mathcal{A}^{\pi_{set}}_{\varepsilon}(s) := \{a \in \mathcal{A} | \pi_{set}(a|s) \geq \varepsilon \}$, in other words, $\mathcal{A}(s)$ is the high-confidence support set of the distribution $\pi_{set}$, and we train for a parametric $\pi_{set}$.
%%SL.5.20: I don't actually understand at this point what you are doing. Are you optimizing a neural net that denotes \pi_set? or something else?

% \paragraph{Deriving the update:} Let $\hat{Q}_k$ be the Q-function at the k-th step of the algorithm. Actor-critic Q-learning algorithms maintain a parameterized policy, $\pi_k$ that is updated towards the maximizing the Q-function.
% %-- $\pi_{k+1}(s) := \max_{\pi \in \Delta_{|S|}} E_{a \sim \pi(\cdot|s)} [\hat{Q}_{k}(s, a)]$. 
% In order to reduce the number of moving parts, we let the actor in this case serve both its regular function of maximizing the Q-function while also constraining the action distribution close to $\mathcal{A}^\dataset_\varepsilon$, which is the the task of $\pi_{set}$. We use the bound derived on Q-values to update the policy in the direction of maximizing a conservative estimate of the true Q-value -- $$ \pi_{k+1}(s) := \max_{\pi \in \Delta_{|S|}} E_{a \sim \pi(\cdot|s)} [\hat{Q}_{k}(s, a)] - \lambda \sqrt{ \operatorname{var_k}E_{a \sim \pi(\cdot |s) }[\hat{Q}_k(s, a)]}$$
% %TODO{may want to mention that this amounts to subtracting a constant times the std, which sounds reasonable}
% We still need to account for the problem of specifying support divergence. In order to enforce this constraint, we use a measure of support matching between the training distribution $\Pi$ and the policy $\pi(\cdot|s)$, which we choose to be a sampled version of the Maximum Mean Discrepancy(MMD) Distance between $\Pi$ and the actor $\pi$. Sampled MMD distance between two probability distributions $P$ and $Q$ is given by, $\operatorname{MMD}(P, Q)$, where $x_1, \cdots, x_n \sim P$ and $y_1, \cdots, y_m \sim Q$ is given by:\\
% $$\operatorname{MMD}^2(\{x_1, \cdots, x_n\}, \{y_1, \cdots, y_m\}) = \frac{1}{n^2} \sum_{i, i'} k(x_i, x_{i'}) - \frac{2}{nm} \sum_{i, j} k(x_i, y_j) + \frac{1}{m^2} \sum_{j, j'} k(y_j, y_{j'})
% $$
% When the number of samples $n$ is an intermediate number (4-10), the above sampled objective can also be approximately considered as a distance between a uniform distribution over the high confidence support set of the distribution $P$ and the distribution $Q$ -- therefore, if trained perfectly, $Q$ should have the same support as $P$. That is, $\operatorname{MMD}(P, Q)$ is a reasonable proxy for $\operatorname{MMD}(\mathcal{U}(\mathcal{A}_{\varepsilon}(P)), Q)$. 
% %\TODO{what does it mean MMD between a set and distribution}
% The expression for $\operatorname{MMD}$ does not use the density function of either distribution, thereby making it suited as an approximate way of support matching.

Putting it all together, the overall optimization problem in the policy improvement step becomes
\begin{multline}
    \label{eqn:policy_update}
   \pi_{k+1}(s) := \max_{\pi \in \Delta_{|S|}} E_{a \sim \pi(\cdot|s)} [\hat{Q}_{k}(s, a)] - \lambda \sqrt{ \operatorname{var_k}E_{a \sim \pi(\cdot |s) }[\hat{Q}_k(s, a)]}\\
   \text{~~s.t.~~} \mathbb{E}_{s \sim \mathcal{D}} [\operatorname{MMD}(\mathcal{D}(s), \pi(\cdot|s))] \leq \varepsilon
\end{multline}
where $\varepsilon$ is an approximately chosen threshold. We choose a threshold of $\varepsilon=0.05$ in all our experiments. 
% We use an ensemble of $M$ Q-functions, $\{Q_{\theta_i} \}_{i=1}^M$ trained on the same data starting from different initializations for modeling $\operatorname{var}(\hat{Q}|\mathcal{F}_{1:t})$, using sample variance of the ensemble. 
The Algorithm is summarized in Algorithm~\ref{algo:bear_ql}.
% $\operatorname{var}(\hat{Q}_k(s, a)) \approx \frac{1}{M} \sum_{i=1}^{M} (\hat{Q}_{\theta_i, k}(s, a) - \bar{Q}_{\theta, k}(s, a))^2$, where $\bar{Q}_{\theta, k}(s, a) = \frac{1}{M} \sum_{i=1}^{M} \hat{Q}_{\theta_i, k}(s, a)$ is the sample mean of the ensemble. 

%AK.05.15: Note to Sergey: this is the actor-critic version, optional depends on results.
% Another variant of the above approach can be where this single policy improvement step can be decomposed into two decoupled steps -- (1) Learning a policy $\pi_{set}$, whose high-confidence set defines the support set $\mathcal{A}_{\varepsilon}(s)$ at a state $s$, by minimizing the sampling error in $\hat{Q}_k$ and accounting for the deviation from the dataset, and then, (2) Learning to maximize the expected Q-function $\hat{Q}_k$ on this set $\mathcal{A}_{\varepsilon}(s)$, in practice obtained by sampling from $\pi_{set}$. In practice, we found using Equation~\ref{eqn:policy_update} working better than the latter approach and hence, we stick to this formulation for our experiments. The overall algorithm is summarized in Algorithm~\ref{alg:q_learning}, and the actor-critic version is described in Algorithm~\ref{alg:actor_critic}.   

\begin{algorithm}[H]

\small
\caption{BEAR Q-Learning}
\label{alg:q_learning}
\begin{algorithmic}[1]
    \INPUT: Dataset $\mathcal{D}$, target network update rate $\tau$, mini-batch size $N$, sampled actions for MMD $n$, minimum $\lambda$
    \STATE Initialize Q-ensemble $\{Q_{\theta_i} \}_{i=1}^{K}$, actor $\pi_{\phi}$, Lagrange multiplier $\alpha$, target networks $\{ Q_{\theta'_i} \}_{i=1}^K$, and a target actor $\pi_{\phi'}$, with $\phi' \leftarrow \phi, \theta'_i \leftarrow \theta_i$
    \FOR{$t$ in \{1, \dots, N\}}
        \STATE Sample mini-batch of transitions $(s, a, r, s') \sim \mathcal{D}$\\
        \textbf{Q-update:}
            \STATE Sample $p$ action samples, $\{a_i \sim \pi_{\phi'}(\cdot|s')\}_{i=1}^p$
            \STATE Define $y = \max_{a_i} [ \lambda \min_{j=1,..,K} Q_{\theta'_j}(s', a_i) + (1 - \lambda) \max_{j=1,..,K} Q_{\theta'_j}(s', a_i)]$
            \STATE $\forall i, \theta_i \leftarrow \arg \min_{\theta_i} (Q_{\theta_i}(s, a) - (r + \gamma y(s, a)))^2$\\
        \textbf{Policy-update:}
        \STATE Sample actions $\{ \hat{a}_i \sim \pi_{\phi}(\cdot | s) \}_{i=1}^{m}$ and $\{ a_j \sim \mathcal{D}(s)\}_{j=1}^{n}$, $n$ preferably an intermediate integer(1-10)
        \STATE Update $\phi$, $\alpha$ by minimizing Equation~\ref{eqn:policy_update} by using dual gradient descent with Lagrange multiplier $\alpha$
        \STATE \textbf{Update Target Networks: } $\theta'_i \leftarrow \tau \theta_i + (1 - \tau)\theta'_i$; $\phi' \leftarrow \tau \phi + (1 -\tau) \phi'$ 
    \ENDFOR
\end{algorithmic}
\label{algo:bear_ql}
\end{algorithm}

To summarize Algorithm~\ref{algo:bear_ql}: The actor is updated towards maximizing the Q-function while still being forced to remain in the valid search space defined by $\Pi$. The Q-function uses actions sampled from the actor to then perform set-constrained Q-learning, over a reduced set of policies. The maximization step in the actor-update empirically helps, but can be coupled with maximization in Step 5. Similar to \cite{fujimoto2018off} we use a soft-minimum to compute target values for updating Q-functions. Implementation and other details are present in Appendix ?.
%%SL.5.22: Remember to fill this in.

% \begin{algorithm}[H]
% \small
% \caption{BEAR Actor-Critic}
% \label{alg:actor_critic}
% \begin{algorithmic}[1]
%     \INPUT: Dataset $\mathcal{D}$, target network update rate $\tau$, mini-batch size $N$, sampled actions for MMD $n$, minimum $\lambda$, policy gradient clipping constants $\beta_1, \beta_2; \beta_1 \leq \beta_2$, MMD threshold constant $\varepsilon$
%     \STATE Initialize Q-ensemble $\{Q_{\theta_i} \}_{i=1}^{M}$, actor $\pi_{\phi}$, set-determining policy $\pi_{set}$, Lagrange multiplier $\alpha$, target networks $\{ Q_{\theta'_i} \}_{i=1}^M$, and a target actor $\pi_{\phi'}$, with $\phi' \leftarrow \phi, \theta'_i \leftarrow \theta_i$
%     \FOR{$t$ in \{1, \dots, N\}}
%         \STATE Sample mini-batch of transitions $(s, a, r, s') \sim \mathcal{D}$\\
%         \textbf{Q-update:}
%             \STATE Sample $m$ action samples, $\{a_i \sim \pi_{\phi'}(\cdot|s')\}_{i=1}^n$
%             \STATE Define $y = \frac{1}{m} \sum_{a_i} [ \lambda \min_{j=1,..,M} Q_{\theta'_j}(s', a_i) + (1 - \lambda) \max_{j=1,..,M} Q_{\theta'_j}(s', a_i)]$
%             \STATE $\forall i, \theta_i \leftarrow \arg \min_{\theta_i} (Q_{\theta_i}(s, a) - (r + \gamma y))^2$\\
%         \textbf{Set-update and Actor-update:}
%         \STATE Sample actions $A_1(s) \equiv \{ \hat{a}_i \sim \pi_{set}(\cdot | s) \}_{i=1}^{m}$ and $A_2(s) \equiv \{ a_j \sim \mathcal{D}(s)\}_{j=1}^{n}$, $n << m$
%         \STATE Update $\pi_{set}, \alpha$: $$ \pi_{set}, \alpha \leftarrow \arg \min_{\pi_{set}} \max_{\alpha \geq 0} \sqrt{\frac{(1 - \delta) \operatorname{var_k}E_{a \sim \pi_{set}(\cdot |s) }[\hat{Q}_k(s, a)]}{\delta}} + \alpha \mathbb{E}_{s \sim \mathcal{D}} ([\operatorname{MMD}(A_1, A_2)] -  \varepsilon) $$
%         \STATE Update $\phi$ using Importance Sampled Policy Gradient: 
%         $$ \pi_{\phi} \leftarrow  \max_{\pi_{\phi}} \mathbb{E}_{s \sim \mathcal{D}} \mathbb{E}_{a \sim \pi_{set}(\cdot|s)} \Big( \Big[ \frac{\pi_\phi(a|s)}{\pi_{set}(a|s)} \Big]_{\beta_1}^{\beta_2} Q(s, a) \Big)$$
%         \STATE \textbf{Update Target Networks: } $\theta'_i \leftarrow \tau \theta_i + (1 - \tau)\theta'_i$; $\phi' \leftarrow \tau \phi + (1 -\tau) \phi'$ 
%     \ENDFOR
% \end{algorithmic}
% \end{algorithm}


% Let $\bar{Q}(\cdot, \cdot)$ be the delayed target network, and $Q(\cdot, \cdot)$ be the current Q-function. Define $d_i$ be the the TD error for the $i^{th}$ datapoint.
% $$
% d_{i}(Q ; \bar{Q}, \pi)=R_{t}+\gamma \bar{Q}\left(s'_{i}, \pi_{set} \left(s'_i\right)\right)-Q\left(s_{i}, a_{i}\right)

% $$
% Further we define the empirical loss function by
% $$
% \hat{L}_{N}(Q ; \bar{Q}, \pi)=\frac{1}{N} \sum_{t=1}^{N} \frac{d_{t}^{2}(Q ; \bar{Q}, \pi_{set})}{\lambda(\mathcal{A})}
% $$
% where normalization $\lambda{\mathcal{A}}$ is introduced for mathematical convenience. Then, each policy evaluation step can be written as:  

% If we solely backup from actions present in our dataset, there is no way the algorithm can perform better than the policy that collected the data. The capacity of Q-learning and other ADP algorithms to ``stitch'' together performant sub-trajectories is lost. Hence, our method does allow the agent to backup from actions that occur outside the dataset, while still being constrained to not go farther away from the support of $\mathcal{D}$. In principle, a measure of distance from a given dataset can only be obtained using Bayesian Approaches (?). In practice, we use the variance of the ensemble as a measure to approximately quantify closeness to the support set. Our overall approach is described in the next paragraph.




% Our problem setting does not allow any interaction with the environment, and only lets us use the dataset $\mathcal{D}$. Since we see a limited subset of state-action pairs from the environment, the expected estimate of the Q-function conditioned on all training history in our case, $\mathbb{E}(\hat{Q}|\mathcal{F}_t)$, is biased. \TODO{aviral: finish this argument} 

% We train an ensemble of $N$ parametric Q-functions, $Q_{\theta_1}, \cdots, Q_{\theta_N}$ by using bootstrap masks on the data points of the dataset $\mathcal{D}$. This is done to simulate epistemic variance. To make sure that the actions chosen for backing up Q-functions are valid, we learn a set selection policy, $\pi_{set}$ -- a policy that can provide high densities to actions that don't propagate errors.   

\newcommand{\CC}{\cellcolor{Gray}}
\definecolor{Gray}{gray}{0.9}

\subsection{Experimental Evaluation of Conservative Data Sharing}
\label{sec:cds_exp}

We conduct experiments to answer six main questions: \textbf{(1)} can \cdsmethodname\ prevent performance degradation when sharing data as observed in Section~\ref{sec:cds_analysis}?, \textbf{(2)} how does \cdsmethodname\ compare to vanilla multi-task offline RL methods and prior data sharing methods?
\textbf{(3)} can \cdsmethodname\ handle sparse reward settings, where data sharing is particularly important due to scarce supervision signal? \textbf{(4)} can \cdsmethodname\ handle goal-conditioned offline RL settings where the offline dataset is undirected and highly suboptimal? \textbf{(5)} Can \cdsmethodname\ scale to complex visual observations? \arxiv{\textbf{(6)} Can \cdsmethodname\ be combined with any offline RL algorithms? Besides these questions, we visualize CDS weights for better interpretation of the data sharing scheme learned by CDS in Figure~\ref{fig:antmaze_vis} in Appendix~\ref{app:cds_vis}}.

\begin{wrapfigure}{r}{0.45\textwidth}
    \vspace{-0.65cm}
    \centering
    \includegraphics[width=0.95\linewidth]{chapters/cds/env.png}
    \vspace{-0.32cm}
    \caption{\footnotesize  Environments (from left to right): walker2d {run forward}, walker2d {run backward}, walker2d {jump},  Meta-World {door open/close} and {drawer open/close} and vision-based pick-place tasks in \citep{kalashnikov2021mt}.}
    \label{fig:env}
    \vspace{-0.4cm}
\end{wrapfigure}
\textbf{Comparisons.} To answer these questions, we consider the following prior methods. On tasks with low dimensional state spaces, we compare with the online multi-task relabeling approach \textbf{HIPI}~\citep{eysenbach2020rewriting}, which uses inverse RL to infer for which tasks the datapoints are optimal and in practice routes a transition to task with the highest Q-value. We adapt HIPI to the offline setting by applying its data routing strategy to a conservative offline RL algorithm.
We also compare to na\"ively sharing data across all tasks (denoted as \textbf{Sharing All}) and vanilla multi-task offline RL method without any data sharing (denoted as \textbf{No Sharing}). On image-based domains, we compare \cdsmethodname\ to the data sharing strategy based on human-defined skills~\citep{kalashnikov2021mt} (denoted as \textbf{Skill}), which manually groups tasks into different skills (e.g. skill ``pick'' and skill ``place'') and only routes an episode to target tasks that belongs to the same skill.
In these domains, we also compare to \textbf{HIPI}, \textbf{Sharing All} and \textbf{No Sharing}. \arxiv{Beyond these multi-task RL approaches with data sharing, to assess the importance of data sharing in offline RL, we perform an additional comparison to other alternatives to data sharing in multi-task offline RL settings. One traditionally considered approach is to use data from other tasks for some form of ``pre-training'' before learning to solve the actual task. We instantiate this idea by considering a method from \citet{yang2021representation} that conducts contrastive representation learning on the multi-task datasets to extract shared representation between tasks and then runs multi-task RL on the learned representations. We discuss this comparison in detail in Table~\ref{tbl:pretrain_comparison} in Appendix~\ref{app:pretrain_comparison}.} 
To answer question (6), we use CQL~\citep{kumar2020conservative} (a Q-function regularization method) and BRAC~\citep{wu2019behavior} (a policy-constraint method) as the base offline RL algorithms for all methods. \arxiv{We discuss evaluations of CDS with CQL in the main text and include the results of CDS with BRAC in Table~\ref{tbl:brac_comparison} in Appendix~\ref{app:brac_results}.} For more details on setup and hyperparameters, see Appendix~\ref{app:cds_details}.


\textbf{Multi-task environments.} We consider a number of multi-task reinforcement learning problems on environments visualized in Figure~\ref{fig:env}. 
% To answer questions (1) and (2), we consider three locomotion environments from OpenAI Gym~\citep{brockman2016openai} with dense rewards: halfcheetah, walker2d, and ant. Each environment has three tasks, \texttt{run forward}, \texttt{run backward} and \texttt{jump}, as used in prior offline RL work~\citep{yu2020mopo}.
{To answer questions (1) and (2), we consider the walker2d locomotion environment from OpenAI Gym~\citep{brockman2016openai} with dense rewards. We use three tasks, \texttt{run forward}, \texttt{run backward} and \texttt{jump}, as proposed in prior offline RL work~\citep{yu2020mopo}.}
To answer question (3), we also evaluate on robotic manipulation domains using environments from the Meta-World benchmark~\citep{yu2020metaworld}. We consider four tasks: \texttt{door open}, \texttt{door close}, \texttt{drawer open} and \texttt{drawer close}. Meaningful data sharing requires a consistent state representation across tasks, so we put both the door and the drawer on the same table, as shown in Figure~\ref{fig:env}. Each task has a sparse reward of 1 when the success condition is met and 0 otherwise. To answer question (4), we consider maze navigation tasks where the temporal ``stitching'' ability of an offline RL algorithm is crucial to obtain good performance. We create goal reaching tasks using the ant robot in the medium and hard mazes from D4RL~\citep{fu2020d4rl}. The set of goals is a fixed discrete set of size 7 and 3 for large and medium mazes, respectively. Following \citet{fu2020d4rl}, a reward of +1 is given and the episode terminates if the state is within a threshold radius of the goal. Finally, to explore how \cdsmethodname\ scales to image-based manipulation tasks (question (5)), we utilize a simulation environment similar to the real-world setup presented in~\citep{kalashnikov2021mt}. This environment, which was utilized by \citet{kalashnikov2021mt} as a representative and realistic simulation of a real-world robotic manipulation problem, consists of 10 image-based manipulation tasks that involve different combinations of picking specific objects (banana, bottle, sausage, milk box, food box, can and carrot) and placing them in one of the three fixtures (bowl, plate and divider plate) (see example task images in Fig.~\ref{fig:env}).
More environment details are in the appendix. We report the average return for locomotion tasks and success rate for AntMaze and both manipluation environments, averaged over 6 and 3 random seeds for environments with low-dimensional inputs and image inputs respectively.

\begin{table}[h]
\vspace{0.1cm}
  \centering
  \scriptsize
  \def\arraystretch{0.9}
  \setlength{\tabcolsep}{0.42em}
  \vspace{-0.4cm}
\resizebox{0.95\linewidth}{!}{\begin{tabular}{cc|cccc}
  \toprule
 \multicolumn{1}{c}{\multirow{1.5}[2]{*}{Environment}} & \multicolumn{1}{c}{\multirow{1.5}[2]{*}{Dataset types / Tasks}}\vline &
 \multicolumn{3}{c}{$D_\text{KL}(\pi, \pi_\beta)$}\\
& \multicolumn{1}{c}{} \vline& \multicolumn{1}{c}{\textbf{No Sharing}}  & \multicolumn{1}{c}{\textbf{Sharing All}} & \multicolumn{1}{c}{\textbf{CDS (basic) (ours)}}  & \multicolumn{1}{c}{\textbf{CDS (ours)}}\\
\midrule
  &medium-replay / run forward & \textbf{1.49} & 7.76 & 14.31 & \textbf{1.49}\\
  walker2d& medium / run backward &  \textbf{1.91} & 12.2 & 8.26 & 6.09\\
  %\rowcolor{Gray}
  & \cellcolor{yellow} expert / jump & \cellcolor{yellow} 3.12 & \cellcolor{yellow} 27.5 & \cellcolor{yellow} 13.25  & \cellcolor{yellow} \textbf{2.91}\\
    \bottomrule
    \end{tabular}}
    \vspace{0.1cm}
         \caption{\footnotesize Measuring $D_\text{KL}(\pi, \pi_\beta)$ on the walker2d environment.  \textbf{Sharing All} degrades the performance on task \text{jump} with limited expert data as discussed in Table~\ref{tab:analysis}. \cdsmethodname\ manages to obtain a $\behavior$ after data sharing that is closer to the single-task optimal policy in terms of the KL divergence compared to \textbf{No Sharing} and \textbf{Sharing All} on task \texttt{jump} (highlighted in yellow). Since \cdsmethodname\ also achieves better performance, this analysis suggests that reducing distribution shift is important for effective offline data sharing.
     \label{tab:analysis_cds}
     \vspace{-0.3cm}
     }
\end{table}

\textbf{Multi-task datasets.}  Following the analysis in Section~\ref{sec:cds_analysis}, we intentionally construct datasets with a variety of heterogeneous behavior policies to test if \cdsmethodname\ can provide effective data sharing to improve performance while avoiding harmful data sharing that exacerbates distributional shift. For the locomotion domain, we use a large, diverse dataset (medium-replay) for \texttt{run forward}, a medium-sized dataset for \texttt{run backward}, and an expert dataset with limited data for \texttt{run jump}. For Meta-World, we consider medium-replay datasets with 152K transitions for task \texttt{door close} and \texttt{drawer open} and expert datasets with only 2K transitions for task \texttt{door open} and \texttt{drawer close}. For AntMaze, we modify the D4RL datasets for antmaze-*-play environments to construct two kinds of multi-task datasets: an ``undirected'' dataset, where data is equally divided between different tasks and the rewards are correspondingly relabeled, and a ``directed'' dataset, where a trajectory is associated with the goal closest to the final state of the trajectory. This means that the per-task data in the undirected setting may not be relevant to reaching the goal of interest. Thus, data-sharing is crucial for good performance: methods that do not effectively perform data sharing and train on largely task-irrelevant data are expected to perform worse. Finally, for image-based manipulation tasks, we collect datasets for all the tasks individually by running online RL~\cite{kalashnikov2018scalable} until the task reaches medium-level performance (40\% for picking tasks and 80\% placing tasks). At that point, we merge the entire replay buffers from different tasks creating a final dataset of 100K RL episodes with 25 transitions for each episode.

\begin{table*}[t!]
\centering
\vspace*{0.1cm}
\scriptsize
\resizebox{\textwidth}{!}{\begin{tabular}{l|l|r|r|r|r|r}
\toprule
\textbf{Environment} & \textbf{Tasks / Dataset type} & \textbf{\cdsmethodname\ (ours)} & \textbf{\cdsmethodname\ (basic)} & \textbf{HIPI}~\cite{eysenbach2020rewriting}& \textbf{Sharing All} & \textbf{No Sharing}\\ \midrule
% & run forward / medium-replay & 2587.7  & \textbf{2626.1} & 2605.0 & \textbf{2632.5}\\
% halfcheetah & run backward / medium & 2519.5  & \textbf{2634.4} & \textbf{2636.7} & \textbf{2630.7}\\
% & jump / expert & \textbf{4298.2} & 4113.4 & 712.3 & -1978.3\\
% & \CC \textbf{average} & \CC \textbf{3135.1} & \CC \textbf{3124.7} & \CC 1984.7 & \CC 1095.0\\
% \midrule
& run forward / medium-replay & \textbf{1057.9}$\pm$121.6 & 968.6$\pm$188.6 & 695.5$\pm$61.9 & 701.4$\pm$47.0 & 590.1$\pm$48.6\\
walker2d & run backward / medium & 564.8$\pm$47.7 & 594.5$\pm$22.7 & 626.0$\pm$48.0& \textbf{756.7}$\pm$76.7& 614.7$\pm$87.3\\
& jump / expert & 1418.2$\pm$138.4 & 1501.8$\pm$115.1  & \textbf{1603.7}$\pm$146.8 & 885.1$\pm$152.9 & 1575.2$\pm$70.9\\
& \CC \textbf{average} & \CC 1013.6$\pm$71.5 &\CC \textbf{1021.6}$\pm$76.9 & \CC 975.1$\pm$45.1 & \CC 781.0$\pm$100.8 & \CC 926.6$\pm$37.7\\\midrule
% & run forward / medium-replay & 2350.1 & \textbf{2658.9} & 1175.0 & 2126.7\\
% ant & run backward / medium & 1435.7 & 1208.2 & 1488.7 & \textbf{2021.7}\\
% & jump / expert & \textbf{2781.3} & 2670.4 & 133.8 & 495.8\\
% & \CC \textbf{average} & \CC \textbf{2189.0} & \CC \textbf{2179.2} & \CC 932.5 & \CC 1548.1\\
% \midrule
& door open / expert & \textbf{58.4\%}$\pm$9.3\% & 30.1\%$\pm$16.6\% & 26.5\%$\pm$20.5\% & 34.3\%$\pm$17.9\% & 14.5\%$\pm$12.7\\
& door close / medium-replay & \textbf{65.3\%}$\pm$27.7\% & 41.5\%$\pm$28.2\% & 1.3\%$\pm$5.3\% & 48.3\%$\pm$27.3\% & 4.0\%$\pm$6.1\% \\
Meta-World~\citep{yu2020metaworld}& drawer open / medium-replay & \textbf{57.9\%}$\pm$16.2\% & 39.4\%$\pm$16.9\% & 41.2\%$\pm$24.9\% & 55.1\%$\pm$9.4\% & 16.0\%$\pm$17.5\%\\
& drawer close / expert & 98.8\%$\pm$0.7\% & 86.3\%$\pm$0.9\% & 62.2\%$\pm$33.4\% & \textbf{100.0\%}$\pm$0\% & 99.0\%$\pm$0.7\%\\
& \CC \textbf{average} & \CC \textbf{70.1\%}$\pm$8.1\% & \CC 49.3\%$\pm$16.0\% & \CC 32.8\%$\pm$18.7\% & \CC 59.4\%$\pm$5.7\% & \CC 33.4\%$\pm$8.3\%\\
\midrule
& large maze (7 tasks) / undirected & \textbf{22.8}\% $\pm$ 4.5\% & 10.0\% $\pm$ 5.9\% & 1.3\% $\pm$ 2.3\%  & 16.7\% $\pm$ 7.0\% & 13.3\% $\pm$ 8.6\% \\
AntMaze~\citep{fu2020d4rl}  & large maze (7 tasks) / directed &  \textbf{24.6\%} $\pm$ 4.7\% & 0.0\% $\pm$ 0.0\% & 11.8\% $\pm$ 5.4\% & 20.6\% $\pm$ 4.4\% & 19.2\% $\pm$ 8.0\% \\
& medium maze (3 tasks) / undirected &  \textbf{36.7\%} $\pm$ 6.2\% & 0.0\% $\pm$ 0.0\% & 8.6\% $\pm$ 3.2\% & 22.9\% $\pm$ 3.6\% & 21.6\% $\pm$ 7.1\% \\
& medium maze (3 tasks) / directed &  \textbf{18.5}\% $\pm$ 6.0\% & 0.0\% $\pm$ 0.0\% & 8.3\% $\pm$ 9.1\% & 12.4\% $\pm$ 5.4\% & \textbf{17.0\%} $\pm$ 3.2\% \\
\bottomrule
\end{tabular}}
\vspace{-0.2cm}
\caption{\footnotesize Results for multi-task locomotion (walker2d), robotic manipulation (Meta-World) and navigation environments (AntMaze) with low-dimensional state inputs.
% On the locomotion environment walker2d, we include three tasks with different styles of datasets, \texttt{run forward} + a medium replay dataset, \texttt{run backward} + a medium dataset and \texttt{jump} + an expert dataset with limited data. On the multi-task robotic manipulation domain, we consider four tasks from Meta-World~\citep{yu2020metaworld}, door open, door close, drawer open and drawer close with medium-replay, expert, medium-replay and expert datasets respectively.
% % Similar to locomotion tasks, we also used limited amount of expert trajectories for the expert dataset.
% On the antmaze navigation task, we consider two maze layouts (medium/large) from D4RL~\citep{fu2020d4rl} with the directed and undirected datasets.
 \arxiv{Numbers are averaged across 6 seeds, $\pm$ the 95$\%$-confidence interval.} We include per-task performance for walker2d and Meta-World domains and the overall performance averaged across tasks (highlighted in gray) for all three domains. We bold the highest score across all methods. \cdsmethodname\ achieves the best or comparable performance on all of these environments.
}
\label{tbl:gym}
\normalsize
\vspace{-0.3cm}
\end{table*}
%%AK: things seemed a bit diverging with CDS (basic) on antmaze, I guess it is fine for now since it is not the main method, but we could revisit back later during camera-ready.

\begin{table*}[t!]
\small{
\centering
\vspace*{0.1cm}
\resizebox{\textwidth}{!}{\begin{tabular}{l|r|r|r|r|r}
\toprule
\textbf{Task Name} & \textbf{\cdsmethodname\ (ours)}& \textbf{HIPI}~\cite{eysenbach2020rewriting} & \textbf{Skill~\cite{kalashnikov2021mt}} & \textbf{Sharing All} & \textbf{No Sharing}\\ \midrule
\texttt{lift-banana} & \textbf{53.1\%}$\pm$3.2\% & 48.3\%$\pm$6.0\%  & 32.1\%$\pm$9.5\% & 41.8\%$\pm$4.2\% & 20.0\%$\pm$6.0\%\\
\texttt{lift-bottle} & \textbf{74.0\%}$\pm$6.3\% & 64.4\%$\pm$7.7\%  & 55.9\%$\pm$9.6\% & 60.1\%$\pm$10.2\% & 49.7\%$\pm$8.7\%\\
\texttt{lift-sausage} & \textbf{71.8\%}$\pm$3.9\%  & 71.0\%$\pm$7.7\%  & 68.8\%$\pm$9.3\% & 70.0\%$\pm$7.0\% & 60.9\%$\pm$6.6\%\\
\texttt{lift-milk}& \textbf{83.4\%}$\pm$5.2\% & 79.0\%$\pm$3.9\% & 68.2\%$\pm$3.5\% & 72.5\%$\pm$5.3\% & 68.4\%$\pm$6.1\%\\

\texttt{lift-food} & 61.4\%$\pm$9.5\% & \textbf{62.6\%}$\pm$6.3\% & 41.5\%$\pm$12.1\% & 58.5\%$\pm$7.0\% & 39.1\%$\pm$7.0\%\\
\texttt{lift-can} & 65.5\%$\pm$6.9\% & \textbf{67.8\%}$\pm$6.8\%  & 50.8\%$\pm$12.5\% & 57.7\%$\pm$7.2\% & 49.1\%$\pm$9.8\%\\
\texttt{lift-carrot} & \textbf{83.8\%}$\pm$3.5\% & 78.8\%$\pm$6.9\% & 66.0\%$\pm$7.0\%& 75.2\%$\pm$7.6\%& 69.4\%$\pm$7.6\%\\
\texttt{place-bowl} & \textbf{81.0\%}$\pm$8.1\%  & 77.2\%$\pm$8.9\% & 80.8\%$\pm$6.9\% & 70.8\%$\pm$7.8\% & 80.3\%$\pm$8.6\%\\
\texttt{place-plate} & 85.8\%$\pm$6.6\%  & 83.6\%$\pm$7.9\% & 78.4\%$\pm$9.6\% & 78.7\%$\pm$7.6\% & \textbf{86.1}\%$\pm$7.7\%\\
\texttt{place-divider-plate} & \textbf{87.8\%}$\pm$7.6\%  & 78.0\%$\pm$10.5\% & 80.8\%$\pm$5.3\% & 79.2\%$\pm$6.3\% & 85.0\%$\pm$5.9\%\\
%\midrule
\CC \textbf{average} & \CC \textbf{74.8\%}$\pm$6.4\%  & \CC 71.1\%$\pm$7.5\% & \CC 62.3\%$\pm$8.9\% & \CC 66.4\%$\pm$7.2\% & \CC 60.8\%$\pm$7.5\%\\
\bottomrule
\end{tabular}}
\vspace{-0.1cm}
\caption{\footnotesize Results for multi-task vision-based robotic manipulation from \citep{kalashnikov2021mt}. \arxiv{Numbers are averaged across 3 seeds, $\pm$ the 95$\%$ confidence interval.} We consider 7 tasks denoted as \texttt{lift-object} where the goal of each task is to lift a different object and 3 tasks denoted as \texttt{place-fixture} that aim to place a lifted object onto different fixtures. \cdsmethodname\ outperforms both a skill-based data sharing strategy~\citep{kalashnikov2021mt} (\textbf{Skill}) and other data sharing methods on the average success rate (highlighted in gray) and 7 out of 10 per-task success rates.
}
\label{tbl:mtopt}
}
\vspace{-0.6cm}
\end{table*}


% \subsection{Evaluating CDS on Multi-Task Offline RL}
\textbf{Results on domains with low-dimensional states.} We present the results on all non-vision environments in Table~\ref{tbl:gym}.
% , but leave the results of halfcheetah and ant to Appendix~\ref{app:additional_exp}. 
\cdsmethodname\ achieves the 
best average performance across all environments except that on walker2d, it achieves the second best performance, obtaining slightly worse the other variant \cdsmethodname\ (basic). On the locomotion domain, we observe the most 
significant improvement on task \texttt{jump} on all three environments. We interpret this as strength of conservative data sharing, which mitigates the distribution shift that can be introduced by routing large amount of other task data to the task with limited data and narrow distribution. We also validate this by measuring the $D_\text{KL}(\pi, \pi_\beta)$ in Table~\ref{tab:analysis_cds} where $\pi_\beta$ is the behavior policy after we perform \cdsmethodname\
to share data. As shown in Table~\ref{tab:analysis_cds}, \cdsmethodname\ achieves lower KL divergence
 between the single-task optimal policy and the behavior policy after data sharing on task \texttt{jump} with limited expert data, whereas \textbf{Sharing All} results in much higher KL divergence compared to \textbf{No Sharing} as discussed in Section~\ref{sec:cds_analysis} and Table~\ref{tab:analysis}. Hence, \cdsmethodname\ is able to mitigate distribution shift when sharing data and result in performance boost.


On the Meta-World tasks, we find that the agent without data sharing completely fails to solve most of the tasks due to the low quality of the medium replay datasets and the insufficient data for the expert datasets. \textbf{Sharing All} improves performance since in the sparse reward settings, data sharing can introduce more supervision signal and help training. \cdsmethodname\ further improves over \textbf{Sharing All}, suggesting that \cdsmethodname\ can not only prevent harmful data sharing, but also lead to more effective multi-task learning compared to \textbf{Sharing All} in scenarios where data sharing is imperative. It's worth noting that \cdsmethodname\ (basic) performs worse than \cdsmethodname\ and \textbf{Sharing All}, indicating that relabeling data that only mitigates distributional shift is too pessimistic and might not be sufficient to discover the shared structure across tasks.

%\textbf{Goal-conditioned AntMazes~\citep{fu2020d4rl}.} 

On AntMaze, we observe that \cdsmethodname\ performs better than \textbf{Sharing All} and significantly outperforms HIPI in all four settings. Perhaps surprisingly, \textbf{No Sharing} is a strong baseline, however, is outperformed by \cdsmethodname\ with the harder undirected data. Moreover, \cdsmethodname\ performs on-par or better in the undirected setting compared to the directed setting, indicating the effectiveness of \cdsmethodname\ in routing data in challenging settings. 




% Concretely, we consider three different styles of datasets in gym domains: medium-replay for \texttt{run forward}, medium for \texttt{run backward}, and expert for \texttt{run jump}, following the convention proposed in \citep{fu2020d4rl}.

% Since it is common to generate a large number of data from multiple policies with different levels of performance, we use 120K, 120K and 250K datapoints 
% %%CF.5.22: are you referring to different checkpoints or different #s of datapoints?
% %%TY.5.26: I mean different number of datapoints.
% medium-replay datasets in halfcheetah, walker2d and ant respectively. Meanwhile, it is usually time-consuming to collect medium or high quality data in the real world and thus we provide 25K, 25K and 50K datapoints for medium datasets and 5K, 5K and 20K datapoints for expert datasets for the aforementioned three environments respectively.
% As analyzed in Section~\ref{sec:analysis} and Table~\ref{tab:analysis}, na\"ively sharing data across all tasks would hurt multi-task performance when behavior policies differ in each task. To test if \cdsmethodname\ can mitigate this issue, we adopt medium-replay datasets for task \texttt{run forward}, medium datasets for task \texttt{run backward} and expert datasets for task \texttt{jump}.










% \subsection{Results on image-based robotic manipulation tasks}
% \label{sec:mtopt_results}

\textbf{Results on image-based robotic manipulation domains.}  %\textbf{Skill} manually defines two oracle skills ``picking'' and ``placing'' and shares all picking tasks share data with each other but does not share data with any of the placing tasks and vice-versa. 
Here, we compare \cdsmethodname\ to the hand-designed \textbf{Skill} sharing strategy, in addition to the other methods. \arxiv{Given that \cdsmethodname\ achieves significantly better performance than \cdsmethodname\ (basic) on low-dimensional robotic manipulation tasks in Meta-World, we only evaluate \cdsmethodname\ in the vision-based robotic manipulation domains.}  Since \cdsmethodname\ is applicable to any offline multi-task RL algorithm, we employ it as a separate data-sharing strategy in \citep{kalashnikov2021mt} while keeping the model architecture and all the other hyperparameters constant, which allows us to carefully evaluate the influence of data sharing in isolation. The results are reported in Table~\ref{tbl:mtopt}. \cdsmethodname\ outperforms both \textbf{Skill} and other approaches, indicating that \cdsmethodname\ is able to scale to high-dimensional observation inputs and can effectively remove the need for manual curation of data sharing strategies.



\section{Conclusion}
\vspace{-5pt}
\label{sec:conclusion}
In this paper, we study the multi-task offline RL setting, focusing on the problem of sharing offline data across tasks for better multi-task learning. Through empirical analysis, we identify that na\"{i}vely sharing data across tasks generally helps learning but can significantly hurt performance in scenarios where excessive distribution shift is introduced. To address this challenge, we present conservative data sharing (CDS), which relabels data to a task when the conservative Q-value of the given transition is better than the expected conservative Q-value of the target task. On multitask locomotion, manipulation, navigation, and vision-based manipulation domains, CDS consistently outperforms or achieves comparable performance to existing data sharing approaches. While CDS attains superior results, it is not able to handle data sharing in settings where dynamics vary across tasks. Another limitation is that CDS requires functional forms of rewards. We leave these as future work. 
% Another limitation of CDS is that it requires access to the functional form of the reward. Though this is a common assumption as discussed in Section~\ref{sec:prelims}, an interesting future direction would be removing this dependence and replacing it with either a learned reward predictor or other supervision signal like language instructions.

Finally, regarding negative societal impact, one potential risk the reward design in RL which is challenging to obtain in the real world. For example, it is generally infeasible to specify rewards for (offline) RL in real-world applications, making (offline) RL hard to scale to real robots. Moreover, training procedures of ML models are generally compute-intensive and collecting large datasets for offline RL is also burdensome, which cause inequitable access to these models and datasets.  We would like push RL community to address these risks in the near future.

\begin{ack}
Use unnumbered first level headings for the acknowledgments. All acknowledgments
go at the end of the paper before the list of references. Moreover, you are required to declare
funding (financial activities supporting the submitted work) and competing interests (related financial activities outside the submitted work).
More information about this disclosure can be found at: \url{https://neurips.cc/Conferences/2021/PaperInformation/FundingDisclosure}.

Do {\bf not} include this section in the anonymized submission, only in the final paper. You can use the \texttt{ack} environment provided in the style file to autmoatically hide this section in the anonymized submission.
\end{ack}

\bibliography{reference}
\bibliographystyle{plainnat}

%%%%%%%%%%%%%%%%%%%%%%%%%%%%%%%%%%%%%%%%%%%%%%%%%%%%%%%%%%%%
\section*{Checklist}


\begin{enumerate}

\item For all authors...
\begin{enumerate}
  \item Do the main claims made in the abstract and introduction accurately reflect the paper's contributions and scope?
    \answerYes{}
  \item Did you describe the limitations of your work?
    \answerYes{Please see Section~\ref{sec:conclusion}.}
  \item Did you discuss any potential negative societal impacts of your work?
    \answerYes{Please see Section~\ref{sec:conclusion}.}
  \item Have you read the ethics review guidelines and ensured that your paper conforms to them?
    \answerYes{}
\end{enumerate}

\item If you are including theoretical results...
\begin{enumerate}
  \item Did you state the full set of assumptions of all theoretical results?
    \answerYes{}
	\item Did you include complete proofs of all theoretical results?
    \answerYes{Proofs can be found in Appendix~\ref{app:proofs}.}
\end{enumerate}

\item If you ran experiments...
\begin{enumerate}
  \item Did you include the code, data, and instructions needed to reproduce the main experimental results (either in the supplemental material or as a URL)?
    \answerYes{We provide code, data and instructions for Gym and Antmaze experiments and code/data for all experiments will be made public with the final.}
  \item Did you specify all the training details (e.g., data splits, hyperparameters, how they were chosen)?
    \answerYes{Please see Appendix~\ref{app:details}}
	\item Did you report error bars (e.g., with respect to the random seed after running experiments multiple times)?
    \answerYes{Please see Appendix~\ref{app:details} for the full results with error bars.}
	\item Did you include the total amount of compute and the type of resources used (e.g., type of GPUs, internal cluster, or cloud provider)?
    \answerYes{Please see Appendix~\ref{app:details}}
\end{enumerate}

\item If you are using existing assets (e.g., code, data, models) or curating/releasing new assets...
\begin{enumerate}
  \item If your work uses existing assets, did you cite the creators?
    \answerYes{Yes we cite \citet{fu2020d4rl}, \citet{yu2020metaworld}, \citet{kalashnikov2021mt}.}
  \item Did you mention the license of the assets?
    \answerYes{Please see Appendix~\ref{app:details}}
  \item Did you include any new assets either in the supplemental material or as a URL?
    \answerYes{We provide datasets for some of our tasks with the supplementary material.}
  \item Did you discuss whether and how consent was obtained from people whose data you're using/curating?
    \answerYes{Datasets in \citep{fu2020d4rl,yu2020metaworld} are open-sourced and we corresponded with the authors of \citep{kalashnikov2021mt} to access the dataset.}
  \item Did you discuss whether the data you are using/curating contains personally identifiable information or offensive content?
    \answerYes{There is no personally identifiable information or offensive content, since the tasks are simulated RL tasks.}
\end{enumerate}

\item If you used crowdsourcing or conducted research with human subjects...
\begin{enumerate}
  \item Did you include the full text of instructions given to participants and screenshots, if applicable?
    \answerNA{}
  \item Did you describe any potential participant risks, with links to Institutional Review Board (IRB) approvals, if applicable?
    \answerNA{}
  \item Did you include the estimated hourly wage paid to participants and the total amount spent on participant compensation?
    \answerNA{}
\end{enumerate}

\end{enumerate}

%%%%%%%%%%%%%%%%%%%%%%%%%%%%%%%%%%%%%%%%%%%%%%%%%%%%%%%%%%%%
\newpage
\appendix

\section{COMBO Proofs from Section~\ref{sec:combo_theory}}
\label{app:combo_proofs}

In this section, we provide proofs for theoretical results in Section~\ref{sec:combo_theory}. Before the proofs, we note that all statements are proven in the case of finite state space (i.e., $|\states| < \infty$) and finite action space (i.e., $|\actions| < \infty$) we define some commonly appearing notation symbols appearing in the proof: 
\begin{itemize}
\vspace{-5pt}
    \item $P_{\mdp}$ and $r_{\mdp}$ (or $P$ and $r$ with no subscript for notational simplicity) denote the dynamics and reward function of the actual MDP $\mdp$
    \vspace{-5pt}
    \item $P_{\mdpbar}$ and $r_{\mdpbar}$ denote the dynamics and reward of the empirical MDP $\mdpbar$ generated from the transitions in the dataset
    \vspace{-5pt}
    \item $P_{\mdphat}$ and $r_{\mdphat}$ denote the dynamics and reward of the MDP induced by the learned model $\mdphat$
\end{itemize}
\vspace{-5pt}
We also assume that whenever the cardinality of a particular state or state-action pair in the offline dataset $\data$, denoted by $|\mathcal{D}(\bs, \mathbf{a})|$, appears in the denominator, we assume it is non-zero. For any non-existent $(\bs, \mathbf{a}) \notin \data$, we can simply set $|\data(\bs, \mathbf{a})|$ to be a small value $< 1$, which prevents any bound from producing trivially $\infty$ values.

\subsection{A Useful Lemma and Its Proof}
\label{app:proof_lemma}

Before proving our main results, we first show that the penalty
term in equation \ref{eqn:combo_iterate} is positive in expectation. Such a positive penalty is important to combat any overestimation that may
arise as a result of using $\bellmanhat$.

\begin{lemma}[(Interpolation Lemma]
\label{thm:line_thm}
For any $f \in [0, 1]$, and any given $\rho(\bs, \mathbf{a}) \in \Delta^{|\states||\actions|}$, let $d_f$ be an f-interpolation of $\rho$ and $\data$, i.e., $d_f(\bs, \mathbf{a}) := f d(\bs, \mathbf{a}) + (1-f) \rho(\bs, \mathbf{a})$. For a given iteration $k$ of Equation~\ref{eqn:combo_iterate}, we restate the definition of the expected penalty under $\rho(\bs, \mathbf{a})$ in Eq.~\ref{eqn:expected_penalty}: 
\begin{equation*}
 \nu(\rho, f) := \E_{\bs, \mathbf{a} \sim \rho(\bs, \mathbf{a})}\left[\frac{\rho(\bs, \mathbf{a}) - d(\bs, \mathbf{a})}{d_f(\bs, \mathbf{a})} \right].
\end{equation*}
Then $\nu(\rho, f)$ satisfies, (1) $\nu(\rho, f) \geq 0,~~ \forall \rho, f$, (2) $\nu(\rho, f)$ is monotonically increasing in $f$ for a fixed $\rho$, and (3) $\nu(\rho, f) = 0$ iff $\forall~ \bs, \mathbf{a}, ~\rho(\bs, \mathbf{a}) = d(\bs, \mathbf{a}) \text{~or~} f = 0$. 
\end{lemma}
\begin{proof}
To prove this lemma, we use algebraic manipulation on the expression for quantity $\nu(\rho, f)$ and show that it is indeed positive and monotonically increasing in $f \in [0, 1]$.
\begin{align}
    \nu(\rho, f) &= \sum_{\bs, \mathbf{a}} \rho(\bs, \mathbf{a}) \left(\frac{\rho(\bs, \mathbf{a}) - d(\bs, \mathbf{a})}{f d(\bs, \mathbf{a}) + (1 - f) \rho(\bs, \mathbf{a})}\right)\nonumber \\
    &= \sum_{\bs, \mathbf{a}} \rho(\bs, \mathbf{a}) \left(\frac{\rho(\bs, \mathbf{a}) - d(\bs, \mathbf{a})}{\rho(\bs, \mathbf{a}) + f ( d(\bs, \mathbf{a}) - \rho(\bs, \mathbf{a}))}\right)\\
    \implies \frac{d \nu(\rho, f)}{d f} &= \sum_{\bs, \mathbf{a}} \rho(\bs, \mathbf{a}) \left(\rho(\bs, \mathbf{a}) - d(\bs, \mathbf{a})\right)^2 \cdot \left(\frac{1}{(\rho(\bs, \mathbf{a}) + f ( d(\bs, \mathbf{a}) - \rho(\bs, \mathbf{a}))}\right)^2 \geq 0\nonumber\\
    &~~~\forall f \in [0, 1].
\end{align}
Since the derivative of $\nu(\rho, f)$ with respect to $f$ is always positive, it is an increasing function of $f$ for a fixed $\rho$, and this proves the second part (2) of the Lemma. Using this property, we can show the part (1) of the Lemma as follows:
\begin{align}
    \forall f \in (0, 1],~ \nu(\rho, f) \geq \nu(\rho, 0) = \sum_{\bs, \mathbf{a}} \rho(\bs, \mathbf{a}) \frac{\rho(\bs, \mathbf{a}) - d(\bs, \mathbf{a})}{\rho(\bs, \mathbf{a})} &= \sum_{\bs, \mathbf{a}} \left( \rho(\bs, \mathbf{a}) - d(\bs, \mathbf{a}) \right)\nonumber\\
    &= 1 - 1 = 0.
\end{align}
Finally, to prove the third part (3) of this Lemma, note that when $f = 0$, $\nu(\rho, f) = 0$ (as shown above), and similarly by setting $\rho(\bs, \mathbf{a}) = d(\bs, \mathbf{a})$ note that we obtain $\nu(\rho, f) = 0$. To prove the only if side of (3), assume that $f \neq 0$ and $\rho(\bs, \mathbf{a}) \neq d(\bs, \mathbf{a})$ and we will show that in this case $\nu(\rho,f) \neq 0$. When $d(\bs, \mathbf{a}) \neq \rho(\bs, \mathbf{a})$, the derivative $\frac{d \nu(\rho,f)}{d f} > 0$ (i.e., strictly positive) and hence the function $\nu(\rho, f)$ is a strictly increasing function of $f$. Thus, in this case, $\nu(\rho, f) > 0 = \nu(\rho, 0)~ \forall f > 0$. Thus we have shown that if $\rho(\bs, \mathbf{a}) \neq d(\bs, \mathbf{a})$ and $f > 0$, $\nu(\rho, f) \neq 0$, which completes our proof for the only if side of (3). 
\end{proof}

\subsection{Proof of Proposition~\ref{thm:lower_bound}}
\label{app:proof_lower_bound}
Before proving this proposition, we provide a bound on the Bellman backup in the empirical MDP, $\bellman_{\mdpbar}$. To do so, we formally define the standard concentration properties of the reward and transition dynamics in the empirical MDP, $\mdpbar$, that we assume so as to prove Proposition~\ref{thm:line_thm}. Following prior work~\citep{osband2017posterior,jaksch2010near,kumar2020conservative}, we assume:
\begin{assumption}
\label{assumption:conc}
    $\forall~ \bs, \mathbf{a} \in \mdp$, the following relationships hold with high probability, $\geq 1 - \delta$
    \begin{equation*}
        |r_{\mdpbar}(\bs, \mathbf{a}) - r(\bs, \mathbf{a})| \leq \frac{C_{r, \delta}}{\sqrt{|\mathcal{D}(\bs, \mathbf{a})|}}, ~~~ ||P_{\mdpbar}(\bs'|\bs, \mathbf{a}) - P(\bs'|\bs, \mathbf{a})||_{1} \leq \frac{C_{P, \delta}}{\sqrt{|\mathcal{D}(\bs, \mathbf{a})|}}.
    \end{equation*}
\end{assumption}
Under this assumption and assuming that the reward function in the MDP, $r(\bs, \mathbf{a})$ is bounded, as $|r(\bs, \mathbf{a})| \leq R_{\max}$, we can bound the difference between the empirical Bellman operator, $\bellman_{\mdpbar}$ and the actual MDP, $\bellman_\mdp$,
\begin{align*}
    \left\vert\left({\bellman_{\mdpbar}}^\policy \hat{Q}^k \right) - \left({\bellman}^\policy_\mdp \hat{Q}^k \right)\right\vert &= \left\vert\left(r_{\mdpbar}(\bs, \mathbf{a}) - r_\mdp(\bs, \mathbf{a})\right)\right.\\
    &\left.+ \gamma \sum_{\bs'} \left({P}_{\mdpbar}(\bs'|\bs, \mathbf{a}) - P_\mdp(\bs'|\bs,\mathbf{a})\right) \E_{\policy(\mathbf{a}'|\bs')}\left[\hat{Q}^k(\bs' , \mathbf{a}')\right]\right\vert\\
    &\leq \left\vert r_{\mdpbar}(\bs, \mathbf{a}) - r_\mdp(\bs, \mathbf{a})\right\vert\\
    &+ \gamma \left\vert \sum_{\bs'} \left({P}_{\mdpbar}(\bs'|\bs, \mathbf{a}) - P_\mdp(\bs'|\bs,\mathbf{a})\right) \E_{\policy(\mathbf{a}'|\bs')}\left[\hat{Q}^k(\bs' , \mathbf{a}')\right]\right\vert\\
    &\leq \frac{C_{r, \delta} + \gamma C_{P, \delta} 2R_{\max} / (1 - \gamma)}{\sqrt{|\mathcal{D}(\bs, \mathbf{a})|}}. 
\end{align*}
Thus the overestimation due to sampling error in the empirical MDP, $\mdpbar$ is bounded as a function of a bigger constant, $C_{r, P, \delta}$ that can be expressed as a function of $C_{r, \delta}$ and $C_{P, \delta}$, and depends on $\delta$ via a $\sqrt{\log (1/\delta)}$ dependency. For the purposes of proving Proposition~\ref{thm:Q_bound}, we assume that:
\begin{equation}
\label{eqn:sampling_error}
    \forall \bs, \mathbf{a}, ~~\left\vert\left({\bellman_{\mdpbar}}^\policy \hat{Q}^k \right) - \left({\bellman}^\policy_\mdp \hat{Q}^k \right)\right\vert  \leq \frac{C_{r, T, \delta} R_{\max}}{(1 - \gamma) \sqrt{|\mathcal{D}(\bs, \mathbf{a})|}}.
\end{equation}

Next, we provide a bound on the error between the bellman backup induced by the learned dynamics model and the learned reward, $\bellman_{\mdphat}$, and the actual Bellman backup, $\bellman_{\mdp}$. To do so, we note that:
\begin{align}
    \left\vert\left({\bellman_{\mdphat}}^\policy \hat{Q}^k \right) - \left({\bellman}^\policy_\mdp \hat{Q}^k \right)\right\vert &= \left\vert\left(r_{\mdphat}(\bs, \mathbf{a}) - r_\mdp(\bs, \mathbf{a})\right)\right.\\
    &\left.+ \gamma \sum_{\bs'} \left({P}_{\mdphat}(\bs'|\bs, \mathbf{a}) - P_\mdp(\bs'|\bs,\mathbf{a})\right) \E_{\policy(\mathbf{a}'|\bs')}\left[\hat{Q}^k(\bs' , \mathbf{a}')\right]\right\vert \nonumber\\ 
    &\leq |r_{\mdphat}(\bs, \mathbf{a}) - r_\mdp(\bs, \mathbf{a})| + \gamma \frac{2 R_{\max}}{1 - \gamma} D(P, P_{\mdphat}),
    \label{eqn:model_error} 
\end{align}
where $D(P, P_{\mdphat})$ is the total-variation divergence between the learned dynamics model and the actual MDP. Now, we show that the asymptotic Q-function learned by COMBO lower-bounds the actual Q-function of any
policy $\pi$ with high probability for a large enough $\beta \geq 0$. We will use Equations~\ref{eqn:sampling_error} and \ref{eqn:model_error} to prove such a result.

\begin{proposition}[Asymptotic lower-bound]
\label{thm:Q_bound}
Let $P^\pi$ denote the Hadamard product of the dynamics $P$ and a given policy $\pi$ in the actual MDP and let $S^\pi := (I - \gamma P^\pi)^{-1}$. Let $D$ denote the total-variation divergence between two probability distributions. For any $\pi(\mathbf{a}|\bs)$, the Q-function obtained by recursively applying Equation~\ref{eqn:combo_iterate}, with $\hat{{\bellman}}^\pi = f \bellman_{\mdpbar}^\pi + (1 - f) \bellman_{\mdphat}^\pi$, with probability at least $1 - \delta$, results in $\hat{Q}^\pi$ that satisfies:
\begin{align*}
    \forall \bs, \mathbf{a},~ \hat{Q}^\pi(\bs, \mathbf{a}) \leq  Q^\pi(\bs, \mathbf{a}) &- \beta \cdot \left[ S^\pi \left[ \frac{\rho - d}{d_f} \right] \right](\bs, \mathbf{a}) + f \left[ S^\pi \left[ \frac{C_{r, T, \delta} R_{\max}}{(1 - \gamma) \sqrt{|\data|}} \right] \right](\bs, \mathbf{a})\\
    +&~ (1 - f) \left[ S^\pi \left[ |r - r_{\mdphat}| + \frac{ 2 \gamma  R_{\max}}{1 - \gamma} D(P, P_{\mdphat}) \right]  \right]\!\! (\bs, \mathbf{a}).
\end{align*}
\end{proposition}
\begin{proof}
We first note that the Bellman backup $\hat{\bellman}^\pi$ induces the following Q-function iterates as per Equation~\ref{eqn:combo_iterate},
\begin{align*}
    \hat{Q}^{k+1}(\bs, \mathbf{a}) &= \left(\hat{\bellman}^\pi \hat{Q}^k\right)(\bs, \mathbf{a}) - \beta \frac{\rho(\bs, \mathbf{a}) - d(\bs, \mathbf{a})}{d_f(\bs, \mathbf{a})}\\
    &=  f \left(\bellman^\pi_{\mdpbar} \hat{Q}^k \right) (\bs, \mathbf{a}) + (1 - f) \left(\bellman^\pi_{\mdphat} \hat{Q}^k \right) (\bs, \mathbf{a}) - \beta \frac{\rho(\bs, \mathbf{a}) - d(\bs, \mathbf{a})}{d_f(\bs, \mathbf{a})}\\
    &= \left(\bellman^\pi \hat{Q}^k\right)(\bs, \mathbf{a}) - \beta \frac{\rho(\bs, \mathbf{a}) - d(\bs, \mathbf{a})}{d_f(\bs, \mathbf{a})} + (1 - f) \left({\bellman_{\mdphat}}^\policy \hat{Q}^k - {\bellman}^\policy \hat{Q}^k \right)(\bs, \mathbf{a})\\
    &+ f  \left({\bellman_{\mdpbar}}^\policy \hat{Q}^k - {\bellman}^\policy \hat{Q}^k \right)(\bs, \mathbf{a})\\
   \forall \bs, \mathbf{a},~ \hat{Q}^{k+1} &\leq \left(\bellman^\pi \hat{Q}^k\right) - \beta \frac{\rho - d}{d_f} + (1 - f) \left[|r_{\mdphat} - r_\mdp| + \frac{2 \gamma R_{\max}}{1 - \gamma} D(P, P_{\mdphat}) \right] + f \frac{C_{r, T, \delta} R_{\max}}{(1 - \gamma) \sqrt{|\data|}} 
\end{align*}
Since the RHS upper bounds the Q-function pointwise for each $(\bs, \mathbf{a})$, the fixed point of the Bellman iteration process will be pointwise smaller than the fixed point of the Q-function found by solving for the RHS via equality. Thus, we get that
\begin{align*}
    \hat{Q}^\pi(\bs, \mathbf{a}) &\leq \underbrace{ S^\pi r_{\mdp}}_{= Q^\pi(\bs, \mathbf{a})} -\beta \left[ S^\pi \left[ \frac{\rho - d}{d_f} \right] \right](\bs, \mathbf{a}) +~ f \left[ S^\pi \left[ \frac{C_{r, T, \delta} R_{\max}}{(1 - \gamma) \sqrt{|\data|}} \right] \right](\bs, \mathbf{a})\\
    &+~ (1 - f) \left[ S^\pi \left[ |r - r_{\mdphat}| + \frac{ 2 \gamma  R_{\max}}{1 - \gamma} D(P, P_{\mdphat}) \right]  \right]\!\! (\bs, \mathbf{a}),  
\end{align*}
which completes the proof of this proposition.
\end{proof}

Next, we use the result and proof technique from Proposition~\ref{thm:Q_bound} to prove Corollary~\ref{thm:lower_bound}, that in expectation under the initial state-distribution, the expected Q-value is indeed a lower-bound. 

\begin{corollary}[Corollary~\ref{thm:lower_bound} restated]
For a sufficiently large $\beta$, we have a lower-bound that
$\E_{\bs \sim \mu_0, \mathbf{a} \sim \policy(\cdot|\bs)}[\hat{Q}^\pi(\bs, \mathbf{a})] \leq \E_{\bs \sim \mu_0, \mathbf{a} \sim \policy(\cdot|\bs)}[Q^\pi(\bs, \mathbf{a})]$, 
where $\mu_0(\bs)$ is the initial state distribution. 
Furthermore, when $\epsilon_{\text{s}}$ is small, such as in the large sample regime; or when the model bias $\epsilon_{\text{m}}$ is small, a small $\beta$ is sufficient along with an appropriate choice of $f$.
\end{corollary}

\begin{proof}
To prove this corollary, we note a slightly different variant of Proposition~\ref{thm:Q_bound}. To observe this, we will deviate from the proof of Proposition~\ref{thm:Q_bound} slightly and will aim to express the inequality using $\bellman_{\mdphat}$, the Bellman operator defined by the learned model and the reward function. Denoting $(I - \gamma P_{\mdphat})^{-1}$ as $S_{\mdphat}^\pi$, doing this will intuitively allow us to obtain $\beta \left(\mu(\bs) \policy(\mathbf{a}|\bs)\right)^T \left(S_{\mdphat}^\pi \left[\frac{\rho - d}{d_f} \right]\right)(\bs, \mathbf{a})$ as the conservative penalty which can be controlled by choosing $\beta$ appropriately so as to nullify the potential overestimation caused due to other terms. Formally,
\begin{align*}
    \hat{Q}^{k+1}(\bs, \mathbf{a}) &= \left(\hat{\bellman}^\pi \hat{Q}^k\right)(\bs, \mathbf{a}) - \beta \frac{\rho(\bs, \mathbf{a}) - d(\bs, \mathbf{a})}{d_f(\bs, \mathbf{a})} = \left(\bellman^\pi_{\mdphat} \hat{Q}^k \right)(\bs, \mathbf{a}) -  \beta \frac{\rho(\bs, \mathbf{a}) - d(\bs, \mathbf{a})}{d_f(\bs, \mathbf{a})}\\
    &+ f \underbrace{\left(\bellman^\pi_{\mdpbar} - \bellman^\pi_{\mdphat} \hat{Q}^k \right)(\bs, \mathbf{a})}_{:= \Delta(\bs, \mathbf{a})}
\end{align*}
By controlling $\Delta(\bs, \mathbf{a})$ using the pointwise triangle inequality:
\begin{equation}
    \forall \bs, \mathbf{a}, ~\left\vert \bellman^\pi_{\mdpbar} \hat{Q}^k - \bellman^\pi_{\mdphat} \hat{Q}^k \right\vert \leq \left\vert \bellman^\pi \hat{Q}^k - \bellman^\pi_{\mdphat} \hat{Q}^k \right\vert + \left\vert \bellman^\pi_{\mdpbar} \hat{Q}^k - \bellman^\pi \hat{Q}^k \right\vert,
\end{equation}
and then iterating the backup $\bellman^\pi_{\mdphat}$ to its fixed point and finally noting that $\rho(\bs, \mathbf{a}) = \left((\mu \cdot \pi)^T S^\pi_{\mdphat}\right)(\bs, \mathbf{a})$, we obtain:
\begin{equation}
    \E_{\mu, \pi}[\hat{Q}^\pi(\bs, \mathbf{a})] \leq \E_{\mu, \pi}[Q^\pi_{\mdphat}(\bs, \mathbf{a})] - \beta~ \E_{\rho(\bs, \mathbf{a})}\left[\frac{\rho(\bs, \mathbf{a}) - d(\bs, \mathbf{a})}{d_f(\bs, \mathbf{a})}\right] + \mathrm{terms~ independent~ of~} \beta.
\end{equation}
%%AK: there is one more term in the equation above, fit it in one line...
The terms marked as ``terms independent of $\beta$'' correspond to the additional positive error terms obtained by iterating $\left\vert \bellman^\pi \hat{Q}^k - \bellman^\pi_{\mdphat} \hat{Q}^k \right\vert$ and $\left\vert \bellman^\pi_{\mdpbar} \hat{Q}^k - \bellman^\pi \hat{Q}^k \right\vert$, which can be bounded similar to the proof of Proposition~\ref{thm:Q_bound} above. Now by replacing the model Q-function, $\E_{\mu, \pi}[Q^\pi_{\mdphat}(\bs, \mathbf{a})]$ with the actual Q-function, $\E_{\mu, \pi}[Q^\pi(\bs, \mathbf{a})]$ and adding an error term corresponding to model error to the bound, we obtain that:
\begin{equation}
\label{eqn:lower_bound_eqn}
    \E_{\mu, \pi}[\hat{Q}^\pi(\bs, \mathbf{a})] \leq \E_{\mu, \pi}[Q^\pi(\bs, \mathbf{a})] + \mathrm{terms~ independent~ of~} \beta - \beta~ \underbrace{\E_{\rho(\bs, \mathbf{a})}\left[\frac{\rho(\bs, \mathbf{a}) - d(\bs, \mathbf{a})}{d_f(\bs, \mathbf{a})}\right]}_{= \nu(\rho, f) > 0}.
\end{equation}
Hence, by choosing $\beta$ large enough, we obtain the desired lower bound guarantee. 
\end{proof}

\begin{remark}[\underline{\textbf{COMBO does not underestimate at every $\bs \in \mathcal{D}$ unlike CQL.}}]
\label{remak:tighter_lower_bound}
Before concluding this section, we discuss how the bound obtained by COMBO (Equation~\ref{eqn:lower_bound_eqn}) is tighter than CQL. CQL learns a Q-function such that the value of the policy under the resulting Q-function lower-bounds the true value function at each state $\bs \in \mathcal{D}$ individually (in the absence of no sampling error), i.e., $\forall \bs \in \mathcal{D}, \hat{V}^\pi_{\text{CQL}}(\bs) \leq V^\pi(\bs)$, whereas the bound in COMBO is only valid in expectation of the value function over the initial state distribution, i.e., $\E_{\bs \sim \mu_0(\bs)}[\hat{V}^\pi_{\text{COMBO}}(\bs)] \leq \E_{\bs \sim \mu_0(\bs)}[V^\pi(\bs)]$, and the value function at a given state may not be a lower-bound. For instance, COMBO can overestimate the value of a state more frequent in the dataset distribution $d(\bs, \mathbf{a})$ but not so frequent in the $\rho(\bs, \mathbf{a})$ marginal distribution of the policy under the learned model $\mdphat$. To see this more formally, note that the expected penalty added in the effective Bellman backup performed by COMBO (Equation~\ref{eqn:combo_iterate}), in expectation under the dataset distribution $d(\bs, \mathbf{a})$, $\widetilde{\nu}(\rho, d, f)$ is actually \textbf{\textit{negative}}:
\begin{align*}
    \widetilde{\nu}(\rho, d, f) = \sum_{\bs, \mathbf{a}} d(\bs, \mathbf{a}) \frac{\rho(\bs, \mathbf{a}) - d(\bs, \mathbf{a})}{d_f(\bs, \mathbf{a})} = - \sum_{\bs, \mathbf{a}} d(\bs, \mathbf{a}) \frac{d(\bs, \mathbf{a}) - \rho(\bs, \mathbf{a})}{f d(\bs, \mathbf{a}) + (1 - f) \rho(\bs, \mathbf{a})} < 0,
\end{align*}
where the final inequality follows via a direct application of the proof of Lemma~\ref{thm:line_thm}. Thus, COMBO actually \emph{overestimates} the values at atleast some states (in the dataset) unlike CQL.   
\end{remark}

\subsection{Proof of Proposition~\ref{prop:less_conservative}}
\label{app:proof_less_conservative}

In this section, we will provide a proof for Proposition~\ref{prop:less_conservative}, and show that the COMBO can be less conservative in terms of the estimated value. To recall, let $\Delta^\pi_\text{COMBO} := \E_{\bs, \mathbf{a} \sim d_{\mdpbar}(\bs), \pi(\mathbf{a}|\bs)}\left[\hat{Q}^\pi(\bs, \mathbf{a} \right]$ and let $\Delta^\pi_\text{CQL} := \E_{\bs, \mathbf{a} \sim d_{\mdpbar}, \pi(\mathbf{a}|\bs)} \left[\hat{Q}^\pi_\text{CQL}(\bs, \mathbf{a}) \right]$. From \citet{kumar2020conservative}, we obtain that $\hat{Q}^\pi_\text{CQL}(\bs, \mathbf{a}) := Q^\pi(\bs, \mathbf{a}) - \beta \frac{\pi(\mathbf{a}|\bs) - \pi_\beta(\mathbf{a}|\bs)}{\pi_\beta(\mathbf{a}|\bs)}$. We shall derive the condition for the real data fraction $f=1$ for COMBO, thus making sure that $d_f(\bs) = d^{\pi_\beta}(\bs)$. To derive the condition when $\Delta^\pi_\text{COMBO} \geq \Delta^\pi_\text{CQL}$, we note the following simplifications:
\begin{align}
    & \Delta^\pi_\text{COMBO} \geq \Delta^\pi_\text{CQL} \\
    \implies & \sum_{\bs, \mathbf{a}} d_{\mdpbar}(\bs) \pi(\mathbf{a}|\bs) \hat{Q}^\pi(\bs, \mathbf{a}) \geq \sum_{\bs, \mathbf{a}} d_{\mdpbar}(\bs) \pi(\mathbf{a}|\bs) \hat{Q}^\pi_\text{CQL}(\bs, \mathbf{a}) \\
    \label{eqn:cql_vs_combo_terms}
    \implies & \beta \sum_{\bs, \mathbf{a}} d_{\mdpbar}(\bs)\pi(\mathbf{a}|\bs) \left( \frac{\rho(\bs, \mathbf{a}) - d^{\pi_\beta}(\bs) \pi_\beta(\mathbf{a}|\bs)}{d^{\pi_\beta}(\bs) \pi_\beta(\mathbf{a}|\bs)} \right) \leq \beta \sum_{\bs, \mathbf{a}} d_{\mdpbar}(\bs)\pi(\mathbf{a}|\bs) \left(\frac{\pi(\mathbf{a}|\bs) - \pi_\beta(\mathbf{a}|\bs)}{\pi_\beta(\mathbf{a}|\bs)} \right).
\end{align}
Now, in the expression on the left-hand side, we add and subtract $d^{\pi_\beta}(\bs) \pi(\mathbf{a}|\bs)$ from the numerator inside the paranthesis.
\begin{align}
    & \sum_{\bs, \mathbf{a}} d_{\mdpbar}(\bs, \mathbf{a}) \left( \frac{\rho(\bs, \mathbf{a}) - d^{\pi_\beta}(\bs) \pi_\beta(\mathbf{a}|\bs)}{d^{\pi_\beta}(\bs) \pi_\beta(\mathbf{a}|\bs)} \right)\\
    &= \sum_{\bs, \mathbf{a}} d_{\mdpbar}(\bs, \mathbf{a}) \left( \frac{\rho(\bs, \mathbf{a}) - d^{\pi_\beta}(\bs) \pi(\mathbf{a}|\bs) + d^{\pi_\beta}(\bs) \pi(\mathbf{a}|\bs) - d^{\pi_\beta}(\bs) \pi_\beta(\mathbf{a}|\bs)}{d^{\pi_\beta}(\bs) \pi_\beta(\mathbf{a}|\bs)} \right)\\
    &= \underbrace{\sum_{\bs, \mathbf{a}} d_{\mdpbar}(\bs, \mathbf{a}) \frac{\pi(\mathbf{a}|\bs) - \pi_\beta(\mathbf{a}|\bs)}{\pi_\beta(\mathbf{a}|\bs)}}_{(1)} + \sum_{\bs, \mathbf{a}} d_{\mdpbar}(\bs, \mathbf{a}) \cdot \frac{\rho(\bs) - d^{\pi_\beta}(\bs)}{d^{\pi_\beta}(\bs)} \cdot \frac{\pi(\mathbf{a}|\bs)}{\pi_\beta(\mathbf{a}|\bs)}
\end{align}
The term marked $(1)$ is identical to the CQL term that appears on the right in Equation~\ref{eqn:cql_vs_combo_terms}. Thus the inequality in Equation~\ref{eqn:cql_vs_combo_terms} is satisfied when the second term above is negative. To show this, first note that $d^{\pi_\beta}(\bs) = d_{\mdpbar}(\bs)$ which results in a cancellation. Finally, re-arranging the second term into expectations gives us the desired result. An analogous condition can be derived when $f \neq 1$, but we omit that derivation as it will be hard to interpret terms appear in the final inequality.

\subsection{Proof of Proposition~\ref{thm:policy_improvement}}
\label{app:proof_policy_improvement}

To prove the policy improvement result in Proposition~\ref{thm:policy_improvement}, we first observe that using Equation~\ref{eqn:combo_iterate} for Bellman backups amounts to finding a policy that maximizes the return of the policy in the a modified ``f-interpolant'' MDP which admits the Bellman backup $\bellmanhat^\pi$, and is induced by a linear interpolation of backups in the empirical MDP $\mdpbar$ and the MDP induced by a dynamics model $\mdphat$ and the return of a policy $\pi$ in this effective f-interpolant MDP is denoted by $J(\mdpbar, \mdphat, f, \pi)$. Alongside this, the return is penalized by the conservative penalty where $\rho^\pi$ denotes the marginal state-action distribution of policy $\pi$ in the learned model $\mdphat$. 
\begin{equation}
    \hat{J}(f, \pi) = J(\mdpbar, \mdphat, f, \pi)  - \beta \frac{\nu(\rho^\pi, f)}{1 - \gamma}.
\label{eqn:penalized_objective}
\end{equation}
We will require bounds on the return of a policy $\pi$ in this f-interpolant MDP, $J(\mdpbar, \mdphat, f, \pi)$, which we first prove separately as Lemma~\ref{lemma:interpolant_regular_bound} below and then move to the proof of Proposition~\ref{thm:policy_improvement}.

\begin{lemma}[Bound on return in f-interpolant MDP]
\label{lemma:interpolant_regular_bound}
For any two MDPs, $\mdp_1$ and $\mdp_2$, with the same state-space, action-space and discount factor, and for a given fraction $f \in [0, 1]$, define the f-interpolant MDP $\mdp_f$ as the MDP on the same state-space, action-space and with the same discount as the MDP with dynamics: $P_{\mdp_f} := f P_{\mdp_1} + (1 - f) P_{\mdp_2}$ and reward function: $r_{\mdp_f} := f r_{\mdp_1} + (1 - f) r_{\mdp_2}$. Then, given any auxiliary MDP, $\mdp$, the return of any policy $\pi$ in $\mdp_f$, $J(\pi, \mdp_f)$, also denoted by $J(\mdp_1, \mdp_2, f, \pi)$, lies in the interval:
\begin{equation*}
    \big[ J(\pi, \mdp) - \alpha,~~ J(\pi, \mdp)+ \alpha \big], \text{~~~~~~~~~~~~where~} \alpha \text{~is given by:~}
\end{equation*}
\begin{align}
    \alpha &= \frac{2 \gamma (1 - f)}{(1 - \gamma)^2} R_{\max} D \left(P_{\mdp_2}, P_{\mdp}\right) + \frac{\gamma f}{1 - \gamma} \left\vert \E_{d^\pi_{\mdp} \pi} \left[ \left(P^\pi_{\mdp} - P^\pi_{\mdp_1}\right) Q^\pi_{\mdp} \right]\right\vert  \nonumber\\
   & + \frac{f}{1 - \gamma} \E_{\bs, \mathbf{a} \sim d^\pi_{\mdp} \pi}[|r_{\mdp_1}(\bs, \mathbf{a}) - r_{\mdp}(\bs, \mathbf{a})|] + \frac{1 - f}{1 - \gamma} \E_{\bs, \mathbf{a} \sim d^\pi_{\mdp} \pi}[|r_{\mdp_2}(\bs, \mathbf{a}) - r_{\mdp}(\bs, \mathbf{a})|].  \label{eqn:alpha_expr}
\end{align}
\end{lemma}
\begin{proof}
To prove this lemma, we note two general inequalities. First, note that for a fixed transition dynamics, say $P$, the return decomposes linearly in the components of the reward as the expected return is linear in the reward function:
\begin{equation*}
    J(P, r_{\mdp_f}) = J(P, f r_{\mdp_1} + (1 - f) r_{\mdp_2}) = f J (P, r_{\mdp_1}) + (1 - f) J(P, r_{\mdp_2}).  
\end{equation*}
As a result, we can bound $J(P, r_{\mdp_f})$ using $J(P, r)$ for a new reward function $r$ of the auxiliary MDP, $\mdp$, as follows
\begin{align*}
     J(P, r_{\mdp_f}) &= J(P, f r_{\mdp_1} + (1 - f) r_{\mdp_2}) = J (P, r + f (r_{\mdp_1} - r) + (1 -f) (r_{\mdp_2} - r)\\
     &= J(P, r) + f J(P, r_{\mdp_1} - r) + (1 - f) J(P, r_{\mdp_2} - r)\\
     &= J(P, r) + \frac{f}{1 - \gamma} \E_{\bs, \mathbf{a} \sim d^\pi_{\mdp}(\bs) \pi(\mathbf{a}|\bs)}\left[ r_{\mdp_1}(\bs, \mathbf{a}) - r(\bs, \mathbf{a}) \right]\\
     &+ \frac{1 - f}{1 - \gamma} \E_{\bs, \mathbf{a} \sim d^\pi_{\mdp}(\bs) \pi(\mathbf{a}|\bs)} \left[ r_{\mdp_2}(\bs, \mathbf{a}) - r(\bs, \mathbf{a}) \right].
\end{align*}
Second, note that for a given reward function, $r$, but a linear combination of dynamics, the following bound holds:
\begin{align*}
    J(P_{\mdp_f}, r) &= J(f P_{\mdp_1} + (1 - f) P_{\mdp_2}, r)\\
    &= J ( P_{\mdp} +  f( P_{\mdp_1} - P_{\mdp}) + (1 - f) (P_{\mdp_2} - P_{\mdp}), r)\\ 
    &= J (P_{\mdp}, r) - \frac{\gamma (1 - f)}{1 - \gamma} \E_{\bs, \mathbf{a} \sim d^\pi_{\mdp}(\bs) \pi(\mathbf{a}|\bs)} \left[ \left(P^\pi_{\mdp_2} - P^\pi_{\mdp}\right) Q^\pi_{\mdp}  \right]\\
    &- \frac{\gamma f}{1 - \gamma} \E_{\bs, \mathbf{a} \sim d^\pi_{\mdp}(\bs) \pi(\mathbf{a}|\bs)} \left[ \left(P^\pi_{\mdp} - P^\pi_{\mdp_1}\right) Q^\pi_{\mdp}  \right]\\
    &\in \left[ J( P_{\mdp}, r) ~\pm~ \left(\frac{\gamma f}{(1 - \gamma)} \left\vert \E_{\bs, \mathbf{a} \sim d^\pi_{\mdp}(\bs) \pi(\mathbf{a}|\bs)}\left[ \left(P^\pi_{\mdp} - P^\pi_{\mdp_1}\right) Q^\pi_{\mdp} \right] \right\vert\right.\right.\\
    &\left.\left.+ \frac{2 \gamma (1 -f) R_{\max}}{(1 - \gamma)^2} D(P_{\mdp_2}, P_{\mdp}) \right) \right].
    % &\in \left[J (P_{\mdp_1}, r) ~~\pm~~ \frac{\gamma (1 -f) R_{\max}}{(1 - \gamma)^2} D(P_{\mdp}, P_{\mdp_2}) ~\pm~ (1 - f) \frac{\gamma}{1 - \gamma}  \E_{\bs, \mathbf{a} \sim d^\pi_{\mdp_1}(\bs) \pi(\mathbf{a}|\bs)} \left[ \left(P^\pi_{\mdp} - P^\pi_{\mdp_1}\right) Q^\pi  \right] \right]. 
\end{align*}
To observe the third equality, we utilize the result on the difference between returns of a policy $\pi$ on two different MDPs, $P_{\mdp_1}$ and $P_{\mdp_f}$ from \citet{ajksbook} (Chapter 2, Lemma 2.2, Simulation Lemma), and additionally incorporate the auxiliary MDP $\mdp$ in the expression via addition and subtraction in the previous (second) step. In the fourth step, we finally bound one term that corresponds to the learned model via the total-variation divergence $D(P_{\mdp_2}, P_{\mdp})$ and the other term corresponding to the empirical MDP $\mdpbar$ is left in its expectation form to be bounded later. 

Using the above bounds on return for reward-mixtures and dynamics-mixtures, proving this lemma is straightforward:
\begin{align*}
    & J(\mdp_1, \mdp_2, f, \pi) := J(P_{\mdp_f}, f r_{\mdp_1} + (1 - f) r_{\mdp_2}) = J(f P_{\mdp_1} + (1 -f) P_{\mdp_2}, r_{\mdp_f})\\
    &\in \left[ J(P_{\mdp_f}, r_{\mdp}) ~\pm\right.\\
    &\left.~ \underbrace{\left(\frac{f}{1 - \gamma} \E_{\bs, \mathbf{a} \sim d^\pi_{\mdp} \pi}[|r_{\mdp_1}(\bs, \mathbf{a}) - r_{\mdp}(\bs, \mathbf{a})|] + \frac{1 - f}{1 - \gamma} \E_{\bs, \mathbf{a} \sim d^\pi_{\mdp} \pi}[|r_{\mdp_2}(\bs, \mathbf{a}) - r_{\mdp}(\bs, \mathbf{a})|] \right)}_{:= \Delta_R} \right],
    % ~\pm~ \left(\frac{2 \gamma f (1 - f)}{(1 - \gamma)^2} R_{\max} D \left(P_{\mdp_2}, P_{\mdp}\right) + \frac{2 \gamma f (1 - f)}{1 - \gamma} \E_{d^\pi_{\mdp_1}} \left\vert \left[ \left(P^\pi_{\mdp} - P^\pi_{\mdp_1}\right) Q^\pi  \right] \right\vert \right) \right],
\end{align*}
where the second step holds via linear decomposition of the return of $\pi$ in $\mdp_f$ with respect to the reward interpolation, and bounding the terms that appear in the reward difference. For convenience, we refer to these offset terms due to the reward as $\Delta_R$. For the final part of this proof, we bound $J(P_{\mdp_f}, r_{\mdp})$ in terms of the return on the actual MDP, $J(P_{\mdp}, r_{\mdp})$, using the inequality proved above that provides intervals for mixture dynamics but a fixed reward function. Thus, the overall bound is given by $J(\pi, \mdp_f) \in [J(\pi, \mdp) - \alpha, J(\pi, \mdp) + \alpha]$, where $\alpha$ is given by:
\begin{align}
\label{eqn:alpha_expr_repeat}
    \alpha = \frac{2 \gamma (1 - f)}{(1 - \gamma)^2} & R_{\max} D \left(P_{\mdp_2}, P_{\mdp}\right) + \frac{\gamma f}{1 - \gamma} \left\vert \E_{d^\pi_{\mdp} \pi} \left[ \left(P^\pi_{\mdp} - P^\pi_{\mdp_1}\right) Q^\pi_{\mdp} \right]\right\vert + \Delta_R.
\end{align}
This concludes the proof of this lemma.
\end{proof}



Finally, we prove Theorem~\ref{thm:policy_improvement} that shows how policy optimization with respect to $\hat{J}(f, \pi)$ affects the performance in the actual MD by using Equation~\ref{eqn:penalized_objective} and building on the  analysis of pure model-free algorithms from \citet{kumar2020conservative}. We restate a more complete statement of the theorem below and present the constants at the end of the proof. 

\begin{theorem}[Formal version of Proposition~\ref{thm:policy_improvement}]
Let $\hat{\pi}_{\text{out}}(\mathbf{a}|\bs)$ be the policy obtained by COMBO.
%%CF: Would be nice to have this definition outside of the theorem so that the theorem is shorter/simpler
Then, the policy ${\pi}_{\text{out}}(\mathbf{a}|\bs)$ is a $\zeta$-safe policy improvement over ${\behavior}$ in the actual MDP $\mdp$, i.e., $J({\pi}_{\text{out}}, \mdp) \geq J({\behavior}, \mdp) - \zeta$, with probability at least $1 - \delta$, where $\zeta$ is given by (where $\rho^\beta(\bs, \mathbf{a}) := d^\behavior_{\mdphat}(\bs, \mathbf{a})$):
\begin{align*}
&\mathcal{O}\left(\frac{\gamma f}{(1 - \gamma)^2}\right) {\left[ \E_{\bs \sim d^{\pi_{\text{out}}}_{\mdp}}\left[ \sqrt{\frac{|\actions|}{|\data(\bs)|} (\mathrm{D}_{\text{CQL}}({\pi}_{\text{out}}, \behavior) + 1)} \right] \right]}\\
&+ \mathcal{O}\left(\frac{\gamma (1 - f)}{(1 - \gamma)^2}\right) {\mathrm{D_{TV}}(P_{\mdp}, P_{\mdphat})} - \beta \frac{\nu(\rho^{\pi_{\text{out}}}, f) - \nu(\rho^\beta, f)}{(1 - \gamma)}.
    % &- \underbrace{\left({J}(\mdpbar, \mdphat, f, \pi) - {J}(\mdpbar, \mdphat, f, \behavior) \right)}_{:= (3),~~ \geq \beta \frac{\nu(\rho, f)}{(1 - \gamma)}} 
\end{align*}
\end{theorem}

\begin{proof}
We first note that since policy improvement is not being performed in the same MDP, $\mdp$ as the f-interpolant MDP, $\mdp_f$, we need to upper and lower bound the amount of improvement occurring in the actual MDP due to the f-interpolant MDP. As a result our first is to relate $J(\pi, \mdp)$ and $J(\pi, \mdp_f) := J(\mdpbar, \mdphat, f, \pi)$ for any given policy $\pi$.

\textbf{Step 1: Bounding the return in the actual MDP due to optimization in the f-interpolant MDP.} By directly applying Lemma~\ref{lemma:interpolant_regular_bound} stated and proved previously, we obtain the following upper and lower-bounds on the return of a policy $\pi$:
\begin{equation*}
    J(\mdpbar, \mdphat, f, \pi) \in \left[ J(\pi, \mdp) - \alpha,~~ J(\pi, \mdp) + \alpha \right],
\end{equation*}
where $\alpha$ is shown in Equation~\ref{eqn:alpha_expr}. As a result, we just need to bound the terms appearing the expression of $\alpha$ to obtain a bound on the return differences. We first note that the terms in the expression for $\alpha$ are of two types: \textbf{(1)} terms that depend only on the reward function differences (captured in $\Delta_R$ in Equation~\ref{eqn:alpha_expr_repeat}), and \textbf{(2)} terms that depend on the dynamics (the other two terms in Equation~\ref{eqn:alpha_expr_repeat}). 

To bound $\Delta_R$, we simply appeal to concentration inequalities on reward (Assumption~\ref{assumption:conc}), and bound $\Delta_R$ as:
\begin{align*}
\Delta_R &:= \frac{f}{1 - \gamma} \E_{\bs, \mathbf{a} \sim d^\pi_{\mdp} \pi}[|r_{\mdp_1}(\bs, \mathbf{a}) - r_{\mdp}(\bs, \mathbf{a})|] + \frac{1 - f}{1 - \gamma} \E_{\bs, \mathbf{a} \sim d^\pi_{\mdp} \pi}[|r_{\mdp_2}(\bs, \mathbf{a}) - r_{\mdp}(\bs, \mathbf{a})|]\\
&\leq \frac{C_{r, \delta}}{1 - \gamma} \E_{\bs, \mathbf{a} \sim d^\pi_{\mdp}\pi} \left[\frac{1}{\sqrt{D(\bs, \mathbf{a})}}\right] + \frac{1}{1 - \gamma} ||R_{\mdp} - R_{\mdphat}|| := \Delta_R^u.
\end{align*}
Note that both of these terms are of the order of $\mathcal{O}(1/ (1 - \gamma))$ and hence they don't figure in the informal bound in Theorem~\ref{thm:policy_improvement} in the main text, as these are dominated by terms that grow quadratically with the horizon.
% First, we use algebraic manipulation to obtain the following decompositionof the difference in the return of $\pi_{\text{out}}$ and $\pi_\beta$ in the actual MDP, $\mdp$:
% \begin{align*}
%     J(\pi_{\text{out}}, \mdp) - J(\behavior, \mdp) &= f \left(J(\pi_{\text{out}}, \mdp) - J(\pi_{\text{out}}, \mdpbar) \right) + (1 - f) \left(J(\pi_{\text{out}}, \mdp) - J(\pi_{\text{out}}, \mdphat) \right)~~~ \text{(a): policy difference}\\
%     &+ f (J(\pi_{\text{out}}, \mdpbar) - J(\behavior, \mdpbar)) + (1 - f) \left(J(\pi_{\text{out}}, \mdphat) - J(\behavior, \mdphat) \right)~~~\text{(b): policy improvement} \\
%     &+ f \left(J(\behavior, \mdpbar) - J(\behavior, \mdp) \right) + (1 - f) \left(J(\behavior, \mdphat) - J(\behavior, \mdp) \right)~~~~~~ \text{(c): behavior difference}
% \end{align*}
% Terms (a) and (c) correspond to a weighted sum of the difference in the return estimates of the policies in the empirical MDP $\mdpbar$ and the actual MDP $\mdp$ and the model-induced MDP $\mdphat$, and the actual MDP $\mdp$. 
To bound the remaining terms in the expression for $\alpha$, we utilize a result directly from \citet{kumar2020conservative} for the empirical MDP, $\mdpbar$, which holds for any policy $\pi(\mathbf{a}|\bs)$, as shown below.
\begin{align*}
   &\frac{\gamma}{(1 - \gamma)} \left\vert \E_{\bs, \mathbf{a} \sim d^\pi_{\mdp}(\bs) \pi(\mathbf{a}|\bs)}\left[ \left(P^\pi_{\mdp} - P^\pi_{\mdp_1}\right) Q^\pi_{\mdp} \right] \right\vert \\
   &\leq \frac{2 \gamma R_{\max} C_{P, \delta}}{(1 - \gamma)^2} \mathbb{E}_{\bs \sim d^{\policy}_{\mdpbar}(\bs)}\left[ \frac{\sqrt{|\mathcal{A}|}}{\sqrt{|\mathcal{D}(\bs)|}} \sqrt{ D_{\text{CQL}}(\policy, \behavior)(\bs) + 1} \right].
    %%AK: technically I think this is a naive result and certainly the MOReL paper was not the first one to come up with this... so unclear if we should be citing it for this result...
\end{align*}

\textbf{Step 2: Incorporate policy improvement in the f-inrerpolant MDP.} Now we incorporate the improvement of policy $\pi_{\text{out}}$ over the policy $\behavior$ on a weighted mixture of $\mdphat$ and $\mdpbar$. In what follows, we derive a lower-bound on this improvement by using the fact that policy $\pi_{\text{out}}$ is obtained by maximizing $\hat{J}(f, \pi)$ from Equation~\ref{eqn:penalized_objective}. As a direct consequence of Equation~\ref{eqn:penalized_objective}, we note that 
\begin{equation}
\label{eqn:improvement_expanded}
    \hat{J}(f, \pi_{\text{out}}) =  J(\mdpbar, \mdphat, f, \pi_{\text{out}}) - \beta \frac{\nu(\rho^\pi, f)}{1 - \gamma} \geq \hat{J}(f, \behavior) =  J(\mdpbar, \mdphat, f, \behavior) - \beta {\frac{\nu(\rho^\beta, f)}{1 - \gamma}}
\end{equation}
% Now, observe that we can both upper and lower-bound $J(\mdpbar, \mdphat, f, \pi)$ in terms of the return of policy $\pi$, individually in each MDP, $\mdpbar$ and $\mdphat$. We state this result more formally in Lemma~\ref{lemma:interpolant_regular_bound}.

% Next, we will use the upper bound on $J(\mdpbar, \mdphat, f, \pi)$ from Lemma~\ref{lemma:interpolant_regular_bound} for policy $\pi = \pi_{\text{out}}$ and a lower-bound on $J(\mdpbar, \mdphat, f, \pi)$ for policy $\pi = \behavior$, in the case when the auxiliary MDP is given by $\mdp$ (the actual MDP) to replace the expressions for $J(\mdpbar, \mdphat, f, \pi_{\text{out}})$ and $J(\mdpbar, \mdphat, f, \behavior)$ in the improvement equation~\ref{eqn:improvement_expanded}. Thus using Lemma~\ref{lemma:interpolant_regular_bound} we obtain the following inequality:
Following \textbf{Step 1}, we will use the upper bound on $J(\mdpbar, \mdphat, f, \pi)$ for policy $\pi = \pi_{\text{out}}$ and a lower-bound on $J(\mdpbar, \mdphat, f, \pi)$ for policy $\pi = \behavior$ and obtain the following inequality:
\begin{align*}
    J(\pi_{\text{out}}, \mdp) - \beta \frac{\nu(\rho^\pi, f)}{1 - \gamma} ~&\geq~ \Big\{ J(\behavior, \mdp) - \beta \frac{\nu(\rho^\beta, f)}{1 - \gamma}
    - \frac{4 \gamma (1 - f) R_{\max}}{(1 - \gamma)^2} D(P_{\mdp}, P_{\mdphat}) \\ 
    &- \underbrace{\frac{2 \gamma f}{(1 - \gamma)}\left\vert\E_{d^{\pi_{\text{out}}}_{\mdp}} \left[ \left(P^{\pi_{\text{out}}}_{\mdp} - P^{\pi_{\text{out}}}_{\mdpbar}\right) Q^{\pi_{\text{out}}}_{\mdp}  \right] \right\vert}_{:= (*)}\nonumber\\
    &- \underbrace{\frac{4 \gamma R_{\max} C_{P, \delta} f}{(1 - \gamma)^2} \E_{\bs \sim d^\behavior_{\mdp}}\left[ \sqrt{\frac{|\actions|}{|\data(\bs)|}}\right]}_{:= (\wedge)} - \Delta_R^u \Big\}.
\end{align*}
The term marked by $(*)$ in the above expression can be upper bounded by the concentration properties of the dynamics as done in Step 1 in this proof: 
\begin{align}
\label{eqn:bound_mdp_mdphat}
    (*) \leq \frac{4 \gamma f C_{P, \delta} R_{\max}}{(1 - \gamma)^2} \mathbb{E}_{\bs \sim d^{{\pi_{\text{out}}}}_{\mdp}(\bs)}\left[ \frac{\sqrt{|\mathcal{A}|}}{\sqrt{|\mathcal{D}(\bs)|}} \sqrt{ D_{\text{CQL}}({\pi_{\text{out}}}, \behavior)(\bs) + 1} \right]. 
\end{align}
Finally, using Equation~\ref{eqn:bound_mdp_mdphat}, we can lower-bound the policy return difference as:
\begin{align*}
\begin{small}
    J(\pi_{\text{out}}, \mdp) - J(\behavior, \mdp) \geq \beta \frac{\nu(\rho^\pi, f)}{1 - \gamma} - \beta \frac{\nu(\rho^\beta, f)}{1 - \gamma} - \frac{4 \gamma (1 -f) R_{\max}}{(1 - \gamma)^2} D(P_{\mdp}, P_{\mdphat}) - (*) - \Delta_R^u.
\end{small}
\end{align*}
Plugging the bounds for terms (a), (b) and (c) in the expression for $\zeta$ where $J(\pi_{\text{out}}, \mdp) - J(\behavior, \mdp) \geq \zeta$, we obtain:
\begin{align}
\zeta &= \left({\frac{4f \gamma R_{\max} C_{P, \delta}}{(1 - \gamma)^2}} \right)\mathbb{E}_{\bs \sim d^{\policy_{\text{out}}}_{\mdp}(\bs)}\left[ \frac{\sqrt{|\mathcal{A}|}}{\sqrt{|\mathcal{D}(\bs)|}} \sqrt{ D_{\text{CQL}}(\policy_{\text{out}}, \behavior)(\bs) + 1} \right]  + (\wedge) - \Delta_R^u \nonumber\\
\label{eqn:zeta_expression}
&~~~~~~~~~~~~+ \frac{4 (1 -f) \gamma R_{\max}}{(1 - \gamma)^2} D(P_{\mdp}, P_{\mdphat}) - \beta \frac{\nu(\rho^\pi, f)}{1 - \gamma} + \beta \frac{\nu(\rho^\beta, f)}{1 - \gamma}.
\end{align}
\end{proof}

\begin{remark}[\underline{\textbf{Interpretation of Proposition~\ref{thm:policy_improvement}}}] 
\label{remark:remark1}
Now we will interpret the theoretical expression for $\zeta$ in Equation~\ref{eqn:zeta_expression}, and discuss the scenarios when it is \emph{negative}. When the expression for $\zeta$ is negative, the policy $\pi_{\text{out}}$ is an improvement over $\behavior$ in the original MDP, $\mdp$. 

\begin{itemize}
    \item First note that we have never used the fact that the learned model $P_{\mdphat}$ is close to the actual MDP, $P_{\mdp}$ on the states visited by the behavior policy $\behavior$ in our analysis. We will use this fact now: in practical scenarios, $\nu(\rho^\beta, f)$ is expected to be smaller than $\nu(\rho^\pi, f)$, since $\nu(\rho^\beta, f)$ is directly controlled by the difference and density ratio of $\rho^\beta(\bs, \mathbf{a})$ and $d(\bs, \mathbf{a})$: $\nu(\rho^\beta, f) \leq \nu(\rho^\beta, f=1) = \sum_{\bs, \mathbf{a}} d^\behavior_{\mdphat}(\bs, \mathbf{a}) \left(d^\behavior_{\mdphat}(\bs, \mathbf{a})/d^\behavior_{\mdpbar}(\bs, \mathbf{a}) - 1\right)^2$ by Lemma~\ref{thm:line_thm} which is expected to be small for the behavior policy $\behavior$ in cases when the behavior policy marginal in the empirical MDP, $d^\behavior_{\mdpbar}(\bs, \mathbf{a})$, is broad. This is a direct consequence of the fact that the learned dynamics integrated with the policy under the learned model: $P_{\mdphat}^\behavior$ is closer to its counterpart in the empirical MDP:  $P_{\mdpbar}^\behavior$ for $\behavior$. Note that this is not true for any other policy besides the behavior policy that performs several counterfactual actions in a rollout and deviates from the data. For such a learned policy $\pi$, we incur an extra error which depends on the importance ratio of policy densities, compounded over the horizon and manifests as the $D_{\mathrm{CQL}}$ term (similar to Equation~\ref{eqn:bound_mdp_mdphat}, or Lemma D.4.1 in \citet{kumar2020conservative}). Thus, in practice, we argue that we are interested in situations where $\nu(\rho^\pi, f) > \nu(\rho^\beta, f)$, in which case by increasing $\beta$, we can make the expression for $\zeta$ in Equation~\ref{eqn:zeta_expression} negative, allowing for policy improvement.
    \item In addition, note that when $f$ is close to 1, the bound reverts to a standard model-free policy improvement bound and when $f$ is close to 0, the bound reverts to a typical model-based policy improvement bound. In scenarios with high sampling error (i.e. smaller $|\mathcal{D}(\bs)|$), if we can learn a good model, i.e., $D(P_{\mdp}, P_{\mdphat})$ is small, we can attain policy improvement better than model-free methods by relying on the learned model by setting $f$ closer to 0. A similar argument can be made in reverse for handling cases when learning an accurate dynamics model is hard. 
\end{itemize}
\end{remark}

% \begin{theorem}[Upper bound on $\nu(\rho, f)$]
% If the distributions $\rho(\bs, \mathbf{a})$ and $d(\bs, \mathbf{a})$ are such that $\sum_{\bs, \mathbf{a}} (\rho(\bs, \mathbf{a}) - d(\bs, \mathbf{a}))^2 \leq \varepsilon$, then the value of $\nu(\rho, f) \leq $. 
% \end{theorem}
% \begin{proof}
% To obtain a bound on $\nu(\rho, f)$, we solve the following optimization problem over $\rho$:
% \begin{align*}
%     \max_{\rho}&~~~ \nu(\rho, f):= \sum_{\bs, \mathbf{a}} \rho(\bs, \mathbf{a}) \frac{\rho(\bs, \mathbf{a}) - d(\bs, \mathbf{a})}{f d (\bs, \mathbf{a}) + (1 - f) \rho(\bs, \mathbf{a})}\\
%     &\text{s.t.}~~ \sum_{\bs, \mathbf{a}} (\rho(\bs, \mathbf{a}) - d(\bs, \mathbf{a}))^2 \leq \varepsilon, ~~ \sum_{\bs, \mathbf{a}} \rho(\bs, \mathbf{a}) = 1, ~~ \rho(\bs, \mathbf{a}) \geq 0.
% \end{align*}
% We first note that for any optimal $\rho=\rho^*$, the objective value is largest for $f = 1$, and thus, solving the above optimization problem for $f=1$ gives an upper bound on the objective value. Converting the problem for $f=1$ to a minimization problem and writing out the Lagrangian for optimization, we obtain:
% \begin{multline}
%     \mathcal{L}(\rho; \lambda, \alpha, \eta) = -\sum_{\bs, \mathbf{a}} d(\bs, \mathbf{a}) \frac{\rho(\bs, \mathbf{a})}{d(\bs, \mathbf{a})}  \left( \frac{\rho(\bs, \mathbf{a})}{d(\bs, \mathbf{a})} - 1 \right) + \lambda \left(\sum_{\bs, \mathbf{a}} d(\bs, \mathbf{a})^2 \left(\frac{\rho(\bs, \mathbf{a})}{d(\bs, \mathbf{a})} - 1 \right)^2 - \varepsilon \right) \\ - \eta \left(\sum_{\bs, \mathbf{a}} d(\bs, \mathbf{a}) \frac{\rho(\bs, \mathbf{a})}{d(\bs, \mathbf{a})} - 1 \right) - \sum_{\bs, \mathbf{a}} \alpha(\bs, \mathbf{a}) \frac{\rho(\bs, \mathbf{a})}{d(\bs, \mathbf{a})}.
% \end{multline}
% Noting the change of variable transformation: $w(\bs, \mathbf{a}) := \frac{\rho(\bs, \mathbf{a})}{d(\bs, \mathbf{a})} - 1$, we obtain the following optimization problem:
% \begin{equation*}
%     \mathcal{L}(w; \lambda, \alpha, \eta) = -\sum_{\bs, \mathbf{a}} d(\bs, \mathbf{a}) w(\bs, \mathbf{a})^2 + \lambda \left( \sum_{\bs, \mathbf{a}} d(\bs, \mathbf{a})^2 w(\bs, \mathbf{a})^2 - \varepsilon \right) - \eta \sum_{\bs, \mathbf{a}} d(\bs, \mathbf{a}) w(\bs, \mathbf{a}) - \sum_{\bs, \mathbf{a}} \alpha(\bs, \mathbf{a}) (w(\bs, \mathbf{a}) + 1).
% \end{equation*}
% Taking the derivative with respect to $w(\bs, \mathbf{a})$ and utilizing KKT conditions we obtain
% \begin{align}
% &- 2 d(\bs, \mathbf{a}) w(\bs, \mathbf{a}) + 2 \lambda d(\bs, \mathbf{a})^2 w(\bs, \mathbf{a}) - \eta d(\bs, \mathbf{a}) - \alpha(\bs, \mathbf{a}) = 0   \label{eq:grad}\\
% &\lambda \left( \sum_{\bs, \mathbf{a}} d(\bs, \mathbf{a})^2 w(\bs, \mathbf{a})^2 - \varepsilon \right) = 0.  \label{eq:slack1}\\
% & \alpha(\bs, \mathbf{a}) (w(\bs, \mathbf{a}) + 1) = 0 ~~ \forall \bs, \mathbf{a}.  \label{eq:slack2}
% \end{align}
% Multiplying Equation~\ref{eq:grad} by $w(\bs, \mathbf{a})$ and adding both LHS and RHS over $(\bs, \mathbf{a})$ we obtain:
% \begin{equation}
%     \label{eq:temp_add}
%     - 2 \sum_{\bs, \mathbf{a}} d(\bs, \mathbf{a}) w(\bs, \mathbf{a})^2 + 2 \underbrace{\lambda \sum_{\bs, \mathbf{a}} d(\bs, \mathbf{a})^2 w(\bs, \mathbf{a})^2}_{= \lambda \varepsilon} - \underbrace{\eta \sum_{\bs, \mathbf{a}} d(\bs, \mathbf{a}) w(\bs, \mathbf{a})}_{= \eta \times 0 = 0} = \sum_{\bs, \mathbf{a}} \alpha(\bs, \mathbf{a}) w(\bs, \mathbf{a}),
% \end{equation}
% and similarly, adding Equation~\ref{eq:grad} over $(\bs, \mathbf{a})$ we get:
% \begin{equation}
%     \label{eq:simple_add}
%     - 2 \underbrace{\sum_{\bs, \mathbf{a}} d(\bs, \mathbf{a}) w(\bs, \mathbf{a})}_{= 0} + 2 \lambda \sum_{\bs, \mathbf{a}} d(\bs, \mathbf{a})^2 w(\bs, \mathbf{a}) - \eta = \sum_{\bs, \mathbf{a}} \alpha(\bs, \mathbf{a}).
% \end{equation}
% Finally, from Equation~\ref{eq:grad}, we get that the value of $w(\bs, \mathbf{a})$ is given by:
% \begin{equation*}
%     w(\bs, \mathbf{a}) = \frac{\eta d(\bs, \mathbf{a}) + \alpha (\bs, \mathbf{a})}{2 \lambda d(\bs, \mathbf{a})^2 - 2 d(\bs, \mathbf{a})}
% \end{equation*}
% Adding Equations~\ref{eq:temp_add} and \ref{eq:simple_add}, we obtain:
% \begin{equation*}
%     2 \sum_{\bs, \mathbf{a}} d(\bs, \mathbf{a}) w(\bs, \mathbf{a}) \left[\lambda d(\bs, \mathbf{a}) - w(\bs, \mathbf{a}) \right] + \lambda \varepsilon - \eta = 0. 
% \end{equation*}
% \end{proof}

\section{Experimental Details for COMBO}
\label{app:details}

In this section, we include all details of our empirical evaluations of COMBO.

\subsection{Practical algorithm implementation details}
\label{app:combo_details}

\paragraph{Model training.}

In the setting where the observation space is low-dimensional, as mentioned in Section~\ref{sec:combo},  we represent the model as a probabilistic neural network that outputs a Gaussian distribution over the next state and reward given the current state and action: $$\widehat{T}_\theta(\bs_{t+1}, r| \bs, \mathbf{a}) = \mathcal{N}(\mu_\theta(\bs_t, \mathbf{a}_t), \Sigma_\theta(\bs_t, \mathbf{a}_t)).$$ We train an ensemble of $7$ such dynamics models following \cite{janner2019mbpo} and pick the best $5$ models based on the validation prediction error on a held-out set that contains $1000$ transitions in the offline dataset $\data$. During model rollouts, we randomly pick one dynamics model from the best $5$ models. Each model in the ensemble is represented as a 4-layer feedforward neural network with $200$ hidden units. For the generalization experiments in Section~\ref{sec:generalization_exps}, we additionally use a two-head architecture to output the mean and variance after the last hidden layer following \cite{yu2020mopo}.

In the image-based setting, we follow \citet{Rafailov2020LOMPO} and use a variational model with the following components:

\begin{gather}
\begin{aligned}
&\text{Image encoder:} && \mathbf{h}_t=E_\theta(\bo_t) \\
&\text{Inference model:} && \bs_t \sim q_\theta(\bs_t|\mathbf{h}_t, \bs_{t-1}, \mathbf{a}_{t-1})\\
&\text{Latent transition model:} &&\bs_t \sim \widehat{T}_\theta(\bs_t| \bs_{t-1}, \mathbf{a}_{t-1})\\
&\text{Reward predictor:} && r_t \sim p_\theta(r_t|\bs_t) \\
&\text{Image decoder:} && \bo_t \sim D_\theta(\bo_t|\bs_t).
\label{eq:latent_model}
\end{aligned}
\end{gather}%

We train the model using the evidence lower bound:

$$\max_{\theta}\sum_{\tau=0}^{T-1}\Big[\mathbb{E}_{q_{\theta}}[\log D_{\theta}(\bo_{\tau+1}|\bs_{\tau+1})]\Big]-\mathbb{E}_{q_{\theta}}\Big[D_{KL}[q_{\theta}(\bo_{\tau+1}, \bs_{\tau+1}|\bs_{\tau}, \mathbf{a}_{\tau})\|\widehat{T}_{\theta_{\tau}}(\bs_{\tau+1}, a_{\tau+1})]\Big]$$

At each step $\tau$ we sample a latent forward model $\widehat{T}_{\theta_{\tau}}$ from a fixed set of $K$ models $[\widehat{T}_{\theta_1},\ldots, \widehat{T}_{\theta_K}]$. For the encoder $E_{\theta}$ we use a convolutional neural network with kernel size 4 and stride 2. For the Walker environment we use 4 layers, while the Door Opening task has 5 layers. The $D_{\theta}$ is a transposed convolutional network with stride 2 and kernel sizes $[5,5,6,6]$ and $[5,5,5,6,6]$ respectively. The inference network has a two-level structure similar to \citet{Hafner2019PlanNet} with a deterministic path using a GRU cell with 256 units and a stochastic path implemented as a conditional diagonal Gaussian with 128 units. We only train an ensemble of stochastic forward models, which are also implemented as conditional diagonal Gaussians.


\paragraph{Policy Optimization.} We sample a batch size of $256$ transitions for the critic and policy learning. We set $f = 0.5$, which means we sample $50\%$ of the batch of transitions from $\data$ and another $50\%$ from $\data_\text{model}$. The equal split between the offline data and the model rollouts strikes the balance between conservatism and generalization in our experiments as shown in our experimental results in Section~\ref{sec:combo_exp}. We represent the Q-networks and policy as 3-layer feedforward neural networks with $256$ hidden units.

For the choice of $\rho(\bs,\mathbf{a})$ in Equation~\ref{eq:implicit_update}, we can obtain the Q-values that lower-bound the true value of the learned policy $\pi$ by setting $\rho(\bs,\mathbf{a}) = d^\policy_{\mdphat} (\bs) \pi(\mathbf{a} | \bs)$. However, as discussed in \cite{kumar2020conservative}, computing $\pi$ by alternating the full off-policy evaluation for the policy $\hat{\pi}^k$ at each iteration $k$ and one step of policy improvement is computationally expensive. Instead, following \cite{kumar2020conservative}, we pick a particular distribution $\psi(\mathbf{a}|\bs)$ that approximates the the policy that maximizes the Q-function at the current iteration and set $\rho(\bs,\mathbf{a}) = d^\policy_{\mdphat} (\bs) \psi(\mathbf{a} | \bs)$. We formulate the new objective as follows:
\begin{small}
\begin{align}
    \hat{Q}^{k+1} \leftarrow& \arg\min_{Q}\beta\left(\E_{\bs \sim d^\policy_{\mdphat} (\bs), \mathbf{a}\sim \psi(\mathbf{a} | \bs)}\!\left[Q(\bs,\mathbf{a})\right]-\E_{\bs, \mathbf{a} \sim \data}\left[Q(\bs,\mathbf{a})\right]\right)\nonumber\\
    &+ \frac{1}{2}\E_{\bs, \mathbf{a}, \bs' \sim d_f}\left[ \left(Q(\bs, \mathbf{a}) - \widehat{\bellman}^\policy\hat{Q}^k(\bs, \mathbf{a}))\right)^2 \right] + \mathcal{R}(\psi),
    \label{eq:combo_update_practical}
\end{align}
\end{small}
where $\mathcal{R}(\psi)$ is a regularizer on $\psi$. In practice, we pick $\mathcal{R}(\psi)$ to be the $-D_\text{KL}(\psi(\mathbf{a}|\bs)\|\text{Unif}(\mathbf{a}))$ and under such a regularization, the first term in Equation~\ref{eq:combo_update_practical} corresponds to computing softmax of the Q-values at any state $\bs$ as follows:
\begin{small}
\begin{align}
    \hat{Q}^{k+1} \leftarrow& \arg\min_{Q}\max_\psi\beta\left(\E_{\bs \sim d^\policy_{\mdphat} (\bs)}\!\left[\log\sum_\mathbf{a} Q(\bs,\mathbf{a})\right]-\E_{\bs, \mathbf{a} \sim \data}\left[Q(\bs,\mathbf{a})\right]\right) \nonumber\\
    &+ \frac{1}{2}\E_{\bs, \mathbf{a}, \bs' \sim d_f}\left[ \left(Q(\bs, \mathbf{a}) - \widehat{\bellman}^\policy\hat{Q}^k(\bs, \mathbf{a}))\right)^2 \right].
    \label{eq:combo_logsumexp}
\end{align}
\end{small}
We estimate the \texttt{log-sum-exp} term in Equation~\ref{eq:combo_logsumexp} by sampling $10$ actions at every state $\bs$ in the batch from a uniform policy $\text{Unif}(\mathbf{a})$ and the current learned policy $\pi(\mathbf{a}|\bs)$ with importance sampling following \cite{kumar2020conservative}.

\subsection{Hyperparameter Selection for COMBO}
\label{app:hyperparameter}

\neurips{In this section, we discuss the hyperparameters that we use for COMBO. In the D4RL and generalization experiments, our method are built upon the implementation of MOPO provided at: \url{https://github.com/tianheyu927/mopo}. The hyperparameters used in COMBO that relates to the backbone RL algorithm SAC such as twin Q-functions and number of gradient steps follow from those used in MOPO with the exception of smaller critic and policy learning rates, which we will discuss below. In the image-based domains, COMBO is built upon LOMPO without any changes to the parameters used there. For the evaluation of COMBO, we follow the evaluation protocol in D4RL~\citep{fu2020d4rl} and a variety of prior offline RL works~\citep{kumar2020conservative,yu2020mopo,kidambi2020morel} and report the normalized score of the smooth undiscounted averaged return over $3$ random seeds for all environments except \texttt{sawyer-door-close} and \texttt{sawyer-door} where we report the average success rate over $3$ random seeds.}

\neurips{We now list the additional hyperparameters as follows.
\begin{itemize}
    \item \textbf{Rollout length $h$.} We perform a short-horizon model rollouts in COMBO similar to \citet{yu2020mopo} and \citet{Rafailov2020LOMPO}. For the D4RL experiments and generalization experiments, we followed the defaults used in MOPO and used $h = 1$ for walker2d and \texttt{sawyer-door-close}, $h=5$ for hopper, halfcheetah and \texttt{halfcheetah-jump}, and $h=25$ for \texttt{ant-angle}. In the image-based domain we used rollout length of $h=5$ for both the the \texttt{walker-walk} and \texttt{sawyer-door-open} environments following the same hyperparameters used in \citet{Rafailov2020LOMPO}.
    \item \textbf{Q-function and policy learning rates.} On state-based domains, we searched over $\{1e-4, 3e-4\}$ for the Q-function learning rate and $\{1e-5, 3e-5, 1e-4\}$ for the policy learning rate. 
    We found that $3e-4$ for the Q-function learning rate (also used previously in \citet{kumar2020conservative}) and $1e-4$ for the policy learning rate (also recommended previously in \citet{kumar2020conservative} for gym domains) work well for almost all domains except that on walker2d where a smaller Q-function learning rate of $1e-4$ and a correspondingly smaller policy learning rate of $1e-5$ works the best. In the image-based domains, we followed the defaults from prior work \citep{Rafailov2020LOMPO} and used $3e-4$ for both the policy and Q-function.
    
    \item \textbf{Conservative coefficient $\beta$.} 
    % As noted in our theoretical results in Lemma~\ref{thm:line_thm}, the amount of conservatism depends on the choice of fraction $f$ and $\rho(\bs, \mathbf{a})$. In principle, we only need to control one of these factors, $\rho$, $f$, $\beta$ to obtain the right degree of conservatism. Since we do not alter $f$ and $\rho(\bs, \mathbf{a})$ for different quality datasets (see Appendix~\ref{app:combo_details} for our choice of $f$; $\rho$ was chosen based on model-prediction error as discussed next) we instead choose values of $\beta$ for different dataset types.
    We searched over $\{0.5, 1.0, 5.0\}$ for $\beta$, which correspond to low conservatism, medium conservatism and high conservatism.  A larger $\beta$ would be desirable in more narrow dataset distributions with lower-coverage of the state-action space that propagates error in a backup whereas a smaller $\beta$ is desirable with diverse dataset distributions. On the D4RL experiments, we found that $\beta = 0.5$ works well for halfcheetah agnostic of dataset quality, while on hopper and walker2d, we found that the more ``narrow'' dataset distributions: medium and medium-expert datasets work best with larger $\beta = 5.0$ whereas more ``diverse'' dataset distributions: random and medium-replay datasets work best with smaller $\beta$ ($\beta = 0.5$ for walker2d and $\beta = 1.0$ for hopper) which is consistent with the intuition. 
    % An intuitive explanation would be that on medium and medium-expert datasets where the data distribution is narrow, we need to be more conservative and hence large $\beta$ while on random and medium-replay datasets where the distribution is diverse and cover most of the state space, we require less conservatism. 
    On generalization experiments, $\beta = 1.0$ works best for all environments. In the image-domains we use $\beta=0.5$ for the medium-replay \texttt{walker-walk} task and and $\beta=1.0$ for all other domains, which again is in accordance with the impact of $\beta$ on performance.
    
    
    \item \textbf{Choice of $\rho(\bs,\mathbf{a})$.} We first decouple $\rho(\bs,\mathbf{a}) = \rho(\bs)\rho(\mathbf{a}|\bs)$ for convenience. As discussed in Appendix~\ref{app:combo_details}, we use $\rho(\mathbf{a}|\bs)$ as the soft-maximum of the Q-values and estimated with \texttt{log-sum-exp}. For $\rho(\bs)$, we searched over $\{d^\policy_{\mdphat}, \rho(\bs)=d_f\}$.  We found that $d^\policy_{\mdphat}$ works better the hopper task in D4RL while $d_f$ is better for the rest of the environments. For the remaining domains, we found $\rho(\bs)=d_f$ works well.
    
    
    \item \textbf{Choice of $\mu(\mathbf{a}|\bs)$.} For the rollout policy $\mu$, we searched $\{\text{Unif}(\mathbf{a}), \pi(\mathbf{a}|\bs)\}$, i.e. the set that contains a random policy and a current learned policy. We found that $\mu(\mathbf{a}|\bs) = \text{Unif}(\mathbf{a})$ works well on the hopper task in D4RL and also in the $\texttt{ant-angle}$ generalization experiment. For the remaining state-based environments, we discovered that $\mu(\mathbf{a}|\bs) = \pi(\mathbf{a}|\bs)$ excels. In the image-based domain, we found that $\mu(\mathbf{a}|\bs) = \text{Unif}(\mathbf{a})$ works well in the \texttt{walker-walk} domain and  $\mu(\mathbf{a}|\bs) = \pi(\mathbf{a}|\bs)$ is better for the \texttt{sawyer-door} environment. 
    % Similar to the choice of $\rho(\bs)$, 
    We observed that
    $\mu(\mathbf{a}|\bs) = \text{Unif}(\mathbf{a})$ behaves less conservatively and is suitable to tasks where dynamics models can be learned fairly precisely.
    \item \textbf{Choice of Backup.} Following CQL~\citep{kumar2020conservative}, we use the standard deterministic backup for COMBO.
    \item \textbf{Choice of $f$.} For the ratio between model rollouts and offline data $f$, we searched $\{0.5, 0.8\}$. We found that $f = 0.8$ works well on the medium and medium-expert in the walker2d task in D4RL. For the remaining tasks, we find $f = 0.5$ works well.
\end{itemize}}

\subsection{Automatic Hyperparameter Selection Rule for COMBO}

It is common in prior work on offline RL to select various hyperparameters using online policy rollouts~\citep{yu2020mopo,kidambi2020morel,argenson2020model,lee2021representation}. Requiring online rollouts to tune hyperparameters contradicts the main aim of offline RL, which is to learn entirely from offline data. Therefore, we attempted to devise an automated rule for tuning important hyperparameters such as $\beta$ and $f$ in a fully offline manner in COMBO. We search over a discrete set of hyperparameters for each task as dicussed above, and use the value of the regularization term $\mathbb{E}_{\mathbf{s}, \mathbf{a} \sim \rho(\mathbf{s},\mathbf{a})}\!\left[Q(\mathbf{s},\mathbf{a})\right]\!-\!\mathbb{E}_{\mathbf{s}, \mathbf{a} \sim \data}\!\left[Q(\mathbf{s},\mathbf{a})\right]$ (shown in Eq.~\ref{eq:implicit_update}) to evaluate the hyperparameters. This automated rule picks the hyperparameter setting which achieves the lowest regularization objective, which indicates that the Q-values on unseen model-predicted state-action tuples are not overestimated.
%%CF.9.30: The ICLR AC also wanted to see a discussion of how this offline selection scheme compared to prior methods for offline selection. Maybe discuss this somewhere? (perhaps in the appendix if space is short?)
%%TY.10.1: I discussed this in Appendix B.2.
%%SL.10.2: I slightly rephrased the paragraph above in a way that hopefully further avoids potential misunderstandings.

Below, we provide additional experimental validation showing the effiacy of this automatic hyperparameter selection rule from above. As shown in Table~\ref{tab:beta_selection},~\ref{tab:mu_selection}, ~\ref{tab:rho_selection} and \ref{tab:f_selection}, the above proposed automatic hyperparameter selection rule is able to pick the hyperparameters $\beta$, $\mu(\mathbf{a}|\bs)$, $\rho(\bs)$ and $f$ and  that correspond to the best policy performance based on the regularization value.

\begin{table}[ht]
    \centering
    \scriptsize
    \resizebox{1.0\textwidth}{!}{\begin{tabular}{l|r|r|r|r|}
    \toprule
    Task & $\beta=0.5$ & $\beta=0.5$ & $\beta=5.0$ & $\beta=5.0$\\
 & performance & regularizer value & performance & regularizer value\\
 \midrule
halfcheetah-medium &  \textbf{54.2}  & \textbf{-778.6}  & 40.8  & -236.8  \\
halfcheetah-medium-replay &  \textbf{55.1} & \textbf{28.9} & 9.3 & 283.9\\ 
halfcheetah-medium-expert & 89.4 & 189.8 & \textbf{90.0}  & \textbf{6.5}\\
hopper-medium      &  75.0  & -740.7  &\textbf{97.2}  & \textbf{-2035.9}\\
hopper-medium-replay & \textbf{89.5} & \textbf{37.7} & 28.3       & 107.2\\
hopper-medium-expert & \textbf{111.1}       & \textbf{-705.6}    & 75.3 &       -64.1\\
walker2d-medium        &  1.9  & 51.5  & \textbf{81.9}  & \textbf{-1991.2}\\
walker2d-medium-replay & \textbf{56.0}       & \textbf{-157.9}    & 27.0       & 53.6\\
walker2d-medium-expert & 10.3       & -788.3    &\textbf{103.3}       & \textbf{-3891.4}\\
    \bottomrule
\end{tabular}}
\caption{\footnotesize We include our automatic hyperparameter selection rule of $\beta$ on a set of representative D4RL environments. We show the policy performance (bold with the higher number) and the regularizer value (bold with the lower number). Lower regularizer value consistently corresponds to the higher policy return, suggesting the effectiveness of our automatic selection rule.}
\label{tab:beta_selection}
\end{table}

\begin{table}[ht]
    \centering
    \scriptsize
    \resizebox{1.0\textwidth}{!}{\begin{tabular}{l|r|r|r|r|}
    \toprule
    Task & $\mu(\mathbf{a}|\bs)=\text{Unif}(\mathbf{a})$ & $\mu(\mathbf{a}|\bs)=\text{Unif}(\mathbf{a})$            &$\mu(\mathbf{a}|\bs)=\pi(\mathbf{a}|\bs)$&$\mu(\mathbf{a}|\bs)=\pi(\mathbf{a}|\bs)$\\
 & performance & regularizer value & performance & regularizer value\\
 \midrule
hopper-medium        & \textbf{97.2}  & \textbf{-2035.9} &  52.6  & -14.9  \\
walker2d-medium        &  7.9  & -106.8  & \textbf{81.9}  & \textbf{-1991.2} \\
    \bottomrule
    \end{tabular}}
    \caption{\footnotesize We include our automatic hyperparameter selection rule of $\mu(\mathbf{a}|\bs)$ on the medium datasets in the hopper and walker2d environments from D4RL. We follow the same convention defined in Table~\ref{tab:beta_selection} and find that our automatic selection rule can effectively select $\mu$ offline.}
    \label{tab:mu_selection}
\end{table}

\begin{table}[ht]
    \centering
    \scriptsize
    \resizebox{0.9\textwidth}{!}{\begin{tabular}{l|r|r|r|r|}
    \toprule
    Task & $\rho(\bs) = d^\pi_{\hat{\mathcal{M}}} $&$\rho(\bs) = d^\pi_{\hat{\mathcal{M}}}$            &$\rho(\bs) = d_f$&$\rho(\bs) = d_f$\\
 & performance & regularizer value & performance & regularizer value\\
 \midrule
hopper-medium        & \textbf{97.2}  & \textbf{-2035.9} &  56.0  & -6.0  \\
walker2d-medium        &  1.8  & 14617.4  & \textbf{81.9}  & \textbf{-1991.2} \\
    \bottomrule
    \end{tabular}}
    \caption{\footnotesize We include our automatic hyperparameter selection rule of $\rho(\bs)$ on the medium datasets in the hopper and walker2d environments from D4RL. We follow the same convention defined in Table~\ref{tab:beta_selection} and find that our automatic selection rule can effectively select $\rho$ offline.}
    \label{tab:rho_selection}
\end{table}

\begin{table}[ht]
    \centering
    \scriptsize
    \resizebox{0.9\textwidth}{!}{\begin{tabular}{l|r|r|r|r|}
    \toprule
    Task & $f = 0.5 $&$f = 0.5$            &$f = 0.8$&$f = 0.8$\\
 & performance & regularizer value & performance & regularizer value\\
 \midrule
hopper-medium        & \textbf{97.2}  & \textbf{-2035.9} &  93.8  & -21.3  \\
walker2d-medium        &  70.9  & -1707.0  & \textbf{81.9}  & \textbf{-1991.2} \\
    \bottomrule
    \end{tabular}}
    \caption{\footnotesize We include our automatic hyperparameter selection rule of $f$ on the medium datasets in the hopper and walker2d environments from D4RL. We follow the same convention defined in Table~\ref{tab:beta_selection} and find that our automatic selection rule can effectively select $f$ offline.}
    \label{tab:f_selection}
\end{table}

\subsection{Details of generalization environments}
\label{app:ood_details}

For \texttt{halfcheetah-jump} and \texttt{ant-angle}, we follow the same environment used in MOPO. For \texttt{sawyer-door-close}, we train the \texttt{sawyer-door} environment in \url{https://github.com/rlworkgroup/metaworld} with dense rewards for opening the door until convergence. We collect $50000$ transitions with half of the data collected by the final expert policy and a policy that reaches the performance of about half the expert level performance. We relabel the reward such that the reward is $1$ when the door is fully closed and $0$ otherwise. Hence, the offline RL agent is required to learn the behavior that is different from the behavior policy in a sparse reward setting. We provide the datasets in the following anonymous link\footnote{The datasets of the generalization environments are available at the following link: \url{https://drive.google.com/file/d/1pn6dS5OgPQVp_ivGws-tmWdZoU7m_LvC/view?usp=sharing}.}.

\subsection{Details of image-based environments}
\label{app:image_details}

\begin{figure}[ht]
    \centering
    \includegraphics[width=0.25\textwidth]{chapters/combo/walker_task.png}
    \includegraphics[width=0.25\textwidth]{chapters/combo/dooropen_task.png}
    \vspace{-0.2cm}
    \caption{\footnotesize Our image-based environments: The observations are $64\times 64$ and $128\times 128$ raw RGB images for the \texttt{walker-walk} and \texttt{sawyer-door} tasks respectively. The \texttt{sawyer-door-close} environment used in in Section~\ref{sec:generalization_exps} also uses the \texttt{sawyer-door} environment.}
    \label{fig:visual}
\end{figure}


We visualize our image-based environments in Figure~\ref{fig:visual}. We use the standard \texttt{walker-walk} environment from \citet{tassa2018deepmind} with $64\times64$ pixel observations and an action repeat of 2. Datasets were constructed the same way as \citet{fu2020d4rl} with 200 trajectories each. For the \texttt{sawyer-door} we use $128\times128$ pixel observations. The medium-expert dataset contains 1000 rollouts (with a rollout length of 50 steps) covering the state distribution from grasping the door handle to opening the door. The expert dataset contains 1000 trajectories samples from a fully trained (stochastic) policy. The data was obtained from the training process of a stochastic SAC policy using dense reward function as defined in \citet{yu2020metaworld}. However, we relabel the rewards, so an agent receives a reward of 1 when the door is fully open and 0 otherwise. This aims to evaluate offline-RL performance in a sparse-reward setting. All the datasets are from \citep{Rafailov2020LOMPO}.


\section{Comparing COMBO to the Naive Combination of CQL and MBPO}
\label{app:cql_mbpo}

\iclr{In this section, we stress the distinction between COMBO and a direct combination of two previous methods CQL and MBPO (denoted as CQL + MBPO). CQL+MBPO performs Q-value regularization using CQL while expanding the offline data with MBPO-style model rollouts. While COMBO utilizes Q-value regularization similar to CQL, the effect is very different. CQL only penalizes the Q-value on unseen actions on the states observed in the dataset whereas COMBO penalizes Q-values on states generated by the learned model while maximizing Q values on state-action tuples in the dataset. Additionally, COMBO also utilizes MBPO-style model rollouts for also augmenting samples for training Q-functions.

To empirically demonstrate the consequences of this distinction, CQL + MBPO performs quite a bit worse than COMBO on generalization experiments (Section~\ref{sec:generalization_exps}) as shown in Table~\ref{tbl:cql_mbpo}. The results are averaged across 6 random seeds ($\pm$ denotes 95\%-confidence interval of the various runs). This suggests that carefully considering the state distribution, as done in COMBO, is crucial.}

\begin{table}[ht]
    \centering
    \scriptsize
    \resizebox{0.7\textwidth}{!}{\begin{tabular}{l|r|r|r|r|}
    \toprule 
    %
    %
    %
    \textbf{Environment} & \stackanchor{\textbf{Batch}}{\textbf{Mean}} & \stackanchor{\textbf{Batch}}{\textbf{Max}} & \stackanchor{\textbf{COMBO}}{\textbf{(Ours)}} & \textbf{CQL+MBPO}\\ \midrule
    halfcheetah-jump & -1022.6 & 1808.6 & \textbf{5392.7}$\pm$575.5 & 4053.4$\pm$176.9\\
    ant-angle & 866.7 & 2311.9 & \textbf{2764.8}$\pm$43.6 & 809.2$\pm$135.4\\
    sawyer-door-close & 5\% & 100\% & \textbf{100}\%$\pm$0.0\% & 62.7\%$\pm$24.8\%\\
    \bottomrule
    \end{tabular}}
    \vspace{-0.2cm}
    \caption{
    \footnotesize Comparison between COMBO and CQL+MBPO on tasks that require out-of-distribution generalization. Results are in average returns of \texttt{halfcheetah-jump} and \texttt{ant-angle} and average success rate of \texttt{sawyer-door-close}. All results are averaged over 6 random seeds, $\pm$ the $95\%$-confidence interval.
    }
    \vspace{-0.3cm}
    \label{tbl:cql_mbpo}
    \normalsize
    \end{table}
    

% \subsection{Computation Complexity}

% For the D4RL and generalization experiments, COMBO is trained on a single NVIDIA GeForce RTX 2080 Ti for one day. For the image-based experiments, we utilized a single NVIDIA GeForce RTX 2070. We trained the \texttt{walker-walk} tasks for a day and the \texttt{sawyer-door-open} tasks for about two days.

% \subsection{License of datasets}

% We acknowledge that all datasets used in this paper use the MIT license.

% % \vspace{1cm}
% \section{Empirical Evidence on Challenges of Uncertainty Quantification}
% \label{app:uq}

% \begin{figure}[t]
%     \centering
%     \includegraphics[width=0.47\linewidth]{halfcheetah_medium_corr_var_ood.png}
%     \includegraphics[width=0.47\linewidth]{halfcheetah_medium_corr_lip_ens_ood.png}
%     \includegraphics[width=0.47\linewidth]{hopper_medium_corr_var_ood.png}
%     \includegraphics[width=0.47\linewidth]{hopper_medium_corr_lip_ens_ood.png}
%     \includegraphics[width=0.47\linewidth]{walker_medium_corr_var_ood.png}
%     \includegraphics[width=0.47\linewidth]{walker_medium_corr_lip_ens_ood.png}
%     \vspace{-0.2cm}
%     \caption{\footnotesize
%     %
%     We visualize the correlation between the model error and two uncertainty quantification methods maximum learned variance over the ensemble (left column) and variance of the model prediction over the ensemble (right column) on three D4RL medium datasets (from the top row to the bottom row: halfcheetah, hopper and walker) where MOPO performs poorly compared to model-free methods. We show that \textbf{Max Var} tends to be overly conservative and overestimating the model error while \textbf{Ens. Var} is on the opposite. Such visualizations corroborate that uncertainty quantification is challenging with deep neural networks and could lead to poor performance in model-based offline RL. In the meantime, COMBO addresses this issue by removing the burden of performing uncertainty quantification.}
%     \label{fig:uq}
%     \vspace{-0.3cm}
% \end{figure}

% In this section, we perform empirical evaluations to show that uncertainty quantification with deep neural networks, especially in the setting of dynamics model learning, is challenging and could cause problems with uncertainty-based model-based offline RL methods such as MOReL~\citep{kidambi2020morel} and MOPO~\citep{yu2020mopo}. In our evaluations, we consider two uncertainty quantification methods, maximum learned variance over the ensemble (denoted as \textbf{Max Var}) $\max_{i=1,\dots,N}\|\Sigma^i_\theta(\bs,\mathbf{a})\|_\text{F}$ (used in MOPO) and the variance of the model prediction over the ensemble (denoted as \textbf{Ens. Var}) $\max_{i=1,\dots,N}\|\mu^i_\theta(\bs,\mathbf{a}) - \frac{1}{N}\sum_{j=1}^N\mu^j_\theta(\bs,\mathbf{a})\|_2$ (used in MOPO and MOReL) where we use an ensemble of $N$ probabilistic dynamics models $\{\widehat{T}^i_\theta(\bs_{t+1}, r| \bs, \mathbf{a}) = \mathcal{N}(\mu^i_\theta(\bs_t, \mathbf{a}_t), \Sigma^i_\theta(\bs_t, \mathbf{a}_t))\}_{i=1}^N$.

% As shown in Table~\ref{tbl:d4rl}, MOPO performs underwhelmingly on medium datasets in the D4RL datasets where the dataset is collected with a single policy and hence with relatively narrow data coverage of the whole state space. To empirically analyze the poor performance of MOPO on those datasets, we visualize the correlation between the true model error and two uncertainty quantification methods \textbf{Max Var} and \textbf{Ens. Var}. We normalize both the model error and the uncertainty estimates to be within scale $[0, 1]$. As shown in Figure~\ref{fig:uq}, on all three medium datasets, \textbf{Max Var} tends to be overly conservative and \textbf{Ens. Var} behaves too optimistic to correctly quantify the true model error, suggesting that uncertainty estimation used by MOPO is not accurate and might be the major factor that results in its poor performance. Meanwhile, COMBO circumvents challenging uncertainty quantification problem and achieves much better performances on those medium datasets, indicating the effectiveness and the robustness of the method.

\end{document}
