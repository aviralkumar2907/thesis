\vspace{-0.2cm}
\subsection{Notation and Problem Statement}
\vspace{-0.2cm}
\label{sec:prelims}
% In this section, we introduce notation, present our problem formulation, and review the base offline RL methods that our algorithm utilizes.

\textbf{Multi-task offline RL.} The goal in multi-task RL is to find a policy that maximizes expected return in a multi-task Markov decision process (MDP), defined as   $\mdp =(\states, \actions, P, \gamma, \{R_i, i \}_{i=1}^N)$, with state space $\states$, action space $\actions$, dynamics $P(\bs' | \bs, \mathbf{a})$, a discount factor $\gamma \in [0, 1)$, and a finite set of task indices $1, \cdots, N$
with corresponding reward functions $R_1, \cdots, R_N$. Each task $i$ presents a different reward function $R_i$, but we assume that the dynamics $P$ are shared across tasks. While this setting is not fully general, there are a wide variety of practical problem settings for which only the reward changes including various goal navigation tasks~\cite{fu2020d4rl}, distinct object manipulation objectives~\cite{xie2018few}, and different user preferences~\cite{christiano2017deep}.
In this work, we focus on learning a policy $\pi(\mathbf{a}|\bs, i)$, which in practice could be modelled as independent policies $\{\pi_1(\mathbf{a}|\bs), \cdots, \pi_N(\mathbf{a}|\bs)\}$ that do not share any parameters, or as a single task-conditioned policy, $\pi(\mathbf{a}|\bs, i)$ with parameter sharing. Our goal in this chapter is to analyze and devise methods for data sharing and the choice of parameter sharing is orthogonal, and can be made independently.
We formulate the policy optimization problem as finding a policy that maximizes expected return over all the tasks: $\pi^*(\mathbf{a}|\bs, \cdot) := \arg \max_{\pi} \mathbb{E}_{i \sim [N]} \mathbb{E}_{\pi(\cdot|\cdot, i)}[\sum_{t} \gamma^t R_i(\bs_t, \mathbf{a}_t)]$. 
% The Q-function, $Q^\pi(\bs, \mathbf{a}, i)$, of a policy $\pi(\cdot|\cdot, i)$ is the long-term discounted reward obtained in task $i$ by executing action $\mathbf{a}$ at state $\bs$ and following policy $\pi$ thereafter.
%%CF.5.22: the above paragraph seems to be a lot more than just notation and definitions, especially the part about the reward being the same across tasks (since that is part of our problem definition). How about you have the multi-task offline RL paragraph header for this paragraph? (and then no header for the next paragraph)
%%CF.5.22: Need to make it clear that we assume that we know the form of the reward function r_i, a common assumption in goal-conditioned RL~\cite{}, and which often holds in robotics applications through the use of learned classifiers~\cite{annie_flo,qtopt} and discriminators~\cite{vice,chen_dvd}.

%%SL.5.23: A subtlety in the problem formulation: a "task conditioned policy" is really just an architectural choice. Basically, we have a different policy for each task, but these policies may share parameters. It might be good to explain this detail though, because the point of our paper is how to share data, *not* how to share model weights. So in principle all this stuff would apply just as well if we had totally separate policies per task.

Standard offline RL is concerned with learning policies $\pi(\mathbf{a}|\bs)$ using only a given static dataset of transitions $\mathcal{D} =  \{(\bs_j, \mathbf{a}_j, \bs'_j, r_j)\}_{j=1}^N$, collected by a behavior policy $\pi_\beta(\mathbf{a}|\bs)$, without any additional environment interaction. In the multi-task offline RL setting, the dataset $\mathcal{D}$ is partitioned into per-task subsets, $\mathcal{D} = \cup_{i=1}^N \mathcal{D}_i$,
where $\mathcal{D}_i$ consists of experience from task $i$. While algorithms can choose to train the policy for task $i$ (i.e., $\pi(\cdot|\cdot, i)$) only on $\mathcal{D}_{i}$, in this paper, we are interested in data-sharing schemes that correspond to relabeling data from a different task, $j \neq i$ with the reward function $r_i$, and learn $\pi(\cdot|\cdot, i)$ on the combined data. To be able to do so, we assume access to the functional form of the reward $r_i$, a common assumption in goal-conditioned RL~\cite{andrychowicz2017hindsight,eysenbach2020rewriting}, and which often holds in robotics applications through the use of learned classifiers~\cite{xie2018few,kalashnikov2018scalable}, and discriminators~\cite{fu2018variational,chen2021learning}.

We assume that relabeling data $\mathcal{D}_j$ from task $j$ to task $i$ generates a dataset $\mathcal{D}_{j \rightarrow i}$, which is then additionally used to train on task $i$. Thus, the effective dataset for task $i$ after relabeling is given by $\mathcal{D}^\mathrm{eff}_i := \mathcal{D}_i \cup \left( \cup_{j \neq i} \mathcal{D}_{j \rightarrow i} \right)$. This notation simply formalizes data sharing and relabeling strategies explored in prior work~\citep{eysenbach2020rewriting,kalashnikov2021mt}. Our aim in this paper will be to improve on this na\"{i}ve strategy, which we will show leads to better results. 

\textbf{Offline RL algorithms.} For our analysis in this chapter, we will abstract conservative offline RL algorithms into a generic constrained policy optimization problem~\citep{kumar2020conservative}:
\begin{equation}
\label{eqn:generic_offline_rl}
    \pi^*(\mathbf{a}|\bs) := \arg \max_{\pi}~~ J_{\mathcal{D}}(\pi) - \alpha D(\pi, \pi_\beta).
    %%SL.5.23: maybe label it as "offline RL with a policy constraint" or something?
\end{equation}
$J_{\mathcal{D}}(\pi)$ denotes the average return of policy $\pi$ in the empirical MDP induced by the transitions in the dataset, and $D(\pi, \pi_\beta)$ denotes a divergence measure (e.g., KL-divergence~\citep{jaques2019way,wu2019behavior}, MMD distance~\citep{kumar2019stabilizing} or $D_{\text{CQL}}$~\citep{kumar2020conservative}) between the learned policy $\pi$ and the behavior policy $\pi_\beta$. 
% With the goal of maximizing the actual return of the policy $J(\pi) = \mathbb{E}_{\pi}[\sum_t \gamma^t r(\bs_t, \mathbf{a}_t)]$,
%%SL.5.23: This sentence is ordered backwards, the first clause should come at the end...
% but with access only to a finite dataset, these algorithms can generally be viewed as solving some instantiation of the following policy-optimization problem~\citep{kumar2020conservative}, which provides certain policy improvement guarantees~\citep{laroche2019safe,petrik2016safe,kumar2020conservative}:
% \begin{equation}
% \label{eqn:generic_offline_rl}
%     \pi^*(\mathbf{a}|\bs) := \arg \max_{\pi}~~ J_{\mathcal{D}}(\pi) - \alpha D(\pi, \pi_\beta)~~~~~~~~~~~~~ \text{(Offline RL)}.
%     %%SL.5.23: maybe label it as "offline RL with a policy constraint" or something?
% \end{equation}
% $J_{\mathcal{D}}(\pi)$ denotes the average return of policy $\pi$ in the empirical MDP induced by the transitions in the dataset, and $D(\pi, \pi_\beta)$ denotes a divergence measure (e.g., KL-divergence~\citep{jaques2019way}, MMD distance~\citep{kumar2019stabilizing}) between the learned policy $\pi$ and the behavior policy $\pi_\beta$. 

In the multi-task offline RL setting with data-sharing, the generic optimization problem in Equation~\ref{eqn:generic_offline_rl} for a task $i$ utilizes the effective dataset $\mathcal{D}^\mathrm{eff}_i$. In addition, we define $\pi_\beta^\mathrm{eff}(\mathbf{a}|\bs, i)$ as the effective behavior policy for task $i$ and it is given by: $\pi_\beta^\mathrm{eff}(\mathbf{a}|\bs, i) := |\mathcal{D}^\mathrm{eff}_i(\bs, \mathbf{a})| / |\mathcal{D}^\mathrm{eff}_i(\bs)|$. Hence, the counterpart of Equation~\ref{eqn:generic_offline_rl} in the multi-task offline RL setting with data sharing is given by:
\begin{equation}
\label{eqn:generic_multitask_offline_rl}
     \forall i \in [N], ~~\pi^*(\mathbf{a}|\bs, i) := \arg \max_{\pi}~~ J_{\mathcal{D}^\mathrm{eff}_i}(\pi) - \alpha D(\pi, \pi^\mathrm{eff}_\beta).
     %%CF.5.22: are J_\data(\pi, \task_i) and D(\pi, \pi_\beta; \mathcal{T}_i) ever defined? and should it be J_\mathcal{D} or J_\mathcal{D}(\task) or something else?
\end{equation}
We will utilize this generic optimization problem to motivate our method in Section~\ref{sec:cds_method}.

% Given an MDP $\mdp =(\states, \actions, R, P, \gamma)$ \citep{puterman1994markov}, with state space $\states$, action space $\actions$, a reward function $R(\bs, \mathbf{a})$, dynamics $P(\bs' | \bs, \mathbf{a})$ and a discount factor $\gamma \in [0, 1)$, the goal of  The Q-function $Q^\pi(\bs, \mathbf{a})$ for a policy $\pi(\mathbf{a}|\bs)$ is the expected long-term discounted reward obtained by executing action $\mathbf{a}$ at state $\bs$ and following $\pi(\mathbf{a}|\bs)$ thereafter, 
% $Q^\pi(\bs, \mathbf{a}) := \E\left[ \sum\nolimits_{t=0}^\infty \gamma^t R(\bs_t, \mathbf{a}_t) \right]$.

%%CF.5.18: I think it's pretty important to define the problem statement here. We do make the assumption that only the reward varies across tasks, and that we have access to the reward function. These assumptions aren't entirely standard in MTRL, so it's important to be upfront about them. (It could also make sense to include a mention of it in the intro, so that we are even more upfront about it)