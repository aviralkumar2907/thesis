\documentclass{article}


% if you need to pass options to natbib, use, e.g.:
%     \PassOptionsToPackage{numbers, compress}{natbib}
% before loading neurips_2023


% ready for submission

% from: https://twitter.com/zicokolter/status/1397569757990301702
\usepackage[nonatbib]{neurips_2023}
\usepackage[numbers]{natbib}

% to compile a preprint version, e.g., for submission to arXiv, add add the
% [preprint] option:
%     \usepackage[preprint]{neurips_2023}


% to compile a camera-ready version, add the [final] option, e.g.:
%     \usepackage[final]{neurips_2023}


% to avoid loading the natbib package, add option nonatbib:
%    \usepackage[nonatbib]{neurips_2023}


\usepackage[utf8]{inputenc} % allow utf-8 input
\usepackage[T1]{fontenc}    % use 8-bit T1 fonts
\usepackage{xcolor}
\usepackage[hidelinks]{hyperref}       % hyperlinks
\usepackage{url}            % simple URL typesetting
\definecolor{citecolor}{HTML}{0071bc}
\hypersetup{
    colorlinks,
    linkcolor={citecolor},
    citecolor={citecolor},
    urlcolor={citecolor}
}
\usepackage{booktabs}       % professional-quality tables
\usepackage{amsfonts}       % blackboard math symbols
\usepackage{amsmath}       % blackboard math
\usepackage{nicefrac}       % compact symbols for 1/2, etc.
\usepackage{microtype}      % microtypography
\usepackage{xcolor}         % colors

% New Packages
\usepackage{multirow}
\usepackage{subcaption}
\usepackage{wrapfig}
\usepackage{lipsum}
\usepackage{listings}

\usepackage{amsmath}
\usepackage{amssymb}
\usepackage{mathtools}
\usepackage{amsthm}
\usepackage{bbm}
\usepackage{algorithm}


\usepackage{algpseudocode}
\usepackage{setspace}

\usepackage{color}
\definecolor{deepblue}{rgb}{0,0,0.5}
\definecolor{deepred}{rgb}{0.6,0,0}
\definecolor{deepgreen}{rgb}{0,0.5,0}
% Python style for highlighting
\newcommand\pythonstyle{\lstset{
basicstyle=\ttfamily\scriptsize,
language=Python,
morekeywords={self, clip, exp, mse_loss, uniform_sample, concatenate, logsumexp},              % Add keywords here
keywordstyle=\color{deepblue},
emph={MyClass,__init__},          % Custom highlighting
emphstyle=\color{deepred},    % Custom highlighting style
stringstyle=\color{deepgreen},
frame=single,                         % Any extra options here
showstringspaces=false
}}

% Python environment
\lstnewenvironment{python}[1][]
{
\pythonstyle
\lstset{#1}
}
{}

% Python for external files
\newcommand\pythonexternal[2][]{{
\pythonstyle
\lstinputlisting[#1]{#2}}}

% Python for inline
\newcommand\pythoninline[1]{{\pythonstyle\lstinline!#1!}}



\makeatletter
\def\mathcolor#1#{\@mathcolor{#1}}
\def\@mathcolor#1#2#3{%
  \protect\leavevmode
  \begingroup
    \color#1{#2}#3%
  \endgroup
}
\makeatother

\usepackage[textsize=tiny]{todonotes}
\usepackage{setspace}

\input{macro.tex}


\setlength{\parskip}{3pt}


\title{\methodname: Calibrated Offline RL Pre-Training for Efficient Online Fine-Tuning}

\begin{document}

\maketitle


\begin{abstract}
A compelling use case of offline reinforcement learning (RL) is to obtain a policy initialization from existing datasets followed by fast online fine-tuning with limited interaction. However, existing offline RL methods tend to behave poorly during fine-tuning. In this paper, we study the fine-tuning problem in the context of conservative offline RL methods and we devise an approach for learning an effective initialization from offline data that also enables fast online fine-tuning capabilities. Our approach, calibrated Q-learning (\methodname), accomplishes this by learning a conservative value function initialization that underestimates the value of the learned policy from offline data, while also ensuring that the learned Q-values are at a reasonable scale. We refer to this property as calibration, and define it formally as providing a lower bound on the true value function of the learned policy and an upper bound on the value of some other (suboptimal) reference policy, which may simply be the behavior policy. We show that a conservative offline RL algorithm that also learns a calibrated value function leads to effective online fine-tuning, enabling us to take the benefits of offline initializations in online fine-tuning. In practice, \methodname\ can be implemented on top of the conservative Q learning (CQL)~\cite{kumar2020conservative} for offline RL within a one-line code change. Empirically, \methodname\ outperforms state-of-the-art methods on {\bf 9/11} fine-tuning benchmark tasks that we study in this paper.     

\end{abstract}
\vspace{-0.2cm}
\section{Introduction}
\label{sec:calql_introduction}
\vspace{-0.2cm}

Modern machine learning successes follow a common recipe: pre-training models on general-purpose, Internet-scale data, followed by fine-tuning the pre-trained initialization on a limited amount of data for the task of interest~\cite{he2022masked,devlin2018bert}. How can we translate such a recipe to sequential decision-making problems? A natural way to instantiate this paradigm is to utilize offline RL for initializing value functions and policies from static datasets, followed by online fine-tuning to improve this initialization with limited active interaction. If successful, such a recipe might enable effective online RL with much fewer samples than current methods that learn from scratch.

Many algorithms for offline RL have been applied to online fine-tuning. Empirical results across such works suggest a counter-intuitive trend: policy initializations obtained from more effective offline RL methods tend to exhibit worse online fine-tuning performance, even within the same task (see Table 2 of~\cite{kostrikov2021offline} \& Figure 4 of \cite{xiao2023the}). On the other end, online RL methods training from scratch (or RL from demonstrations~\cite{vecerik2017leveraging},
where the replay buffer is seeded with the offline data) seem to improve online at a significantly faster rate. But these online methods require actively collecting data by rolling out policies from scratch, which inherits similar limitations to na\"ive online RL methods in problems where data collection is expensive or dangerous. Overall, these results suggest that it is challenging to devise an offline RL algorithm that both acquires a good initialization from prior data and also enables efficient fine-tuning.

\begin{wrapfigure}{r}{0.5\columnwidth}
    % \vspace{-1.6cm}
    \vspace{-0.3cm}
    \begin{center}
    {\includegraphics[width=0.85\linewidth]{chapters/cal_ql/figs-sample/Teaser_V2.pdf}}
    \vspace{-0.4cm}
    \caption{
    \footnotesize{We study \textbf{offline RL pre-training followed by online RL fine-tuning}. Some prior offline RL methods tend to exhibit slow performance improvement in this setting (yellow), resulting in worse asymptotic performance. Others suffer from initial performance degradation once online fine-tuning begins (red), resulting in a high cumulative regret. We develop an approach that ``\emph{calibrates}'' the value function to attain a fast improvement with a smaller regret (blue).}}
    \vspace{-0.7cm}
    \label{fig:teaser}
    \end{center}
\end{wrapfigure}
How can we devise a method to learn an effective policy initialization that also improves during fine-tuning? We have shown that one can learn a good offline initialization by optimizing the policy against a \emph{conservative} value function obtained from an offline dataset. But, as we show in Section~\ref{sec:calql_empirical_analysis}, conservatism alone is insufficient for efficient online fine-tuning. Conservative methods often tend to ``unlearn'' the policy initialization learned from offline data and waste samples collected via online interaction in recovering this initialization. We find that the ``unlearning'' phenomenon is a consequence of the fact that value estimates produced via conservative methods can be significantly lower than the ground-truth return of \emph{any} valid policy. Having Q-value estimates that do not lie on a similar scale as the return of a valid policy is problematic. Because once fine-tuning begins, actions executed in the environment for exploration that are actually worse than the policy learned from offline data could erroneously appear better, if their ground-truth return value is larger than the learned conservative value estimate. Hence, subsequent policy optimization will degrade the policy performance until the method recovers.  

% Our approach
If we can ensure that the conservative value estimates learned using the offline data are \emph{calibrated}, meaning that these estimates are on a similar scale as the true return values, then we can avoid the unlearning phenomenon caused by conservative methods (see the formal definition in~\ref{cond:calibration}). Of course, we cannot enforce such a condition perfectly, since it would require eliminating all errors in the value function. Instead, we devise a method for ensuring that the learned values upper bound the true values of some \emph{reference policy} whose values can be estimated more easily (e.g., the behavior policy), while still lower bounding the values of the learned policy. Though this does not perfectly ensure that the learned values are correct, we show that it still leads to sample-efficient online fine-tuning. Thus, our practical method, \textbf{calibrated Q-learning} \textbf{(\methodname)}, learns conservative value functions that are ``calibrated'' against the behavior policy, via a simple modification to existing conservative methods.

The main contribution of this chapter is \methodname, a method for acquiring an offline initialization that facilitates online fine-tuning. \methodname\ aims to learn conservative value functions that are calibrated with respect to a reference policy (e.g., the behavior policy). Our analysis of \methodname\ shows that \methodname\ attains stronger guarantees on cumulative regret during fine-tuning. In practice, \methodname\ can be implemented on top of conservative Q-learning~\cite{kumar2020conservative}, a prior offline RL method, without any additional hyperparameters. We evaluate \methodname\ across a range of benchmark tasks from \cite{fu2020d4rl}, \cite{singh2020cog} and \cite{AWAC}, including robotic manipulation and navigation. We show that \methodname\ matches or outperforms the best methods on all tasks, in some cases by 30-40\%.


\vspace{-0.2cm}
\section{Related Work}
\vspace{-0.3cm}
Several prior works suggest that online RL methods typically require a large number of samples~\cite{silver2016mastering,vinyals2019grandmaster,ye2020towards,kakade2002approximately,zhai2022computational,gupta2022unpacking,li2022understanding} to learn from scratch. We can utilize offline data to accelerate online RL algorithms. Prior works do this in a variety of ways: incorporating the offline data into the replay buffer of online RL~\cite{schaal1996learning,vecerik2017leveraging,hester2018deep,song2023hybrid}, utilizing auxiliary behavioral cloning losses with policy gradients~\cite{rajeswaran2017learning,kang2018policy,zhu2018reinforcement,zhu2019dexterous}, or extracting a high-level skill space for downstream online RL~\cite{gupta2019relay,ajay2020opal}. While these methods improve the sample efficiency of online RL from scratch, as we will also show in our results, they do not eliminate the need to actively roll out poor policies for data collection.

To address this issue, a different line of work first runs offline RL for learning a good policy and value initialization from the offline data, followed by online fine-tuning~\cite{nair2020accelerating,kostrikov2021offlineb,lyu2022mildly,beeson2022improving,wu2022supported,lee2022offline,mark2022fine}. These approaches typically employ offline RL methods based on policy constraints or pessimism~\cite{fujimoto2018off,siegel2020keep,guo2020batch,ghasemipour2021emaq,kostrikov2021offlineb,singh2020cog,lee2022offline} on the offline data, then continue training with the same method on a combination of offline and online data once fine-tuning begins~\cite{nachum2019algaedice,kidambi2020morel,yu2020mopo,kumar2020conservative,buckman2020importance}. Although pessimism is crucial for offline RL~\cite{jin2021pessimism,cheng2022adversarially}, using pessimism or constraints for fine-tuning~\cite{nair2020accelerating,kostrikov2021offlineb,lyu2022mildly} slows down fine-tuning or leads to initial unlearning, as we will show in Section~\ref{sec:empirical_analysis}. In effect, these prior methods either fail to improve as fast as online RL or lose the initialization from offline RL. We aim to address this limitation by understanding some conditions on the offline initialization that enable fast fine-tuning. 

Our approach is most related to methods that utilize a pessimistic offline RL algorithm for offline training but incorporate exploration in fine-tuning~\cite{lee2022offline,mark2022fine,wu2022supported}. In contrast to these works, our method aims to learn a better offline initialization that enables standard online fine-tuning. Our approach fine-tunes na\"ively without ensembles~\cite{lee2022offline} or exploration~\cite{mark2022fine} and, as we show in our experiments, this alone is enough to outperform approaches that employ explicit optimism during data collection.
\vspace{-0.2cm}
\section{Problem Statement and Additional Notation}
\vspace{-0.15cm}
% The goal in RL is to learn the optimal policy for an MDP $\mathcal{M} = (\mathcal{S}, \mathcal{A}, P, r, \rho, \gamma)$. $\mathcal{S}, \mathcal{A}$ denote the state and action spaces.  $P(s' | s, a)$ and $r(s,a)$
% are the dynamics and reward functions. $\rho(s)$ denotes the initial state distribution.  $\gamma \in (0,1)$ denotes the discount factor. Formally, the goal is to learn a policy $\pi:\mc S\mapsto \mc A$ that maximizes cumulative discounted value function, denoted by $V^\pi(s) = {\frac{1}{1-\gamma}\sum_{t} \bb E_{a_t \sim \pi(s_t)}\brac{\gamma^t r(s_t, a_t)|s_0=s}}$. The Q-function of a given policy $\pi$ is defined as ${Q^\pi(s,a) = {\frac{1}{1-\gamma}\sum_{t} \bb E_{a_t \sim \pi(s_t)}\brac{\gamma^t r(s_t, a_t)|s_0=s,a_0=a}}}$, and we use $Q_\theta^\pi$ to denote the estimate of the Q-function of a policy $\pi$ as obtained via a neural network with parameters $\theta$.

Given access to an offline dataset $\mathcal{D} = \{(s, a, r, s^\prime)\}$ collected using a behavior policy $\behavior$, we aim to first train a good policy and value function using the offline dataset $\mathcal{D}$ alone, followed by an online phase that utilizes online interaction in $\mathcal{M}$. Our goal during fine-tuning is to obtain the optimal policy with the smallest number of online samples. This can be expressed as minimizing the \textbf{cumulative regret} over rounds of online interaction: 
\begin{align}
\regret(K) := \bb E_{s\sim \rho}\sum_{k=1}^K\brac{V^{\star}(s) - V^{\pi^k}(s)}. 
\end{align}
% As we demonstrate in Section~\ref{sec:experiments}, existing methods face challenges in this setting.

% Our approach will build on the conservative Q-learning (CQL)~\cite{kumar2020conservative} algorithm.
% CQL imposes an additional regularizer that penalizes the learned Q-function on out-of-distribution (OOD) actions while compensating for this pessimism on actions seen within the training dataset. Assuming that the value function is represented by a function, $Q_\theta$, the training objective of CQL is given by
% \begin{align}
%     \label{eqn:cql_training}
%     \!\!\!\!\!\min_\theta {\color[HTML]{6C8EBF} {\alpha \underbrace{\left(\mathbb{E}_{s \sim \mathcal{D}, a \sim \pi} \left[Q_\theta(s,a)\right] - \mathbb{E}_{s, a \sim \mathcal{D}}\left[Q_\theta(s,a)\right]\right)}_{\text{Conservative regularizer }\mathcal{R}(\theta)}}} + \frac{1}{2} {\mathbb{E}_{s, a, s^\prime\sim \mathcal{D}}\left[\left(Q_\theta(s, a) - \bellman^\pi\bar{Q}(s, a)\right)^2 \right]},
% \end{align}
% where $\bellman^\pi \bar{Q} (s, a)$ is the backup operator applied to a delayed target Q-network, $\bar{Q}$: $\bellman^\policy \bar{Q}(s, a) := r(s, a) + \gamma \E_{a^\prime \sim \pi(a^\prime|s^\prime)}[\bar{Q}(s^\prime, a^\prime)]$. The second term is the standard TD error~\cite{lillicrap2015continuous,fujimoto2018addressing,haarnoja2018sacapps}. The first term  $\mathcal{R}(\theta)$ (in {\color[HTML]{6C8EBF} blue}) is a conservative regularizer that aims to prevent overestimation in the Q-values for OOD actions by minimizing the Q-values under the policy $\pi(\ba|\bs)$, and counterbalances by maximizing the Q-values of the actions in the dataset following the behavior policy $\pi_\beta$.

% \vspace{-0.2cm}
\vspace{-0.4cm}
\section{When Can Offline RL Initializations Enable Fast Online Fine-Tuning?}
\label{sec:empirical}
\vspace{-0.2cm}

A starting point for offline pre-training and online fine-tuning is to simply initialize the value function with one that is produced by an existing offline RL method and then perform fine-tuning. However, we empirically find that initializations learned by many offline RL algorithms can perform poorly during fine-tuning. We will study the reasons for this poor performance for the subset of conservative methods to motivate and develop our approach for online fine-tuning, calibrated Q-learning. 

\vspace{-0.2cm}
\subsection{Empirical Analysis}
\label{sec:calql_empirical_analysis}
\vspace{-0.2cm}

\begin{wrapfigure}{r}{0.38\columnwidth}
\vspace{-1.2cm}
\begin{center}
{\includegraphics[width=0.75\linewidth]{chapters/cal_ql/figs-sample/sec41-final-utd5.pdf}}
\vspace{-0.3cm}
\caption{\label{fig:cql_iql_finetune}\footnotesize{\textbf{Multiple prior offline RL algorithms suffer from difficulties} during fine-tuning including poor asymptotic performance and initial unlearning.}}
\vspace{-0.5cm}
\end{center}
\end{wrapfigure}
Offline RL followed by online fine-tuning typically poses non-trivial challenges for a variety of methods. While analysis in prior work~\citep{nair2020accelerating} notes challenges for a subset of offline RL methods, in Figure~\ref{fig:cql_iql_finetune}, we evaluate the fine-tuning performance of a variety of prior offline RL methods (CQL~\citep{kumar2020conservative}, IQL~\citep{kostrikov2021offlineb}, TD3+BC~\citep{fujimoto2021minimalist}, AWAC~\citep{nair2020accelerating}) on a particular diagnostic instance of a visual pick-and-place task with a distractor object and sparse binary rewards~\citep{singh2020cog}, and find that all methods struggle to attain the best possible performance, quickly. More details about this task are in Appendix~\ref{appendix:env_details}. 

% The analysis by \citet{nair2020accelerating} highlights the limitations of explicit policy constraint methods for fine-tuning. Therefore, in this section, we study a representative \emph{implicit} policy constraint method, implicit Q-learning (IQL)~\cite{kostrikov2021offlineb} that attains good performance on benchmark tasks, and a conservative method, CQL~\cite{kumar2020conservative}.
%%SL.5.7: I really think we should just remove the IQL discussion here. We don't actually "study" IQL in any meaningful way here, and the reference to Nair et al. is pretty tortured too. We could just say we'll use CQL as our starting point, though the issues we observe have also been noted with other RL methods (and we can reference Nair for that, since the AWAC paper also observes the "dip").
%%SZ.5.9: agree
% We study the task of fine-tuning a robot policy on a visual pick-and-place task with a distractor object and sparse binary rewards, from prior work~\cite{singh2020cog}. 
% More details about the offline dataset are in Appendix~\ref{appendix:env_details}.  

% We present the learning curves for both methods in online fine-tuning in Figure~\ref{fig:cql_iql_finetune}. 
While the offline Q-function initialization obtained from all methods attains a similar (normalized) return of around 0.5, they suffer from difficulties during fine-tuning: TD3+BC, IQL, AWAC attain slow asymptotic performance and CQL unlearns the offline initialization, followed by spending a large amount of online interaction to recover the offline performance again, before any further improvement. This initial unlearning appears in multiple tasks as we show in Appendix~\ref{app:cql_dip_zoom_in}. In this work, we focus on developing effective fine-tuning strategies on top of conservative value estimation methods like CQL. To do so, we next study the potential reason behind the initial unlearning in CQL.  

\begin{wrapfigure}{r}{0.6\linewidth}
\vspace{-0.5cm}
\begin{center}
\centerline{\includegraphics[width=0.99\linewidth]{chapters/cal_ql/figs-sample/CQL-q-values.pdf}}
\vspace{-0.25cm}
\caption{\label{fig:cql_q_value} \footnotesize{\textbf{The evolution of the average Q-value and the success rate of CQL over the course of offline pre-training and online fine-tuning.} Fine-tuning begins at 50K steps. The red-colored part denotes the period of performance recovery which also coincides with the period of Q-value adjustment.}}
\end{center}
\vspace{-0.9cm}
\end{wrapfigure}
\textbf{Why does CQL unlearn initially?} To understand why CQL unlearns initially, we inspect the learned Q-values averaged over the dataset in Figure~\ref{fig:cql_q_value}. Observe that the Q-values learned by CQL in the offline phase are \emph{much} smaller than their ground-truth value (as expected), but these Q-values drastically jump and adjust in scale when fine-tuning begins. In fact, we observe that performance recovery (red segment in Figure~\ref{fig:cql_q_value}) {\em coincides} with a period where the range of Q-values changes to match the true range. This is as expected: as a conservative Q-function experiences new online data, actions much worse than the offline policy on the rollout states appear to attain higher rewards compared to the highly underestimated offline Q-function, which in turn deceives the policy optimizer into unlearning the initial policy. We illustrate this idea visually in Figure~\ref{fig:calql_idea}. Once the Q-function has adjusted and the range of Q-values closely matches the true range, then fine-tuning can proceed normally, after the dip. 

\textbf{To summarize,} our empirical analysis indicates that methods existing fine-tuning methods suffer from difficulties such as initial unlearning or poor asymptotic performance. In particular, we observed that conservative methods can attain good asymptotic performance, but ``waste'' samples to correct the learned Q-function. Thus, we attempt to develop a fine-tuning method that builds on top of an existing conservative method, CQL. 
% but aims to ``calibrate'' the Q-function so that the initial dip in performance can be avoided. 

\vspace{-0.2cm}
\subsection{Conditions on the Offline Initialization that Enable Fast Fine-Tuning}
\vspace{-0.2cm}
Our observations from the preceding discussion motivate two conclusions in regard to the offline Q-initialization for fast fine-tuning: \textbf{(a)} methods that learn \textbf{conservative} Q-functions can attain good asymptotic performance, and \textbf{(b)} if the learned Q-values closely match the range of ground-truth Q-values on the task, then online fine-tuning does not need to devote samples to unlearn and then recover the offline initialization. One approach to formalize this intuition of Q-values lying on a similar scale as the ground-truth Q-function is via the requirement that the conservative Q-values learned by the conservative method must be lower-bounded by the Q-value of a sub-optimal reference policy. This will prevent conservatism from learning overly small Q-values. We will refer to this property as ``calibration'' with respect to the reference policy.
\begin{tcolorbox}[colback=blue!6!white,colframe=black,boxsep=0pt,top=-3pt,bottom=2pt]
\vspace{2mm}
\begin{definition}[Calibration]
\label{cond:calibration}
An estimated Q-function ${Q}_\theta^\pi$ for a given policy $\pi$ is said to be calibrated with respect to a reference policy $\mu$ if:
\begin{align} 
    \mathbb{E}_{\mathbf{a} \sim \pi}\left[Q_\theta^\pi(\bs, \mathbf{a})\right] \geq \mathbb{E}_{\mathbf{a} \sim \mu}\left[Q^\mu(\bs, \mathbf{a})\right] := V^\mu(s), \forall s \in \mc \mathcal{D}.
\end{align}
\end{definition}
\end{tcolorbox}

If the learned Q-function ${Q}^\pi_\theta$ is calibrated with respect to a policy $\mu$ that is worse than $\pi$, it would prevent unlearning during fine-tuning that we observed in the case of CQL.
This is because the policy optimizer would not unlearn $\pi$ in favor of a policy that is worse than the reference policy $\mu$ upon observing new online data as the expected value of $\pi$ is constrained to be larger than $V^\mu$: $\mathbb{E}_{\mathbf{a} \sim \pi}\left[{Q}^\pi_\theta(\bs, \mathbf{a})\right] \geq V^\mu(\bs)$.
Our practical approach \methodname\ will enforce calibration with respect to a policy $\mu$ whose ground-truth value, $V^\mu(\bs)$, can be estimated reliably without bootstrapping error (e.g., the behavior policy induced by the dataset). This is the key idea behind our method (as we will discuss next) and is visually illustrated in Figure~\ref{fig:calql_idea}.

\vspace{-0.2cm}
\vspace{-0.2cm}
\section{\methodname: Calibrated Q-Learning}
\label{sec:empirical-method}
\vspace{-0.25cm}
Our approach, calibrated Q-learning (\methodname) aims to learn a conservative and calibrated value function initializations from an offline dataset. To this end, \methodname\ builds on CQL from Chapter~\ref{chapter:cql} and then constrains the learned Q-function to produce Q-values larger than the Q-value of a reference policy $\mu$ per Definition~\ref{cond:calibration}. In principle, our approach can utilize many different choices of reference policies, but for developing a practical method, we simply utilize the behavior policy as our reference policy.  

\niparagraph{\textbf{Calibrating CQL.}} We can constrain the learned Q-function $Q^\pi_\theta$ to be larger than $V^\mu$ via a simple change to the CQL training objective from Chapter~\ref{chapter:cql}: masking out the push down of the learned Q-value on out-of-distribution (OOD) actions in CQL if the Q-function is not calibrated, i.e., if $\mathbb{E}_{a \sim \pi}\left[Q^\pi_\theta(\bs, \mathbf{a})\right] \leq V^\mu(\bs)$. \methodname\ modifies the CQL regularizer, $\mathcal{R}(\theta)$ in this manner: 
\begin{align}
\label{eqn:cal_ql_training}
\!\!\!\!\!\!\mathbb{E}_{\bs \sim \mathcal{D}, \mathbf{a} \sim \pi} \brac{{\color[HTML]{B85450}{\max \left( Q_\theta(s,a), V^\mu(s) \right)}} } - \mathbb{E}_{\bs, \mathbf{a} \sim \mathcal{D}}\left[Q_\theta(\bs,\mathbf{a})\right],
\end{align}
where the changes from standard CQL are depicted in {\color[HTML]{B85450} red}. As long as $\alpha$ in CQL is large, for any state-action pair where the learned Q-value is smaller than $Q^\mu$, the Q-function in Equation~\ref{eqn:cal_ql_training} will upper bound $Q^\mu$ in a tabular setting. Of course, as with any practical RL method, with function approximators and gradient-based optimizers, we cannot guarantee that we can enforce this condition for every state-action pair, but in our experiments, we find that Equation~\ref{eqn:cal_ql_training} is sufficient to enforce the calibration in expectation over the states in the dataset.         

\begin{wrapfigure}{r}{0.65\linewidth}
\centering
\vspace{-0.2cm}
\includegraphics[trim={0 0 2.7cm 0},clip,width=0.98\linewidth]{chapters/cal_ql/figs-sample/figure_for_calql_final.pdf}
\vspace{-0.2cm}
\caption{
\footnotesize{\textbf{Intuition behind policy unlearning with CQL and the idea behind \methodname.} The plot visualizes a slice of the learned Q-function and the ground-truth values for a given state. Erroneous peaks on suboptimal actions (x-axis) arise when updating CQL Q-functions with online data. This in turn can lead the policy to deviate away from high-reward actions covered by the dataset in favor of erroneous new actions, resulting in deterioration of the pre-trained policy. In contrast, \methodname\ corrects the scale of the learned Q-values by using a reference value function, such that actions with worse Q-values than the reference value function do not erroneously appear optimal in fine-tuning.}}
\label{fig:calql_idea}
\vspace{-0.4cm}
\end{wrapfigure}

\niparagraph{\textbf{Pseudo-code and implementation details.}} Our implementation of \methodname\ directly builds on the implementation of CQL from \citet{geng2022jaxcql}. We present a pseudo-code for \methodname\ in Algorithm~\ref{alg:practical_alg}. Additionally, we list the hyperparameters $\alpha$ for the CQL algorithm and our baselines for each suite of tasks in Appendix \ref{app:calql_hyperparam}. Following the protocol in prior work~\citep{kostrikov2021iql,song2023hybrid}, the practical implementation of \methodname\ trains on a mixture of the offline data and the new online data, weighted in some proportion during fine-tuning. To get $V^\mu(\bs)$, we can fit a function approximator $Q^\mu_\theta$ or $V^\mu_\theta$ to the return-to-go values via regression, but we observed that also simply utilizing the return-to-go estimates for tasks that end in a terminal was sufficient for our use case. We show in  Section~\ref{sec:calql_experiments}, how this simple change to the objective drastically improves over prior fine-tuning results.

\vspace{-0.2cm}
\vspace{-0.15cm}
\section{Theoretical Analysis of \methodname}
\label{sec:theory}
\vspace{-0.25cm}

We will now analyze the cumulative regret attained over online fine-tuning, when the value function is pre-trained with Cal-QL, and show that enforcing calibration (Defintion~\ref{cond:calibration}) leads to a favorable regret bound during the online phase. Our analysis utilizes tools from \citet{song2023hybrid}, but studies the impact of calibration on fine-tuning. We also remark that we simplify the treatment of certain aspects (e.g., how to incorporate pessimism) as it allows us to cleanly demonstrate benefits of calibration.  

{\textbf{Notation \& terminology.}} In our analysis, we will consider an idealized version of \methodname\ for simplicity. Specifically, following prior work~\citep{song2023hybrid} under the bilinear model~\citep{du2021bilinear}, we will operate in a finite-horizon setting with a horizon $H$. We denote the learned Q-function at each learning iteration $k$ for a given $(\bs, \mathbf{a})$ pair and time-step $h$ by $Q_{\theta}^k(\bs, \mathbf{a})$. For any given policy $\pi$, let $C_\pi\geq1$ denote the concentrability coefficient such that $C_\pi:=\max_{f\in \mc C}\frac{\sum_{h=0}^{H-1}\bb E_{s,a\sim d_h^\pi}[\mc T f_{h+1}(s,a)-f_h(s,a)]}{\sqrt{\sum_{h=0}^{H-1}\bb E_{s,a\sim \nu_h}(\mc T f_{h+1}(s,a)-f_h(s,a))^2}}$,
i.e., a coefficient that quantifies the distribution shift between the policy $\pi$ and the dataset $\mathcal{D}$, in terms of the ratio of Bellman errors averaged under $\pi$ and the dataset $\mathcal{D}$. Note that $\mc C$ represents the Q-function class and we assume $\mc C$ has a bellman-bilinear rank~\citep{du2021bilinear} of $d$. We also use $C_\pi^\mu$ to denote the concentrability coefficient over a subset of {\em calibrated} Q-functions w.r.t. a reference policy $\mu$: $C^\mu_\pi:=\max_{f\in \mc C,f(s,a)\geq Q^\mu(s,a)}\frac{\sum_{h=0}^{H-1}\bb E_{s,a\sim d_h^\pi}[\mc T f_{h+1}(s,a)-f_h(s,a)]}{\sqrt{\sum_{h=0}^{H-1}\bb E_{s,a\sim \nu_h}(\mc T f_{h+1}(s,a)-f_h(s,a))^2}}$, which provides $C^\mu_\pi\leq C_\pi$. Similar to $\mc C$, let $d_\mu$ denote the bellman bilinear rank of $\mc C_\mu$ -- 
the calibrated Q-function class w.r.t. the reference policy $\mu$.
Intuitively, we have $\mc C_\mu\subset\mc C$, which implies that $d_\mu\leq d$. The formal definitions are provided in Appendix~\ref{appendix:notations}.
We will use $\pi^k$ to denote the arg-max policy induced by $Q^k_\theta$. 

% \vspace{-0.4cm}
% \subsection{Intuition} 
% \vspace{-0.4cm}

{\textbf{Intuition.}} We intuitively discuss how calibration and conservatism enable \methodname\ to attain a smaller regret compared to not imposing calibration. Our goal is to bound the cumulative regret of online fine-tuning, ${\sum_{k} \mathbb{E}_{\bs_0 \sim \rho}[V^{\pi^\star}(\bs_0) - V^{\pi^k}(\bs_0)]}$. We can decompose this expression into two terms: 
\vspace{-0.05cm}
\begin{align}
    \resizebox{.87\textwidth}{!}{$\regret(K) = \underbrace{\sum_{k=1}^K \mathbb{E}_{\bs_0 \sim \rho} \brac{V^{\star}(\bs_0) -\max_a {Q}_{\theta}^k(\bs_0,\mathbf{a})}}_{(i) ~:=~ \text{miscalibration}}
    +\underbrace{\sum_{k=1}^K \mathbb{E}_{\bs_0 \sim \rho} \brac{ \max_a {Q}_{\theta}^k(\bs_0,\mathbf{a}) - V^{\pi^{k}}(\bs_0)}}_{(ii) ~:=~ \text{overestimation}}$.}
\label{eq:regret-decomposition}
\end{align}
This decomposition of regret into terms (i) and (ii) is instructive. Term (ii) corresponds to the amount of over-estimation in the learned value function, which is expected to be small if a conservative RL algorithm is used for training. Term (i) is the difference between the ground-truth value of the optimal policy and the learned Q-function and is negative if the learned Q-function were calibrated against the optimal policy (per Definition~\ref{cond:calibration}). Of course, this is not always possible because we do not know $V^\star$ a priori. But note that when \methodname\ utilizes a reference policy $\mu$ with a high value $V^\mu$, close to $V^\star$, then the learned Q-function $Q_\theta$ is calibrated with respect to $Q^\mu$ per Condition~\ref{cond:calibration} and term (i) can still be controlled. Therefore, controlling this regret requires striking a balance between learning a calibrated (term (i)) and conservative (term (ii)) Q-function. We now formalize this intuition and defer the detailed proof to Appendix~\ref{appdendix:derivation-CalQL}. 

% \vspace{-0.2cm}
% \subsection{Theorem Statement}
% \vspace{-0.15cm}

\begin{theorem}[Informal regret bound of \methodname]
\label{thm:main-thm-informal}
    With high probability, \methodname{} obtains the following bound on total regret accumulated during online fine-tuning: \vspace{-0.15cm}
\begin{equation*}
    \begin{split}
        \regret(K) = \wt{O}\Big(\min\big\{C_{\pi^\star}^\mu H\sqrt{dK\log\paren{|\mc F|}},
        \;K\mathbb{E}_{\rho}[V^{\star}(\bs_0) -V^\mu(\bs_0)]+H\sqrt{d_\mu K\log\paren{|\mc F|}}\big\}\Big),
    \end{split}
\end{equation*}
where $\mc F$ is the functional class of the Q-function.
\end{theorem}

{\textbf{Comparison to~\citet{song2023hybrid}.}}
\citet{song2023hybrid} analyzes an online RL algorithm that utilizes offline data without imposing conservatism or calibration. We now compare Theorem~\ref{thm:main-thm-informal} to Theorem 1 of \citet{song2023hybrid} to understand the impact of these conditions on the final regret guarantee. Theorem 1 of \citet{song2023hybrid} presents a regret bound: $\regret(K) = \wt{O}\left( C_{\pi^\star} H \sqrt{d K \log\paren{|\mc F|}} \right)$ and we note some improvements in our guarantee, that we also verify via experiments in Section~\ref{subsec:diagonistic}: \textbf{(a)} for the setting where the reference policy $\mu$ contains near-optimal behavior, i.e., $V^\star - V^\mu \lesssim O(H\sqrt{d\log\paren{|\mc F|}/K})$, \methodname\ can enable a tighter regret guarantee compared to \citet{song2023hybrid}; \textbf{(b)} as we show in Appendix~\ref{subsec:CalQL-assumptions}, the concentrability coefficient $C^\mu_{\pi^\star}$ appearing in our guarantee is no larger than the one that appears in Theorem 1 of \citet{song2023hybrid}, providing another source of improvement; and \textbf{(c)} finally, in the case where the reference policy has broad coverage \emph{and} is highly sub-optimal, \methodname\ reverts back to the guarantee from \citep{song2023hybrid}, meaning that \methodname\ improves upon this prior work.


\vspace{-0.2cm}
\vspace{-0.09cm}
\section{Experimental Evaluation}
\vspace{-0.22cm}
\label{sec:experiments}
The goal of our experimental evaluation is to study how well \methodname\ can facilitate sample-efficient online fine-tuning. To this end, we compare \methodname\ with several other state-of-the-art fine-tuning methods on a variety of offline RL benchmark tasks from D4RL~\cite{fu2020d4rl}, \citet{singh2020cog}, and \citet{nair2020accelerating}, evaluating performance before and after fine-tuning. We also study the effectiveness of \methodname\ on higher-dimensional tasks, where the policy and value function must process raw image observations. Finally, we perform empirical studies to understand the efficacy of \methodname\ with different dataset compositions and the impact of errors in the reference value function estimation.
% to understand the efficacy of \methodname\ with different dataset compositions 
% Finally, we perform several empirical studies to understand the efficacy of \methodname\ with different dataset compositions and to understand the impact of errors in reference function value estimation on \methodname.  

\begin{wrapfigure}{r}{0.5\linewidth}
\centering
\vspace{-0.45cm}
\includegraphics[width=0.95\linewidth]{chapters/cal_ql/figs-sample/envs_final.pdf}
\vspace{-0.25cm}
\caption{
\footnotesize{\textbf{Tasks:} We evaluate \methodname\ on a diverse set of benchmark problems: \texttt{AntMaze} and \texttt{Frankakitchen} domains from \cite{fu2020d4rl}, \texttt{Adroit} tasks from \cite{nair2020accelerating} and a vision-based robotic manipulation task from \cite{kumar2022pre}.}}
\label{fig:envs}
\vspace{-0.2cm}
\end{wrapfigure}
%%AK.5.6: Do we need this figure to be in the main paper given that we are tight on space?

\textbf{Offline RL tasks and datasets.} We evaluate \methodname\ on a number of benchmark tasks and datasets used by prior works~\cite{kostrikov2021offlineb,nair2020accelerating} to evaluate fine-tuning performance: \textbf{(1)} The {\texttt{AntMaze}} tasks from D4RL~\cite{fu2020d4rl} that require controlling an ant quadruped robot to navigate from a starting point to a desired goal location in a maze. The reward is +1 if the agent reaches within a pre-specified small radius around the goal and 0 otherwise. 
% We consider two kinds of maze layouts (medium and large mazes from \cite{fu2020d4rl}) and two data compositions: \textbf{play} and \textbf{diverse} that vary in coverage of actions at different regions of the state space and sub-optimality of the behavior policy. 
\textbf{(2)} The \texttt{FrankaKitchen} tasks from D4RL require controlling a 9-DoF Franka robot to attain a desired configuration of a kitchen. To succeed, a policy 
must complete four sub-tasks in the kitchen within a single rollout, and it receives a binary reward of +1/0 for every sub-task it completes. \textbf{(3)} Three \texttt{Adroit} dexterous manipulation tasks~\cite{rajeswaran2018dapg,kostrikov2021offlineb,nair2020accelerating} that require learning complex manipulation skills on a 28-DoF five-fingered hand to \textbf{(a)} manipulate a pen in-hand to a desired configuration (\texttt{pen-binary}), \textbf{(b)} open a door by unlatching the handle (\texttt{door-binary}), and \textbf{(c)} relocating a ball to a desired location (\texttt{relocate-binary}). The agent obtains a sparse binary +1/0 reward if it succeeds in solving the task. Each of these tasks only provides a narrow offline dataset consisting of 25 demonstrations collected via human teleoperation and additional trajectories collected by a BC policy.
Finally, to evaluate the efficacy of \methodname\ on a task where we learn from raw visual observations, we study \textbf{(4)} a pick-and-place task from prior work~\cite{singh2020cog,kumar2022pre} that requires learning to pick a ball and place it in a bowl, in the presence of distractors.


\iffalse

\textbf{Offline RL tasks and datasets.} We evaluate \methodname\ on a number of benchmark tasks and datasets used by prior works~\cite{kostrikov2021offlineb,nair2020accelerating} to evaluate fine-tuning performance: \textbf{(1)} The {\texttt{AntMaze}} tasks from D4RL~\cite{fu2020d4rl} that require controlling an 8-DoF ant quadruped robot to navigate from a starting point to a desired goal location in a maze. The reward is +1 if the agent reaches within a pre-specified small radius around the goal and 0 otherwise. We consider two kinds of maze layouts (medium and large mazes from \cite{fu2020d4rl}) and two data compositions: \textbf{play} and \textbf{diverse} that vary in coverage of actions at different regions of the state space and sub-optimality of the behavior policy. \textbf{(2)} The \texttt{FrankaKitchen} tasks from D4RL require controlling a 9-DoF Franka robot to attain a desired configuration of a kitchen. To succeed, a policy 
must complete four sub-tasks in the kitchen within a single rollout, and it receives a binary reward of +1/0 for every sub-task it completes. \textbf{(3)} Three \texttt{Adroit} dexterous manipulation tasks~\cite{rajeswaran2018dapg,kostrikov2021offlineb,nair2020accelerating} that require learning complex manipulation skills on a 28-DoF five-fingered hand to \textbf{(a)} manipulate a pen in-hand to a desired configuration (\texttt{pen-binary}), \textbf{(b)} open a door by unlatching the handle (\texttt{door-binary}), and \textbf{(c)} relocating a ball to a desired location (\texttt{relocate-binary}). An agent obtains a sparse binary +1/0 reward if it succeeds in solving the task. Each of these tasks only provides an extremely narrow offline dataset consisting of 25 demonstrations collected via human teleoperation and additional trajectories collected by a BC policy.
Finally, to evaluate the efficacy of \methodname\ on more challenging tasks where we must learn from raw visual observations, we study \textbf{(4)} a pick-and-place task from prior work~\cite{singh2020cog,kumar2022pre} that requires learning to pick a ball and place it in a bowl, in the presence of distractors.

\fi

\begin{figure}[t]
\begin{center}    
{\includegraphics[clip,width=1\linewidth]{chapters/cal_ql/figs-sample/antmaze-final-caption.pdf}} 
{\includegraphics[clip,width=1\linewidth]{chapters/cal_ql/figs-sample/antmaze-final-plots-only.pdf}}\\{\includegraphics[clip,width=1\linewidth]{chapters/cal_ql/figs-sample/kitchen-cog-final.pdf}} {\includegraphics[clip,width=0.75\linewidth]{chapters/cal_ql/figs-sample/adroit-final.pdf}}
\end{center}
\vspace{-0.45cm}
\caption{\label{fig:all_tasks} \footnotesize{\textbf{Online fine-tuning after offline initialization on the benchmark tasks}. The plots show the online fine-tuning phase \emph{after} pre-training for each method (except SAC-based approaches which are not pre-trained). Observe that \methodname\ consistently matches or exceeds the speed and final performance of the best prior method and is the only algorithm to do so across all tasks. (6 seeds)}}
\vspace{-0.6cm}
\end{figure}

\noindent \textbf{Comparisons, prior methods, and evaluation protocol.} We compare \methodname\ to running online SAC~\cite{haarnoja2018soft} from scratch, as well as prior approaches that leverage offline data. This includes na\"ively fine-tuning offline RL methods such as CQL~\cite{kumar2020conservative} and IQL~\cite{kostrikov2021offlineb}, as well as fine-tuning with AWAC~\citep{nair2020accelerating}, O3F~\cite{mark2022fine} and online decision transformer (ODT)~\citep{zheng2022online}, methods specifically designed for offline RL followed by online fine-tuning. In addition, we also compare to a baseline that trains SAC~\cite{haarnoja2018soft} using both online data and offline data (denoted by ``SAC + offline data'') that mimics DDPGfD~\citep{vecerik2017leveraging} but utilizes SAC instead of DDPG. We also compare to Hybrid RL~\citep{song2023hybrid}, a recently proposed method that improves the sample efficiency of the ``SAC + offline data'' approach, and ``CQL+SAC'', which first pre-train with CQL and then run fine-tuning with SAC on a mixture of offline and online data without conservatism. More details of each method can be found in Appendix~\ref{app:hyperparam}.
We present learning curves for online fine-tuning and also quantitatively evaluate each method on its ability to improve the initialization learned from offline data measured in terms of \textbf{(i)} final performance after a pre-defined number of steps per domain and \textbf{(ii)} the cumulative regret over the course of online fine-tuning. In Section~\ref{subsec:highutd}, we run \methodname\ with a higher update-to-data (UTD) ratio and compare it to RLPD~\cite{rlpd}, a more sample-efficient version of ``SAC + offline data''.

\vspace{-0.3cm}
\subsection{Empirical Results} 
\vspace{-0.2cm}

We first present a comparison of \methodname\ in terms of the normalized performance before and after fine-tuning in Table~\ref{tab:performance} and the cumulative regret in a fixed number of online steps in Table~\ref{tab:results_regret}. Following the protocol of \cite{fu2020d4rl}, we normalize the average return values for each domain with respect to the highest possible return (+4 in FrankaKitchen; +1 in other tasks; see Appendix~\ref{appendix:normalized_score} for more details).  

\textbf{\methodname\ improves the offline initialization significantly.} Observe in Table~\ref{tab:performance} and Figure~\ref{fig:all_tasks} that while the performance of offline initialization acquired by \methodname\ is comparable to that of other methods such as CQL and IQL, \methodname\ is able to improve over its offline initialization the most by \textbf{106.9\%} in aggregate and achieve the best fine-tuned performance in \textbf{9 out of 11} tasks.

\textbf{\methodname\ enables fast fine-tuning.} 
% To understand the efficacy of \methodname\ in enabling learning quickly during online fine-tuning, we measure the cumulative regret accumulated over the course of fine-tuning. 
Observe in Table~\ref{tab:results_regret} that \methodname\ achieves the smallest regret on \textbf{8 out of 11} tasks, attaining an average regret of 0.22 which improves over the next best method (IQL) by \textbf{42\%}. Intuitively, this means that \methodname\ does not require running highly sub-optimal policies. In tasks such as \texttt{relocate-binary}, \methodname\ enjoys the fast online learning benefits associated with na\"ive online RL methods that incorporate the offline data in the replay buffer (SAC + offline data and \methodname\ are the only two methods to  attain a score of $\geq$ 90\% on this task) unlike prior offline RL methods. As shown in Figure~\ref{fig:all_tasks}, in the \texttt{kitchen} and \texttt{antmaze} domains, \methodname\ brings the benefits of fast online learning together with a good offline initialization, improving drastically on the regret metric. Finally, observe that the initial unlearning at the beginning of fine-tuning with conservative methods observed in Section~\ref{sec:empirical_analysis} is greatly alleviated in all tasks (see Appendix~\ref{app:cql_dip_zoom_in} for details).


\begin{table*}[h]
\scriptsize{
\begin{center}

\vspace{-0.1cm}
\!\!\!\!\scalebox{0.89}{
\begin{tabular}{l|c|c|c|c|c|c|c|c|c||c}
Task & CQL & IQL & AWAC & O3F & ODT & CQL+SAC & Hybrid SRL & SAC+od & SAC & Cal-QL (Ours) \\ \hline \hline
\texttt{large-diverse}  & 25 $\rightarrow$ 87 & 40 $\rightarrow$ 59 & 00 $\rightarrow$ 00 & 59 $\rightarrow$ 28 & 00 $\rightarrow$ 01 & 36 $\rightarrow$ 00 &  $\rightarrow$ 00 &  $\rightarrow$ 00 &  $\rightarrow$ 00 & 33 $\rightarrow$  \textbf{95} \\
\texttt{large-play}  & 34 $\rightarrow$ 76 & 41 $\rightarrow$ 51 & 00 $\rightarrow$ 00 & 68 $\rightarrow$ 01 & 00 $\rightarrow$ 00 & 21 $\rightarrow$ 00 &  $\rightarrow$ 00 &  $\rightarrow$ 00 &  $\rightarrow$ 00 & 26 $\rightarrow$  \textbf{90} \\
\texttt{medium-diverse}  & 65 $\rightarrow$  \textbf{98} & 70 $\rightarrow$ 92 & 00 $\rightarrow$ 00 & 92 $\rightarrow$ 97 & 00 $\rightarrow$ 03 & 64 $\rightarrow$ \textbf{98} &  $\rightarrow$ 02 &  $\rightarrow$ 68 &  $\rightarrow$ 00 & 75 $\rightarrow$ \textbf{98} \\
\texttt{medium-play}  & 62 $\rightarrow$ 98 & 72 $\rightarrow$ 94 & 00 $\rightarrow$ 00 & 89 $\rightarrow$ \textbf{99} & 00 $\rightarrow$ 05 & 67 $\rightarrow$ 98 &  $\rightarrow$ 25 &  $\rightarrow$ 96 &  $\rightarrow$ 00 & 54 $\rightarrow$ 97 \\  \hline
\texttt{partial}  & 71 $\rightarrow$ 75 & 40 $\rightarrow$ 60 & 01 $\rightarrow$ 13 & 11 $\rightarrow$ 22 & - & 71 $\rightarrow$ 00 &  $\rightarrow$ 00 &  $\rightarrow$ 07 &  $\rightarrow$ 03 & 67 $\rightarrow$ \textbf{79} \\
\texttt{mixed}  & 56 $\rightarrow$ 50 & 48 $\rightarrow$ 48 & 02 $\rightarrow$ 12 & 06 $\rightarrow$ 33 & - & 59 $\rightarrow$ 01 &   $\rightarrow$ 01 &   $\rightarrow$ 00 &   $\rightarrow$ 02 & 38 $\rightarrow$ \textbf{80} \\
\texttt{complete} & 13 $\rightarrow$ 34 & 57 $\rightarrow$ 50 & 01 $\rightarrow$ 08 & 17 $\rightarrow$ 41 & - & 21 $\rightarrow$ 06 &  $\rightarrow$ 00 & $\rightarrow$ 05 &  $\rightarrow$ 06 & 22 $\rightarrow$ \textbf{68} \\  \hline
\texttt{pen} & 55 $\rightarrow$ 13 & 88 $\rightarrow$ 92 & 88 $\rightarrow$ 92 & 91 $\rightarrow$ 89 & - & 48 $\rightarrow$ 10 &  $\rightarrow$ 54 &  $\rightarrow$ 17 &  $\rightarrow$ 11 & 79 $\rightarrow$ \textbf{99} \\
\texttt{door} & 22 $\rightarrow$ 88 & 41 $\rightarrow$ 88 & 29 $\rightarrow$ 13 & 04 $\rightarrow$ 08 & - & 29 $\rightarrow$ 66 &  $\rightarrow$ 88 &  $\rightarrow$ 39 &  $\rightarrow$ 17 & 35 $\rightarrow$ \textbf{92} \\
\texttt{relocate} & 06 $\rightarrow$ 69 & 06 $\rightarrow$ 45 & 06 $\rightarrow$ 08 & 03 $\rightarrow$ 35 & - & 01 $\rightarrow$ 00 &  $\rightarrow$ \textbf{99} &  $\rightarrow$ 16 &  $\rightarrow$ 00 & 03 $\rightarrow$ 98 \\  \hline
\texttt{manipulation} & 50 $\rightarrow$ 97 & 49 $\rightarrow$ 81 & 50 $\rightarrow$ 73 & - & - & 42 $\rightarrow$ 41 &  $\rightarrow$ 00 &  $\rightarrow$ 01 &  $\rightarrow$ 01 & 49 $\rightarrow$ \textbf{99} \\ \hline \hline
\textbf{average} & 42 $\rightarrow$ 71 & 50 $\rightarrow$ 69 & 16 $\rightarrow$ 20 & 44 $\rightarrow$ 45 & 00 $\rightarrow$ 02 & 42 $\rightarrow$ 29 &  $\rightarrow$ 24 &  $\rightarrow$ 23 &  $\rightarrow$ 04 & 44 $\rightarrow$ \textbf{90} \\ \hline
\textbf{improvement} & + 71.0\%  & + 37.7\%  & + 23.7\%  & + 3.0\%  & N/A  & - 30.3\%  & N/A &  N/A &  N/A & \textbf{+ 106.9\%}  \\
\end{tabular}}
\vspace{-0.15cm}
a\caption{\footnotesize{\textbf{Normalized score before \& after online fine-tuning.} Observe that \methodname\ improves over the best prior fine-tuning method and attains a much larger performance improvement over the course of online fine-tuning. The numbers represent the normalized score out of 100 following the convention in \citep{fu2020d4rl}.} \label{tab:performance}}

\vspace{-0.3cm}
\end{center}
}
\end{table*}

% \begin{wraptable}{r}{0.66\linewidth}
\begin{table*}[h]

\vspace{-0.4cm}
\small{
\begin{center}
\scalebox{0.85}{
\begin{tabular}{l|c|c|c|c|c|c|c|c|c||c}
Task & CQL & IQL & AWAC & O3F & ODT & CQL+SAC & Hybrid RL & SAC+od & SAC & Cal-QL (Ours)\\ \hline \hline 
\texttt{large-diverse}  & 0.35 & 0.46 & 1.00 & 0.62 & 0.98 & 0.99 & 1.00 & 1.00 & 1.00 & \textbf{0.20}  \\
\texttt{large-play}     & 0.32 & 0.52 & 1.00 & 0.91 & 1.00 & 0.99 & 1.00 & 1.00 & 1.00 & \textbf{0.28}  \\
\texttt{medium-diverse} & 0.06 & 0.08 & 0.99 & \textbf{0.03} & 0.95 & 0.06 & 0.98 & 0.77 & 1.00 & 0.05  \\
\texttt{medium-play}    & 0.09 & 0.10 & 0.99 & \textbf{0.04} & 0.96 & 0.06 & 0.90 & 0.47 & 1.00 & 0.07  \\ \hline
\texttt{partial}                   & 0.31 & 0.49 & 0.89 & 0.78 & - & 0.97 & 0.98 & 0.98 & 0.92 & \textbf{0.27}  \\
\texttt{mixed}                     & 0.55 & 0.60 & 0.88 & 0.72 & - & 0.97 & 0.99 & 1.00 & 0.91 & \textbf{0.27}  \\
\texttt{complete}                  & 0.70 & 0.53 & 0.97 & 0.66 & - & 0.99 & 0.99 & 0.96 & 0.91 & \textbf{0.44}  \\ \hline
\texttt{pen}                       & 0.86 & \textbf{0.11} & 0.12 & 0.13 & - & 0.90 & 0.56 & 0.75 & 0.87 & \textbf{0.11}  \\
\texttt{door}                      & 0.36 & 0.25 & 0.81 & 0.82 & - & \textbf{0.23} & 0.35 & 0.60 & 0.94 & \textbf{0.23}  \\
\texttt{relocate}                  & 0.71 & 0.74 & 0.95 & 0.71 & - & 0.86 & \textbf{0.30} & 0.89 & 1.00 & 0.43  \\ \hline
\texttt{manipulation}              & 0.15 & 0.32 & 0.38 & - & - & 0.61 & 1.00 & 1.00 & 0.99 & \textbf{0.11}   \\ \hline \hline
\textbf{average}              & 0.41 & 0.38 & 0.82 & 0.54 & 0.97 & 0.69 & 0.82 & 0.86 & 0.96 & \textbf{0.22} 

\end{tabular}}
\vspace{-0.05cm}
\caption{\footnotesize{\textbf{Cumulative regret averaged over the steps of fine-tuning.} The smaller the better and 1.00 is the worst. \methodname\ attains the smallest overall regret, achieving the best performance among 8 / 11 tasks.}}
\label{tab:results_regret}

\vspace{-0.6cm}
\end{center}}
% \end{wraptable}
\end{table*}

\vspace{-0.1cm}
\subsection{\methodname\ With High Update-to-Data (UTD) Ratio}
\label{subsec:highutd}
\vspace{-0.2cm}


We can further enhance the online sample efficiency of \methodname\ by increasing the number of gradient steps per environment step made by the algorithm. The number of updates per environment step is usually called the update-to-data (UTD) ratio.
% In Figure~\ref{fig:utd-cog}, we present the performance of \methodname\ with $\text{UTD}=5$ on the \texttt{visual-manipulation} task. We can see that \methodname\ can achieve higher sample efficiency by using $\text{UTD}=5$, and the initial unlearning disappears completely which we studied previously in Section~\ref{sec:empirical_analysis}. Interestingly, while prior fine-tuning methods such as IQL do not improve with a higher UTD (see Figure~\ref{fig:utd-cog}), \methodname\ and CQL benefit from using a higher UTD ratio. We will further discuss this in Appendix~\ref{appendix:more_discussion_on_finetuning}.
In standard online RL, running off-policy Q-learning with a high UTD value (e.g., 20, compared to the typical value of 1) often results in challenges pertaining to overfitting~\citep{li2023efficient, redq, rlpd, replaybarrier}. As expected, we noticed that running \methodname\ with a high UTD value also leads these overfitting challenges. To address these challenges in high UTD settings, we combine \methodname\ with the Q-function architecture in recent work, RLPD~\citep{rlpd} (i.e., we utilized layer normalization in the Q-function and ensembles akin to \citep{redq}), that attempts to tackle overfitting challenges. Note that \methodname\ still first pre-trains on the offline dataset using Equation~\ref{eqn:cal_ql_training} followed by online fine-tuning, unlike RLPD that runs online RL right from the start.
In Figure~\ref{fig:rlpd}, we compare \methodname\ ($\text{UTD}=20$) with RLPD~\cite{rlpd} ($\text{UTD}=20$) and also \methodname\ ($\text{UTD}=1$) as a baseline. Observe that \methodname\ ($\text{UTD}=20$) improves over \methodname\ ($\text{UTD}=1$) and training from scratch (RLPD).   

\iffalse
%%AK: This is the original version of this section

We can further enhance the online sample efficiency of \methodname\ by increasing the number of gradient steps per environment step made by the algorithm. The number of updates per environment step is usually called the update-to-data (UTD) ratio. We will now evaluate \methodname\ with a high UTD ratio. In Figure~\ref{fig:utd-cog}, we present the performance of \methodname\ with $\text{UTD}=5$ on the \texttt{visual-manipulation} task. We can see that \methodname\ can achieve higher sample efficiency by using $\text{UTD}=5$, and the initial unlearning disappears completely which we studied previously in Section~\ref{sec:empirical_analysis}. Interestingly, while prior fine-tuning methods such as IQL do not improve with a higher UTD (see Figure~\ref{fig:utd-cog}), \methodname\ and CQL benefit from using a higher UTD ratio. We will further discuss this in Appendix~\ref{appendix:more_discussion_on_finetuning}.

We noticed that simply running \methodname\ with an even higher UTD value (e.g., 20) leads to previously-observed challenges pertaining to overfitting~\citep{li2023efficient, redq, rlpd, replaybarrier}. To address these challenges in high UTD settings, we ran \methodname\ in conjunction with the Q-function architecture in \cite{rlpd} (i.e., we utilized layer normalization in the Q-function and ensembles akin to \cite{redq}) to prevent overfitting. Note that \methodname\ still first pre-trains on the offline dataset using Equation~\ref{eqn:cal_ql_training} followed by online fine-tuning, unlike RLPD that runs online RL from scratch on offline and online data. 
In Figure~\ref{fig:rlpd}, we compare \methodname\ ($\text{UTD}=20$) with RLPD~\cite{rlpd} ($\text{UTD}=20$) and also \methodname\ ($\text{UTD}=1$) as a baseline. Observe that \methodname\ ($\text{UTD}=20$) improves over \methodname\ ($\text{UTD}=1$) and training from scratch (RLPD).   

\begin{figure}[ht]
\vspace{-0.3cm}
\begin{center}
\centerline{\includegraphics[width=0.55\textwidth]{chapters/cal_ql/figs-sample/cog-highutd-final.pdf}}
\vspace{-0.3cm}
\caption{\label{fig:utd-cog}\footnotesize{\textbf{UTD ablation}: We observe that using a higher UTD ratio can lead to higher sample efficiency for \methodname\ and CQL but not for IQL.}}
\end{center}
\vspace{-0.9cm}
\end{figure}

\fi

\begin{figure}[h]
\begin{center}    
{\includegraphics[clip,width=1\linewidth]{chapters/cal_ql/figs-sample/antmaze-utd20-rlpd-final.pdf}} {\includegraphics[clip,width=0.75\linewidth]{chapters/cal_ql/figs-sample/adroit-rlpd-final.pdf}}
\end{center}
\vspace{-0.3cm}
\caption{\label{fig:rlpd} \footnotesize{\textbf{\methodname\ with UTD=20}. Incorporating design choices from RLPD enables \methodname\ to achieve sample-efficient fine-tuning with UTD=20. Specifically, \methodname\ generally attains similar or higher asymptotic performance as RLPD, while also exhibiting a smaller cumulative regret.}}
\vspace{-0.6cm}
\end{figure}





\vspace{-0.2cm}
\subsection{Understanding the Behavior of \methodname}
\label{subsec:diagonistic}
\vspace{-0.2cm}

In this section, we aim to understand the behavior of \methodname\ by performing controlled experiments that modify the dataset composition, and by investigating various metrics to understand the properties of scenarios where utilizing \methodname\ is especially important for online fine-tuning.         

\textbf{Effect of data composition.} To understand the efficacy of \methodname\ with different data compositions, we ran it on a newly constructed fine-tuning task on the medium-size \texttt{AntMaze} domain with a low-coverage offline dataset, which is generated via a scripted controller that starts from a fixed initial position and navigates the ant to a fixed goal position. In Figure~\ref{fig:ant-narrow}, we plot the performance of \methodname\ and baseline CQL (for comparison) on this task, alongside the trend of average Q-values over the course of offline pre-training (to the left of the dashed vertical line, before 250 training epochs) and online fine-tuning (to the right of the vertical dashed line, after 250 training epochs), and the trend of \emph{bounding rate}, i.e., the fraction of transitions in the data-buffer for which the constraint in \methodname\ actively lower-bounds the learned Q-function with the reference value. For comparison, we also plot these quantities for a diverse dataset with high coverage on the task (we use the \texttt{antmaze-medium-diverse} from \cite{fu2020d4rl} as a representative diverse dataset) in Figure~\ref{fig:ant-narrow}. 

Observe that for the diverse dataset, both na\"ive CQL and \methodname\ perform similarly, and indeed, the learned Q-values behave similarly for both of these methods. In this setting, online learning doesn't spend samples to correct the Q-function when fine-tuning begins leading to a low bounding rate, almost always close to 0. Instead, with the narrow dataset, we observe that the Q-values learned by na\"ive CQL are much smaller, and are corrected once fine-tuning begins. This correction co-occurs with a drop in performance (solid blue line on left), and na\"ive CQL is unable to recover from this drop. \methodname\, which calibrates the scale of the Q-function for many more samples in the dataset, stably transitions to online fine-tuning with no unlearning (solid red line on left). 

\begin{wrapfigure}{r}{0.54\textwidth}
\vspace{-0.35cm}
\centering
\includegraphics[width=0.97\linewidth]{chapters/cal_ql/figs-sample/diagnose-1.png}
\vspace{-0.3cm}
\caption{
\footnotesize{\textbf{Performance of \methodname\ with data compositions.} \methodname\ is most effective with narrow datasets, where Q-values need to be corrected at the beginning of fine-tuning.}
}
\label{fig:ant-narrow}
\vspace{-0.35cm}
\end{wrapfigure}


This suggests that in settings with narrow datasets (e.g., in the experiment above and in the \texttt{adroit} and \texttt{visual-manipulation} domains from Figure~\ref{fig:all_tasks}), Q-values learned by na\"ive conservative methods are more likely to be smaller than the ground-truth Q-function of the behavior policy due to function approximation errors. Hence utilizing \methodname\ to calibrate the Q-function against the behavior policy can be significantly helpful. On the other hand, with significantly high-coverage datasets, especially in problems where the behavior policy is also random and sub-optimal, Q-values learned by na\"ive methods are likely to already be calibrated with respect to those of the behavior policy. Therefore no explicit calibration might be needed (and indeed, the bounding rate tends to be very close to 0 as shown in Figure~\ref{fig:ant-narrow}). In this case, \methodname\ will revert back to standard CQL, as we observe in the case of the diverse dataset above. This intuition is also reflected in Theorem~\ref{thm:main-thm-informal}: when the reference policy $\mu$ is close to a narrow, expert policy, we would expect \methodname\ to be especially effective in controlling the efficiency of online fine-tuning. \textbf{We also present a diagnostic study of \methodname\ when the reference value function is estimated by fitting a neural network in Appendix~\ref{app:nn_value_function}}, and find that estimation errors in this model of the reference function do not affect performance significantly. 




\vspace{-0.2cm}
\vspace{-0.2cm}
\section{Discussion and Limitations}
\vspace{-0.2cm}

In this chapter, we developed \methodname\, a method for acquiring conservative offline initializations that facilitate fast online fine-tuning. \methodname\ learns conservative value functions that are constrained to be larger than the value function of a reference policy. This form of calibration allows us to avoid initial unlearning when fine-tuning with conservative methods, while also retaining the effective asymptotic performance that these methods exhibit. Our theoretical and experimental results highlight the benefit of \methodname\ in enabling fast online fine-tuning. 
% Specifically, \methodname\ improves over a number of prior approaches on 9/11 tasks that we study. 
While \methodname\ outperforms prior methods, we believe that we can develop even more effective methods by adjusting calibration and conservatism more carefully. A limitation of our work is that we do not consider fine-tuning setups where pre-training and fine-tuning tasks are different, but this is an interesting avenue for future work. Theoretically, our analysis can be improved via a more precise treatment of pessimism, and this is an avenue for future work.

\bibliographystyle{abbrvnat}
\bibliography{reference.bib}

\appendix
\vspace{-0.2cm}
\section{Implementation details of Cal-QL}
\vspace{-0.2cm}
Our algorithm, \methodname\ is illustrated in Algorithm~\ref{alg:practical_alg}. 

\vspace{-0.2cm}
\subsection{\methodname\ Algorithm}
\vspace{-0.2cm}
We use $J_Q(\theta)$ to denote the conservative regularizer for the Q network update:
\begin{align}
    \label{eqn:calql_training_J}
    J_Q(\theta):=&\;{{\alpha \underbrace{\left(\mathbb{E}_{\bs \sim \mathcal{D}, \mathbf{a} \sim \pi} \brac{{\color[HTML]{B85450}{\max \left( Q_\theta(s,a), Q^\mu(s,a) \right)}} } - \mathbb{E}_{s, a \sim \mathcal{D}}\left[Q_\theta(s,a)\right]\right)}_{\text{Calibrated conservative regularizer }\mathcal{R}(\theta)}}}\\
    &\;+ \frac{1}{2} {\mathbb{E}_{s, a, s^\prime\sim \mathcal{D}}\left[\left(Q_\theta(s, a) - \bellman^\pi\mathbf{a}r{Q}(s, a)\right)^2 \right]}.
\end{align}

\begin{center}
\begin{minipage}{0.64\linewidth}
    \begin{algorithm}[H]
    \caption{\methodname\ pseudo-code}
    \label{alg:practical_alg}
    \begin{algorithmic}[1]
        \State Initialize Q-function, $Q_\theta$, a policy, $\pi_\phi$
        \For{step $t$ in \{1, \dots, N\}}
            \State Train the Q-function using the conservative regularizer in Eq.~\ref{eqn:calql_training_J}:
            \begin{equation}
                \theta_t := \theta_{t-1} - \eta_Q \nabla_\theta J_Q(\theta)
            \end{equation}
            \State Improve policy $\pi_\phi$ with SAC-style update:
            \begin{equation}
                \phi_{t} := \phi_{t-1} + \eta_\pi \mathbb{E}_{\bs \sim \mathcal{D}, \mathbf{a} \sim \pi_\phi(\cdot|\bs)}[Q_\theta(\bs, \mathbf{a})\! -\! \log \pi_\phi(\mathbf{a}|\bs)]
            \end{equation}
        \EndFor
    \end{algorithmic}
    \end{algorithm}
\end{minipage}
\end{center}

\vspace{-0.2cm}
\section{Regret Analysis of \methodname}
\vspace{-0.2cm}
We provide a theoretical version of \methodname\ in Algorithm~\ref{algo:CalQL-thm}. Policy fine-tuning has been studied in different settings~\citep{xie2021policy,song2023hybrid,wagenmaker2022leveraging}. Our analysis largely adopts the settings and results in \citet{song2023hybrid}, with additional changes in Assumption~\ref{assump:conservative-realizability-completeness}, Assumption~\ref{assump:conservative-bilinear-rank}, and Definition~\ref{def:bellman-error-coeff-ref}. Note that the goal of this proof is to demonstrate that
{\em a pessimistic functional class (Assumption~\ref{assump:conservative-realizability-completeness})} allows one to utilize the offline data efficiently, rather than providing a new analytical technique for regret analysis. Note that we use $f$ instead of $Q_\theta$ in the main text to denote the estimated $Q$ function for notation simplicity.

\begin{algorithm}[H]

\caption{Theoretical version of \methodname}\label{algo:CalQL-thm}
  \begin{algorithmic}[1]
    \State \textbf{Input:} Value function class $\mc F$, \# total iterations $K$, offline dataset $\mc D_h^\nu$ of size $\moff$ for $h\in[H-1]$.
    \State Initialize $f_h^1(s,a)=0,\forall (s,a)$.
    \For{$k=1,\dots,K$}
    \State Let $\pi^t$ be the greedy policy w.r.t. $f^k$ \Comment{I.e., $\pi^k_h(s)=\arg\max_af_h^k(s,a)$.}
    \State For each $h$, collect $\mon$ online tuples $\mc D_h^k\sim d_h^{\pi^k}$ \Comment{online data collection}
    \State Set $f_H^{k+1}(s,a)=0,\forall (s,a)$.

    \For{$h=H-1,\dots 0$} \Comment{FQI with offline and online data}
        \State Estimate $f_h^{k+1}$ using {\color{purple} conservative} least squares on the aggregated data: {\color{purple} \Comment{ I.e., CQL regularized class $\mc C_h$}}
        {\scriptsize
        \begin{equation}
            \label{eq:least-square-our-method}f_h^{k+1}\leftarrow\underset{f\in {\color{purple}\mc C_h}}{\arg\min}\Brac{\EmpiricalOffline\brac{f(s,a)-r-\max_{a^\prime}f_{h+1}^{k+1}(s^\prime,a^\prime)}^2 + \sum_{\tau=1}^K\EmpiricalOnline\brac{f(s,a)-r-\max_{a^\prime}f_{h+1}^{k+1}(s^\prime,a^\prime)}^2}  
        \end{equation}}
        \State {\color{purple} $f_h^{k+1} = \max\{f_h^{k+1}, Q^\mr{ref}_h\}$ \Comment{Set a reference policy for calibration (Definition~\ref{cond:calibration})} }
    \EndFor
    \EndFor
    \State \textbf{Output:} $\pi^K$
  \end{algorithmic}
\end{algorithm}


% \subsection{Preliminaries}
% In this subsection, we follow most of the notations and definitions in~\citet{song2023hybrid}. In particular, we consider the finite horizon cases, where the value function and Q function are defined as:
% \begin{align}
%     V_h^\pi(s) &= \bb E\brac{\sum_{\tau=h}^{H-1}r_\tau|\pi, s_h=s}\\
%     Q_h^\pi(s,a) &= \bb E\brac{\sum_{\tau=h}^{H-1}r_\tau|\pi,s_h=s,a_h=a}.
% \end{align}
% We also define the Bellman operator $\mc T$ such that $\forall f:\mc S\times \mc A$:
% \begin{equation}
%     \mc T f(s,a) = \bb E_{s,a}[R(s,a)] + \bb E_{s^\prime \sim P(s,a)}\max_{a^\prime} f(s^\prime, a^\prime),\;\forall (s,a)\in \mc S\times \mc A,
% \end{equation}
% where $R(s,a)\in \Delta[0,1]$ represents a stochastic reward function.

\vspace{-0.2cm}
\subsection{Notations}
\label{appendix:notations}
\vspace{-0.2cm}
\begin{itemize}
    \item Feature covariance matrix $\Sigma_{k;h}$: 
    \begin{equation}
    \label{eq:covariance-matrix}
        \mb \Sigma_{k;h}=\sum_{\tau=1}^kX_h(f^\tau)(X_h(f^\tau))^\top+\lambda \mb I
    \end{equation}
    \item Matrix Norm \citet{zanette2021provable}: for a matrix $\Sigma$, the matrix norm $\norm{\mb u}{\mb \Sigma}$ is defined as:
    \begin{equation}
        \norm{\mb u}{\mb \Sigma}=\sqrt{\mb u\mb \Sigma \mb u^\top}
    \end{equation}
    \item Weighted $\ell^2$ norm: for a given distribution $\beta\in\Delta(\mc S\times \mc A)$ and a function $f:\mc S\times \mc A\mapsto \bb R$, we denote the weighted $\ell^2$ norm as:
    \begin{equation}
        \norm{f}{2,\beta}^2:=\sqrt{\bb E_{(s,a)\sim \beta}f^2(s,a)}
    \end{equation}
    \item A stochastic reward function $R(s,a)\in \Delta([0,1])$
    \item For each offline data distribution $\nu = \{\nu_0,\dots,\nu_{H-1}\}$, the offline data set at time step $h$ ($\nu_h$) contains data samples $(s,a,r,s^\prime)$, where $(s,a)\sim\nu_h,r\in R(s,a),s^\prime\sim P(s,a)$.
    \item Given a policy $\pi:=\{\pi_0,\dots,\pi_{H-1}\}$, where $\pi_h:\mc S\mapsto \Delta (\mc A)$, $d^\pi_h\in\Delta(s,a)$ denotes the state-action occupancy induced by $\pi$ at step $h$.
    \item We consider the value-based function approximation setting, where we are given a function class $\mc C=\mc C_0\times\dots\mc C_{H-1}$ with $\mc C_{h}\subset\mc S\times \mc A\mapsto[0, \Vmax]$.
    \item A policy $\pi^f$ is defined as the greedy policy w.r.t. $f$: $\pi_h^f(s)=\arg\max_af_h(s,a)$. Specifically, at iteration $k$, we use $\pi^k$ to denote the greedy policy w.r.t. $f^k$.
\end{itemize}


\subsection{Assumptions and Defintions}
\label{subsec:CalQL-assumptions}
\begin{assumption}[Pessimistic Realizability and Completeness] 
\label{assump:conservative-realizability-completeness}
For any policy $\pi^e$, we say $\mc C_h$ is a pessimistic function class w.r.t. $\pi^e$, if for any $h$, we have $Q^{\pi^e}_h\in \mc C_h$, and additionally, for any $f_{h+1}\in \mc C_{h+1}$, we have $\mc T f_{h+1}\in \mc C_h$ and $f_h(s,a)\leq Q_h^{\pi^e}(s,a),\forall (s,a)\in \mc S\times \mc A$.
\end{assumption}

\begin{definition}[Bilinear model~\citet{du2021bilinear}]
\label{def:bilinear-model}
We say that the MDP together with the function class $\mc F$ is a bilinear model of rand $d$ of for any $h\in [H-1]$, there exist two (known) mappings $X_h,W_h:\mc F\mapsto \bb R^d$ with $\max_f\norm{X_h(f)}{2}\leq B_X$ and $\max_f\norm{W_h(f)}{2}\leq B_W$ such that 
\begin{equation}
    \label{eq:bilinear-model}
    \forall f, g \in \mc F:\quad \abs{\bb E_{s,a\sim d_h^{\pi^f}} [g_h(s,a)-\mc Tg_{h+1}(s,a)]} = \abs{\innerprod{X_h(f)}{W_h(g)}}.
\end{equation}
\end{definition}

\begin{assumption}[Bilinear Rank of Reference Policies]
\label{assump:conservative-bilinear-rank}
Suppose $Q^{\mr{ref}}\in\mc C_\mr{ref}\subset\mc C$, where $\mc C_\mr{ref}$ is the function class of our reference policy, we assume the Bilinear rank of $\mc C_\mr{ref}$ is $d_\mr{ref}$ and $d_\mr{ref}\leq d$.
\end{assumption}


\begin{definition}[Calibrated Bellman error transfer coefficient]
\label{def:bellman-error-coeff-ref}
For any policy $\pi$, we define the calibrated transfer coefficient w.r.t. to a reference policy $\pi^\mr{ref}$ as 
\begin{equation}
    C^\mr{ref}_\pi:=\max_{f\in \mc C,f(s,a)\geq Q^\mr{ref}(s,a)}\frac{\sum_{h=0}^{H-1}\bb E_{s,a\sim d_h^\pi}[\mc T f_{h+1}(s,a)-f_h(s,a)]}{\sqrt{\sum_{h=0}^{H-1}\bb E_{s,a\sim \nu_h}(\mc T f_{h+1}(s,a)-f_h(s,a))^2}},
\end{equation}
where $Q^\mr{ref} = Q^{\pi^\mr{ref}}$.
\end{definition}

\vspace{-0.2cm}
\subsection{Discussions on the Assumptions}
\vspace{-0.2cm}
The pessimistic realizability and completeness assumption (Assumption~\ref{assump:conservative-realizability-completeness}) is motivated by some theoretical studies of the pessimistic offline methods~\cite{xie2021bellman,cheng2022adversarially} with regularizers:

\begin{align}
    \label{eq:dual-conservative-formulation}
    \min_\theta  {\alpha \underbrace{\left(\mathbb{E}_{s \sim \mathcal{D}, a \sim \pi} \left[Q_\theta(s,a)\right] - \mathbb{E}_{s, a \sim \mathcal{D}}\left[Q_\theta(s,a)\right]\right)}_{\text{Conservative regularizer }\mathcal{R}(\theta)}} + \frac{1}{2} {\mathbb{E}_{s, a, s^\prime\sim \mathcal{D}}\left[\left(Q_\theta(s, a) - \bellman^\pi\mathbf{a}r{Q}(s, a)\right)^2 \right]}.
\end{align}
Since the goal of the conservative regularizer $\mc R(\theta)$ intrinsically wants to enforce
\begin{equation}
    Q_\theta(s,\pi(s)) \leq Q_\theta(s,\pi^e(s)),
\end{equation}
where $\pi$ is the training policy and $\pi^e$ is the reference (behavior) policy. One can consider \eqref{eq:dual-conservative-formulation} as the Lagrange duality formulation of the following primal optimization problem:{\small
\begin{equation}
    \min_\theta    {\mathbb{E}_{s, a, s^\prime\sim \mathcal{D}}\left[\left(Q_\theta(s, a) - \bellman^\pi\mathbf{a}r{Q}(s, a)\right)^2 \right]},\;\text{subject to}\; \mathbb{E}_{s \sim \mathcal{D}, a \sim \pi} \left[Q_\theta(s,a)\right] \leq \mathbb{E}_{s\sim \mathcal{D},a\sim\pi^e}\left[Q_\theta(s,a)\right],
\end{equation}
}
where the constraint set is equivalent to Assumption~\ref{assump:conservative-realizability-completeness}. Although Assumption~\ref{assump:conservative-realizability-completeness} directly characterizes the constraint set of the primal form of~\eqref{eq:dual-conservative-formulation} the exact theoretical connection between the pessimistic realizability and the regularized bellman consistency equation is beyond the scope of this work and we would like to leave that for future studies.

Assumption~\ref{assump:conservative-realizability-completeness} allows us to restrict the functional class of interest to a smaller conservative function class $\mc C\subset\mc F$, which leads to a smaller Bellman rank of the reference policy $(d_\mr{ref}\leq d)$ suggested in Assumption~\ref{assump:conservative-bilinear-rank}, and a smaller concentrability coefficient $(C^\mr{ref}_\pi\leq C_\pi)$ defined in Definition~\ref{def:bellman-error-coeff-ref}, and \ref{def:bellman-error-coeff-hy-q}. Assumption~\ref{assump:conservative-bilinear-rank} and Definition~\ref{def:bellman-error-coeff-hy-q} provide the Bellman Bilinear rank and Bellman error transfer coefficient of the pessimistic functional class $\mc C$ of interest.

\vspace{-0.2cm}
\subsection{Proof Structure Overview}
\vspace{-0.2cm}
We provide an overview of the proof structure and its dependency on different assumptions below:
\begin{itemize}
    \item Theorem~\ref{thm:CalQL-regret-main}: the total regret is decomposed into {\em offline regrets} and {\em online regrets}.
    \begin{itemize}
        \item Bounding {\em offline regrets}, requiring Definition~\ref{def:bellman-error-coeff-ref} and the following lemmas:
        \begin{itemize}
            \item Performance difference lemma w.r.t. a comparator policy (Lemma~\ref{lemma:perf-diff-comparator}).
            \item Least square generalization bound (Lemma~\ref{lemma:bellman-error-bound}), requiring Assumption~\ref{assump:conservative-realizability-completeness}.
        \end{itemize}
        \item Bounding {\em online regrets}, requiring Definition~\ref{def:bilinear-model}
        \begin{itemize}
            \item Performance difference lemma for the online error (Lemma~\ref{lemma:perf-diff-online}).
            \item Least square generalization bound (Lemma~\ref{lemma:bellman-error-bound}), requiring Assumption~\ref{assump:conservative-realizability-completeness}.
            \item Upper bounds with the bilinear model assumption (Lemma~\ref{lemma:bilinear-upper-bound}).
            \item Applying Elliptical Potential Lemma~\cite{lattimore2020bandit} with bellman rank $d$ and $d_\mr{ref}$ (Lemma~\ref{lemma:bound-inverse-cov-norm}), requiring Assumption~\ref{assump:conservative-bilinear-rank}.
        \end{itemize}
    \end{itemize}
\end{itemize}

\subsection{Our Results}
\label{appdendix:derivation-CalQL}
\begin{theorem}[Formal Result of Theorem~\ref{thm:main-thm-informal}]
\label{thm:CalQL-regret-main}
    Fix $\delta\in(0,1),\moff=K,\mon=1$, suppose and the function class $\mc C$ follows Assumption~\ref{assump:conservative-realizability-completeness} w.r.t. $\pi^e$. Suppose the underlying MDP admits Bilinear rank $d$ on function class $\mc C$ and $d_\mr{ref}$ on $\mc C_\mr{ref}$, respectively, then with probability at least $1-\delta$, Algorithm~\ref{algo:CalQL-thm} obtains the following bound on cumulative suboptimality w.r.t. any comparator policy $\pi^e$:{\small
    \begin{equation}
        \begin{split}
        \sum_{t=1}^KV^{\pi^e}-V^{\pi^k} =\wt{O}\paren{\min\Brac{C_{\pi^e}^\mr{ref}H\sqrt{dK\log\paren{|\mc F|/\delta}},\;K\paren{V^{\pi^e}-V^\mr{ref}}+H\sqrt{d_\mr{ref}K\log\paren{|\mc F|/\delta}}}}.
        \end{split}
    \end{equation}
    }
\end{theorem}
\noindent Note that Theorem~\ref{thm:CalQL-regret-main} provides a guarantee for {\em any} comparator policy $\pi^\e$, which can be directly applied to $\pi^\star$ described in our informal result (Theorem~\ref{thm:main-thm-informal}). We also change the notation
for the reference policy from $\mu$ in Theorem~\ref{thm:main-thm-informal} to $\pi^\mr{ref}$ (similarly, $d_\mr{ref},V^\mr{ref},C_{\pi^e}^\mr{ref}$ correspond to $d_\mu,V^\mu,C_{\pi^e}^\mu$ in Theorem~\ref{thm:main-thm-informal}) for notation consistency in the proof. Our proof of Theorem~\ref{thm:CalQL-regret-main} largely follows the proof of Theorem 1 of~\citep{song2023hybrid}, and the major changes are caused by Assumption~\ref{assump:conservative-realizability-completeness}, Assumption~\ref{assump:conservative-bilinear-rank}, and Definition~\ref{def:bellman-error-coeff-ref}.
\begin{proof}
    \label{proof:CalQL-main-thm}
    Let $\mc E_k $ denote the event that $\Brac{f_0^k(s,a)\leq Q^\mr{ref}(s,a)}$ and $\mathbf{a}r{\mc E}_k $ denote the event that $\Brac{f_0^k(s,a) > Q^\mr{ref}(s,a)}$. Let $V^\mr{ref}(s) = \max_aQ^\mr{ref}(s,a)$, we start by noting that
\begin{equation}
    \label{eq:conservative-main-thm-derivation0}
    \begin{split}
        &\sum_{k=1}^KV^{\pi^e}-V^{\pi^{f^k}} = \sum_{k=1}^K\bb E_{s\sim\rho}\brac{V_0^{\pi^e}(s)-V^{\pi^{f^k}}_0(s)}\\
        =&\;\underbrace{\sum_{k=1}^K\bb E_{s\sim\rho}\brac{\indicator{\mathbf{a}r{\mc E}_k}\paren{V_0^{\pi^e}(s)-V^{\mr{ref}}(s)}}}_{\Gamma_0}+\underbrace{\sum_{k=1}^K\bb E_{s\sim\rho}\brac{\indicator{\mathbf{a}r{\mc E}_k}\paren{V^{\mr{ref}}(s)-\max_a f_0^k(s,a)}}}_{=0, \text{ by the definition of }\mathbf{a}r{\mc E}_k\text{ and line 9 of Algorithm~\ref{algo:CalQL-thm}}}\\
        &+\underbrace{\sum_{t=1}^K\bb E_{s\sim\rho}\brac{\indicator{\mathbf{a}r{\mc E}_k}\paren{\max_a f_0^k(s,a)-V^{\pi^{f^k}}_0(s)}}}_{\Gamma_1}+\underbrace{\sum_{k=1}^K\bb E_{s\sim\rho}\brac{\indicator{{\mc E}_k}\paren{V_0^{\pi^e}(s)-\max_a f_0^k(s,a)}}}_{\Gamma_2}\\
        &+\underbrace{\sum_{t=1}^T\bb E_{s\sim\rho}\brac{\indicator{{\mc E}_k}\paren{\max_a f_0^k(s,a)-V^{\pi^{f^k}}_0(s)}}}_{\Gamma_3}.\\
    \end{split}
\end{equation}
Let $K_1 = \sum_{k=1}^K\indicator{f_0^k(s,a)>Q^\mr{ref}(s,a)}$ and $K_2 = \sum_{k=1}^K\indicator{f_0^k(s,a)\leq Q^\mr{ref}(s,a)}$ (or equivalently $K_1 = \sum_{k=1}^K\indicator{\mathbf{a}r{\mc E}_k}$, $K_2=\sum_{k=1}^K\indicator{{\mc E}_k}$). For $\Gamma_0$, we have
\begin{equation}
    \label{eq:derivation_gamma0}
    \Gamma_0 = K_2\bb E_{s\sim\rho}\paren{V^{\pi^e}(s)-V^{\mr{ref}}(s)}.
\end{equation}
For $\Gamma_2$, we have 

\begin{equation}
    \label{eq:derivation_gamma2}
    \begin{split}
        \Gamma_2 = &\;\sum_{k=1}^K\bb E_{s\sim\rho}\brac{\indicator{{\mc E}_k}\paren{V_0^{\pi^e}(s)-\max_a f_0^k(s,a)}}\\
        \overset{(i)}{\leq} & \sum_{k=1}^K\indicator{\mc E_k}\sum_{h=0}^{H-1}\bb E_{s,a\sim d_h^{\pi^e}}\brac{\mc T f_{h+1}^k(s,a)-f_h^k(s,a)}\\
        \overset{(ii)}{\leq} & \sum_{k=1}^K \brac{C_{\pi^e}^\mr{ref}\cdot\indicator{\mc E_k}\sqrt{\sum_{h=0}^{H-1}\bb E_{s,a\sim\nu_h}\brac{\paren{f_h^k(s,a)-\mc T f_{h+1}^k(s,a)}^2}}} \\
        \overset{(iii)}{\leq} & K_1C_{\pi^e}^\mr{ref}\sqrt{H\cdot \Deltaoff},
    \end{split}
\end{equation}
where $\Deltaoff$ is similarly defined as~\citet{song2023hybrid} (See \eqref{eq:delta-off} of Lemma~\ref{lemma:bellman-error-bound}). Inequality $(i)$ holds because of Lemma~\ref{lemma:perf-diff-comparator}, inequality $(ii)$ holds by the definition of $C_{\pi^e}^\mr{ref}$ (Definition~\ref{def:bellman-error-coeff-ref}), inequality $(iii)$ holds by applying Lemma~\ref{lemma:bellman-error-bound} with the function class satisfying Assumption~\ref{assump:conservative-realizability-completeness}, and Definition~\ref{def:bellman-error-coeff-ref}. Note that the telescoping decomposition technique in the above equation also appears in~\citep{xie2020q,jin2021bellman,du2021bilinear}. Next, we will bound $\Gamma_1+\Gamma_3$:
\begin{equation}
    \begin{split}
        \Gamma_1+\Gamma_3  =& \sum_{k=1}^K\paren{\indicator{\mc E_k}+\indicator{\mathbf{a}r{\mc E}_k}}\bb E_{s\sim d_0}\brac{\max_a f_0^k(s,a)-V_0^{\pi^{f^k}}(s)}\\
        \overset{(i)}{\leq}& \sum_{k=1}^K\paren{\indicator{\mc E_k}+\indicator{\mathbf{a}r{\mc E}_k}}\sum_{h=0}^{H-1}\abs{\bb E_{s,a\sim d_h^{\pi^{f^k}}}\brac{f_h^k(s,a)-\mc T f_{h+1}^k(s,a)}}\\
        \overset{(ii)}{=}&\sum_{t=1}^K\brac{\paren{\indicator{\mc E_k}+\indicator{\mathbf{a}r{\mc E}_k}}\sum_{h=0}^{H-1}\abs{\innerprod{X_h(f^k)}{W_h(f^k)}}}\\
        \overset{(iii)}{\leq}&\sum_{k=1}^K\brac{\paren{\indicator{\mc E_k}+\indicator{\mathbf{a}r{\mc E}_k}}\sum_{h=0}^{H-1}\norm{X_h(f^k)}{\mb \Sigma^{-1}_{k-1;h}}\sqrt{\Deltaon+\lambda B_W^2}},
    \end{split}
\end{equation}
where $\Deltaon$ is similarly defined as~\citet{song2023hybrid} (See \eqref{eq:delta-on} of Lemma~\ref{lemma:bellman-error-bound}). Inequality~$(i)$ holds by Lemma~\ref{lemma:perf-diff-online}, equation~$(ii)$ holds by the definition of Bilinear model (\eqref{eq:bilinear-model} in Definition~\ref{def:bilinear-model}), inequality~$(iii)$ holds by Lemma~\ref{lemma:bilinear-upper-bound} and Lemma~\ref{lemma:bellman-error-bound} with the function class satisfying Assumption~\ref{assump:conservative-realizability-completeness}. Using Lemma~\ref{lemma:bound-inverse-cov-norm}, we have that 
{\small
\begin{equation}
    \label{eq:derivation_gamma13}
    \begin{split}
        &\Gamma_1+\Gamma_3\\
        \leq&\sum_{k=1}^K\brac{\paren{\indicator{\mc E_k}+\indicator{\mathbf{a}r{\mc E}_k}}\sum_{h=0}^{H-1}\norm{X_h(f^k)}{\mb \Sigma^{-1}_{k-1;h}}\sqrt{\Deltaon+\lambda B_W^2}}\\
        \overset{(i)}{\leq} &H\sqrt{2d\log\paren{1+\frac{K_1B_X^2}{\lambda d}}\cdot(\Deltaon+\lambda B_W^2)\cdot K_1}+H\sqrt{2d_\mr{ref}\log\paren{1+\frac{K_2B_X^2}{\lambda d_\mr{ref}}}\cdot(\Deltaon+\lambda B_W^2)\cdot K_2}\\
        \overset{(ii)}{\leq} &H\paren{\sqrt{2d\log\paren{1+\frac{K_1}{ d}}\cdot(\Deltaon+B_X^2 B_W^2)\cdot K_1}+\sqrt{2d_\mr{ref}\log\paren{1+\frac{K_2}{d_\mr{ref}}}\cdot(\Deltaon+B_X^2 B_W^2)\cdot K_2}},
    \end{split}
\end{equation}
}
where the first part of inequality $(i)$ holds by the assumption that the underlying MDPs have bellman rank $d$ (Definition~\ref{def:bilinear-model}) when $\mathbf{a}r{\mc E}_k$ happens, and the second part of inequality $(i)$ holds by the assumption 
that $\mc C_\mr{ref}$ has bilinear rank $d_\mr{ref}$ (Assumption~\ref{assump:conservative-bilinear-rank}) $\mc C_\mr{ref}$ has bellman rank $d_\mr{ref}$ when $\mc E_k$ happens. Inequality $(ii)$ holds by plugging in $\lambda = B_X^2$. Substituting \eqref{eq:derivation_gamma0}, inequality~\ref{eq:derivation_gamma2}, and inequality \eqref{eq:derivation_gamma13} into \eqref{eq:conservative-main-thm-derivation0}, we have 
{\small
\begin{equation}
    \begin{split}
        &\sum_{t=1}^K V^{\pi^e}-V^{\pi^{f^k}}\leq \Gamma_0 + \Gamma_2 + \Gamma_1 + \Gamma_3 \leq K_2\paren{V^{\pi^e}(s)-V^{\mr{ref}}(s)} + K_1C_{\pi^e}^\mr{ref}\sqrt{H\cdot \Deltaoff}\\
        +& H\paren{\sqrt{2d\log\paren{1+\frac{K_1}{ d}}\cdot(\Deltaon+B_X^2 B_W^2)\cdot K_1}+\sqrt{2d_\mr{ref}\log\paren{1+\frac{K_2}{d_\mr{ref}}}\cdot(\Deltaon+B_X^2 B_W^2)\cdot K_2}}
    \end{split}
\end{equation}
}

Plugging in the values of $\Deltaon,\Deltaoff$ from \eqref{eq:delta-off} and \eqref{eq:delta-on}, and using the subadditivity of the square root function, we have 
{\small
\begin{equation}
    \begin{split}
        &\sum_{k=1}^K V^{\pi^e}-V^{\pi^{f^k}}\\
        \leq&\;K_2\paren{V^{\pi^e}(s)-V^{\mr{ref}}(s)}+
        16\Vmax  C_{\pi^e}^\mr{ref}K_1\sqrt{\frac{H}{\moff}\log\paren{\frac{2HK_1|\mc F|}{\delta}}}\\
        \;&+ \paren{16V_{\max}\sqrt{\frac{1}{\mon}\log\paren{\frac{2HK_1|\mc F|}{\delta}}}+B_XB_W}\cdot H\sqrt{2dK_1\log\paren{1+\frac{K_1}{d}}}\\
        \;&+ \paren{16V_{\max}\sqrt{\frac{1}{\mon}\log\paren{\frac{2HK_2|\mc F|}{\delta}}}+B_XB_W}\cdot H\sqrt{2d_\mr{ref}K_2\log\paren{1+\frac{K_2}{d_\mr{ref}}}}.
    \end{split}
\end{equation}
}
Setting $\moff = K,\mon=1$ in the above equation completes the proof, we have
\begin{equation}
    \begin{split}
        &\sum_{k=1}^KV^{\pi^e}-V^{\pi^k}\\
        \leq&\; \wt{O}\paren{C_{\pi^e}^\mr{ref}\sqrt{HK_1\log\paren{|\mc F|/\delta}}} + \wt{O}\paren{H \sqrt{dK_1\log\paren{|\mc F|/\delta}}}\\
        &+K_2\paren{V^{\pi^e}(s)-V^{\mr{ref}}(s)}+\wt{O}\paren{H \sqrt{d_\mr{ref}K_2\log\paren{|\mc F|/\delta}}}\\
        \leq &\begin{cases}
            \wt{O}\paren{C^\mr{ref}_{\pi^e}H\sqrt{dK_1\log\paren{|\mc F|/\delta}}}&\text{ if } K_1\gg K_2,\\
            \wt{O}\paren{K_2\paren{V^{\pi^e}-V^\mr{ref}}+H\sqrt{d_\mr{ref}K_2\log\paren{|\mc F|/\delta}}}&\text{ otherwise.} 
        \end{cases}\\
        \leq&\; \wt{O}\paren{\min\Brac{C_{\pi^e}^\mr{ref}H\sqrt{dK\log\paren{|\mc F|/\delta}},\;K\paren{V^{\pi^e}-V^\mr{ref}}+H\sqrt{d_\mr{ref}K\log\paren{|\mc F|/\delta}}}},
    \end{split} 
\end{equation}
where the last inequality holds because $K_1,K_2\leq K$, which completes the proof.
\end{proof}


\section{Key Results of HyQ~\citep{song2023hybrid}}
In this section, we restate the major theoretical results of Hy-Q~\citep{song2023hybrid}. 
% \begin{algorithm}[ht]

%   \caption{Hybrid Q-learning using offline and online data~\citep{song2023hybrid}}\label{algo:HybridQ}
%   \begin{algorithmic}[1]
%     \State \textbf{Input:} Value function class: $\mc F$, \# iterations: $K$, offline dataset $\mc D_h^\nu$ of size $\moff$ for $h\in[H-1]$.
%     \State Initialize $f_h^1(s,a)=0,\forall (s,a)$.
%     \For{$k=1,\dots,K$}
%     \State Let $\pi^k$ be the greedy policy w.r.t. $f^k$ \Comment{I.e., $\pi^k_h(s)=\arg\max_af_h^k(s,a)$.}
%     \State For each $h$, collect $\mon$ online tuples $\mc D_h^k\sim d_h^{\pi^k}$ \Comment{online data collection}
%     \State Set $f_H^{k+1}(s,a)=0,\forall (s,a)$.
%     \For{$h=H-1,\dots 0$} \Comment{FQI with offline and online data}
%         \State Estimate $f_h^{k+1}$ using least squares regression on the aggregated data:
%         {\small
%         \begin{equation}
%             f_h^{k+1}\leftarrow\underset{f\in \mc F_h}{\arg\min}\Brac{\EmpiricalOffline\brac{f(s,a)-r-\max_{a^\prime}f_{h+1}^{k+1}(s^\prime,a^\prime)}^2 + \sum_{\tau=1}^k\EmpiricalOnline\brac{f(s,a)-r-\max_{a^\prime}f_{h+1}^{k+1}(s^\prime,a^\prime)}^2}  
%         \end{equation}}
%     \EndFor
%     \EndFor
%     \State \textbf{Output:} $\pi^K$
%   \end{algorithmic}
% \end{algorithm}
\vspace{-0.2cm}
\subsection{Assumptions}
\label{subsec:HyQ-assumptions}
\vspace{-0.2cm}

\begin{assumption}[Realizability and Bellman completeness]
\label{assump:realizability-completeness}
For any $h$, we have $Q^\star_h\in \mc F_h$, and additionally, for any $f_{h+1}\in \mc F_{h+1}$, we have $\mc T f_{h+1}\in \mc F_h$.
\end{assumption}



\begin{definition}[Bellman error transfer coefficient]
\label{def:bellman-error-coeff-hy-q}
For any policy $\pi$, we define the transfer coefficient as 
\begin{equation}
    C_\pi:=\max\Brac{0,\max_{f\in \mc F}\frac{\sum_{h=0}^{H-1}\bb E_{s,a\sim d_h^\pi}[\mc T f_{h+1}(s,a)-f_h(s,a)]}{\sqrt{\sum_{h=0}^{H-1}\bb E_{s,a\sim \nu_h}(\mc T f_{h+1}(s,a)-f_h(s,a))^2}}}.
\end{equation}
\end{definition}

\subsection{Main Theorem of Hy-Q}
\label{appendix:main-thm-HyQ}
\begin{theorem}[Theorem 1 of~\citet{song2023hybrid}]
\label{thm:HyQ-regret-decomposition}
    Fix $\delta\in(0,1),\moff=K,\mon=1$, and suppose that the underlying MDP admits Bilinear rank $d$ (Definition~\ref{def:bilinear-model}), and the function class $\mc F$ satisfies Assumption~\ref{assump:realizability-completeness}. Then with probability at least $1-\delta$, HyQ obtains the following bound on cumulative suboptimality w.r.t. any comparator policy $\pi^e$:
    \begin{equation}
        \begin{split}
        \regret(K)=\;\wt{O}\paren{\max\{C_{\pi^e},1\}\Vmax B_XB_W\sqrt{dH^2K\cdot\log(|\mc F|/\delta)}}.
        \end{split}
    \end{equation}
\end{theorem}


\subsection{Key Lemmas}
\subsubsection{Least Squares Generalization and Applications}
% \begin{lemma}[Lemma 3 of~\citet{song2023hybrid}, least squares generalization bound]
%     \label{lemma:least-squares-generalization}
%     Let $R>0,\delta\in(0,1)$, we consider a sequential function estimation setting, with an instance space $\mc X$ and target space $\mc Y$. Let $\mc H:\mc S\mapsto[-R,R]$ be a class of real valued functions. Let $\mc D=\{(x_1,y_1),\dots,(x_T,y_T)\}$ be a dataset of $T$ points where $x_t\sim\rho_t:=\rho_t(x_{1:t-1},y_{1:t-1})$, and $y_t$ is sampled via the conditional probability $p(\cdot|x_t)$:
%     \begin{equation}
%         y_t~\sim p(\cdot |x_t):=h^\star(x_t)+\eps_t,
%     \end{equation}
%     where the function $h^\star$ satisfies approximate realizablity:
%     \begin{equation}
%         \inf_{h\in \mc H}\frac{1}{T}\sum_{t=1}^T\bb E_{x\sim\rho_t}\brac{(h^\star(x)-h(x))^2}\leq \gamma,
%     \end{equation}
%     and $\{\eps_i\}_{i=1}^n$ are independent random variables such that $\bb E[y_t|x_t]=h^\star(x_t)$. Additionally, suppose that $\max_t|y_t|\leq R$ and $\max_x|h^\star(x)|\leq R$. Then the least square solution $\wh{h}\leftarrow\arg\min_{h\in \mc H}\sum_{t=1}^T(h(x_t)-y_t)^2$ satisfies with probability at least $1-\delta$,
%     \begin{equation}
%         \label{eq:least-square-error-bound}
%         \sum_{t=1}^T\bb E_{x\sim\rho_t}\brac{\paren{\wh{h}(x)-h^\star(x)}^2}\leq 3\gamma T+256R^2\log(2|\mc H|/\delta).
%     \end{equation}
% \end{lemma}

\begin{lemma}[Lemma 7 of~\citet{song2023hybrid}, Online and Offline Bellman Error Bound for FQI]
    \label{lemma:bellman-error-bound}
    Let $\delta\in(0,1)$ and $\forall h\in[H-1],k\in[K]$, let $f_h^{k+1}$ be the estimated value function for time step $h$ computed via least square regression using samples in the dataset $\{\mc D^\nu_h,\mc D^1_h,\dots,\mc D^T_h\}$ in \eqref{eq:least-square-our-method} in the iteration $t$ of Algorithm~\ref{algo:CalQL-thm}. Then with probability at least $1-\delta$, for any $h\in[H-1]$ and $k\in[K]$, we have
    \begin{equation}
        \label{eq:delta-off}
        \norm{f_h^{k+1}-\mc Tf_{h+1}^{k+1}}{2,\nu_h}^2\leq\frac{1}{\moff}256V^2_{\max}\log(2HK|\mc F|/\delta)=:\Deltaoff
    \end{equation}
    and
    \begin{equation}
        \label{eq:delta-on}
        \sum_{\tau=1}^k\norm{f_h^{k+1}-\mc Tf_{h+1}^{k+1}}{2,\mu^\tau_h}^2\leq\frac{1}{\mon}256V^2_{\max}\log(2HK|\mc F|/\delta)=:\Deltaon,
    \end{equation}
    where $\nu_h$ denotes the offline data distribution at time $h$, and the distribution $\mu_h^\tau\in\Delta(s,a)$ is defined such that $s,a\sim d^{\pi^\tau}_h$.
\end{lemma}
% \begin{proof}
%     \label{proof:bellman-error-bound}
%     Fix $t\in[T],h\in[H-1],f_{h+1}^{t+1}\in \mc F_{h+1}$ and consider the regression problem
%     \begin{equation*}
%         f_{h}^{h+1}\leftarrow\underset{f\in \mc F_h}{\arg\min}\Brac{\EmpiricalOffline\paren{f(s,a)-r-\max_{a^\prime}f_{h+1}^{t+1}(s^\prime,a^\prime)}^2+\sum_{\tau=1}^t\EmpiricalOnline\paren{f(s,a)-r-\max_{a^\prime}f^{t+1}_{h+1}(s^\prime,a^\prime)}^2}.
%     \end{equation*}
%     The above equation can be considered as a regression problem on the dataset $\mc D$ consisting of $n = \moff + t\cdot \mon$ samples $\{(x_i,y_i)\}_{i\leq n}$ where 
%     \begin{equation}
%         x_i = (s_h^i,a_h^i),\;\text{and}\; y^i=r^i+\max_a f_{h+1}^{t+1}(s_{h+1},a).
%     \end{equation}
%     Define $\mc D$ such that the first $\moff$ samples $\{(x_i,y_i)\}_{i\leq \moff}=\mc D_h^\nu$, the next $\mon$ samples ${\{(x_i,y_i)\}_{i=\moff+1}^{\moff+\mon}=\mc D_h^1}$, and so on: $\{(x_i,y_i)\}_{i=\moff+(\tau-1)\mon+1}^{\moff+\tau\mon}=\mc D_h^\tau$. Note that: (a) for any sample 
%     \begin{equation*}
%         \paren{x=(s_h,a_h),y=r+\max_af_{h+1}^{t+1}(s_{h+1},a)} \in \mc D,
%     \end{equation*}
%     by definition, we have
%     \begin{equation}
%         \bb E[y|x] = \bb E_{s_{h+1\sim P(s_h,a_h),r\sim R(s_h,a_h)}}\brac{r+\max_a f_{h+1}^{t+1}(s_{h+1},a)} = \mc T f_{h+1}^{t+1}(s_h,a_h)\leq g(s_h,a_h),
%     \end{equation}
%     where the last line holds since the Bellman completeness assumption (Assumption~\ref{assump:realizability-completeness}) implies existence of such a function $g$; (b) for any sample, $|y|\leq \Vmax$ and $f(s,a)\leq \Vmax$ hold $\forall (s,a)$; (c) the construction of $\mc D$ implies that for each iteration $t$ the sample $(x_t,y_t)$ is generated from the following procedure: $x_t$ is sampled from the data generation scheme $\mc D^t(x_{1:t-1,y_{1:t-1}})$ and $y_t$ is sampled from some conditional probability distribution $p(\cdot|x_t)$ as defined in Lemma~\ref{lemma:least-squares-generalization}; (d) the samples in $\mc D_h^\mu$ are drawn from the offline distribution $\nu_h$, and the samples in $\mc D_h^\tau$ are  drawn such that $s_h\sim d_h^{\pi^t}$ and $a_h\sim \pi^{f^t}(s_h)$. Thus, using Lemma~\ref{lemma:least-squares-generalization}, the least square regression regression solution $f_h^{t+1}$ satisfies
%     \begin{equation}
%         \sum_{i=1}^n \bb E\brac{\paren{f_h^{t+1}(s^i,a^i)-\mc T f_{h+1}^{t+1}(s^i,a^i)}^2|\mc D_i}\leq 256\Vmax^2\log(2|\mc F|/\delta),
%     \end{equation}
%     with probability at least $1-\delta$. Note that the above equation does not contain the $2n\gamma$ term in \eqref{eq:least-square-error-bound} because the Bellman completeness assumption (Assumption~\ref{assump:realizability-completeness}) implies the approximate realizability term $\gamma =0$. Now we apply a union bound for all $h,t$, substituting $\delta = \delta/HT$, the above inequality implies that, for a fixed $h,t$,
%     \begin{equation}
%         \sum_{i=1}^n \bb E\brac{\paren{f_h^{t+1}(s^i,a^i)-\mc T f_{h+1}^{t+1}(s^i,a^i)}^2|\mc D_i}>256\Vmax^2\log(2HT|\mc F|/\delta),
%     \end{equation}
%     holds with probability at most $\delta/HT$. Hence, with probability at least $1-\delta$,
%     \begin{equation}
%         \sum_{i=1}^n \bb E\brac{\paren{f_h^{t+1}(s^i,a^i)-\mc T f_{h+1}^{t+1}(s^i,a^i)}^2|\mc D_i}\leq256\Vmax^2\log(2HT|\mc F|/\delta),
%     \end{equation}
%     holds $\forall h,t$, which directly leads to the desired results.
% \end{proof}

\subsubsection{Bounding Offline Suboptimality via Performance Difference Lemma}
\begin{lemma}[Lemma 5 of~\citet{song2023hybrid}, performance difference lemma of w.r.t. $\pi^e$] 
\label{lemma:perf-diff-comparator}
Let $\pi^e=(\pi_0^e,\dots,\pi_{H-1}^e)$ be a comparator policy and consider any value function $f=(f_0,\dots,f_{H-1})$, where $f_h:\mc S\times \mc A\mapsto \bb R$. Then we have
\begin{equation}
    \bb E_{s\sim d_0}\brac{V_0^{\pi^e}(s)-\max_a f_0(s,a)}\leq \sum_{i=1}^{H-1}\bb E_{s,a\sim d_i^{\pi^e}}\brac{\mc Tf_{i+1}(s,a)-f_i(s,a)},
\end{equation}
where we define $f_H(s,a)=0,\forall (s,a)$.
\end{lemma}
% \begin{proof}
%     \label{proof:perf-diff-comparator}
%     This proof is similar to Lemma~\ref{lemma:perf-diff-online}. We start with the fact that $\max_af(s,a)\geq f(s,a^\prime),\forall a^\prime$, hence
%     \begin{equation}
%         \begin{split}
%             &\bb E_{s\sim d_0}\brac{V_0^{\pi^e}(s)-\max_af_0(s,a)}\leq \bb E_{s,a\sim d^{\pi^e}_0}\brac{Q_0^{\pi^e}(s,a)-f_0(s,a)}\\
%             =\;& \bb E_{s,a\sim d^{\pi^e}_0}\brac{Q_0^{\pi^e}(s,a) -\mc T f_1(s,a) + \mc T f_1(s,a) -f_0(s,a)}\\
%             =\;&\bb E_{s,a\sim d^{\pi^e}_0}\brac{\bb E_{s^\prime\sim\mc P(s,a)}V_1^{\pi^e}(s^\prime)-\max_a f_1(s,a)}+\bb E_{s,a\sim d^{\pi^e}_0}\brac{\mc T f_1(s,a)-f_0(s,a)}\\
%              =\;&\bb E_{s\sim d^{\pi^e}_1}\brac{V_1^{\pi^e}(s)-\max_a f_1(s,a)}+\bb E_{s,a\sim d^{\pi^e}_0}\brac{\mc T f_1(s,a)-f_0(s,a)}.
%         \end{split}
%     \end{equation}
%     Recursively apply the above inequality, we have
%     \begin{equation}
%         \bb E_{s\sim d_0}\brac{V_0^{\pi^e}(s)-\max_af_0(s,a)}\leq \bb E_{s\sim d^{\pi^e}_H}\brac{V_H^{\pi^e}(s)-\max_a f_H(s,a)}+\sum_{h=0}^{H-1}\bb E_{s,a\sim d^{\pi^e}_h}\brac{\mc T f_{h+1}(s,a)-f_h(s,a)},
%     \end{equation}
%     plugging in $f_H(s,a)=0$ and $V_H^{\pi^f}=0$, we obtain
%     \begin{equation}
%         \bb E_{s\sim d_0}\brac{V_0^{\pi^e}(s)-\max_af_0(s,a)}\leq \sum_{h=0}^{H-1}\bb E_{s,a\sim d^{\pi^e}_h}\brac{\mc T f_{h+1}(s,a)-f_h(s,a)},
%     \end{equation}
%     which completes the proof.
% \end{proof}


\subsubsection{Bounding Online Suboptimality via Performance Difference Lemma}
\begin{lemma}[Lemma 4 of~\citet{song2023hybrid}, performance difference lemma]
    \label{lemma:perf-diff-online}
    For any function $f=(f_0,\dots,f_{H-1})$ where $f_h:\mc S\times \mc A\mapsto \bb R$ and $h\in[H-1]$, we have
    \begin{equation}
        \bb E_{s\sim d_0}\brac{\max_a f_0(s,a)-V_0^{\pi^f}(s)}\leq \sum_{h=0}^{H-1}\abs{\bb E_{s,a\sim d_h^{\pi^f}}\brac{f_h(s,a)-\mc Tf_{h+1}(s,a)}},
    \end{equation}
    where we define $f_H(s,a)=0,\forall s,a$.
\end{lemma}

\begin{lemma}[Lemma 8 of \citet{song2023hybrid}, upper bounding bilinear class]
    \label{lemma:bilinear-upper-bound}
    For any $k\geq2$ and $h\in [H-1]$, we have
    \begin{equation}
        \abs{\innerprod{W_h(f^k)}{X_h(f^k)}}\leq \norm{X_h(f^k)}{\mb \Sigma^{-1}_{k-1;h}}\sqrt{\sum_{i=1}^{k-1}\bb E_{s,a\sim d_h^{f^i}}\brac{\paren{f_h^k-\mc T f_{h+1}^k}^2}+\lambda B_W^2},
    \end{equation}
    where $\mb \Sigma_{k-1;h}$ is defined as \eqref{eq:covariance-matrix} and we use $d_h^{f^k}$ to denote $d_h^{\pi^{f^k}}$.
\end{lemma}
% \begin{proof}
%     \label{proof:bilinear-upper-bound}
%     By Cauchy-Schwarz inequality, we have
%     \begin{equation}
%         \begin{split}
%             &\abs{\innerprod{W_h(f^t)}{X_h(f^t)}}\leq\norm{X_h(f^t)}{\mb \Sigma^{-1}_{t-1;h}}\norm{W_h(f^t)}{\mb \Sigma_{t-1;h}}\\
%             =&\;\norm{X_h(f^t)}{\mb \Sigma^{-1}_{t-1;h}}\sqrt{W_h(f^t)^\top \mb \Sigma_{t-1;h}W_h(f^t)}\\
%             =&\;\norm{X_h(f^t)}{\mb \Sigma^{-1}_{t-1;h}}\sqrt{W_h(f^t)^\top \paren{\sum_{i=1}^{t-1}X_h(f^i)X_h(f^i)^\top+\lambda \mb I}W_h(f^t)}\\
%             =& \; \norm{X_h(f^t)}{\mb \Sigma^{-1}_{t-1;h}}\sqrt{\sum_{i=1}^{t-1}\abs{\innerprod{W_h(f^t)}{X_h(f^i)}}^2+\lambda\norm{W_h(f^t)}{2}^2}\\
%             \overset{(i)}{\leq} & \; \norm{X_h(f^t)}{\mb \Sigma^{-1}_{t-1;h}}\sqrt{\sum_{i=1}^{t-1}\abs{\innerprod{W_h(f^t)}{X_h(f^i)}}^2+\lambda B_W^2}\\
%             \overset{(ii)}{\leq} & \; \norm{X_h(f^t)}{\mb \Sigma^{-1}_{t-1;h}}\sqrt{\sum_{i=1}^{t-1}\bb E_{s,a\sim d_h^{f^i}}\brac{\paren{f_h^t-\mc T f_{h+1}^t}^2}+\lambda B_W^2},
%         \end{split}
%     \end{equation}
%     where inequality $(i)$ holds by the Bilinear class assumption (Assumption~\ref{assump:realizability-completeness}) and inequality $(ii)$ holds by \eqref{eq:bilinear-model} from Definition~\ref{def:bilinear-model}:
%     \begin{equation}
%         \abs{\innerprod{W_h(f^t)}{X_h(f^i)}}^2=\paren{\bb E_{s,a\sim d_h^{f^i}}\brac{f^t_h-\mc T f^t_{h+1}}}^2\leq \bb E_{s,a\sim d_h^{f^i}}\brac{\paren{f_h^t-\mc T f_{h+1}^t}^2}.
%     \end{equation}
%     The last holds by Jensen's inequality, which, completes the proof.
% \end{proof}
\begin{lemma}[Lemma 6 of~\citet{song2023hybrid}, bounding the inverse covariance norm]
    \label{lemma:bound-inverse-cov-norm}
    Let $X_h(f^1),\dots,X_h(f^K)\in \bb R^d$ be a sequence of vectors with $\norm{X_h(f^k)}{2}\leq B_X<\infty,\forall k\leq K$. Then we have
    \begin{equation}
        \sum_{k=1}^K\norm{X_h(f^k)}{\mb \Sigma^{-1}_{k-1;h}}\leq\sqrt{2dK\log\paren{1+\frac{KB_X^2}{\lambda d}}},
    \end{equation}
    where we define $\mb \Sigma_{k;h}:=\sum_{\tau=1}^kX_h(f^\tau)X_h(f^\tau)^T+\lambda \mb I$ and we assume $\lambda\geq B_X^2$ holds $\forall k\in[K]$.
\end{lemma}


\section{Environment Details}
\label{appendix:env_details}
\subsection{Antmaze} 
The Antmaze navigation tasks aim to control an 8-DoF ant quadruped robot to move from a starting point to a desired goal in a maze. The agent will receive sparse +1/0 rewards depending on whether it reaches the goal or not. We study each method on ``medium'' and ``hard'' (shown in Figure~\ref{fig:envs}) mazes which are difficult to solve, using the following datasets from D4RL~\citep{fu2020d4rl}: \texttt{large-diverse}, \texttt{large-play}, \texttt{medium-diverse}, and \texttt{medium-play}. The difference between ``diverse'' and ``play'' datasets is the optimality of the trajectories they contain. The ``diverse'' datasets contain the trajectories commanded to a random goal from random starting points, while the ``play'' datasets contain the trajectories commanded to specific locations which are not necessarily the goal. We used an episode length of 1000 for each task. For \methodname, CQL, and IQL, we pre-trained the agent using the offline dataset for 1M steps. We then trained online fine-tuning for 1M environment steps for each method.


\subsection{Franka Kitchen} 
The Franka Kitchen domain require controlling a 9-DoF Franka robot to arrange a kitchen environment into a desired configuration. The configuration is decomposed into 4 subtasks, and the agent will receive rewards of $0$, $+1$, $+2$, $+3$, or $+4$ depending on how many subtasks it has managed to solve. To solve the whole task and reach the desired configuration, it is important to learn not only how to solve each subtask, but also to figure out the correct order to solve. We study this domain using datasets with three different optimalities: \texttt{kitchen-complete}, \texttt{kitchen-partial}, and \texttt{kitchen-mixed}. The ``complete'' dataset contains the trajectories of the robot performing the whole task completely. The ``partial'' dataset partially contains some complete demonstrations, while others are incomplete demonstrations solving the subtasks. The ``mixed'' dataset only contains incomplete data without any complete demonstrations, which is hard and requires the highest degree of stitching and generalization ability. We used an episode length of 1000 for each task. For \methodname, CQL, and IQL, we pre-trained the agent using the offline dataset for 500K steps. We then performed online fine-tuning for 1.25M environment steps for each method.

\subsection{Adroit} 
The Adroit domain involves controlling a 24-DoF shadow hand robot. There are 3 tasks we consider in this domain: \texttt{pen-binary}, \texttt{relocate-binary}, \texttt{relocate-binary}. These tasks comprise a limited set of narrow human expert data distributions ($\sim 25$) with additional trajectories collected by a behavior-cloned policy. We truncated each trajectory and used the positive segments (terminate when the positive reward signal is found) for all methods. This domain has a very narrow dataset distribution and a large action space. In addition, learning in this domain is difficult due to the sparse reward, which leads to exploration challenges. We utilized a variant of the dataset used in prior work \cite{AWAC} to have a standard comparison with SOTA offline fine-tuning experiments that consider this domain. For the offline learning phase, we pre-trained the agent for 20K steps. We then performed online fine-tuning for 300K environment steps for the \texttt{pen-binary} task, and 1M environment steps for the \texttt{door-binary} and \texttt{relocate-binary} tasks. The episode length is 100, 200, and 200 for \texttt{pen-binary}, \texttt{door-binary}, and \texttt{relocate-binary} respectively.

\subsection{Visual Manipulation Domain} 
The Visual Manipulation domain consists of a pick-and-place task. This task is a multitask formulation explored in the work, Pre-training for Robots (PTR) \cite{2022arXiv221005178K}. Here each task is defined as placing an object in a bin. A distractor object was present in the scene as an adversarial object which the agent had to avoid picking. There were 10 unique objects and no overlap between the task objects and the interfering/distractor objects. The episode length is 40. For the offline phase, we pre-trained the policy with offline data for 50K steps. We then performed online fine-tuning for 100K environment steps for each method, taking 5 gradient steps per environment step.

\section{Experiment Details}
\label{app:calql_hyperparam}

\subsection{Normalized Scores}
\label{appendix:normalized_score}

The \texttt{visual-manipulation}, \texttt{adroit}, and \texttt{antmaze} domains are all goal-oriented, sparse reward tasks. In these domains, we computed the normalized metric as simply the goal achieved rate for each method. For example, in the visual manipulation environment, if the object was placed successfully in the bin, a $+1$ reward was given to the agent and the task is completed. Similarly, for the \texttt{door-binary} task in the adroit tasks, we considered the success rate of opening the door. For the \texttt{kitchen} task, the task is to solve a series of 4 sub-tasks that need to be solved in an iterative manner. The normalized score is computed simply as $\frac{\# \text{tasks solved}}{\text{total tasks}}$.

\subsection{Mixing Ratio Hyperparameter}
\label{appendix:mixing_ratio_overview}
In this work, we explore the mixing ratio parameter $m$, which is used during the online fine-tuning phase. The mixing ratio is either a value in the range $\left[0, 1\right]$ or the value -1. If this mixing ratio is within $\left[0, 1\right]$, it represents what percentage of offline and online data is seen in each batch when fine-tuning. For example, if the mixing ratio $m=0.25$, that means for each batch we sample 25\% from the offline data and 75\% from online data. Instead, if the mixing ratio is -1, the buffers are appended to each other and sampled uniformly.

\subsection{Details and Hyperparameters for CQL and \methodname\ }
We list the hyperparameters for CQL and \methodname\ in Table~\ref{table:hyper_all_cql}. We utilized a variant of Bellman backup that computes the target value by performing a maximization over target values computed for $k$ actions sampled from the policy at the next state, where we used $k=4$ in visual pick and place domain and $k=10$ in others. In the Antmaze domain, we used the dual version of CQL~\citep{kumar2020conservative} and conducted ablations over the value of the threshold of the CQL regularizer $\mathcal{R}(\theta)$ (target action gap) instead of $\alpha$. In the visual-manipulation domain which is not presented in the original paper, we swept over the alpha values of $\alpha=0.5, 1, 5, 10$, and utilized separate $\alpha$ values for offline ($\alpha=5$) and online ($\alpha=0.5$) phases for the final results. We built our code upon the CQL implementation from \url{https://github.com/young-geng/JaxCQL}~\cite{geng2022jaxcql}. We used a single NVIDIA TITAN RTX chip to run each of our experiments.

\subsection{Details and Hyperparameters for IQL}
We list the hyperparameters for IQL in Table~\ref{table:hyper_all_iql}. To conduct our experiments, we used the official implementation of IQL provided by the authors~\citep{kostrikov2021offlineb}, and primarily followed their recommended parameters, which they previously ablated over in their work. In the visual-manipulation domain which is not presented in the original paper, we performed a parameter sweep over expectile $\tau = 0.5, 0.6, 0.7, 0.8, 0.9, 0.95, 0.99$ and temperature $\beta = 1, 3, 5, 10, 25, 50$ and selected the best-performing values of $\tau = 0.7$ and $\beta = 10$ for our final results. In addition, as the second best-performing method in the visual-manipulation domain, we also attempted to use separate $\beta$ values for IQL, for a fair comparison with CQL and \methodname. However, we found that it has little to no effect, as shown in Figure~\ref{fig:beta_ablation}. 

\subsection{Details and Hyperparameters for AWAC and ODT}
We used the JAX implementation of AWAC from \url{https://github.com/ikostrikov/jaxrl}~\cite{jaxrl}. We primarily followed the author's recommended parameters, where we used the Lagrange multiplier $\lambda = 1.0$ for the Antmaze and Franka Kitchen domains, and $\lambda = 0.3$ for the Adroit domain. In the visual-manipulation domain, we performed a parameter sweep over $\lambda = 0.1, 0.3, 1, 3, 10$ and selected the best-performing value of $\lambda = 1$ for our final results. For ODT, we used the author's official implementation from \url{https://github.com/facebookresearch/online-dt}, with the author's recommended parameters they used in the Antmaze domain. In addition, in support of our result of AWAC and ODT (as shown in Table~\ref{tab:performance}), the poor performance of Decision Transformers and AWAC in the Antmaze domain can also be observed in Table 1 and Table 2 of the IQL paper~\citep{kostrikov2021offlineb}.

\subsection{Details and Hyperparameters for SAC, SAC + Offline Data, and CQL + SAC}
We used the standard hyperparameters for SAC as derived from the original implementation in~\cite{sac}. We used the same other hyperparameters as CQL and \methodname. We used automatic entropy tuning for the policy and critic entropy terms, with a target entropy of the negative action dimension. For SAC, the agent is only trained with the online explored data. For SAC + Offline Data, the offline data and online explored data is combined together and sampled uniformly. For Hybrid RL, we use the same mixing ratio used for CQL and \methodname\ presented in Table~\ref{table:hyper_all_cql}. For CQL + SAC, we first pre-train with CQL and then run online fine-tuning using both offline and online data, also using the same mixing ratio presented in Table~\ref{table:hyper_all_cql}. 

\begin{table}[H]
\centering
\caption{CQL, \methodname\ Hyperparameters}
\label{table:hyper_all_cql} 
\scalebox{0.80}{
\begin{tabular}{c|c|c|c|c} % p{3.cm} p{2.cm} p{2.cm}  p{2.cm}  p{2.5cm}}
  \toprule
    \textbf{Hyperparameters} & \textbf{Adroit} & \textbf{Kitchen}  & \textbf{Antmaze} & \textbf{Manipulation}\\         
  \midrule
    $\alpha$ & 1  & 5  & - & 5 (online: 0.5) \\
    target action gap &  -  & - & 0.8 & - \\
    mixing ratio &  -1, 0.25, \textbf{0.5} & -1, \textbf{0.25}, 0.5 & 0.5 & 0.2, \textbf{0.5}, 0.7, 0.9 \\
   % policy architecture &  512-512  & 512-512-512 & 256-256 & ConvNet \\
   % policy learning rate &  1e-4  & 1e-4 & 1e-4 & 1e-4\\
   % critic architecture &  512-512-512  & 512-512-512 & 256-256-256-256 & ConvNet\\
   % critic learning rate &  3e-4  & 3e-4 & 3e-4 & 3e-4\\
   %  reward scale &  10  & 1 & 10 & 11 \\
   %  reward bias &  5  & 1  &  -5 & -1 \\
   %  batch size &  256  & 256 & 256 & 64 \\
    
  \bottomrule
\end{tabular} }
\end{table}


\begin{table}[H]
\centering
\vspace{-0.7cm}
\caption{IQL Hyperparameters}
\label{table:hyper_all_iql} 
\scalebox{0.80}{
\begin{tabular}{c|c|c|c|c} % {p{3.cm} p{2.cm} p{2.cm}  p{2.cm}  p{2.5cm}}
  \toprule
    \textbf{Hyperparameters} & \textbf{Adroit} & \textbf{Kitchen}  & \textbf{Antmaze} & \textbf{Manipulation}\\         
  \midrule
    expectile $\tau$ & 0.8  &  0.7 & 0.9 & 0.7 \\
    temperature $\beta$ & 3  & 0.5  & 10  & 10  \\
    mixing ratio &  -1, \textbf{0.2}, 0.5 & -1, \textbf{0.25}, 0.5   & 0.5 &  0.2, \textbf{0.5}, 0.7, 0.9 \\
   % policy architecture &  256-256  & 256-256 & 256-256 & ConvNet \\
   % policy learning rate &  3e-4  & 3e-4 & 3e-4 & 1e-4\\
   % critic architecture &   256-256 & 256-256 & 256-256 & ConvNet\\
   % critic learning rate &  3e-4  & 3e-4 & 3e-4 & 3e-4\\
   %  batch size &  256  & 256 & 256 & 64 \\
  \bottomrule
\end{tabular} }
\end{table}




\section{Discussion on Limitations of Existing Fine-Tuning Methods}
\label{appendix:more_discussion_on_finetuning}

\begin{figure}[h]
\begin{center}
\centerline{\includegraphics[width=0.45\textwidth]{chapters/cal_ql/figs-sample/iql-beta.pdf}}

\caption{\label{fig:beta_ablation}\footnotesize{\textbf{Abalation on IQL's online temperature values}: The change in the temperature $\beta$ used in online fine-tuning phase has little to no effect on the sample efficiency.}}
\end{center}
\vspace{-0.8cm}

\end{figure}

\begin{figure}[h]

\begin{center}
\centerline{\includegraphics[width=0.8\textwidth]{chapters/cal_ql/figs-sample/iql-analysis-final.png}}

\caption{\label{fig:cql_iql_finetune_app}\footnotesize{\textbf{IQL and CQL:} Step 0 on the x-axis is the performance after offline pre-training. Observe while CQL suffers from initial policy unlearning, IQL improves slower throughout fine-tuning.}}

\end{center}
\vspace{-0.9cm}

\end{figure}

In this section, we aim to highlight some potential reasons behind the slow improvement of other methods in our empirical analysis experiment in Section~\ref{sec:calql_empirical_analysis}, and specifically, we use IQL for the analysis. We first swept over the temperature $\beta$ values used in the online fine-tuning phase for IQL, which controls the constraint on how closely the learned policy should match the behavior policy. As shown in Figure~\ref{fig:beta_ablation}, the change in the temperature $\beta$ has little to no effect on the sample efficiency. Another natural hypothesis is that IQL improves slowly because we are not making enough updates per unit of data collected by the environment. To investigate this, we ran IQL with \textbf{(a)} five times as many gradient steps per step of data collection ($\text{UTD}=5$), and \textbf{(b)} with a more aggressive policy update. Observe in Figure~\ref{fig:cql_iql_finetune_app} that \textbf{(a)} does not improve the asymptotic performance of IQL, although it does improve CQL meaning that there is room for improvement on this task by making more gradient updates. Observe in Figure~\ref{fig:cql_iql_finetune_app} that \textbf{(b)} often induces policy unlearning, similar to the failure mode in CQL. These two observations together indicate that a policy constraint approach can slow down learning asymptotically, and we cannot increase the speed by making more aggressive updates as this causes the policy to find erroneously optimistic out-of-distribution actions, and unlearn the policy learned from offline data. 


\section{Impact of Estimation Errors in the Reference Value Function}
\label{app:nn_value_function}

\begin{figure}[t]
\vspace{-0.3cm}
\begin{center}
\includegraphics[width=0.35\linewidth]{chapters/cal_ql/figs-sample/kitchen-analysis.png}
\vspace{-0.2cm}
\caption{\footnotesize{The performance of \methodname\ using a neural net approximator for the reference value function is comparable to using the Monte-Carlo return.}}
\label{fig:kitchen-regress}
\vspace{-0.5cm}
\end{center}
\end{figure}

In our experiments, we compute the reference value functions using Monte-Carlo return estimates. However, this may not be available in all tasks. How does \methodname\ behave when reference value functions must be estimated using the offline dataset itself? To answer this, we ran an experiment on the \texttt{kitchen} domain, where instead of using an estimate for $Q^\mu$ based on the Monte-Carlo return, we train a neural network function approximator $Q^\mu_\theta$ to approximate $Q^\mu$ via supervised regression on to Monte-Carlo return, which is then utilized by \methodname. 
Observe in Figure \ref{fig:kitchen-regress}, that the performance of \methodname\ largely remains unaltered. This implies as long as we can obtain a reasonable function approximator to estimate the Q-function of the reference policy (in this case, the behavior policy), errors in the reference Q-function do not affect the performance of \methodname\ significantly.

\vspace{-0.2cm}
\section{Initial Unlearning of CQL During Online Fine-Tuning}
\label{app:cql_dip_zoom_in}
\vspace{-0.2cm}
\begin{figure}[h]
\vspace{-0.5cm}
\begin{center}    
{\includegraphics[clip,width=1.0\linewidth]{chapters/cal_ql/figs-sample/kitchen-cog-zoom-in.pdf}} {\includegraphics[clip,width=0.75\linewidth]{chapters/cal_ql/figs-sample/adroit-zoom-in.pdf}}
\end{center}
\vspace{-0.35cm}
\caption{\label{fig:dip_zoom} \footnotesize{While CQL experiences initial unlearning, \methodname\ effectively mitigates it and quickly recovers its performance.}}
\vspace{-0.4cm}
\end{figure}
In this section, we show the learning curves of CQL and \methodname\ from Figure~\ref{fig:all_tasks} and zoom in on the x-axis to provide a clearer visualization of CQL's initial unlearning in the Franka Kitchen, Adroit, and the visual-manipulation domains. As depicted in Figure~\ref{fig:dip_zoom}, it is evident across all tasks that while CQL experiences initial unlearning, \methodname\ can effectively mitigate it and quickly recovers its performance. Regarding the Antmaze domain, as we discussed in section~\ref{subsec:diagonistic}, CQL does not exhibit initial unlearning since the default dataset has a high coverage of data. However, we can observe a similar phenomenon if we narrow down the dataset distribution (as shown in Figure~\ref{fig:ant-narrow}).




\iffalse

\section{Discussion of Policy Unlearning in Fine-Tuning}

In this section, we provide another explanation for the performance drop shown in Figure~\ref{fig:cql_q_value} following the tabular setting. Specifically, in Section~\ref{subsec:reward-bias}, we show that running bellman iteration w.r.t. the CQL objective function~\eqref{eqn:cql_training} will converge to another Q-values w.r.t. a different (or biased) reward function $r^\pi_{\alpha,\beta}=r(\bs,\mathbf{a}) - \alpha\brac{ \frac{\pi(\mathbf{a}|\bs)}{\pi_\beta(\mathbf{a}|\bs)} - 1 }$, where $\alpha$ is the regularizer of the CQL objective, and $\pi_\beta$ is the behavior policy. Note that this estimate only becomes unbiased when $\pi=\pi_\beta$.

In the fine-tuning phase, suppose we are using the CQL objective with another behavior policy $\pi_{\beta^\prime}$ and $\alpha^\prime$, then the Bellman operator will converge to a Q-value w.r.t. another (biased) reward function $r^\pi_{\alpha^\prime,\beta^\prime}=r(\bs,\mathbf{a}) - \alpha^\prime\brac{ \frac{\pi(\mathbf{a}|\bs)}{\pi_{\beta^\prime}(\mathbf{a}|\bs)} - 1 }$. In Theorem~\ref{thm:mixing-dip-main}, we provide an if and only if condition under which the Q-values in the fine-tuning phase remain unchanged. But since the condition in Theorem~\ref{thm:mixing-dip-main} is generally hard to satisfy. Hence, it may lead to a performance difference gap in the fine-tuning phase in terms of the Q-values, as observed in Figure~\ref{fig:cql_q_value}.

\subsection{Bellman Consistency Equation of CQL}
 
\label{subsec:reward-bias}


To analyze policy unlearning with conservative methods at the beginning of online fine-tuning, we consider a tabular setting, where we are learning conservative value functions using a generic policy-iteration style offline RL method based on CQL~\citep{kumar2020conservative}. Our goal is to understand the differences in a policy obtained by running one round of policy improvement with and without additional online data. Key to our analysis is the Bellman backup induced by CQL (Equation~\ref{eqn:cql_training}):
\begin{align}
\label{eq:cql-bellman-consistency}
\footnotesize{
    Q^\pi(\bs, \mathbf{a}) = \left(\mathcal{B}^\pi Q^\pi \right)(\bs, \mathbf{a}) - \alpha \left[ \frac{\pi(\mathbf{a}|\bs)}{\pi_\beta(\mathbf{a}|\bs)} - 1 \right].}
\end{align}
By expanding $\mc B^\pi$, Equation~\ref{eq:cql-bellman-consistency} can also be interpreted as running standard Q-iteration in an MDP with a pessimistic reward function, which depends upon the learned policy $\pi$, the behavior policy $\pi_\beta$ induced by the dataset, and the coefficient $\alpha$ from Equation~\ref{eqn:cql_training}:
$r^\pi_{\alpha,\beta}(\bs,\mathbf{a}) = r(\bs,\mathbf{a}) - \alpha\brac{ \frac{\pi(\mathbf{a}|\bs)}{\pi_\beta(\mathbf{a}|\bs)} - 1 }$. This means that once online fine-tuning commences with a new regularizer $\alpha$, and a new behavior policy $\pi_{\beta^\prime}$ induced by online data added to the buffer, the pessimistic reward function $r^\pi_{\alpha,\beta}$ may {\em bias towards} $r^\pi_{\alpha^\prime,\beta^\prime}$. Hence, the policy improvement on the resulting Q-function with the online data may simply not lead to any policy improvement on the ground-truth reward function. We will first provide a condition such that the pessimistic Q-function is invariant to such reward bias during the online fine-tuning, and then show how our approach alleviates the reward bias.


\paragraph{Performance difference during fine-tuning.} 
Consider the fixed points from solving Equation~\ref{eq:cql-bellman-consistency} w.r.t. the biased rewards $r^\pi_{\alpha,\beta}$ and $r^\pi_{\alpha^\prime,\beta^\prime}$, Theorem~\ref{thm:mixing-dip-main} provides a necessary and sufficient condition for the fixed points to be invariant to reward bias.
\begin{theorem}[Invariant Conservative Q Functions]
\label{thm:mixing-dip-main}
Let $Q$ and $Q^\prime$ denote the conservative value function from solving the fixed point Equation~\ref{eq:cql-bellman-consistency}
with regularizers $\alpha,\alpha^\prime$ and behavior policies $\pi_\beta,\pi_{\beta^\prime}$ respectively. Then for a given policy $\pi$, $Q(s,a) = Q^\prime(s,a),\forall s\in \mc S$ if and only if
\begin{equation}
    \label{eq:inv-cond}
    \footnotesize{
    \frac{\alpha}{\pi_\beta(a|s)}-\frac{\alpha^\prime}{\pi_{\beta^\prime}(a|s)}=\frac{\alpha-\alpha^\prime}{\pi(a|s)},\;\forall (s,a)\in \mc S\times \mc A.}
\end{equation}
\end{theorem}
The proof of Theorem~\ref{thm:mixing-dip-main} is provided in Appendix~\ref{appendix:proof:mixing-dip-main}. Theorem~\ref{thm:mixing-dip-main} implies that one shall expect a performance change in the fine-tuning phase, whenever Equation~\ref{eq:inv-cond} becomes invalid. Unfortunately, Equation~\ref{eq:inv-cond} is generally hard to enforce in practice since the new behavior policy $\pi_{\beta^\prime}$ and the updated policy $\pi$ may be intractable, as suggested by the performance dip in Section~\ref{sec:calql_empirical_analysis}.

\paragraph{Preventing unlearning via calibration.}
Since the standard CQL training objective suffers from reward bias, which may continue to deteriorate during fine-tuning. The next corollary provides a condition under which $r^\pi_{\alpha,\beta}(s,a)$ becomes unbiased.
\begin{corollary}
\label{coro:calibration-prevent-dip}
The reward function $r^\pi_{\alpha,\beta}(s,a)$ induced by the CQL training (Equation~\ref{eq:cql-bellman-consistency}) becomes unbiased when $\pi = \pi_\beta$: $r^{\pi_\beta}_{\alpha,\beta}(s,a)=r(s,a),\forall (s,a)\in \mc S\times \mc A.$
\end{corollary}
The proof of Corollary~\ref{coro:calibration-prevent-dip} is straight forward as substituting $\pi = \pi_\beta$ will set $\brac{\frac{\pi(a|s)}{\pi_\beta(a|s)}-1}=0$, which makes Equation~\ref{eq:cql-bellman-consistency} become a Bellman backup w.r.t. the original reward $r(s,a)$. Corollary~\ref{coro:calibration-prevent-dip} implies that the CQL-induced Bellman backup (Equation~\ref{eqn:cql_training}) becomes unbiased when the updating policy equals the behavior policy. Hence, if one aims at using a neural network $Q_\theta$ to approximate the CQL Bellman backup w.r.t. behavior policy $\pi_\beta$, one shall consider using $Q^{\pi_\beta}(s,a)$ to calibrate the bias. This empirically implies that we shall use $Q^{\pi_\beta}(s,a)$ and $Q^{\pi_{\beta^\prime}}(s,a)$ to calibrate the $Q_\theta$ during offline and online phase, as we have presented in Definition~\ref{cond:calibration} and Section~\ref{sec:empirical-method}.

\subsection{Notations}
In this subsection, we provide the notations for deriving the Bellman Consistency Equation of the conservative Bellman Consistency equation (CQL Objective)~\eqref{eq:cql-bellman-consistency}. Our matrix notation follows~\cite{li2020breaking}. 
\begin{itemize}
    \item We consider the infinite horizon tabular setting where $\mc S$ and $\mc A$ are discrete and finite, $\gamma \in (0,1)$ is a discount factor and $r:\mc S\times \mc A \mapsto[0,1]$ is the reward function;
    \item The value function $V^\pi(s)$ of a state w.r.t. policy $\pi$ is defined as 
    \begin{equation}
        V^\pi(s) := \bb E\brac{\sum_{t=0}^\infty \gamma^t r(s^t,a^t)|s^0=s},\;\forall s \in \mc S;
    \end{equation}
    \item The Q function $Q^\pi(s,a)$ of a state action pair $(s,a)$ w.r.t. a policy $\pi$ is defined by
    \begin{equation}
        Q^\pi(s,a):=\bb E\brac{\sum_{t=0}^\infty\gamma^t r(s^t,a^t)|s^0=s,a^0=a}\forall (s,a)\in \mc S\times \mc A;
    \end{equation}
    \item $\mb P\in \bb R^{|\mc S||\mc A|\times |\mc S|}$ is a matrix of the transition kernel $P$;
    \item $\mb P^\pi\in\bb R^{|\mc S||\mc A|\times |\mc S||\mc A|}$ and  $\mb P_\pi\in\bb R^{|\mc S|\times |\mc S|}$ two square probability transition matrices induced by the policy $\pi$ over the state-action pair and the states respectively, defined by
    \begin{equation}
        \mb P^\pi := \mb P \mb \Pi^\pi,\quad \mb P_\pi := \mb \Pi^\pi\mb P;
    \end{equation}
    \item $\mb \Pi^\pi\in \{0,1\}^{|\mc S|\times |\mc S||\mc A|}$ is a projection matrix:
    \begin{equation}
        \mb \Pi^\pi = 
        \begin{pmatrix}
            \mb e^\top_{\pi(1)} & & & \\
             & \mb e^\top_{\pi(2)} & & \\
             & & \ddots & \\
             & & & \mb e^\top_{\pi(|\mc S|)}.
        \end{pmatrix}
    \end{equation}
    \item $\mb r\in \bb R^{|\mc S||\mc A|}$ is the reward function
    \item $\mb r^\pi \in \bb R^{|\mc S|}$ is the reward function following policy $\pi$, simply we have $\mb r^\pi = \mb \Pi^\pi\mb r$.
\end{itemize}
\subsection{Derivations of Conservative Bellman Consistency Equation~\ref{eq:cql-bellman-consistency}}
$\forall (s,a)\in \mc S\times \mc A$, the tabular CQL optimization (Equation~\ref{eqn:cql_training}) aims at solving the following optimization problem:
\label{appendix:cql-critical-point}
\begin{equation}
    \begin{split}
        \min_{Q(s,a)} \alpha \brac{ \underset{s\sim \mc D,a\sim \pi}{\bb E}Q(s,a)-\underset{(s,a)\sim \mc D,a\sim \pi_\beta}{\bb E}Q(s,a)}+\frac{1}{2}\underset{s\sim \mc D,a\sim \pi_\beta}{\bb E}\brac{\paren{Q(s,a)-\mc B^\pi Q(s,a)}^2}.
    \end{split}
\end{equation}
If we rewrite $x = Q(s,a)$ and define function $f(x)$ as
\begin{equation}
    f(x) := \alpha \brac{ \underset{s\sim \mc D,a\sim \pi}{\bb E}x-\underset{(s,a)\sim \mc D,a\sim \pi_\beta}{\bb E}x}+\frac{1}{2}\underset{s\sim \mc D,a\sim \pi_\beta}{\bb E}\brac{\paren{x-\mc B^\pi x}^2},
\end{equation}
setting $f^\prime(x) = 0$ yields
\begin{equation}
    \alpha \paren{\pi(a|s) - \pi_\beta(a|s)} + \pi_\beta(x - \mc B^\pi x) = 0,
\end{equation}
which leads to 
\begin{equation}
    x = \mc B^\pi x - \alpha\brac{\frac{\pi(a|s)}{\pi_\beta(a|s)}-1}.
\end{equation}

\subsection{Conservative Bellman Consistency Equations}
\paragraph{Q Functions.}
Considering the matrix form and point-wise Bellman consistency equation, we have
\begin{align}
    \mb Q^\pi &= \mb r + \gamma \mb P^\pi \mb Q^\pi\implies \mb Q^\pi = \paren{\mb I - \gamma \mb P^\pi}^{-1}\mb r\\
    Q^\pi(s,a) &= r(s,a) + \gamma \brac{\mb P^\pi \mb Q^\pi}_{(s,a)},\;\forall (s,a)\in \mc S\times \mc A,
\end{align}
where we use $\brac{\mb P^\pi \mb Q^\pi}_{(s,a)}$ to denote the entries $(s,a)^\text{th}$ of the matrix $\mb P^\pi \mb Q^\pi$.
Now recall the point-wise Bellman consistency equation of the CQL objective~\eqref{eq:cql-bellman-consistency}, we also have the following point-wise consistency equation:
\begin{align}
    Q_{\alpha,\beta}^\pi(s,a) &= r(s,a) + \gamma \brac{\mb P^\pi \mb Q_{\alpha,\beta}^\pi}_{(s,a)} - \alpha\brac{\frac{\pi(a|s)}{\pi_\beta(a|s)}-1},\;\forall (s,a)\in \mc S\times \mc A,\\
    &= r_{\alpha,\beta} + \gamma \brac{\mb P^\pi \mb Q^\pi}_{(s,a)},
\end{align}
where 
\begin{equation}
    \label{eq:new-bellman-reward-cql}
    r^\pi_{\alpha,\beta}(s,a) := r(s,a) - \alpha\brac{\frac{\pi(a|s)}{\pi_\beta(a|s)}-1},\;\forall (s,a)\in \mc S\times \mc A.
\end{equation}
Hence, we can similarly have the bellman-consistency equation of CQL in matrix form:
\begin{equation}
    \mb Q_{\alpha,\beta} = \mb r_{\alpha,\beta} + \gamma \mb P^\pi \mb Q^\pi_{\alpha,\beta} \implies \mb Q^\pi_{\alpha,\beta}=\paren{\mb I - \gamma \mb P^\pi}^{-1}\mb r_{\alpha,\beta}.
\end{equation}
\paragraph{Value Functions.} Now considering the Bellman Consistency equation of the Value function, we have
\begin{align}
    \mb V^\pi &= \mb r^\pi + \gamma \mb P_\pi \mb V^\pi\implies\mb V^\pi = (\mb I - \gamma \mb P_\pi)^{-1}\mb r^\pi\\
    \mb V_{\alpha,\beta}^{\pi} &= \mb r^\pi_{\alpha,\beta} + \gamma \mb P_\pi \implies \mb V_{\alpha,\beta}^\pi = (\mb I - \gamma \mb P_\pi)^{-1}\mb r_{\alpha,\beta}^\pi.
\end{align}


\paragraph{Summary.} In summary, for a given reward function $\mb r$, a fixed policy $\pi$, a behavior policy $\pi_\beta$, and a fixed constant $\alpha$, the {\em policy evaluation} for CQL satisfies:
\begin{equation}
    \label{eq:cql-consistency-equation}
    \begin{split}
         \mb V_{\alpha,\beta}^{\pi} & = (\mb I - \gamma \mb P_\pi)^{-1}\mb r_{\alpha,\beta}^\pi = (\mb I - \gamma \mb P_\pi)^{-1}\mb \Pi^\pi\brac{\mb r - \alpha\paren{\mb\pi/\mb\pi_\beta-\mb 1}} \\
         \mb Q^\pi_{\alpha,\beta}&=\paren{\mb I - \gamma \mb P^\pi}^{-1}\mb r_{\alpha,\beta}=\paren{\mb I - \gamma \mb P^\pi}^{-1}\brac{\mb r - \alpha\paren{\mb\pi/\mb\pi_\beta-\mb 1}}.
    \end{split}
\end{equation}
where $\pi/\pi_\beta\in \bb R^{|\mc S|\times |\mc A|}$ is a vector whose $(s,a)$ entry denotes $\pi(a|s)/\pi_\beta(a|s)$ and $\mb 1=\{1,1,\dots,1\}^\top\in \bb R^{|\mc S||\mc A|}$.

\subsection{Proof of Theorem~\ref{thm:mixing-dip-main}}
\label{appendix:proof:mixing-dip-main}
\begin{theorem}[Invariant Conservative Q Functions]
Let $Q^\pi_{\alpha,\beta}$ and $Q^\pi_{\alpha^\prime,\beta^\prime}$ denote the conservative value function from solving the conservative bellman consistency equation (Equation~\ref{eq:cql-bellman-consistency} and~\ref{eq:cql-consistency-equation}) with regularizers $\alpha,\alpha^\prime$ and behavior policies $\pi_\beta,\pi_{\beta^\prime}$ respectively. Then for a given policy $\pi$, $Q^\pi_{\alpha,\beta}(s) = Q^\pi_{\alpha^\prime,\beta^\prime}(s),\forall s\in \mc S$ if and only if
\begin{equation}
    \frac{\alpha}{\pi_\beta(a|s)}-\frac{\alpha^\prime}{\pi_{\beta^\prime}(a|s)}=\frac{\alpha-\alpha^\prime}{\pi(a|s)},\;\forall (s,a)\in \mc S\times \mc A.
\end{equation}
\end{theorem}
\begin{proof}
\label{proof:mixing-dip-main}
% Now consider we updated our behavior policy by
% \begin{equation}
%     \mb \pi_{\beta^\prime} = \delta \mb \pi_{\beta} + (1-\delta) \mb \pi_\mr{on}, 
% \end{equation}
% where
% \begin{equation}
%     \pi_{\beta^\prime}(a|s) = 
%     \begin{cases}
%         \pi_\beta(a|s),\;&\text{with probability } \delta,\\
%         \pi_\mr{on}(a|s),\;&\text{with probability } 1-\delta.\\
%     \end{cases}
% \end{equation}
By the conservative Bellman Consistency~\eqref{eq:cql-consistency-equation}, we know that changing a behavior policy from $\pi_\beta$ to $\pi_{\beta^\prime}$ and changing the regularize from $\alpha$ to $\alpha^\prime$, we have
\begin{equation}
    \begin{split}
        \mb Q_{\alpha,\beta}^{\pi} - \mb Q_{\alpha^\prime,\beta^\prime}^{\pi} & = (\mb I - \gamma \mb P^\pi)^{-1}(\mb r_{\alpha,\beta}^\pi-\mb r_{\alpha^\prime,\beta^\prime}^\pi)  \\
        & =(\mb I - \gamma \mb P^\pi)^{-1}\brac{\alpha\paren{\frac{\mb\pi}{\mb\pi_\beta}-\mb 1}-\alpha^\prime\paren{\frac{\mb\pi}{\mb\pi_{\beta^\prime}}-\mb 1}}.
    \end{split} 
\end{equation}
Since $(\mb I - \gamma \mb P)^{-1}$ is a square and full rank matrix, $ \mb Q_{\alpha,\beta}^{\pi} - \mb Q_{\alpha^\prime,\beta^\prime}^{\pi} = \mb 0$ holds if and only if
\begin{equation}
    \alpha\paren{\frac{\mb\pi}{\mb\pi_\beta}-\mb 1}-\alpha^\prime\paren{\frac{\mb\pi}{\mb\pi_{\beta^\prime}}-\mb 1} = \mb 0\implies \frac{\alpha}{\pi_\beta(a|s)}-\frac{\alpha^\prime}{\pi_{\beta^\prime}(a|s)}=\frac{\alpha-\alpha^\prime}{\pi(a|s)},\;\forall (s,a)\in \mc S\times \mc A,
\end{equation}
which finishes the proof.
\end{proof}

\fi


\end{document}
