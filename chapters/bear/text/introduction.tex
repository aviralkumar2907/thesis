\section{Introduction}
\label{sec:intro}
\vspace{-5pt}
% Deep reinforcement learning holds promise as a general framework for solving decision making problems. However, the standard trial-and-error paradigm of RL, where an agent begins \textit{tabular rasa}, without any prior knowledge, is impractical for many real-world scenarios, such as when failures may be too costly or when sample efficiency is a major concern. Instead, it is often desirable for an agent to be able to safely learn from large banks of prior experience, such as those available for autonomous driving~\citep{yu2018bdd}, aerial drones~\citep{majdik2017zurich}, or robotic manipulation~\citep{levine2018handeye}. \textit{Off-policy} reinforcement learning methods hold promise for being able to leverage such prior experience, and have been demonstrated to be relatively sample efficient ~\citep{gu2016qprop,haarnoja2018sac,munos2016safe} while being able to solve complex control tasks in robotics~\cite{} and simulated environments~\cite{}. However, while these methods are in principle off-policy, they often fail to learn when presented with arbitrary off-policy data -- even ``high-quality'' data such as expert data from optimal policies~\citep{deBruin2015importance,fujimoto2018off,fu2019diagnosing}. Effective learning typically requires at least some amount of on-policy data collection~\cite{something}.
% %%SL.5.11: fill in above citation
% This sensitivity to the training data distribution is alarming, and one would hope that an off-policy algorithm will be able to learn reasonable policies before being deployed in the real world, without performance on the task decreasing as learning progresses.

One of the primary drivers
%\blfootnote{Correspondence to: Aviral Kumar (\texttt{aviralk@berkeley.edu})} 
of the success of machine learning methods in open-world perception settings, such as computer vision~\cite{he2016resnet} and NLP~\cite{devlin2018bert}, has been the ability of high-capacity function approximators, such as deep neural networks, to learn generalizable models from large amounts of data. Reinforcement learning (RL) has proven comparatively difficult to scale to unstructured real-world settings because most RL algorithms require active data collection. As a result, RL algorithms can learn complex behaviors in simulation, where data collection is straightforward, 
%~\cite{} \TODO{what's a good reference here?},
%%SL.5.20: I think we could just omit any citation here, and cite all the RL stuff in the related work section. No need to needlessly spark a debate about whether Atari games, Go, or GTA driving require more or less generalization.
but real-world performance is limited by the expense of active data collection. 
%If we can develop effective off-policy RL methods, we would be able to train autonomous agents using large previously collected datasets. 
In some domains, such as autonomous driving~\cite{yu2018bdd} and recommender systems~\citep{bennett2007netflix}, previously collected datasets are plentiful. Algorithms that can utilize such datasets effectively would not only make real-world RL more practical, but also would enable substantially better generalization by incorporating diverse prior experience.  

In principle, off-policy RL algorithms can leverage this data; however, in practice, off-policy algorithms are limited in their ability to learn entirely from off-policy data. %Our aim in this paper is to devise off-policy RL algorithms that are stable and performant when trained entirely on off-policy experience, without any on-policy data collection.  
Recent off-policy RL methods   (e.g.,~\citep{haarnoja2018sac,munos2016safe,kalashnikov18qtopt,impala2018}) have demonstrated sample-efficient performance on complex tasks in robotics~\cite{kalashnikov18qtopt} and simulated environments~\cite{mujoco}. 
However, these methods can still fail to learn when presented with arbitrary off-policy data without the opportunity to collect more experience from the environment. This issue persists even when the off-policy data comes from effective expert policies, which in principle should address any exploration challenge~\citep{deBruin2015importance,fujimoto2018off,fu2019diagnosing}. This sensitivity to the training data distribution is a limitation of practical off-policy RL algorithms, and one would hope that an off-policy algorithm should be able to learn reasonable policies through training on static datasets before being deployed in the real world. %, without performance on the task decreasing as learning progresses. 
In this paper, we aim to develop off-policy, value-based RL methods that can learn from large, static datasets. As we show, a crucial challenge in applying value-based methods to off-policy scenarios arises in the bootstrapping process employed
%%SL.5.20: I think the above sentence is a distraction -- this isn't actually the issue that we are dealing with, is it?
when Q-functions are evaluated on out of \textit{out-of-distribution} action inputs for computing the backup when training from off-policy data. This may introduce errors in the Q-function and the algorithm is unable to collect new data in order to remedy those errors, making training unstable and potentially diverging. %Through systematic experiments, we demonstrate that this issue can lead to unstable, diverging, behavior (Sec.~\ref{sec:Problem Description}) during training.  %%SL.5.11: the above sentence doesn't actually say what we do -- what does it mean "we focus"? what are we doing?
%This includes state-of-the-art methods based on Q-learning~\citep{mnih2015humanlevel}, as well as actor-critic methods such as DDPG~\citep{lillicrap2015ddpg}, TD3~\cite{fujimoto18addressing}, and SAC~\citep{haarnoja2018sac}. This class of methods have been the focus of several theoretical analysis works that highlight the instabilities and sources of error that arise from the bootstrapping process employed by ADP methods~\citep{farahmand2010error}.
% In a standard supervised learning setup in machine learning, discrepancies between training performance and testing performance are often attributed to ``overfitting.'' However in reinforcement learning, agents suffer from substantial distribution shift, since updating the policy will change the distribution of states that the agent will experience. Simply obtaining more off-policy data from the same distribution is insufficient to guarantee stability in the learning process -- the design of stable off-policy algorithms must be observant of the fact that models will be evaluated at states with little data support over the course of training. In order to address problems related to learning from off-policy distributions, we analyze and address two major sources of error that arise from off-policy learning: bootstrapping error and evaluation error \textcolor{red}{TODO: make up better names for these}.
%Most of these ADP methods use bootstrapping to perform fixed point iteration with function approximators, which outputs an optimal policy at convergence. In this work, we analyse one major source of error that arises in these algorithms when learning with  static datasets -- which we call \textit{bootstrapping error}, and define next.
%%SL.5.11: We need to focus. The above paragraph casts the net a bit too broadly. If we focus on out-of-distribution actions, let's just motive that directly, instead of getting distracted with problems that are not the primary focus of our work. My recommendation would be to rewrite the above paragraph to focus exclusively on out-of-distribution action issues, and leave the rest for the discussion section.
%As Bellman error is typically minimized via supervised learning, the Q-values outside of training data. Moreover, function approximators used to model Q-functions in practice, have no guarantees whatsoever to produce reasonable values when queried with out-of-distribution inputs and can often generalize in unpredictable and undesired ways. These errors are hence, largely uncontrolled and these values can propagate to Q-values of neighboring states (which are then propagated further on subsequent iterations). 
%%SL.4.23: I think the above paragraph is reasonable, but it somewhat sweeps under the rug the critical observation: The issue here is that function approximators have no guarantees whatsoever to produce reasonable values when queried with out-of-distribution inputs. But this idea is missing in the above paragraph, and instead the issue is described in a way that is needlessly convoluted. Can we just state much more clearly that the problem is due to out-of-distribution inputs? This is good because it's not really an issue that prior work has thoroughly studied or addressed, due to lack of focus on function approximation. 
% The second source of error occurs from \textit{evaluating} the policy at states and actions not seen during training. When a policy is executed in an environment, it may inadvertently visit states that have not been experienced before. This can cause the policy to drift away from the training data during evaluation, causing compounding error over the course of an episode. This can happen, for instance, when training data exclusively comes from an expert policy, and during evaluation the policy visits a suboptimal state. Such issues have been extensively studied in imitation learning~\citep{ross2011dagger}, and we demonstrate that even in absence of bootstrapping error, this evaluation error can cause policies to be arbitrarily bad (Sec.~\ref{}).
%%SL.4.23: do we actually have a solution to this?
%%SL.4.23: The current introduction gets across the main idea, but it's a bit hard to parse and a bit technical. Can we more clearly draw out the common themes and state them out front? At a very fundamental level, both of these issues are issues due to out-of-distribution inputs, but they are also different from each other in perhaps a surprising way. Describing this simple fact early will make things look more coherent, and less like two disconnected and overly technical ideas.
%%SL.5.11: please address the above...
%%SL.5.11: also, the above discussion says nothing about how we address any of these problems
Our primary contribution is an analysis of error accumulation in the bootstrapping process due to out-of-distribution inputs and a practical way of addressing this error. 
%%SL.5.20: This seems like a pretty disappointing way to state the contribution. Can you more clearly state that the main contribution of our work is an analysis of bootstrapping error and a practical method for addressing it?
First, we formalize and analyze the reasons for instability and poor performance when learning from off-policy data. We show that, through careful action selection, error propagation through the Q-function can be mitigated. We then propose a principled algorithm called \emph{bootstrapping error accumulation reduction} (BEAR) to control bootstrapping error in practice, which uses the notion of \emph{support-set matching} to prevent error accumulation. Through systematic experiments, we show the effectiveness of our method on continuous-control MuJoCo tasks, with a variety of off-policy datasets: generated by a random, suboptimal, or optimal policies. BEAR is consistently robust to the training dataset, matching or exceeding the state-of-the-art in all cases, whereas existing algorithms only perform well for specific datasets.
%%SL.4.23: another sentence about the results?
%%JF.4.25: I think we can just put a sentence on results in when we actually have them.

%%SL.5.11: We should rewrite the last paragraph to more directly describe what we do. Here is a potential phrasing:
% The main contribution in our paper is an analysis of several methods for stabilizing off-policy reinforcement learning. We first analyze the reasons for instability and poor performance when learning from off-policy data with approximate dynamic programming algorithms, using Q-learning and soft actor-critic~\cite{} in our analysis. We identify the out-of-distribution action problem and discuss its causes, and then propose three potential solutions. First, we show that actor-critic algorithms mitigate the issue both in theory and in practice, but only to a limited degree. Such methods remain susceptible to overtraining. Second, we show that a robust variant of importance sampling can greatly alleviate the out-of-distribution action problem, at the cost of severely conservative learning with poor final performance. Finally, we show that a combination of a pessimistic Q-value bound and a distributional support constraint mitigates the issue while maintaining good final performance. We analyze our method when learning from off-policy data with differing quality, ranging from random to near optimal, and on a range of discrete and continuous tasks. Our results show that mitigating the problem of out-of-distribution actions greatly improves the final performance and stability of off-policy RL.
