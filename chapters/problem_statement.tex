\documentclass[../thesis.tex]{subfiles}

\usepackage{wrapfig}
\usepackage{cite}
% \usepackage{natbib}
% \usepackage[round]{natbib}

\begin{document}

% \blfootnote{This chapter is based on \cite{fu2019diagnosing}, published at ICML 2019, which is joint work with Justin Fu, Matthew Soh, and Sergey Levine.}

% Off-policy reinforcement learning aims to leverage experience collected from prior policies for sample-efficient learning. However, in practice, commonly used off-policy approximate dynamic programming methods based on Q-learning and actor-critic methods are highly sensitive to the data distribution, and can make only limited progress without collecting additional on-policy data. As a step towards more robust off-policy algorithms, we study the setting where the off-policy experience is fixed and there is no further interaction with the environment. We identify \emph{bootstrapping error} as a key source of instability in current methods. Bootstrapping error is due to bootstrapping from actions that lie outside of the training data distribution, and it accumulates via the Bellman backup operator. We theoretically analyze bootstrapping error, and demonstrate how carefully constraining action selection in the backup can mitigate it. Based on our analysis, we propose a practical algorithm, bootstrapping error accumulation reduction (BEAR). We demonstrate that BEAR is able to learn robustly from different off-policy distributions, including random and suboptimal demonstrations, on a range of continuous control tasks.

%%SL.4.23: maybe make above sentence more explicit? readers might not actually realize what kinds of issues Q-learning has with off-policy data (also, it's not just Q-learning, maybe say "commonly used dynamic programming methods, such as Q-learning and actor-critic algorithms,...")
% In this paper, we analyze and control two sources of error that arise from off-policy training - 
% %%SL.4.23: Somehow understates our contribution, maybe say something like: In this paper, we aim to systematically analyze and correct the issues that result in poor off-policy training. We identify [whatever] (and make it clear that this is our contribution)
% bootstrapping error which arises from errors propagating via the Bellman backup operator, and evaluation error arising from executing a policy at states not seen during training. We propose OUR METHOD, an actor-critic method which carefully selects actions so as to minimize the accumulation of such errors. We demonstrate that OUR METHOD is able to learn robustly from different off-policy distributions on a wide range of challenging continuous control tasks.


% \section{Introduction}
\label{sec:intro}

% One of the primary drivers
% %\blfootnote{Correspondence to: Aviral Kumar (\texttt{aviralk@berkeley.edu})} 
% of the success of machine learning methods in open-world perception settings, such as computer vision~\cite{he2016resnet} and NLP~\cite{devlin2018bert}, has been the ability of high-capacity function approximators, such as deep neural networks, to learn generalizable models from large amounts of data. Reinforcement learning (RL) has proven comparatively difficult to scale to unstructured real-world settings because most RL algorithms require active data collection. As a result, RL algorithms can learn complex behaviors in simulation, where data collection is straightforward, 
% %~\cite{} \TODO{what's a good reference here?},
% %%SL.5.20: I think we could just omit any citation here, and cite all the RL stuff in the related work section. No need to needlessly spark a debate about whether Atari games, Go, or GTA driving require more or less generalization.
% but real-world performance is limited by the expense of active data collection. 
% %If we can develop effective off-policy RL methods, we would be able to train autonomous agents using large previously collected datasets. 
% In some domains, such as autonomous driving~\cite{yu2018bdd} and recommender systems~\citep{bennett2007netflix}, previously collected datasets are plentiful. Algorithms that can utilize such datasets effectively would not only make real-world RL more practical, but also would enable substantially better generalization by incorporating diverse prior experience.  

% aim in this paper is to devise off-policy RL algorithms that are stable and performant when trained entirely on off-policy experience, without any on-policy data collection.  
% Recent off-policy RL methods   (e.g.,~\citep{haarnoja2018sac,munos2016safe,kalashnikov18qtopt,impala2018}) have demonstrated sample-efficient performance on complex tasks in robotics~\cite{kalashnikov18qtopt} and simulated environments~\cite{mujoco}. 
% However, these methods can still fail to learn when presented with arbitrary off-policy data without the opportunity to collect more experience from the environment. This issue persists even when the off-policy data comes from effective expert policies, which in principle should address any exploration challenge~\citep{deBruin2015importance,fujimoto2018off,fu2019diagnosing}. This sensitivity to the training data distribution is a limitation of practical off-policy RL algorithms, and one would hope that an off-policy algorithm should be able to learn reasonable policies through training on static datasets before being deployed in the real world. %, without performance on the task decreasing as learning progresses. 

While state-of-the-art off-policy RL methods   (e.g.,~\citep{haarnoja2018sac,munos2016safe,kalashnikov18qtopt,impala2018}) have demonstrated sample-efficient performance on complex tasks in robotics~\cite{kalashnikov18qtopt} and in simulation~\cite{mujoco}, previously, we saw that these methods fail to learn when presented with arbitrary offline datasets. This issue persists even when the off-policy data comes from effective expert policies, which in principle should address any exploration challenge~\citep{deBruin2015importance,fujimoto2018off,fu2019diagnosing}. In this chapter, we aim to understand the root cause behind this inability and then develop off-policy, value-based RL methods that can learn from static, offline datasets. 

We show that a crucial challenge in applying value-based methods to off-policy scenarios arises in the bootstrapping process employed
when Q-functions are evaluated on out of \textit{out-of-distribution} action inputs for computing the backup when training from off-policy data. This may introduce errors in the Q-function and the algorithm is unable to collect new data in order to remedy those errors, making training unstable and divergent. We will first formalize and analyze the reasons for instability and poor performance when learning from off-policy data. Then, we will show that, through careful action selection, error propagation through the Q-function can be mitigated. We will then propose a principled algorithm called \emph{bootstrapping error accumulation reduction} (BEAR) to control bootstrapping error in practice, which uses the notion of \emph{support-set matching} to prevent error accumulation. Finally, through systematic experiments, we will show the effectiveness of our method on continuous-control MuJoCo Gym tasks, with a variety of off-policy datasets: generated by a random, suboptimal, or optimal policies. BEAR is consistently robust to the training dataset, matching or exceeding the state-of-the-art in all cases, whereas existing algorithms only perform well for specific datasets.


%Through systematic experiments, we demonstrate that this issue can lead to unstable, diverging, behavior (Sec.~\ref{sec:Problem Description}) during training.  %%SL.5.11: the above sentence doesn't actually say what we do -- what does it mean "we focus"? what are we doing?
%This includes state-of-the-art methods based on Q-learning~\citep{mnih2015humanlevel}, as well as actor-critic methods such as DDPG~\citep{lillicrap2015ddpg}, TD3~\cite{fujimoto18addressing}, and SAC~\citep{haarnoja2018sac}. This class of methods have been the focus of several theoretical analysis works that highlight the instabilities and sources of error that arise from the bootstrapping process employed by ADP methods~\citep{farahmand2010error}.
% In a standard supervised learning setup in machine learning, discrepancies between training performance and testing performance are often attributed to ``overfitting.'' However in reinforcement learning, agents suffer from substantial distribution shift, since updating the policy will change the distribution of states that the agent will experience. Simply obtaining more off-policy data from the same distribution is insufficient to guarantee stability in the learning process -- the design of stable off-policy algorithms must be observant of the fact that models will be evaluated at states with little data support over the course of training. In order to address problems related to learning from off-policy distributions, we analyze and address two major sources of error that arise from off-policy learning: bootstrapping error and evaluation error \textcolor{red}{TODO: make up better names for these}.
%Most of these ADP methods use bootstrapping to perform fixed point iteration with function approximators, which outputs an optimal policy at convergence. In this work, we analyse one major source of error that arises in these algorithms when learning with  static datasets -- which we call \textit{bootstrapping error}, and define next.
%%SL.5.11: We need to focus. The above paragraph casts the net a bit too broadly. If we focus on out-of-distribution actions, let's just motive that directly, instead of getting distracted with problems that are not the primary focus of our work. My recommendation would be to rewrite the above paragraph to focus exclusively on out-of-distribution action issues, and leave the rest for the discussion section.
%As Bellman error is typically minimized via supervised learning, the Q-values outside of training data. Moreover, function approximators used to model Q-functions in practice, have no guarantees whatsoever to produce reasonable values when queried with out-of-distribution inputs and can often generalize in unpredictable and undesired ways. These errors are hence, largely uncontrolled and these values can propagate to Q-values of neighboring states (which are then propagated further on subsequent iterations). 
%%SL.4.23: I think the above paragraph is reasonable, but it somewhat sweeps under the rug the critical observation: The issue here is that function approximators have no guarantees whatsoever to produce reasonable values when queried with out-of-distribution inputs. But this idea is missing in the above paragraph, and instead the issue is described in a way that is needlessly convoluted. Can we just state much more clearly that the problem is due to out-of-distribution inputs? This is good because it's not really an issue that prior work has thoroughly studied or addressed, due to lack of focus on function approximation. 
% The second source of error occurs from \textit{evaluating} the policy at states and actions not seen during training. When a policy is executed in an environment, it may inadvertently visit states that have not been experienced before. This can cause the policy to drift away from the training data during evaluation, causing compounding error over the course of an episode. This can happen, for instance, when training data exclusively comes from an expert policy, and during evaluation the policy visits a suboptimal state. Such issues have been extensively studied in imitation learning~\citep{ross2011dagger}, and we demonstrate that even in absence of bootstrapping error, this evaluation error can cause policies to be arbitrarily bad (Sec.~\ref{}).
%%SL.4.23: do we actually have a solution to this?
%%SL.4.23: The current introduction gets across the main idea, but it's a bit hard to parse and a bit technical. Can we more clearly draw out the common themes and state them out front? At a very fundamental level, both of these issues are issues due to out-of-distribution inputs, but they are also different from each other in perhaps a surprising way. Describing this simple fact early will make things look more coherent, and less like two disconnected and overly technical ideas.
%%SL.5.11: please address the above...
%%SL.5.11: also, the above discussion says nothing about how we address any of these problems
% Our primary contribution is 
%%SL.5.20: This seems like a pretty disappointing way to state the contribution. Can you more clearly state that the main contribution of our work is an analysis of bootstrapping error and a practical method for addressing it?
%%SL.4.23: another sentence about the results?
%%JF.4.25: I think we can just put a sentence on results in when we actually have them.

%%SL.5.11: We should rewrite the last paragraph to more directly describe what we do. Here is a potential phrasing:
% The main contribution in our paper is an analysis of several methods for stabilizing off-policy reinforcement learning. We first analyze the reasons for instability and poor performance when learning from off-policy data with approximate dynamic programming algorithms, using Q-learning and soft actor-critic~\cite{} in our analysis. We identify the out-of-distribution action problem and discuss its causes, and then propose three potential solutions. First, we show that actor-critic algorithms mitigate the issue both in theory and in practice, but only to a limited degree. Such methods remain susceptible to overtraining. Second, we show that a robust variant of importance sampling can greatly alleviate the out-of-distribution action problem, at the cost of severely conservative learning with poor final performance. Finally, we show that a combination of a pessimistic Q-value bound and a distributional support constraint mitigates the issue while maintaining good final performance. We analyze our method when learning from off-policy data with differing quality, ranging from random to near optimal, and on a range of discrete and continuous tasks. Our results show that mitigating the problem of out-of-distribution actions greatly improves the final performance and stability of off-policy RL.

% \section{Related Work}
\label{sec:related}
\vspace{-10pt}
In this work, we study off-policy reinforcement learning with static datasets. Errors arising from inadequate sampling, distributional shift, and function approximation have been rigorously studied as ``error propagation'' in approximate dynamic programming (ADP)~\citep{bertsekas1996ndp,munos2003errorapi,farahmand2010error,bruno2015approximate}. These works often study how Bellman errors accumulate and propagate to nearby states via bootstrapping. In this work, we build upon tools from this analysis to show that performing Bellman backups on static datasets leads to error accumulation due to out-of-distribution values. Our approach is motivated as reducing the rate of propagation of error propagation between states.

% One way to reduce off-policy errors in practice is using importance sampling~\cite{precup2001offpol,sutton2016etd,hallak2017coptd,gelada2019off,munos2016safe} and minimize an importance sampled objective. In this work, we tackle the problem of inaccurate target values in Q-learning and the developments in importance sampling for off-policy evaluation are orthogonal and complementary to our method.
%%SL.5.20: I think we can heavily trim this paragraph down to basically one sentence with a row of citations if we don't end up making a big deal out of importance sampling. This will also help to mitigate our current problem with the new stuff beginning very very late in the paper.

Our approach constrains actor updates so that the actions remain in the support of the training dataset distribution. Several works have explored similar ideas in the context of off-policy learning learning in online settings. \citet{kakade2002cpi} shows that large policy updates can be destructive, and propose a conservative policy iteration scheme which constrains actor updates to be small for provably convergent learning. \citet{grau-moya2018soft} use a learned prior over actions in the maximum entropy RL framework~\citep{levine2018rlasinference} and justify it as a regularizer based on mutual information. However, none of these methods use static datasets. Importance Sampling based distribution re-weighting~\cite{munos2016safe,gelada2019off,precup2001offpol,mahmood2015emphatic} has also been explored primarily in the context of off-policy policy evaluation.

Most closely related to our work is batch-constrained Q-learning (BCQ)~\citep{fujimoto2018off} and SPIBB~\citep{laroche2019spibb},
which also discuss instability arising from previously unseen actions. \citet{fujimoto2018off} show convergence properties of an action-constrained Bellman backup operator in tabular, error-free settings. We prove stronger results under approximation errors and provide a bound on the \emph{suboptimality} of the solution. This is crucial as it drives the design choices for a practical algorithm. 
As a consequence, although we experimentally find that \citep{fujimoto2018off} outperforms standard Q-learning methods when the off-policy data is collected by an expert, BEAR outperforms \cite{fujimoto2018off} when the off-policy data is collected by a suboptimal policy, as is common in real-life applications. Empirically, we find BEAR  achieves stronger and more consistent results than BCQ across a wide variety of datasets and environments. As we explain below, the BCQ constraint is too aggressive;  BCQ generally fails to substantially improve over the behavior policy, while our method actually improves when the data collection policy is suboptimal or random. SPIBB~\citep{laroche2019spibb}, like BEAR, is an algorithm based on constraining the learned policy to the support of a behavior policy. However, the authors do not extend safe performance guarantees from the batch-constrained case to the relaxed support-constrained case, and do not evaluate on high-dimensional control tasks. REM~\citep{agarwal19striving} is a concurrent work that uses a random convex combination of an ensemble of Q-networks to perform offline reinforcement learning from a static dataset consisting of interaction data generated while training a DQN agent.

% \section{Background}
\label{sec:background}
\vspace{-0.1in}
We represent the environment as a Markov decision process (MDP) defined by a tuple $(\mathcal{S}, \mathcal{A}, \trans, R, \rhoinit, \gamma)$, where $\mathcal{S}$ is the state space, $\mathcal{A}$ is the action space, $\trans(s' | s, a)$ is the transition distribution, $\rhoinit(s)$ is the initial state distribution, $R(s,a)$ is the reward function, and $\gamma \in (0,1)$ is the discount factor. The goal in RL is to find a policy $\pi(a|s)$ that maximizes the expected cumulative discounted rewards which is also known as the return. The notation $\mu_\pi(s)$ denotes the discounted state marginal of a policy $\pi$, defined as the average state visited by the policy, $\sum_{t=0}^\infty \gamma^t p^t_\pi(s)$. $\trans^\pi$ is shorthand for the transition matrix from $s$ to $s'$ following a certain policy $\pi$, $p(s'|s) = E_{\pi}[p(s'|s,a)]$.

Q-learning learns the optimal state-action value function 
% \mbox{$\pi^* = \argmax{\pi} E_{s_{t+1} \sim \trans(\cdot|s_t, a_t), a_t \sim \pi(\cdot|s_t)}\left[\sum_{t=0}^\infty \gamma^t R(s_t, a_t)\right]$}. 
$Q^*(s,a)$, which represents the expected cumulative discounted reward starting in $s$ taking action $a$ and then acting optimally thereafter. The optimal policy can be recovered from $Q^*$ by choosing the maximizing action. Q-learning algorithms are based on iterating the Bellman optimality operator $\backup$, defined as
\begin{align*}
(\backup \hat{Q})(s, a) \coloneqq R(s, a) + \gamma \expec_{T(s'|s,a)}[\max_{a'}\hat{Q}(s', a')].~~~~~
%V(s') \coloneqq: \max_{a'} Q(s', a')    .
\end{align*}
%Tabular Q-learning is a dynamic programming algorithm that iterates the Bellman backup $Q^{t+1} \leftarrow \backup Q^t$. 
%Because the Bellman backup is a $\gamma$-contraction in the L-$\infty$ norm, and $Q^*$ (the Q-values of $\pi^*$) is its fixed point, Q-iteration can be shown to converge to $Q^*$~\citep{suttonrlbook}. A deterministic optimal policy can then be obtained as $\pi^*(s) = \argmax{a} Q^*(s,a)$.
When the state space is large, we represent $\hat{Q}$ as a hypothesis from the set of function approximators $\Qclass$ (e.g., neural networks). In theory, the estimate of the $Q$-function is updated by projecting $\backup \hat{Q}$ into $\Qclass$ (i.e., minimizing the mean squared Bellman error $\expec_{\nu}[( Q - \backup \hat{Q})^2]$, where $\nu$ is the state occupancy measure under the behaviour policy). This is also referred to a \emph{Q-iteration}. In practice, an empirical estimate of $\backup \hat{Q}$ is formed with samples, and treated as a supervised $\ell_2$ regression target to form the next approximate $Q$-function iterate. %Q-function is updated by minimizing %a $\mu$-weighted
%%SL.5.20: what is mu?
%$\ell_2$-projection onto $\Qclass$ (chosen class of Q-function approximators):
%%SL.5.20: \Qclass was never defined?
% $Q^{t+1} \leftarrow \Projmu(\backup \bar{Q)}$
%via supervised learning. The values produced by the Bellman backup, $(\backup \bar{Q})(s,a)$, are commonly referred to as \textit{target values}. The \textit{target network}, $\bar{Q}$, is usually delayed in time, for stability reasons, and is updated periodically to match the current Q-network. 
%%SL.5.20: I think the replay buffer discussion can be deleted to make this more concise (in general, there is too much background), since it is not actually relevant for off-policy RL -- in off-policy RL, there is just a buffer of off-policy experience, and the notion of updating online doesn't make sense anyway.
In large action spaces (e.g., continuous), the maximization $\max_{a'} Q(s', a')$
is generally intractable. Actor-critic methods~\cite{suttonrlbook,fujimoto18addressing,haarnoja2018sac} address this by additionally learning a policy $\pi_{\theta}$ that maximizes the $Q$-function. %by maximizing the expected Q-value under its specified action distribution, such that $\pi_\theta \leftarrow \max_{\theta} \mathbb{E}_{s \in \mathcal{B}}[Q(s, \pi_\theta(s)]$.
In this work, we study off-policy learning from a static dataset of transitions $\dataset = \{(s, a, s', R(s, a))\}$, collected under an unknown behavior policy $\beta(\cdot|s)$. We denote the distribution over states and actions induced by $\beta$ as $\mu(s,a)$.
%$Q_k$ refers to the Q-values at the $k$-th step of Q-learning. $Q*$ denotes the optimal Q-function. 


%%SL.5.20: In general, the background section is too long right now. We need to find some way to trim this, we should have the new stuff (Sec 4) start no later than the middle of page 3. Maybe we can trim the related work section a bit too, and tighten up the rhetoric in Section 1.

% \section{Out-of-Distribution Actions in Q-Learning}
\label{sec:Problem Description}

% Q-learning and other ADP methods which rely on iterating the Bellman backup operator are particularly susceptible to out-of-distribution inputs, because any errors incurred on these inputs can be propagated to neighbor states via the backup and keep compounding over iterations of the algorithm. Unfortunately, error on a single state can propagate to other states and can potentially cause inaccurate predictions across the entire Q-function. As we will show, these inaccuracies do affect the performance of off-policy algorithms in practice.

\begin{figure}
\vspace{-10pt}
\begin{center}
\vspace{-0.1in}
    \includegraphics[width=0.45\linewidth]{chapters/bear/images/cheetah_divergence.pdf}
    ~
    \includegraphics[width=0.45\linewidth]{chapters/bear/images/cheetah_divergence_q_val.pdf}
  \end{center}
 \vspace{-10pt}
 %%SL.5.22: Very important: the y-axes are not labeled right now, and it took me a while to figure out which plot was showing what. What is log(Q)? I guess you're trying to show that the right plot has Bellman error (?), while the left has performance? A couple more things: (1) always put space before ( (you often omit this space) (2) consider a caption like this (once the figures are labeled more clearly): Off-policy learning with SAC on HalfCheetah-v2 for different dataset sizes ($n$). The performance (left) does not correlate with $n$, while the Q-values (right) diverge or saturate at values far from the actual return.
  \caption{ \footnotesize Performance of SAC on HalfCheetah-v2 (return (left) and $\log$ Q-values (right)) with off-policy expert data w.r.t. number of training samples ($n$). Note the large discrepancy between returns (which are negative) and $\log$ Q-values (which have large positive values), which is not solved with additional samples.} 
 \vspace{-15pt}
 \label{fig:divergence}
\end{figure}
Q-learning methods often fail to learn on static, off-policy data, as shown in Figure \ref{fig:divergence}. At first glance, this resembles overfitting, but increasing the size of the static dataset does not rectify the problem, suggesting the issue is more complex. We can understand the source of this instability by examining the form of the Bellman backup. Although minimizing the mean squared Bellman error corresponds to a supervised regression problem, the targets for this regression are themselves derived from the current Q-function estimate. The targets are calculated by maximizing the learned $Q$-values with respect to the action at the next state. However, the $Q$-function estimator is only reliable on inputs from the same distribution as its training set. As a result, na\"{i}vely maximizing the value may evaluate the $\hat{Q}$ estimator on actions that lie far outside of the training distribution, resulting in pathological values that incur large error. We refer to these actions as out-of-distribution (OOD) actions. 

Formally, let $\valerr_k(\bs, \mathbf{a}) = |Q_k(\bs, \mathbf{a}) - Q^*(\bs, \mathbf{a})|$ denote the total error at iteration $k$ of Q-learning, and let $\projerr_k(\bs, \mathbf{a}) = |Q_k(\bs, \mathbf{a}) - \mathcal{B} Q_{k-1}(\bs, \mathbf{a})|$ denote the current Bellman error. Then, we have \mbox{$\valerr_k(\bs, \mathbf{a}) \le \projerr_k(\bs, \mathbf{a}) + \gamma \max_{\mathbf{a}'} \expec_{\bs'}[\valerr_{k-1}(\bs', \mathbf{a}')]$}. In other words, errors from $(\bs', \mathbf{a}')$ are discounted, then accumulated with new errors $\projerr_k(\bs, \mathbf{a})$ from the current iteration. We expect $\projerr_k(\bs, \mathbf{a})$ to be high on OOD states and actions, as errors at these state-actions are never directly minimized while training.

To mitigate bootstrapping error, we can restrict the policy to ensure that it output actions that lie in the support of the training distribution. This is distinct from previous work~\citep{jaques2019way} which implicitly constrains the \emph{distribution} of the learned policy to be close to the behavior policy, similarly to behavioral cloning~\cite{Schaal99isimitation}.
While this is sufficient to ensure that actions lie in the training set with high probability, it is overly restrictive. For example, if the behavior policy is close to uniform, the learned policy will behave randomly, resulting in poor performance, even when the data is sufficient to learn a strong policy (see Figure~\ref{fig:gridworld}
for an illustration). {Formally, this means that a learned policy $\pi(\mathbf{a}| \bs)$ has positive density\textit{ only where} the density of the behaviour policy $\beta(\mathbf{a}|s)$ is more than a threshold (i.e., $\forall \mathbf{a}, \beta(\mathbf{a}|\bs) \leq \varepsilon \implies \pi(\mathbf{a}|\bs) = 0$), instead of a closeness constraint on the value of the density $\pi(\mathbf{a}|\bs)$ and $\beta(\mathbf{a}|\bs)$.}
Our analysis instead reveals a tradeoff between staying within the data distribution and finding a suboptimal solution when the constraint is too restrictive. Our analysis motivates us to restrict the support of the learned policy, but not the probabilities of the actions lying within the support. This avoids evaluating the Q-function estimator on OOD actions, but remains flexible in order to find a performant policy. Our proposed algorithm leverages this insight. 

\section{Formal Analysis and Distribution-Constrained Backups}
\label{sec:dist_constrained}
In this section, we define and analyze a backup operator that restricts the set of policies used in the maximization of the Q-function, and we derive performance bounds which depend on the restricted set. This provides motivation for constraining policy support to the data distribution. We begin with the definition of a distribution-constrained operator:

\begin{tcolorbox}[colback=blue!6!white,colframe=black,boxsep=0pt,top=3pt,bottom=5pt]
\begin{definition}[Distribution-constrained operators]
Given a set of policies $\Pi$%, we define the set-constrained policy improvement operator as:
%$\greedyPi(Q, s) = \argmax{\pi \in \Pi}~ \expec_{a \sim \pi}[Q(s, a)]$,
%and 
, the distribution-constrained backup operator is defined as:
%\[ \TPi Q(s, a) \coloneqq \expec \big[ R(s, a) + \gamma \expec_{\trans(s' | s, a)}\left[\max_{\pi \in \Pi} \expec_{\pi}[Q(s', a')] \right] \big] \]
\begin{align*}
\TPi Q(\mathbf{s}, \mathbf{a}) \defeq \expec \big[ R(\bs, \mathbf{a}) + \gamma \max_{\pi \in \Pi} \expec_{\trans(\bs' | \bs, \mathbf{a})}\left[V(\bs') \right] \big]
\ \ \ \ \ \ \ \ \ \ \ \ 
V(\bs) \defeq \max_{\pi \in \Pi} \expec_{\pi}[Q(\mathbf{s}, \mathbf{a})]\ \ .
\end{align*}
\end{definition}
\end{tcolorbox}
This backup operator satisfies properties of the standard Bellman backup, such as convergence to a fixed point, as discussed in Appendix~\ref{app:constrained_backup}. To analyze the (sub)optimality of performing this backup under approximation error, we first quantify two sources of error. The first is a \emph{suboptimality bias}. The optimal policy may lie outside the policy constraint set, and thus a suboptimal solution will be found. The second arises from distribution shift between the training distribution and the policies used for backups. This formalizes the notion of OOD actions. %and states.
To capture suboptimality in the final solution, we define a \emph{suboptimality constant}, which measures how far $\pi^*$ is from $\Pi$. 

\begin{definition}[Suboptimality constant]
The suboptimality constant is defined as:
\[ \alpha(\Pi) = \max_{\bs, \mathbf{a}} |\TPi Q^*(\bs, \mathbf{a}) - \backup Q^*(\bs, \mathbf{a})|. \]
\end{definition}
\vspace{-10pt}
Next, we define a concentrability coefficient~\citep{munos2005erroravi}, which quantifies how far the visitation distribution generated by policies from $\Pi$ is  from the training data distribution. This constant captures the degree to which states and actions are out of distribution.
\begin{tcolorbox}[colback=blue!6!white,colframe=black,boxsep=0pt,top=3pt,bottom=5pt]
\begin{assumption}[Concentrability]
Let $\rhoinit$ denote the initial state distribution, and $\mu(\bs, \mathbf{a})$ denote the distribution of the training data over $\mathcal{S} \times \mathcal{A}$, with marginal $\mu(\bs)$ over $\mathcal{S}$. Suppose there exist coefficients $c(k)$ such that for any $\pi_1, ... \pi_k \in \Pi$ and $s \in \mathcal{S}$:
\[
\rhoinit P^{\pi_1}P^{\pi_2}...P^{\pi_k}(s) \le c(k) \mu(\bs),
\]
where $P^{\pi_i}$ is the transition operator on states induced by $\pi_i$.
Then, define the concentrability coefficient $C(\Pi)$ as
\[
C(\Pi) \defeq (1-\gamma)^2\sum_{k=1}^\infty k\gamma^{k-1}c(k).
\] \label{assumption:conc} \end{assumption} 
\end{tcolorbox}
% \vspace{-10pt}
To provide some intuition for $C(\Pi)$, if $\mu$ was generated by a single policy $\pi$, and $\Pi = \{\pi\}$ was a singleton set, then we would have $C(\Pi)=1$, which is the smallest possible value. However, if $\Pi$ contained policies far from $\pi$, the value could be large, potentially infinite if the support of $\Pi$ is not contained in $\pi$. Now, we bound the performance of approximate distribution-constrained Q-iteration:

\begin{tcolorbox}[colback=blue!6!white,colframe=black,boxsep=0pt,top=3pt,bottom=5pt]
\begin{theorem}
\label{thm:avi_bound}
Suppose we run approximate distribution-constrained value iteration with a set constrained backup $\TPi$. Assume that $\delta(\bs,\mathbf{a}) \ge \max_k |Q_k(\bs, \mathbf{a}) - \TPi Q_{k-1}(\bs, \mathbf{a})|$ bounds the Bellman error. Then,
\[\lim_{k \to \infty} \expec_{\rhoinit}[|V^{\pi_k}(\bs) - V^*(\bs)|] \le
\frac{\gamma}{(1-\gamma)^2}\left[ C(\Pi)\expec_\mu[\max_{\pi \in \Pi} \expec_{\pi}[\projerr(\bs, \mathbf{a})]] + \frac{1-\gamma}{\gamma}\alpha(\Pi) \right]
\]
\end{theorem}
\end{tcolorbox}
\begin{proof} See Appendix~\ref{app:error_prop}, Theorem~\ref{thm:avi_bound_proof} \end{proof}
This bound formalizes the tradeoff between keeping policies chosen during backups close to the data (captured by $C(\Pi)$) and keeping the set $\Pi$ large enough to capture well-performing policies (captured by $\alpha(\Pi)$). When we expand the set of policies $\Pi$, we are increasing $C(\Pi)$ but decreasing $\alpha(\Pi)$. An example of this tradeoff, and how a careful choice of $\Pi$ can yield superior results, is given in a tabular gridworld example in Fig.~\ref{fig:gridworld}, where we visualize errors accumulated during distribution-constrained Q-iteration for different choices of $\Pi$. 

Finally, we motivate the use of support sets to construct $\Pi$. We are interested in the case where $\Pi_\epsilon = \{ \pi ~|~ \pi( \mathbf{a} | \bs) = 0 \text{ whenever } \beta( \mathbf{a} | \bs) < \epsilon \}$, where $\beta$ is the behavior policy (i.e., $\Pi$ is the set of policies that have support in the probable regions of the behavior policy). Defining $\Pi_\epsilon$ in this way allows us 
%Why should we use support in order to construct $\Pi$ in the distribution-constrained operator? If any policy in $\Pi$ takes an action which has no support in the data, $C(\Pi)$ could potentially be infinite. Keeping policies in the support of the data distribution is a reasonable choice as it allows us 
to bound the concentrability coefficient:

\begin{tcolorbox}[colback=blue!6!white,colframe=black,boxsep=0pt,top=3pt,bottom=5pt]
\begin{theorem}
\label{thm:conc_coeff_bound}
% Assume the data distribution $\mu$ is generated by a policy $\beta$, such that $\mu(s,a) = d_\beta(s,a)$. Let $\mu_\beta(s)$ be the state-visitation marginal for $\beta$. Let us define $\Pi_\epsilon = \{ \pi ~|~ \pi( a | s) = 0 \text{ whenever } \beta( a | s) < \epsilon \}$. Let $\mu_{\Pi}(s)$ be the maximum discounted visitation marginal of a state generated by some sequence of policies $\{\pi_i\}_{i} \in \Pi$ and let $f(\epsilon) \defeq \min_{s \in \mathcal{S}, \mu_\Pi(s) > 0} [\mu(s)]$. Then, Assumption~\ref{assumption:conc} is satisfied for a $C(\Pieps)$ that is bounded as:
% \[
% C(\Pi_\epsilon) \leq C(\beta) \cdot \Big(1 + \frac{\gamma}{(1 - \gamma) f(\epsilon)} (1 - \epsilon)\Big)
% \]
Assume the data distribution $\mu$ is generated by a behavior policy $\beta$. %, such that $\mu(s,a) = d_\beta(s,a)$. 
Let $\mu(\bs)$ be the marginal state distribution under the data distribution. Define $\Pieps = \{ \pi ~|~ \pi( \mathbf{a} | \bs) = 0 \text{ whenever } \beta( \mathbf{a} | \bs) < \epsilon \}$ and let $\mu_\Pieps$ be the highest discounted marginal state distribution starting from the initial state distribution $\rho$ and following policies $\pi \in \Pieps$ at each time step thereafter. Then, there exists a concentrability coefficient $C(\Pieps)$ which is bounded:
\[
C(\Pi_\epsilon) \leq C(\beta) \cdot \Big(1 + \frac{\gamma}{(1 - \gamma) f(\epsilon)} (1 - \epsilon)\Big)
\]
where $f(\epsilon) \defeq \min_{\bs \in \mathcal{S}, \mu_\Pieps(\bs) > 0} [\mu(\bs)] > 0$.
\end{theorem}
\end{tcolorbox}
% \vspace{-10pt}
\begin{proof} See Appendix~\ref{app:error_prop}, Theorem~\ref{thm:conc_coeff_proof} \end{proof}
% \vspace{-10pt}
Qualitatively, $f(\epsilon)$ is the minimum discounted visitation marginal of a state under the behaviour policy if only actions which are more than $\epsilon$ likely are executed in the environment. Thus, using support sets gives us a single lever, $\epsilon$, which simultaneously trades off the value of $C(\Pi)$ and $\alpha(\Pi)$. Not only can we provide theoretical guarantees, we will see in our experiments (Sec.~\ref{sec:experiments}) that constructing $\Pi$ in this way provides a simple and effective method for implementing distribution-constrained algorithms. 

Intuitively, this means we can prevent an increase in overall error in the Q-estimate by selecting policies supported on the support of the training action distribution, which would ensure roughly bounded projection error $\delta_k(\mathbf{s}, \mathbf{a})$ while reducing the suboptimality bias, potentially by a large amount. Bounded error $\delta_k(\bs, \mathbf{a})$ on the support set of the training distribution is a reasonable assumption when using highly expressive function approximators, such as deep networks, especially if we are willing to reweight the transition set~\cite{Schaul2016PrioritizedER,fu2019diagnosing}. We further elaborate on this point in Appendix~\ref{app:bearql-more}.

\begin{figure}
    \centering
    \vspace{-0.1in}
    \includegraphics[width=0.9\textwidth]{chapters/bear/images/gridworld}
    \caption{ \footnotesize Visualized error propagation in Q-learning for various choices of the constraint set $\Pi$:
    unconstrained (top row) distribution-constrained (middle),
    and constrained to the behavior policy (policy-evaluation, bottom). Triangles represent Q-values for actions that move in different directions. The task (left) is to reach the bottom-left corner (G) from the top-left (S), but the behaviour policy (visualized as arrows in the task image, support state-action pairs are shown in black on the support set image) travels to the bottom-right with a small amount of $\epsilon$-greedy exploration. Dark values indicate high error, and light values indicate low error. Standard backups propagate large errors from the low-support regions into the high-support regions, leading to high error. Policy evaluation reduces error propagation from low-support regions, but introduces significant suboptimality bias, as the data policy is not optimal. A carefully chosen distribution-constrained backup strikes a balance between these two extremes, by confining error propagation in the low-support region while introducing minimal suboptimality bias.}
    \label{fig:gridworld}
    \vspace{-0.1in}
\end{figure}

% \subsection{Choosing Backup Policies for OOD Action Error Reduction}
% \label{sec:choosing_policies}
% Argument in Sec.~\ref{sec:tradeoff} tells us that, with a careful selection of the policy under which the target value is computed, the overall error of value estimates from the optimal value function $\|V^* - V_k\|$ can be reduced. How should we search for a policy that minimizes the overall error? Our choice is to backup from policies which maintain high-support over the action set of the data.
% %%SL.5.22: I think it's not obvious to readers that "policy for the backup" means the distribution over the actions under which the target value is calculated. -- addressed

% To justify this choice,
% %%SL.5.22: What choice? -- choice of backing up from any policy that maintains high support over data.
% we note that the error analysis relies on being able to quantify $\delta_k(s, a)$ (the per-state-action bellman error) for OOD actions. Outside of the support of the data distribution, it is hard to provide guarantees on $\delta_k$. However, when $a$ lies inside the support of the training distribution for a given state $s$, high-capacity function approximators trained with supervised learning are expected to produce a bounded error, given enough samples.
% %Even if they don't produce bounded error on such in-support inputs, techniques such as Prioritized Replay~\cite{Schaul2016PrioritizedER} can be employed to ensure bounded error on all in-support inputs. 
% %Furthermore, often the quantity of interest is the Bellman error weighted by the inverse density of the behaviour policy~\cite{antos07fitted}, which depends only on the support of the behaviour policy and this error metric is the equal for two policies provided they share the same support.
% Therefore, backing up from all actions that have non-negligible support under the training distribution is sufficient (but not necessary) to prevent error accumulation. Hence, we restrict the set $\Pi$
% %%SL.5.22: Did we define \Pi before? since we cut the set backup operator stuff, now this is much harder to follow. Maybe we can bring it back (but call it something else)?
% of policies used for distribution-constrained backups to the set of policies that are supported on the probable regions of the behaviour policy. That is, $\Pi = \{ \pi | \pi( a | s) = 0 \text{ whenever } \beta( a | s) < \epsilon \}$, where $\beta$ is the behavior policy (i.e., the set of policies that have support in the probable regions of the behavior policy). This means that we are allowed to backup from any action distribution supported over the support of the behaviour policy. Previous work~\cite{fujimoto2018off} restricts the choice of actions to be a distribution close to the behaviour policy. 

%%SL.5.22: I don't really understand what the above paragraph is saying. Read literally, it seems to say "prior work does something similar, and in the worst case we are equally bad." That's not very satisfying. Maybe just delete this paragraph, or rephrase if that's not what you meant?
%Now, explain why this does a good job of balancing the terms. Next, we explain how this bound motivates the use of set-constrained backups to reduce accumulation of bootstrapping error. \TODO{explanation about $\delta1$ goes here} -- addressed -- removed this paragraph


% we need to determine how to formulate the appropriate constraint and how to implement so as to back up only values of policies in $\Pi$.
% %%SL.5.20: Rephrase. In order to develop a practical algorithm based on the set-constrained backup, we need to determine how to formulate the appropriate constraint and how to implement so as to back up only values of policies in $\Pi$.
% Intuitively, we would like $\Pi(s)$ for a particular state $s$ to contain only those policies that permit actions within the support of the dataset distribution. Instead of inferring $\Pi$, we use a notion of divergence between the uniform distribution over the support-set of the current policy and the current policy for optimization.  

% %%SL.5.20: Rephrase. Intuitively, we would like $\Pi(s)$ for a particular state $s$ to contain only those policies that permit actions withi
% In order confidence support set perform the $\max$ on the high-over actions from only these policies, we need to define a tractable objective. Instead of inferring the set of policies $\Pi$ we rather resort to specifying a notion of divergence between the set $\mathcal{A}_\varepsilon^\dataset$ and the current policy, $\operatorname{Divergence}(\mathcal{A}^{\mathcal{D}}_{\varepsilon}(s), \pi)$ thereby fitting the problem of inferring $\Pi$ in an optimization setup.
% %%SL.5.20: I don't really understand the above sentence. Try rewriting it to be clearer?
% Next, we move on to presenting our method, which we call \emph{bootstrap error accumulation reduction} (BEAR).

\section{Bootstrapping Error Accumulation Reduction (BEAR)}
\label{sec:bear}
% \vspace{-0.1in}
We now propose a practical actor-critic algorithm (built on the framework of TD3~\cite{fujimoto18addressing} or SAC~\cite{haarnoja2018sac}) that uses distribution-constrained backups to reduce accumulation of bootstrapping error. The key insight is that we can search for a policy with the same support as the training distribution, while preventing accidental error accumulation.
Our algorithm has two main components. Analogous to BCQ~\citep{fujimoto18addressing}, we use $K$ Q-functions and use the minimum Q-value for policy improvement, and design a constraint which will be used for searching over the set of policies $\Pieps$, which share the same support as the behavior policy. Both of these components will appear as modifications of the policy improvement step in actor-critic style algorithms. We also note that policy improvement can be performed with the mean of the K Q-functions, and we found that this scheme works as good in our experiments. 

% n components. Analogous to BBCQ~\citep{fujimoto2018off}, we use two Q-functions and linearly combine their predictions the Q-function for policy improvement, and design a constraint which will be used for searching over the set of policies $\Pieps$, which share the same support as the behaviour policy. Both of these components will appear as modifications of the policy improvement step in actor-critic style algorithms.

We denote the set of Q-functions as: $\hat{Q}_1, \cdots, \hat{Q}_K$.
% compute a conservative estimate of the Q-values: $\frac{1}{K} \sum_{i=1}^K \hat{Q}_i (s, a) - \lambda \sqrt{\operatorname{var}_k \hat{Q}_k(s, a)}$, where $\lambda \in \mathbb{R}^+$ is a hyperparameter. %We use this value as a conservative estimate of the Q-function. This can be derived using Cantelli's inequality. 
Then, the policy is updated to maximize the conservative estimate of the Q-values within $\Pieps$: 
% \vspace{-10pt}
$$ \pi_\phi(\bs) := \max_{\pi \in \Pieps} \expec_{a \sim \pi(\cdot|\bs)} \left[\min_{j=1,..,K} \hat{Q}_j(\bs, \mathbf{a})\right] $$
% \lambda \sqrt{ \operatorname{var_k}\expec_{a \sim \pi(\cdot |s) }[\hat{Q}_k(s, a)]}.$$
% \vspace{-5pt}
% Let $\mathcal{F}_t$ be the sigma-algebra generated by the training procedure until iteration $t$, and let $\operatorname{var}_{t} \hat{Q}(s,a) := \mathbb{E}[(\hat{Q}_t(s, a) - \mathbb{E}[(\hat{Q}_t(s, a) | \mathcal{F}_t))^2|\mathcal{F}_t]$
%%SL.5.20: use mbox. And for clarity, it might be good to indicate what the expectation is over (and use [ instead of ( for E so that parens don't get cluttered). Also, what is up with this (s,a) hanging out at the end? do you mean to put (s,a) inside (after \hat{Q})?
% denote the variance of the Q-function $\hat{Q}_t$, at time $t$ during training. Then, for each state-action pair $(s, a)$, 
% ${Pr (\hat{Q}_t \geq \mathbb{E}(\hat{Q}_t|\mathcal{F}_t) + \sqrt{\frac{(1 - \delta) \operatorname{var}_{t} \hat{Q}_t }{\delta}})  \leq \delta}$
%%SL.5.20: can you state in words what this means for the purpose of this section? also, rhetoric-wise, amybe better state as a theorem (it's kind of obvious, but still) and then after say that this is easy to show via Cantelli's inequality or something?

%%SL.5.20: It's not clear what the concentration bound is actually used for.

 %In the above concentration bound, $\mathbb{E}(\hat{Q}_t|\mathcal{F}_t)$ refers to the true Q-value, which can be obtained given no stochasticity in the procedure.


%%SL.5.20: The logical thread here is broken. What are you doing with set divergence? State the issue first, then th e resolution, else it's hard for the reader to follow.
In practice, the behavior policy $\beta$ is unknown, so we need an approximate way to constrain $\pi$ to $\Pi$. We define a differentiable constraint that approximately constrains $\pi$ to $\Pi$, and then approximately solve the constrained optimization problem via dual gradient descent.  We use the sampled version of maximum mean discrepancy (MMD)~\cite{gretton2012kernel}
%%SL.5.22: Alg names are not capitalized unless they contain proper nouns, put a space after the words and before open paren (I fixed it above, but this issue happens often, please take this comment into account) -- Thanks for pointing this out!
between the unknown behavior policy $\beta$ and the actor $\pi$ because it can be estimated based solely on samples from the distributions. Given samples $x_1, \cdots, x_n \sim P$ and $y_1, \cdots, y_m \sim Q$, the sampled MMD between $P$ and $Q$ is given by:\\
$$\operatorname{MMD}^2(\{x_1, \cdots, x_n\}, \{y_1, \cdots, y_m\}) = \frac{1}{n^2} \sum_{i, i'} k(x_i, x_{i'}) - \frac{2}{nm} \sum_{i, j} k(x_i, y_j) + \frac{1}{m^2} \sum_{j, j'} k(y_j, y_{j'}).
$$
Here, $k(\cdot, \cdot)$ is any universal kernel. In our experiments, we find both Laplacian and Gaussian kernels work well.
%As the $\operatorname{MMD}$ distance does not depend on the density function of either distribution, minimizing it using samples is a reasonable proxy for enforcing that $Q$ lies inside the support of $P$. This is because, 
The expression for MMD does not involve the density of either distribution and it can be optimized directly through samples. Empirically we find that, in the low-intermediate sample regime, the sampled MMD between $P$ and $Q$ is similar to the MMD between a uniform distribution over $P$'s support and $Q$, which makes MMD roughly suited for constraining distributions to a given support set. (See Appendix~\ref{app:mmd} for numerical simulations justifying this approach).

% and hence, we parameterize the set $\mathcal{A}^{\mathcal{D}}_{\varepsilon}(s)$ as a distribution $\pi_{set}(a|s)$ such that $\mathcal{A}(s) := \mathcal{A}^{\pi_{set}}_{\varepsilon}(s) := \{a \in \mathcal{A} | \pi_{set}(a|s) \geq \varepsilon \}$, in other words, $\mathcal{A}(s)$ is the high-confidence support set of the distribution $\pi_{set}$, and we train for a parametric $\pi_{set}$.
%%SL.5.20: I don't actually understand at this point what you are doing. Are you optimizing a neural net that denotes \pi_set? or something else?

% \paragraph{Deriving the update:} Let $\hat{Q}_k$ be the Q-function at the k-th step of the algorithm. Actor-critic Q-learning algorithms maintain a parameterized policy, $\pi_k$ that is updated towards the maximizing the Q-function.
% %-- $\pi_{k+1}(s) := \max_{\pi \in \Delta_{|S|}} E_{a \sim \pi(\cdot|s)} [\hat{Q}_{k}(s, a)]$. 
% In order to reduce the number of moving parts, we let the actor in this case serve both its regular function of maximizing the Q-function while also constraining the action distribution close to $\mathcal{A}^\dataset_\varepsilon$, which is the the task of $\pi_{set}$. We use the bound derived on Q-values to update the policy in the direction of maximizing a conservative estimate of the true Q-value -- $$ \pi_{k+1}(s) := \max_{\pi \in \Delta_{|S|}} E_{a \sim \pi(\cdot|s)} [\hat{Q}_{k}(s, a)] - \lambda \sqrt{ \operatorname{var_k}E_{a \sim \pi(\cdot |s) }[\hat{Q}_k(s, a)]}$$
% %TODO{may want to mention that this amounts to subtracting a constant times the std, which sounds reasonable}
% We still need to account for the problem of specifying support divergence. In order to enforce this constraint, we use a measure of support matching between the training distribution $\Pi$ and the policy $\pi(\cdot|s)$, which we choose to be a sampled version of the Maximum Mean Discrepancy(MMD) Distance between $\Pi$ and the actor $\pi$. Sampled MMD distance between two probability distributions $P$ and $Q$ is given by, $\operatorname{MMD}(P, Q)$, where $x_1, \cdots, x_n \sim P$ and $y_1, \cdots, y_m \sim Q$ is given by:\\
% $$\operatorname{MMD}^2(\{x_1, \cdots, x_n\}, \{y_1, \cdots, y_m\}) = \frac{1}{n^2} \sum_{i, i'} k(x_i, x_{i'}) - \frac{2}{nm} \sum_{i, j} k(x_i, y_j) + \frac{1}{m^2} \sum_{j, j'} k(y_j, y_{j'})
% $$
% When the number of samples $n$ is an intermediate number (4-10), the above sampled objective can also be approximately considered as a distance between a uniform distribution over the high confidence support set of the distribution $P$ and the distribution $Q$ -- therefore, if trained perfectly, $Q$ should have the same support as $P$. That is, $\operatorname{MMD}(P, Q)$ is a reasonable proxy for $\operatorname{MMD}(\mathcal{U}(\mathcal{A}_{\varepsilon}(P)), Q)$. 
% %\TODO{what does it mean MMD between a set and distribution}
% The expression for $\operatorname{MMD}$ does not use the density function of either distribution, thereby making it suited as an approximate way of support matching.

Putting together, the optimization problem in the policy improvement step is
% \vspace{-5pt}
\begin{multline}
    \label{eqn:policy_update}
   \pi_\phi := \max_{\pi \in \Delta_{|S|}} \expec_{\bs \sim \mathcal{D}} \expec_{\mathbf{a} \sim \pi(\cdot|\bs)} \left[\min_{j=1,..,K} \hat{Q}_j(\bs, \mathbf{a})\right] 
%   - \lambda \sqrt{ \operatorname{var_k}\expec_{a \sim \pi(\cdot |s) }[\hat{Q}_k(s, a)]}\\
   \text{~~s.t.~~} \mathbb{E}_{\bs \sim \mathcal{D}} [\operatorname{MMD}(\mathcal{D}(\bs), \pi(\cdot|\bs))] \leq \varepsilon \quad
\end{multline}
where $\varepsilon$ is an approximately chosen threshold. We choose a threshold of $\varepsilon=0.05$ in our experiments. The algorithm is summarized in Algorithm~\ref{algo:bear_ql}. 
% Step 5 of the algorithm performs a stochastic version of the distribution-constrained backup, where Dirac-delta policies $\delta_{a_i}, \cdots, \delta_{a_p},(~\forall~i, \delta_{a_i} \in \Pi)$ are sampled, an expectation of the target Q-value under these Dirac-delta policies is computed and then the maximum value across these policies is backed up as defined by the backup operator. We provide more explanation in Appendix \ref{app:bearql-more}.

\textbf{How does BEAR connect with distribution-constrained backups described in Section 4.1?} Step 5 of the algorithm restricts $\pi_\phi$ to lie in the support of $\beta$. This insight is formally justified in Theorems 4.1 \& 4.2 ($C(\Pi_\varepsilon)$ is bounded). Computing distribution-constrained backup exactly by maximizing over $\pi \in \Pi_\varepsilon$ is intractable in practice. As an approximation, we sample Dirac policies in the support of $\beta$ (Alg 1, Line 5) and perform empirical maximization to compute the backup. As the maximization is performed over a \textit{narrower} set of Dirac policies ($\{ \delta_{\mathbf{a}_i} \} \subseteq \Pi_\varepsilon$), the bound in the above Theorem still holds. Empirically, we show in Section~\ref{sec:experiments} that this approximation is sufficient to outperform previous methods. This connection is briefly discussed in Appendix C.2.
% $\operatorname{var}(\hat{Q}_k(s, a)) \approx \frac{1}{M} \sum_{i=1}^{M} (\hat{Q}_{\theta_i, k}(s, a) - \bar{Q}_{\theta, k}(s, a))^2$, where $\bar{Q}_{\theta, k}(s, a) = \frac{1}{M} \sum_{i=1}^{M} \hat{Q}_{\theta_i, k}(s, a)$ is the sample mean of the ensemble. 

%AK.05.15: Note to Sergey: this is the actor-critic version, optional depends on results.
% Another variant of the above approach can be where this single policy improvement step can be decomposed into two decoupled steps -- (1) Learning a policy $\pi_{set}$, whose high-confidence set defines the support set $\mathcal{A}_{\varepsilon}(s)$ at a state $s$, by minimizing the sampling error in $\hat{Q}_k$ and accounting for the deviation from the dataset, and then, (2) Learning to maximize the expected Q-function $\hat{Q}_k$ on this set $\mathcal{A}_{\varepsilon}(s)$, in practice obtained by sampling from $\pi_{set}$. In practice, we found using Equation~\ref{eqn:policy_update} working better than the latter approach and hence, we stick to this formulation for our experiments. The overall algorithm is summarized in Algorithm~\ref{alg:q_learning}, and the actor-critic version is described in Algorithm~\ref{alg:actor_critic}.   
% \vspace{-5pt}
\begin{algorithm}[h]
\small
\caption{Q-learning variant of BEAR (BEAR)}
\label{alg:q_learning}
\begin{algorithmic}[1]
    \State Dataset $\mathcal{D}$, target network rate $\tau$, batch size $N$, sampled actions for MMD $n$, minimum $\lambda$
    \State Initialize Q-ensemble $\{Q_{\theta_i} \}_{i=1}^{K}$, actor $\pi_{\phi}$, multiplier $\alpha$, target networks $\{ Q_{\theta'_i} \}_{i=1}^K$, and a target actor $\pi_{\phi'}$, with $\phi' \leftarrow \phi, \theta'_i \leftarrow \theta_i$
    \ForAll{$t$ in \{1, \dots, N\}}
        \State Sample mini-batch of transitions $(\bs, \mathbf{a}, r, \bs') \sim \mathcal{D}$\\
        \textbf{Q-update:}
            \State Sample $p$ action samples, $\{\mathbf{a}_i \sim \pi_{\phi'}(\cdot|\bs')\}_{i=1}^p$
            \State Define $y(\bs, \mathbf{a}) := \max_{\mathbf{a}_i} [ \lambda \min_{j=1,..,K} Q_{\theta'_j}(\bs', \mathbf{a}_i) + (1 - \lambda) \max_{j=1,..,K} Q_{\theta'_j}(\bs', \mathbf{a}_i)]$
            \State $\forall i, \theta_i \leftarrow \arg \min_{\theta_i} (Q_{\theta_i}(\bs, \mathbf{a}) - (r + \gamma y(\bs, \mathbf{a})))^2$\\
        \textbf{Policy-update:}
        \State Sample actions $\{ \hat{\mathbf{a}}_i \sim \pi_{\phi}(\cdot | \bs) \}_{i=1}^{m}$ and $\{ \mathbf{a}_j \sim \mathcal{D}(\bs)\}_{j=1}^{n}$. % $n$ preferably an intermediate integer(1-10)
        \State Update $\phi$, $\alpha$ by minimizing Equation~\ref{eqn:policy_update} with Lagrange multiplier $\alpha$.
        \State \textbf{Update Target Networks: } $\theta'_i \leftarrow \tau \theta_i + (1 - \tau)\theta'_i$; $\phi' \leftarrow \tau \phi + (1 -\tau) \phi'$ 
    \EndFor
\end{algorithmic}
\label{algo:bear_ql}
\end{algorithm}

In summary, the actor is updated towards maximizing the Q-function while still being constrained to remain in the valid search space defined by $\Pieps$. The Q-function uses actions sampled from the actor to then perform distribution-constrained Q-learning, over a reduced set of policies. {At test time, we sample $p$ actions from $\pi_\phi(\bs)$ and the Q-value maximizing action out of these is executed in the environment.}  %The maximization step in the actor-update empirically helps, but can be coupled with maximization in Step 5. Similar to \cite{fujimoto2018off} we use a soft-minimum to compute target values for updating Q-functions. 
Implementation and other details are present in Appendix \ref{app:additional_details}.
%%SL.5.22: Remember to fill this in.

% \begin{algorithm}[H]
% \small
% \caption{BEAR Actor-Critic}
% \label{alg:actor_critic}
% \begin{algorithmic}[1]
%     \INPUT: Dataset $\mathcal{D}$, target network update rate $\tau$, mini-batch size $N$, sampled actions for MMD $n$, minimum $\lambda$, policy gradient clipping constants $\beta_1, \beta_2; \beta_1 \leq \beta_2$, MMD threshold constant $\varepsilon$
%     \STATE Initialize Q-ensemble $\{Q_{\theta_i} \}_{i=1}^{M}$, actor $\pi_{\phi}$, set-determining policy $\pi_{set}$, Lagrange multiplier $\alpha$, target networks $\{ Q_{\theta'_i} \}_{i=1}^M$, and a target actor $\pi_{\phi'}$, with $\phi' \leftarrow \phi, \theta'_i \leftarrow \theta_i$
%     \FOR{$t$ in \{1, \dots, N\}}
%         \STATE Sample mini-batch of transitions $(s, a, r, s') \sim \mathcal{D}$\\
%         \textbf{Q-update:}
%             \STATE Sample $m$ action samples, $\{a_i \sim \pi_{\phi'}(\cdot|s')\}_{i=1}^n$
%             \STATE Define $y = \frac{1}{m} \sum_{a_i} [ \lambda \min_{j=1,..,M} Q_{\theta'_j}(s', a_i) + (1 - \lambda) \max_{j=1,..,M} Q_{\theta'_j}(s', a_i)]$
%             \STATE $\forall i, \theta_i \leftarrow \arg \min_{\theta_i} (Q_{\theta_i}(s, a) - (r + \gamma y))^2$\\
%         \textbf{Set-update and Actor-update:}
%         \STATE Sample actions $A_1(s) \equiv \{ \hat{a}_i \sim \pi_{set}(\cdot | s) \}_{i=1}^{m}$ and $A_2(s) \equiv \{ a_j \sim \mathcal{D}(s)\}_{j=1}^{n}$, $n << m$
%         \STATE Update $\pi_{set}, \alpha$: $$ \pi_{set}, \alpha \leftarrow \arg \min_{\pi_{set}} \max_{\alpha \geq 0} \sqrt{\frac{(1 - \delta) \operatorname{var_k}E_{a \sim \pi_{set}(\cdot |s) }[\hat{Q}_k(s, a)]}{\delta}} + \alpha \mathbb{E}_{s \sim \mathcal{D}} ([\operatorname{MMD}(A_1, A_2)] -  \varepsilon) $$
%         \STATE Update $\phi$ using Importance Sampled Policy Gradient: 
%         $$ \pi_{\phi} \leftarrow  \max_{\pi_{\phi}} \mathbb{E}_{s \sim \mathcal{D}} \mathbb{E}_{a \sim \pi_{set}(\cdot|s)} \Big( \Big[ \frac{\pi_\phi(a|s)}{\pi_{set}(a|s)} \Big]_{\beta_1}^{\beta_2} Q(s, a) \Big)$$
%         \STATE \textbf{Update Target Networks: } $\theta'_i \leftarrow \tau \theta_i + (1 - \tau)\theta'_i$; $\phi' \leftarrow \tau \phi + (1 -\tau) \phi'$ 
%     \ENDFOR
% \end{algorithmic}
% \end{algorithm}


% Let $\bar{Q}(\cdot, \cdot)$ be the delayed target network, and $Q(\cdot, \cdot)$ be the current Q-function. Define $d_i$ be the the TD error for the $i^{th}$ datapoint.
% $$
% d_{i}(Q ; \bar{Q}, \pi)=R_{t}+\gamma \bar{Q}\left(s'_{i}, \pi_{set} \left(s'_i\right)\right)-Q\left(s_{i}, a_{i}\right)

% $$
% Further we define the empirical loss function by
% $$
% \hat{L}_{N}(Q ; \bar{Q}, \pi)=\frac{1}{N} \sum_{t=1}^{N} \frac{d_{t}^{2}(Q ; \bar{Q}, \pi_{set})}{\lambda(\mathcal{A})}
% $$
% where normalization $\lambda{\mathcal{A}}$ is introduced for mathematical convenience. Then, each policy evaluation step can be written as:  

% If we solely backup from actions present in our dataset, there is no way the algorithm can perform better than the policy that collected the data. The capacity of Q-learning and other ADP algorithms to ``stitch'' together performant sub-trajectories is lost. Hence, our method does allow the agent to backup from actions that occur outside the dataset, while still being constrained to not go farther away from the support of $\mathcal{D}$. In principle, a measure of distance from a given dataset can only be obtained using Bayesian Approaches (?). In practice, we use the variance of the ensemble as a measure to approximately quantify closeness to the support set. Our overall approach is described in the next paragraph.




% Our problem setting does not allow any interaction with the environment, and only lets us use the dataset $\mathcal{D}$. Since we see a limited subset of state-action pairs from the environment, the expected estimate of the Q-function conditioned on all training history in our case, $\mathbb{E}(\hat{Q}|\mathcal{F}_t)$, is biased. \TODO{aviral: finish this argument} 

% We train an ensemble of $N$ parametric Q-functions, $Q_{\theta_1}, \cdots, Q_{\theta_N}$ by using bootstrap masks on the data points of the dataset $\mathcal{D}$. This is done to simulate epistemic variance. To make sure that the actions chosen for backing up Q-functions are valid, we learn a set selection policy, $\pi_{set}$ -- a policy that can provide high densities to actions that don't propagate errors.   
% \section{Experimental Evaluation}
\label{sec:experiments}
In our experiments, we study how BEAR performs when learning from static offline data on a variety of continuous control benchmark tasks. We evaluate our algorithm in three settings: when the dataset $\dataset$ is generated by \textbf{(1)} a completely random behavior policy, \textbf{(2)} a partially trained, medium scoring policy, and \textbf{(3)} an optimal policy. Condition \textbf{(2)} is of particular interest, as it captures many common use-cases in practice, such as learning from imperfect demonstration data (e.g., of the sort that are commonly available for autonomous driving~\cite{DBLP:conf/iclr/GaoXLYLD18}), or reusing previously collected experience during off-policy RL. We compare our method to several prior methods: a baseline actor-critic algorithm (TD3), the BCQ  algorithm~\citep{fujimoto2018off}, which aims to address a similar problem, as discussed in Section~\ref{sec:Problem Description}, KL-control~\citep{jacques19way} (which solves a KL-penalized RL problem similarly to maximum entropy RL), a static version of DQfD~\citep{hester2018dqfd} (where a constraint to upweight Q-values of state-action pairs observed in the dataset is added as an auxiliary loss on top a regular actor-critic algorithm), and a behavior cloning (BC) baseline, which simply imitates the data distribution. This serves to measure whether each method actually performs effective RL, or simply copies the data. We report the average evaluation return over 5 seeds of the policy given by the learned algorithm, in the form of a learning curve as a function of number of gradient steps taken by the algorithm. These samples are only collected for evaluation, and are not used for training.

\subsection{Performance on Medium-Quality Data}

We first discuss the evaluation of condition with ``mediocre'' data \textbf{(2)}, as this condition resembles the settings where we expect training on offline data to be most useful. We collected one million transitions from a partially trained policy, so as to simulate imperfect demonstration data or data from a mediocre prior policy.
In this scenario, we found that BEAR consistently outperforms BCQ~\cite{fujimoto2018off} and a na\"ive off-policy RL baseline (TD3) (by large margins), as shown in Figure~\ref{fig:mediocre}. This scenario is the most relevant from an application point of view, as access to optimal data may not be feasible, and random data might have inadequate exploration to efficient learn a good policy. We also evaluate the accuracy with which the learned Q-functions predict actual policy returns. These trends are provided in Appendix~\ref{app:q_vs_mc}. 
% Note that the performance of BCQ often tracks the performance of the BC baseline, suggesting that BCQ primarily imitates the data. 
Our KL-control baseline uses automatic temperature tuning~\citep{haarnoja2018sac}. We find that KL-control usually performs similar or worse to BC, whereas DQfD tends to diverge, and often exhibits a huge variance across different runs (for example, HalfCheetah-v2 environment).  

\begin{figure}
    \centering
    \begin{subfigure}[t]{0.23\textwidth}
        \centering
        \includegraphics[width=0.99\linewidth]{chapters/bear/images/images_camera_ready/cheetah_mediocre_camera_ready.pdf}
    \end{subfigure}
    \begin{subfigure}[t]{0.23\textwidth}
        \centering
        \includegraphics[width=0.99\linewidth]{chapters/bear/images/images_camera_ready/walker_mediocre_camera_ready.pdf}
        % \caption{}
    \end{subfigure}
    ~
    \begin{subfigure}[t]{0.23\textwidth}
        \centering
        \includegraphics[width=0.99\linewidth]{chapters/bear/images/images_camera_ready/hopper_mediocre_camera_ready.pdf}
        % \caption{}
    \end{subfigure}
    ~
    \begin{subfigure}[t]{0.23\textwidth}
        \centering
        \includegraphics[width=0.99\linewidth]{chapters/bear/images/images_camera_ready/ant_mediocre_camera_ready.pdf}
        % \caption{}
    \end{subfigure}
    \caption{\label{fig:mediocre} \footnotesize Average performance of BEAR, BCQ, Na\"ive RL and BC on medium-quality data averaged over 5 seeds. BEAR outperforms both BCQ and Na\"ive RL. Average return over the training data is indicated by the magenta line. One step on the x-axis corresponds to 1000 gradient steps.}
\end{figure}

% \begin{figure*}[t!]
%     \centering
%     \vspace{-0.05in}
%     \begin{subfigure}[t]{0.23\textwidth}
%         \centering
%         \includegraphics[width=0.99\linewidth]{chapters/bear/images/cheetah_mediocre.pdf}
%     \end{subfigure}
%     \begin{subfigure}[t]{0.23\textwidth}
%         \centering
%         \includegraphics[width=0.99\linewidth]{chapters/bear/images/walker_mediocre_final_again.pdf}
%         % \caption{}
%     \end{subfigure}
%     ~
%     \begin{subfigure}[t]{0.23\textwidth}
%         \centering
%         \includegraphics[width=0.99\linewidth]{chapters/bear/images/hopper_mediocre_final_new.pdf}
%         % \caption{}
%     \end{subfigure}
%     ~
%     \begin{subfigure}[t]{0.23\textwidth}
%         \centering
%         \includegraphics[width=0.99\linewidth]{chapters/bear/images/ant_mediocre_final.pdf}
%         % \caption{}
%     \end{subfigure}
%     \caption{ \footnotesize Average performance of BEAR, BCQ, Na\"ive RL and BC on medium-quality data averaged over 5 seeds. BEAR outperforms both BCQ and Na\"ive RL. Average return over the training data is indicated by the magenta line. One step on the x-axis corresponds to 1000 gradient steps.}
%     \label{fig:mediocre}
%     \vspace{-0.1in}
% \end{figure*}

% \vspace{-5pt}
\subsection{Performance on Random and Optimal Datasets}
In Figure~\ref{fig:optimal_random}, we show the performance of each method when trained on data from a random policy (top) and a near-optimal policy (bottom). In both cases, our method BEAR achieves good results, consistently exceeding the average dataset return on random data, and matching the optimal policy return on optimal data. Na\"{i}ve RL also often does well on random data. For a random data policy, all actions are in-distribution, since they all have equal probability. This is consistent with our hypothesis that OOD actions are one of the main sources of error in off-policy learning on static datasets. The prior BCQ method~\cite{fujimoto2018off} performs well on optimal data but performs poorly on random data, where the constraint is too strict. These results show that BEAR is more robust to the dataset composition, and can learn consistently in a variety of settings. We find that KL-control and DQfD can be unstable in these settings.  

{Finally, in Figure \ref{fig:humanoid}, we  show that BEAR outperforms other considered prior methods in the challenging Humanoid-v2 environment as well, in two cases -- Medium-quality data and random data.}

\begin{figure}
        \centering
        \includegraphics[width=0.4\linewidth]{chapters/bear/images/images_camera_ready/humanoid_mediocre_camera_ready.pdf}
       ~
        \includegraphics[width=0.4\linewidth]{chapters/bear/images/images_camera_ready/humanoid_random_camera_ready.pdf}
      \caption{\label{fig:humanoid} \footnotesize Performance of BEAR, BCQ, Na\"ive RL and BC on medium-quality (left) and random (right) data in the Humanoid-v2 environment. Note that BEAR outperforms prior methods.}
\end{figure}

%With random data, BCQ is expected to not perform well, as it constrains actions to the actions seen in the dataset at a particular state. On the other hand, a na\"ive off-policy RL algorithm is expected to perform well in these settings. In Figure~\ref{fig:optimal_random}, we show that BEAR outperforms \cite{fujimoto2018off} drastically while still performing comparable to the na\"ive RL algorithm in HalfCheetah-v2, and Hopper-v2, and outperforming it on Walker2d-v2 and Ant-v2 tasks. On optimal data, as the suboptimality bias is small, the best solution is to imitate the behavior policy. BEAR learns to imitate the behavior policy, and maintains stably there. In this setting, na\"ive-RL algorithm fails to learn (and mostly converges to the minimum possible reward, that can be obtained in the environment). Overall, BEAR is robust to the dataset composition, and can consistently perform in all settings -- mediocre, random and optimal data. Figure~\ref{fig:optimal_random} summarizes the average evaluation return of the learned policy as a function of training steps.

% show that DQNs are empirically less stable than off-policy actor critic algorithms(TD3). In all cases, note that BCQ~\cite{fujimoto18addressing} often ends up imitating the baseline, which explains the reason for poor performance on random data, and fast convergence to optimal performance on optimal data. However, the versatility of BEAR demonstrates its use as a practical algorithm irrespective of the quality of the dataset $\dataset$. We also examine the amount of deviation from the true MC-returns of the actor and the learned estimate of the Q-value, which again suggests that BEAR can learn reliable estimates of Q-values without excessive overestimation as observed in TD3, while achieving better expected return performance over BCQ. \TODO{Q vs MC figure left}

\begin{figure}
    \centering
    \begin{subfigure}[t]{0.23\textwidth}
        \centering
        % \includegraphics[width=0.99\linewidth]{chapters/bear/images/cheetah_mediocre_final.pdf}
        \includegraphics[width=0.99\linewidth]{chapters/bear/images/images_camera_ready/cheetah_random_final_camera_ready.pdf}
        \includegraphics[width=0.99\linewidth]{chapters/bear/images/images_camera_ready/cheetah_optimal_camera_ready_new.pdf}
        % \caption{}
    \end{subfigure}%
    ~ 
    \begin{subfigure}[t]{0.23\textwidth}
        \centering
        % \includegraphics[width=0.99\linewidth]{chapters/bear/images/walker_mediocre_final.pdf}
        \includegraphics[width=0.99\linewidth]{chapters/bear/images/images_camera_ready/walker_random_camera_ready.pdf}
        \includegraphics[width=0.99\linewidth]{chapters/bear/images/images_camera_ready/walker_optimal_camera_ready.pdf}
        % \caption{}
    \end{subfigure}
    ~
    \begin{subfigure}[t]{0.23\textwidth}
        \centering
        % \includegraphics[width=0.99\linewidth]{chapters/bear/images/hopper_mediocre_final.pdf}
        \includegraphics[width=0.99\linewidth]{chapters/bear/images/images_camera_ready/hopper_random_camera_ready.pdf}
        \includegraphics[width=0.99\linewidth]{chapters/bear/images/images_camera_ready/hopper_optimal_camera_ready.pdf}
        % \caption{}
    \end{subfigure}
    ~
    \begin{subfigure}[t]{0.23\textwidth}
        \centering
        % \includegraphics[width=0.99\linewidth]{chapters/bear/images/ant_mediocre_final.pdf}
        \includegraphics[width=0.99\linewidth]{chapters/bear/images/images_camera_ready/ant_random_camera_ready.pdf}
        \includegraphics[width=0.99\linewidth]{chapters/bear/images/images_camera_ready/ant_optimal_camera_ready.pdf}
        % \caption{}
    \end{subfigure}
    \caption{\label{fig:optimal_random} \footnotesize Average performance of BEAR, BCQ, Na\"ive RL and BC on random data (top row) and optimal data (bottom row) over 5 seeds. BEAR is the only algorithm capable of learning in both scenarios. Na\"{i}ve RL cannot handle optimal data, since it does not illustrate mistakes, and BCQ favors a behavioral cloning strategy (performs quite close to behavior cloning in most cases), causing it to fail on random data. Average return over the training dataset is indicated by the dashed magenta line.}
\end{figure}

% \begin{figure*}[t!]
% \vspace{-0.1in}
%     \centering
%     \begin{subfigure}[t]{0.23\textwidth}
%         \centering
%         % \includegraphics[width=0.99\linewidth]{chapters/bear/images/cheetah_mediocre_final.pdf}
%         \includegraphics[width=0.99\linewidth]{chapters/bear/images/cheetah_random_final.pdf}
%         \includegraphics[width=0.99\linewidth]{chapters/bear/images/cheetah_optimal_final.pdf}
%         % \caption{}
%     \end{subfigure}%
%     ~ 
%     \begin{subfigure}[t]{0.23\textwidth}
%         \centering
%         % \includegraphics[width=0.99\linewidth]{chapters/bear/images/walker_mediocre_final.pdf}
%         \includegraphics[width=0.99\linewidth]{chapters/bear/images/walker_random_final.pdf}
%         \includegraphics[width=0.99\linewidth]{chapters/bear/images/walker_optimal_final.pdf}
%         % \caption{}
%     \end{subfigure}
%     ~
%     \begin{subfigure}[t]{0.23\textwidth}
%         \centering
%         % \includegraphics[width=0.99\linewidth]{chapters/bear/images/hopper_mediocre_final.pdf}
%         \includegraphics[width=0.99\linewidth]{chapters/bear/images/hopper_random_final.pdf}
%         \includegraphics[width=0.99\linewidth]{chapters/bear/images/hopper_optimal_final.pdf}
%         % \caption{}
%     \end{subfigure}
%     ~
%     \begin{subfigure}[t]{0.23\textwidth}
%         \centering
%         % \includegraphics[width=0.99\linewidth]{chapters/bear/images/ant_mediocre_final.pdf}
%         \includegraphics[width=0.99\linewidth]{chapters/bear/images/ant_random_final.pdf}
%         \includegraphics[width=0.99\linewidth]{chapters/bear/images/ant_optimal_final.pdf}
%         % \caption{}
%     \end{subfigure}
%     \caption{\footnotesize Average performance of BEAR, BCQ, Na\"ive RL and BC on random data (top row) and optimal data (bottom row) over 5 seeds. BEAR is the only algorithm capable of learning in both scenarios. Na\"{i}ve RL cannot handle optimal data, since it does not illustrate mistakes, and BCQ favors a behavioral cloning strategy (performs quite close to behavior cloning in most cases), causing it to fail on random data. Average return over the training dataset is indicated by the dashed magenta line.}
%     \label{fig:optimal_random}
%     \vspace{-0.1in}
% \end{figure*}

% \begin{figure*}[t!]
%     \centering
%     \begin{subfigure}[t]{0.31\textwidth}
%         \centering
%         \includegraphics[width=0.99\linewidth]{chapters/bear/images/random_halfcheetah.png}
%         \caption{ }
%     \end{subfigure}%
%     ~ 
%     \begin{subfigure}[t]{0.31\textwidth}
%         \centering
%         \includegraphics[width=0.99\linewidth]{chapters/bear/images/mediocre_walker.png}
%         \caption{ }
%     \end{subfigure}
%     ~
%     \begin{subfigure}[t]{0.31\textwidth}
%         \centering
%         \includegraphics[width=0.99\linewidth]{chapters/bear/images/random_hopper.png}
%         \caption{ }
%     \end{subfigure}
%     \caption{Random Data}
% \end{figure*}

% \begin{figure*}[t!]
%     \centering
%     \begin{subfigure}[t]{0.23\textwidth}
%         \centering
%         \includegraphics[width=0.99\linewidth]{chapters/bear/images/cheetah_random.pdf}
%         \caption{ }
%     \end{subfigure}%
%     ~ 
%     \begin{subfigure}[t]{0.23\textwidth}
%         \centering
%         \includegraphics[width=0.99\linewidth]{chapters/bear/images/walker_random.pdf}
%         \caption{ }
%     \end{subfigure}
%     ~
%     \begin{subfigure}[t]{0.23\textwidth}
%         \centering
%         \includegraphics[width=0.99\linewidth]{chapters/bear/images/hopper_random.pdf}
%         \caption{ }
%     \end{subfigure}
%     ~
%     \begin{subfigure}[t]{0.23\textwidth}
%         \centering
%         \includegraphics[width=0.99\linewidth]{chapters/bear/images/ant_mediocre.pdf}
%         \caption{ }
%     \end{subfigure}
%     \caption{Random Data}
% \end{figure*}

% \begin{figure*}[t!]
%     \centering
%     \begin{subfigure}[t]{0.23\textwidth}
%         \centering
%         \includegraphics[width=0.99\linewidth]{chapters/bear/images/cheetah_optimal.pdf}
%         \caption{ }
%     \end{subfigure}%
%     ~ 
%     \begin{subfigure}[t]{0.23\textwidth}
%         \centering
%         \includegraphics[width=0.99\linewidth]{chapters/bear/images/walker_optimal.pdf}
%         \caption{ }
%     \end{subfigure}
%     ~
%     \begin{subfigure}[t]{0.23\textwidth}
%         \centering
%         \includegraphics[width=0.99\linewidth]{chapters/bear/images/hopper_optimal.pdf}
%         \caption{ }
%     \end{subfigure}
%     ~
%     \begin{subfigure}[t]{0.23\textwidth}
%         \centering
%         \includegraphics[width=0.99\linewidth]{chapters/bear/images/ant_optimal.pdf}
%         \caption{ }
%     \end{subfigure}
%     \caption{Optimal Data}
% \end{figure*}

\iffalse

\vspace{-5pt}
\subsection{Analysis of BEAR}
\label{subsec:ablations}
In this section, we aim to analyze different components of our method via an ablation study. Our first ablation studies the support constraint discussed in Section~\ref{sec:bear}, which uses MMD to measure support. We replace it with a more standard KL-divergence distribution constraint, which measures similarity in density. 
% \TODO{how is this done if we don't have the behavior policy? do you assume access for this study? -- we train a behavior policy on the data and then constrain to that with KL or MMD (so it should be a fair comparison)}
Our hypothesis is that this should provide a more conservative constraint, since matching distributions is not necessary for matching support. KL-divergence performs well in some cases, such as with optimal data, but as shown in Figure~\ref{fig:ablations}, it performs worse than MMD on medium-quality data. Even when KL-divergence is hand tuned fully, so as to prevent instability issues it still performs worse than a not-well tuned MMD constraint. We provide the results for this setting in the Appendix. We also vary the number of samples $n$ that are used to compute the MMD constraint. We find that smaller n ($\approx$ 4 or 5) gives better performance. Although the difference is not large, consistently better performance with 4 samples leans in favour of our hypothesis that an intermediate number of samples works well for support matching, and hence is less restrictive.

\fi
% Next, we study whether using a conservative Q-value estimate by subtracting the variance in the ensemble helps with learning. As shown in Figure~\ref{fig:ablations}, the conservative estimate 
%  makes a comparatively smaller difference than the use of MMD, providing some benefit on one task, while somewhat hurting performance on another.
% The ensemble produces more conservative estimates, which can result in underestimation in practice, and prevent overestimation divergence.

%The third factor in the ablation study is whether the usage of conservative estimates of Q-values subtracting the variance of the $Q$-ensemble helps. We find that on Hopper, usage of ensembles helps, whereas on Walker2d using ensembles hurts as the algorithm tends to underestimate Q-values. Figure~\ref{subfig:ensembles_ablation} demonstrates the average trend on 2 environments -- Hopper and Walker.




% \subsection{Generalization performance on datasets collected using a mixture of markovian policies.}
% We finally test our BEAR method in the case where the dataset $\dataset$ cannot be generated by a \TODO{Sergey and George: What's your opinion on having this section about non-markovian policies? This was one reason why Fujimoto got rejected. {https://openreview.net/forum?id=S1zlmnA5K7\&noteId=HJeQ-p0F2Q} }
% \TODO{Exp List: - Ant multiple + Point Mass multiple
%                 - with BCQ, BEAR, BEAR with n=1, BEAR without ensemble}

% \begin{figure}
%     \centering
%     \includegraphics{}
%     \caption{Caption}
%     \label{fig:my_label}
% \end{figure}
% \section{Discussion and Future Work}
\vspace{-5pt}
The goal in our work was to study off-policy reinforcement learning with static datasets. We theoretically and empirically analyze how error propagates in off-policy RL due to the use of out-of-distribution actions for computing the target values in the Bellman backup. Our experiments suggest that this source of error is one of the primary issues afflicting off-policy RL: increasing the number of samples does not appear to mitigate the degradation issue (Figure~\ref{fig:divergence}), and training with na\"{i}ve RL on data from a random policy, where there are no out-of-distribution actions, shows much less degradation than training on data from more focused policies (Figure~\ref{fig:optimal_random}). Armed with this insight, we develop a method for mitigating the effect of out-of-distribution actions, which we call BEAR-QL. BEAR-QL constrains the backup to use actions that have non-negligible support under the data distribution, but without being overly conservative in constraining the learned policy. We observe experimentally that BEAR-QL achieves good performance across a range of tasks, and across a range of dataset compositions, learning well on random, medium-quality, and expert data.

% \vspace{-0.15in}
\begin{wrapfigure}{r}{0.51\textwidth}
        \includegraphics[width=0.48\linewidth]{chapters/bear/images/kl_vs_mmd_ablation_final.pdf}
       ~
        \includegraphics[width=0.48\linewidth]{chapters/bear/images/num_samples_ablation.pdf}
        % ~
        % \includegraphics[width=0.31\linewidth]{images/ensembles_ablation_final.pdf}
      \caption{\footnotesize Average return (averaged Hopper-v2 and Walker2d-v2) as a function of train steps for ablation studies from Section~\ref{subsec:ablations}. (a) MMD constrained optimization is more stable and leads to better returns, (b) 4 sample MMD is more performant than 10.}
    %   and (c) Ensemble variance has mixed benefit.}
      \label{fig:ablations}
\vspace{-10pt}
\end{wrapfigure}

While BEAR-QL substantially stabilizes off-policy RL, we believe that this problem merits further study. One limitation of our current method is that, although the learned policies are more performant than those acquired with na\"{i}ve RL, performance sometimes still tends to degrade for long learning runs. An exciting direction for future work would be to develop an early stopping condition for RL, perhaps by generalizing the notion of validation error to reinforcement learning. {A limitation of approaches that perform constrained-action selection is that they can be overly conservative when compared to methods that constrain state-distributions directly, especially with datasets collected from mixtures of policies. We leave it to future work to design algorithms that can directly constrain state distributions. A theoretically robust method for support matching efficiently in high-dimensional continuous action spaces is a question for future research. Perhaps methods from outside RL, predominantly used in domain adaptation, such as using asymmetric f-divergences~\citep{wu19domain} can be used for support restriction.} Another promising future direction is to examine how well BEAR-QL can work on large-scale off-policy learning problems, of the sort that are likely to arise in domains such as robotics, autonomous driving, operations research, and commerce. If RL algorithms can learn effectively from large-scale off-policy datasets, reinforcement learning can become a truly data-driven discipline, benefiting from the same advantage in generalization that has been seen in recent years in supervised learning fields, where large datasets have enabled rapid progress in terms of accuracy and generalization~\cite{imagenet_cvpr09}.

\section*{Acknowledgements}
We thank Kristian Hartikainen for sharing implementations of RL algorithms and for help in debugging certain issues. We thank Matthew Soh for help in setting up environments. We thank Aurick Zhou, Chelsea Finn, Abhishek Gupta, Kelvin Xu and Rishabh Agarwal for informative discussions. We thank Ofir Nachum for comments on an earlier draft of this paper. We thank Google, NVIDIA, and Amazon for providing computational resources. This research was supported by Berkeley DeepDrive, JPMorgan Chase \& Co., NSF IIS-1651843 and IIS-1614653, the DARPA Assured Autonomy program, and ARL DCIST CRA W911NF-17-2-0181.

% Q-learning methods are a common class of algorithms used in reinforcement learning (RL). However, their behavior with function approximation, especially with neural networks, is poorly understood theoretically and empirically. In this work, we aim to experimentally investigate potential issues in Q-learning, by means of a "unit testing" framework where we can utilize oracles to disentangle sources of error. 
% Specifically, we investigate questions related to function approximation, sampling error and nonstationarity, and where available, verify if trends found in oracle settings hold true with deep RL methods.
% We find that large neural network architectures have many benefits with regards to learning stability; offer several practical compensations for overfitting; and develop a novel sampling method based on explicitly compensating for function approximation error that yields fair improvement on high-dimensional continuous control domains. 

\vspace{-0.4cm}
\begin{AIbox}{\large{\textbf{Abstract}}}
\vspace{4mm}
In this chapter, we will discuss our notion and background definition, followed by a discussion of the problem statement of offline reinforcement learning (RL).   
\vspace{2mm}
\end{AIbox}


\section{Reinforcement Learning Preliminaries}
\label{sec:rl_prelims}

In this section, we will define reinforcement learning (RL) concepts, following standard definitions~\citep{suttonrlbook}. RL addresses the problem of learning to control a dynamical system. The dynamical system is fully defined by a fully-observed or partially-observed Markov decision process (MDP). We only consider problems where the dnynamical system is defined by a fully-observed MDP in this dissertation. 

\begin{definition}[Markov decision process]
A Markov decision process is defined as a tuple \mbox{$\mdp = (\states,\actions,\transitions,\initstate,\reward,\discount)$}, where $\states$ is a set of states $\bs \in \states$, which may be either discrete or continuous (i.e., multi-dimensional vectors), $\actions$ is a set of actions $\mathbf{a} \in \actions$, which similarly can be discrete or continuous, $\transitions$ defines a conditional probability distribution of the form $\transitions(\bs_{t+1} | \bs_t, \mathbf{a}_t)$ that describes the dynamics of the system,\footnote{We will sometimes use time subscripts (i.e., $\bs_{t+1}$ follows $\bs_t$), and sometimes ``prime'' notation (i.e., $\bs'$ is the state that follows $\bs$). Explicit time subscripts can help clarify the notation in finite-horizon settings, while ``prime'' notation is simpler in infinite-horizon settings where absolute time step indices are less meaningful.} $\initstate$ defines the initial state distribution $\initstate(\bs_0)$, $\reward : \states \times \actions \rightarrow \real$ defines a reward function, and $\gamma \in (0, 1]$ is a scalar discount factor.
\end{definition}

% We will use the fully-observed formalism in most of this article, though the definition for the partially observed Markov decision process (POMDP) is also provided for completeness. The MDP definition can be extended to the partially observed setting as follows:

% \begin{definition}[Partially observed Markov decision process]
% The partially observed Markov decision process is defined as a tuple \mbox{$\mdp = (\states,\actions,\observations,\transitions,\initstate,\obsfunc,\reward,\discount)$}, where $\states$, $\actions$, $\transitions$, $\initstate$, $\reward$, and $\discount$ are defined as before, $\observations$ is a set of observations, where each observation is given by $\bo \in \observations$, and $\obsfunc$ is an emission function, which defines the distribution $\obsfunc(\bo_t | \bs_t)$.
% \end{definition}

The final goal in a reinforcement learning problem is to learn a policy, which defines a distribution over actions conditioned on states, $\policy(\mathbf{a}_t | \bs_t)$. From these definitions, we can derive the \emph{trajectory distribution}. The trajectory is a sequence of states and actions of length $H$, given by $\tau = (\bs_0, \mathbf{a}_0, \dots, \bs_H, \mathbf{a}_H)$, where $H$ may be infinite. The trajectory distribution $p_\policy$ for a given MDP $\mdp$ and policy $\policy$ is given by
\[
p_\policy(\traj) = \initstate(\bs_0) \prod_{t=0}^H \policy(\mathbf{a}_t | \bs_t) \transitions(\bs_{t+1} | \bs_t, \mathbf{a}_t).
\]
% This definition can easily be extended into the partially observed setting by including the observations $\bo_t$ and emission function $\obsfunc(\bo_t | \bs_t)$. 
The reinforcement learning objective, $J(\policy)$, can then be written as an expectation under this trajectory distribution:
\begin{equation}
J(\policy) = \E_{\traj \sim p_\policy(\traj)}\left[
\sum_{t=0}^H \discount^t \reward(\bs_t, \mathbf{a}_t)
\right]. \label{eq:rl_objective}
\end{equation}
When $H$ is infinite, it is typical to consider the expected reward under the $\gamma$-discounted stationary distribution of the learned policy. Formally, we can refer to the marginals of the trajectory distribution $p_\policy(\traj)$. We will use $\freq^\policy(\bs)$ to refer to the overall state visitation frequency, averaged over the time steps, and $\freq^\policy_t(\bs_t)$ to refer to the state visitation frequency at time step $t$. Alternatively, we can consider the undiscounted expected reward under the stationary distribution of the Markov chain $(\bs_t, \mathbf{a}_t)$ defined by $\policy(\mathbf{a}_t | \bs_t) \transitions(\bs_{t+1} | \bs_t, \mathbf{a}_t)$, under ergodicity assumptions. For discussions in this dissertation, we will consider the discounted marginal setting for simplicity.  

% In many cases, 

Next, we will briefly summarize some types reinforcement learning algorithms and present definitions. At a high level, all standard reinforcement learning algorithms follow the same basic learning loop: the agent \emph{interacts} with the MDP $\mdp$ by using some sort of \emph{behavior policy}, which may or may not match $\policy(\mathbf{a}|\bs)$, by observing the current state $\bs_t$, selecting an action $\mathbf{a}_t$, and then observing the resulting next state $\bs_{t+1}$ and reward value $\reward_t = \reward(\bs_t,\mathbf{a}_t)$. This may repeat for multiple steps, and the agent then uses the observed transitions $(\bs_t,\mathbf{a}_t,\bs_{t+1},\reward_t)$ to update its policy. This update might also utilize previously observed transitions. We will use $\data = \{ (\bs^i_t,\mathbf{a}^i_t,\bs^i_{t+1},\reward^i_t) \}$ to denote the set of transitions that are available for the agent to use for updating the policy (``learning''), which may consist of either all transitions seen so far, or some subset thereof.

\niparagraph{\large{Dynamic Programming with Function Approximators}} 

One way to optimize the reinforcmeent learning objective relies on the following observation: if we can accurately estimate a state or state-action \emph{value function}, then it is easy to then recover a near-optimal policy. A value function provides an estimate of the expected cumulative reward that will be obtained by following some policy $\policy(\mathbf{a}_t|\bs_t)$ when starting from a given state $\bs_t$, in the case of the state-value function $V^\policy(\bs_t)$, or when starting from a state-action tuple $(\bs_t,\mathbf{a}_t)$, in the case of the state-action value function $Q^\policy(\bs_t,\mathbf{a}_t)$. We can define these value functions as:
\begin{align*}
V^\policy(\bs_t) &= \E_{\traj \sim p_\policy(\traj | \bs_t)} \left[
\sum_{t' = t}^H \gamma^{t' - t} \reward(\bs_t, \mathbf{a}_t)
\right] \\
Q^\policy(\bs_t,\mathbf{a}_t) &= \E_{\traj \sim p_\policy(\traj | \bs_t, \mathbf{a}_t)} \left[
\sum_{t' = t}^H \gamma^{t' - t} \reward(\bs_t, \mathbf{a}_t)
\right].
\end{align*}
From this, we can derive recursive definitions for value functions:
\begin{align*}
V^\policy(\bs_t) &= \E_{\mathbf{a}_t \sim \policy(\mathbf{a}_t | \bs_t)}\left[
Q^\policy(\bs_t,\mathbf{a}_t)
\right] \\
Q^\policy(\bs_t, \mathbf{a}_t) &= \reward(\bs_t,\mathbf{a}_t) + \discount \E_{\bs_{t+1} \sim \transitions(\bs_{t+1}|\bs_t,\mathbf{a}_t)}\left[
V^\policy(\bs_{t+1})
\right].
\end{align*}
We can combine these two equations to express the $Q^\policy(\bs_t,\mathbf{a}_t)$ in terms of $Q^\policy(\bs_{t+1},\mathbf{a}_{t+1})$:
\begin{align}
Q^\policy(\bs_t, \mathbf{a}_t) &= \reward(\bs_t,\mathbf{a}_t) + \discount \E_{\bs_{t+1} \sim \transitions(\bs_{t+1}|\bs_t,\mathbf{a}_t), \mathbf{a}_{t+1} \sim \policy(\mathbf{a}_{t+1} | \bs_{t+1})}\left[
Q^\policy(\bs_{t+1},\mathbf{a}_{t+1})
\right]. \label{eq:qeq}
\end{align}
We can also express these in terms of the \emph{Bellman operator} for the policy $\policy$, which we denote $\bellman^\policy$. For example, Equation~(\ref{eq:qeq}) can be written as ${Q}^\policy = \bellman^\policy {Q}^\policy$, where ${Q}^\policy$ (with abuse of notation) now denotes the Q-function $Q^\policy$ represented as a vector of length $|\states|\times |\actions|$. Before moving on to deriving learning algorithms based on these definitions, we briefly discuss some properties of the Bellman operator. This Bellman operator has a unique fixed point that corresponds to the true Q-function for the policy $\policy(\mathbf{a}|\bs)$, which can be obtained by repeating the iteration ${Q}^\policy_{k+1} = \bellman^\policy {Q}^\policy_k$, and it can be shown that $\lim_{k \rightarrow \infty} {Q}^\policy_k = {Q}^\policy$, which obeys Equation~(\ref{eq:qeq})~\citep{suttonrlbook}. The proof for this follows from the observation that $\bellman^\policy$ is a contraction in the $\ell_\infty$ norm~\citep{lagoudakis2003least}.

Based on these definitions, we can derive two commonly used algorithms based on dynamic programming: Q-learning and actor-critic methods. To derive Q-learning, we express the policy implicitly in terms of the Q-function, as \mbox{$\policy(\mathbf{a}_t | \bs_t) = \delta(\mathbf{a}_t = \arg\max Q(\bs_t,\mathbf{a}_t))$}, and only learn the Q-function $Q(\bs_t,\mathbf{a}_t)$. By substituting this (implicit) policy into the above dynamic programming equation, we obtain the following condition on the optimal Q-function:
\begin{equation}
Q^\star(\bs_t, \mathbf{a}_t) = \reward(\bs_t,\mathbf{a}_t) + \discount \E_{\bs_{t+1} \sim \transitions(\bs_{t+1}|\bs_t,\mathbf{a}_t)}\left[
\max_{\mathbf{a}_{t+1}} Q^\star(\bs_{t+1},\mathbf{a}_{t+1})
\right]. \label{eq:q_learning_equation}
\end{equation}
We can again express this as ${Q} = \bellman^\star {Q}$ in vector notation, where $\bellman^\star$ now refers to the Bellman optimality operator. Note however that this operator is not linear, due to the maximization on the right-hand side in Equation~(\ref{eq:q_learning_equation}). To turn this equation into a learning algorithm, we can minimize the difference between the left-hand side and right-hand side of this equation with respect to the parameters of a parametric Q-function estimator with parameters $\phi$, $Q_\phi(\bs_t,\mathbf{a}_t)$. There are a number of variants of this Q-learning procedure, including variants that fully minimize the difference between the left-hand side and right-hand side of the above equation at each iteration, commonly referred to as fitted Q-iteration~\citep{ernst2005tree,Riedmiller2005}, and variants that take a single gradient step, such as the original Q-learning method~\citep{watkins1992q}. The commonly used variant in deep reinforcement learning is a kind of hybrid of these two methods, employing a replay buffer~\citep{lin1992self} and taking gradient steps on the Bellman error objective concurrently with data collection~\citep{mnih2013playing}. We write out a general recipe for Q-learning methods in Algorithm~\ref{alg:qlearning}.

\begin{algorithm}[ht]
\caption{Generic Q-learning (includes FQI and DQN as special cases) \label{alg:qlearning}}
\begin{algorithmic}[1]
\State initialize $\phi_0$
\State initialize $\policy_0(\mathbf{a}|\bs) = \epsilon \mathcal{U}(\mathbf{a}) + (1-\epsilon)\delta(\mathbf{a} = \arg\max_{\mathbf{a}} Q_{\phi_0}(\bs,\mathbf{a}))$ \Comment{Use $\epsilon$-greedy exploration}
\State initialize replay buffer $\data = \emptyset$ as a ring buffer of fixed size
\State initialize $\bs \sim \initstate(\bs)$
\For{iteration $k \in [0, \dots, K]$}
\For{step $s \in [0, \dots, S-1]$}
\State $\mathbf{a} \sim \policy_k(\mathbf{a}|\bs)$ \Comment{sample action from exploration policy}
\State $\bs' \sim p(\bs' | \bs, \mathbf{a})$ \Comment{sample next state from MDP}
\State $\data \leftarrow \data \cup \{(\bs,\mathbf{a},\bs',\reward(\bs,\mathbf{a}))\}$ \Comment{append to buffer, purging old data if buffer too big}
\EndFor
\State $\phi_{k,0} \leftarrow \phi_k$
\For{gradient step $g \in [0, \dots, G-1]$}
\State sample batch $batch \subset \data$ \Comment{$B = \{ (\bs_i, \mathbf{a}_i, \bs'_i, r_t ) \}$}
\State estimate error $\en(B,\phi_{k,g}) = \sum_i \left( Q_{\phi_{k,g}} - (r_i + \discount \max_{\mathbf{a}'} Q_{\phi_k}(\bs',\mathbf{a}')) \right)^2$
\State update parameters: $\phi_{k,g+1} \leftarrow \phi_{k,g} - \alpha \nabla_{\phi_{k,g}} \en(B,\phi_{k,g})$
\EndFor
\State $\phi_{k+1} \leftarrow \phi_{k,G}$ \Comment{update parameters}
\EndFor
\end{algorithmic}
\end{algorithm}

Classic Q-learning can be derived as the limiting case where the buffer size is 1, and we take $G=1$ gradient steps and collect $S=1$ transition samples per iteration, while classic fitted Q-iteration runs the inner gradient descent phase to convergence (i.e., $G=\infty$), and uses a buffer size equal to the number of sampling steps $S$. Note that many modern implementations also employ a \emph{target network}, where the target value $r_i + \discount \max_{\mathbf{a}'} Q_{\phi_k}(\bs',\mathbf{a}')$ actually uses $\phi_{L}$, where $L$ is a lagged iteration (e.g., the last $k$ that is a multiple of 1000). Note that these approximations violate the assumptions under which Q-learning algorithms can be proven to converge. However, recent work suggests that high-capacity function approximators, which correspond to a very large set $\qset$, generally do tend to make this method convergent in practice, yielding a Q-function that is close to ${Q}^\policy$~\citep{fu2019diagnosing,van2018deep}.

\begin{algorithm}[H]
    \caption{Generic off-policy actor-critic \label{alg:actorcritic}}
    \begin{algorithmic}[1]
    \State initialize $\phi_0$
    \State initialize $\theta_0$
    \State initialize replay buffer $\data = \emptyset$ as a ring buffer of fixed size
    \State initialize $\bs \sim \initstate(\bs)$
    \For{iteration $k \in [0, \dots, K]$}
    \For{step $s \in [0, \dots, S-1]$}
    \State $\mathbf{a} \sim \policy_{\theta_k}(\mathbf{a}|\bs)$ \Comment{sample action from current policy}
    \State $\bs' \sim p(\bs' | \bs, \mathbf{a})$ \Comment{sample next state from MDP}
    \State $\data \leftarrow \data \cup \{(\bs,\mathbf{a},\bs',\reward(\bs,\mathbf{a}))\}$ \Comment{append to buffer, purging old data if buffer too big}
    \EndFor
    \State $\phi_{k,0} \leftarrow \phi_k$
    \For{gradient step $g \in [0, \dots, G_Q-1]$}
    \State sample batch $batch \subset \data$ \Comment{$B = \{ (\bs_i, \mathbf{a}_i, \bs'_i, r_t ) \}$}
    \State estimate error $\en(B,\phi_{k,g}) = \sum_i \left( Q_{\phi_{k,g}} - (r_i + \discount \E_{\mathbf{a}'\sim \policy_k(\mathbf{a}'|\bs')} Q_{\phi_k}(\bs',\mathbf{a}')) \right)^2$
    \State update parameters: $\phi_{k,g+1} \leftarrow \phi_{k,g} - \alpha_Q \nabla_{\phi_{k,g}} \en(B,\phi_{k,g})$
    \EndFor
    \State $\phi_{k+1} \leftarrow \phi_{k,G_Q}$ \Comment{update Q-function parameters}
    \State $\theta_{k,0} \leftarrow \theta_k$
    \For{gradient step $g \in [0,\dots G_\policy-1]$}
    \State sample batch of states $\{\bs_i\}$ from $\data$
    \State for each $\bs_i$, sample $\mathbf{a}_i \sim \policy_{\theta_{k,g}}(\mathbf{a}|\bs_i)$ \Comment{do not use actions in the buffer!}
    \State for each $(\bs_i,\mathbf{a}_i)$, compute $\hat{A}(\bs_i, \mathbf{a}_i) = Q_{\phi_{k+1}}(\bs_i,\mathbf{a}_i) - \E_{\mathbf{a}\sim\policy_{k,g}(\mathbf{a}|\bs_i)}[Q_{\phi_{k+1}}(\bs_i,\mathbf{a})]$
    \State $\nabla_{\theta_{k,g}} J(\policy_{\theta_{k,g}}) \approx \frac{1}{N} \nabla_{\theta_{k,g}} \log \policy_{\theta_{k,g}}(\bs_i, \mathbf{a}_i) \hat{A}(\bs_i, \mathbf{a}_i)$
    \State $\theta_{k,g+1} \leftarrow \theta_{k,g} + \alpha_\policy \nabla_{\theta_{k,g}} J(\policy_{\theta_{k,g}})$
    \EndFor
    \State $\theta{k+1} \leftarrow \theta_{k,G_\policy}$ \Comment{update policy parameters}
    \EndFor
    \end{algorithmic}
\end{algorithm}

\niparagraph{\large{Actor-Critic Algorithms}} 

Actor-critic algorithms employ \emph{both} a parameterized policy and a parameterized value function, and use the value function to provide training signal for the policy. Typically, the learned value function is used to provide a better estimate of policy performance (or advantage $\hat{A}(\bs,\mathbf{a})$) in a policy gradient objective. There are a number of different variants of actor-critic methods, including on-policy variants that directly estimate $V^\policy(\bs)$~\citep{konda2000actor}, and off-policy variants that estimate $Q^\policy(\bs,\mathbf{a})$ via a parameterized state-action value function $Q^\policy_\phi(\bs,\mathbf{a})$~\citep{haarnoja2018sac,haarnoja2017reinforcement,heess2015learning}. 

We will focus on the latter class of algorithms, since they can be easily extended to the offline setting. The basic design of such an algorithm is a straightforward combination of the ideas in dynamic programming and policy gradients. Unlike Q-learning, which directly attempts to learn the optimal Q-function, actor-critic methods aim to learn the Q-function corresponding to the current parameterized policy $\policy_\theta(\mathbf{a} | \bs)$, which must obey the equation
\[
Q^\policy(\bs_t, \mathbf{a}_t) = \reward(\bs_t,\mathbf{a}_t) + \discount \E_{\bs_{t+1} \sim \transitions(\bs_{t+1}|\bs_t,\mathbf{a}_t), \mathbf{a}_{t+1} \sim \policy_\theta(\mathbf{a}_{t+1} | \bs_{t+1})}\left[
Q^\policy(\bs_{t+1},\mathbf{a}_{t+1})
\right].
\]
As before, this equation can be expressed in vector form in terms of the Bellman operator for the policy, ${Q}^\policy = \bellman^\policy {Q}^\policy$, where ${Q}^\policy$ denotes the Q-function $Q^\policy$ represented as a vector of length $|\states|\times |\actions|$. We can now instantiate a complete algorithm based on this idea, shown in Algorithm~\ref{alg:actorcritic}.

For more details, we refer the reader to standard textbooks and prior works~\citep{sb-irl-98,konda2000actor}.
Actor-critic algorithms are closely related with another class of methods that frequently arises in dynamic programming, called policy iteration (PI)~\citep{lagoudakis2003least}. Policy iteration consists of two phases: policy evaluation and policy improvement. The policy evaluation phase computes the Q-function for the current policy $\policy$, $Q^\policy$, by solving for the fixed point such that $Q^\policy = \bellman^\policy Q^\policy$. This can be done via gradient updates, analogously to line 15 in Algorithm~\ref{alg:actorcritic}. The next policy iterate is then computed in the policy improvement phase, by choosing the action that greedily maximizes the Q-value at each state, such that $\policy_{k+1}(\mathbf{a}|\bs) = \delta(\mathbf{a} = \arg\max_{\mathbf{a}} Q^{\policy_k}(\bs,\mathbf{a}))$, or by using a gradient based update procedure as is employed in Algorithm~\ref{alg:actorcritic} on line 24. In the absence of function approximation (e.g., with tabular representations) policy iteration produces a monotonically improving sequence of policies, and converges to the optimal policy. Policy iteration can be obtained as a special case of the generic actor-critic algorithm in Algorithm~\ref{alg:actorcritic} when we set $G_Q = \infty$ and $G_\pi = \infty$, when the buffer $\mathcal{D}$ consists of each and every transition of the MDP.

%Besides the model-free reinforcement learning algorithms discussed in this section, a wide range of model-based reinforcement learning algorithms have also been proposed in the literature. Such methods utilize explicit estimates of the transition or dynamics function $\transitions(\bs_{t+1}|\bs_t,\mathbf{a}_t)$, and then use this estimated dynamics function in a variety of ways~\citep{sutton1991dyna,deisenroth2011pilco,nagabandi2018neural,pets,pipps,simpl,mbpo}. A discussion of model-based reinforcement learning methods is outside the scope of this tutorial, though we will briefly discuss the potential of model-based offline reinforcement learning methods in Section~\ref{sec:discussion}.

\niparagraph{\large{Model-Based Reinforcement Learning}} 

Model-based reinforcement learning is a general term that refers to a broad class of methods that utilize explicit estimates of the transition or dynamics function $\transitions(\bs_{t+1}|\bs_t,\mathbf{a}_t)$, parameterized by a parameter vector $\psi$, which we will denote $\transitions_\psi(\bs_{t+1}|\bs_t,\mathbf{a}_t)$. There is no single recipe for a model-based reinforcement learning method. Some commonly used model-based reinforcement learning algorithms learn only the dynamics model $\transitions_\psi(\bs_{t+1}|\bs_t,\mathbf{a}_t)$, and then utilize it for planning at test time, often by means of model-predictive control (MPC)~\citep{tassa2012synthesis} with various trajectory optimization methods~\citep{nagabandi2018neural,pets}. Other model-based reinforcement learning methods utilize a learned policy $\policy_\theta(\mathbf{a}_t|\bs_t)$ in addition to the dynamics model, and employ backpropagation through time to optimize the policy with respect to the expected reward objective~\citep{deisenroth2011pilco}. Yet another set of algorithms employ the model to generate ``synthetic'' samples to augment the sample set available to model-free reinforcement learning methods. The classic Dyna algorithm uses this recipe in combination with Q-learning and one-step predictions via the model from previously seen states~\citep{sutton1991dyna}, while a variety of recently proposed algorithms employ synthetic model-based rollouts with policy gradients~\citep{pipps,simpl} and actor-critic algorithms~\citep{mbpo}. 

% Since there are so many variants of model-based reinforcement learning algorithms, we will not go into detail on each of them in this section, but we will discuss some considerations for offline model-based reinforcement learning in Section~\ref{sec:model_based}.

\textbf{Comparison of different types of RL methods.} One might wonder how these different classes of reinforcement learning methods compare with each other. In practice, value-based RL methods that use pproximate dynamic programming (Q-learning and actor-critic) train Q-functions to match non-stationary target values, resulting in an optimization problem different from standard supervised learning. This results in error propagation, a phenomenon that we discuss theoretically and empirically in Chapter~\ref{chapter:bear} when learning value functions. On the other hand, training a model of the transition dynamics is a supervised learning problem, which can be tackled using well-studied tools in empirical risk minimization theoretically and benefits directly from advances in supervised deep learning in practice. That said, trajectory rollouts in model-based RL algorithms diverge from rollouts in the ground-truth model over longer horizons, once errors in fitting the model compound together over steps of a trajectory. This is further exacerbated when the policy aims to optimize the cumulative reward under the learned model, as we elaborate theoretically and empirically in Chapter~\ref{chapter:cql}. From a theoretical standpoint, value-based RL methods based on approximate dynamic programming require stronger assumptions of completeness in order to learn a policy effectively, although model-based RL methods do not require this sort of an assumption (see \citet{sun2019model} for a detailed discussion of when model-based RL methods can perform better).         

\section{Problem Statement: Offline Reinforcement Learning}

The offline reinforcement learning problem can be defined as a \emph{dataset-driven} formulation of the RL problem. The end goal is still to optimize the objective in Equation~(\ref{eq:rl_objective}). However, the agent no longer has the ability to interact with the environment and collect additional transitions using the behavior policy. Instead, the learning algorithm is provided with a \emph{static} dataset of transitions, $\data = \{ (\bs^i_t,\mathbf{a}^i_t,\bs^i_{t+1},\reward^i_t) \}$, and must learn the best policy it can using this dataset. This formulation more closely resembles the standard supervised learning problem statement, and we can regard $\data$ as the \emph{training set} for the policy. In essence, offline reinforcement learning requires the learning algorithm to derive a sufficient understanding of the dynamical system underlying the MDP $\mdp$ entirely from a fixed dataset, and then construct a policy $\policy(\mathbf{a}|\bs)$ that attains the largest possible cumulative reward \emph{when it is actually used to interact with the MDP}. We will use $\behavior$ to denote the distribution over states and actions in $\data$, such that we assume that the state-action tuples $(\bs,\mathbf{a}) \in \data$ are sampled according to $\bs \sim \freq^{\behavior}(\bs)$, and the actions are sampled according to the behavior policy, such that $\mathbf{a} \sim \behavior(\mathbf{a}|\bs)$.

%%AK: Change this para
% This problem statement has been presented under a number of different names. The term ``off-policy reinforcement learning'' is typically used as an umbrella term to denote all reinforcement learning algorithms that can employ datasets of transitions $\data$ where the corresponding actions in each transition were collected with any policy \emph{other} than the current policy $\policy(\mathbf{a}|\bs)$. Q-learning algorithms, actor-critic algorithms that utilize Q-functions, and many model-based reinforcement learning algorithm are off-policy algorithms. However, off-policy algorithms still often employ additional interaction (i.e., online data collection) during the learning process. Therefore, the term ``fully off-policy'' is sometimes used to indicate that no additional online data collection is performed. Another commonly used term is ``batch reinforcement learning''~\citep{ernst2005tree,riedmiller2005neural,lange2012batch}. While this term has been used widely in the literature, it can also cause some amount of confusion, since the use of a ``batch'' in an iterative learning algorithm can also refer to a method that consumes a batch of data, updates a model, and then obtains a different batch, as opposed to a traditional online learning algorithm, which consumes one sample at a time. In fact, \citet{lange2012batch} further introduces qualifiers  ``pure'' and ``growing'' batch reinforcement learning to clarify this. %The ``batch'' in ``batch reinforcement learning'' remains fixed throughout training. 
% To avoid this confusion, we will instead use the term ``offline reinforcement learning'' in this tutorial.

The offline reinforcement learning problem can be approached using algorithms from many categories of examples discussed above, and in principle any off-policy RL algorithm \emph{could} be used as an offline RL algorithm. For example, a simple offline RL method can be obtained simply by using Q-learning without additional online exploration, using $\data$ to pre-populate the data buffer. This corresponds to changing the initialization of $\data$ in Algorithm~\ref{alg:qlearning}, and setting $S=0$. However, as we will discuss in subsequent chapters, not all such methods are effective in the offline setting.  

We will also consider an extension of the offline reinforcement learning problem, where the goal is to first learn an offline policy initialization, followed by fine-tuning the learned policy with limited amounts of online, actively-collected data. We will define a bit of terminology and notation for this problem in Chapter~\ref{chapter:calql}, which studies this problem.  

\end{document}