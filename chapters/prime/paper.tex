\documentclass{article}
\usepackage{iclr2022_conference,times}
\usepackage{amsmath}
\usepackage{amssymb}
\usepackage{mathtools}
\usepackage{natbib}
\usepackage{enumerate}
\usepackage{tikz}
\usepackage{graphicx}

\DeclarePairedDelimiter\abs{\lvert}{\rvert}%
\DeclarePairedDelimiter\norm{\lVert}{\rVert}%
\DeclarePairedDelimiter\ceil{\lceil}{\rceil}
\DeclarePairedDelimiter\floor{\lfloor}{\rfloor}

\newcommand{\TODO}[1]{\textcolor{red}{TODO: #1}}
\newcommand{\textdiff}[1]{\textcolor{red}{#1}}
\newcommand{\citemissing}{\textcolor{red}{(cite?)}}

\newcommand{\normt}[1]{\left\lVert#1\right\rVert_2}
\newcommand{\normtmu}[1]{\left\lVert#1\right\rVert_{2, \mu}}
\newcommand{\normm}[1]{\left\lVert#1\right\rVert}
\newcommand{\norminf}[1]{\left\lVert#1\right\rVert_\infty}
\newcommand{\normtt}[1]{\left\lVert#1\right\rVert^2_2}

\newcommand{\half}{\frac{1}{2}}
\newcommand{\fourth}{\frac{1}{4}}
\newcommand{\vect}[1]{\overrightafrrow{\textbf{#1}}}
\newcommand{\phat}{\hat{p}}
\newcommand{\KL}[2]{D_{KL}(#1||#2)}
\newcommand{\TV}[2]{D_{TV}(#1||#2)}
\newcommand{\ind}[1]{1[#1]}
\newcommand{\pardiv}[1]{\frac{\partial}{\partial #1}}
\newcommand{\parHess}[1]{\frac{\partial^2}{\partial #1 ^2}}
\newcommand{\Xhat}{{\hat{X}}}
\newcommand{\xhat}{{\hat{x}}}
\newcommand{\defeq}{\mathrel{\stackrel{\makebox[0pt]{\mbox{\normalfont\tiny def}}}{=}}}

\newcommand{\argmax}[1]{\underset{#1}{\textrm{argmax}}\ }
\newcommand{\argmin}[1]{\underset{#1}{\textrm{argmin}}\ }
\newcommand{\grad}[1]{\nabla{#1}}
\newcommand{\innerp}[2]{\langle{#1,#2}\rangle}
\newcommand{\Hess}[1]{\nabla^2{#1}}
\newcommand{\EXP}[1]{\text{exp}\{#1\}}

% debug q
\newcommand{\Proj}{\Pi}
\newcommand{\Projmu}{\Pi_\mu}
\newcommand{\trans}{T}
\newcommand{\backup}{\mathcal{T}}
\newcommand{\Qclass}{\mathcal{Q}}
\newcommand{\ReplayBuffer}{\mathcal{B}}
\newcommand{\ltwonorm}{L_2}
\newcommand{\lpnorm}{L_p}
\newcommand{\linfnorm}{L_\infty}

\newcommand{\UniformVec}{U[-1,1]^{32}}
\usepackage{packages}

\usepackage{fleqn, tabularx}
\usepackage{colortbl}
\usepackage{transparent}

\definecolor{Gray}{gray}{0.9}

\makeatletter
\newcommand{\mybox}{%
    \collectbox{%
        \setlength{\fboxsep}{1pt}%
        \fbox{\BOXCONTENT}%
    }%
}
\makeatother
\newcommand{\CC}{\cellcolor{Gray}}

\title{\centering{Data-Driven Offline Optimization for\\ Architecting Hardware Accelerators}}

\iclrfinalcopy

\author{%
  \quad \quad \quad \quad \quad \quad \quad \quad \quad {Aviral Kumar$^{\dagger, *}$~~~ Amir Yazdanbakhsh$^{\dagger}$} \vspace{0.05cm}\\
  \quad \quad \quad \quad \quad \quad \quad \quad \textbf{Milad Hashemi~~~ Kevin Swersky~~~ Sergey Levine$^{*}$} \vspace{0.2cm}\\
  \quad \quad \quad \quad \quad \quad Google Research ~~$^*$ UC Berkeley~~~~ ($^\dagger$ Equal Contribution) \vspace{0.05cm}\\
  \quad \quad \quad \quad \quad \quad~~~  \texttt{aviralk@berkeley.edu, ayazdan@google.com}
  % examples of more authors
  % \And
  % Coauthor \\
  % Affiliation \\
  % Address \\
  % \texttt{email} \\
  % \AND
  % Coauthor \\
  % Affiliation \\
  % Address \\
  % \texttt{email} \\
  % \And
  % Coauthor \\
  % Affiliation \\
  % Address \\
  % \texttt{email} \\
  % \And
  % Coauthor \\
  % Affiliation \\
  % Address \\
  % \texttt{email} \\
}

\begin{document}

\part*{\Large{Significant Publication 3: \\ Data-Driven Offline Optimization For Architecting Hardware Accelerators}}

\section*{Author List} Aviral Kumar, Amir Yazdanbakhsh, Milad Hashemi, Kevin Swersky, Sergey Levine

\section*{Significance of the Paper} 

\textbf{Summary.} This paper utilizes ideas from my offline RL work for tackling the problem of hardware accelerator design. The goal in this work is to design an accelerator that optimizes some performance metric (e.g., latency) to optimize the learned model. I show how a transition from simulators to datasets can enable us to design accelerators for never-before-seen workloads, in a zero-shot manner, using data of successful and failed accelerator designs for other workloads. This is quite promising since building cycle-accurate simulators can be time-consuming and my work provides a fully dataset-driven approach to tackle this problem.  


\textbf{Significance.} The main significance of the results in the paper is that it shows how even in the challenging problem of designing accelerators, where we must strive to find a small set of good accelerators from a sea of infeasible designs, dataset-driven design approaches can be performant even when provided with bad previously generated designs, whereas simulator-driven approaches may not (while also requiring more manual effort). Our analysis of the designed accelerators in Appendix~\ref{app:analysis} show how our method can automatically tune compute against memory, which of interest in future architecture design. 

More broadly, this work opens up an exciting possibilities for machine learning in computer architecture. It shows the promise of utilizing databases of both successful and failed accelerators (including invalid accelerator designs) for certain workloads, to automatically discover accelerators for \emph{new} workloads. This kind of transfer across workloads is quite exciting as more and more machine learning models start getting deployed into IoT devices.

This work was featured in the \href{https://ai.googleblog.com/2022/03/offline-optimization-for-architecting.html}{Google AI blog} \& media outlets \href{https://medium.com/syncedreview/google-uc-berkeleys-data-driven-offline-optimization-approach-significantly-boosts-hardware-70746a7ff3cc}{[(a)}, \href{https://analyticsindiamag.com/data-driven-vs-simulation-driven-approach-to-build-hardware-accelerators-according-to-google/}{(b)}, \href{https://www.marktechpost.com/2022/03/23/google-ai-and-uc-berkely-researchers-introduce-a-deep-learning-approach-called-prime-that-generates-ai-chip-architectures-by-drawing-from-existing-blueprints-and-performance-figures/}{(c)}, \href{https://www.theregister.com/2022/03/18/google_deep_learning_chip_design/}{(d)]}.
 

\newpage

\maketitle

\vspace{-0.3cm}
\begin{abstract}
\label{sec:abstract}
%
To attain higher efficiency, the industry has gradually reformed towards application-specific hardware accelerators.
%
While such a paradigm shift is already starting to show promising results, designers need to spend considerable manual effort and perform large number of time-consuming simulations to find accelerators that can accelerate multiple target applications while obeying design constraints.
%
Moreover, such a ``simulation-driven'' approach must be re-run from scratch every time the set of target applications or design constraints change. 
%
An alternative paradigm is to use a ``data-driven'', offline approach that utilizes logged simulation data, to architect hardware accelerators, without needing any form of simulations.
%
Such an approach not only alleviates the need to run time-consuming simulation, but also enables data reuse and applies even when set of target applications changes.
%
In this paper, we develop such a data-driven offline optimization method for designing hardware accelerators, dubbed \methodname, that enjoys all of these properties.
%
Our approach learns a conservative, robust estimate of the desired cost function, utilizes infeasible points and optimizes the design against this estimate without any additional simulator queries during optimization.
%
\methodname\ architects accelerators---tailored towards both single- and multi-applications---improving performance upon stat-of-the-art simulation-driven methods by about 1.54$\times$ and 1.20$\times$, while considerably reducing the required total simulation time by 93\% and 99\%, respectively.
%
In addition, \methodname\ also architects effective accelerators for unseen applications in a zero-shot setting, outperforming simulation-based methods by 1.26$\times$\footnote{A more accessible blog post summarizing this paper is available on the Google AI blog at: \url{https://ai.googleblog.com/2022/03/offline-optimization-for-architecting.html}}.
%
\end{abstract}
\section{Introduction}
\label{sec:intro}
%% motivation, brief description of the problem statement
The death of Moore's Law~\citep{esmaeilzadeh2011dark} and its spiraling effect on the semiconductor industry have driven the growth of specialized hardware accelerators. These specialized accelerators are tailored to specific applications~\citep{yazdanbakhsh2021apollo,reagen2017case,prac_dse:mascots:2019,shi2020learned}. To design specialized accelerators, designers first spend considerable amounts of time developing simulators that closely model the real accelerator performance, and then optimize the accelerator using the simulator. While such simulators can automate accelerator design, this requires a large number of simulator queries for each new design, both in terms of simulation time and compute requirements, and this cost increases with the size of the design space~\citep{yazdanbakhsh2021evaluation,shi2020learned,hegdemind}.
%
Moreover, most of the accelerators in the design space are typically infeasible~\citep{hegdemind,yazdanbakhsh2021apollo} because of build errors in silicon or compilation/mapping failures. 
%
When the target applications change or a new application is added, the complete simulation-driven procedure is generally repeated.
%
To make such approaches efficient and practically viable, designers typically ``bake-in'' constraints or otherwise narrow the search space, but such constraints can leave out high-performing solutions~\citep{dmazerunner,timeloop,marvel}.

%% Transition to the new paradigm, define data-driven stuff, etc
An alternate approach, proposed in this work, is to devise a \textit{data-driven} optimization method that only utilizes a database of previously tested accelerator designs, annotated with measured performance metrics, to produce new optimized designs \emph{without} additional active queries to an explicit silicon or a cycle-accurate simulator.
%
Such a data-driven approach provides three key benefits: \textbf{(1)} it significantly shortens the recurring cost of running large-scale simulation sweeps, \textbf{(2)} it alleviates the need to explicitly bake in domain knowledge or search space pruning, and {\textbf{(3)} it enables data re-use by empowering the designer to optimize accelerators for new unseen applications, by the virtue of effective generalization.}
%
While data-driven approaches have shown promising results in biology~\citep{fu2021offline,brookes19a,trabucco2021conservative},
using offline optimization methods to design accelerators 
has been challenging primarily due to the abundance of infeasible design points~\citep{yazdanbakhsh2021apollo,hegdemind}.

%% The key idea of our method
The key contribution of this paper is a data-driven approach, \primemethodname\,, to automatically architect high-performing application-specific accelerators by using only previously collected offline data.
%
\primemethodname\ learns a robust surrogate model of the task objective function from an existing offline dataset, and finds high-performing application-specific accelerators by optimizing the architectural parameters against this learned surrogate function, as shown in Figure~\ref{fig:overview}. While na\"ively learned
%
\begin{figure}[t!]
    \centering
    \vspace{-0.1cm}
    \includegraphics[width=0.8\linewidth]{chapters/prime/figs/overview/prime-overview.pdf}
    \vspace{-0.2cm}
    \caption{\footnotesize{\textbf{Overview of \primemethodname.} We use a one-time collected dataset of prior accelerator designs, including TPU-style~\citep{yazdanbakhsh2021evaluation}, NVDLA-style~\citep{nvdla}, and ShiDianNao-style~\citep{shidiannao} accelerators to train a conservative surrogate model, which is used to design accelerators to meet desired goals and constraints.}}
    \vspace{-0.4cm}
    \label{fig:overview}
\end{figure}
surrogate functions usually produces poor-performing, out-of-distribution designs that appear quite optimistic under the learned surrogate~\citep{kumar2019model,brookes19a,trabucco2021conservative}.
%
The robust surrogate in \primemethodname\ is explicitly trained to prevent overestimation on ``adversarial'' designs that would be found during optimization.
%
Furthermore, in contrast to prior works that discard infeasible points~\citep{hegdemind,trabucco2021conservative}, our proposed method instead incorporates infeasible points when learning the conservative surrogate by treating them as additional negative samples.
%
During evaluation, \primemethodname\ optimizes the learned conservative surrogate.

Our results show that \primemethodname\ architects hardware accelerators that improve over the best design in the training dataset, on average, by 2.46$\times$ (up to 6.7$\times$) when specializing for a single application. 
%
In this case, \primemethodname\ also improves over the best conventional simulator-driven optimization methods by 1.54$\times$ (up to 6.6$\times$).
%
These performance improvements are obtained while reducing the total simulation time to merely 7\% and 1\% of that of the simulator-driven methods for single-task and multi-task optimization, respectively.
%
More importantly, a contextual version of \primemethodname\ can design accelerators that are \emph{jointly optimal} for a set of \textit{nine} applications without requiring any additional domain information.
%
In this challenging setting, \primemethodname\ improves over simulator-driven methods, which tend to scale poorly as more applications are added, by 1.38$\times$.
%
Finally, we show that the surrogates trained with \primemethodname\ on a set of training applications can be readily used to obtain accelerators for \textit{unseen} target applications, without any retraining on the new application.
%
Even in this \emph{zero-shot} optimization scenario, \primemethodname\ outperforms simulator-based methods that require re-training and active simulation queries by up to 1.67$\times$.
%
In summary, \primemethodname\ allows us to effectively address the shortcomings of simulation-driven approaches, it: (1) significantly reduces the simulation time, (2) enables data reuse and enjoys generalization properties, and (3) does not require domain-specific engineering or search space pruning.
%

\vspace{-0.1cm}
\section{Background on Hardware Accelerators}
\vspace{-0.1cm}
\label{sec:background}
The goal of specialized hardware accelerators---Google TPUs~\citep{jouppi2017datacenter,edgetpu:arxiv:2020}, Nvidia GPUs~\citep{nvidia}, GraphCore~\citep{graphcore}---is to improve the performance of specific applications, such as machine learning models. To design such accelerators, architects typically create a parameterized design and sweep over parameters using simulation.

\niparagraph{Target hardware accelerators.}
%
Our primary evaluation uses an industry-grade and highly parameterized template-based accelerator following prior work~\citep{yazdanbakhsh2021evaluation}.
%
This template enables architects to determine the organization of various components, such as compute units, memory cells, memory, etc., by searching for these configurations in a discrete design space. Some ML applications may have large memory requirements (e.g., large language models~\citep{brown2020language}) demanding sufficient on-chip memory resources, while others may benefit from more compute blocks. The hardware design workflow directly selects the values of these parameters.
%
In addition to this accelerator and to further show the generality of our method to other accelerator design problems, we evaluate two distinct dataflow accelerators with different search spaces, namely NVDLA-style~\citep{nvdla} and ShiDianNao-style~\citep{shidiannao} from~\citet{kao2020confuciux} (See Section~\ref{sec:eval} and Appendix~\ref{sec:dla_shi_fast} for a detailed discussion; See Table~\ref{table:dla_shi} for results).

\begin{wrapfigure}{r}{0.47\textwidth}
    \centering
    \vspace{-0.1in}
    \includegraphics[width=0.98\linewidth]{chapters/prime/figs/accelerator/template-accel.png}
    \vspace{-0.1cm}
    \caption{An industry-level machine learning accelerator~\cite{yazdanbakhsh2021evaluation}.}
    \label{fig:template_accel}
    \vspace{-0.5cm}
    \label{fig:accels}
\end{wrapfigure} 

\niparagraph{How does an accelerator work?}
%
We briefly explain the computation flow on our template-based accelerators (Figure~\ref{fig:template_accel}) and refer the readers to Appendix~\ref{sec:dla_shi_fast} for details on other accelerators. This template-based accelerator is a 2D array of processing elements (PEs). Each PE is capable of performing matrix multiplications in a single instruction multiple data (SIMD) paradigm~\citep{simd}. A controller orchestrates the data transfer (both activations and model parameters) between off-chip DRAM memory and the on-chip buffers and also reads in and manages the instructions (e.g. convolution, pooling, etc.) for execution. The computation stages on such accelerators start by sending a set of activations to the compute lanes, executing them in SIMD manner, and either storing the partial computation results or offloading them back into off-chip memory. Compared to prior works~\citep{hegdemind,shidiannao,kao2020confuciux}, this parameterization is unique---it includes multiple compute lanes per each PE and enables SIMD execution model within each compute lane---and yields a distinct accelerator search space accompanied by an end-to-end simulation framework. 
% More details in Appendix~\ref{sec:dla_shi_fast}.
\vspace{-0.2cm}
\section{Problem Statement, Training Data and Evaluation Protocol}
\label{sec:accel}
\vspace{-0.2cm}
%
Our template-based parameterization maps the accelerator, denoted as $\rvx$, to a discrete design space, $\rvx = [\rvx_1, \rvx_2, \cdots, \rvx_K]$, and each $\rvx_i$ is a discrete-valued variable representing one component of the microarchitectural template, as shown in Table~\ref{tab:arch_params} (See Appendix~\ref{sec:dla_shi_fast} for the description of other accelerator search spaces studied in our work). In our context, the accelerator $\rvx$ plays the same role as an action $\mathbf{a}$.

%
An accelerator design maybe be infeasible due to various reasons, such as a compilation failure or the limitations of physical implementation, and we denote the set of all such feasibility criterion as $\mathrm{Feasible}(\rvx)$. The feasibility criterion depends on both the target software and the underlying hardware, and it is not easy to identify if a given $\rvx$ is infeasible without running explicit simulation. We will require our design procedure to not only learn the value of the objective function but also to learn to navigate through infeasible solutions to performant feasible solutions $\rvx^*$ satisfying $\mathrm{Feasible}(\rvx^*) = 1$. 

Our training dataset $\mathcal{D}$ consists of a modest set of accelerators $\rvx_i$ that are randomly sampled from the design space and evaluated by the hardware simulator. We partition the dataset $\mathcal{D}$ into two subsets, $\mathcal{D}_\text{feasible}$ and $\mathcal{D}_\text{infeasible}$. Let $f(\rvx)$ denote the desired objective (e.g., latency, power, etc.) we intend to optimize over the space of accelerators $\rvx$. We do not possess functional access to $f(\rvx)$, and the optimizer can only access $f(\rvx)$ values for accelerators $\rvx$ in the feasible partition of the data, $\mathcal{D}_\text{feasible}$.
%
For all infeasible accelerators, the simulator does not provide any value of $f(\rvx)$.
In addition to satisfying feasibility, the optimizer must handle explicit constraints on parameters such as area and power~\citep{flynn2011computer}. In our applications, we impose an explicit area constraint, $\mathrm{Area}(\rvx) \leq \review{\alpha_0}$, though additional explicit constraints are also possible. 
%
To account for different constraints, we formulate this task as a constrained optimization problem.
%
Formally:  
\begin{equation}
\label{eqn:opt_prob}
\vspace{-0.2cm}
\begin{aligned}
\min_{\rvx}~ & f(\rvx)
~~~\textrm{s.t.}~~~\mathrm{Area}(\rvx) \leq \review{\alpha_0}, ~~~ \mathrm{Feasible}(\rvx) = 1 \\
\quad &~\text{on}~~ \mathcal{D} = \mathcal{D}_\text{feasible} \cup \mathcal{D}_\text{infeasible} = \{(\rvx_1, \rvy_1), \cdots, (\rvx_N, \rvy_N)\} \cup \{\rvx'_1, \cdots, \rvx'_{N'}\}
\end{aligned}
\end{equation}

While Equation~\ref{eqn:opt_prob} may appear similar to a typical black-box optimization problem, solving it over the space of accelerator designs is challenging due to the large number of infeasible points, the need to handle explicit design constraints, and the difficulty in navigating the non-smooth landscape (See Figure~\ref{fig:ds_dist_all} and Figure~\ref{fig:appx_ds} in the Appendix) of the objective function.
%


\begin{table*}[t!]
\small
\centering
% \vspace*{0.1cm}
\caption{The accelerator design space parameters for the primary accelerator search space targeted in this work. The maximum possible number of accelerator designs (including feasible and infeasible designs) is 452,760,000. \methodname\ only uses a small randomly sampled subset of the search space.}
\label{tab:arch_params}
\vspace{-0.3cm}
\resizebox{0.8\textwidth}{!}{\begin{tabular}{l|l||l|l}
\toprule
\textbf{Accelerator Parameter} &\textbf{\# Discrete Values} & \textbf{Accelerator Parameter} & \textbf{\# Discrete Values}\\\midrule
\# of PEs-X & 10 & \# of PEs-Y & 10\\\hline
PE Memory & 7 & \# of Cores & 7\\\hline
Core Memory & 11 & \# of Compute Lanes & 10\\\hline
Instruction Memory & 4 & Parameter Memory & 5\\\hline
Activation Memory & 7 & DRAM Bandwidth & 6\\
\bottomrule
\end{tabular}}
\vspace{-0.2cm}
\end{table*}

\niparagraph{What makes optimization over accelerators challenging?} Compared to other domains where model-based optimization methods have been applied~\citep{brookes19a,trabucco2021conservative}, optimizing accelerators introduces a number of practical challenges.
%
First, accelerator design spaces typically feature a narrow manifold of feasible accelerators within a sea of infeasible points~\citep{prac_dse:mascots:2019,shi2020learned,gelbart2014bayesian}, as visualized in Figure~\ref{fig:ds_dist_all} and Appendix (Figure~\ref{fig:tsne_infeasible}).
%
While some of these infeasible points can be identified via simple rules (e.g. estimating chip area usage), most infeasible points correspond to failures during compilation or hardware simulation. These infeasible points are generally not straightforward to formulate into the optimization problem and requires simulation~\citep{shi2020learned,timeloop,yazdanbakhsh2021apollo}.

Second, the optimization objective can exhibit high sensitivity to small variations in some architecture parameters (Figure~\ref{fig:appx_ds_memory}) in some regions of the design space, but remain relatively insensitive in other parts, resulting in a complex optimization landscape. This suggests that optimization algorithms based on local parameter updates (e.g., gradient ascent,  evolutionary schemes, etc.) may have a challenging task traversing the nearly flat landscape of the objective, which can lead to poor performance.
\begin{figure}[t]
    \begin{center}
    \begin{minipage}{\linewidth}
    \centering
    \includegraphics[width=0.35\linewidth]{chapters/prime/figs/dataset/dist.pdf}
    \vspace{-0.1cm}
    \label{fig:ds_dist}
    \centering
    \includegraphics[width=0.35\linewidth]{chapters/prime/figs/dataset/dist_large.pdf}
    \label{fig:ds_dist_large}
    \end{minipage}
    \end{center}
    \vspace{-0.2cm}
    \caption{\footnotesize{\textbf{Left:} histogram of infeasible (right orange bar with large score values) and feasible (left cluster of bars) data points for MobileNetEdgeTPU; \textbf{Right:} zoomed-in histogram (different number of bins) focused on feasible points highlighting the variable latencies.}}
    \vspace{-0.2cm}
    \label{fig:ds_dist_all}
\end{figure}

\niparagraph{Training dataset.}
%
We used an offline dataset $\mathcal{D}$ of (accelerator parameters, latency) via random sampling from the space of 452M possible accelerator configurations.
%
Our method is only provided with a relatively modest set of feasible points ($\leq 8000$ points) for training, and these points are the \emph{worst-performing} feasible points across the pool of randomly sampled data.
%
This dataset is meant to reflect an easily obtainable and an application-agnostic dataset of accelerators that could have been generated once and stored to disk, or might come from real physical experiments. 
%
We emphasize that no assumptions or domain knowledge about the application use case was made during dataset collection.
%
Table~\ref{tab:models} depicts the list of target applications, evaluated in this work, includes three variations of MobileNet~\citep{edgetpu:arxiv:2020,mnv2:arxiv:2018,mnv3:cvpr:2019}, three in-house industry-level models for object detection (M4, M5, M6; names redacted to prevent anonymity violation), a U-net model~\citep{unet}, and two RNN-based encoder-decoder language models~\citep{trnn01,trnn02,trnn03,trnn04}. 
%
These applications span the gamut from small models, such as \msix, with only 0.4~MB model parameters that demands less on-chip memory, to the medium-sized models ($\geq$~5~MB), such as MobileNetV3 and \mfour models, and large models ($\geq$~19~MB), such as t-RNNs, hence requiring larger on-chip memory. 
%
\begin{table*}[t!]
\vspace{0.05cm}
\small{
\centering
% \vspace*{0.1cm}
\caption{\footnotesize The description of the applications, their domains, number of (convolutions, depth-wise convolutions, feed-forward) XLA ops, model parameter size, instruction sizes in bytes, number of compute operations.}
\label{tab:models}
\vspace{-0.1in}
\resizebox{\textwidth}{!}{\begin{tabular}{l|l|c|r|r|r}
\toprule
\textbf{Name}&\textbf{Domain}&\textbf{\# of XLA Ops (Conv, D/W, FF)}&\textbf{Model Param}&\textbf{Instr. Size}&\textbf{\# of Compute Ops.}\\\midrule
{\textbf{MobileNetEdgeTPU}} &Image Class.&(45, 13, 1)&3.87~MB&476,736&1,989,811,168\\\hline
{\textbf{MobileNetV2}}&Image Class.&(35, 17, 1)&3.31~MB&416,032&609,353,376\\\hline
{\textbf{MobileNetV3}}&Image Class.&(32, 15, 17)&5.20~MB&1,331,360&449,219,600\\\hline
\textbf{\mfour}&Object Det.&(32, 13, 2)&6.23~MB&317,600&3,471,920,128\\\hline
\textbf{\mfive}&Object Det.&(47, 27, 0)&2.16~MB&328,672&939,752,960\\\hline
\textbf{\msix}&Object Det.&(53, 33, 2)&0.41~MB&369,952&228,146,848\\\hline
\textbf{U-Net}&Image Seg.&(35, 0, 0)&3.69~MB&224,992&13,707,214,848\\\hline
\textbf{t-RNN Dec}&Speech Rec.&(0, 0, 19)&19~MB&915,008&40,116,224\\\hline
\textbf{t-RNN Enc}&Speech Rec.&(0, 0, 18)&21.62~MB&909,696&45,621,248\\
\bottomrule
\end{tabular}}
}
\vspace{-0.25cm}
\end{table*}

\niparagraph{Evaluation protocol.}
%
To compare simulator-driven methods and our data-driven method, we limit the number of feasible points (costly to evaluate) that can be used by any algorithm to equal amounts. We still provide infeasible points to any method and leave it up to the optimization method to use it or not.
This ensures our comparisons are fair in terms of the amount of data available to each method.
However, it is worthwhile to note that in contrast to our method where \emph{worse-quality} data points from small offline dataset are used, the simulator-driven methods have an inherent advantage because they can steer the query process towards the points that are more likely to be better in terms of performance.
Following prior work in data-driven design~\citep{brookes19a}, we evaluate each run of a method by first sampling the top $n=256$ design candidates according to the algorithm's predictions, evaluating all of these under the ground truth objective function and recording the performance of the best accelerator design. The final reported results is the median of ground truth objective values across five independent runs.

\vspace{-0.2cm}
\section{\primemethodname: Architecting Accelerators via Conservative Models}
\label{sec:prime_method}
\vspace{-0.2cm}
%
As shown in Figure~\ref{fig:prime_method}, our method first learns a conservative surrogate model of the optimization objective using the offline dataset. Then, it optimizes this learned model using a discrete optimizer. The optimization process does not require access to a simulator, nor to real-world experiments beyond the initial dataset, except when evaluating the final top-performing $n=256$ designs (Section~\ref{sec:accel}). This is built on the principle of conservative value estimation for offline RL from Chapter~\ref{chapter:cql}: while conservative value estimation methods that we have discussed so far in this dissertation attempt to estimate a pessimistic value-function, our approach \primemethodname\ is a special sub-case that involves a single decision-making step (as opposed to making a sequence of decisions).

\begin{figure}
    \centering
    \vspace{-0.3cm}
    \includegraphics[width=0.7\linewidth]{chapters/prime/figs/overview/mbo-method.pdf}
    \vspace{-0.1cm}
    \caption{\small{Overview of \primemethodname\ which trains a conservative model ${f}_\theta(\rvx_i)$ using Equation~\ref{eqn:final_training}. Our neural net model for ${f}_\theta(\rvx)$ utilizes two transformer layers~\citep{vaswani2017attention}, and a multi-headed architecture which is pooled via a soft-attention layer.}}
    \vspace{-0.3cm}
    \label{fig:prime_method}
\end{figure}

\vspace{-0.2cm}
\subsection{Learning Conservative Models Using Logged Offline Data}
\vspace{-0.2cm}
Our goal is to utilize a logged dataset of feasible accelerator designs labeled with the desired performance metric (e.g., latency),  $\mathcal{D}_\text{feasible}$, and additional infeasible designs, $\mathcal{D}_\text{infeasible}$ to learn a mapping ${f}_\theta: \mathcal{X} \rightarrow \mathbb{R}$, that maps the accelerator configuration $\rvx$ to its corresponding metric $y$. This learned surrogate model can then be optimized by the optimizer. While a straightforward approach for learning such a mapping is to train it via supervised regression (which is equivalent to standard temporal-difference learning for value estimation in one-step decision-making problems, where an episode terminates as soon as one action is performed and a terminal reward is collected), by minimizing the mean-squared error $\E_{\rvx_i, y_i \sim \mathcal{D}}[(f_\theta(\rvx_i) - y_i)^2]$, as we have seen in this dissertation and in other prior works~\citep{kumar2019model}, such predictive models can arbitrarily overestimate the value of an unseen input $\rvx_i$. This can cause the optimizer to find a solution $\rvx^*$ that performs poorly in the simulator but looks promising under the learned model. We empirically validate this overestimation hypothesis and find it to confound the optimizer in on our problem domain as well (See Figure~\ref{fig:cali_plot} in Appendix). 

To prevent overestimated values at unseen inputs from confounding the optimizer, we follow the conservative value estimation recipe from Chapter~\ref{chapter:cql} (specifically, Equation~\ref{eqn:modified_policy_eval}) and train $f_\theta(\rvx)$ with an additional term that explicitly maximizes the function value $f_\theta(\rvx)$ at out-of-distribution $\rvx$ values. Note that instead of minimizing the value $f_\theta(\rvx)$ on unseen $\rvx$, we maximize this value because the problem in Equation~\ref{eqn:opt_prob} requires us to find the minimum of the ground-truth function. Such unseen designs $\rvx$ are ``negative mined'' by running a few iterations of a stochastic optimization procedure that aims to maximize $f_\theta$ in the inner loop. In the context of this single-step decision-making problem, this procedure is analogous to adversarial training~\citep{goodfellow2014explaining}. Equation~\ref{eqn:training} formalizes this objective:
\newcommand{\editcolor}{black}
\begin{equation}
\label{eqn:training}
    \theta^* := \arg \min_\theta~~ \mathcal{L}(\theta):= \E_{\rvx_i, y_i \sim \mathcal{D}_{\text{feasible}}} \left[ (f_\theta(\rvx_i) - y_i)^2 \right] - \alpha \E_{\rvx^{-}_i \sim \textcolor{\editcolor}{\mathrm{Opt}(f_\theta)}} \left[f_\theta(\rvx^-_i) \right].
\end{equation}
$\rvx^{-}_i$ denotes the negative samples produced from an optimizer $\mathrm{Opt}(\cdot)$ that attempts to maximize the current learned objective model, $f_\theta$. We will discuss our choice of $\mathrm{Opt}$ in the Appendix~\ref{app:prime_details}.  
 
\vspace{-0.2cm}
\subsection{Incorporating Design Constraints by Training on Infeasible Points}
\vspace{-0.2cm}
While conservative value estimation methods provide us with a recipe for optimizing Equation~\ref{eqn:training} via a conservative surrogate model, this is not enough when optimizing over accelerators, as we will also show empirically (Appendix~\ref{app:prime_additional_experiments}).
This is because while explicitly minimizing for out-of-distribution designs constrains the design procedure to the data, it does not provide any information about accelerator design constraints. Fortunately, this information can be provided by infeasible points, $\mathcal{D}_\text{infeasible}$. The training procedure in Equation~\ref{eqn:training} provides a simple way to do incorporate such infeasible points: we simply incorporate $\rvx'_i \sim \mathcal{D}_{\text{infeasible}}$ as additional out-of-distribution samples and maximize the prediction at these points. This gives rise to our final objective:
\begin{align}
\label{eqn:final_training}
    \min_\theta~~ \mathcal{L}^\text{inf}(\theta) := \mathcal{L}(\theta)
    - \textcolor{blue}{\beta \E_{\rvx'_i \sim \mathcal{D}_{\text{infeasible}}}\left[ f_\theta(\rvx'_i) \right]}
\end{align}

\vspace{-0.2cm}
\subsection{Optimizing Multiple Applications and Zero-Shot Design}
\vspace{-0.2cm}
%
One of the central benefits of an offline learning approach is that it enables learning powerful models that generalize over the space of applications, potentially being effective for new unseen application domains. In our experiments, we evaluate \primemethodname\ on designing accelerators for multiple applications denoted as $k=1, \cdots, K$, jointly or for a novel unseen application. In this case, we utilized a dataset $\mathcal{D} = \{\mathcal{D}_1, \cdots, \mathcal{D}_K\}$, where each $\mathcal{D}_k$ consists of a set of accelerator designs, annotated with the latency value and the feasibility criterion for a given application $k$. While there are a few overlapping designs in different parts of the dataset annotated for different applications, most of the designs only appear in one part. To train a single conservative model $f_\theta(\cdot)$ for multiple applications, we extend the training procedure in Equation~\ref{eqn:final_training} to incorporate \textit{context vectors} $\rvc_k \in \mathbb{R}^d$ for various applications driven by a list of application properties in Table~\ref{tab:models}. A context vector is akin to a state in the context of full sequential reinforcement learning. 

The learned function in this setting is now conditioned on the context $f_\theta(\rvx, \rvc_k)$. We train $f_\theta$ via the objective in Equation~\ref{eqn:final_training}, but in expectation over all the contexts and their corresponding datasets: $\min_\theta \E_{\textcolor{blue}{k \sim [K]}}\left[\mathcal{L}^\text{inf}_k(\theta)\right]$. Once such a contextual model is learned, we can either optimize the average models across a set of contexts $\{\rvc_i, \rvc_2, \cdots, \rvc_n\}$ to obtain an accelerator that is optimal for multiple applications simultaneously on an average (``multi-model'' optimization), or optimize this contextual model for a novel context vector, corresponding to an unseen application (``zero-shot'' generalization). In this case, \primemethodname\ is not allowed to train on any data corresponding to this new unseen application.  While such zero-shot generalization might appear surprising at first, note that the context vectors are not simply one-hot vectors, but consist of parameters with semantic information, which the conservative model can generalize over.

\niparagraph{Learned conservative model optimization.} Prior work~\citep{yazdanbakhsh2021apollo} has shown that the most effective optimizers for accelerator design are meta-heuristic/evolutionary optimizers. We therefore choose to utilize, firefly~\citep{yang2010nature,yang2010eagle,liu2013adaptive} to optimize our conservative model. This algorithm maintains a set of optimization candidates (a.k.a. ``fireflies'') and jointly update them towards regions of low objective value, while adjusting their relative distances appropriately to ensure multiple high-performing, but diverse solutions. We discuss additional details in Appendix~\ref{sec:prime_practical_implementation}.

\niparagraph{Cross validation: which model and checkpoint should we evaluate?} Similarly to supervised learning, models trained via Equation~\ref{eqn:final_training} can overfit, leading to poor solutions. Thus, we require a procedure to select which hyperparameters and checkpoints should actually be used for the design. This is crucial, because we cannot arbitrarily evaluate as many models as we want against the simulator. While effective methods for model selection have been hard to develop in offline reinforcement learning~\citep{trabucco2021conservative,trabucco2021designbench}, we devised a simple scheme using a validation set for choosing the values of $\alpha$ and $\beta$ (Equation~\ref{eqn:final_training}), as well as which checkpoint to utilize for generating the design. For each training run, we hold out the best 20\% of the points out of the training set and use them \textit{only} for cross-validation as follows. Typical cross-validation strategies in supervised learning involve tracking validation error (or risk), but since our model is trained conservatively, its predictions may not match the ground truth, making such validation risk values unsuitable for our use case. Instead, we track Kendall's ranking correlation between the predictions of the learned model $f_\theta(\rvx_i)$ and the ground truth values $y_i$ (Appendix~\ref{app:prime_details}) for the held-out points for each run and then pick values of $\alpha$, $\beta$ and the checkpoint that attain the highest validation ranking correlation.
We present the pseudo-code for \primemethodname\ (Algorithm~\ref{alg:prime}) and implementation details in Appendix~\ref{sec:prime_practical_implementation}.
\vspace{-0.4cm}
\section{Related Work}
\label{sec:prime_related}
\vspace{-0.2cm}
%
Optimizing hardware accelerators has become more important recently. Prior works~\citep{bo:frontiers:2020,flexibo:arxiv:2020,cnn_gen:cyber:2020,prac_dse:mascots:2019,accel_gen:dac:2018,spatial:pldi:2018,automomml:hpc:2016,opentuner:pact:2014,hegdemind,magnet,autodnnchip} mainly rely on expensive-to-query hardware simulators to navigate the search space \review{and/or target single-application accelerators}. For example, HyperMapper~\citep{prac_dse:mascots:2019} targets compiler optimization for FPGAs by continuously interacting with the simulator in a design space with relatively few infeasible points. Mind Mappings~\citep{hegdemind}, optimizes software mappings to a fixed hardware provided access to millions of feasible points and throws away infeasible points during learning. {MAGNet~\citep{magnet} uses a combination of pruning heuristics and online Bayesian optimization to generate accelerators for image classification models in a single-application setting.} {AutoDNNChip~\citep{autodnnchip} uses two-level online optimization to generate customized  accelerators for ASIC and FPAG platforms.} In contrast, \primemethodname~, does not only learn a conservative model of the objective function from offline data but can also leverage information from infeasible points and can work with just a few thousand feasible points. {In addition, we devise a contextual version of \primemethodname\ that is effective in designing accelerators that are jointly optimized for multiple applications, different from prior work.} Finally, to our knowledge, our work, is the first to demonstrate generalization to unseen applications for accelerator design, outperforming state-of-the-art online methods.

A popular approach for solving black-box optimization problems is model-based optimization (MBO)~\citep{snoek15scalable,shahriari2016TakingTH,snoek2012practical}. Most of the classical MBO methods fail to scale to high-dimensions, and have been extended with neural networks~\citep{snoek15scalable,snoek2012practical,kim2018attentive,garnelo18neural,garnelo18conditional,p3bo:arxiv:2020,angermueller2019model,mirhoseini2020chip}. While these methods work well in the active setting, they are susceptible to out-of-distribution inputs~\citep{trabucco2021designbench} in the offline, data-driven setting. To prevent this, many methods for offline model-based optimization constrain the optimizer to the manifold of valid, in-distribution inputs~\citep{brookes19a,fannjiang2020autofocused,kumar2019model}. However, modeling the manifold of valid inputs can be challenging for accelerators. \primemethodname\ dispenses with the need for generative modeling, while still avoiding out-of-distribution inputs. \primemethodname\ takes a conservative value estimation approach. However, unlike these approaches, \primemethodname\ can handle constraints by learning from infeasible data. In addition, while prior works in this area mostly restricted their design problem to a single application, we show that \primemethodname\ is effective for multi-application optimization and zero-shot generalization.
\section{Experimental Evaluation}
\label{sec:eval}
%
Our evaluations aim to answer the following questions: \textbf{Q(1)} {Can \primemethodname\ design accelerators tailored for a given application that are better than the best observed configuration in the training dataset, and comparable to or better than state-of-the-art simulation-driven methods under a given simulator-query budget?} \textbf{Q(2)} {Does \primemethodname\ reduce the total simulation time compared to other methods?} \textbf{Q(3)} {Can \primemethodname\ produce hardware accelerators for a family of different applications?} \textbf{Q(4)} {Can \primemethodname\ trained for a family of applications extrapolate to designing a high-performing accelerator for a new, unseen application, thereby enabling data reuse?} Additionally, we ablate various properties of \primemethodname\ (Appendix~\ref{sec:appx_ablation}) and evaluate its efficacy in designing accelerators with distinct dataflow architectures, with a larger search space (up to 2.5$\times10^{114}$ possible candidates).

\begin{wrapfigure}{r}{0.35\textwidth}
    \centering
    \vspace{-0.45cm}
    \includegraphics[width=0.98\linewidth]{chapters/prime/figs/motivation/simulation_time.png}
    \vspace{-0.3cm}
    \caption{\footnotesize{Comparing the total simulation time of \primemethodname\ (\review{for \primemethodname\ this is the total time for a forward-pass through the trained surrogate on a CPU}) and evolutionary method on MobileNetEdgeTPU. \primemethodname\ only requires about \textbf{7\%} of the total simulation time of the online method.}}
    \vspace{-0.35cm}
    \label{fig:convergence_time}
\end{wrapfigure}
\niparagraph{Baselines and comparisons.}
%
We compare \primemethodname\ against three online optimization methods that actively query the simulator: \textbf{(1)} evolutionary search with the firefly optimizer~\citep{yazdanbakhsh2021apollo} (``Evolutionary''), which is the shown to outperform other online methods for accelerator design; \textbf{(2)} Bayesian Optimization (``Bayes Opt'')~\citep{vizier:sigkdd:2017},
\textbf{(3)} MBO~\citep{angermueller2019model}. 
%
%
In all the experiments, we grant all the methods the same number of feasible points. Note that our method do not get to select these points, and use the same exact offline points across all the runs, while the online methods can actively select which points to query, and therefore require new queries for every run. 
%
``$\mathcal{D}$(Best in Training)'' denotes the best latency value in the training dataset used in \primemethodname. We also present ablation results with different components of our method removed in Appendix~\ref{sec:appx_ablation}, where we observe that utilizing both infeasible points and negative sampling are generally important for attaining good results. 
%
Appendix~\ref{app:additional_experiments} presents additional comparisons to COMs~\citep{trabucco2021conservative}---which only obtains negative samples via gradient ascent on the learned surrogate and does not utilize infeasible points---and P3BO~\citep{p3bo:arxiv:2020}---an state-of-the-art online method in biology. 

\niparagraph{Architecting application-specific accelerators.}
%
We first evaluate \primemethodname\ in designing specialized accelerators for each of the applications in Table~\ref{tab:models}.
%
We train a conservative surrogate using the method in Section~\ref{sec:method} on the logged dataset for each application separately.
%
The area constraint $\alpha$ (Equation~\ref{eqn:opt_prob}) is set to $\alpha = 29~\text{mm}^2$, a realistic budget for accelerators~\citep{yazdanbakhsh2021apollo}. Table~\ref{table:results_single_task} summarizes the results.
%
On average, the best accelerators designed by \primemethodname\ outperforms the best accelerator configuration in the training dataset (last row Table~\ref{table:results_single_task}), by 2.46$\times$.
%
\begin{table*}[t!]
\small
\centering
\vspace*{0.1cm}
\caption{\label{table:results_single_task}Optimized objective values (i.e., latency in milliseconds) obtained by various methods for the task of learning accelerators specialized to a given application. Lower latency is better. \textbf{From left to right}: our method, online Bayesian optimization (``Bayes Opt''), online evolutionary algorithm (``Evolutionary''), and the best design in the training dataset. On average (last row), \methodname\ improves over the best in the dataset by 2.46$\times$ (up to 6.69$\times$ in t-RNN Dec) and outperforms best online optimization methods by 1.54$\times$ (up to 6.62$\times$ in t-RNN Enc). The best accelerator configurations identified is highlighted in bold.}
\resizebox{0.95\textwidth}{!}{% <------ Don't forget this %
\begin{tabular}{l|l|l|l|l|l}
\toprule
&&\multicolumn{3}{c|}{\textbf{Online Optimization}}&\\\cline{3-5}
\textbf{Application} & \textbf{\methodname} & \textbf{Bayes Opt} & \textbf{Evolutionary}&\textbf{MBO}&$\mathcal{D}$ \textbf{(Best in Training)}\\\midrule
{MobileNetEdgeTPU}&\textbf{298.50}&319.00&320.28&332.97&354.13\\\hline 
{MobileNetV2}&\textbf{207.43}&240.56&238.58&244.98&410.83\\\hline
{MobileNetV3}&\textbf{454.30}&534.15&501.27&535.34&938.41\\\hline
{\mfour}&\textbf{370.45}&396.36&383.58&405.60&779.98\\\hline
{\mfive}&208.21&201.59&\textbf{198.86}&219.53&449.38\\\hline
{\msix}&131.46&121.83&120.49&\textbf{119.56}&369.85\\\hline
{U-Net}&\textbf{740.27}&872.23&791.64&888.16&1333.18\\\hline
{t-RNN Dec}&\textbf{132.88}&771.11&770.93&771.70&890.22\\\hline
{t-RNN Enc}&\textbf{130.67}&865.07&865.07&866.28&584.70\\\bottomrule
\CC \textbf{Geomean of \methodname's Improvement}&\CC~1.0$\times$&\CC~\texttt{\textbf{1.58$\times$}}&\CC~\texttt{\textbf{1.54$\times$}}&\CC~\texttt{\textbf{1.61$\times$}}&\CC~\texttt{\textbf{2.46$\times$}}\\
\bottomrule
\end{tabular}% <------ Don't forget this %
}
\vspace{-0.2cm}
\end{table*} 
%
\primemethodname\ also outperforms the accelerators in the best online method by 1.54$\times$ (up to 5.80$\times$ and 6.62$\times$ in t-RNN Dec and t-RNN Enc, respectively).
%
Moreover, perhaps surprisingly, \primemethodname\ generates accelerators that are better than all the online optimization methods in 7/9 domains, and performs on par in several other scenarios (on average only 6.8$\%$ slowdown compared to the best accelerator with online methods in \mfive and \msix). 
%
These results indicates that offline optimization of accelerators using \primemethodname\ can be more data-efficient compared to online methods with active simulation queries.
%
To answer \textbf{Q(2)}, we compare the total simulation time of \primemethodname\ and the best evolutionary approach from Table~\ref{table:results_single_task} on the MobileNetEdgeTPU domain.
%
On average, not only that \primemethodname\ outperforms the best online method \review{that we evaluate}, but also considerably reduces the total simulation time by 93\%, as shown in Figure~\ref{fig:convergence_time}.
% 
Even the total simulation time to the first occurrence of the final design that is eventually returned by the online methods is about 11$\times$ what \primemethodname\ requires to fine a better design.
%
This indicates that data-driven \primemethodname\ is much more preferred in terms of the performance-time trade-off. \review{The fact that our offline approach \primemethodname\ outperforms the online evolutionary method (and also other state-of-the-art online MBO methods; see Table~\ref{table:p3bo_vs_prime}) is surprising, and we suspect this is because online methods get stuck early-on during optimization, while utilizing offline data allows us \primemethodname\ to find better solutions via generalization (see Appendix~\ref{app:online_opt_details}).} 

\begin{wrapfigure}{r}{0.35\textwidth}
    \centering
    \vspace{-0.5cm}
    \includegraphics[width=0.98\linewidth]{chapters/prime/figs/motivation/convergence_time_seven_models.png}
    \vspace{-0.25cm}
    \caption{\footnotesize{Comparing the total simulation time needed by \primemethodname\ and online methods on seven models (Area $\leq$ 100mm$^2$)~. \primemethodname\ only requires about 1\%, 6\%, and 0.9\% of the total simulation time of Evolutionary, MBO, and Bayes Opt, respectively, although \primemethodname\ outperforms the best online method by 41\%.}}
    \vspace{-0.6cm}
    \label{fig:convergence_time_seven_models}
\end{wrapfigure}
\niparagraph{Architecting accelerators for multiple applications.}
%
To answer \textbf{Q(3)}, we evaluate the efficacy of the contextual version of \primemethodname\ in designing an accelerator that attains the lowest latency averaged over a set of application domains.
%
\begin{table*}[t!]
\small{
\centering
% \vspace*{0.1cm}
\caption{\label{table:results_multi_task} Optimized average latency (the lower, the better) across multiple applications (up to ten applications) for \primemethodname\ and best online algorithms (Evolutionary, MBO). Each row show the (Best, Median) of average latency across five runs. The geometric mean of \primemethodname's improvement over other methods (last row) indicates that \primemethodname\ is at least 21\% better.}
\vspace{-0.1cm}
\resizebox{\textwidth}{!}{\begin{tabular}{l|l|l|l|l}
\toprule
\textbf{Applications}&\textbf{Area}&\textbf{\primemethodname\ (Ours)}&\textbf{Evolutionary~ (Online)}&\textbf{MBO~(Online)}\\\midrule
% 3 Models
{MobileNet~(EdgeTPU, V2, V3)}&29~mm$^2$&(310.21, 334.70)&(\textbf{315.72}, \textbf{325.69})&(342.02, 351.92)\\\hline
% 4 Models
{MobileNet~(V2, V3), \mfive, \msix}&29~mm$^2$&(\textbf{268.47}, \textbf{271.25})
&(288.67, 288.68)&(295.21, 307.09)\\\hline
% 6 Models
{MobileNet~(EdgeTPU, V2, V3), \mfour, \mfive, \msix}&29~mm$^2$&(\textbf{311.39}, \textbf{313.76})&(314.31, 316.65)&(321.48, 339.27)\\\hline
% 7 Models
{MobileNet~(EdgeTPU, V2, V3), \mfour, \mfive, \msix, U-Net, t-RNN-Enc}&29~mm$^2$&\textbf{(305.47, 310.09)}&(404.06, 404.59)&(404.06, 412.90)\\\hline
{MobileNet~(EdgeTPU, V2, V3), \mfour, \mfive, \msix, t-RNN-Enc}&100~mm$^2$&\textbf{(286.45, 287.98)}&(404.25, 404.59)&(404.06, 404.94)\\\hline
% 9 Models
{MobileNet~(EdgeTPU, V2, V3), \mfour, \mfive, \msix, t-RNN~(Dec, Enc)}&29~mm$^2$&(\textbf{426.65}, \textbf{426.65})&(586.55, 586.55)&(626.62, 692.61)\\\hline
{MobileNet~(EdgeTPU, V2, V3), \mfour, \mfive, \msix, U-Net, t-RNN~(Dec, Enc)}&100~mm$^2$&(\textbf{383.57}, \textbf{385.56})&(518.58, 519.37)&(526.37, 530.99)\\\bottomrule
\CC \textbf{Geomean of \primemethodname's Improvement}&\CC~---&\CC~\texttt{\textbf{(1.0$\times$, 1.0$\times$)}}&\CC~\texttt{\textbf{(1.21$\times$, 1.20$\times$)}}&\CC~\texttt{\textbf{(1.24$\times$, 1.27$\times$)}}\\
\bottomrule
\end{tabular}}
}
\vspace{-0.2cm}
\end{table*}
%
As discussed previously, the training data used does not label a given accelerator with latency values corresponding to each application, and thus, \primemethodname\ must extrapolate accurately to estimate the latency of an accelerator for a context it is not paired with in the training dataset.
%
This also means that \primemethodname\ cannot simply return the accelerator with the best average latency and must run non-trivial optimization. 
%
We evaluate our method in seven different scenarios (Table~\ref{table:results_multi_task}), 
%
%
comprising various combinations of models from Table~\ref{tab:models} and under different area constraints, where the smallest set consists of the three MobileNet variants and the largest set consists of nine models from image classification, object detection, image segmentation, and speech recognition.
%
This scenario is also especially challenging for online methods since the number of jointly feasible designs is expected to drop significantly as more applications are added.
%
For instance, for the case of the MobileNet variants, the training dataset only consists of a few (20-30) accelerator configurations that are jointly feasible and high-performing (Appendix~\ref{sec:appx_multi_task_tsne}---Figure~\ref{fig:tsne_overlap}). 

Table~\ref{table:results_multi_task} shows that, on average, \primemethodname\ finds accelerators that outperform the best online method by 1.2$\times$ (up to 41\%).
%
While \primemethodname\ performs similar to online methods in the smallest three-model scenario (first row), it outperforms online methods as the number of applications increases and the set of applications become more diverse.
%
In addition, comparing with the best jointly feasible design point across the target applications, \primemethodname\ finds significantly better accelerators (3.95$\times$).
%
Finally, as the number of model increases the total simulation time difference between online methods and \primemethodname\ further widens (Figure~\ref{fig:convergence_time_seven_models}).
%
These results indicate that \primemethodname\ is effective in designing accelerators jointly optimized across multiple applications while reusing the same dataset as for the single-task, and scales more favorably than its simulation-driven counterparts.
%
Appendix~\ref{app:analysis} expounds the details of the designed accelerators for nine applications, comparing our method and the best online method.

\niparagraph{Accelerating previously unseen applications (``zero-shot'' optimization).}
%
Finally, we answer \textbf{Q(4)} by demonstrating that our data-driven offline method, \primemethodname\ enables effective data reuse by using logged accelerator data from a set of applications to design an accelerator for an unseen new application, without requiring any training on data from the new unseen application(s).
%
We train a contextual version of \primemethodname\ using a set of ``training applications'' and then optimize an accelerator using the learned surrogate with different contexts corresponding to ``test applications,'' without any additional query to the test application dataset.
%
Table~\ref{table:zero_shot} shows, on average, \primemethodname\ outperforms the best online method by 1.26$\times$ (up to 66$\%$) and only 2$\%$ slowdown in 1/4 cases.
%
Note that the difference in performance increases as the number of training applications increases.
%
These results show the effectiveness of \primemethodname\ in the zero-shot setting (more results in Appendix~\ref{sec:app_zero_shot}).

\vspace{-0.1cm}
\niparagraph{Applying \primemethodname\ on other accelerator architectures and dataflows.}
%
Finally, to assess the the generalizability of \primemethodname\ to other accelerator architectures~\citep{kao2020confuciux}, we evaluate \primemethodname\ to optimize latency of two style of dataflow accelerators---NVDLA-style and ShiDianNao-style---across three applications (Appendix~\ref{sec:dla_shi_fast} details the methodology).
%
As shown in Table~\ref{table:dla_shi}, \primemethodname\ outperforms the online evolutionary method by 6\% and improves over the best point in the training dataset by 3.75$\times$.
%
This demonstrates the efficacy of \primemethodname\ with different dataflows and large design spaces.
%
\begin{table*}[t!]
\small
\centering
\vspace*{-0.1cm}
\caption{\label{table:zero_shot}Optimized objective values (i.e., latency in milliseconds) under zero-shot setting. Lower latency is better. From left to right: the applications used to train the surrogate model in \methodname\, the target applications for which the accelerator is optimized for, the area constraint of the accelerator, \methodname's (best, median) latency, and best online method's (best, median) latency. \methodname\ does not use any additional data from the target applications. On average (last row), \methodname\ yields optimized accelerator for target applications (with zero query to the target applications' dataset) with 1.26$\times$ (up to 1.66$\times$) lower latency over the best online method. The best accelerator configurations identified is highlighted in bold.}
\vspace{-0.1cm}
\resizebox{\textwidth}{!}{% <------ Don't forget this %
\begin{tabular}{l|l|l|l|l}
\toprule
\textbf{Train Applications}&\textbf{Test Applications}&\textbf{Area}&\textbf{\primemethodname\ (Ours)}&\textbf{Evolutionary~(Online)}\\\midrule
% Train = MobilenetEdge, MobilenetV3, Target = MobilenetV2
MobileNet~(EdgeTPU, V3)&{MobileNetV2}&29~mm$^2$&(\textbf{311.39}, \textbf{313.76})&(314.31, 316.65)\\\hline
MobileNet~(V2, V3), \mfive, \msix&{MobileNetEdge, \mfour}&29~mm$^2$&(357.05, 364.92)&(\textbf{354.59}, \textbf{357.29})\\\hline
MobileNet~(EdgeTPU, V2, V3), \mfour, \mfive, \msix, t-RNN Enc&{U-Net, t-RNN Dec}&29~mm$^2$&(\textbf{745.87}, \textbf{745.91})&(1075.91, 1127.64)\\\hline
MobileNet~(EdgeTPU, V2, V3),\mfour, \mfive, \msix, t-RNN Enc&{U-Net, t-RNN Dec}&100~mm$^2$&(\textbf{517.76}, \textbf{517.89})&(859.76, 861.69)\\\bottomrule
\CC \textbf{Geomean of \primemethodname's Improvement}&\CC---&\CC---&\CC~(1.0$\times$, 1.0$\times$)&\CC~(\texttt{\textbf{1.24$\times$}}, \texttt{\textbf{1.26$\times$}})\\
\bottomrule
\end{tabular}% <------ Don't forget this %
}
% \vspace{-0.1cm}
\end{table*}
\begin{table*}[t!]
\small
\centering
\renewcommand{\arraystretch}{1.1}
\vspace*{-0.05cm}
\caption{\label{table:dla_shi}Optimized objective values (i.e. total number of cycles) for two different dataflow architectures, NVDLA-style~\citep{nvdla} and ShiDianNao-style~\citep{shidiannao}, across three classes of applications. The maximum search space for the studied accelerators are $\approx$~2.5$\times$10$^{114}$. \primemethodname\ generalizes to other classes of accelerators with larger search space and outperforms the best online method by 1.06$\times$ and the best data seen in training by 3.75$\times$ (last column). The best accelerator configurations is highlighted in bold.}
\resizebox{0.95\textwidth}{!}{\begin{tabular}{l|l|l|l|l}
\toprule
\textbf{Applications}&\textbf{Dataflow}&\textbf{\primemethodname}&\textbf{Evolutionary ~(Online)}&\textbf{$\mathcal{D}$ (Best in Training)}\\\midrule
% MobileNetV2
MobileNetV2&NVDLA&\textbf{2.51$\times$10$^7$}&2.70$\times$10$^7$&1.32$\times$10$^8$\\\hline
MobileNetV2&ShiDianNao&\textbf{2.65$\times$10$^7$}&2.84$\times$10$^7$&1.27$\times$10$^8$\\\bottomrule
% ResNet50
ResNet50&NVDLA&\textbf{2.83$\times$10$^8$}&3.13$\times$10$^8$&1.63$\times$10$^9$\\\hline
ResNet50&ShiDianNao&\textbf{3.44$\times$10$^8$}&3.74$\times$10$^8$&2.05$\times$10$^9$\\\bottomrule
% Transformer
Transformer&NVDLA&\textbf{7.8$\times$10$^8$}&\textbf{7.8$\times$10$^8$}&1.3$\times$10$^9$\\\hline
Transformer&ShiDianNao&\textbf{7.8$\times$10$^8$}&\textbf{7.8$\times$10$^8$}&1.5$\times$10$^9$\\\bottomrule
\CC \textbf{Geomean of \primemethodname's Improvement}&\CC---&\CC~\texttt{\textbf{1.0$\times$}}&\CC~\texttt{\textbf{1.06$\times$}}&\CC~\texttt{\textbf{3.75$\times$}}\\
\bottomrule
\end{tabular}}
\vspace{-0.4cm}
\end{table*}
% \vspace{-0.2cm}
\section{Discussion}
\label{sec:discussion}
\vspace{-0.1cm}
%
In this work, we present a data-driven offline optimization method, \primemethodname\ to automatically architect hardware accelerators. Our method learns a conservative model of the objective function by leveraging  infeasible data points to better model the desired objective function of the accelerator using a one-time collected dataset of accelerators, thereby alleviating the need for time-consuming simulation.
%
Our results show that, on average, our method outperforms the best designs observed in the logged data by 2.46$\times$ and improves over the best simulator-driven approach by about 1.54$\times$. 
%
In the more challenging setting of designing accelerators jointly optimal for multiple applications or for new, unseen applications, zero-shot, \primemethodname\ outperforms simulator-driven methods by 1.2$\times$, while reducing the total simulation time by 99\%.
%
{The efficacy of \primemethodname\ highlights the potential for utilizing the logged offline data in an accelerator design pipeline. While \primemethodname\ outperforms the online methods we utilize, in principle, a strong online method can be devised by running \primemethodname\ in the inner loop. Our goal is to not advocate that offline methods must replace online methods, but that training a strong offline optimization algorithm on offline datasets of low-performing designs can be a highly effective ingredient in hardware accelerator design.}


\section*{Acknowledgements}
%
We thank the ``Learn to Design Accelerators'' and EdgeTPU teams at Google for their invaluable feedback.
%
In addition, we extend our gratitude to the Vizier team, Christof Angermueller, Sheng-Chun Kao, Samira Khan, Stella Aslibekyan, and Xinyang Geng for their help with experiment setups and insightful comments. 


\bibliography{paper.bib}
\bibliographystyle{iclr2022_conference}

\appendix
\newpage
\section{Additional Experiments}
In this section, we present additional experiments compared to the method of \citet{trabucco2021conservative}, present some additional results obtained by jointly optimizing multiple applications (Appendix~\ref{app:additional_experiments_multi}), provide an analysis of the designed accelerators (Appendix~\ref{app:analysis}) and finally, discuss how our trained conservative surrogate can be used with a different evaluation time constraint (Appendix~\ref{app:transfer}).

\subsection{Comparison to Other Baseline Methods}
\label{app:additional_experiments}

% \paragraph{Comparison to COMs.} 
% %
% In this section, we perform a comparative evaluation of \primemethodname\ to the COMs method~\citep{trabucco2021conservative}. 
% %
% Like several offline reinforcement learning algorithms~\citep{kumar2020conservative}, our method, \primemethodname\, and COMs are based on the key idea of learning a conservative surrogate of the desired objective function, such that it does not overestimate the value of unseen data points, which prevents the optimizer from finding accelerators that appear promising under the learned model but are not actually promising under the actual objective. 
% %
% The key differences between our method and COMs are:
% %
% \textbf{(1)} \primemethodname\ uses an evolutionary optimizer ($\mathrm{Opt}(\cdot)$) for negative sampling compared to gradient ascent of COMs, which can be vastly beneficial in discrete design spaces as our results show empirically,
% %
% \textbf{(2)} \primemethodname\ can explicitly learn from infeasible data points provided to the algorithm, while COMs does not have a mechanism to incorporate the infeasible points into the learning of surrogate.
% %
% To further assess the importance of these differences in practice, we run COMs on three tasks from Table~\ref{table:results_single_task}, and present a comparison our method, COMs, and Standard method in Table~\ref{table:coms_vs_prime}.
% %
% The ``Standard'' method represents a surrogate model without utilizing any infeasible points.
% %
% On average, \primemethodname\ outperforms COMs by 1.17$\times$ (up to 1.24$\times$ in \msix).

% \begin{table}[H]
% \small
% \centering
% \vspace*{0.1cm}
% \caption{Optimized objective values (i.e., latency in milliseconds) obtained by \primemethodname\ and COMs~\citep{trabucco2021conservative} when optimizing over single applications (MobilenetV2, MobilenetV3 and \msix), extending Table~\ref{table:results_single_task}. Note that \primemethodname\ outperforms COMs. However, COMs improves over baseline ``Standard'' method (last column).}
% \label{table:coms_vs_prime}
% \resizebox{0.8\textwidth}{!}{
% \begin{tabular}{l||l||l||l}
% \toprule
% \textbf{Application}&\textbf{\primemethodname~}\textbf{(Ours)}&\textbf{COMs}&\textbf{Standard}\\\midrule
% MobileNetV2&\textbf{207.43}&251.58&374.52\\\hline
% MobileNetV3&\textbf{454.30}&485.66&575.75\\\hline
% \msix&\textbf{131.46}&163.94&180.24\\\bottomrule
% \CC \textbf{Geomean of \primemethodname's Improvement}&\CC~\texttt{\textbf{1.0$\times$}}&\CC~\texttt{\textbf{1.17$\times$}}&\CC~\texttt{\textbf{1.46$\times$}}\\
% \bottomrule
% \end{tabular}
% }
% \vspace{-0.1cm}
% \end{table}

% \paragraph{Comparison to generative offline MBO methods.} 
% %
% We provide a comparison between \primemethodname\ and prior offline MBO methods based on generative models~\citep{kumar2019model}. 
% %
% We evaluate model inversion networks (MINs)~\citep{kumar2019model} on our accelerator data. However, we were unable to train a discrete
% %
% \begin{wrapfigure}{r}{0.3\textwidth}
%     \vspace{-0.1in}
%     \includegraphics[width=\linewidth]{chapters/prime/figs/latent.png}
%     \vspace{-0.1in}
% \end{wrapfigure}
% objective-conditioned GAN model to 0.5 discriminator accuracy on our offline dataset, and often observed a collapse of the discriminator.
% %
% As a result, we trained a $\delta-$VAE~\citep{razavi2019preventing}, conditioned on the objective function (i.e., latency).
% %
% A standard VAE~\citep{kingma2013auto} suffered from posterior collapse and thus informed our choice of utilizing a $\delta-$VAE. 
% %
% The latent space of a trained objective-conditioned $\delta-$VAE corresponding to accelerators on a held-out validation dataset (not used for training) is visualized in the t-SNE plot in the figure on the right.
% %
% This is a 2D t-SNE of the accelerators configurations (\S Table~\ref{tab:arch_params}). 
% %
% The color of a point denotes the latency value of the corresponding accelerator configuration, partitioned into three bins. 
% %
% Observe that while we would expect these objective conditioned models to disentangle accelerators with different objective values in the latent space, the models we trained did not exhibit such a structure, which will hamper optimization.
% %
% While our method \primemethodname\ could also benefit from a generative optimizer (i.e., by using a generative optimizer in place of $\mathrm{Opt}(\cdot)$ with a conservative surrogate), we leave it for future work to design effective generative optimizers on the accelerator manifold.


\paragraph{Comparison to P3BO.}
%
We perform a comparison against P3BO, a state-of-the-art online method in biological sequence design~\citep{p3bo:arxiv:2020}.
%
On average, \primemethodname\ outperforms the P3BO method by 2.5$\times$ (up to 8.7$\times$ in U-Net).
%
In addition, we present the comparison between the total simulation runtime of the P3BO and Evolutionary methods in Figure~\ref{fig:converg_p3bo_evolutionary}.
%
Note that, not only the total simulation time of P3BO is around 3.1$\times$ higher than the Evolutionary method, but also the latency of final optimized accelerator is around 18\% for MobileNetEdgeTPU.
%
On the other hand, the total simulation time of \primemethodname\ for the task of accelerator design for MobileNetEdgeTPU is lower than both methods (only 7\% of the Evolutionary method as shown in Figure~\ref{fig:convergence_time}).
%
\begin{table}[H]
\small
\centering
\vspace*{0.1cm}
\caption{Optimized objective values (i.e., latency in milliseconds) obtained by \primemethodname\ and P3BO~\citep{p3bo:arxiv:2020} when optimizing over single applications (MobileNetEdgeTPU, \mfour, t-RNN Dec, t-RNN Enc, and U-Net). On average, \primemethodname\ outperforms P3BO by 2.5$\times$.}
\label{table:p3bo_vs_prime}
\resizebox{0.7\textwidth}{!}{
\begin{tabular}{l||l||l}
\toprule
\textbf{Application}&\textbf{\primemethodname~}\textbf{(Ours)}&\textbf{P3BO}\\\midrule
MobileNetEdgeTPU&\textbf{298.50}&376.02\\\hline
\mfour&\textbf{370.45}&483.39\\\hline
U-Net&\textbf{740.27}&771.70\\\hline
t-RNN Dec&\textbf{132.88}&865.12\\\hline
t-RNN Enc&\textbf{130.67}&1139.48\\\bottomrule
\CC \textbf{Geomean of \primemethodname's Improvement}&\CC~\texttt{\textbf{1.0$\times$}}&\CC~\texttt{\textbf{2.5$\times$}}\\
\bottomrule
\end{tabular}
}
\vspace{-0.1cm}
\end{table}
%
\begin{figure}[t!]
    \centering
    \includegraphics[width=0.5\textwidth]{chapters/prime/figs/convergence_time_v2_evolutionary_vs_p3bo.png}
    \caption{Comparing the total simulation time needed by the P3BO and Evolutionary method on MobileNetEdgeTPU. Note that, not only the total simulation time of P3BO is around 3.1$\times$ higher than the Evolutionary method, but also the latency of final optimized accelerator is around 18\% for MobileNetEdgeTPU. The total simulation time of our method is around 7\% of the Evolutionary method (See Figure~\ref{fig:convergence_time}).}
    \label{fig:converg_p3bo_evolutionary}
    \vspace{-0.1in}
\end{figure}
%

%



\subsection{Learned Surrogate Model Reuse for Accelerator Design}
\label{app:additional_experiments_multi}
%
Extending our results in Table~\ref{table:results_multi_task}, we present another variant of optimizing accelerators jointly for multiple applications. In that scenario, the learned conservative model is reused to architect an accelerator for a subset of applications used for training. We train a contextual conservative model on the variants of MobileNet (Table~\ref{tab:models}) as discussed in Section~\ref{sec:method}, but generated optimized designs by only optimizing the average surrogate on only two variants of MobileNet (MobileNetEdgeTPU and MobileNetV2). This tests the ability of our approach \primemethodname\ to provide a general contextual conservative surrogates, that can be trained \textit{only once} and optimized multiple times with respect to different subsets of applications. Observe in Table~\ref{table:results_edge_v2}, \primemethodname\ architects high-performing accelerator configurations (better than the best point in the dataset by 3.29$\times$ -- last column) while outperforming the online optimization methods by 7\%.
%
\begin{table}[t!]
\small
\renewcommand{\arraystretch}{1.2}
\centering
\caption{Optimized objective values (i.e., latency in milliseconds) obtained by our \primemethodname\ when using the jointly optimized model on three variants of MobileNets and use for MobileNetEdgeTPU and MobileNetV2 for different dataset configurations. \primemethodname\ outperforms the best online method by \textbf{7\%} and finds an accelerator that is \textbf{3.29$\times$} better than the best accelerator in the training dataset (last row). The best accelerator configuration is highlighted in bold.}
\label{table:results_edge_v2}
\resizebox{\textwidth}{!}{
\begin{tabular}{l||l||l|l||l|l|l||l}
\toprule
&\multicolumn{3}{c||}{\textbf{\primemethodname}}& &\multicolumn{2}{c||}{\textbf{Online Optimization}}&\\\cline{2-4}\cline{6-7}
\textbf{Applications}&\textbf{All}& \textbf{-Opt}&\textbf{-Infeasible}&\textbf{Standard}& \textbf{Bayes Opt}&\textbf{Evolutionary}&\textbf{$\mathcal{D}$ (Best in Training)}\\\midrule
(MobileNetEdgeTPU, MobileNetV2)&\textbf{253.85}&297.36&264.85&341.12&275.21&271.71&834.68\\
\bottomrule
\end{tabular}
}
\end{table}


%
\subsection{Learned Surrogate Model Reuse under Different Design Constraint}
\label{app:transfer}

\begin{table}[t!]
    \small
    \renewcommand{\arraystretch}{1.2}
    \centering
    \caption{Optimized objective values (i.e., latency in milliseconds) obtained by various methods for the task of learning accelerators specialized to MobileNetEdgeTPU under chip area budget constraint 18~mm$^2$ reusing the already learned model by our method for MobileNetEdgeTPU (shown in Table~\ref{table:results_single_task}). Lower latency/runtime is better. From left to right: our method, our method without negative sampling (``\primemethodname-$\mathrm{Opt}$'') and without utilizing infeasible points (``\primemethodname-Infeasible''), standard surrogate (``Standard''), online Bayesian optimization (``Bayes Opt''), online evolutionary algorithm (``Evolutionary'') and the best design in the training dataset. Note that \primemethodname\ improves over the best in the dataset by 12\%, outperforms the best online optimization method by 4.4\%. The best accelerator configuration is highlighted in bold.}
    \label{table:results_edge_18}
    \resizebox{\textwidth}{!}{
    \begin{tabular}{l||l||l|l||l|l|l||l}
    \toprule
    &\multicolumn{3}{c||}{\textbf{\primemethodname}}& &\multicolumn{2}{c||}{\textbf{Online Optimization}}&\\\cline{2-4}\cline{6-7}
    \textbf{Applications}&\textbf{All}& \textbf{-Opt}&\textbf{-Infeasible}&\textbf{Standard}& \textbf{Bayes Opt}&\textbf{Evolutionary}&\textbf{$\mathcal{D}$ (Best in Training)}\\\midrule
    MobileNetEdgeTPU, Area $\leq$ 18~mm$^2$&\textbf{315.15}&433.81&351.22&470.09&331.05&329.13&354.13\\
    \bottomrule
    \end{tabular}
    }
    \end{table}
    %
%
We also test the robustness of our approach in handling variable constraints at test-time such as different chip area budget.
%
We evaluate the learned conservative surrogate trained via \primemethodname\ under a reduced value of the area threshold, $\alpha$, in Equation~\ref{eqn:opt_prob}.
%
To do so, we utilize a variant of rejection sampling -- we take the learned model trained for a default area constraint $\alpha = 29~\mathrm{mm}^2$ and then reject all optimized accelerator configurations which do not satisfy a reduces area constraint: $\mathrm{Area}(\rvx) \leq \alpha_0 = 18~\mathrm{mm}^2$.
%
Table~\ref{table:results_edge_18} summarizes the results for this scenario for the MobileNetEdgeTPU~\citep{edgetpu:arxiv:2020} application under the new area constraint ($\alpha = 18~\mathrm{mm}^2$).
%
A method that produces diverse designs which are both high-performing and are spread across diverse values of the area constraint are expected to perform better.
%
As shown in Table~\ref{table:results_edge_18}, \primemethodname\ provides better accelerator than the best online optimization from scratch with the new constraint value by 4.4\%, even when \primemethodname\ does not train its conservative surrogate with this unseen test-time design constraint.
%
Note that, when the design constraint changes, online methods generally need to restart the optimization process from scratch and undergo costly queries to the simulator.
%
This would impose additional overhead in terms of total simulation time (\S~Figure~\ref{fig:convergence_time} and Figure~\ref{fig:convergence_time_seven_models}). However, the results in Table~\ref{table:results_edge_18} shows that our learned surrogate model can be reused under different test-time design constraint eliminating additional queries to the simulator.
%



\subsection{Analysis of Designed Accelerators}
\label{app:analysis}
%
\begin{table}[t!]
\small
\renewcommand{\arraystretch}{1.2}
\centering
\caption{Per application latency for the best accelerator design suggested by \primemethodname\ and the Evolutionary method according to Table~\ref{table:results_multi_task} for multi-task accelerator design (nine applications and area constraint 100~mm$^2$). \primemethodname\ outperforms the Evolutionary method by 1.35$\times$.}
\label{table:per_application_latency}
\resizebox{0.9\textwidth}{!}{
\begin{tabular}{l|r|r|r}
\toprule
&\multicolumn{2}{c|}{\textbf{Latency (ms)}}&\\\cline{2-3}
\textbf{Applications}&\textbf{\primemethodname}&\textbf{Evolutionary~(Online)}&\textbf{Improvement of \primemethodname\ over Evolutionary}\\\midrule
{MobileNetEdgeTPU}&\textbf{288.38}&319.98&1.10$\times$\\\hline
{MobileNetV2}&\textbf{216.27}&255.95&1.18$\times$\\\hline
{MobileNetV3}&\textbf{487.46}&573.57&1.17$\times$\\\hline
{\mfour}&\textbf{400.88}&406.28&1.01$\times$\\\hline
{\mfive}&248.18&\textbf{239.18}&0.96$\times$\\\hline
{\msix}&164.98&\textbf{148.83}&0.90$\times$\\\hline
{U-Net}&1268.73&\textbf{908.86}&0.71$\times$\\\hline
{t-RNN Dec}&\textbf{191.83}&862.14&5.13$\times$\\\hline
{t-RNN Enc}&\textbf{185.41}&952.44&4.49$\times$\\\bottomrule
\CC \textbf{Average~(Latency in ms)}&\CC~\texttt{\textbf{383.57}}&\CC~\texttt{\textbf{518.58}}&\CC~\texttt{\textbf{1.35$\times$}}\\\bottomrule
\end{tabular}
}
\end{table}
\begin{table}[t!]
\small
\renewcommand{\arraystretch}{1.2}
\centering
\caption{Optimized accelerator configurations (See Table~\ref{tab:arch_params}) found by \primemethodname\ and the Evolutionary method for multi-task accelerator design (nine applications and area constraint 100~mm$^2$). Last row shows the accelerator area in mm$^2$. \primemethodname\ reduces the overall chip area usage by 1.97$\times$. The difference in the accelerator configurations are shaded in gray.}
\label{table:best_accelerator_config}
\resizebox{0.6\textwidth}{!}{
\begin{tabular}{l|r|r}
\toprule
&\multicolumn{2}{c}{\textbf{Parameter Value}}\\\cline{2-3}
\textbf{Accelerator Parameter}&\textbf{\primemethodname}&\textbf{Evolutionary~(Online)}\\\midrule
\# of PEs-X&4&4\\\hline
\CC{}\# of PEs-Y&\CC~6&\CC~8\\\hline
\CC{}\# of Cores&\CC~64&\CC~128\\\hline
\CC{}\# of Compute Lanes&\CC~4&\CC~6\\\hline
\CC{}PE Memory&\CC~2,097,152&\CC~1,048,576\\\hline
Core Memory&131,072&131,072\\\hline
\CC{}Instruction Memory&\CC~32,768&\CC~8,192\\\hline
Parameter Memory&4,096&4,096\\\hline
\CC{}Activation Memory&\CC~512&\CC~2,048\\\hline
DRAM Bandwidth~(Gbps)&30&30\\\bottomrule
\CC~\textbf{Chip Area~(mm$^2$)}&\CC~\texttt{\textbf{46.78}}&\CC~\texttt{\textbf{92.05}}\\\bottomrule
\end{tabular}
}
\end{table}
%
In this section, we overview the best accelerator configurations that \primemethodname\ and the Evolutionary method identified for multi-task accelerator design (See Table~\ref{table:results_multi_task}), when the number of target applications are nine and the area constraint is set to 100~mm$^2$.
%
The average latencies of the best accelerators found by \primemethodname\ and the Evolutionary method across nine target applications are \textbf{383.57~ms} and \textbf{518.58~ms}, respectively.
%
In this setting, our method outperforms the best online method by \textbf{1.35$\times$}.
%
Table~\ref{table:per_application_latency} shows per application latencies for the accelerator suggested by our method and the Evolutionary method.
%
The last column shows the latency improvement of \primemethodname\ over the Evolutionary method. Interestingly, while the latency of the accelerator found by our method for MobileNetEdgeTPU, MobileNetV2, MobileNetV3, \mfour, t-RNN Dec, and t-RNN Enc are better, the accelerator identified by the online method yields lower latency in \mfive, \msix, and U-Net.

%
To better understand the trade-off in design of each accelerator designed by our method and the Evolutionary method, we present all the accelerator parameters (See Table~\ref{tab:arch_params}) in Table~\ref{table:best_accelerator_config}.
%
The accelerator parameters that are different between each of the designed accelerator are shaded in gray (e.g. \# of PEs-Y, \# of Cores, \# of Compute Lanes, PE Memory, Instruction Memory, and Activation Memory).
%
Last row of Table~\ref{table:best_accelerator_config} depicts the overall chip area usage in mm$^2$. \primemethodname\ not only outperforms the Evolutionary algorithm in reducing the average latency across the set of target applications, but also reduces the overall chip area usage by \textbf{1.97$\times$}.
%
Studying the identified accelerator configuration, we observe that \primemethodname\ trade-offs compute (\colorbox{lightgray}{\textbf{64 cores vs. 128 cores}}) for larger PE memory size (\colorbox{lightgray}{\textbf{2,097,152 vs. 1,048,576}}). These results show that \primemethodname\ favors PE memory size to accommodate for the larger memory requirements in t-RNN Dec and t-RNN Enc (See Table~\ref{tab:models} Model Parameters) where large gains lie.
%
Favoring larger on-chip memory comes at the expense of lower compute power in the accelerator. This reduction in the accelerator's compute power leads to higher latency for the models with large number of compute operations, namely \mfive, \msix, and U-Net (See last row in Table~\ref{tab:models}).
%
\mfour is an interesting case where both compute power and on-chip memory is favored by the model (6.23~MB model parameters and 3,471,920,128 number of compute operations). This is the reason that the latency of this model on both accelerators, designed by our method and the Evolutionary method, are comparable (400.88~ms in \primemethodname\ vs. 406.28~ms in the online method).


\subsection{Comparison with Online Methods in Zero-Shot Setting}
\label{sec:app_zero_shot}
%
We evaluated the Evolutionary (online) method under two protocols for the last two rows of Table~\ref{table:zero_shot}: first, we picked the best designs (top-performing 256 designs similar to the \primemethodname\ setting in Section~\ref{sec:method}) found by the evolutionary algorithm on the training set of applications and evaluated them on the target applications and second, we let the evolutionary algorithm continue simulator-driven optimization on the target applications.
%
The latter is unfair, in that the online approach is allowed access to querying more designs in the simulator. Nevertheless, we found that in either configuration, the evolutionary approach performed worse than \primemethodname\, which does not access training data from the target application domain.
%
For the area constraint 29~mm$^2$ and 100~mm$^2$, the Evolutionary algorithm reduces the latency from 1127.64~$\rightarrow$~820.11 and 861.69~$\rightarrow$~552.64, respectively, although still worse than \primemethodname.
%
In the second experiment in which we \emph{unfairly} allow the evolutionary algorithm to continue optimizing on the target application, the Evolutionary algorithm suggests worse designs than Table~\ref{table:zero_shot} (e.g. 29~mm$^2$: 1127.64~$\rightarrow$~1181.66 and 100~mm$^2$: 861.69~$\rightarrow$~861.66).
%


\subsection{\primemethodname\ Ablation Study}
\label{sec:appx_ablation}



\begin{table}[t]
\captionsetup{font=small}
\vspace{5pt}
\centering
\caption{\label{table:appx_results_single_task}{Optimized objective values (i.e., latency in milliseconds) obtained by various methods for the task of learning accelerators specialized to a given application. Lower latency/runtime is better. From left to right: our method, our method without negative sampling (``\methodname-$\mathrm{Opt}$'') and without utilizing infeasible points (``\methodname-Infeasible''), standard surrogate (``Standard''), online Bayesian optimization (``Bayes Opt''), online evolutionary algorithm (``Evolutionary'') and the best design in the training dataset. Note that, in all the applications \methodname\ improves over the best in the dataset, outperforms online optimization methods in 7/9 applications and the complete version of \methodname\ generally performs best. The best accelerator designs are in bold.}}
\renewcommand{\arraystretch}{1.2}
\fontsize{7}{7}\selectfont
\begin{tabular}{l||r|r|r||r||r|r||r}
\hline
&\multicolumn{3}{c||}{\textbf{\methodname}}&&\multicolumn{2}{c||}{\textbf{Online Optimization}}&\\\cline{2-4}\cline{6-7}
\textbf{Application}&\textbf{All}& \textbf{-Opt}&\textbf{-Infeasible}&\textbf{Standard}&\textbf{Bayes Opt}&\textbf{Evolutionary}&\textbf{$\mathcal{D}$ (Best in Training)}\\\midrule
MobileNetEdge&\textbf{298.50}&435.40&322.20&411.12&319.00&320.28&354.13\\\hline
MobileNetV2&\textbf{207.43}&281.01&214.71&374.52&240.56&238.58&410.83\\\hline
MobileNetV3&\textbf{454.30}&489.45&483.96&575.75&534.15&501.27&938.41\\\hline
\mfour&\textbf{370.45}&478.32&432.78&1139.76&396.36&383.58&779.98\\\hline
\mfive&208.21&319.61&246.80&307.57&201.59&\textbf{198.86}&449.38\\\hline
\msix&131.46&197.70&162.12&180.24&121.83&\textbf{120.49}&369.85\\\hline
U-Net&\textbf{740.27}&740.27&765.59&763.10&872.23&791.64&1333.18\\\hline
t-RNN Dec&\textbf{132.88}&172.06&135.47&136.20&771.11&770.93&890.22\\\hline
t-RNN Enc&\textbf{130.67}&134.84&137.28&150.21&865.07&865.07&584.70\\\bottomrule
\end{tabular}
\normalsize
\vspace{-16pt}
\end{table}

Here we ablate over variants of our method: (1) $\mathrm{Opt}$ was not used for negative sampling (``\primemethodname-$\mathrm{Opt}$'' in Table~\ref{table:appx_results_single_task}) (2)  infeasible points were not used (``\primemethodname-Infeasible'' in Table~\ref{table:appx_results_single_task}).
%
As shown in Table~\ref{table:appx_results_single_task}, the variants of our method generally performs worse compared to the case when both negative sampling and infeasible data points are utilized in training the surrogate model.

\subsection{{Comparison with Human-Engineered Accelerators}}
\label{app:human_engineered}
%
{In this section, we compare the optimized accelerator design found by \primemethodname\ that is targeted towards single applications to the manually optimized EdgeTPU design~\citep{yazdanbakhsh2021evaluation,edgetpu:arxiv:2020}.
%
EdgeTPU accelerators are primarily optimized towards running applications in image classification, particularly, MobileNetV2,  MobileNetV3 and MobileNetEdgeTPU.
%
The goal of this comparison is to present the potential benefit of \primemethodname for a dedicated application when compared to human designs.
%
For this comparison, we utilize an area constraint of 27~mm$^2$ and a DRAM bandwidth of 25 Gbps, to match the specifications of the EdgeTPU accelerator.}
%

{Table~\ref{table:edgetpu_comparison} shows the summary of results in two sections, namely ``Latency'' and ``Chip Area''. The first and second under each section show the results for \primemethodname\ and EdgeTPU, respectively.
%
The final column for each section shows the improvement of the design suggested by \primemethodname\ over EdgeTPU.
%
On average (as shown in the last row), \primemethodname\ finds accelerator designs that are 2.69$\times$ (up to 11.84$\times$ in t-RNN Enc) better than EdgeTPU in terms of latency. Our method achieves this improvement while, on average, reducing the chip area usage by 1.50$\times$ (up to 2.28$\times$ in MobileNetV3). Even on the MobileNet image-classification domains, we attain an average improvement of 1.85$\times$.}

\begin{table}[t!]
\small
\vspace{0.1in}
\renewcommand{\arraystretch}{1.2}
\centering
\caption{\review{The comparison between the accelerator designs suggested by \primemethodname\ and EdgeTPU~\citep{yazdanbakhsh2021evaluation,edgetpu:arxiv:2020} for single model specialization. On average (last row), with single-model specialization our method reduces the latency by 2.69$\times$ while minimizes the chip area usage by 1.50$\times$.}}
\label{table:edgetpu_comparison}
\resizebox{\textwidth}{!}{
\begin{tabular}{l|r|r|r||r|r|r}
\toprule
&\multicolumn{3}{c||}{\textbf{Latency~(milliseconds)}}&\multicolumn{3}{c}{\textbf{Chip Area~(mm$^2$)}}\\\cline{2-7}
\textbf{Application}&\textbf{\primemethodname}&\textbf{EdgeTPU}&\textbf{Improvement}&\textbf{\primemethodname}&\textbf{EdgeTPU}&\textbf{Improvement}\\\midrule
MobileNetEdgeTPU&294.34&523.48&1.78$\times$&18.03&27&1.50$\times$\\\hline
MobileNetV2&208.72&408.24&1.96$\times$&17.11&27&1.58$\times$\\\hline
MobileNetV3&459.59&831.80&1.81$\times$&11.86&27&2.28$\times$\\\hline
\mfour&370.45&675.53&1.82$\times$&19.12&27&1.41$\times$\\\hline
\mfive&208.42&377.32&1.81$\times$&22.84&27&1.18$\times$\\\hline
\msix&132.98&234.88&1.77$\times$&16.93&27&1.59$\times$\\\hline
U-Net&1465.70&2409.73&1.64$\times$&25.27&27&1.07$\times$\\\hline
t-RNN Dec&132.43&1384.44&10.45$\times$&14.82&27&1.82$\times$\\\hline
t-RNN Enc&130.45&1545.07&11.84$\times$&19.87&27&1.36$\times$\\\bottomrule
\CC~\textbf{Average Improvement}&\CC~---&\CC~---&\CC~\textbf{2.69$\times$}&\CC~---&\CC~---&\CC~\textbf{1.50$\times$}\\\bottomrule
\end{tabular}
}
\end{table}
%



\subsection{{Zero-Shot Results on All Applications}}
\review{In this section, we present the results of zero-shot optimization from Table~\ref{table:zero_shot} on all the nine applications we study in the paper (i.e., test applications = all nine models: MobileNet (EdgeTPU, V2, V3), \msix, \mfive, \mfour, t-RNN (Enc and Dec), and U-Net). We investigate this for two sets of training applications and two different area budgets. As shown in Table~\ref{table:zero_shot_all}, we find that \primemethodname\ does perform well compared to the online evolutionary method.}

\begin{table*}[t]
\small
\centering
\vspace*{0.0cm}
\caption{\label{table:zero_shot_all}\review{Optimized objective values (i.e., latency in milliseconds) under zero-shot setting when the test applications include all the nine evaluated models (e.g. MobileNet (EdgeTPU, V2, V3), \mfour, \mfive, \msix, t-RNN Dec, t-RNN Enc, U-Net). Lower latency is better. From \textbf{left} to \textbf{right}: the applications used to train the surrogate model in \primemethodname\, the target applications for which the accelerator is optimized for, the area constraint of the accelerator, \primemethodname's (best, median) latency, and best online method's (best, median) latency. The best accelerator configurations identified is highlighted in bold.}}
\vspace{-0.1cm}
\resizebox{\textwidth}{!}{% <------ Don't forget this %
\begin{tabular}{l|l|l|l}
\toprule
\textbf{Train Applications}&\textbf{Area}&\textbf{\primemethodname}&\textbf{Evolutionary~(Online)}\\\midrule
% Train = MobilenetEdge, MobilenetV3, Target = MobilenetV2
% MobileNet~(Edge, V3)&{MobileNetV2}&29~mm$^2$&(\textbf{311.39}, \textbf{313.76})&(314.31, 316.65)\\\hline
% MobileNet~(V2, V3), \mfive, \msix&{MobileNetEdge, \mfour}&29~mm$^2$&(357.05, 364.92)&(\textbf{354.59}, \textbf{357.29})\\\hline
% Area 29
MobileNet~(EdgeTPU,V2,V3), \mfour, \mfive, \msix, t-RNN Enc&29~mm$^2$&(\textbf{426.65}, \textbf{427.94})&(586.55, 586.55)\\\hline
% Area 100
MobileNet~(EdgeTPU,V2,V3), \mfour, \mfive, \msix, t-RNN Enc&100~mm$^2$&(\textbf{365.95}, \textbf{366.64})&(518.58, 519.37)\\\bottomrule
\CC \textbf{Geomean of \primemethodname's Improvement}&\CC--- &\CC~(1.0$\times$, 1.0$\times$)&\CC~(\texttt{\textbf{1.40$\times$}}, \texttt{\textbf{1.39$\times$}})\\
\bottomrule
\end{tabular}% <------ Don't forget this %
}
% \vspace{-0.2cm}
\end{table*}


\subsection{{Different Train and Validation Splits}}
{In the main paper, we used the worst 80\% of the feasible points in the training dataset for training and used the remaining 20\% of the points for cross-validation using our strategy based on Kendall's rank correlation. In this section, we explore some alternative training-validation split strategies to see how they impact the results. To do so, we consider two alternative strategies: \textbf{(1)} training on 95\% of the worst designs, validation on top 5\% of the designs, and \textbf{(2)} training on the top 80\% of the designs and validation on the worst 20\% of the designs. We apply these strategies to MobileNetEdgeTPU, \msix and t-RNN Enc models from Table~\ref{table:results_single_task}, and present a comparative evaluation in Table~\ref{table:train_val_split} below.}

{\textbf{Results.} As shown in Table~\ref{table:train_val_split}, we find that cross-validating using the best 5\% of the points in the dataset led to a reduced latency (298.50 $\rightarrow$ 273.30) on MobileNetEdgeTPU, and retained the same performance on \msix. However, it increased the latency on t-RNN Enc (130.67 $\rightarrow$ 137.45). This indicates at the possibility that while top 5\% of the datapoints can provide a better signal for cross-validation in some cases, this might also hurt performance if the size of the 5\% dataset becomes extremely small (as in the case of t-RNN Enc, the total dataset size is much smaller than either MobileNetEdgeTPU or \msix).}

{The strategy of cross-validating using the worst 20\% of the points hurt performance on \msix and t-RNN Enc, which is perhaps as expected, since the worst 20\% of the points may not be indicative of the best points found during optimization. However, while it improves performance on the MobileNetEdgeTPU application compared to the split used in the main paper but it is still worse than using the top 5\% of the points for validation.}
%
\begin{table}[H]
\small
\centering
% \vspace*{0.1cm}
\caption{\review{Performance of \primemethodname\ (as measured by median latency of the accelerator found across five runs) under various train-test splits on three applications studied in Table~\ref{table:results_single_task}.}}
\label{table:train_val_split}
\resizebox{0.99\textwidth}{!}{
\begin{tabular}{l|c|c|c}
\toprule
\textbf{Applications}&\textbf{Best 5\% Validation}&\textbf{Best 20\% Validation (Table~\ref{table:results_single_task})}&\textbf{Worst 20\% Validation}\\\midrule
MobileNetEdgeTPU&273.30&298.50&286.53\\\hline
\msix&131.46&131.46&142.68\\\hline
t-RNN Enc&137.45&130.67&135.71\\\bottomrule
\end{tabular}
}
\vspace{-0.1cm}
\end{table}
%


\section{Details of \primemethodname}
\label{app:details}
%
In this section, we provide training details of our method \primemethodname\ including hyperparameters and compute requirements and details of different tasks. 
%
\subsection{Hyperparameter and Training Details}
\label{sec:practical_implementation}
%
Algorithm~\ref{alg:prime} outlines our overall system for accelerator design.
%
\primemethodname\ parameterizes the function $f_\theta(\rvx)$ as a deep neural network as shown in Figure~\ref{fig:method}. 
%
The architecture of $f_\theta(\rvx)$ first embeds the discrete-valued accelerator configuration $\rvx$ into a continuous-valued 640-dimensional embedding via two layers of a self-attention transformer~\citep{vaswani2017attention}. 
%
Rather than directly converting this 640-dimensional embedding into a scalar output via a simple feed-forward network, which we found a bit unstable to train with Equation~\ref{eqn:final_training}, possibly due to the presence of competing objectives for a comparison), we pass the 640-dimensional embedding into $M$ different networks that map it to $M$ different scalar predictions $(f^i_\theta(\rvx))_{i=1}^M$. 
%
Finally, akin to attention~\citep{vaswani2017attention} and mixture of experts~\citep{shazeer2017outrageously}, we train an additional head to predict weights $(w_i)_{i=1}^M \geq 0$ of a linear combination of the predictions at different heads that would be equal to the final prediction: $f_\theta(\rvx) = \sum_{i=1}^K w_i f^i_\theta(\rvx)$.
%
Such an architecture allows the model to use different predictions $f^i_\theta(\rvx)$, depending upon the input, which allows for more stable training. To train $f_\theta(\rvx)$, we utilize the Adam~\citep{kingma2014adam} optimizer.
%
Equation~\ref{eqn:final_training} utilizes a procedure $\mathrm{Opt}$ that maximizes the learned function approximately. We utilize the same technique as Section~\ref{sec:method} (``optimizing the learned surrogate'') to obtain these negative samples.
%
We periodically refresh $\mathrm{Opt}$, once in every 20K gradient steps on $f_\theta(\rvx)$ over training.
\begin{algorithm}[ht] %b
  \caption{Training the conservative surrogate in PRIME}\label{alg:prime}
  \begin{algorithmic}[1]
    \State Initialize a neural model $f_{\theta_0}(\rvx)$ and a set $M = 23$ of negative particles to be updated by the firefly optimizer $\{\rvx^{-}_1(0), \cdots,  \rvx^{-}_i(0), \rvx^{-}_M(0)\}$ to random configurations from the design space.
    \For{iteration $i = 0, 1, 2, 3, \dots $ until convergence}
        \For{firefly update step  $t = 0, 1, \dots,  4$  \hfill \Comment{\textbf{Inner loop}}}
          \State Update the $M$ fireflies according to the firefly update rule in Equation~\ref{eqn:firefly1},\\ ~~~~~~~~~~~~towards maximizing $f_{\theta_i}(\rvx)$ according to: \hfill \Comment{\textbf{Negative mining}}\\ ~~~~~~~~~~~~~~~~~~~~~~~~~~~~~~~~${\rvx_{i}(ti+ 1) = \rvx_{i}(ti) + \beta(\rvx_{i}(ti) - \rvx_{j}(ti)) + {\eta}\epsilon_{ti}}$
      \EndFor
      \State Find the best firefly found in these steps to be used as the negative sample:\\ ~~~~~~~~~~~~~~~~~~~~~~$\rvx^{-}_i = \arg \min \{f_{\theta_i}(\rvx^{-}_1(t i)), \cdots, f_{\theta_i}(\rvx^{-}_M(t i))\}$ \hfill \Comment{\textbf{Find negative sample}}
      \State Run one gradient step on $\theta_i$ using Equation~\ref{eqn:final_training} with $\rvx^{-}_i$ as the negative sample
      \State \textbf{if} $i \% p == 0$, (p = 20000), then: \hfill \Comment{\textbf{Periodically reinitialize the optimizer}} \\~~~~~~~~~~~~~~~Reinitialize firefly particles  $\{\rvx^{-}_1(0), \cdots,  \rvx^{-}_i(0), \rvx^{-}_M(0)\}$ to random designs.
    \EndFor
    \State Return the final model $f_{\theta^*}(\rvx)$
  \end{algorithmic}
\end{algorithm}

The hyperparameters for training the conservative surrogate in Equations~\ref{eqn:final_training} and its contextual version are as follows:
\begin{itemize}
    \item \textbf{Architecture of ${f}_\theta(\rvx)$}. As indicated in Figure~\ref{fig:method}, our architecture takes in list of categorical (one-hot) values of different accelerator parameters (listed in Table~\ref{tab:arch_params}), converts each parameter into $64$-dimensional embedding, thus obtaining a $10 \times 64$ sized matrix for each accelerator, and then runs two layers of self-attention~\citep{vaswani2017attention} on it. The resulting $10 \times 64$ output is flattened to a vector in $\mathbb{R}^{640}$ and fed into $M = 7$ different prediction networks that give rise to $f^1_\theta(\rvx), \cdots, f^M_\theta(\rvx)$, and an additional attention 2-layer feed-forward network (layer sizes $=[256, 256]$) that determines weights $w_1, \cdots, w_M$, such that $w_i \geq 0$ and $\sum_{i=1}^M w_i = 1$. Finally the output is simply $f_\theta(\rvx) = \sum_i w_i f^i_\theta(\rvx)$. 
    \item \textbf{Optimizer/learning rate for training $f_\theta(\rvx)$}. Adam, $1e-4$, default $\beta_1 = 0.9$, $\beta_2 = 0.999$.
    \item \textbf{Validation set split.} Top 20\% high scoring points in the training dataset are used to provide a validation set for deciding coefficients $\alpha$, $\beta$ and the checkpoint to evaluate.
    \item \textbf{Ranges of $\alpha$, $\beta$.} We trained several $f_\theta(\rvx)$ models with $\alpha \in [0.0, 0.01, 0.1, 0.5, 1.0, 5.0]$ and $\beta \in [0.0, 0.01, 5.0, 0.1, 1.0]$. Then we selected the best values of $\alpha$ and $\beta$ based on the highest Kendall's ranking correlation on the validation set. Kendall's ranking correlation between two sets of objective values: $S = \{y_1, y_2, \cdots, y_N\}$ corresponding to ground truth latency values on the validation set and $S' = \{y'_1, y'_2, \cdots, y'_N \}$ corresponding to the predicted latency values on the validation set is given by $\tau$ equal to:
    \begin{equation}
        \tau = \frac{\sum_{i, j}^{N, N} \mathbb{I}[(y_i - y_j) (y'_i - y'_j) > 0] - \sum_{i, j}^{N, N} \mathbb{I}[(y_i - y_j) (y'_i - y'_j) \leq 0]}{N \cdot (N - 1)}.   
    \end{equation}
    \item \textbf{Clipping $f_\theta(\rvx)$ during training}. Equation~\ref{eqn:final_training} increases the value of the learned function $f_\theta(\rvx)$ at $\rvx = \rvx_0 \in \mathcal{D}_\text{infeasible}$ and $\rvx^- \sim \mathrm{Opt}(f_\theta)$. We found that with the small dataset, these linear objectives can run into numerical instability, and produce $+\infty$ predictions. To avoid this, we clip the predicted function value both above and below by $\pm 10000.0$, where the valid range of ground-truth values is $\mathcal{O}(1000)$.
    \item \textbf{Negative sampling with $\mathrm{Opt}(\cdot)$}. As discussed in Section~\ref{sec:method}, we utilize the firefly optimizer for both the negative sampling step and the final optimization of the learned conservative surrogate. When used during negative sampling, we refresh (i.e., reinitialize) the firefly parameters after every $p = 20000$ gradient steps of training the conservative surrogate, and run $t = 5$ steps of firefly optimization per gradient step taken on the conservative surrogate. 
    \item \textbf{Details of firefly:} The initial population of fireflies depends on the number of accelerator configurations ($\mathcal{C}$) following the formula $10+\round{(\mathcal{C}^{1.2}+\mathcal{C})\times0.5}$. In our setting with ten accelerator parameters (See Table~\ref{tab:arch_params}), the initial population of fireflies is 23. We use the same hyperparameters: $\gamma = 1.0, \beta_0 = 1.0$, for the optimizer in all the experiments and never modify it. The update to a particular optimization particle (i.e., a firefly) $\rvx_i$, at the $t$-th step of optimization is given by: 
    \begin{equation}
    \label{eqn:firefly1}
    \begin{aligned}
    {\rvx_{i}(t+1) = \rvx_{i}(t) + \beta(\rvx_{i}(t) - \rvx_{j}(t)) + ~\text{i.i.d. Gaussian noise},}
    \end{aligned}
    \end{equation}
    where $\rvx_j(t), j \neq i$ is a different firefly that achieves a better objective value compared to $\rvx_i$ and the function $\beta$ is given by: $\beta(r) = \beta_{0}e^{-\gamma{}r^2}$.
    \item \review{\textbf{Training set details:} The training dataset sizes for the studied applications are shown in Table~\ref{table:dataset_size}. To recap, to generate the dataset, we first randomly sampled accelerators from the deign space, and evaluated them for the target application, and constituted the training set from the worst-performing feasible accelerators for the given application. Since different applications admit different feasibility criteria (differences in compilation, hardware realization, and etc.), the dataset sizes for each application are different, as the number of feasible points is different. Note however that as mentioned in the main text, these datasets all contain $\leq 8000$ feasible points.}
    
    \review{\textbf{Discussion on data quality:} In the cases of t-RNN Dec, t-RNN Enc, and U-Net, we find that the number of feasible points is much smaller compared to other applications, and we suspect this is because our random sampling procedure does not find enough feasible points. This is a limitation of our data collection strategy and we intentionally chose this na\"ive strategy to keep data collection simple. Other techniques for improving data collection and making sure that the data does not consist of only infeasible points includes strategies such as utilizing logged data from past runs of online evolutionary methods, mixed with some data collected via random sampling to improve coverage of the design space.} 
\end{itemize}

\begin{table}[t]
    \small
    \centering
    \vspace*{0.1cm}
    \caption{\review{Dataset sizes for various applications that we study in this paper. Observe that all of the datasets are smaller than $8000$.}}
    \label{table:dataset_size}
    \resizebox{0.5\textwidth}{!}{
    \begin{tabular}{l||r}
    \toprule
    \textbf{Application}& \textbf{Dataset size}\\\midrule
    MobileNetEdgeTPU & 7697 \\\hline
    MobileNetV2 & 7620 \\ \hline
    MobileNetV3 & 5687 \\ \hline
    \mfour & 3763 \\ \hline
    \mfive & 5735 \\ \hline
    \msix & 7529 \\ \hline
    U-Net & 557\\\hline
    t-RNN Dec & 1211\\\hline
    t-RNN Enc & 1240\\\bottomrule
    \end{tabular}
    }
    \vspace{-0.1cm}
\end{table}

\subsubsection{\review{Details of Firefly Used for Our Online Evolutionary Method}}
\label{app:online_opt_details}

\review{In this section, we discuss some details for firefly optimization used in the online evolutionary method.}

{\textbf{Stopping criterion:} We stopped the firefly optimization when the latency of the best design found did not improve over the previous 1000 iterations, but we also made sure to run firefly optimization for at least 8000 iterations, to make sure that both the online and offline methods match in terms of the data budget. We also provide the convergence curves for firefly optimization on various single-application problems from Table~\ref{table:results_single_task} in Figure~\ref{fig:convergence_curves}.}

{\textbf{What happens if we run firefly optimization for longer?} We also experimented with running the evolutionary methods for longer (i.e., 32k simulator accesses compared to 8k), to check if this improves the performance of the evolutionary approach. As shown in Table~\ref{table:online_budget}, we find that while this procedure does improve performance in some cases, the performance does not improve much beyond 8k steps. This indicates that there is a possibility that online methods can perform better than \primemethodname\ if they are run for many more optimization iterations against the simulator, but they may not be as data-efficient as \primemethodname.}

\begin{table*}[t!]
\small
\centering
\vspace*{0.0cm}
\caption{\label{table:online_budget}\review{Comparing the latency of the accelerators designed by the evolutionary approach for variable number of simulator access budgets (8k and 32k). Even with $4\times$ as much allowed simulator interaction, online methods are unable to perform that well in our case.}}
\resizebox{0.95\textwidth}{!}{% <------ Don't forget this %
\begin{tabular}{l|l|l|l|l}
\toprule
&&&\multicolumn{2}{c}{\textbf{Evolutionary~(Online)}}\\\cline{4-5}
\textbf{Application}&\textbf{Area}&\textbf{\primemethodname}&\textbf{8k data points}&\textbf{32k data points}\\\midrule
MobileNetEdgeTPU&29~mm$^2$&298.50&320.28&311.35\\\hline
t-RNN Dec&29~mm$^2$&132.88&770.93&770.63\\\hline
t-RNN Enc&29~mm$^2$&130.67&865.07&865.07\\\bottomrule
\CC \textbf{Geomean of \primemethodname's Improvement}&\CC---&\CC~(1.0$\times$, 1.0$\times$)&\CC~\texttt{\textbf{3.45$\times$}}&\CC~\texttt{\textbf{3.42$\times$}}\\
\bottomrule
\end{tabular}% <------ Don't forget this %
}
\vspace{-0.3cm}
\end{table*}
%

{\textbf{Hyperparameter tuning for firefly:} Since the online optimization algorithms we run have access to querying the simulator over the course of training, we can simply utilize the value of the latest proposed design as a way to perform early stopping and hyperparameter tuning. A na\"ive way to perform hyperparameter tuning for such evolutionary methods is to run the algorithm for multiple rounds with multiple hyperparameters, however this is compute and time intensive. Therefore, we adopted a dynamic hyperparameter tuning strategy. Our implementation of the firefly optimizer tunes hyperparameters by scoring a set of hyperparameters based on its best performance over a sliding window of $T$ data points. This allows us to adapt to the best hyperparameters on the fly, within the course of optimization, effectively balancing the number of runs that need to be run in the simulator and hyperparameter tuning.}
\review{This dynamic hyperparameter tuning strategy requires some initial coverage of the hyperparameter space before hyperparameter tuning begins, and therefore, this tuning begins only after $750$ datapoints. 
After this initial phase, every $T = 50$ iterations, the parameters $\gamma$ and $\beta_0$ are updated via an evolutionary scoring strategy towards their best value.}

{\textbf{Discussion of t-RNN Enc and t-RNN Dec}. Finally, we discuss the results of the evolutionary approach on the t-RNN Enc and t-RNN Dec tasks, for which the convergence plots are shown in Figures~\ref{fig:conv_rnn_dec} and \ref{fig:conv_enc}. Observe that the best solution found by this optimization procedure converges quite quickly in this case (with about 1000 iterations) and the evolutionary method, despite the dynamic hyperparameter tuning is unable to find a better solution. We hypothesize that this is because the performance of a local optimization method may suffer heavily due to the poor landscape of the objective function, and it may get stuck if it continuously observes only infeasible points over the course of optimization.}

\begin{figure}[ht]
\centering
\subfloat[MobileNetEdgeTPU]{\label{fig:conv_edge}\includegraphics[width=0.32\linewidth]{chapters/prime/figs/convergence/mobilenetedge.pdf}}
\subfloat[MobileNetV2]{\label{fig:conv_v2}\includegraphics[width=0.32\linewidth]{chapters/prime/figs/convergence/mobilenetv2.pdf}}
\subfloat[MobileNetV3]{\label{fig:conv_v3}\includegraphics[width=0.32\linewidth]{chapters/prime/figs/convergence/mobilenetv3.pdf}}\\
%
\subfloat[\mfour]{\label{fig:conv_m4}\includegraphics[width=0.32\linewidth]{chapters/prime/figs/convergence/m4.pdf}}
\subfloat[\mfive]{\label{fig:conv_m5}\includegraphics[width=0.32\linewidth]{chapters/prime/figs/convergence/m5.pdf}}
\subfloat[\msix]{\label{fig:conv_m6}\includegraphics[width=0.32\linewidth]{chapters/prime/figs/convergence/m6.pdf}}\\
%
\subfloat[U-Net]{\label{fig:conv_unet}\includegraphics[width=0.32\linewidth]{chapters/prime/figs/convergence/unet.pdf}}
\subfloat[t-RNN Dec]{\label{fig:conv_rnn_dec}\includegraphics[width=0.32\linewidth]{chapters/prime/figs/convergence/rnn-dec.pdf}}
\subfloat[t-RNN Enc]{\label{fig:conv_enc}\includegraphics[width=0.32\linewidth]{chapters/prime/figs/convergence/rnn-enc.pdf}}
\caption{\review{Optimization behavior of Firefly optimizer (Online Evolutionary). Observe that the optimization procedure converges and plateaus very quickly (at least 1000 iterations in advance) and hence we stop at 8000 iterations. In the case of t-RNN Enc and t-RNN Dec, we find that the evolutionary algorithm performs poorly and we suspect this is because it saturates quite quickly to a suboptimal solution and is unable to escape. This is also evident from Figures~\ref{fig:conv_rnn_dec} and \ref{fig:conv_enc}, where we observe that online optimization plateaus the fastest for these RNN applications. }}
\label{fig:convergence_curves} 
% \vspace{-0.5cm}
\end{figure}


\subsubsection{{Exact Hyperparameters Found By Our Cross-Validation Strategy}}

{In this section, we present the exact hyperparameters found by our cross-validation strategy discussed in Section~\ref{sec:method}. To recap, our offline cross-validation strategy finds the early stopping checkpoint and selects the values of $\alpha$ and $\beta$ in Equation~\ref{eqn:final_training} that attain the highest rank correlation on a held-out validation set consisting of top 20\% of the dataset feasible samples. The values of $\alpha$, $\beta$ and checkpoint selected for the experiments in Table 3, Table 4, Table 5 and 6 are shown in Table~\ref{table:hparams}.} 


\begin{table*}[t!]
\small
\centering
\renewcommand{\arraystretch}{1.1}
\vspace*{-0.05cm}
\caption{\label{table:hparams} \footnotesize \review{Hyperparameters $\alpha$, $\beta$ and checkpoint index (measured in terms of gradient steps on the learned conservative model) for \primemethodname\ found by our \textbf{offline} cross-validation strategy discussed in Section~\ref{sec:method}, that is based on the Kendall's rank correlation on the validation set (note that no simulator queries were used to tune hyperparameters). In the case of the multi-task and zero-shot scenarios, when training on more than one application, the batch size used for training \primemethodname\ increases to $N$-fold, where $N$ is the number of applications in the training set, therefore we likely find that even a few gradient steps are good enough.}}
\vspace{-0.1cm}
\resizebox{1.0\textwidth}{!}{\begin{tabular}{l|l|l|l|l}
\toprule
\textbf{Table}&\textbf{Application}&\textbf{$\alpha$}&\textbf{$\beta$}&\textbf{Checkpoint Index}\\\midrule
Table~\ref{table:results_single_task} & MobileNetEdgeTPU & 0.01 & 5.0 & 80000 \\\hline
Table~\ref{table:results_single_task} & MobileNetV2 & 5.0 & 5.0 & 120000 \\\hline
Table~\ref{table:results_single_task} & MobileNetV3 & 5.0 & 0.01 & 80000 \\\hline
Table~\ref{table:results_single_task} & \mfour & 0.1 & 0.0 & 80000 \\\hline
Table~\ref{table:results_single_task} & \mfive & 5.0 & 1.0 & 80000 \\\hline
Table~\ref{table:results_single_task} & \msix & 1.0 & 1.0 & 60000 \\\hline
Table~\ref{table:results_single_task} & U-Net & 0.0 & 1.0 & 100000 \\\hline
Table~\ref{table:results_single_task} & t-RNN Dec & 1.0 & 0.0 & 60000 \\\hline
Table~\ref{table:results_single_task} & t-RNN Enc & 0.0 & 0.1 & 60000 \\\midrule \hline
Table~\ref{table:results_multi_task} & MobileNet (EdgeTPU, V2, V3) & 5.0 & 0.01  & 60000 \\\hline
Table~\ref{table:results_multi_task} & MobileNet (V2, V3), \mfive, \msix & 0.0 & 5.0 & 30000 \\\hline
Table~\ref{table:results_multi_task} & MobileNet (EdgeTPU, V2, V3), \mfour, \mfive, \msix & 0.5 & 0.0 & 100000 \\\hline
Table~\ref{table:results_multi_task} & MobileNet (EdgeTPU, V2, V3), \mfour, \mfive, \msix, t-RNN Enc (Area 29.0) & 0.0 & 1.0  & 20000 \\\hline
Table~\ref{table:results_multi_task} & MobileNet (EdgeTPU, V2, V3), \mfour, \mfive, \msix, t-RNN Enc (Area 100.0) & 0.0 & 0.0 & 20000 \\\hline
Table~\ref{table:results_multi_task} & MobileNet (EdgeTPU, V2, V3), \mfour, \mfive, \msix, U-Net, t-RNN (Enc, Dec) (Area 29.0) &  0.01 & 0.01 & 10000 \\\hline
Table~\ref{table:results_multi_task} & MobileNet (EdgeTPU, V2, V3), \mfour, \mfive, \msix, U-Net, t-RNN (Enc, Dec) (Area 100.0) & 0.01 & 0.1  & 20000  \\\midrule\hline
Table~\ref{table:zero_shot} & Train (Zero-Shot): MobileNet (EdgeTPU, V3) & 5.0 & 0.01 & 60000\\\hline
Table~\ref{table:zero_shot} & Train (Zero-Shot): MobileNet (V2, V3), \mfive, \msix &  0.0 & 5.0 & 30000  \\\hline
Table~\ref{table:zero_shot} & Train (Zero-Shot): MobileNet (EdgeTPU, V2, V3), \mfour, \mfive, \msix, t-RNN Enc (Area 29.0) & 0.0 & 1.0 & 20000 \\\hline
Table~\ref{table:zero_shot} & Train (Zero-Shot): MobileNet (EdgeTPU, V2, V3), \mfour, \mfive, \msix, t-RNN Enc (Area 100.0) & 0.1 & 5.0 & 20000 \\ \midrule\hline
Table~\ref{table:dla_shi} & MobileNetV2 (NVDLA) & 0.0 & 1.0 & 40000 \\\hline
Table~\ref{table:dla_shi} & MobileNetV2 (ShinDianNao) & 0.0 & 0.0 & 40000 \\\hline
Table~\ref{table:dla_shi} & ResNet 50 (NVDLA) & 0.01 & 0.0 & 40000 \\\hline
Table~\ref{table:dla_shi} & ResNet 50 (ShinDianNao) & 0.0 & 0.0 & 75000 \\\hline
Table~\ref{table:dla_shi} & Transformer (NVDLA) & 0.01 & 1.0 & 200000 \\\hline
Table~\ref{table:dla_shi} & Transformer (ShinDianNao) & 0.0 & 0.1 & 100000 \\
\bottomrule
% \CC \textbf{Geomean of \primemethodname's Improvement}&\CC---&\CC~\texttt{\textbf{1.0$\times$}}&\CC~\texttt{\textbf{1.06$\times$}}&\CC~\texttt{\textbf{3.75$\times$}}\\
% \bottomrule
\end{tabular}}
\vspace{-0.1cm}
\end{table*}

%
\subsection{Details of Architecting Accelerators for Multiple Applications Simultaneously}
%
\label{sec:appx_multi_task_tsne}
Now we will provide details of the tasks from Table~\ref{table:results_multi_task} where the goal is to architect an accelerator which is jointly optimized for multiple application models.
%
For such tasks, we augment data-points for each model with the context vector $c_k$ from Table~\ref{tab:models} that summarizes certain parameters for each application.
%
For entries in this context vector that have extremely high magnitudes (e.g., model parameters and number of compute operations), we normalize the values by the sum of values across the applications considered to only encode the relative scale, and not the absolute value which is not required.
%
To better visualize the number of feasible accelerators for joint optimization, Figure~\ref{fig:tsne_overlap} show the tSNE plot (raw architecture configurations are used as input) of high-performing accelerator configurations. The blue-colored dots are the jointly feasible accelerators in the combined dataset, and note that these data points are no more than 20-30 in total. The highlighted red star presents the best design suggested by \primemethodname\ with average latency of 334.70 (Table~\ref{table:results_multi_task}).
%
This indicates that this contextual, multi-application problem poses a challenge for data-driven methods: these methods need to produce optimized designs even though very few accelerators are jointly feasible in the combined dataset.
% 
Despite this limitation, \primemethodname\ successfully finds more efficient accelerator configurations that attain low latency values on each of the applications jointly, as shown in Table~\ref{table:results_multi_task}.

\begin{figure}[t]
    \centering
    \includegraphics[width=0.7\textwidth]{chapters/prime/figs/tsne/tsne-overlap.pdf}
    \caption{tSNE plot of the joint dataset and randomly sampled infeasible data points. The blue points show the accelerator configurations that are jointly feasible for all the applications. The highlighted point with red star shows the best design proposed by \primemethodname\. The rest of the points show the infeasible points.}
    \label{fig:tsne_overlap}
    \vspace{-0.1in}
\end{figure}
%


\subsection{Dataset Sensitivity to Accelerator Parameters}
\label{sec:app_ds_sensitivity}
%
We visualize the sensitivity of the objective function (e.g. latency) with respect to the changes in certain accelerator parameters, such as memory size (Table~\ref{tab:arch_params}), in Figure~\ref{fig:appx_ds_memory}, illustrating this sensitivity.
%
As shown in the Figure, the latency objective that we seek to optimize can exhibit high sensitivity to small variations in the architecture parameters, making the optimization landscape particularly ill-behaved.
%
Thus, a small change in one of the discrete parameters, can induce a large change in the optimization objective.
%
This characteristic of the dataset further makes the optimization task challenging. 


\section{Overview of Accelerators and Search Space}
\label{sec:dla_shi_fast}
%
This section briefly discuss the additional accelerators (similar to \citep{kao2020confuciux}) that we evaluate in this work, namely NVDLA~\citep{nvidia} and ShiDianNao~\citep{shidiannao}, and their corresponding search spaces.

\niparagraph{NVDLA: Nvidia Deep Learning Accelerator}
%
NVDLA~\citep{nvdla} is an open architecture inference accelerator designed and maintained by Nvidia.
%
In compared to other inference accelerators, NVDLA is a weight stationary accelerator. That is, it retains the model parameters on each processing elements and parallelizes the computations across input and output channels.
%
NVDLA-style dataflow accelerators generally yield better performance for the computations of layers at the later processing stages. This is because these layers generally have larger model parameters that could benefit from less data movement associated to the model parameters.

\niparagraph{ShiDianNao: Vision Accelerator}
%
Figure~\ref{fig:shi_dla} shows the high-level schematic of ShiDianNao accelerator~\citep{shidiannao}.
%
ShiDianNao-style dataflow accelerator is an output-stationary accelerator.
%
\begin{figure}[t!]
\centering
    \includegraphics[width=0.35\linewidth]{chapters/prime/figs/accelerator/shidian.pdf}
    \caption{Overview of ShiDianNao dataflow accelerator. This dataflow accelerators exhibits an output-stationary dataflow where it keeps the partial results stationary within each processing elements (PEs).}
    \label{fig:shi_dla}
    \vspace{-0.1in} 
\end{figure}
That is, it keeps the partial results inside each PE and instead move the model parameters and input channel data. 
%
As such, in compared to NVDLA-style accelerators, ShiDianNao provides better performance for the computations of the layers with large output channels (generally first few layers of a model).

\niparagraph{Search space of dataflow accelerators.}
%
We follow a similar methodology as \citep{kao2020confuciux} to evaluate additional hardware accelerators, discussed in the previous paragraphs.
%
We use MAESTRO~\citep{maestro}, an analytical cost model, that supports the performance modeling of various dataflow accelerators.
%
In this joint accelerator design and dataflow optimization problem, the total number of parameters to be optimized is up to 106---the tuple of (\# of PEs, Buffers) per per model layer---with each parameter taking one of 12 discrete values. This makes the hardware search space consist of $\approx$~2.5$\times10^{114}$ accelerator configurations.
%
We also note that while the method proposed in~\citep{kao2020confuciux} treats the accelerator design problem as a sequential decision making problem, and uses reinforcement learning techniques, \primemethodname\ simply designs the whole accelerator in a single step, treating it as a model-based optimization problem.

\begin{figure}[t]
    \centering
    %
    \subfloat[]{
    \includegraphics[width=0.45\textwidth]{chapters/prime/figs/dataset/dist.pdf}
    \label{fig:appx_ds_dist}}
    %
    \subfloat[]{
    \includegraphics[width=0.45\textwidth]{chapters/prime/figs/dataset/core-memory.pdf}
    \label{fig:appx_ds_memory}}
    \caption{The (a) histogram of infeasible (orange bar with large score values)/feasible (blue bars) data points and (b) the sensitivity of runtime to the size of core memory for the MobileNetEdgeTPU~\citep{efficientnet:2020} dataset.} 
    \label{fig:appx_ds}
    \vspace{-0.4cm}
\end{figure}

\begin{table*}[t!]
\small{
\begin{center}
% \vspace*{0.1cm}
\caption{\footnotesize {The evaluated applications, their model parameter size, number of compute operations, and normalized compute-to-memory ratio.}}
\label{tab:compute_memory_ratio}
\vspace{-0.1in}
\resizebox{0.8\textwidth}{!}{\begin{tabular}{l|c|r|r}
\toprule
\textbf{Name}&\textbf{Model Param}&\textbf{\# of Compute Ops.}&\textbf{Normalized Compute-to-Memory Ratio}\\\midrule
{\textbf{MobileNetEdgeTPU}}&3.87~MB&1,989,811,168&1.38e-1\\\hline
{\textbf{MobileNetV2}}&3.31~MB&609,353,376&4.96e-2\\\hline
{\textbf{MobileNetV3}}&5.20~MB&449,219,600&2.33e-2\\\hline
\textbf{\mfour}&6.23~MB&3,471,920,128&1.5e-1\\\hline
\textbf{\mfive}&2.16~MB&939,752,960&1.17e-1\\\hline
\textbf{\msix}&0.41~MB&228,146,848&1.5e-1\\\hline
\textbf{U-Net}&3.69~MB&13,707,214,848&1.0\\\hline
\textbf{t-RNN Dec}&19~MB&40,116,224&\textbf{5.68e-4}\\\hline
\textbf{t-RNN Enc}&21.62~MB&45,621,248&\textbf{5.68e-4}\\
\bottomrule
\end{tabular}}
\end{center}
\vspace{-0.1cm}
}
% \vspace{-0.6cm}
\end{table*}

\begin{figure}[ht]
    \centering
    \includegraphics[width=0.45\textwidth]{chapters/prime/figs/tsne/tsne_infeasible.pdf}
    \caption{tSNE plot of the infeasible and feasible hardware accelerator designs. Note that feasible designs (shown in blue) are embedded in a sea of infeasible designs (shown in red), which makes this a challenging domain for optimization methods.}
    \label{fig:tsne_infeasible}
    \vspace{-0.1in}
\end{figure}

\begin{figure}[ht]
    \centering
    \includegraphics[width=0.4\textwidth]{chapters/prime/figs/results/overestimation.pdf}
    \caption{To verify if the overestimation hypothesis--that optimizing an accelerator against a na\"ive standard surrogate model is likely to find optimizers that appear promising under the learned model, but do not actually attain low-latency values--in our domain, we plot a calibration plot of the top accelerator designs found by optimizing a na\"ively trained standard surrogate model. In the scatter plot, we represent each accelerator as a point with its x-coordinate equal to the actual latency obtained by running this accelerator design in the simulator and the y-coordinate equal to the predicted latency under the learned surrogate. Note that for a large chunk of designs, the predicted latency is much smaller than their actual latency (i.e., these designs lie beneath the $y=x$ line in the plot above). This means that optimizing designs under a na\"ive surrogate model is prone to finding designs that appear overly promising (i.e., attain lower predicted latency values), but are not actually promising. This confirms the presence of the overestimation hypothesis on our problem domain.}
    \label{fig:cali_plot}
    \vspace{-0.1in}
\end{figure}

\begin{figure}[ht]
    \centering
    \includegraphics[width=0.4\textwidth]{chapters/prime/figs/overestimation_prime.pdf}
    \includegraphics[width=0.4\textwidth]{chapters/prime/figs/zoomed_in.pdf}
    \caption{\review{Plot showing the calibration plot of the predicted (y-axis) and actual latencies (x-axis) of accelerators found by \primemethodname. Compared to Figure~\ref{fig:cali_plot}, observe that all the acclerator configurations lie above $y=x$, meaning that \primemethodname\ predicts a higher latency (y-axis) compared to the actual latency. This means that \primemethodname\ does not think that accelerators that attain high-latency values under the simulator, are actually good.  We also provide a zoomed-in version of the plot on the right, which shows that there are accelerators do have meaningfully distinct latency predictions under \primemethodname. Observe in the zoomed-in plot that the designs that attain small predicted latencies also perform relatively better under the actual latency compared to the designs that attain larger predicted latency of $\sim$ 14000-16000 under the \primemethodname\ surrogate. Optimizing against \primemethodname\ is still effective because optimization just needs relative correctness of values, not absolutely correct latency predictions.}}
    \label{fig:overestimation_prime_plot}
    \vspace{-0.1in}
\end{figure}

% \newpage

\section{{Subset of Applications for Good Zero-Shot Performance}}
%
\begin{table*}[t]
\small
\centering
\vspace*{0.0cm}
\caption{\label{table:zero_shot_all_2} \footnotesize \review{Additional ablation study under zero-shot setting when the test applications include all the nine evaluated models (e.g. MobileNet (EdgeTPU, V2, V3), \mfour, \mfive, \msix, t-RNN Dec, t-RNN Enc, U-Net). Lower latency is better. From \textbf{left} to \textbf{right}: the applications used to train the surrogate model in \primemethodname, the area constraint of the accelerator, \primemethodname's (best, median) latency.}}
\vspace{-0.1cm}
\resizebox{0.9\textwidth}{!}{% <------ Don't forget this %
\begin{tabular}{l|l|l}
\toprule
\textbf{Train Applications}&\textbf{Area}&\textbf{\primemethodname} \textbf{(best, median)} \\\midrule
% Area 29
\textbf{(1)} MobileNet~(V2, V3), \mfive, \msix&29~mm$^2$&({426.65}, {427.35})\\\hline
% Area 29
\textbf{(2)} MobileNet~(V2, V3), \mfive, t-RNN Enc&29~mm$^2$&({461.79}, {464.87})\\\hline
% Area 29
\textbf{(3)} MobileNet~(EdgeTPU, V2, V3), \mfour, \mfive, \msix, t-RNN Enc&29~mm$^2$&({426.65}, {427.94})\\\bottomrule
\end{tabular}% <------ Don't forget this %
}
% \vspace{-0.2cm}
\end{table*}
%
{In this section, we present the results of an ablation study with the goal to identify a subset of applications such that training on data from only these applications yields good zero-shot performance across all the nine applications studied in this work.
%
Since we cannot train \primemethodname\ for every subset of applications because the space of subsets of all applications is exponentially large, we utilized some heuristics in devising the subset of applications we would train on, with the goal to make interesting observations that allow us to devise rough guidelines for performing application selection.}

{\textbf{Our heuristic for devising subsets of applications:} Building on the intuition that applications with very different compute to memory ratios (shown in Table~\ref{tab:compute_memory_ratio}) may require different accelerator designs -- for example, if our goal is to run a compute-intensive application, we likely need an accelerator design with more compute units -- we study two subsets of training applications: \textbf{(1)} MobileNetV2, MobileNetV3, \msix, \mfive, and \textbf{(2)} MobileNetV2, MobilenetV3, \mfive, t-RNN Enc.
%
Note that, these two combinations only differ in whether some RNN application was used in training or not.
%
As shown in Table~\ref{tab:compute_memory_ratio}, the t-RNN applications admit a very different compute to memory ratio, for instance, while this ratio is $5.68e-4$ for t-RNN Enc and t-RNN Dec, it is much different $\sim 0.01-0.2$ for other models MobileNetEdgeTPU, MobileNetV2, MobileNetV3, \mfive, and \msix. This means that likely t-RNN Enc and Dec will require different kinds of accelerators for good performance compared to the other applications.}

{\textbf{Results:} We present the performance of zero-shot evaluating the designed accelerator obtained by training on combinations \textbf{(1)} and \textbf{(2)}, and also the accelerator obtained by training on \textbf{(3)} seven applications from Table~\ref{table:zero_shot} in Table~\ref{table:zero_shot_all_2} as reference. We make some key takeaways from the results: 
}
\begin{itemize}
    \item The performance of both configuration \textbf{(1)} and training with seven applications (\textbf{(3)}, last row of Table~\ref{table:zero_shot_all_2}) are similar.
    \item In case \textbf{(2)}, when the training applications consist of four applications in which one application is t-RNN Enc, with drastically different compute to memory ratio (Table~\ref{tab:compute_memory_ratio}), the performance on an average across all applications becomes slightly worse (compare the performance in \textbf{(2)} vs \textbf{(3)}).
\end{itemize}

{\textbf{Conclusion and guidance on selecting good applications:} The above results indicate that only a few applications (e.g., four applications in case \textbf{(1)}) can be enough for good performance on all nine applications. While this set may not be not minimal, it is certainly a much smaller set compared to the nine applications considered.
Adding an RNN application in case \textbf{(2)} increases latency in compared to case \textbf{(1)}, because t-RNN Enc likely admits a very different optimal accelerator compared to the other applications due to a very different compute/memory ratio, which in turn skews the generalization of the surrogate learned by \primemethodname\ when trained only on this limited set of four applications. However, when seven applications are provided in case \textbf{(3)}, even when the set of training applications includes t-RNN, its contribution on the \primemethodname\ surrogate is reduced since many other compute intensive applications are also provided in the training set and the resulting accelerator performs well.}
%

{\textbf{Practitioner guidance:} The primary practitioner guidance we can conclude here is that \emph{the models used for training must be representative of the overall distribution of the target models that we want to zero-shot generalize to}. Since a number of our applications are compute intensive, we were able to simply utilize set \textbf{(1)} to attain good performance on all the applications. On the other hand, in case \textbf{(2)}, when the t-RNN Enc application was over-represented -- while seven of nine applications we considered were primarily compute intensive, one out of four applications we used for training in case \textbf{(2)} were memory intensive -- this hurt performance on the overall set.
%
Therefore, we believe that ensuring that the training subset of applications is adequately aligned with the overall set of applications in terms of compute/memory ratio statistic is imperative for favorable zero-shot performance.}

{\textbf{For a practitioner deciding between zero-shot generalization and additional data collection}, it may make sense to test if the target application admits a similar value of the compute to memory ratio as an already existing application. If it does, then the practitioner might be able to utilize the zero-shot generalization, as is indicated with the good performance of case \textbf{(1)}, whereas if the compute/memory ratio is heavily different from any seen application, zero-shot generalization to the target application may be worse. Finally, making sure that the training applications adequately reflect the compute/memory ratio statistic for the overall target set is important.}


\end{document}