\begin{abstract}
\label{sec:abstract}
%
To attain higher efficiency, the industry has gradually reformed towards application-specific hardware accelerators.
%
While such a paradigm shift is already starting to show promising results, designers need to spend considerable manual effort and perform large number of time-consuming simulations to find accelerators that can accelerate multiple target applications while obeying design constraints.
%
Moreover, such a ``simulation-driven'' approach must be re-run from scratch every time the set of target applications or design constraints change. 
%
An alternative paradigm is to use a ``data-driven'', offline approach that utilizes logged simulation data, to architect hardware accelerators, without needing any form of simulations.
%
Such an approach not only alleviates the need to run time-consuming simulation, but also enables data reuse and applies even when set of target applications changes.
%
In this paper, we develop such a data-driven offline optimization method for designing hardware accelerators, dubbed \methodname, that enjoys all of these properties.
%
Our approach learns a conservative, robust estimate of the desired cost function, utilizes infeasible points and optimizes the design against this estimate without any additional simulator queries during optimization.
%
\methodname\ architects accelerators---tailored towards both single- and multi-applications---improving performance upon stat-of-the-art simulation-driven methods by about 1.54$\times$ and 1.20$\times$, while considerably reducing the required total simulation time by 93\% and 99\%, respectively.
%
In addition, \methodname\ also architects effective accelerators for unseen applications in a zero-shot setting, outperforming simulation-based methods by 1.26$\times$\footnote{A more accessible blog post summarizing this paper is available on the Google AI blog at: \url{https://ai.googleblog.com/2022/03/offline-optimization-for-architecting.html}}.
%
\end{abstract}